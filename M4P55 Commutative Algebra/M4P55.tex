\def\module{M4P55 Commutative Algebra}
\def\lecturer{Prof Alexei Skorobogatov}
\def\term{Autumn 2019}
\def\cover{}
\def\syllabus{}
\def\thm{section}

\documentclass{article}

% Packages

\usepackage{amssymb}
\usepackage{amsthm}
\usepackage[UKenglish]{babel}
\usepackage{commath}
\usepackage{enumitem}
\usepackage{etoolbox}
\usepackage{fancyhdr}
\usepackage[margin=1in]{geometry}
\usepackage{graphicx}
\usepackage[hidelinks]{hyperref}
\usepackage[utf8]{inputenc}
\usepackage{listings}
\usepackage{mathtools}
\usepackage{stmaryrd}
\usepackage{tikz-cd}
\usepackage{csquotes}

% Formatting

\addto\captionsUKenglish{\renewcommand{\abstractname}{Syllabus}}
\delimitershortfall5pt
\ifx\thm\undefined\newtheorem{n}{}\else\newtheorem{n}{}[\thm]\fi
\newcommand\newoperator[1]{\ifcsdef{#1}{\cslet{#1}{\relax}}{}\csdef{#1}{\operatorname{#1}}}
\setlength{\parindent}{0cm}

% Environments

\theoremstyle{plain}
\newtheorem{algorithm}[n]{Algorithm}
\newtheorem*{algorithm*}{Algorithm}
\newtheorem{algorithm**}{Algorithm}
\newtheorem{conjecture}[n]{Conjecture}
\newtheorem*{conjecture*}{Conjecture}
\newtheorem{conjecture**}{Conjecture}
\newtheorem{corollary}[n]{Corollary}
\newtheorem*{corollary*}{Corollary}
\newtheorem{corollary**}{Corollary}
\newtheorem{lemma}[n]{Lemma}
\newtheorem*{lemma*}{Lemma}
\newtheorem{lemma**}{Lemma}
\newtheorem{proposition}[n]{Proposition}
\newtheorem*{proposition*}{Proposition}
\newtheorem{proposition**}{Proposition}
\newtheorem{theorem}[n]{Theorem}
\newtheorem*{theorem*}{Theorem}
\newtheorem{theorem**}{Theorem}

\theoremstyle{definition}
\newtheorem{aim}[n]{Aim}
\newtheorem*{aim*}{Aim}
\newtheorem{aim**}{Aim}
\newtheorem{axiom}[n]{Axiom}
\newtheorem*{axiom*}{Axiom}
\newtheorem{axiom**}{Axiom}
\newtheorem{condition}[n]{Condition}
\newtheorem*{condition*}{Condition}
\newtheorem{condition**}{Condition}
\newtheorem{definition}[n]{Definition}
\newtheorem*{definition*}{Definition}
\newtheorem{definition**}{Definition}
\newtheorem{example}[n]{Example}
\newtheorem*{example*}{Example}
\newtheorem{example**}{Example}
\newtheorem{exercise}[n]{Exercise}
\newtheorem*{exercise*}{Exercise}
\newtheorem{exercise**}{Exercise}
\newtheorem{fact}[n]{Fact}
\newtheorem*{fact*}{Fact}
\newtheorem{fact**}{Fact}
\newtheorem{goal}[n]{Goal}
\newtheorem*{goal*}{Goal}
\newtheorem{goal**}{Goal}
\newtheorem{law}[n]{Law}
\newtheorem*{law*}{Law}
\newtheorem{law**}{Law}
\newtheorem{plan}[n]{Plan}
\newtheorem*{plan*}{Plan}
\newtheorem{plan**}{Plan}
\newtheorem{problem}[n]{Problem}
\newtheorem*{problem*}{Problem}
\newtheorem{problem**}{Problem}
\newtheorem{question}[n]{Question}
\newtheorem*{question*}{Question}
\newtheorem{question**}{Question}
\newtheorem{warning}[n]{Warning}
\newtheorem*{warning*}{Warning}
\newtheorem{warning**}{Warning}
\newtheorem{acknowledgements}[n]{Acknowledgements}
\newtheorem*{acknowledgements*}{Acknowledgements}
\newtheorem{acknowledgements**}{Acknowledgements}
\newtheorem{annotations}[n]{Annotations}
\newtheorem*{annotations*}{Annotations}
\newtheorem{annotations**}{Annotations}
\newtheorem{assumption}[n]{Assumption}
\newtheorem*{assumption*}{Assumption}
\newtheorem{assumption**}{Assumption}
\newtheorem{conclusion}[n]{Conclusion}
\newtheorem*{conclusion*}{Conclusion}
\newtheorem{conclusion**}{Conclusion}
\newtheorem{claim}[n]{Claim}
\newtheorem*{claim*}{Claim}
\newtheorem{claim**}{Claim}
\newtheorem{notation}[n]{Notation}
\newtheorem*{notation*}{Notation}
\newtheorem{notation**}{Notation}
\newtheorem{note}[n]{Note}
\newtheorem*{note*}{Note}
\newtheorem{note**}{Note}
\newtheorem{remark}[n]{Remark}
\newtheorem*{remark*}{Remark}
\newtheorem{remark**}{Remark}

% Lectures

\newcommand{\lecture}[3]{ % Lecture
  \marginpar{
    Lecture #1 \\
    #2 \\
    #3
  }
}

% Blackboard

\renewcommand{\AA}{\mathbb{A}} % Blackboard A
\newcommand{\BB}{\mathbb{B}}   % Blackboard B
\newcommand{\CC}{\mathbb{C}}   % Blackboard C
\newcommand{\DD}{\mathbb{D}}   % Blackboard D
\newcommand{\EE}{\mathbb{E}}   % Blackboard E
\newcommand{\FF}{\mathbb{F}}   % Blackboard F
\newcommand{\GG}{\mathbb{G}}   % Blackboard G
\newcommand{\HH}{\mathbb{H}}   % Blackboard H
\newcommand{\II}{\mathbb{I}}   % Blackboard I
\newcommand{\JJ}{\mathbb{J}}   % Blackboard J
\newcommand{\KK}{\mathbb{K}}   % Blackboard K
\newcommand{\LL}{\mathbb{L}}   % Blackboard L
\newcommand{\MM}{\mathbb{M}}   % Blackboard M
\newcommand{\NN}{\mathbb{N}}   % Blackboard N
\newcommand{\OO}{\mathbb{O}}   % Blackboard O
\newcommand{\PP}{\mathbb{P}}   % Blackboard P
\newcommand{\QQ}{\mathbb{Q}}   % Blackboard Q
\newcommand{\RR}{\mathbb{R}}   % Blackboard R
\renewcommand{\SS}{\mathbb{S}} % Blackboard S
\newcommand{\TT}{\mathbb{T}}   % Blackboard T
\newcommand{\UU}{\mathbb{U}}   % Blackboard U
\newcommand{\VV}{\mathbb{V}}   % Blackboard V
\newcommand{\WW}{\mathbb{W}}   % Blackboard W
\newcommand{\XX}{\mathbb{X}}   % Blackboard X
\newcommand{\YY}{\mathbb{Y}}   % Blackboard Y
\newcommand{\ZZ}{\mathbb{Z}}   % Blackboard Z

% Brackets

\renewcommand{\eval}[1]{\left. #1 \right|}          % Evaluation
\newcommand{\br}{\del}                              % Brackets
\newcommand{\abr}[1]{\left\langle #1 \right\rangle} % Angle brackets
\newcommand{\fbr}[1]{\left\lfloor #1 \right\rfloor} % Floor brackets
\newcommand{\lbr}[1]{\left\lfloor #1 \right\rfloor} % Ceiling brackets
\newcommand{\st}{\ \middle| \ }                     % Such that

% Calligraphic

\newcommand{\AAA}{\mathcal{A}} % Calligraphic A
\newcommand{\BBB}{\mathcal{B}} % Calligraphic B
\newcommand{\CCC}{\mathcal{C}} % Calligraphic C
\newcommand{\DDD}{\mathcal{D}} % Calligraphic D
\newcommand{\EEE}{\mathcal{E}} % Calligraphic E
\newcommand{\FFF}{\mathcal{F}} % Calligraphic F
\newcommand{\GGG}{\mathcal{G}} % Calligraphic G
\newcommand{\HHH}{\mathcal{H}} % Calligraphic H
\newcommand{\III}{\mathcal{I}} % Calligraphic I
\newcommand{\JJJ}{\mathcal{J}} % Calligraphic J
\newcommand{\KKK}{\mathcal{K}} % Calligraphic K
\newcommand{\LLL}{\mathcal{L}} % Calligraphic L
\newcommand{\MMM}{\mathcal{M}} % Calligraphic M
\newcommand{\NNN}{\mathcal{N}} % Calligraphic N
\newcommand{\OOO}{\mathcal{O}} % Calligraphic O
\newcommand{\PPP}{\mathcal{P}} % Calligraphic P
\newcommand{\QQQ}{\mathcal{Q}} % Calligraphic Q
\newcommand{\RRR}{\mathcal{R}} % Calligraphic R
\newcommand{\SSS}{\mathcal{S}} % Calligraphic S
\newcommand{\TTT}{\mathcal{T}} % Calligraphic T
\newcommand{\UUU}{\mathcal{U}} % Calligraphic U
\newcommand{\VVV}{\mathcal{V}} % Calligraphic V
\newcommand{\WWW}{\mathcal{W}} % Calligraphic W
\newcommand{\XXX}{\mathcal{X}} % Calligraphic X
\newcommand{\YYY}{\mathcal{Y}} % Calligraphic Y
\newcommand{\ZZZ}{\mathcal{Z}} % Calligraphic Z

% Fraktur

\newcommand{\aaa}{\mathfrak{a}}   % Fraktur a
\newcommand{\bbb}{\mathfrak{b}}   % Fraktur b
\newcommand{\ccc}{\mathfrak{c}}   % Fraktur c
\newcommand{\ddd}{\mathfrak{d}}   % Fraktur d
\newcommand{\eee}{\mathfrak{e}}   % Fraktur e
\newcommand{\fff}{\mathfrak{f}}   % Fraktur f
\renewcommand{\ggg}{\mathfrak{g}} % Fraktur g
\newcommand{\hhh}{\mathfrak{h}}   % Fraktur h
\newcommand{\iii}{\mathfrak{i}}   % Fraktur i
\newcommand{\jjj}{\mathfrak{j}}   % Fraktur j
\newcommand{\kkk}{\mathfrak{k}}   % Fraktur k
\renewcommand{\lll}{\mathfrak{l}} % Fraktur l
\newcommand{\mmm}{\mathfrak{m}}   % Fraktur m
\newcommand{\nnn}{\mathfrak{n}}   % Fraktur n
\newcommand{\ooo}{\mathfrak{o}}   % Fraktur o
\newcommand{\ppp}{\mathfrak{p}}   % Fraktur p
\newcommand{\qqq}{\mathfrak{q}}   % Fraktur q
\newcommand{\rrr}{\mathfrak{r}}   % Fraktur r
\newcommand{\sss}{\mathfrak{s}}   % Fraktur s
\newcommand{\ttt}{\mathfrak{t}}   % Fraktur t
\newcommand{\uuu}{\mathfrak{u}}   % Fraktur u
\newcommand{\vvv}{\mathfrak{v}}   % Fraktur v
\newcommand{\www}{\mathfrak{w}}   % Fraktur w
\newcommand{\xxx}{\mathfrak{x}}   % Fraktur x
\newcommand{\yyy}{\mathfrak{y}}   % Fraktur y
\newcommand{\zzz}{\mathfrak{z}}   % Fraktur z

% Geometry

\newcommand{\CP}{\mathbb{CP}}                                              % Complex projective space
\newcommand{\iintd}[4]{\iint_{#1} \, #2 \, \dif #3 \, \dif #4}             % Double integral
\newcommand{\RP}{\mathbb{RP}}                                              % Real projective space
\newcommand{\intd}[4]{\int_{#1}^{#2} \, #3 \, \dif #4}                     % Single integral
\newcommand{\iiintd}[5]{\iint_{#1} \, #2 \, \dif #3 \, \dif #4 \, \dif #5} % Triple integral

% Logic

\newcommand{\iffb}[2]{\br{#1 \leftrightarrow #2}} % Biconditional
\newcommand{\andb}[2]{\br{#1 \land #2}}           % Conjunction
\newcommand{\orb}[2]{\br{#1 \lor #2}}             % Disjunction
\newcommand{\nib}[2]{\br{#1 \notin #2}}           % Element of
\newcommand{\eqb}[2]{\br{#1 = #2}}                % Equal to
\newcommand{\teb}[1]{\br{\exists #1}}             % Existential quantifier
\newcommand{\impb}[2]{\br{#1 \rightarrow #2}}     % Implication
\newcommand{\ltb}[2]{\br{#1 < #2}}                % Less than
\newcommand{\leb}[2]{\br{#1 \le #2}}              % Less than or equal to
\newcommand{\notb}[1]{\br{\neg #1}}               % Negation
\newcommand{\inb}[2]{\br{#1 \in #2}}              % Not element of
\newcommand{\neb}[2]{\br{#1 \ne #2}}              % Not equal to
\newcommand{\subb}[2]{\br{#1 \subseteq #2}}       % Subset
\newcommand{\fab}[1]{\br{\forall #1}}             % Universal quantifier

% Maps

\newcommand{\bijection}[7][]{    % Bijection
  \ifx &#1&
    \begin{array}{rcl}
      #2 & \longleftrightarrow & #3 \\
      #4 & \longmapsto         & #5 \\
      #6 & \longmapsfrom       & #7
    \end{array}
  \else
    \begin{array}{ccrcl}
      #1 & : & #2 & \longrightarrow & #3 \\
         &   & #4 & \longmapsto     & #5 \\
         &   & #6 & \longmapsfrom   & #7
    \end{array}
  \fi
}
\newcommand{\birational}[7][]{   % Birational map
  \ifx &#1&
    \begin{array}{rcl}
      #2 & \dashrightarrow & #3 \\
      #4 & \longmapsto     & #5 \\
      #6 & \longmapsfrom   & #7
    \end{array}
  \else
    \begin{array}{ccrcl}
      #1 & : & #2 & \dashrightarrow & #3 \\
         &   & #4 & \longmapsto     & #5 \\
         &   & #6 & \longmapsfrom   & #7
    \end{array}
  \fi
}
\newcommand{\correspondence}[2]{ % Correspondence
  \cbr{
    \begin{array}{c}
      #1
    \end{array}
  }
  \qquad
  \leftrightsquigarrow
  \qquad
  \cbr{
    \begin{array}{c}
      #2
    \end{array}
  }
}
\newcommand{\function}[5][]{     % Function
  \ifx &#1&
    \begin{array}{rcl}
      #2 & \longrightarrow & #3 \\
      #4 & \longmapsto     & #5
    \end{array}
  \else
    \begin{array}{ccrcl}
      #1 & : & #2 & \longrightarrow & #3 \\
         &   & #4 & \longmapsto     & #5
    \end{array}
  \fi
}
\newcommand{\functions}[7][]{    % Functions
  \ifx &#1&
    \begin{array}{rcl}
      #2 & \longrightarrow & #3 \\
      #4 & \longmapsto     & #5 \\
      #6 & \longmapsto     & #7
    \end{array}
  \else
    \begin{array}{ccrcl}
      #1 & : & #2 & \longrightarrow & #3 \\
         &   & #4 & \longmapsto     & #5 \\
         &   & #6 & \longmapsto     & #7
    \end{array}
  \fi
}
\newcommand{\rational}[5][]{     % Rational map
  \ifx &#1&
    \begin{array}{rcl}
      #2 & \dashrightarrow & #3 \\
      #4 & \longmapsto     & #5
    \end{array}
  \else
    \begin{array}{ccrcl}
      #1 & : & #2 & \dashrightarrow & #3 \\
         &   & #4 & \longmapsto     & #5
    \end{array}
  \fi
}

% Matrices

\newcommand{\onebytwo}[2]{      % One by two matrix
  \begin{pmatrix}
    #1 & #2
  \end{pmatrix}
}
\newcommand{\onebythree}[3]{    % One by three matrix
  \begin{pmatrix}
    #1 & #2 & #3
  \end{pmatrix}
}
\newcommand{\twobyone}[2]{      % Two by one matrix
  \begin{pmatrix}
    #1 \\
    #2
  \end{pmatrix}
}
\newcommand{\twobytwo}[4]{      % Two by two matrix
  \begin{pmatrix}
    #1 & #2 \\
    #3 & #4
  \end{pmatrix}
}
\newcommand{\threebyone}[3]{    % Three by one matrix
  \begin{pmatrix}
    #1 \\
    #2 \\
    #3
  \end{pmatrix}
}
\newcommand{\threebythree}[9]{  % Three by three matrix
  \begin{pmatrix}
    #1 & #2 & #3 \\
    #4 & #5 & #6 \\
    #7 & #8 & #9
  \end{pmatrix}
}
\newcommand{\twobytwosmall}[4]{ % Two by two small matrix
  \begin{psmallmatrix}
    #1 & #2 \\
    #3 & #4
  \end{psmallmatrix}
}

% Number theory

\renewcommand{\symbol}[2]{\br{\tfrac{#1}{#2}}} % Power residue symbol
\newcommand{\unit}[1]{\br{\ZZ / #1\ZZ}^\times} % Unit group

% Operators

\newoperator{ab}    % Abelian
\newoperator{AG}    % Affine geometry
\newoperator{alg}   % Algebraic
\newoperator{Ann}   % Annihilator
\newoperator{area}  % Area
\newoperator{Aut}   % Automorphism
\newoperator{card}  % Cardinality
\newoperator{ch}    % Characteristic
\newoperator{Cl}    % Class
\newoperator{Coker} % Cokernel
\newoperator{col}   % Column
\newoperator{Corr}  % Correspondence
\newoperator{diam}  % Diameter
\newoperator{Disc}  % Discriminant
\newoperator{dom}   % Domain
\newoperator{Eig}   % Eigenvalue
\newoperator{Em}    % Embedding
\newoperator{End}   % Endomorphism
\newoperator{fin}   % Finite
\newoperator{Fix}   % Fixed
\newoperator{Frac}  % Fraction
\newoperator{Frob}  % Frobenius
\newoperator{Fun}   % Function
\newoperator{Gal}   % Galois
\newoperator{GL}    % General linear
\newoperator{Ham}   % Hamming
\newoperator{Homeo} % Homeomorphism
\newoperator{Hom}   % Homomorphism
\newoperator{id}    % Identity
\newoperator{Im}    % Image
\newoperator{Ind}   % Index
\newoperator{Ker}   % Kernel
\newoperator{lcm}   % Least common multiple
\newoperator{Mat}   % Matrix
\newoperator{mult}  % Multiplicity
\newoperator{new}   % New
\newoperator{Nm}    % Norm
\newoperator{old}   % Old
\newoperator{op}    % Opposite
\newoperator{ord}   % Order
\newoperator{Pay}   % Payley
\newoperator{PG}    % Projective geometry
\newoperator{PGL}   % Projective general linear
\newoperator{PSL}   % Projective special linear
\newoperator{rad}   % Radical
\newoperator{ran}   % Range
\newoperator{Res}   % Residue
\newoperator{rk}    % Rank
\newoperator{Re}    % Real
\newoperator{row}   % Row
\newoperator{sgn}   % Sign
\newoperator{Sing}  % Singular
\newoperator{SK}    % Skeleton
\newoperator{sp}    % Span
\newoperator{SL}    % Special linear
\newoperator{SO}    % Special orthogonal
\newoperator{Spec}  % Spectrum
\newoperator{Stab}  % Stabiliser
\newoperator{star}  % Star
\newoperator{srg}   % Strongly regular graph
\newoperator{supp}  % Support
\newoperator{Sym}   % Symmetric
\newoperator{tors}  % Torsion
\newoperator{Tr}    % Trace
\newoperator{vol}   % Volume
\newoperator{wt}    % Weight

% Roman

\newcommand{\A}{\mathrm{A}}   % Roman A
\newcommand{\B}{\mathrm{B}}   % Roman B
\newcommand{\C}{\mathrm{C}}   % Roman C
\newcommand{\D}{\mathrm{D}}   % Roman D
\newcommand{\E}{\mathrm{E}}   % Roman E
\newcommand{\F}{\mathrm{F}}   % Roman F
\newcommand{\G}{\mathrm{G}}   % Roman G
\renewcommand{\H}{\mathrm{H}} % Roman H
\newcommand{\I}{\mathrm{I}}   % Roman I
\newcommand{\J}{\mathrm{J}}   % Roman J
\newcommand{\K}{\mathrm{K}}   % Roman K
\renewcommand{\L}{\mathrm{L}} % Roman L
\newcommand{\M}{\mathrm{M}}   % Roman M
\newcommand{\N}{\mathrm{N}}   % Roman N
\renewcommand{\O}{\mathrm{O}} % Roman O
\renewcommand{\P}{\mathrm{P}} % Roman P
\newcommand{\Q}{\mathrm{Q}}   % Roman Q
\newcommand{\R}{\mathrm{R}}   % Roman R
\renewcommand{\S}{\mathrm{S}} % Roman S
\newcommand{\T}{\mathrm{T}}   % Roman T
\newcommand{\U}{\mathrm{U}}   % Roman U
\newcommand{\V}{\mathrm{V}}   % Roman V
\newcommand{\W}{\mathrm{W}}   % Roman W
\newcommand{\X}{\mathrm{X}}   % Roman X
\newcommand{\Y}{\mathrm{Y}}   % Roman Y
\newcommand{\Z}{\mathrm{Z}}   % Roman Z

\renewcommand{\a}{\mathrm{a}} % Roman a
\renewcommand{\b}{\mathrm{b}} % Roman b
\renewcommand{\c}{\mathrm{c}} % Roman c
\renewcommand{\d}{\mathrm{d}} % Roman d
\newcommand{\e}{\mathrm{e}}   % Roman e
\newcommand{\f}{\mathrm{f}}   % Roman f
\newcommand{\g}{\mathrm{g}}   % Roman g
\newcommand{\h}{\mathrm{h}}   % Roman h
\renewcommand{\i}{\mathrm{i}} % Roman i
\renewcommand{\j}{\mathrm{j}} % Roman j
\renewcommand{\k}{\mathrm{k}} % Roman k
\renewcommand{\l}{\mathrm{l}} % Roman l
\newcommand{\m}{\mathrm{m}}   % Roman m
\renewcommand{\n}{\mathrm{n}} % Roman n
\renewcommand{\o}{\mathrm{o}} % Roman o
\newcommand{\p}{\mathrm{p}}   % Roman p
\newcommand{\q}{\mathrm{q}}   % Roman q
\renewcommand{\r}{\mathrm{r}} % Roman r
\newcommand{\s}{\mathrm{s}}   % Roman s
\renewcommand{\t}{\mathrm{t}} % Roman t
\renewcommand{\u}{\mathrm{u}} % Roman u
\renewcommand{\v}{\mathrm{v}} % Roman v
\newcommand{\w}{\mathrm{w}}   % Roman w
\newcommand{\x}{\mathrm{x}}   % Roman x
\newcommand{\y}{\mathrm{y}}   % Roman y
\newcommand{\z}{\mathrm{z}}   % Roman z

% Tikz

\tikzset{
  arrow symbol/.style={"#1" description, allow upside down, auto=false, draw=none, sloped},
  subset/.style={arrow symbol={\subset}},
  cong/.style={arrow symbol={\cong}}
}

% Fancy header

\pagestyle{fancy}
\lhead{\module}
\rhead{\nouppercase{\leftmark}}

% Make title

\title{\module}
\author{Lectured by \lecturer \\ Typed by David Kurniadi Angdinata}
\date{\term}

\begin{document}

% Title page
\maketitle
\cover
\vfill
\begin{abstract}
\noindent\syllabus
\end{abstract}

\pagebreak

% Contents page
\tableofcontents

\pagebreak

% Document page
\setcounter{section}{-1}

\section{Introduction}

\lecture{1}{Thursday}{03/10/19}

The prerequisites are
\begin{itemize}
\item groups,
\item rings,
\item fields, and
\item a solid linear algebra.
\end{itemize}

This course is good for
\begin{itemize}
\item algebraic geometry, and
\item algebraic number theory.
\end{itemize}

The following are books.
\begin{itemize}
\item M Reid, Undergraduate commutative algebra, 1995
\item M F Atiyah and I G Macdonald, Introduction to commutative algebra, 1969
\end{itemize}

The following is the structure of the course.
\begin{itemize}
\item Generalities on rings, such as ideals, and examples.
\item Localisation of rings between a ring $ R $ and the fraction field $ K $ of $ R $, such as $ \ZZ $ and $ \QQ $.
\item Finiteness conditions of Noetherian rings and Artinian rings.
\item Integral closure and normal rings, such as $ \ZZ\sbr{i} \subset \QQ\br{i} $ and $ \ZZ\sbr{\sqrt{-3}} \subset \ZZ\sbr{\tfrac{1 + \sqrt{-3}}{2}} \subset \QQ\br{\sqrt{-3}} $.
\item Discrete valuation rings.
\item Completion of rings with topology.
\end{itemize}

\pagebreak

\section{Rings and ideals}

\begin{definition}
A \textbf{commutative ring} is a set $ \br{A, +, \cdot, 0, 1} $ such that
\begin{enumerate}
\item $ \br{A, +, 0} $ is an abelian group,
\item for all $ x, y, z \in A $,
\begin{itemize}
\item $ \br{x \cdot y} \cdot z = x \cdot \br{y \cdot z} $,
\item $ x \cdot y = y \cdot x $,
\item $ x \cdot \br{y + z} = x \cdot y + x \cdot z $, and
\end{itemize}
\item for all $ x \in A $, $ x \cdot 1 = 1 \cdot x = x $.
\end{enumerate}
\end{definition}

\begin{remark}
\hfill
\begin{itemize}
\item One is uniquely determined by $ 3 $, since $ 1' = 1' \cdot 1 = 1 $.
\item If $ 1 = 0 $, then $ 0 = x \cdot 0 = x \cdot 1 = x $, since
$$ x \cdot 0 = x \cdot \br{0 + 0} = x \cdot 0 + x \cdot 0, $$
so $ x \cdot 0 = 0 $. So every element is zero. Hence $ R = \cbr{0} $.
\end{itemize}
\end{remark}

\begin{definition}
A \textbf{homomorphism of rings} $ f : A \to B $ is a map such that for all $ x, y \in A $,
$$ f\br{x + y} = f\br{x} + f\br{y}, \qquad f\br{xy} = f\br{x}f\br{y}, \qquad f\br{1} = 1. $$
\end{definition}

\begin{example*}
If $ A \subset B $ is closed under $ + $ and $ \cdot $, and $ 1 \in A $, then
$$ \function{A}{B}{x}{x} $$
is a homomorphism.
\end{example*}

\begin{remark}
\hfill
\begin{itemize}
\item A composition of homomorphisms is a homomorphism.
\item An \textbf{isomorphism} is a bijective homomorphism.
\end{itemize}
\end{remark}

\begin{definition}
A subset $ I $ of a ring $ A $ is an \textbf{ideal} if $ I $ is a subgroup of the additive group $ \br{A, +} $ which is closed under multiplication by elements of $ A $, so $ xI \subset I $ for any $ x \in A $. Sometimes this is written as $ I \triangleleft A $. In this case the \textbf{quotient group} $ A / I $ is naturally a ring, where $ \br{x + I}\br{y + I} $ is defined as $ xy + I $.
\end{definition}

\begin{proposition}
Let $ I $ be an ideal of a commutative ring $ A $. Then there is a natural bijection between the ideals $ J \subset A $ such that $ I \subset J $ and the ideals of $ A / I $.
\end{proposition}

\begin{proof}
Let
$$ \function{A}{A / I}{x}{x + I} $$
be the natural surjective map. Send $ J $ to its image under this map.
\end{proof}

\begin{definition}
If $ f : A \to B $ is a homomorphism, then
$$ \Ker f = \cbr{x \in A \st f\br{x} = 0} $$
is an ideal in $ A $, and
$$ \Im f = f\br{A} \cong A / \Ker f \subset B. $$
\end{definition}

\pagebreak

\section{Polynomials and formal power series}

\lecture{2}{Tuesday}{08/10/19}

\begin{definition}
Let $ R $ be a ring. The \textbf{polynomial ring} with coefficients in $ R $ is
$$ R\sbr{x} = \cbr{a_0 + \dots + a_nx^n \st a_i \in R, \ n \in \ZZ_{\ge 0}}. $$
The addition is coefficient-wise, and the multiplication is given by the formula
$$ \br{\sum_{i \ge 0} a_ix^i}\br{\sum_{j \ge 0} b_jx^j} = \sum_{i \ge 0} \br{\sum_{j + k = i, \ j \ge 0, \ k \ge 0} a_jb_k}x^i, $$
where all but finitely many coefficients are zero. Define
$$ R\sbr{x_1, \dots, x_n} = R\sbr{x_1} \dots \sbr{x_n} = \cbr{\sum_{i_1, \dots, i_n \ge 0} a_{i_1, \dots, i_n}x_1^{i_1} \dots x_n^{i_n} \st a_{i_1, \dots, i_n} \in R}, $$
where all but finitely many coefficients $ a_{i_1, \dots, i_n} $ are equal to zero.
\end{definition}

\begin{definition}
The \textbf{ring of formal power series} with coefficients in $ R $ is
$$ R\sbr{\sbr{t}} = \cbr{a_0 + a_1t + \dots \st a_i \in R}. $$
The addition is coefficient-wise, and the multiplication is given by the formula
$$ \br{\sum_{i \ge 0} a_it^i}\br{\sum_{j \ge 0} b_jt^j} = \sum_{i \ge 0} \br{\sum_{j + k = i, \ j \ge 0, \ k \ge 0} a_jb_k}x^i. $$
Define
$$ R\sbr{\sbr{t_1, \dots, t_n}} = R\sbr{\sbr{t_1}} \dots \sbr{\sbr{t_n}}. $$
\end{definition}

In $ R\sbr{\sbr{t}} $ many products equal one unlike in $ R\sbr{t} $, for example $ \br{1 - t}\br{1 + t + \dots} = 1 $.

\section{Zero-divisors, nilpotents, units}

\begin{definition}
Let $ A $ be a ring. An element $ x \in A $ is a \textbf{zero-divisor} if $ x \ne 0 $ but $ xy = 0 $ for some $ y \ne 0 $ in $ A $. A ring without zero-divisors is called an \textbf{integral domain}. An element $ x \in A $ is \textbf{nilpotent} if $ x^n = 0 $ for some $ n \in \ZZ_{> 0} $. A \textbf{unit} $ x \in A $ is an element such that $ xy = 1 $ for some $ y \in A $. The units of $ A $ form a group under multiplication, denoted by $ A^* $, or $ A^\times $.
\end{definition}

\begin{definition}
Let $ x \in A $. Then the set
$$ \abr{x} = \cbr{xy \st y \in A} $$
is an ideal. Such ideals are called \textbf{principal ideals}.
\end{definition}

\begin{remark*}
$ x \in A^* $ if and only if $ \abr{x} = A $, and $ R $ is a field if and only if $ R^* = R \setminus \cbr{0} $.
\end{remark*}

\begin{proposition}
Let $ A $ be a non-zero ring. Then the following are equivalent.
\begin{enumerate}
\item $ A $ is a field.
\item There are no ideals in $ A $ other than $ \abr{0} $ and $ A $.
\item Every non-zero homomorphism $ f : A \to B $ is injective.
\end{enumerate}
\end{proposition}

\begin{proof}
\hfill
\begin{itemize}[leftmargin=0.5in]
\item[$ 1 \implies 2 $] Clear.
\item[$ 2 \implies 3 $] $ \Ker f \subset A $ is an ideal. Since $ f \ne 0 $, $ \Ker f \ne A $. Hence $ \Ker f = 0 $.
\item[$ 3 \implies 1 $] Take any $ x \ne 0 $ in $ A $. Look at $ \abr{x} $. Define $ B = A / \abr{x} $. Then take $ f : A \to B $ to be the natural surjective map. If $ f $ is not identically zero, we get a contradiction with $ 3 $.
\end{itemize}
\end{proof}

\pagebreak

\section{Prime ideals and maximal ideals}

\begin{definition}
An ideal $ I \subset A $ is called \textbf{prime} if $ I \ne A $ and if whenever $ xy \in I $, then $ x \in I $ or $ y \in I $. An ideal $ J \subset A $ is called \textbf{maximal} if there is no ideal $ J' $ such that $ J \subsetneq J' \subsetneq A $.
\end{definition}

\begin{notation*}
The set of prime ideals of $ A $ is called the \textbf{spectrum} of $ A $ and is denoted by $ \Spec A $.
\end{notation*}

\begin{lemma}
An ideal $ I \subset A $ is prime if and only if $ A / I $ is an integral domain.
\end{lemma}

\begin{proof}
Obvious.
\end{proof}

\begin{lemma}
An ideal $ J \subset A $ is maximal if and only if $ A / J $ is a field.
\end{lemma}

\begin{proof}
Obvious.
\end{proof}

\begin{proposition}
If $ f : A \to B $ is a ring homomorphism and $ I \subset B $ is a prime ideal, then $ f^{-1}\br{I} $ is a prime ideal of $ A $.
\end{proposition}

\begin{proof}
It is easy to see that $ f^{-1}\br{I} $ is an ideal in $ A $. Suppose $ xy \in f^{-1}\br{I} $ for some $ x, y \in A $. Then $ f\br{x}f\br{y} = f\br{xy} \in I $. Since $ I $ is prime, $ f\br{x} \in I $ or $ f\br{y} \in I $, so $ x \in f^{-1}\br{I} $ or $ y \in f^{-1}\br{I} $.
\end{proof}

So we get a canonical map
$$ \function[f^*]{\Spec B}{\Spec A}{I \subset B}{f^{-1}\br{I} \subset A}. $$

\lecture{3}{Wednesday}{09/10/19}

\begin{remark}
If $ f : A \to B $ is a ring homomorphism, then $ f^{-1}\br{\ppp} $, where $ \ppp \subset B $ is a prime ideal, is a prime ideal. But this is false for maximal ideals. Let $ A = \ZZ $, let $ B = \QQ $, and let $ f\br{x} = x $. Then $ \abr{0} \subset \QQ $ is a maximal ideal and $ f^{-1}\br{\abr{0}} = \abr{0} \subset \ZZ $ is not a maximal ideal. For example, $ \abr{0} \subsetneq \abr{2} \subsetneq \ZZ $.
\end{remark}

\begin{theorem}
\label{thm:4.6}
Let $ A $ be a non-zero ring. Then $ A $ has at least one maximal ideal. In particular, $ \Spec A $ is not empty.
\end{theorem}

The proof is based on Zorn's lemma. Let $ S $ be a set. Then a \textbf{partial order} is a binary relation $ \le $ such that
\begin{itemize}
\item $ x \le x $ for all $ x \in S $,
\item $ x \le y \le z $ implies that $ x \le z $, and
\item $ x \le y $ and $ y \le x $ imply that $ x = y $,
\end{itemize}
where not all pairs are comparable. A \textbf{chain} $ T \subset S $ is a subset in which every two elements are comparable.

\begin{lemma}[Zorn]
Suppose that $ S $ is a partially ordered set such that every chain $ T \subset S $ has an upper bound, that is an element $ t \in S $ such that $ x \le t $ for all $ x \in T $. Then $ S $ has a maximal element, that is there exists $ s \in S $ such that if $ x \in S $ and $ x \ge s $, then $ x = s $.
\end{lemma}

Zorn's lemma is equivalent to the axiom of choice.

\begin{proof}[Proof of Theorem \ref{thm:4.6}]
Let $ \Sigma $ be the set of all ideals of $ A $ which are not equal to $ A $. Then $ \abr{0} \in \Sigma $, so $ \Sigma \ne \emptyset $. Equip $ \Sigma $ with partial order given by inclusion. Enough to check the assumption of Zorn's lemma. Suppose $ T $ is a chain of ideals, so it is a collection of ideals $ J_i $ for $ i \in T $. Consider instead
$$ I = \bigcup_{i \in T} J_i. $$
Claim that $ T $ is a chain implies that $ I $ is an ideal. Then $ x \in I $ implies that $ x \in J_i $ for some $ i $. Take any $ x, y \in I $. Then $ x \in J_i $ and $ y \in J_k $ for some $ i, k \in T $, so $ T $ is a chain, hence $ i \le k $ or $ k \le i $, so $ J_i \subset J_k $ or $ J_k \subset J_i $. Without loss of generality assume $ J_i \subset J_k $. Then $ x, y \in J_k $, so $ x + y \in J_k \subset I $. Clearly, $ I $ is an upper bound.
\end{proof}

\pagebreak

\begin{corollary}
\label{cor:4.8}
Any ideal of $ A $ is contained in a maximal ideal of $ A $.
\end{corollary}

\begin{proof}
If $ I \subset A $ is an ideal, apply Theorem \ref{thm:4.6} to $ A / I $.
\end{proof}

\begin{corollary}
\label{cor:4.9}
Any non-unit of $ A $ is contained in a maximal ideal.
\end{corollary}

\begin{proof}
Apply Corollary \ref{cor:4.8} to $ \abr{a} $.
\end{proof}

\begin{example*}
The maximal ideals of $ \ZZ $ are $ \abr{p} $, where $ p $ is prime.
\end{example*}

\begin{definition}
A ring $ A $ is \textbf{local} if $ A $ has exactly one maximal ideal.
\end{definition}

\begin{example*}
Any field is a local ring. If $ k $ is a field, then $ k\sbr{\sbr{t}} $ is a local ring.
\end{example*}

\begin{lemma}[Prime avoidance]
Let $ A $ be a ring and let $ \ppp \subset A $ be a prime ideal. Suppose that $ I_1, \dots, I_n $ are ideals in $ A $ such that $ \bigcap_{j = 1}^n I_j \subset \ppp $. Then $ I_j \subset \ppp $ for some $ j $. If, moreover, $ \bigcap_{j = 1}^k I_j = \ppp $, then $ I_j = \ppp $ for some $ j $.
\end{lemma}

\begin{proof}
Suppose that $ I_j $ is not a subset of $ \ppp $ for any $ j $. Then there exists $ x_j \in I_j $ such that $ x_j \notin \ppp $. Hence
$$ x_1, \dots, x_n \in I_1 \dots I_n \subset \bigcap_{j = 1}^n I_j \subset \ppp, $$
so $ x_1\br{x_2 \dots x_n} \in \ppp $. Then $ x_1 \notin \ppp $ implies that $ x_2 \dots x_n \in \ppp $. Since $ \ppp $ is prime we get a contradiction. For the second claim, we know that some $ I_j \subset \ppp $. But $ \ppp = \bigcap_{j = 1}^k I_j \subset I_k $ for all $ k $. Hence $ \ppp = I_j $.
\end{proof}

\section{Nilradical and the Jacobson radical}

\lecture{4}{Thursday}{10/10/19}

\begin{proposition}
The set $ \NNN\br{A} $ consisting of all nilpotents of the ring $ A $ and zero is an ideal. Then $ \NNN\br{A} $ is called the \textbf{nilradical} of $ A $. The quotient $ A / \NNN\br{A} $ has no nilpotents.
\end{proposition}

\begin{proof}
Suppose $ x \in A $ is nilpotent, so $ x^n = 0 $. For any $ a \in A $, $ \br{ax}^n = a^nx^n = 0 $. Let $ x $ and $ y $ be nilpotents. Say $ x^n = y^m = 0 $. Then
$$ \br{x + y}^{n + m} = \sum_{i, j \ge 0, \ i + j = n + m} a_{ij}x^iy^j, \qquad a_{ij} \in A. $$
Clearly, either $ i \ge n $ or $ j \ge m $. Then $ a_{ij}x^iy^j = 0 $. Therefore, $ \br{x + y}^{n + m} = 0 $, hence $ x + y \in \NNN\br{A} $. If $ x + \NNN\br{A} $ is nilpotent in $ A / \NNN\br{A} $, then $ x^n + \NNN\br{A} = \NNN\br{A} $ is the trivial coset. Hence $ x^n \in \NNN\br{A} $. Thus $ \br{x^n}^m = 0 $ for some $ m $.
\end{proof}

\begin{definition}
A ring $ A $ such that $ \NNN\br{A} = 0 $ is called a \textbf{reduced ring}.
\end{definition}

\begin{proposition}
\label{prop:5.3}
$ \NNN\br{A} $ is the intersection of all prime ideals of $ A $.
\end{proposition}

\begin{proof}
\hfill
\begin{itemize}
\item[$ \subset $] Let $ I $ be the intersection of all prime ideals of $ A $. Let $ f \in A $ be such that $ f^n = 0 $. Take any prime ideal $ \ppp \subset A $. We know that $ f^n = 0 \in \ppp $. Then $ f\br{f \dots f} \in \ppp $ and $ \ppp $ prime implies that $ f \in \ppp $, so $ f \in I $.
\item[$ \supset $] Let us prove the converse. Suppose $ f $ is not nilpotent, so $ f^n \ne 0 $ for all $ n \ge 1 $. We will show that there exists a prime ideal $ \ppp \subset A $ that does not contain $ f $. Let us consider all ideals of $ A $ that do not contain $ f^m $, where $ m \in \ZZ_{> 0} $. Let $ \Sigma $ be the set of ideals $ J \subset A $ such that
$$ J \cap \cbr{f^m \st m \ge 1} = \emptyset. $$
The zero ideal $ \abr{0} $ is in $ \Sigma $. So $ \Sigma \ne \emptyset $. Equip $ \Sigma $ with a partial order given by inclusion. Applying Zorn's lemma we obtain that $ \Sigma $ contains a maximal element. Call it $ \ppp $. By construction, $ \ppp \cap \cbr{f^m \st m \ge 1} = \emptyset $, so $ f \notin \ppp $. It remains to prove that $ \ppp $ is prime. Enough to prove that if $ x \notin \ppp $ and $ y \notin \ppp $, then $ xy \notin \ppp $. Consider the ideal $ \ppp + \abr{x} \supsetneq \ppp $. Since $ \ppp $ is maximal in $ \Sigma $, thus $ \ppp + \abr{x} $ is not in $ \Sigma $. By definition of $ \Sigma $ there exists $ n \ge 1 $ such that $ f^n \in \ppp + \abr{x} $. Similarly, there exists $ m \ge 1 $ such that $ f^m \in \ppp + \abr{y} $. Then $ \br{\ppp + \abr{x}}\br{\ppp + \abr{y}} \subset \ppp + \abr{xy} $. In particular, $ f^{n + m} = f^n \cdot f^m \in \ppp + \abr{xy} $. If $ xy \in \ppp $, then $ f^{n + m} \in \ppp $, which is not possible. Therefore, $ xy \notin \ppp $. So $ \ppp $ is a prime ideal that does not contain $ f $.
\end{itemize}
\end{proof}

\pagebreak

\begin{definition}
The \textbf{Jacobson radical} $ \JJJ\br{A} $ is the intersection of all maximal ideals of $ A $.
\end{definition}

\begin{proposition}
$ x \in \JJJ\br{A} $ if and only if $ 1 - xy \in A^* $ for all $ y \in A $.
\end{proposition}

\begin{proof}
\hfill
\begin{itemize}
\item[$ \implies $] Let $ x \in \JJJ\br{A} $. Suppose there exists $ y \in A $ such that $ 1 - xy $ is not a unit. By Corollary \ref{cor:4.9} every non-unit is contained in a maximal ideal. Say $ M \subset A $ is a maximal ideal and $ 1 - xy \in M $. But $ x \in \JJJ\br{A} \subset M $. Then $ 1 = \br{1 - xy} + xy \in M $, but then $ M \ne A $. A contradiction.
\item[$ \impliedby $] Given $ x \in A $ such that $ 1 - xy \in A^* $ for all $ y \in A $, we must have $ x \in \JJJ\br{A} $. If $ x \notin \JJJ\br{A} $, then there exists a maximal ideal $ M \subset A $ such that $ x \notin M $. Then $ M + \abr{x} = A \ni 1 $. Thus $ 1 = m + xy $, where $ y \in A $. But by assumption $ 1 - xy \in A^* $, so $ m \in A^* $. But then $ M = A $. A contradiction.
\end{itemize}
\end{proof}

\begin{definition}
Let $ I $ be an ideal of $ A $. The \textbf{radical} of $ I $ is the set
$$ \rad I = \cbr{x \in A \st \exists n \ge 1, \ x^n \in I}. $$
\end{definition}

\begin{proposition}
The radical of $ I $ is the intersection of all prime ideals of $ A $ that contain $ I $.
\end{proposition}

\begin{proof}
Apply Proposition \ref{prop:5.3} to $ A / I $.
\end{proof}

\lecture{5}{Tuesday}{15/10/19}

\begin{definition}
Let $ I $ be an indexing set. For each $ i \in I $ we are given a ring $ R_i $. Consider the product set $ \prod_{i \in I} R_i $. This is $ \br{x_i}_{i \in I} $ for $ x_i \in R_i $. Define
$$ 0 = \br{0}_{i \in I} \in \prod_{i \in I} R_i, \qquad 1 = \br{1}_{i \in I} \in \prod_{i \in I} R_i. $$
Define addition and multiplication coordinate-wise, so
$$ \br{a_i}_{i \in I} + \br{b_i}_{i \in I} = \br{a_i + b_i}_{i \in I}, \qquad \br{a_i}_{i \in I} \cdot \br{b_i}_{i \in I} = \br{a_i \cdot b_i}_{i \in I}, \qquad \br{a_i}_{i \in I}, \br{b_i}_{i \in I} \in \prod_{i \in I} R_i. $$
Then $ \prod_{i \in I} R_i $ is a ring, the \textbf{product of rings}.
\end{definition}

A warning is if $ I $ has at least two elements, then $ \prod_{i \in I} R_i $ has zero-divisors.

\begin{example*}
$ R_1 \times R_2 $ has $ \br{1, 0} \cdot \br{0, 1} = \br{0, 0} = 0 $.
\end{example*}

If $ h_i : R \to R_i $ is a ring homomorphism for $ i \in I $, then $ \br{h_i}_{i \in I} $ is a ring homomorphism $ R \to \prod_{i \in I} R_i $.

\begin{remark}
Let $ \ppp_i $ for $ i \in I $ be all prime ideals of $ R $. Let $ h_i : R \to R / \ppp_i $. Then
$$ h = \br{h_i}_{i \in I} : R \to \prod_{i \in I} R / \ppp_i $$
is a homomorphism, and
$$ \Ker h = \bigcap_{i \in I} \Ker h_i = \bigcap_{i \in I} \ppp_i = \NNN\br{R}. $$
So there is an injective map
$$ R / \NNN\br{R} \hookrightarrow \prod_{i \in I} R / \ppp_i, $$
a product of integral domains. Now take $ f_j : R \to R / M_j $, so if we take the indexing set $ J $ to be the set of all maximal ideals of $ R $, then we obtain an injective map
$$ R / \JJJ\br{R} \hookrightarrow \prod_{j \in J} R / M_j, $$
a product of fields.
\end{remark}

\pagebreak

\section{Localisation of rings}

\begin{example*}
Fix a prime $ p $. Then
$$ \ZZ \subset \cbr{\dfrac{m}{p^k} \st m \in \ZZ, \ k \in \ZZ_{\ge 0}} \subset \QQ. $$
\end{example*}

\begin{definition}
A subset $ S $ of a ring $ A $ is called a \textbf{multiplicative set} if $ 1 \in S $ and $ 0 \notin S $, and $ S $ is closed under multiplication.
\end{definition}

\begin{example}
\hfill
\begin{itemize}
\item Let $ a \in A $ be a non-nilpotent. Then $ \cbr{1, a, \dots} $ is a multiplicative set.
\item Let $ \ppp \subsetneq A $ be a prime ideal. Then $ A \setminus \ppp $ is a multiplicative set. Indeed, if $ x \notin \ppp $ and $ y \notin \ppp $ then $ xy \notin \ppp $ by the definition of a prime ideal.
\item If we have a family $ \ppp_i $ for $ i \in I $ of prime ideals, then $ A \setminus \bigcup_{i \in I} \ppp_i $ is a multiplicative set.
\item $ A^* $ is a multiplicative set.
\item All non-zero-divisors in $ A $ form a multiplicative set.
\item Let $ I \subsetneq A $ be an ideal. Then $ 1 + I = \cbr{1 + x \st x \in I} $ is a multiplicative set.
\end{itemize}
\end{example}

\begin{definition}
Consider $ A \times S $ and the equivalence relation on $ A \times S $ defined as
$$ \br{a, s} \sim \br{b, t} \qquad \iff \qquad \exists u \in S, \ u\br{at - bs} = 0. $$
Check that this is indeed an equivalence relation. \footnote{Exercise} The following is some notation.
\begin{itemize}
\item The equivalence class of $ \br{a, s} $ is written as $ a / s $. For example, if $ t \in S $, then $ a / s = at / st $.
\item The set of equivalence classes is denoted by $ S^{-1}A $.
\end{itemize}
Define
$$ \dfrac{a}{s} + \dfrac{b}{t} = \dfrac{at + bs}{st}, \qquad \dfrac{a}{s} \cdot \dfrac{b}{t} = \dfrac{ab}{st}, \qquad a, b \in A, \qquad s, t \in S. $$
Need to check that these operations are well-defined. \footnote{Exercise} Define $ \tfrac{0}{1} $ as the zero of $ S^{-1}A $, and $ \tfrac{1}{1} $ as the one of $ S^{-1}A $. Then $ S^{-1}A $ is a ring, the \textbf{localisation of $ A $ with respect to $ S $}.
\end{definition}

\begin{lemma}
There is a ring homomorphism
$$ \function[f]{A}{S^{-1}A}{x}{\dfrac{x}{1}}. $$
This $ f $ is injective if and only if $ S $ has no zero-divisors.
\end{lemma}

\begin{proof}
If $ S $ contains a zero-divisor, say $ u $, then there exists $ a \in A $ for $ a \ne 0 $ such that $ ua = 0 $. Then
$$ f\br{a} = \dfrac{a}{1} = \dfrac{au}{u} = \dfrac{0}{u} = 0. $$
So $ \Ker f $ contains $ a $, hence $ f $ is not injective. If $ f $ has no zero-divisors, then $ u \cdot a = u\br{a - 0} \ne 0 $ if $ a \ne 0 $ and any $ u \in S $. Hence $ f\br{a} \ne 0 $.
\end{proof}

\lecture{6}{Thursday}{16/10/19}

If $ A $ is an integral domain, then $ \Ker f = 0 $. So $ A \hookrightarrow S^{-1}A $.

\pagebreak

\begin{example*}
Let $ R = \ZZ $.
\begin{itemize}
\item If $ S = \cbr{1, a, \dots} $, then
$$ S^{-1}\ZZ = \cbr{\dfrac{n}{a^m} \st n \in \ZZ, \ m \in \ZZ_{\ge 0}}. $$
\item If $ S = \ZZ \setminus p\ZZ $, then
$$ S^{-1}\ZZ = \cbr{\dfrac{n}{m} \st p \nmid m}. $$
\item If $ S = \ZZ \setminus \bigcup_{p_i \ \text{prime}} p_i\ZZ $, then
$$ S^{-1}\ZZ = \cbr{\dfrac{n}{m} \st p_i \nmid m}. $$
\item If $ S = \ZZ^* = \cbr{\pm 1} $, then $ S^{-1}\ZZ = \ZZ $.
\item If $ S = \cbr{\text{all non-zero elements}} $, then $ S^{-1}\ZZ = \QQ $.
\item If $ S = \cbr{1 + I \st I \subset \ZZ \ \text{ideal}} = \cbr{1 + nk \st k \in \ZZ} $, then
$$ S^{-1}\ZZ = \cbr{\dfrac{m}{1 + nk} \st m, k \in \ZZ}, $$
where $ n $ is fixed.
\end{itemize}
\end{example*}

\begin{example*}
Let $ R = k\sbr{x} $, where $ k $ is a field.
\begin{itemize}
\item If $ S = k\sbr{x}^* = k^* $, then $ S^{-1}k\sbr{x} = k\sbr{x} $.
\item If $ S = \cbr{\text{all non-zero elements}} $, then
$$ S^{-1}k\sbr{x} = k\br{x} = \cbr{\dfrac{f\br{x}}{g\br{x}} \st g\br{x} \ \text{arbitrary non-zero polynomial}}. $$
\end{itemize}
\end{example*}

\begin{example}
Let $ k $ be a field, and let $ A = k\sbr{x, y} / \abr{xy} $. Note that $ A $ has zero-divisors, since $ xy = 0 $ in $ A $, but $ x \ne 0 $ in $ A $ and $ y \ne 0 $ in $ A $. Then $ S = \cbr{1, x, \dots} $ is a multiplicative set, since $ x^n \ne 0 $ in $ A $ for $ n = 1, 2, \dots $, because no power of the polynomial $ x $ is in $ \abr{xy} $. What is $ S^{-1}A $? Let $ f : A \to S^{-1}A $. Then $ a \in \Ker f $ if and only if $ a / 1 = 0 / 1 $, if and only if $ u \cdot \br{a \cdot 1 - 0 \cdot 1} = 0 $ for some $ u \in S $, if and only if $ ua = 0 $. Let $ a \ne 0 $. Then $ u = 1 $ is not interesting. Take $ u = x $ and $ a = y $, then $ xy = 0 $, hence $ y \in \Ker f $. Then $ f $ is a homomorphism, hence $ \Ker f $ is an ideal. So $ \abr{y} = yA \subset \Ker f $. In general,
$$ a = \sum_{i, j \ge 0} a_{ij}x^iy^j \equiv a_{00} + \sum_{i \ge 1} a_{i0}x^i + \sum_{j \ge 1} a_{0j}y^j \mod \abr{xy}. $$
Then $ \Ker f = yA = \abr{y} $, since $ \sum_{j \ge 1} a_{0j}y^j $ goes to zero, since it is annihilated by $ x $, and $ x^n \cdot \sum_{i \ge 0} a_ix^i $ is never zero in $ A $. Thus $ f\br{A} = k\sbr{x} $, and
$$ S^{-1}A = \cbr{\dfrac{f\br{x}}{x^n} \st f\br{x} \in k\sbr{x}, \ n \ge 0} = k\sbr{x, x^{-1}} = \cbr{\sum_{i \in \ZZ, \ a_i = 0 \ \text{for almost all} \ i} a_ix^i \st a_i \in k}. $$
\end{example}

\lecture{7}{Thursday}{17/10/19}

\begin{lemma}[Universal property of localisation]
Let $ A $ be a ring, and $ S \subset A $ a multiplicative set. Let $ g : A \to B $ be a ring homomorphism such that $ g\br{s} $ is a unit in $ B $ for all $ s \in S $. Then there exists a unique ring homomorphism $ h : S^{-1}A \to B $ such that $ g = h \circ f $ where $ f : A \to S^{-1}A $ is the canonical map, so
$$
\begin{tikzcd}
A \arrow[swap]{d}{f} \arrow{dr}{g} & \\
S^{-1}A \arrow[swap]{r}{\exists !h} & B
\end{tikzcd}.
$$
\end{lemma}

\pagebreak

\begin{proof}
Define
$$ \function[h]{S^{-1}A}{B}{\dfrac{a}{s}}{\dfrac{g\br{a}}{g\br{s}}}, \qquad a \in A, \qquad s \in S. $$
This is well-defined, that is if $ a / s = b / t $ then $ g\br{a}g\br{s}^{-1} = g\br{b}g\br{t}^{-1} $. \footnote{Exercise} This is a ring homomorphism. \footnote{Exercise} Now easy to check that
$$ \br{h \circ f}\br{a} = h\br{\dfrac{a}{1}} = \dfrac{g\br{a}}{g\br{1}} = \dfrac{g\br{a}}{1} = g\br{a}, \qquad a \in A. $$
Moreover, if $ h' : S^{-1}A \to B $ and $ g = h' \circ f $ then for all $ a \in A $ we have $ \br{h' \circ f}\br{a} = g\br{a} $. Since $ h' $ is a ring homomorphism, for all $ s \in S $, $ h'\br{1 / s} = 1 / h'\br{s / 1} = 1 / g\br{s} $. Hence
$$ h'\br{\dfrac{a}{s}} = h'\br{\dfrac{a}{1}}h'\br{\dfrac{1}{s}} = \dfrac{h'\br{f\br{a}}}{h'\br{f\br{s}}} = \dfrac{g\br{a}}{g\br{s}} = h\br{\dfrac{a}{s}}. $$
\end{proof}

For all ideal $ I \subseteq A $, set
$$ S^{-1}I = \cbr{\dfrac{i}{s} \in S^{-1}A \st i \in I, \ s \in S}, $$
the ideal of $ S^{-1}A $ generated by $ f\br{I} $.

\begin{proposition}
Let $ S \subset A $ be a multiplicative subset, and let $ I_1, \dots, I_n $ be ideals of $ A $. Then
\begin{enumerate}
\item $ S^{-1}\br{I_1 + \dots + I_n} = S^{-1}I_1 + \dots + S^{-1}I_n $,
\item $ S^{-1}\br{I_1 \cdot \dots \cdot I_n} = S^{-1}I_1 \cdot \dots \cdot S^{-1}I_n $,
\item $ S^{-1}\br{\bigcap_{i = 1}^n I_i} = \bigcap_{j = 1}^n S^{-1}I_j $, and
\item $ S^{-1}\br{\rad I} = \rad S^{-1}I $ for every ideal $ I $.
\end{enumerate}
\end{proposition}

\begin{proof}
Exercise. \footnote{Exercise}
\end{proof}

There is a map
$$ \cbr{\text{ideals} \ I \ \text{of} \ A} \to \cbr{\text{ideals} \ S^{-1}I \ \text{of} \ S^{-1}A}. $$

\begin{proposition}
Every ideal of $ S^{-1}A $ is of the form $ S^{-1}I $ for some ideal $ I \subseteq A $.
\end{proposition}

\begin{proof}
Let $ J $ be any ideal of $ S^{-1}A $. Define $ I = f^{-1}A $. Know $ I $ is an ideal of $ A $. Claim that $ J = S^{-1}I $. Say $ a / s \in J $. Since $ J $ is an ideal, $ s\br{a / s} \in J $, so $ a / 1 \in J $, so $ a \in I $. Hence $ a / s \in S^{-1}I $. So $ J \subseteq S^{-1}I $. Conversely, $ f\br{I} = f\br{f^{-1}\br{J}} \subseteq J $. Thus $ S^{-1}I \subseteq J $.
\end{proof}

\begin{theorem}
\label{thm:6.9}
The only prime ideals of $ S^{-1}A $ are of the form $ S^{-1}\ppp $ where $ \ppp $ is a prime ideal of $ A $ such that $ \ppp \cap S = \emptyset $. Hence there is a bijection
$$ \correspondence{\text{prime ideals of} \ S^{-1}A}{\text{prime ideals of} \ A \ \text{that do not intersect} \ S}. $$
\end{theorem}

\begin{proof}
Prove $ S^{-1}\ppp $ is prime if $ \ppp $ is prime and $ \ppp \cap S = \emptyset $. Say $ a / s \cdot b / t \in S^{-1}\ppp $ for $ a / s, b / t \in S^{-1}A $. This implies $ v\br{abu - cst} = 0 $ for some $ u, v \in S $ and $ c \in \ppp $. Hence $ abuv = cstv \in \ppp $, so $ ab \in \ppp $, as $ u $ and $ v $ are units, so $ a \in \ppp $ or $ b \in \ppp $. Hence $ S^{-1}\ppp $ is prime. Next note that $ f^{-1}\br{S^{-1}\ppp} = \ppp $, assuming $ \ppp \cap S = \emptyset $. For if $ a \in A $ lies in $ S^{-1}\ppp $ then by definition there exists $ s \in S $ such that $ sa \in \ppp $. Then $ s $ is a unit and so $ a \in \ppp $. Hence $ \ppp $ is uniquely determined by $ S^{-1}\ppp $. Now let $ \qqq $ be an arbitrary prime ideal of $ S^{-1}A $. Then certainly $ \qqq = S^{-1}I $ for $ I = f^{-1}\br{\qqq} $. But the preimage of a prime ideal is prime. So $ I $ is prime. Moreover, $ I \cap S = \emptyset $ as no $ s \in S $ is in $ \qqq $, since $ \qqq $ is prime, so $ \qqq $ contains no units.
\end{proof}

\pagebreak

\section{\texorpdfstring{$ \Spec R $}{Spec R} as a topological space}

\lecture{8}{Tuesday}{22/10/19}

A set $ X $ with a collection $ \UUU $ of subsets $ U \subset X $ is called a \textbf{topological space} if the following properties hold.
\begin{enumerate}
\item $ \UUU $ contains $ \emptyset $ and $ X $.
\item If $ U $ and $ U' $ are in $ \UUU $, then $ U \cap U' $ is in $ \UUU $.
\item If $ U_i $ are in $ \UUU $, where $ i $ is an element of an indexing set $ S $, then $ \bigcup_{i \in S} U_i $ is in $ \UUU $.
\end{enumerate}
Then the elements of $ \UUU $ are called \textbf{open subsets} of $ X $. The following is an equivalent definition. A set $ X $ with a family $ \VVV $ of subsets $ V \subset X $ is called a \textbf{topological space} if the following properties hold.
\begin{enumerate}
\item $ \VVV $ contains $ \emptyset $ and $ X $.
\item If $ V $ and $ V' $ are in $ \VVV $, then $ V \cup V' $ is in $ \VVV $.
\item If $ V_i $ are in $ \VVV $, where $ i $ is an element of an indexing set $ S $, then $ \bigcap_{i \in S} V_i $ is in $ \VVV $.
\end{enumerate}
Then the elements of $ \UUU $ are called \textbf{closed subsets} of $ X $. For the equivalence, if $ U $ is in $ \UUU $, then define the closed subsets as $ X \setminus U $ for $ U $ in $ \UUU $, and vice versa. Let $ R $ be a ring with unity. Let $ I \subset R $ be an ideal. Let $ \V_I $ be the set of all prime ideals in $ R $ that contain $ I $. Define $ \U_I = \Spec R \setminus \V_I $.

\begin{proposition}
The collection of subsets $ \V_I \subset \Spec R $, for all ideals $ I \subset R $, satisfies $ 1 $, $ 2 $, $ 3 $ of closed subsets, hence defines a topology on $ \Spec R $.
\end{proposition}

\begin{proof}
\hfill
\begin{enumerate}
\item If $ I = 0 $ is the zero ideal, then $ \V_0 = \Spec R $, all prime ideals of $ R $. If $ I = R $, then no prime ideals of $ R $ contain $ R $, so $ \V_R = \emptyset $, so $ 1 $ holds.
\item It is enough to check that $ \V_I \cup \V_J = \V_{IJ} = \V_{I \cap J} $. Note that $ IJ \subset I \cap J $. An element of $ \V_I $ is a prime ideal $ \ppp \supset I $, so $ \ppp \supset IJ $. Conversely, let $ \ppp $ be a prime ideal such that $ IJ \subset \ppp $. Claim that $ I \subset \ppp $ or $ J \subset \ppp $. Suppose not. Then there exists $ x \in I $ such that $ x \notin \ppp $ and there exists $ y \in J $ such that $ y \notin \ppp $. Then $ xy \in IJ \subset \ppp $. This contradicts the definition of prime ideals. So the claim is proved. Thus $ 2 $ holds.
\item $ J_i $ for $ i \in S $ is a collection of ideals. Claim that $ \bigcap_{i \in S} \V_{J_i} = \V_J $, where $ J = \sum_{i \in S} J_i $ is the smallest ideal of $ R $ containing all $ J_i $ for $ i \in S $. The elements of $ J $ are finite sums, where each summand is in some $ J_i $. If $ \ppp \supset J_i $ for $ i \in S $, then $ \ppp \supset J $. Conversely, if $ \ppp \supset J \supset J_i $, then $ \ppp \supset J_i $ for all $ i \in S $.
\end{enumerate}
\end{proof}

Recall that if $ f : A \to B $ is a homomorphism of rings, then $ f^* : \Spec B \to \Spec A $ sends any prime ideal $ \ppp \subset B $ to the inverse image $ f^{-1}\br{\ppp} $, which is a prime ideal in $ A $. This breaks down for maximal ideals.

\begin{example*}
Take $ f : \ZZ \to \QQ $, then $ f^{-1}\br{0} = 0 $, which is not maximal in $ \ZZ $.
\end{example*}

A map of topological spaces is \textbf{continuous} if the inverse image of any open set is open. Equivalently, the inverse images of closed sets are closed.

\begin{proposition}
$ f^* $ is a continuous map.
\end{proposition}

\begin{proof}
Let $ I $ be an ideal in $ A $. We need to show that $ \br{f^*}^{-1}\br{\V_I} = \V_J $ for some ideal $ J $ in $ B $. Let $ J $ be the smallest ideal in $ B $ containing $ f\br{I} $.
\begin{itemize}
\item[$ \subset $] Fix $ \ppp $ in $ \V_I $, a prime ideal in $ A $ such that $ \ppp \supset I $. The elements of the left hand side that are mapped to $ \ppp $ by $ f^* $ are the prime ideals $ \qqq \subset B $ such that $ \ppp = f^{-1}\br{\qqq} $. We have $ I \subset \ppp $, so $ f\br{I} \subset f\br{\ppp} \subset \qqq $, so $ J \subset \qqq $, by definition of $ J $.
\item[$ \supset $] Take any prime ideal $ \qqq \subset B $ such that $ J \subset \qqq $. We have $ I \subset f^{-1}\br{f\br{I}} \subset f^{-1}\br{J} \subset f^{-1}\br{\qqq} $, so $ f^{-1}\br{\qqq} $ is a prime ideal in $ A $ containing $ I $. This ideal is exactly $ f^*\br{\qqq} $, so $ f^*\br{\qqq} $ is in $ \V_I $. Since $ \qqq \in \br{f^*}^{-1}\br{f^*\br{\qqq}} \subset \br{f^*}^{-1}\br{\V_I} $, so we are done.
\end{itemize}
\end{proof}

\pagebreak

\lecture{9}{Wednesday}{23/10/19}

The following are particular cases.
\begin{itemize}
\item Assume $ f $ is surjective. Then $ B \cong A / \Ker f $. Then
$$ \function{\cbr{\text{prime ideals in} \ B}}{\cbr{\text{prime ideals in} \ A \ \text{containing} \ \Ker f}}{\ppp \subset B}{f^{-1}\br{\ppp}}. $$
So in this case $ f^* $ is injective and its image is $ \V_{\Ker f} $.
\item Let $ S $ be a multiplicative set in $ A $. Let $ f : A \to S^{-1}A $ be the associated canonical map. By Theorem \ref{thm:6.9} the prime ideals of $ S^{-1}A $ are $ S^{-1}\ppp $, where $ \ppp $ is a prime ideal in $ A $ such that $ \ppp \cap S = \emptyset $. Thus $ f^* : \Spec S^{-1}A \to \Spec A $ is injective and its image consists of $ \ppp \subset A $ such that $ \ppp \cap S = \emptyset $.
\end{itemize}

\begin{example*}
\hfill
\begin{itemize}
\item Let $ k $ be a field. Then $ \Spec k $ is one point.
\item Let $ R = k\sbr{x} $, an integral domain. This is a PID, so every ideal is $ \abr{p\br{x}} $, where $ p\br{x} \in k\sbr{x} $ is monic. Then $ \abr{p\br{x}} $ is prime if and only if $ p\br{x} $ is irreducible, so
$$ \Spec k\sbr{x} = \cbr{\abr{0}} \cup \cbr{\abr{p\br{x}} \st p\br{x} \ \text{is monic and irreducible}}. $$
In particular, if $ k $ is algebraically closed, such as $ k = \CC $, then
$$ \Spec k\sbr{x} = \cbr{\abr{0}} \cup \cbr{\abr{x - a} \st a \in k}. $$
\item Let $ R = \ZZ $, a PID. Then
$$ \Spec \ZZ = \cbr{\abr{0}} \cup \cbr{\abr{p} \st p \ \text{is a prime number}}. $$
\item Let $ R = \ZZ\sbr{i} $ be the Gaussian integers, a PID. The tautological map $ f : \ZZ \to \ZZ\sbr{i} $ gives rise to $ f^* : \Spec \ZZ\sbr{i} \to \Spec \ZZ $. Take a usual prime $ p $ and decompose $ p $ into a product of primes in $ \ZZ\sbr{i} $.
\begin{itemize}
\item $ 2 = \br{1 + i}\br{1 - i} = -i\br{1 + i}^2 $, where $ 1 + i $ is a prime in $ \ZZ\sbr{i} $.
\item If $ p \equiv 1 \mod 4 $, then $ p = \br{a + bi}\br{a - bi} $. In this case $ a + bi $ and $ a - bi $ are not associated primes.
\item If $ p \equiv 3 \mod 4 $, then $ p $ stays prime in $ \ZZ\sbr{i} $.
\end{itemize}
Then
$$
\begin{array}{cccl}
\Spec \ZZ\sbr{i} & \longrightarrow & \Spec \ZZ & \\
\abr{0} & \longmapsto & \abr{0} & \\
\abr{1 + i} & \longmapsto & \abr{2} & \text{ramified} \\
\abr{3} & \longmapsto & \abr{3} & \text{inert} \\
\abr{1 + 2i}, \abr{1 - 2i} & \longmapsto & \abr{5} & \text{split}
\end{array}.
$$
\item Let $ R $ be an integral domain and let $ k $ be the fraction field of $ R $, so $ f : R \hookrightarrow k $. Then $ \Spec k = \cbr{\abr{0}} $ and $ f^* : \Spec k \to \Spec R $.
\item Let $ k $ be a field, so $ f : k \hookrightarrow k\sbr{x} $. Then $ f^* : \Spec k\sbr{x} \to \Spec k $. If $ \ppp \subset k\sbr{x} $, then $ \ppp \cap k = \cbr{\abr{0}} $, otherwise if $ \ppp $ contains a unit of $ k\sbr{x} $ then $ \ppp = k\sbr{x} $. A contradiction.
\end{itemize}
\end{example*}

Usually, every point of a topological space is a closed subset. But this is not always true. Recall that if $ Y $ is a subset of a topological space $ X $, then the \textbf{closure} of $ Y $ is the smallest closed subset of $ X $ containing $ Y $. It is the same as the intersection of all closed subsets containing $ Y $. Claim that if $ \ppp \subseteq R $ is a prime ideal, then the closure of $ \ppp $ is $ \V_\ppp $. Any closed subset of $ \Spec R $ containing $ \ppp $ is $ \V_J $, where $ J \subset \ppp $. This $ \V_J $ visibly contains $ \V_\ppp $. Hence $ \V_\ppp $ is the intersection of all such $ \V_J $.

\begin{example*}
In $ \Spec \ZZ $, the point $ \abr{p} $ is closed, because $ \V_{\abr{p}} = \cbr{\abr{p}} $. The point $ \abr{0} $ is not closed, as $ \V_{\abr{0}} = \Spec \ZZ $. The closure of $ \abr{0} $ is all of $ \Spec \ZZ $.
\end{example*}

\begin{example*}
Let $ R = k\sbr{\sbr{t}} = \cbr{a_0 + a_1t + \dots \st a_i \in k} $, a local ring. Its unique maximal ideal is $ \abr{t} $. This is also a unique non-zero prime ideal. \footnote{Exercise} All ideals are $ \abr{0} $ and $ \abr{t^n} $. Then $ \Spec k\sbr{\sbr{t}} = \cbr{\abr{0}, \abr{t}} $. Similarly, $ \abr{0} $ is not a closed point, since its closure is $ \Spec k\sbr{\sbr{t}} $, and $ \abr{t} $ is a closed point.
\end{example*}

\pagebreak

\section{Determinants}

\lecture{10}{Thursday}{24/10/19}

Let $ R $ be a commutative ring with unity. Let $ A $ be a matrix $ A = \br{a_{ij}}_{i, j = 1}^n $ for $ a_{ij} \in R $. Then
$$ \det A = \sum_{\pi \in \SSS_n} \sgn \pi \cdot a_{1\pi\br{1}} \cdot \dots \cdot a_{n\pi\br{n}} \in R, $$
where $ \sgn : \SSS_n \to \cbr{\pm 1} $. Let
$$ \M_{ij} = \det\br{A \ \text{without $ j $-th column and $ i $-th row}} \in R. $$
Then
$$ \br{-1}^{j + 1}a_{i1}\M_{j1} + \dots + \br{-1}^{j + n}a_{in}\M_{jn} =
\begin{cases}
\det A & i = j \\
0 & i \ne j
\end{cases}.
$$
Define the \textbf{adjoint matrix} of $ A $ as the $ n \times n $ matrix $ A^\vee $ with entries $ \br{A^\vee}_{ij} = \br{-1}^{i + j}\M_{ji} $, so
$$ A^\vee = \br{\br{-1}^{i + j}\M_{ij}}^\intercal. $$
Then $ A \cdot A^\vee = A^\vee \cdot A = \det A \cdot \I_n $, where $ \I_n $ is the identity matrix.

\section{Modules}

\begin{definition}
Let $ A $ be a commutative ring with unity. An \textbf{$ A $-module} $ M $ is an abelian group with an additional structure $ A \times M \to M $ such that
$$ \lambda\br{x + y} = \lambda x + \lambda y, \qquad \br{\mu + \lambda}x = \mu x + \lambda x, \qquad \mu\br{\lambda x} = \br{\mu\lambda}x, \qquad 1x = x, \qquad \lambda, \mu \in R, \qquad x, y \in M. $$
\end{definition}

\begin{example}
\hfill
\begin{itemize}
\item If $ R $ is a field, then an $ R $-module is the same as a vector space.
\item If $ R = \ZZ $, then an $ R $-module is the same as an abelian group. Remark that if $ G $ is an abelian group then $ n \cdot g = g + \dots + g $.
\item If $ R $ is any ring, then subgroups of $ R $ that are $ R $-modules are the same as ideals.
\item If $ k $ is a field, then $ k\sbr{x} $-modules are vector spaces $ V $ over $ k $ equipped with a linear transformation $ L : V \to V $. Here $ x $ acts on $ V $ as $ L $.
\end{itemize}
\end{example}

\begin{definition}
If $ M $ and $ N $ are $ R $-modules, then a \textbf{homomorphism of $ R $-modules} $ f : M \to N $ is a homomorphism of abelian groups such that $ f\br{rx} = rf\br{x} $ for all $ x \in M $ and all $ r \in R $.
\end{definition}

\begin{definition}
Let $ \Hom_R\br{M, N} $ be the set of $ R $-module homomorphisms $ M \to N $.
\end{definition}

This is an abelian group. Moreover, it is an $ R $-module. If $ r \in R $ and $ f \in \Hom_R\br{M, N} $ then $ r \cdot f $ sends $ x \in M $ to $ rf\br{x} \in N $. Warning that if $ R $ is not commutative $ \Hom_R\br{M, N} $ is just an abelian group.

\begin{definition}
Let $ M $ and $ N $ be submodules of an $ R $-module. Define
$$ \br{N : M} = \cbr{r \in R \st rM \subset N}. $$
\end{definition}

This is an ideal in $ R $.

\begin{example*}
The \textbf{annihilator} of $ M $ is
$$ \br{0 : M} = \cbr{r \in R \st rM = 0} = \Ann M. $$
\end{example*}

\pagebreak

\begin{definition}
An $ R $-module $ M $ is \textbf{finitely generated} if there are elements $ x_1, \dots, x_n \in M $ such that for any $ m \in M $ there are $ r_1, \dots, r_n \in R $ such that $ m = r_1x_1 + \dots + r_nx_n $.
\end{definition}

\begin{example*}
There is a \textbf{free} finitely generated module
$$ R^{\oplus n} = \cbr{\br{t_1, \dots, t_n} \st t_i \in R}, $$
with coordinate-wise addition and multiplication.
\end{example*}

\begin{remark*}
Any finitely generated $ R $-module is a quotient of a free finitely generated $ R $-module. Indeed, define
$$ \function[f_i]{R^{\oplus n}}{M}{\br{t_1, \dots, t_n}}{t_1x_1 + \dots + t_nx_n}. $$
\end{remark*}

Comment that $ JM $ is the smallest submodule of $ M $ containing all elements $ rm $ for $ r \in J $ and $ m \in M $, so
$$ JM = \cbr{\text{finite sums} \ r_1m_1 + \dots + r_km_k} \subset M. $$

\begin{lemma}
\label{lem:9.7}
Let $ A $ be a ring. Let $ M $ be a finitely generated $ A $-module. Let $ J \subset A $ be an ideal such that $ JM = M $. Then there is an $ a \in J $ such that $ \br{1 - a}M = 0 $.
\end{lemma}

\begin{proof}
If $ M = 0 $, then it is fine. Suppose $ M \ne 0 $ and $ m_1, \dots, m_n $ are generators of $ M $. Then $ m_i \in M = JM $, so
$$ m_1 = x_{11}m_1 + \dots + x_{1n}m_n, \qquad \dots, \qquad m_n = x_{n1}m_1 + \dots + x_{nn}m_n, $$
for $ x_{ij} \in J $. Define $ X = \br{x_{ij}}_{i, j = 1}^n $. Then
$$ \threebyone{m_1}{\vdots}{m_n} = X\threebyone{m_1}{\vdots}{m_n} \qquad \iff \qquad \br{\I_n - X}\threebyone{m_1}{\vdots}{m_n} = 0. $$
Consider the adjoint matrix $ \br{\I_n - X}^\vee $. Then
$$ \br{\I_n - X}^\vee\br{\I_n - X}\threebyone{m_1}{\vdots}{m_n} = 0 \qquad \iff \qquad \det \br{\I_n - X}\threebyone{m_1}{\vdots}{m_n} = 0. $$
We have $ \det \br{\I_n - X} \in A $. Then $ \det \br{\I_n - X} $ is a product of diagonal entries $ \prod_{i = 1}^n \br{1 - x_{ii}} $, plus other terms but every non-diagonal term contains at least one factor in $ J $, so is in $ J $. Finally, $ \det \br{\I_n - X} = 1 - a $, where $ a \in J $. Now, $ \br{1 - a}m_i = 0 $ for $ i = 1, \dots, n $. Hence $ \br{1 - a}M = 0 $.
\end{proof}

\lecture{11}{Tuesday}{29/10/19}

\begin{remark*}
If $ M $ is not finitely generated then this is false, such as $ A = \ZZ $ and $ M = \QQ $. If $ p $ is a prime, then $ p\QQ = \QQ $. So for $ J = \abr{p} $ we have $ JM = M $. But no non-zero integer annihilates $ \QQ $, since $ \QQ $ is not a finitely generated $ \ZZ $-module.
\end{remark*}

\begin{corollary}
Let $ R $ be a ring and let $ M $ be a finitely generated $ R $-module. If $ f : M \to M $ is a surjective $ R $-module endomorphism, then $ f $ is an isomorphism.
\end{corollary}

\begin{proof}
Define $ A = R\sbr{t} $. Let us equip $ M $ with the structure of an $ A $-module. Define $ t \cdot m = f\br{m} $ for $ m \in M $. This makes sense because $ f\br{rx} = rf\br{x} $ for all $ r \in R $. Then $ M $ is finitely generated also as an $ A $-module. If $ f\br{M} = M $, then $ tM = M $. Take $ J = \abr{t} \subset A $. By Lemma \ref{lem:9.7} there exists $ a \in \abr{t} $ such that $ \br{1 - a}M = 0 $. Take $ v \in M $ such that $ f\br{v} = 0 $. Then $ tv = 0 $, so $ av = 0 $. Since $ \br{1 - a}v = 0 $, we conclude $ v = 0 $.
\end{proof}

\begin{theorem}[Nakayama's lemma]
Let $ A $ be a ring and let $ J \subset A $ be an ideal contained in the Jacobson radical $ \JJJ\br{A} $. If $ M $ is a finitely generated $ A $-module such that $ JM = M $, then $ M = 0 $.
\end{theorem}

\begin{proof}
Lemma \ref{lem:9.7} implies that there exists $ a \in J $ such that $ \br{1 - a}M = 0 $. But $ a \in \JJJ\br{A} $, so $ 1 - a $ is a unit in $ A $. Then there exists $ u \in A $ such that $ u\br{1 - a} = 1 $. Hence $ M = u\br{1 - a}M = 0 $.
\end{proof}

\begin{corollary}
\label{cor:9.10}
Let $ A $ be a ring and $ J $ an ideal contained in the Jacobson radical of $ A $. Suppose $ M $ is an $ A $-module, and $ N \subset M $ is a submodule such that $ M / N $ is a finitely generated $ A $-module. Then $ M = N + JM $ implies $ M = N $.
\end{corollary}

\begin{proof}
Apply Nakayama's lemma to $ M / N $. Indeed, we have $ M / N = J\br{M / N} $, so $ M / N = 0 $.
\end{proof}

\pagebreak

Recall a ring is local when it has a unique maximal ideal. The quotient is called the \textbf{residue field}.

\begin{example*}
For $ k $ a field, $ k\sbr{\sbr{t}} \supset \abr{t} $ and $ k\sbr{\sbr{t_1, \dots, t_n}} \supset \abr{t_1, \dots, t_n} $ are local rings. \footnote{Exercise}
\end{example*}

\begin{theorem}
\label{thm:9.11}
Let $ R $ be a local ring with maximal ideal $ J $ and residue field $ k = R / J $. Let $ M $ be a finitely generated $ R $-module.
\begin{enumerate}
\item $ M / JM $ is a finite-dimensional vector space over $ k $.
\item Let $ v_1, \dots, v_n $ be a basis of $ M / JM $ as a vector space over $ k $. Choose $ \widetilde{v_1}, \dots, \widetilde{v_n} \in M $ to be representatives of $ v_1, \dots, v_n $ respectively. That is, $ v_i = \widetilde{v_i} + JM $. Then $ \widetilde{v_1}, \dots, \widetilde{v_n} $ generate $ M $ as an $ R $-module. Moreover, this is a minimal set of generators of $ M $. That is, no proper subset generates $ M $.
\item All minimal sets of generators of $ M $ are obtained in this way. In particular, all such sets have $ n $ elements, where $ n = \dim_k M / JM $.
\end{enumerate}
\end{theorem}

\begin{proof}
$ J $ is the Jacobson radical of $ A $.
\begin{enumerate}
\item Any quotient of a finitely generated $ R $-module is a finitely generated $ R $-module. Hence $ M / JM $ is a finitely generated $ R $-module. But if $ x \in J $ then $ x \cdot M / JM = 0 $. So $ R $ acts on $ M / JM $ via the quotient $ k = R / J $. One says that the action of $ R $ descends to an action of $ k $. Thus $ M / JM $ is a $ k $-module, which is finitely generated. In other words, $ M / JM $ is a finite-dimensional $ k $-vector space.
\item Consider
$$ N = R\widetilde{v_1} + \dots R\widetilde{v_n} = \cbr{r_1\widetilde{v_1} + \dots + r_n\widetilde{v_n} \st r_i \in R} \subset M. $$
Then $ M / JM $ is generated by $ v_1, \dots, v_n $, hence $ M = N + JM $, since $ M / JM = N / JN $. By Corollary \ref{cor:9.10} we have $ M = N $. If a proper subset of $ \widetilde{v_1}, \dots, \widetilde{v_n} $ generates $ M $, then a proper subset of $ v_1, \dots, v_n $ generates an $ n $-dimensional vector space. A contradiction.

\lecture{12}{Wednesday}{30/10/19}

\item Suppose $ m_1, \dots, m_n $ is any minimal generating set of the $ R $-module $ M $. Consider $ \overline{m_1}, \dots, \overline{m_n} \in M / JM $. Then $ \overline{m_1}, \dots, \overline{m_n} $ span the vector space $ M / JM $. If this is not a basis, then $ M / JM $ is spanned by a proper subset of $ \overline{m_1}, \dots, \overline{m_n} $. In particular, a basis is a proper subset. By part $ 2 $ a proper subset of $ m_1, \dots, m_n $ generates $ M $. This contradicts the minimality of $ m_1, \dots, m_n $.
\end{enumerate}
\end{proof}

The moral of the story is any finitely generated module $ M $ over a local ring $ R $ has a minimal set of generators, where $ m_1, \dots, m_n $ is a minimal set of generators of $ M $ if and only if $ \overline{m_1}, \dots, \overline{m_n} $ is a basis of the $ k $-vector space $ M / JM $, and $ n $ is well-defined.

\section{Localisation of modules}

Let $ A $ be a ring with a multiplicative set $ S \subset A $.

\begin{definition}
Let $ M $ be an $ A $-module. Consider the set $ M \times S $. Equip it with a relation $ \sim $ such that
$$ \br{m, s} \sim \br{n, t} \qquad \iff \qquad \exists u \in S, \ u\br{mt - ns} = 0. $$
This is an equivalence relation.
\begin{itemize}
\item Define $ S^{-1}M $ as the set of equivalence classes.
\item The equivalence class of $ \br{m, s} $ is written as $ m / s $.
\end{itemize}
Turn $ S^{-1}M $ into a $ S^{-1}A $-module as follows. Let $ \tfrac{0}{1}, \tfrac{1}{1} \in S^{-1}M $, and
$$ \dfrac{m}{s} + \dfrac{b}{t} = \dfrac{mt + bs}{st}, \qquad \dfrac{a}{s} \cdot \dfrac{m}{t} = \dfrac{am}{st}, \qquad a \in A, \qquad m \in M, \qquad s \in S, \qquad t \in S. $$
This is the \textbf{localisation of $ M $ with respect to $ S $}.
\end{definition}

\pagebreak

Now let us consider a particular kind of multiplicative set.

\begin{definition}
Let $ \ppp \subset A $ be a prime ideal. Let $ S = A \setminus \ppp $. This is a multiplicative set. Then the \textbf{localisation $ S^{-1}A $ of $ A $ at $ \ppp $} is written as $ A_\ppp $.
\end{definition}

\begin{theorem}
Let $ \ppp \subset A $ be a prime ideal. Then $ A_\ppp $ is a local ring with unique maximal ideal
$$ \ppp A_\ppp = \cbr{\dfrac{x}{y} \st x \in \ppp, \ y \notin \ppp}. $$
\end{theorem}

\begin{remark*}
In general, a ring $ R $ with an ideal $ J $ is a local ring with maximal ideal $ J $ if and only if $ R^* = R \setminus J $. Indeed, if $ J \subset R $ is a maximal ideal, then for any $ x \in R \setminus J $, $ J + xR $ contains one. This forces $ x $ to be a unit. Conversely, if $ R^* = R \setminus J $ then $ J $ is maximal and is a unique maximal ideal.
\end{remark*}

\begin{proof}
Suppose $ a / s \in A_\ppp^* $. Then $ a / s \cdot b / t = 1 / 1 $ for some $ b \in A $ and $ t \in A \setminus \ppp $. By definition $ u\br{ab - st} = 0 $ for $ u \in A \setminus \ppp $, so $ uab = ust \notin \ppp $, since all factors are in $ S = A \setminus \ppp $. Therefore, $ a \notin \ppp $, hence $ a / s \notin \ppp A_\ppp $. Conversely, if $ a / s \notin \ppp A_\ppp $ for $ s \notin \ppp $, then $ a \notin \ppp $. Thus $ a / s $ is a unit in $ A_\ppp $ because $ a / s \cdot s / a = 1 $.
\end{proof}

\begin{example}
Let $ R = \ZZ $ and $ \ppp = \abr{p} $. Then
$$ p\ZZ_{\abr{p}} = \cbr{\dfrac{x}{y} \st p \mid x, \ p \nmid y} \subset \cbr{\dfrac{x}{y} \st x \in \ZZ, \ p \nmid y} = \ZZ_{\abr{p}} $$
is the unique maximal ideal.
\end{example}

\begin{proposition}
Let $ M $ be an $ A $-module. Consider $ M_\ppp = \br{A \setminus \ppp}^{-1}M $, where $ \ppp \subset A $ is a maximal ideal. Then $ M = 0 $ if and only if $ M_\ppp = 0 $ for any maximal ideal $ \ppp $.
\end{proposition}

\begin{proof}
\hfill
\begin{itemize}
\item[$ \implies $] Obvious.
\item[$ \impliedby $] Assume $ M \ne 0 $, so there exists $ x \in M $ such that $ x \ne 0 $. Define
$$ I = \Ann x = \cbr{a \in A \st ax = 0}, $$
so $ 1 \notin I $ since $ x \ne 0 $. Choose a maximal ideal $ \ppp $ containing $ I $. If $ M_\ppp = 0 $, then $ x / 1 = 0 $. We know that $ x \in \Ker \br{M \to M_\ppp} $ if and only if $ ux = 0 $ for some $ u \in A \setminus \ppp $. A contradiction, since $ I \subset \ppp $.
\end{itemize}
\end{proof}

The following is a corollary. Let $ M $ be a finitely generated $ A $-module. Then $ m_1, \dots, m_n $ generate $ M $ if and only if $ m_1, \dots, m_n $ generate the $ A_\ppp $-module $ M_\ppp $ for any maximal ideal $ \ppp \subset A $. By Theorem \ref{thm:9.11} applied to $ A_\ppp $, this is if and only if the images $ \overline{m_1}, \dots, \overline{m_n} $ in $ M / \ppp M \cong M_\ppp / \ppp M_\ppp $ generate the $ k\br{\ppp} $-vector space for every maximal ideal $ \ppp \subset A $, where $ k\br{\ppp} = A / \ppp $.

\begin{corollary}
Assume $ A $ is an integral domain with field of fractions $ K $. In this case $ A $ is a subring of $ K $. For any prime ideal $ \ppp \subset A $ the local ring $ A_\ppp $ is also a subring of $ K $. Then
$$ A = \bigcap_{\text{all prime ideals} \ \ppp \subset A} A_\ppp, $$
as subsets of $ K $.
\end{corollary}

\begin{proof}
Clearly, $ A \subset A_\ppp $, so the left hand side is in the right hand side. Let us prove that if $ x \in K $ is contained in each $ A_\ppp $, then $ x \in A $. Consider
$$ I = \cbr{a \in A \st ax \in A}. $$
Visibly, $ I $ is an ideal in $ A $. We are given that $ x = m / s $, where $ m \in A $ and $ s \in A \setminus \ppp $. Hence $ s \in I $. So $ I $ contains an element not in $ \ppp $ for every $ \ppp $. Then $ I = A $, because otherwise $ I $ is contained in some maximal ideal but maximal ideals are prime. Hence $ 1 \in I $, so $ x \in A $.
\end{proof}

\lecture{13}{Thursday}{31/10/19}

Lecture 13 is a problem class.

\lecture{14}{Thursday}{05/11/19}

Lecture 14 is a test.

\end{document}