\documentclass{article}

\usepackage{amssymb}
\usepackage{amsthm}
\usepackage[UKenglish]{babel}
\usepackage{enumitem}
\usepackage{fancyhdr}
\usepackage[margin=1in]{geometry}
\usepackage{graphicx}
\usepackage[utf8]{inputenc}
\usepackage{listings}
\usepackage{mathtools}
\usepackage{tikz-cd}
\usepackage{csquotes}

\newcommand{\F}{\mathbb{F}}
\newcommand{\N}{\mathbb{N}}
\newcommand{\Z}{\mathbb{Z}}
\newcommand{\Q}{\mathbb{Q}}
\newcommand{\R}{\mathbb{R}}
\newcommand{\C}{\mathbb{C}}
\newcommand{\A}{\mathbb{A}}
\renewcommand{\P}{\mathbb{P}}

\newcommand{\val}[1]{\left. #1 \right\rvert}
\newcommand{\rb}[1]{\left( #1 \right)}
\renewcommand{\sb}[1]{\left[ #1 \right]}
\newcommand{\cb}[1]{\left\{ #1 \right\}}
\newcommand{\ab}[1]{\left\langle #1 \right\rangle}
\newcommand{\abs}[1]{\left\lvert #1 \right\rvert}
\newcommand{\two}[2]{\begin{pmatrix} #1 \\ #2 \end{pmatrix}}
\newcommand{\three}[3]{\begin{pmatrix} #1 & #2 & #3 \end{pmatrix}}

\theoremstyle{definition}\newtheorem{definition}{Definition}[section]
\theoremstyle{definition}\newtheorem{remark}[definition]{Remark}
\theoremstyle{definition}\newtheorem*{example}{Example}
\theoremstyle{definition}\newtheorem*{note}{Note}
\newtheorem{proposition}[definition]{Proposition}
\newtheorem{lemma}[definition]{Lemma}
\newtheorem{theorem}[definition]{Theorem}
\newtheorem{corollary}[definition]{Corollary}

\pagestyle{fancy}
\lhead{M4P55 Commutative Algebra}
\rhead{Autumn 2018}

\title{M4P55 Commutative Algebra}
\author{Lectured by Prof Alexei Skorobogatov \\ Typeset by David Kurniadi Angdinata}
\date{Autumn 2018}

\setcounter{section}{-1}

\begin{document}

\maketitle

\vfill

\tableofcontents

\pagebreak

\marginpar{Lecture 1 \\ Friday \\ 05/10/18}

\section{Introduction}

\subsection{Outline}

Why study commutative algebra? Number theory and algebraic geometry use this language.

Structure of the course:
\begin{enumerate}
\item Rings, ideals, zero divisors, nilpotents, etc
\item Prime and maximal ideals
\item Radicals of ideals, nilradicals and the Jacobson radicals
\item Localisation
\item Modules, Nakayama's lemma
\item Noetherian and Artinian rings
\item Primary decomposition
\item Valuation rings and discrete valuation rings
\end{enumerate}

\subsection{References}

\begin{enumerate}
\item M Reid, Undergraduate commutative algebra, 1995
\item M Atiyah and I G Macdonald, Introduction to commutative algebra, 1969
\end{enumerate}

\section{Rings and ideals}

\begin{definition}
A commutative \textbf{ring} with 1 is a set $ A $ with two operations $ + $ and $ \cdot $, and two elements $ 0 $ and $ 1 $ such that the following holds.
\begin{enumerate}
\item $ \rb{A, +} $ is a group with zero $ 0 $.
\item Multiplication is
\begin{enumerate}
\item associative ($ \rb{xy}z = x\rb{yz} $ for all $ x, y, z \in A $),
\item commutative ($ xy = yx $ for all $ x, y \in A $), and
\item distributive over addition ($ x\rb{y + z} = xy + xz $ for all $ x, y, z \in A $).
\end{enumerate}
\item $ x \cdot 1 = 1 \cdot x = x $ for all $ x \in A $.
\end{enumerate}
\end{definition}

\begin{example}
$ \Z $ is a ring. The set of even integers $ 2\Z $ is not a ring because it does not contain $ 1 $.
\end{example}

\begin{remark}
Can it happen that $ 0 = 1 $? $ x \cdot 0 = x \cdot \rb{0 + 0} = x \cdot 0 + x \cdot 0 $ gives $ x \cdot 0 = 0 $. But $ x \cdot 1 = x $. Then $ x = 0 $ for all $ x \in A $, so $ A = \cb{0} $.
\end{remark}

Let $ A $ be a commutative ring with $ 1 $.

\begin{definition}
A \textbf{ring homomorphism} $ f : A \to B $ is a homomorphism of abelian groups such that $ f\rb{xy} = f\rb{x}f\rb{y} $ for any $ x, y \in A $ and $ f\rb{1} = 1 $.
\end{definition}

\begin{proposition}
A composition of homomorphisms is a homomorphism.
\end{proposition}

An \textbf{isomorphism} is a bijective homomorphism. If $ f : A \to B $ is an isomorphism, we write $ A \cong B $.

\marginpar{Lecture 2 \\ Monday \\ 08/10/18}

\begin{definition}
A subset $ I \subset A $ is called an \textbf{ideal} if $ I $ is a subgroup of $ \rb{A, +} $ and $ AI = I $. Equivalently, for any $ a \in A $ and any $ x \in I $ we have $ ax \in I $. The \textbf{quotient ring} $ A / I $ is the quotient group $ \cb{a + I \mid a \in A} $, which is actually a ring by $ \rb{a + I}\rb{b + I} = ab + I $. $ 1 + I $ is the $ 1 $ in $ A / I $. $ f : A \to A / I $ such that $ f\rb{a} = a + I $ is a surjective ring homomorphism. An ideal $ I \subset A $ is \textbf{principal} if there is $ r \in A $ such that $ I = rA $.
\end{definition}

\begin{proposition}
There is a natural bijection between the ideals of $ A $ that contain a fixed ideal $ I $ and the ideals of $ A / I $.
\end{proposition}

\begin{proof}
Suppose $ J \subset A $ is an ideal containing $ I $. Then associate to $ J $ its image $ f\rb{J} \subset A / I $. To check this, note that since $ f : A \to A / I $ is surjective, for any $ x \in A / I $ there is a $ y \in A $ such that $ f\rb{y} = x $. Hence $ xf\rb{J} = f\rb{y}f\rb{J} = f\rb{yJ} \subset f\rb{J} $. Conversely, take an ideal $ M \subset A / I $ and associate to it $ f^{-1}\rb{M} \subset A $. This is an ideal in $ A $. To check that for all $ a \in A $ we have $ af^{-1}\rb{M} \subset f^{-1}\rb{M} $, we note that this is equivalent to $ f\rb{a}M \subset M $, which is true. These maps are inverses to each other.
\end{proof}

\begin{definition}
Let $ g : A \to B $ be a homomorphism of rings. The \textbf{image} is the subset $ Im\rb{g} = \cb{x \in B \mid \exists y \in A, \ g\rb{y} = x} $. The \textbf{kernel} is the subset $ Ker\rb{g} = \cb{y \in A \mid g\rb{y} = 0} $.
\end{definition}

The image is a subring of $ \rb{B, +} $ but not necessarily an ideal, but the kernel is.

\begin{example}
Let $ g : \Z \hookrightarrow \Q $. $ 2\Z $ is an ideal in $ \Z $, but not in $ \Q $.
\end{example}

An isomorphism theorem states that $ A / Ker\rb{g} \cong Im\rb{g} = g\rb{A} $ by $ a \mapsto a + Ker\rb{g} $.

\section{Polynomial rings}

Let $ R $ be a ring. Define $ R\sb{X} $ as the ring of polynomials $ \sum_{i = 0}^n a_iX^i $ with coefficients $ a_i \in R $ and
$$ \rb{\sum_{i = 0}^k a_iX^i}\rb{\sum_{j = 0}^m b_jX^i} = \sum_{k = 0}^{n + m} \rb{\sum_{k = i + j} a_ib_j}X^k. $$
Define $ R\sb{X_1, X_2} $ to be the ring $ R\sb{X_1}\sb{X_2} $. In general, $ R\sb{X_1, \dots, X_n} = R\sb{X_1} \dots \sb{X_2} $.

\section{Zero-divisors, nilpotents, units}

\begin{definition}
A \textbf{zero-divisor} in $ A $ is an element $ x \in A $ such that there exists $ y \in A $, $ y \ne 0 $, with the property that $ xy = 0 $. A ring with no non-zero zero-divisors is called an \textbf{integral domain}. A \textbf{nilpotent} is an element $ x \in A $ such that $ x^n = 0 $ for some $ n \ge 1 $. A \textbf{unit} $ a \in A $ is an element such that there exists $ b \in A $ with the property that $ ab = 1 $. Such elements are also called \textbf{invertible}. $ b $ is denoted by $ a^{-1} $. The units form a group under multiplication, denoted by $ A^* $.
\end{definition}

\begin{example}
In $ A = \Z $, $ \Z^* = \cb{1, -1} $ and $ \Z $ is an integral domain. In $ A = \Z / 4 = \cb{4\Z, 1 + 4\Z, 2 + 4\Z, 3 + 4\Z} $, $ 2 + 4\Z $ is a zero-divisor in $ \Z / 4 $ that is also nilpotent.
\end{example}

\marginpar{Lecture 3 \\ Tuesday \\ 09/10/18}

\begin{definition}
A \textbf{field} is a ring in which $ 0 \ne 1 $ and every non-zero element is a unit. So if $ k $ is a field, then $ k \setminus \cb{0} = k^* $.
\end{definition}

\begin{proposition}
Let $ A $ be a non-zero ring. Then the following are equivalent.
\begin{enumerate}
\item $ A $ is a field.
\item The only ideals in $ A $ are $ \rb{0} = \cb{0} $ and $ \rb{1} = A $.
\item Every homomorphism $ A \to B $, where $ B \ne 0 $, is injective.
\end{enumerate}
\end{proposition}

\begin{proof}
\hfill
\begin{enumerate}[leftmargin=2cm]
\item[$ 1 \implies 2 $] Let $ I \subset A $ be a non-zero ideal. Then there exists $ x \in I $, $ x \ne 0 $. Then $ x $ is a unit, i.e. there exists $ y \in A $ such that $ xy = 1 $. For all $ a \in A $, $ a = a.1 = a.y.x \in \rb{x} $. Thus $ I = A $.
\item[$ 2 \implies 3 $] Let $ f : A \to B $. $ Ker\rb{f} $ is an ideal of $ A $. If $ Ker\rb{f} \ne \cb{0} $, then $ Ker\rb{f} = A $. But then $ 1 \in Ker\rb{f} $ and $ f\rb{1} = 0 $ but $ f\rb{1} = 1 $ so in $ B $ we have that $ 0 = 1 $. Then $ B = \cb{0} $, which is a contradiction.
\item[$ 3 \implies 1 $] Let $ x \in A $, $ x \ne 0 $. If $ 1 \in \rb{x} = xA $, then $ x $ is a unit. If $ 1 \notin \rb{x} $, then $ x $ is not a unit. If $ 1 \notin \rb{x} $, then consider the map $ A \to A / \rb{x} $ sending $ a \mapsto a + \rb{x} $. Since $ 1 \notin \rb{x} $, $ 1 + \rb{x} $ is not zero in $ A / \rb{x} $. So this is a non-injective homomorphism to a non-zero ring. This contradicts $ 3 $.
\end{enumerate}
\end{proof}

\section{Prime ideals and maximal ideals}

\begin{definition}
An ideal $ P \subset A $ is a \textbf{prime ideal} if for any $ x, y \in A $, $ xy \in P $ implies $ x \in P $ or $ y \in P $. An ideal $ M \subset A $ is called \textbf{maximal} if there does not exist an ideal $ I $ in $ A $ such that $ M \subsetneq I \subsetneq A $.
\end{definition}

\begin{lemma}
An ideal $ P \subset A $ is prime if and only if $ A / P $ is an integral domain. An ideal $ M \subset A $ is maximal if and only if $ A / M $ is a field.
\end{lemma}

\begin{proof}
Let $ x, y \in A $ such that $ xy \in P $. Then $ \rb{x + P}\rb{y + P} = xy + P = P $. If $ x \notin P $ and $ y \notin P $, then $ x + P \ne P $ and $ y + P \ne P $. These are zero divisors in $ A / P $. Conversely, if $ A / P $ is not an integral domain, then it has zero divisors. So there exists $ x, y \in A $ such that $ \rb{x + P}\rb{y + P} = P $. This implies $ xy \in P $. Since $ P $ is prime, $ x \in P $ or $ y \in P $. So one of $ x + P $ and $ y + P $ is zero in $ A / P $. Recall that there is a bijection between the ideals in $ A $ containing $ M $ with the ideals in $ A / M $. Thus $ M \subset A $ is maximal if and only if the only ideals in $ A / M $ are $ \rb{0} $ and $ \rb{1} $, if and only if $ A / M $ is a field.
\end{proof}

\begin{remark}
Every field is an integral domain, hence every maximal ideal is prime. The converse is false. Take any integral domain which is not a field, such as $ \Z $. Then $ \rb{0} \in \Z $ is a prime ideal which is not a maximal ideal.
\end{remark}

\begin{proposition}
If $ f : A \to B $ is a homomorphism of rings, and $ P \subset B $ is a prime ideal, then $ f^{-1}\rb{P} $ is a prime ideal in $ A $.
\end{proposition}

\begin{proof}
Assume that for some $ x, y \in A $ we have $ xy \in f^{-1}\rb{P} $. Then $ f\rb{xy} = f\rb{x}f\rb{y} \in P $. Then $ f\rb{x} \in P $ or $ f\rb{y} \in P $. Then $ x \in f^{-1}\rb{P} $ or $ y \in f^{-1}\rb{P} $.
\end{proof}

\begin{remark}
This does not hold for maximal ideals. Let $ f : \Z \hookrightarrow \Q $. $ f^{-1}\rb{\rb{0}} = \rb{0} $, but $ \rb{0} $ is maximal in $ \Q $ and not maximal in $ \Z $. But if $ f : A \to B $ is a surjective homomorphism of rings, then $ f^{-1} $ sends maximal ideals of $ B $ to maximal ideals of $ A $. (Exercise)
\end{remark}

\begin{theorem}
\label{thm:4.6}
Every non-zero ring contains at least one maximal ideal.
\end{theorem}

We need Zorn's lemma, which belongs to set theory. A \textbf{partially ordered set} or \textbf{poset} is a set $ S $ equipped with a \textbf{partial order}. By definition it is a reflexive, transitive, antisymmetric binary relation $ \le $,
$$ x \le x, \qquad x \le y, y \le z \implies x \le z, \qquad x \le y, y \le x \implies x = y. $$
We don't require that for arbitrary $ x $ and $ y $ in $ S $, we have either $ x \le y $ or $ y \le x $. A subset $ T \subset S $ is called a \textbf{chain} if for any $ x \in T $, $ y \in T $ we have $ x \le y $ or $ y \le x $. An \textbf{upper bound} for a subset $ T \subset S $ is an element $ x \in S $ such that for any $ t \in T $ we have $ t \le s $. A \textbf{maximal element} in $ S $ is an element $ x \in S $ such that if $ y \in S $ and $ y \ge x $, then $ y = x $.

\marginpar{Lecture 4 \\ Friday \\ 12/10/18}

\begin{theorem}[Zorn's lemma]
If $ S $ is a non-empty partially ordered set such that every chain in $ S $ has an upper bound in $ S $, then $ S $ contains a maximal element.
\end{theorem}

\begin{proof}[Proof of Theorem \ref{thm:4.6}]
Let $ A $ be a non-zero ring. To apply Zorn's lemma it is enough to show that every growing chain of ideals $ I_1 \subset I_2 \subset \dots $, such that $ 1 \in I_i $ for all $ i $, has an upper bound which is an ideal not equal to $ A $, so not containing $ 1 $. Then Zorn's lemma applied to the set of ideals of $ A $ not containing $ 1 $ and ordered by inclusion, implies the existence of a maximal ideal. So we have a chain $ I_j $, where $ j $ is an element of a set $ J $. Consider $ I = \cup_{j \in J} I_j $. Claim that $ I $ is an ideal in $ A $ and $ 1 \notin I $.
\begin{enumerate}
\item $ 1 \notin I $ is clear. Because otherwise $ 1 \in I $ gives $ 1 \in I_j $ for $ j \in J $, but it is a contradiction.
\item For any $ a \in A $ we have $ aI \subset I $, so for all $ x \in I $, $ ax \in I $. But then $ x \in I_j $ for some $ j $. Then $ ax \in I_j \subset I $.
\item Suppose $ x, y \in I $. Must show $ x + y \in I $. There exists $ j_1 \in J $ such that $ x \in I_{j_1} $. Similarly, there exists $ j_2 \in J $ such that $ y \in I_{j_2} $. Recall that $ I_j $ for $ j \in J $ is a chain. Hence either $ j_1 \le j_2 $ or $ j_2 \le j_1 $. This means that either $ I_{j_1} \subset I_{j_2} $ or $ I_{j_2} \subset I_{j_1} $. Without loss of generality assume that $ I_{j_1} \subset I_{j_2} $. Then $ x, y \in I_{j_2} $. Hence $ x + y \in I_{j_2} $, hence $ x + y \in I $. This proves that $ I $ is an ideal not containing $ 1 $.
\end{enumerate}
\end{proof}

\begin{definition}
A ring with a unique maximal ideal is called a \textbf{local ring}.
\end{definition}

\begin{corollary}
\label{cor:4.9}
Let $ I $ be an ideal of $ A $ and $ I \ne A $. Then $ I $ is contained in a maximal ideal of $ A $.
\end{corollary}

\begin{proof}
There is a bijection between the ideals of $ A $ containing $ I $ and the ideals in $ A / I $. If $ I \subset J \subset A $, then $ J \mapsto J / I $. $ J / I $ is an ideal in $ A / I $. By Theorem \ref{thm:4.6}, $ A / I $ contains a maximal ideal, say $ M \subset A / I $. Let $ f : A \to A / I $ be the map sending $ x \mapsto x + I $. Consider $ f^{-1}\rb{M} \subset A $. This is an ideal in $ A $. In general, if $ I \subset J \subset A $ are ideals, then $ f $ induces an isomorphism of rings $ A / J \to \rb{A / I}\rb{J / I} $. For additive groups, this is one of the standard isomorphisms theorems, but this respects multiplication, so is an isomorphism of rings. Now, we know that $ M $ maximal in $ A / I $ implies that $ \rb{A / I} $ is a field. This ring is isomorphic to $ A / f^{-1}\rb{M} $. Hence $ A / f^{-1}\rb{M} $ is also a field. Therefore, $ f^{-1}\rb{M} $ is maximal in $ A $.
\end{proof}

\begin{corollary}
\label{cor:4.10}
Every non-unit is contained in a maximal ideal.
\end{corollary}

\begin{proof}
If $ x \in A $ is a non-unit, consider $ \rb{x} $. $ 1 \notin \rb{x} $, otherwise $ x $ is a unit. By Corollary \ref{cor:4.9} $ \rb{x} $ is contained in a maximal ideal of $ A $.
\end{proof}

\begin{example}
\hfill
\begin{enumerate}
\item Every field is a local ring. In this case $ \rb{0} $ is a maximal ideal.
\item Let $ k $ be a field. Consider the ring of formal power series $ k\sb{\sb{t}} = \cb{a_0 + a_1t + \dots \mid a_i \in k} $, such that
$$ \rb{\sum_{i = 0}^\infty a_it^i}\rb{\sum_{j = 0}^\infty b_jt^j} = a_0b_0 + \rb{a_0b_1 + a_1b_0}t + \dots. $$
Then the principal ideal $ \rb{t} $ is a maximal ideal. Indeed, $ k\sb{\sb{t}} / \rb{t} \cong k $ is a field. (TODO Exercise: $ k\sb{\sb{t}} \setminus \rb{t} = k\sb{\sb{t}}^* $)
\item $ \Z_{\rb{p}} = \cb{a / b \mid a, b \in \Z, \ b \ne 0, \ p \nmid b} $. (TODO Exercise: $ \rb{p} $ is a maximal ideal. There are no other maximal ideals)
\end{enumerate}
\end{example}

If $ A $ is a local ring with maximal ideal $ M $, then $ A / M $ is called the \textbf{residue field} of $ A $.

\begin{lemma}[Prime avoidance]
Let $ A $ be a ring and $ P \subset A $ be a prime ideal. Suppose that $ I_1, \dots, I_n $ are ideals in $ A $ such that $ \cap_{i = 1}^n I_i \subset P $. Then there exists $ j $, $ 1 \le j \le n $, such that $ I_j \subset P $. If $ \cap_{i = 1}^n I_i = P $, then there exists $ j $, $ 1 \le j \le n $, such that $ I_j = P $.
\end{lemma}

\begin{proof}
Suppose our claim is false. Then there exists $ a_j \in I_j $ such that $ a_j \notin P $ for $ j = 1, \dots, n $. Then $ a_1 \dots a_n \in \cap_{j = 1}^n I_i \subset P $. $ \rb{a_1 \dots a_{n - 1}}a_n \in P $ gives $ a_1 \dots a_{n - 1} \in P $ or $ a_n \in P $. But $ a_n \notin P $, so $ a_1 \dots a_{n - 2} \in P $, a contradiction. The second statement follows. We know that $ I_k \subset P $ for some $ k $, $ 1 \le k \le n $, but $ P = \cap_{j = 1}^n I_j \subset I_k $. Hence $ P = I_k $.
\end{proof}

\marginpar{Lecture 5 \\ Monday \\ 15/10/18}

\section{Nilradical and the Jacobson radical}

\begin{proposition}
Let $ A $ be a ring. The set $ N\rb{A} $ of all nilpotent elements of $ A $ is an ideal in $ A $. It is called the \textbf{nilradical} of $ A $. The quotient ring $ A / N\rb{A} $ has no non-zero nilpotents.
\end{proposition}

\begin{proof}
Clearly, if $ x^n = 0 $ and $ y^n = 0 $, then $ \rb{xy}^n = 0 $, if $ n \ge m $. $ \rb{x + y}^{n + m} $ is the sum with coefficients of of monomials in which either the power of $ x $ is $ \ge n $ or the power of $ y $ is $ \ge m $. So this is zero. Let $ a \in A $. Then $ \rb{ax}^n = 0 $. Therefore, $ N\rb{A} $ is an ideal. Now let $ t + N\rb{A} $ for $ t \in A $ be a nilpotent element in $ A / N\rb{A} $. For some $ k $ we have $ t^k + N\rb{A} $ is the trivial coset. That is, $ t^k \in N\rb{A} $. Thus $ \rb{t^k}^l = 0 $ for some $ l > 0 $. Hence $ t \in N\rb{A} $ , so $ t + N\rb{A} $ is the zero element of $ A / N\rb{A} $.
\end{proof}

\begin{proposition}
The nilradical $ N\rb{A} $ is the intersection of all prime ideals of $ A $.
\end{proposition}

\begin{proof}
\hfill
\begin{enumerate}
\item $ N\rb{A} \subset \cap_{P \subset A} P $, where $ P $ is a prime ideal of $ A $. Take $ x \in A $, $ x^n = 0 $. Take a prime ideal $ P \subset A $. We have that $ P \ni x^n = x \dots x $ gives $ x \in P $.
\item Now let $ f \in A $ be a non-nilpotent element, that is $ 0 \notin \cb{f^i \mid i \ge 1} $. Let $ \Sigma $ be the set of ideals of $ A $ that do not intersect $ \cb{f^i \mid i \ge 1} $. $ \Sigma $ contains the zero ideal $ \rb{0} $, so $ \Sigma \ne \emptyset $. Order the elements of $ \Sigma $ by inclusion. Every chain in $ \Sigma $ has an upper bound. If $ I_j $ for $ j \in J $ is a chain, then $ \cup_{j \in J} I_j $ is an ideal of $ A $. Moreover, if $ f^k \in \cup_{j \in J} I_j $, then $ f^k \in I_{j_0} $ for some $ j_0 \in J $, but this is impossible. By Zorn's Lemma, we know that $ \Sigma $ has a maximal element. Call it $ P $. Claim that $ P $ is a prime ideal. To prove this, assume that $ x, y \in A $ such that $ x, y \notin P $. We must show that $ xy \notin P $. Consider $ P + \rb{x} $, all elements of the form $ \alpha + rx $, where $ \alpha \in P $ and $ r \in A $. $ x \notin P $ gives $ P \ne P + \rb{x} $. By construction, $ P $ is maximal in $ \Sigma $, hence $ P + \sigma $ is not in $ \Sigma $, that is, there exists $ n \ge 1 $ such that $ f^n \in P + \rb{x} $. Similarly, there exists $ m $ such that $ f^m \in P + \rb{y} $. Therefore, $ f^{n + m} $ belongs to $ P + \rb{xy} $. If $ xy \in P $, then $ P + \rb{xy} = P $ but then $ f^{n + m} \in P $, which is absurd because $ P \in \Sigma $. Thus $ xy \notin P $. This shows that $ P $ is a prime ideal and $ f \notin P $.
\end{enumerate}
\end{proof}

What happens if we consider the intersection of all maximal ideals of $ A $. This intersection is called the \textbf{Jacobson radical} of $ A $. It is denoted by $ J\rb{A} $.

\begin{proposition}
\label{prop:5.3}
$ x \in J\rb{A} $ if and only if $ 1 - xy $ is a unit in $ A $ for all $ y \in A $.
\end{proposition}

\begin{proof}
Suppose that $ x \in J\rb{A} $, that is $ x $ is contained in every maximal ideal of $ A $, but $ 1 - xy $ is not a unit for some $ y \in A $. By Corollary \ref{cor:4.10} every non-unit is contained in some maximal ideal, so there exists a maximal ideal $ M \subset A $ such that $ 1 - xy \in M $. Since $ x \in M $ we conclude that $ 1 \in M $, which is impossible. Conversely, suppose $ x \notin J\rb{A} $, that is, $ x \notin M $ for some maximal ideal $ M \subset A $. Consider the sum of two ideals $ M + \rb{x} $. This is an ideal in $ A $, such that $ M \subsetneq M + \rb{x} $. Since $ M $ is maximal, we have $ M + \rb{x} = A $. Therefore $ 1 = m + xy $, where $ m \in A $ and $ y \in A $. Now $ 1 - xy = m \in M $ cannot be a unit.
\end{proof}

Let $ I \subset A $ be an ideal. The \textbf{radical} $ rad\rb{I} $ or $ r\rb{I} $ or $ \sqrt{I} $ is defined as $ \cb{x \in A \mid \exists n \ge 1, \ x^n \in I} $.

\begin{proposition}
$ r\rb{I} $ is the intersection of all prime ideals of $ A $ that contain $ I $.
\end{proposition}

\begin{proof}
Use the bijection between ideals containing $ I $ and the ideals in $ A / I $.
\end{proof}

\marginpar{Lecture 6 \\ Tuesday \\ 16/10/18}

\begin{definition}
Let $ J $ be an index set. Suppose we have a ring $ R_j $ for $ j \in J $. $ \prod_{j \in J} R_j $ has a natural structure of a ring. $ 0 $ in $ \prod_{j \in J} R_j $ is $ \rb{0, \dots, 0} $ and $ 1 $ in $ \prod_{j \in J} $ is defined as $ \rb{1, \dots, 1} $, and
$$ \rb{r_j}_{j \in J} + \rb{r'_j}_{j \in J} = \rb{r_j + r'_j}_{j \in J}, \qquad \rb{r_j}_{j \in J} \cdot \rb{r'_j}_{j \in J} = \rb{r_j \cdot r'_j}_{j \in J}. $$
$ \prod_{j \in J} R_j $ is called the \textbf{product of rings} $ R_j $ for $ j \in J $. If $ R $ is a ring equipped with homomorphisms $ f_j : R \to R_j $ for each $ j \in J $, then $ \rb{f_j} : R \to \prod_{j \in J} R_j $ is a homomorphism of rings.
\end{definition}

Recall that $ N\rb{R} = \cap_{P \subset R} P $, where $ P $ are prime ideals of $ R $. Consider the product ring $ \prod_{P \subset R} R / P $. Putting together the canonical surjective maps $ R \to R / P $ by $ x \mapsto x + P $ for all $ P \subset R $ we obtain a homomorphism $ f : R \to \prod_{P \subset R} R / P $. $ Ker\rb{f} = \cup_{P \subset R} Ker\sb{R \to R / P} = \cap_{P \subset R} = N\rb{R} $. Hence we get an injective homomorphism $ R / N\rb{R} \to \prod_{P \subset R} R / P $. Similarly, we get an injective homomorphism $ R / J\rb{R} \to \prod_{M \subset R} R / M $, where $ M $ are maximal ideals of $ R $ and $ J\rb{R} $ is the Jacobson radical of $ R $.

\section{Localisation of rings}

Localisation refers to introducing denominators.

\begin{example}
From $ R = \Z $ to $ \Q = \cb{a / b \mid a, b \in \Z, \ b \ne 0} $.
\end{example}

\begin{definition}
A subset $ S \subset A $ is called a \textbf{multiplicative set} if $ 1 \in S $, $ 0 \not\in S $, and if $ a, b \in S $, then $ ab \in S $, that is $ S $ is closed under multiplication.
\end{definition}

\begin{example}
\hfill
\begin{enumerate}
\item Take any $ a \in A $ which is not nilpotent, that is $ a^n = 0 $ for $ n \ge 1 $. Then $ \cb{1, a, a^2, \dots} $ is a multiplicative set.
\item Let $ P \subset A $ be a prime ideal. Then $ A \setminus P $ is a multiplicative set. Indeed, $ x, y \notin P $ gives $ xy \notin P $.
\item Let $ P_j \subset A $, for $ j \in J $, be a family of prime ideals of $ A $. Then $ A \setminus \cup_{j \in J} P_j = \cap_{j \in J} \rb{A \setminus P_j} $ is a multiplicative set.
\item $ A^* $ is a multiplicative set in $ A $.
\item The set of all non-zero-divisors of $ A $ is a multiplicative set.
\item Let $ I \subset A $ be an ideal. Then $ 1 + I = \cb{1 + x \mid x \in I} $ is a multiplicative set.
\end{enumerate}
\end{example}

\begin{definition}
Let $ A $ be a ring with a multiplicative set $ S $. Consider $ A \times S $, that is the set of pairs of elements $ \rb{a, s} $, where $ a \in A $ and $ s \in S $. Define an equivalence relation $ \sim $ as follows. $ \rb{a, s} \sim \rb{b, t} $ if and only if there exists $ u \in S $ such that $ u\rb{at - bs} = 0 $. Define $ S^{-1}A $ to be the set of equivalence classes of $ \sim $. Write the equivalence class of $ \rb{a, s} $ as $ a / s $. Define multiplication on $ S^{-1}A $ as
$$ \dfrac{a}{s} \cdot \dfrac{b}{t} = \dfrac{ab}{st}. $$
Define addition on $ S^{-1}A $ as
$$ \dfrac{a}{s} + \dfrac{b}{t} = \dfrac{at + bs}{st}. $$
Define $ 0 $ in $ S^{-1}A $ as $ 0 / 1 $ and we define $ 1 $ in $ S^{-1}A $ as $ 1 / 1 $.
\end{definition}

(TODO Exercise: check that if $ \rb{a, s} \sim \rb{a', s'} $ and $ \rb{b, t} \sim \rb{b', t'} $, then $ \rb{ab, st} \sim \rb{a'b', s't'} $)
(TODO Exercise: check that if $ \rb{a, s} \sim \rb{a', s'} $ and $ \rb{b, t} \sim \rb{b', t'} $, then $ \rb{at + bs, st} \sim \rb{a't' + b's', s't'} $)
(TODO Exercise: with this definition $ S^{-1}A $ is a ring)

\begin{remark}
$ \sim $ is indeed an equivalence relation. $ \rb{a, s} \sim \rb{a, s} $, $ \rb{a, s} \sim \rb{b, t} $ gives $ \rb{b, t} \sim \rb{a, s} $. Let us check that if $ \rb{a, s} \sim \rb{b, t} $ and $ \rb{b, t} \sim \rb{c, r} $, then $ \rb{a, s} \sim \rb{c, r} $. There exists $ u, v \in S $ such that $ u\rb{at - bs} = 0 $ and $ v\rb{br - ct} = 0 $. Then $ uv\rb{atr - bsr} = 0 $ and $ uv\rb{brs - cts} = 0 $, so $ uvt\rb{ar - bs} = 0 $.
\end{remark}

\begin{lemma}
Let $ A $ be a ring with a multiplicative set $ S $. Then $ f : A \to S^{-1}A $ defined by $ f\rb{x} = x / 1 $ is a homomorphism of rings. $ Ker\rb{f} = 0 $ if and only if $ S $ contains no zero-divisors.
\end{lemma}

\begin{proof}
$$ f\rb{x + y} = \dfrac{x + y}{1} = \dfrac{x}{1} + \dfrac{y}{1}, \qquad f\rb{xy} = \dfrac{xy}{1} = \dfrac{x}{1} \cdot \dfrac{y}{1}. $$
$ Ker\rb{f} = \cb{x \mid \exists u \in S, \ ux = 0} $ since $ x / 1 = 0 / 1 $ if and only if there exists $ u \in S $ such that $ u\rb{x \cdot 1 - 0 \cdot 1} = 0 $.
\end{proof}

\begin{example}
Let $ k $ be a field. Explore what happens when $ A = k\sb{x, y} / \rb{xy} $ and $ S = \cb{1, x, \dots} $. Determine $ S^{-1}A $ and $ Ker\rb{f} $.
\end{example}

\marginpar{Lecture 7 \\ Friday \\ 19/10/18}

Lecture 7 is a problem class.

\marginpar{Lecture 8 \\ Monday \\ 22/10/18}

\begin{lemma}[Universal property of localisation]
Let $ A $ be a ring with a multiplicative set $ S \subset A $. Suppose $ g : A \to B $ is a homomorphism such that $ g\rb{S} \subset B^* $, that is for all $ s \in S $, $ g\rb{s} $ is a unit in $ B $. Then there exists a unique homomorphism $ h : S^{-1}A \to B $ such that $ g = h \circ f $, where $ f : A \to S^{-1}A $ is the canonical map.
\end{lemma}

\begin{proof}
Define $ h\rb{a / s} = g\rb{a}g\rb{s}^{-1} $ since $ g $ invertible. Check that $ h $ is well-defined, that is if $ a / s = a' / s' $, then $ u\rb{as' - a's} = 0 $ for $ u \in S $. Apply $ g $ and get $ g\rb{u}\rb{g\rb{a}g\rb{s'} - g\rb{a'}g\rb{s}} = 0 $. $ g\rb{u} \in B^* $ and $ g\rb{a}g\rb{s'} = g\rb{a'}g\rb{s} $. Hence $ g\rb{a}g\rb{s}^{-1} = g\rb{a'}g\rb{s'}^{-1} $. Take any $ a \in A $. Then $ f\rb{a} = a / 1 $, hence $ \rb{h \circ f}\rb{a} = g\rb{a} $. Finally, let us show there is only one homomorphism $ h : S^{-1}A \to B $ such that $ g = h \circ f $. Suppose $ h' : S^{-1}A \to B $ is such that $ g = h' \circ f $, so that for any $ a \in A $ we have $ g\rb{a} = h'\rb{a} $. For any $ s \in S $, $ s^{-1} $ is an element of $ S^{-1}A $, and so is $ s $. $ 1 = s^{-1}s $ gives $ 1 = h'\rb{1} = h'\rb{s^{-1}}h'\rb{s} $. Thus $ h'\rb{s^{-1}} = h'\rb{s}^{-1} = g\rb{s}^{-1} $ because $ h' $ on the image of $ A $ in $ S^{-1}A $ is the same as $ g $. Comparing this with the definition of $ h $ we see that $ h' = h $.
\end{proof}

Let $ I \subset A $ be an ideal. Define $ S^{-1}I = \cb{x / s \mid x \in I, \ s \in S} $. This is an ideal in $ S^{-1}I $. It is the ideal generated by $ f\rb{I} \subset S^{-1}A $.

\begin{proposition}
Let $ A $ be a ring with a multiplicative set $ S $. Let $ I_1, \dots, I_n $ be ideals in $ A $. Then
\begin{enumerate}
\item $ S^{-1}\rb{I_1 + \dots + I_n} = S^{-1}I_1 + \dots + S^{-1}I_n $,
\item $ S^{-1}\rb{I_1 \dots I_n} = S^{-1}I_1 \dots S^{-1}I_n $,
\item $ S^{-1}\rb{\cap_{j = 1}^n I_j} = \cap_{j = 1}^n S^{-1}I_j $, and
\item $ r\rb{S^{-1}I} = S^{-1}r\rb{I} $, where $ r\rb{I} $ is the radical of $ I $.
\end{enumerate}
\end{proposition}

\begin{proposition}
\label{prop:6.8}
Every ideal of $ S^{-1}A $ is of the form $ S^{-1}I $ for some ideal $ I \subset A $.
\end{proposition}

\begin{proof}
Start with an ideal $ J \subset S^{-1}A $. Consider $ f^{-1}\rb{J} \subset A $. This is an ideal. Call it $ I $. Claim that $ J = S^{-1}I $. Pick any element $ a / s \in J $. Then $ a \in J $. Since $ f\rb{a} = a / 1 \in J $ we have that $ a \in I $. Therefore, $ a / s \in S^{-1}I $. This proves $ J \subset S^{-1}I $. But it is clear that $ S^{-1}I \subset J $. Indeed, $ x \in I $ then $ x / 1 \in J $. But $ J $ is an ideal, hence $ x / s \in J $.
\end{proof}

\begin{theorem}
\label{thm:6.9}
The prime ideals in $ S^{-1}A $ are the ideals $ S^{-1}P $, where $ P $ is a prime ideal of $ A $ such that $ P \cap S \ne \emptyset $. Thus we have a bijection between the set of prime ideals in $ S^{-1}A $ and the set of prime ideals in $ A $ that do not intersect $ S $.
\end{theorem}

\begin{proof}
Suppose that $ P $ is a prime ideal in $ A $, $ P \cap S \ne \emptyset $. Claim that $ S^{-1}P $ is a prime ideal in $ S^{-1}A $. If $ \rb{a / s}\rb{b / t} \in S^{-1}P $, then $ \rb{a / s}\rb{b / t} = c / u $, where $ c \in P $, $ u \in S $. This is equivalent to $ v\rb{abu - cst} = 0 $ for some $ v \in S $. $ \rb{ab}\rb{vu} = c \in P $ such that $ v \in P $. $ vu \in S $ and $ S \cap P = \emptyset $, so $ vu \notin P $. But $ P \subset A $ is a prime ideal, hence $ ab \in P $. Thus $ a \in P $ gives $ a / s \in S^{-1}P $ or $ b \in P $ gives $ b / t \in S^{-1}P $. This proves $ S^{-1}P \subset S^{-1}A $ is prime. For any ideal $ J \subset S^{-1}A $, we know that $ f^{-1}J $ is an ideal in $ S $. Moreover, if $ J $ is prime, then $ f^{-1}J \subset A $ is prime. Let us show that $ f^{-1}J \cap S = \emptyset $. Otherwise, take $ s \in S \cap f^{-1}J $, so $ s / 1 \in J $. But $ 1 / s \in J^{-1}A $, hence $ 1 = \rb{1 / s}s \in J $, so $ J = S^{-1}A $. But $ J $ is a prime ideal, so $ J \ne S^{-1}A $. To show that $ P \mapsto S^{-1}P $ and $ J \mapsto f^{-1}J $ are the identity maps, we need to check that $ P = f^{-1}\rb{S^{-1}P} $ and $ J = S^{-1}f^{-1}\rb{J} $. $ S^{-1}P = \cb{x / s \mid x \in P, \ s \in S} $. If $ y \in f^{-1}\rb{S^{-1}P} \subset A $  is such that $ f\rb{y} = x / s $, then $ y / 1 = x / s $. Hence $ ys = x \in P $. Since $ P \cap S = \emptyset $, $ s \notin P $. Therefore, $ y \in P $. Hence $ P = f^{-1}\rb{S^{-1}A} $. Now let us prove that $ J = S^{-1}f^{-1}\rb{J} $. But in Proposition \ref{prop:6.8} we showed that there is an ideal $ I \subset A $ such that $ J = S^{-1}I $. In the proof of Proposition \ref{prop:6.8} we have taken $ I = f^{-1}\rb{J} $. So we are done.
\end{proof}

\marginpar{Lecture 9 \\ Tuesday \\ 23/10/18}

\section{Determinants}

\begin{lemma}
Let $ f\rb{x_1, \dots, x_n} \in \Z\sb{x_1, \dots, x_n} $. If $ f $ as a function $ \Z^n \to \Z $ is zero, that is $ f $ only takes zero values on arbitrary elements of $ \Z^n $, then $ f $ is the zero polynomial.
\end{lemma}

\begin{proof}
Induction in $ n $. If $ n = 1 $, then $ f\rb{x} $ is a polynomial with infinitely many roots. So $ f\rb{x} $ is the zero polynomial, so cannot have more than $ \deg\rb{f} $ roots. Assume we know the lemma for $ n - 1 $ variables. Write $ f\rb{x_1, \dots, x_n} = \sum_{i = 0}^N f_i\rb{x_1, \dots, x_{n - 1}}x_n^i $ for $ f_j\rb{x_1, \dots, x_{n - 1}} \in \Z\sb{x_1, \dots, x_{n - 1}} $. Fix $ x_1, \dots, x_{n - 1} $. We get a polynomial in one variable $ x_n $, so this polynomial has zero coefficients. This implies that each $ f_i\rb{x_1, \dots, x_n} $ takes only zero values. By the induction assumption, each $ f_i $ is the zero polynomial.
\end{proof}

\begin{remark}
\label{rem:7.2}
This means that if a polynomial formula with coefficients in $ \Z $ is true in $ \Z $, this is true in an arbitrary commutative ring.
\end{remark}

\begin{example}
$ \rb{x + y}^n = \sum_{i = 0}^n \two{n}{i}x^iy^{n - i} $ is true in any ring.
\end{example}

The underlying fact is the existence of a canonical map $ \Z \to R $ by $ 1 \mapsto 1 $.

\begin{definition}
Let $ R $ be a commutative ring. Let $ A = \rb{a_{ij}} $ be a square matrix for $ 1 \le i \le n $ and $ 1 \le j \le n $, with entries in $ R $. Then $ \det\rb{A} $ is defined as $ \rb{-1}^{i + 1}a_{i1}M_{i1} + \dots \rb{-1}^{i + n}a_{in}M_{in} $ for $ i $ fixed. Here $ M_{ij} $ is the determinant of the $ \rb{n - 1} \times \rb{n - 1} $ submatrix of $ A $ obtained by removing the $ i $-th row and the $ j $-th column.
\end{definition}

\begin{proposition}
$ \det\rb{A} = \rb{-1}^{i + 1}a_{i1}M_{i1} + \dots \rb{-1}^{i + n}a_{in}M_{in} $.
\end{proposition}

\begin{proof}
This is known for matrices with entries in $ \C $, so by Remark \ref{rem:7.2} this holds in any commutative ring.
\end{proof}

\begin{remark}
The official definition is
$$ \det\rb{A} = \sum_{\pi \in S_n} sgn\rb{\pi}a_{1\pi\rb{1}} \dots a_{n\pi\rb{n}}, $$
where $ sgn : S_n \to \cb{\pm 1} $.
\end{remark}

\begin{proposition}
For $ i \ne j $,
$$ \rb{-1}^{j + 1}a_{i1}M_{j1} + \dots + \rb{-1}^{j + n}a_{in}M_{jn} = 0, $$
$$ \rb{-1}^{j + 1}a_{1i}M_{1j} + \dots + \rb{-1}^{j + n}a_{ni}M_{nj} = 0. $$
\end{proposition}

Define the \textbf{adjacent} matrix as an $ n \times n $ matrix $ A^v_{ij} = \rb{-1}^{i + j}M_{ji} $. Putting together all the previous identities we get the following.

\begin{theorem}
$ A \cdot A^v = A^v \cdot A = \det\rb{A} I_n $.
\end{theorem}

\marginpar{Lecture 10 \\ Friday \\ 26/10/18}

\section{Modules}

\begin{definition}
Let $ A $ be a ring. A \textbf{module} $ M $ over $ A $ is an abelian group $ \rb{M, 0, +} $ with an action $ \cdot $ of $ A $ on $ M $, that is $ A \times M \to M $ by $ a \cdot m = am $, such that the following axioms hold.
\begin{enumerate}
\item $ 1 \cdot m = m $ for all $ m \in M $ and $ a \in A $.
\item $ \mu \cdot \rb{\lambda \cdot m} = \rb{\mu\lambda} \cdot m $ $ \lambda, \mu \in A $.
\item $ \lambda\rb{x + y} = \lambda x + \lambda y $ for all $ \lambda \in A $ and $ x, y \in M $.
\item $ \rb{\mu + \lambda}x = \mu x + \lambda x $ for all $ \mu, \lambda \in A $ and $ x \in M $.
\end{enumerate}
\end{definition}

\begin{example}
\hfill
\begin{enumerate}
\item $ M = A $. More generally, consider an ideal $ I \subset A $. $ A $ acts on $ I $ by $ A \times I \to I $ by $ a \cdot x = ax $.
\item If $ A $ is a field, then an $ A $-module is the same as a vector space over this field.
\item Take $ M $ to be any abelian group. Take $ A = \Z $. Define an action of $ \Z $ as follows. $ 1 \cdot m = m $ and $ n \cdot m = \rb{1 + \dots + 1} \cdot m = m + \dots + m = nm $. $ 0 = n + \rb{- n} \in \Z $, then $ 0 = \rb{n + \rb{-n}} \cdot m = nm + \rb{-n}m $. Hence $ \rb{-n} \cdot m = -\rb{n \cdot m} = -\rb{m + \dots + m} $. So, there is exactly one way to equip any abelian group with the structure of a $ \Z $-module.
\item Let $ k $ be a field and let $ A = k\sb{x} $. A $ k\sb{x} $-module is a vector space over $ k $ with extra structure $ x \times M \to M $. This is a linear transformation of $ M $. It can be arbitrary. Thus a $ k\sb{x} $-module is a pair $ \rb{M, f} $, where $ M $ is a $ k $-vector space and $ f : M \to M $ is linear transformation of $ M $.
\end{enumerate}
\end{example}

\begin{definition}
Let $ M $ and $ N $ be $ A $-modules. A map $ f : M \to N $ is called a \textbf{homomorphism of $ A $-modules} if $ f $ is a homomorphism of abelian groups and $ f\rb{a, m} = af\rb{m} $ for any $ a \in A $ and $ m \in M $. If $ f : M \to N $ and $ g : M \to N $ are homomorphisms of $ A $-modules, then so is $ f + g $, so we get $ Hom_A\rb{M, N} $, a group of such homomorphisms. This is also an $ A $-module via the action $ \rb{a, f\rb{a}} \mapsto a \cdot f\rb{a} $.
\end{definition}

\begin{definition}
A \textbf{submodule} $ N \subset M $ is a subgroup, stable under the action of $ A $. Then $ M / N $ is naturally an $ A $-module with $ A $-action inherited from $ M $. Define $ \rb{N : M} = \cb{a \in A \mid raM \subset rN \subset N} $. This is an ideal in $ A $. In particular, can do this when $ N = 0 $. Note $ Ann\rb{M} = \rb{0 : M} = \cb{a \in A \mid aM = 0} $. This is called the \textbf{annihilator} of $ M $.
\end{definition}

\begin{definition}
If $ f : M \to N $ is a homomorphism of $ A $-modules, then $ Ker\rb{f} $ is an $ A $-module and $ Im\rb{f} \cong M / Ker\rb{f} $ is as isomorphism of $ A $-modules.
\end{definition}

\begin{definition}
An $ A $-module $ M $ is \textbf{finitely generated} if there exist $ m_1, \dots, m_n $ in $ M $ such that $ M = \cb{a_1m_1 + \dots + a_nm_n \mid a_i \in A} $.
\end{definition}

\begin{example}
A \textbf{free} $ A $-module of rank $ n $ is the set $ A^n = \cb{\rb{a_1, \dots, a_n} \mid a_i \in A} $ with coordinate-wise addition. $ a \in A $ acts on $ \rb{a_1, \dots, a_n} $ by sending it to $ \rb{aa_1, \dots, aa_n} $. If $ f\rb{1, 0, \dots, 0} = m_1 $, $ f : A^m \to M $ is an example of an $ A $-module homomorphism.
\end{example}

\begin{lemma}
\label{lem:8.7}
Let $ A $ be a ring. Let $ M $ be a finitely generated $ A $-module and let $ A \subset A $ be an ideal such that $ JM = M $, that is sums of $ xm $, where $ x \in J $ and $ m \in M $, give all of $ M $. Then there exists $ a \in J $ such that $ \rb{1 - a}M = 0 $.
\end{lemma}

\begin{proof}
Let $ m_1, \dots, m_n $ be a set of generators of $ M $. $ m_i \in M = JM $, so $ m_i = x_{i1}m_1 + \dots + x_{in}m_n $, where $ x_{ij} \in J $. Let $ X = \rb{x_{ij}}_{1 \le i, j \le n} $, so
$$ \rb{I_n - X}\begin{pmatrix} m_1 \\ \vdots \\ m_n \end{pmatrix} = 0. $$
Let $ \rb{I_n - X}^v $ be the adjunct matrix of $ I_n - X $. Then $ \rb{I_n - X}^v\rb{I_n - X} = \det\rb{I_n - X}I_n $. Hence
$$ \det\rb{I_n - X}\begin{pmatrix} m_1 \\ \vdots \\ m_n \end{pmatrix} = \begin{pmatrix} 0 \\ \vdots \\ 0 \end{pmatrix}. $$
$ \det\rb{I_n - X} = \prod_{i = 1}^n \rb{1 - x_{ii}} + J \equiv 1 \mod J $. So $ \det\rb{I_n - X} = 1 - a $, where $ a \in J $. $ \rb{1 - a}m_i = 0 $ for all $ i $ gives $ \rb{1 - a}M = 0 $.
\end{proof}

\marginpar{Lecture 11 \\ Monday \\ 29/10/18}

\begin{corollary}[Nakayama's lemma]
Let $ A $ be a ring and let $ M $ be an $ A $-module, which is finitely generated. Let $ I \subset A $ be an ideal contained in the Jacobson radical $ J\rb{A} $. Then $ IM = M $ implies $ M = 0 $.
\end{corollary}

\begin{proof}
Lemma \ref{lem:8.7} gives an $ a \in I $ such that $ \rb{1 - a}M $. But $ a \in J\rb{A} $. By Proposition \ref{prop:5.3} $ 1 - a \in A^* $ so that there exists $ u \in A^* $ such that $ u\rb{1 - a} = 1 $, so $ M = 1 \cdot M  = u\rb{1 - a} \cdot M = 0 $.
\end{proof}

Another proof considers $ M = \rb{m_1, \dots, m_n} $. Let us call a generating set minimal, if no proper set is a generating set. Assume that $ m_1, \dots, m_n $ is a minimal generating set. $ IM = M $ implies that $ m_1 = a_1m_1 + \dots + a_nm_n $, where $ a_i \in I $. $ \rb{1 - a_1}m_1 = a_2m_2 + \dots + a_nm_n $. Proposition \ref{prop:5.3} says that $ 1 - a_1 \in A^* $. Hence $ m_1 = \rb{1 - a_1}^{-1}a_2m_2 + \dots + \rb{1 - a_1}^{-1}a_nm_n $. This is a contradiction, because $ m_2, \dots, m_n $ is a generating set.

\section{Localisation of modules}

\begin{definition}
Let $ A $ be a ring with a multiplicative set $ S $, and let $ M $ be an $ A $-module. Define $ \sim $ on $ M \times S $ by $ \rb{m, s} \sim \rb{n, t} $ if and only if there exists $ u \in S $ such that $ u\rb{tm - sn} = 0 $. This is an equivalence relation. Denote the equivalence class of $ \rb{m, s} $ by $ m / s $. Then the set of these equivalence classes form a module denoted by $ S^{-1}M $ over $ S^{-1}A $. The action of $ S^{-1}A $ on $ S^{-1}M $ is $ \rb{a / s}\rb{m / t} = \rb{am / st} $. $ m / s + n / t = \rb{mt + ns} / st $. The zero in $ S^{-1}M $ is $ 0 / 1 $.
\end{definition}

\begin{definition}
Let $ A $ be a ring and let $ P \subset A $ be a prime ideal. Then $ S = A \setminus P $ is a multiplicative set. The ring $ S^{-1}A $ is denoted $ A_P $. It is called the localisation of $ A $ at $ P $. Recall that by Theorem \ref{thm:6.9} the prime ideals of $ A_P $ are of the form $ S^{-1}I $, where $ I \subset A $ is a prime ideal such that $ I \cap \rb{A \setminus P} = \emptyset $, if and only if $ I \subset P $.
\end{definition}

\begin{theorem}
Let $ A $ be a ring with a prime ideal $ P $. Then $ a \in A_P $ is a unit if and only if $ a \notin PA_P = S^{-1}P = \rb{A \setminus P}^{-1}P $. The ideal $ PA_P $ is the unique maximal ideal of $ A_P $. So $ A_P $ is a local ring.
\end{theorem}

\begin{proof}
Suppose $ a / s \in A_P $ is a unit. Then for some $ b / t \in A_P $ we have $ \rb{a / s}\rb{b / t} = 1 $. $ ab / st - 1 / 1 = 0 $ if and only if there exists $ u \in S $ such that $ u\rb{ab - st} = 0 $. $ uab = ust \in S = A \setminus P $. Hence $ a \notin P $, so that $ a / s \notin PA_P $. Conversely, if $ a / s \notin PA_P $, then $ a \notin P $ and $ s \in S $ gives $ a \in S = A \setminus P $. So $ a / s $ is a unit whose inverse is $ s / a $. $ PA_P $ is a maximal ideal, because joining any new element will be the whole ring, as this element must be a unit.
\end{proof}

\begin{example}
$ \Z_{\rb{p}} = \cb{a / b \mid a, b \in \Z, \ \rb{p, b} = 1} $ and
$$ p\Z_{\rb{p}} = \cb{\dfrac{a}{b} \ \Big| \ a, b \in \Z, \ p \mid a, \ \rb{p, b} = 1}, \qquad \Z_{\rb{p}}^* = \cb{\dfrac{a}{b} \ \Big| \ a, b \in \Z, \ p \nmid a, \ \rb{p, b} = 1}. $$
Do the same for $ A = k\sb{x} $ and $ P = \rb{f\rb{x}} $, where $ f\rb{x} $ is irreducible.
\end{example}

\begin{proposition}
Let $ M $ be an $ A $-module. Then $ M = 0 $ if and only if $ M_P = 0 $ for all maximal ideals $ P \subset A $.
\end{proposition}

\begin{proof}
Suppose $ M \ne 0 $. Choose $ x \in M $, $ x \ne 0 $. Define $ I = Ann\rb{x} = \cb{a \in A \mid ax = 0} $. This is an ideal in $ A $, and $ I \ne A $ because $ 1 \cdot x = x $, so $ 1 \notin I $. Let $ P $ be a maximal ideal such that $ I \subset P $. Claim that $ M_P \ne 0 $. Consider $ x / 1 \in M_P $. If $ M_P = 0 $, then $ x / 0 = 0 / 1 $, so $ ux = 0 $ for some $ u \in A \setminus P $. $ u \in I = Ann\rb{x} $ but $ u \notin P $. This is a contradiction because $ I \subset P $.
\end{proof}

\marginpar{Lecture 12 \\ Tuesday \\ 30/10/18}

\section{Chain conditions: Noetherian and Artinian rings}

\begin{lemma}
Let $ \Sigma $ be a partially ordered set. Then the following properties are equivalent.
\begin{enumerate}
\item Every non-empty subset of $ \Sigma $ has a maximal element.
\item Every ascending chain $ x_1 \le x_2 \le \dots $ is stationary, that is there exists $ n $ such that for any $ m \ge 0 $ we have $ x_{n + m} = x_n $.
\end{enumerate}
\end{lemma}

\begin{proof}
\hfill
\begin{enumerate}[leftmargin=2cm]
\item[$ 1 \implies 2 $] Any ascending chain has a maximal element, say $ x_n $. Hence $ x_{m + n} = x_n $, for all $ m \ge 0 $.
\item[$ 2 \implies 1 $] Suppose $ S \subset \Sigma $ does not have a maximal element. Choose $ x_1 \in S $. There exists $ x_2 \in S $ such that $ x_2 > x_1 $. If $ x_1 < \dots < x_2 $ are chosen, then since $ x_n $ is not a maximal element, we can choose $ x_{n + 1} > x_n $. This constructs an ascending chain that is not stationary.
\end{enumerate}
\end{proof}

\begin{definition}
A ring $ A $ is called \textbf{Noetherian} if every ascending chain of ideals in $ A $ is stationary. An $ A $-module $ M $ is Noetherian if every chain of submodules of $ M $ is stationary. In particular, a ring $ A $ is Noetherian if it is a Noetherian module over $ A $. A ring $ A $ is called \textbf{Artinian} if every descending chain of ideals is stationary. An $ A $-module $ M $ is Artinian if every descending chain of submodules is stationary.
\end{definition}

\begin{example}
Let $ \Z \supset \rb{n} $ is Noetherian. $ \rb{a} \subset \ab{b} $ if and only if $ b $ divides $ a $. $ \rb{15} \subsetneq \rb{5} \subsetneq \rb{1} = \Z $. But $ \rb{2} \supsetneq \rb{4} \supsetneq \dots \supsetneq \rb{2^n} \supsetneq \dots $ is an infinite descending chain of ideals so $ \Z $ is not Artinian. If $ A $ is a finite ring, then it is trivially both Noetherian and Artinian.
\end{example}

\begin{proposition}
Let $ A $ be a ring and let $ M $ be an $ A $-module. Then $ M $ is Noetherian if and only if every submodule of $ M $ is finitely generated.
\end{proposition}

\begin{proof}
Suppose $ M $ is Noetherian, but $ N \subset M $ is a submodule that is not finitely generated. Then take $ x_1 \in N $. Since $ N \ne \rb{x_1} $, the submodule generated by $ x_1 $, we can find $ x_2 \in N \setminus \rb{x_1} $. This gives $ \rb{x_1} \subsetneq \rb{x_1, x_2} $ and so on. This produces an ascending chain which is not stationary, a contradiction. Now suppose that every submodule of $ M $ is $ f \cdot g $. Consider any ascending chain $ M_1 \subset M_2 \subset \dots $. Let $ N = \cup_{i \ge 1} M_i $. This is a submodule of $ M $. By assumption $ N = \rb{x_1, \dots, x_n} $ for some $ x_i \in N $. For each $ x_i $ there is an $ M_j $ in our chain such that $ x_i \in M_j $. So there will be some $ M_l $ that contains $ x_1, \dots, x_n $. Then $ N = M_l $. And clearly for any $ m \ge 0 $ we have $ M_l \subset M_{l + m} \subset N = M_l $, so $ M_{l + m} = M_l $. So $ M $ is Noetherian.
\end{proof}

\begin{remark}
Applying this to the $ A $-module $ A $ we see that $ A $ is Noetherian if and only if every ideal is finitely generated. Hence every principal ideal domain is Noetherian.
\end{remark}

\begin{example}
$ \Z $, $ k\sb{x} $, $ k\sb{x_1, \dots, x_n} $. Hilbert's basis theorem says that if $ R $ is Noetherian, then $ R\sb{x} $ is also Noetherian.
\end{example}

\begin{proposition}
\label{prop:10.4}
Let $ A $ be a ring. Let $ M $ be an $ A $-module and $ N \subset M $ a submodule. Then $ M $ is Noetherian if and only if $ N $ and $ M / N $ are both Noetherian $ A $-modules.
\end{proposition}

\begin{proof}
Suppose $ M $ is Noetherian. Then clearly $ N $ is Noetherian. $ M / N $ is Noetherian too. Indeed, let $ L $ be a submodule of $ M / N $. Let $ T $ be the inverse image of $ L $ in $ M $. Then we have a surjective homomorphism of $ A $-modules $ T \to L $. Since $ T $ is finitely generated, so that $ T = \rb{x_1, \dots, x_n} $ for some $ x_i \in T $. Then the images of $ x_1, \dots, x_n $ generate $ L $. Now assume $ N $ and $ M / N $ are Noetherian. This can also be proved using ascending chains. Take any ascending chain $ M_1 \subset M_2 \subset \dots $. Then $ N \cap M_1 \subset N \cap M_2 \subset \dots $ is an ascending chain of submodules of $ N $. Let $ n_1 \in \N $ be such that for all $ i \ge 0 $, $ N \cap M_{n + i} = N \cap M_{n_1} $. Consider $ \rb{M_i + N} / N \subset M / N $. This is just the set of cosets $ x + N $, where $ x \in M_i $. In fact $ \rb{M_i + N} / N \cong M_i / M \cap N $. We obtain an ascending chain $ \rb{M_1 + N} / N \subset \rb{M_2 + N} / N \subset \dots \subset \rb{M_{n_2} + N} / N = \rb{M_{n_1} + N} / N = \dots $. Take $ n = \max\rb{n_1, n_2} $. It works, that is $ M_n = M_{n + 1} = \dots $. Indeed, take any $ x \in M_{n + i} $ for $ i \ge 0 $. Then there exists $ y \in M_n $ such that $ x + N = y + N $. Thus $ x - y \in N \cap M_{n + i} $. But this is $ N \cap M_n $. So there exists $ z \in N \cap M_n $ such that $ x - y = z $. Hence $ x = y + z \in M_n $.
\end{proof}

\marginpar{Lecture 13 \\ Friday \\ 02/11/18}

Lecture 13 is a problem class.

\marginpar{Lecture 14 \\ Monday \\ 05/11/18}

(TODO Exercise: Do the same in the Artinian case)

\begin{corollary}
Let $ A $ be a Noetherian or Artinian ring. Let $ M $ be a finitely generated $ A $-module. Then $ M $ is Noetherian or Artinian.
\end{corollary}

\begin{proof}
Let $ M = \rb{m_1, \dots, m_n} $ for $ m_i \in M $. $ M = \cb{a_1m_1 + \dots + a_nm_n \mid a_i \in A} $. Let $ A^{\oplus n} = \cb{\rb{a_1, \dots, a_n} \mid a_i \in A} $ be a free $ A $-module of rank $ n $. There is a homomorphism of $ A $-modules $ A^{\oplus n} \to M $ sending $ \rb{a_1, \dots, a_n} $ to $ a_1m_1 + \dots + a_nm_n $. It is surjective. By Proposition \ref{prop:10.4} it is enough to show that $ A^{\oplus n} $ is Noetherian. Prove by induction in $ n $. Clearly, $ A $ is Noetherian. $ A^{\oplus \rb{n - 1}} \subset A^{\oplus n} $. The quotient $ A^{\oplus n} / A^{\oplus \rb{n - 1}} \cong A $ by $ \rb{a_1, \dots, a_n} \mapsto a_n $. By Proposition \ref{prop:10.4} $ A^{\oplus \rb{n - 1}} $ and $ A $ Noetherian implies that $ A^{\oplus n} $ is Noetherian too.
\end{proof}

\begin{corollary}
Let $ A $ be a ring and let $ M $ be an $ A $-module. Suppose that we have $ 0 = M_0 \subset \dots M_n = M $ are $ A $-submodules of $ M $. Then $ M $ is Noetherian or Artinian if and only if each quotient $ M_{i + 1} / M_i $ is Noetherian or Artinian.
\end{corollary}

\begin{proof}
Use Proposition \ref{prop:10.4}.
\end{proof}

\begin{lemma}
Let $ A $ be a Noetherian ring. Let $ S \subset A $ be a multiplicative set. Then $ S^{-1}A $ is Noetherian.
\end{lemma}

\begin{proof}
Consider a non-empty set $ \Sigma $ of ideals of $ S^{-1}A $. There is a canonical homomorphism of rings $ f : A \to S^{-1}A $ by $ f\rb{a} = a / 1 $. If $ I $ is an ideal of $ S^{-1}A $, then $ f^{-1}\rb{I} $ is an ideal in $ A $. Then $ I = S^{-1}f^{-1}\rb{I} $. Now $ \Sigma $ gives a non-empty set of ideals of $ A $ under $ I \to f^{-1}\rb{I} $. Let $ J $ be a maximal element of this set. Then $ S^{-1}J $ is a maximal element of $ \Sigma $. Hence $ S^{-1}A $ is Noetherian.
\end{proof}

\section{Primary decomposition}

\begin{definition}
An ideal $ Q $ in a ring $ R $ not equal to $ R $, that is a proper ideal, is called \textbf{primary} if all $ x, y \in R $ such that $ xy \in Q $ we have $ x \in Q $ or $ y^n \in Q $ for some $ n $.
\end{definition}

\begin{example}
Let $ p $ be a prime number. Then $ \rb{p^m} $ for $ m \ge 1 $ is a primary ideal in $ \Z $. $ ab \in \rb{p^m} $ if and only if $ p^m \mid ab $. Consider $ a $. If $ p \nmid a $, then $ p^m \mid b $, hence $ b \in \rb{p^m} $. Otherwise $ p \mid a $, then $ p^m \mid a^m $, so $ a^m \in \rb{p^m} $.
\end{example}

\end{document}