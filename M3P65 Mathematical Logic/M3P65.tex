\def\module{M3P65 Mathematical Logic}
\def\lecturer{Prof David Evans}
\def\term{Autumn 2018}
\def\cover{
\begin{align*}
\textbf{Extensionality} \equiv & \fab{x}\fab{y} \\
& \qquad \iffb{\eqb{x}{y}}{\fab{z}\iffb{\inb{z}{x}}{\inb{z}{y}}} \\
\textbf{Empty} \equiv & \teb{x}\fab{y} \\
& \qquad \notb{\inb{y}{x}} \\
\textbf{Pairing} \equiv & \fab{x}\fab{y}\teb{z}\fab{w} \\
& \qquad \iffb{\inb{w}{z}}{\orb{\eqb{w}{x}}{\eqb{w}{y}}} \\
\textbf{Union} \equiv & \fab{x}\teb{y}\fab{z} \\
& \qquad \iffb{\inb{z}{y}}{\teb{w}\andb{\inb{w}{x}}{\inb{z}{w}}} \\
\textbf{Power} \equiv & \fab{x}\teb{y}\fab{z} \\
& \qquad \iffb{\inb{z}{y}}{\fab{w}\impb{\inb{w}{z}}{\inb{w}{x}}} \\
\textbf{Specification} \equiv & \fab{x}\fab{y_1} \dots \fab{y_n}\teb{z}\fab{w} \\
& \qquad \iffb{\inb{w}{z}}{\andb{\inb{w}{x}}{\phi\rb{x, y_1, \dots, y_n}}} \\
\textbf{Infinity} \equiv & \teb{x} \\
& \qquad (\teb{y}\andb{\inb{y}{x}}{\fab{z}\notb{\inb{z}{y}}} \land \fab{y}(\inb{y}{x} \\
& \qquad \qquad \rightarrow \teb{z}\andb{\inb{z}{x}}{\andb{\fab{w}\impb{\inb{w}{y}}{\inb{w}{z}}}{\teb{w}\andb{\inb{w}{z}}{\notb{\inb{w}{y}}}}})) \\
\textbf{Replacement} \equiv & \fab{x}\fab{y_1} \dots \fab{y_n} \\
& \qquad (\fab{w}\impb{\inb{w}{x}}{\teb{v}\andb{\phi\rb{v, w, y_1, \dots, y_n}}{\fab{u}\impb{\phi\rb{u, w, y_1, \dots, y_n}}{\eqb{u}{v}}}} \\
& \qquad \qquad \rightarrow \teb{z}\fab{v}\impb{\inb{v}{z}}{\teb{w}\andb{\inb{w}{x}}{\phi\rb{v, w, y_1, \dots, y_n}}}) \\
\textbf{Regularity} \equiv & \fab{x} \\
& \qquad \impb{\teb{y}\inb{y}{x}}{\teb{y}\andb{\inb{y}{x}}{\fab{z}\andb{\notb{\inb{z}{x}}}{\notb{\inb{z}{y}}}}} \\
\textbf{Choice} \equiv & \fab{x} \\
& \qquad \impb{\teb{y}\inb{y}{x}}{\teb{\phi : x \to \bigcup x}\fab{z}\impb{\inb{z}{x}}{\inb{\phi\rb{z}}{z}}}
\end{align*}
}
\def\syllabus{Propositional formulas. The formal system $ L $. Soundness and completeness. First-order structures and languages. Bound and free variables. The formal system $ K_\LL $. G\"odel's completeness theorem. Equality. Cardinality. Zermelo-Fraenkel axioms for set theory. Well orderings. Ordinals. Transfinite induction. Transfinite recursion. Cardinals. Axiom of choice. Well ordering principle. Zorn's lemma.}

\def\thm{subsection}

\documentclass{article}

% Packages

\usepackage{amssymb}
\usepackage{amsthm}
\usepackage[UKenglish]{babel}
\usepackage{commath}
\usepackage{enumitem}
\usepackage{etoolbox}
\usepackage{fancyhdr}
\usepackage[margin=1in]{geometry}
\usepackage{graphicx}
\usepackage[hidelinks]{hyperref}
\usepackage[utf8]{inputenc}
\usepackage{listings}
\usepackage{mathtools}
\usepackage{stmaryrd}
\usepackage{tikz-cd}
\usepackage{csquotes}

% Formatting

\addto\captionsUKenglish{\renewcommand{\abstractname}{Syllabus}}
\delimitershortfall5pt
\ifx\thm\undefined\newtheorem{n}{}\else\newtheorem{n}{}[\thm]\fi
\newcommand\newoperator[1]{\ifcsdef{#1}{\cslet{#1}{\relax}}{}\csdef{#1}{\operatorname{#1}}}
\setlength{\parindent}{0cm}

% Environments

\theoremstyle{plain}
\newtheorem{algorithm}[n]{Algorithm}
\newtheorem*{algorithm*}{Algorithm}
\newtheorem{algorithm**}{Algorithm}
\newtheorem{conjecture}[n]{Conjecture}
\newtheorem*{conjecture*}{Conjecture}
\newtheorem{conjecture**}{Conjecture}
\newtheorem{corollary}[n]{Corollary}
\newtheorem*{corollary*}{Corollary}
\newtheorem{corollary**}{Corollary}
\newtheorem{lemma}[n]{Lemma}
\newtheorem*{lemma*}{Lemma}
\newtheorem{lemma**}{Lemma}
\newtheorem{proposition}[n]{Proposition}
\newtheorem*{proposition*}{Proposition}
\newtheorem{proposition**}{Proposition}
\newtheorem{theorem}[n]{Theorem}
\newtheorem*{theorem*}{Theorem}
\newtheorem{theorem**}{Theorem}

\theoremstyle{definition}
\newtheorem{aim}[n]{Aim}
\newtheorem*{aim*}{Aim}
\newtheorem{aim**}{Aim}
\newtheorem{axiom}[n]{Axiom}
\newtheorem*{axiom*}{Axiom}
\newtheorem{axiom**}{Axiom}
\newtheorem{condition}[n]{Condition}
\newtheorem*{condition*}{Condition}
\newtheorem{condition**}{Condition}
\newtheorem{definition}[n]{Definition}
\newtheorem*{definition*}{Definition}
\newtheorem{definition**}{Definition}
\newtheorem{example}[n]{Example}
\newtheorem*{example*}{Example}
\newtheorem{example**}{Example}
\newtheorem{exercise}[n]{Exercise}
\newtheorem*{exercise*}{Exercise}
\newtheorem{exercise**}{Exercise}
\newtheorem{fact}[n]{Fact}
\newtheorem*{fact*}{Fact}
\newtheorem{fact**}{Fact}
\newtheorem{goal}[n]{Goal}
\newtheorem*{goal*}{Goal}
\newtheorem{goal**}{Goal}
\newtheorem{law}[n]{Law}
\newtheorem*{law*}{Law}
\newtheorem{law**}{Law}
\newtheorem{plan}[n]{Plan}
\newtheorem*{plan*}{Plan}
\newtheorem{plan**}{Plan}
\newtheorem{problem}[n]{Problem}
\newtheorem*{problem*}{Problem}
\newtheorem{problem**}{Problem}
\newtheorem{question}[n]{Question}
\newtheorem*{question*}{Question}
\newtheorem{question**}{Question}
\newtheorem{warning}[n]{Warning}
\newtheorem*{warning*}{Warning}
\newtheorem{warning**}{Warning}
\newtheorem{acknowledgements}[n]{Acknowledgements}
\newtheorem*{acknowledgements*}{Acknowledgements}
\newtheorem{acknowledgements**}{Acknowledgements}
\newtheorem{annotations}[n]{Annotations}
\newtheorem*{annotations*}{Annotations}
\newtheorem{annotations**}{Annotations}
\newtheorem{assumption}[n]{Assumption}
\newtheorem*{assumption*}{Assumption}
\newtheorem{assumption**}{Assumption}
\newtheorem{conclusion}[n]{Conclusion}
\newtheorem*{conclusion*}{Conclusion}
\newtheorem{conclusion**}{Conclusion}
\newtheorem{claim}[n]{Claim}
\newtheorem*{claim*}{Claim}
\newtheorem{claim**}{Claim}
\newtheorem{notation}[n]{Notation}
\newtheorem*{notation*}{Notation}
\newtheorem{notation**}{Notation}
\newtheorem{note}[n]{Note}
\newtheorem*{note*}{Note}
\newtheorem{note**}{Note}
\newtheorem{remark}[n]{Remark}
\newtheorem*{remark*}{Remark}
\newtheorem{remark**}{Remark}

% Lectures

\newcommand{\lecture}[3]{ % Lecture
  \marginpar{
    Lecture #1 \\
    #2 \\
    #3
  }
}

% Blackboard

\renewcommand{\AA}{\mathbb{A}} % Blackboard A
\newcommand{\BB}{\mathbb{B}}   % Blackboard B
\newcommand{\CC}{\mathbb{C}}   % Blackboard C
\newcommand{\DD}{\mathbb{D}}   % Blackboard D
\newcommand{\EE}{\mathbb{E}}   % Blackboard E
\newcommand{\FF}{\mathbb{F}}   % Blackboard F
\newcommand{\GG}{\mathbb{G}}   % Blackboard G
\newcommand{\HH}{\mathbb{H}}   % Blackboard H
\newcommand{\II}{\mathbb{I}}   % Blackboard I
\newcommand{\JJ}{\mathbb{J}}   % Blackboard J
\newcommand{\KK}{\mathbb{K}}   % Blackboard K
\newcommand{\LL}{\mathbb{L}}   % Blackboard L
\newcommand{\MM}{\mathbb{M}}   % Blackboard M
\newcommand{\NN}{\mathbb{N}}   % Blackboard N
\newcommand{\OO}{\mathbb{O}}   % Blackboard O
\newcommand{\PP}{\mathbb{P}}   % Blackboard P
\newcommand{\QQ}{\mathbb{Q}}   % Blackboard Q
\newcommand{\RR}{\mathbb{R}}   % Blackboard R
\renewcommand{\SS}{\mathbb{S}} % Blackboard S
\newcommand{\TT}{\mathbb{T}}   % Blackboard T
\newcommand{\UU}{\mathbb{U}}   % Blackboard U
\newcommand{\VV}{\mathbb{V}}   % Blackboard V
\newcommand{\WW}{\mathbb{W}}   % Blackboard W
\newcommand{\XX}{\mathbb{X}}   % Blackboard X
\newcommand{\YY}{\mathbb{Y}}   % Blackboard Y
\newcommand{\ZZ}{\mathbb{Z}}   % Blackboard Z

% Brackets

\renewcommand{\eval}[1]{\left. #1 \right|}          % Evaluation
\newcommand{\br}{\del}                              % Brackets
\newcommand{\abr}[1]{\left\langle #1 \right\rangle} % Angle brackets
\newcommand{\fbr}[1]{\left\lfloor #1 \right\rfloor} % Floor brackets
\newcommand{\lbr}[1]{\left\lfloor #1 \right\rfloor} % Ceiling brackets
\newcommand{\st}{\ \middle| \ }                     % Such that

% Calligraphic

\newcommand{\AAA}{\mathcal{A}} % Calligraphic A
\newcommand{\BBB}{\mathcal{B}} % Calligraphic B
\newcommand{\CCC}{\mathcal{C}} % Calligraphic C
\newcommand{\DDD}{\mathcal{D}} % Calligraphic D
\newcommand{\EEE}{\mathcal{E}} % Calligraphic E
\newcommand{\FFF}{\mathcal{F}} % Calligraphic F
\newcommand{\GGG}{\mathcal{G}} % Calligraphic G
\newcommand{\HHH}{\mathcal{H}} % Calligraphic H
\newcommand{\III}{\mathcal{I}} % Calligraphic I
\newcommand{\JJJ}{\mathcal{J}} % Calligraphic J
\newcommand{\KKK}{\mathcal{K}} % Calligraphic K
\newcommand{\LLL}{\mathcal{L}} % Calligraphic L
\newcommand{\MMM}{\mathcal{M}} % Calligraphic M
\newcommand{\NNN}{\mathcal{N}} % Calligraphic N
\newcommand{\OOO}{\mathcal{O}} % Calligraphic O
\newcommand{\PPP}{\mathcal{P}} % Calligraphic P
\newcommand{\QQQ}{\mathcal{Q}} % Calligraphic Q
\newcommand{\RRR}{\mathcal{R}} % Calligraphic R
\newcommand{\SSS}{\mathcal{S}} % Calligraphic S
\newcommand{\TTT}{\mathcal{T}} % Calligraphic T
\newcommand{\UUU}{\mathcal{U}} % Calligraphic U
\newcommand{\VVV}{\mathcal{V}} % Calligraphic V
\newcommand{\WWW}{\mathcal{W}} % Calligraphic W
\newcommand{\XXX}{\mathcal{X}} % Calligraphic X
\newcommand{\YYY}{\mathcal{Y}} % Calligraphic Y
\newcommand{\ZZZ}{\mathcal{Z}} % Calligraphic Z

% Fraktur

\newcommand{\aaa}{\mathfrak{a}}   % Fraktur a
\newcommand{\bbb}{\mathfrak{b}}   % Fraktur b
\newcommand{\ccc}{\mathfrak{c}}   % Fraktur c
\newcommand{\ddd}{\mathfrak{d}}   % Fraktur d
\newcommand{\eee}{\mathfrak{e}}   % Fraktur e
\newcommand{\fff}{\mathfrak{f}}   % Fraktur f
\renewcommand{\ggg}{\mathfrak{g}} % Fraktur g
\newcommand{\hhh}{\mathfrak{h}}   % Fraktur h
\newcommand{\iii}{\mathfrak{i}}   % Fraktur i
\newcommand{\jjj}{\mathfrak{j}}   % Fraktur j
\newcommand{\kkk}{\mathfrak{k}}   % Fraktur k
\renewcommand{\lll}{\mathfrak{l}} % Fraktur l
\newcommand{\mmm}{\mathfrak{m}}   % Fraktur m
\newcommand{\nnn}{\mathfrak{n}}   % Fraktur n
\newcommand{\ooo}{\mathfrak{o}}   % Fraktur o
\newcommand{\ppp}{\mathfrak{p}}   % Fraktur p
\newcommand{\qqq}{\mathfrak{q}}   % Fraktur q
\newcommand{\rrr}{\mathfrak{r}}   % Fraktur r
\newcommand{\sss}{\mathfrak{s}}   % Fraktur s
\newcommand{\ttt}{\mathfrak{t}}   % Fraktur t
\newcommand{\uuu}{\mathfrak{u}}   % Fraktur u
\newcommand{\vvv}{\mathfrak{v}}   % Fraktur v
\newcommand{\www}{\mathfrak{w}}   % Fraktur w
\newcommand{\xxx}{\mathfrak{x}}   % Fraktur x
\newcommand{\yyy}{\mathfrak{y}}   % Fraktur y
\newcommand{\zzz}{\mathfrak{z}}   % Fraktur z

% Geometry

\newcommand{\CP}{\mathbb{CP}}                                              % Complex projective space
\newcommand{\iintd}[4]{\iint_{#1} \, #2 \, \dif #3 \, \dif #4}             % Double integral
\newcommand{\RP}{\mathbb{RP}}                                              % Real projective space
\newcommand{\intd}[4]{\int_{#1}^{#2} \, #3 \, \dif #4}                     % Single integral
\newcommand{\iiintd}[5]{\iint_{#1} \, #2 \, \dif #3 \, \dif #4 \, \dif #5} % Triple integral

% Logic

\newcommand{\iffb}[2]{\br{#1 \leftrightarrow #2}} % Biconditional
\newcommand{\andb}[2]{\br{#1 \land #2}}           % Conjunction
\newcommand{\orb}[2]{\br{#1 \lor #2}}             % Disjunction
\newcommand{\nib}[2]{\br{#1 \notin #2}}           % Element of
\newcommand{\eqb}[2]{\br{#1 = #2}}                % Equal to
\newcommand{\teb}[1]{\br{\exists #1}}             % Existential quantifier
\newcommand{\impb}[2]{\br{#1 \rightarrow #2}}     % Implication
\newcommand{\ltb}[2]{\br{#1 < #2}}                % Less than
\newcommand{\leb}[2]{\br{#1 \le #2}}              % Less than or equal to
\newcommand{\notb}[1]{\br{\neg #1}}               % Negation
\newcommand{\inb}[2]{\br{#1 \in #2}}              % Not element of
\newcommand{\neb}[2]{\br{#1 \ne #2}}              % Not equal to
\newcommand{\subb}[2]{\br{#1 \subseteq #2}}       % Subset
\newcommand{\fab}[1]{\br{\forall #1}}             % Universal quantifier

% Maps

\newcommand{\bijection}[7][]{    % Bijection
  \ifx &#1&
    \begin{array}{rcl}
      #2 & \longleftrightarrow & #3 \\
      #4 & \longmapsto         & #5 \\
      #6 & \longmapsfrom       & #7
    \end{array}
  \else
    \begin{array}{ccrcl}
      #1 & : & #2 & \longrightarrow & #3 \\
         &   & #4 & \longmapsto     & #5 \\
         &   & #6 & \longmapsfrom   & #7
    \end{array}
  \fi
}
\newcommand{\birational}[7][]{   % Birational map
  \ifx &#1&
    \begin{array}{rcl}
      #2 & \dashrightarrow & #3 \\
      #4 & \longmapsto     & #5 \\
      #6 & \longmapsfrom   & #7
    \end{array}
  \else
    \begin{array}{ccrcl}
      #1 & : & #2 & \dashrightarrow & #3 \\
         &   & #4 & \longmapsto     & #5 \\
         &   & #6 & \longmapsfrom   & #7
    \end{array}
  \fi
}
\newcommand{\correspondence}[2]{ % Correspondence
  \cbr{
    \begin{array}{c}
      #1
    \end{array}
  }
  \qquad
  \leftrightsquigarrow
  \qquad
  \cbr{
    \begin{array}{c}
      #2
    \end{array}
  }
}
\newcommand{\function}[5][]{     % Function
  \ifx &#1&
    \begin{array}{rcl}
      #2 & \longrightarrow & #3 \\
      #4 & \longmapsto     & #5
    \end{array}
  \else
    \begin{array}{ccrcl}
      #1 & : & #2 & \longrightarrow & #3 \\
         &   & #4 & \longmapsto     & #5
    \end{array}
  \fi
}
\newcommand{\functions}[7][]{    % Functions
  \ifx &#1&
    \begin{array}{rcl}
      #2 & \longrightarrow & #3 \\
      #4 & \longmapsto     & #5 \\
      #6 & \longmapsto     & #7
    \end{array}
  \else
    \begin{array}{ccrcl}
      #1 & : & #2 & \longrightarrow & #3 \\
         &   & #4 & \longmapsto     & #5 \\
         &   & #6 & \longmapsto     & #7
    \end{array}
  \fi
}
\newcommand{\rational}[5][]{     % Rational map
  \ifx &#1&
    \begin{array}{rcl}
      #2 & \dashrightarrow & #3 \\
      #4 & \longmapsto     & #5
    \end{array}
  \else
    \begin{array}{ccrcl}
      #1 & : & #2 & \dashrightarrow & #3 \\
         &   & #4 & \longmapsto     & #5
    \end{array}
  \fi
}

% Matrices

\newcommand{\onebytwo}[2]{      % One by two matrix
  \begin{pmatrix}
    #1 & #2
  \end{pmatrix}
}
\newcommand{\onebythree}[3]{    % One by three matrix
  \begin{pmatrix}
    #1 & #2 & #3
  \end{pmatrix}
}
\newcommand{\twobyone}[2]{      % Two by one matrix
  \begin{pmatrix}
    #1 \\
    #2
  \end{pmatrix}
}
\newcommand{\twobytwo}[4]{      % Two by two matrix
  \begin{pmatrix}
    #1 & #2 \\
    #3 & #4
  \end{pmatrix}
}
\newcommand{\threebyone}[3]{    % Three by one matrix
  \begin{pmatrix}
    #1 \\
    #2 \\
    #3
  \end{pmatrix}
}
\newcommand{\threebythree}[9]{  % Three by three matrix
  \begin{pmatrix}
    #1 & #2 & #3 \\
    #4 & #5 & #6 \\
    #7 & #8 & #9
  \end{pmatrix}
}
\newcommand{\twobytwosmall}[4]{ % Two by two small matrix
  \begin{psmallmatrix}
    #1 & #2 \\
    #3 & #4
  \end{psmallmatrix}
}

% Number theory

\renewcommand{\symbol}[2]{\br{\tfrac{#1}{#2}}} % Power residue symbol
\newcommand{\unit}[1]{\br{\ZZ / #1\ZZ}^\times} % Unit group

% Operators

\newoperator{ab}    % Abelian
\newoperator{AG}    % Affine geometry
\newoperator{alg}   % Algebraic
\newoperator{Ann}   % Annihilator
\newoperator{area}  % Area
\newoperator{Aut}   % Automorphism
\newoperator{card}  % Cardinality
\newoperator{ch}    % Characteristic
\newoperator{Cl}    % Class
\newoperator{Coker} % Cokernel
\newoperator{col}   % Column
\newoperator{Corr}  % Correspondence
\newoperator{diam}  % Diameter
\newoperator{Disc}  % Discriminant
\newoperator{dom}   % Domain
\newoperator{Eig}   % Eigenvalue
\newoperator{Em}    % Embedding
\newoperator{End}   % Endomorphism
\newoperator{fin}   % Finite
\newoperator{Fix}   % Fixed
\newoperator{Frac}  % Fraction
\newoperator{Frob}  % Frobenius
\newoperator{Fun}   % Function
\newoperator{Gal}   % Galois
\newoperator{GL}    % General linear
\newoperator{Ham}   % Hamming
\newoperator{Homeo} % Homeomorphism
\newoperator{Hom}   % Homomorphism
\newoperator{id}    % Identity
\newoperator{Im}    % Image
\newoperator{Ind}   % Index
\newoperator{Ker}   % Kernel
\newoperator{lcm}   % Least common multiple
\newoperator{Mat}   % Matrix
\newoperator{mult}  % Multiplicity
\newoperator{new}   % New
\newoperator{Nm}    % Norm
\newoperator{old}   % Old
\newoperator{op}    % Opposite
\newoperator{ord}   % Order
\newoperator{Pay}   % Payley
\newoperator{PG}    % Projective geometry
\newoperator{PGL}   % Projective general linear
\newoperator{PSL}   % Projective special linear
\newoperator{rad}   % Radical
\newoperator{ran}   % Range
\newoperator{Res}   % Residue
\newoperator{rk}    % Rank
\newoperator{Re}    % Real
\newoperator{row}   % Row
\newoperator{sgn}   % Sign
\newoperator{Sing}  % Singular
\newoperator{SK}    % Skeleton
\newoperator{sp}    % Span
\newoperator{SL}    % Special linear
\newoperator{SO}    % Special orthogonal
\newoperator{Spec}  % Spectrum
\newoperator{Stab}  % Stabiliser
\newoperator{star}  % Star
\newoperator{srg}   % Strongly regular graph
\newoperator{supp}  % Support
\newoperator{Sym}   % Symmetric
\newoperator{tors}  % Torsion
\newoperator{Tr}    % Trace
\newoperator{vol}   % Volume
\newoperator{wt}    % Weight

% Roman

\newcommand{\A}{\mathrm{A}}   % Roman A
\newcommand{\B}{\mathrm{B}}   % Roman B
\newcommand{\C}{\mathrm{C}}   % Roman C
\newcommand{\D}{\mathrm{D}}   % Roman D
\newcommand{\E}{\mathrm{E}}   % Roman E
\newcommand{\F}{\mathrm{F}}   % Roman F
\newcommand{\G}{\mathrm{G}}   % Roman G
\renewcommand{\H}{\mathrm{H}} % Roman H
\newcommand{\I}{\mathrm{I}}   % Roman I
\newcommand{\J}{\mathrm{J}}   % Roman J
\newcommand{\K}{\mathrm{K}}   % Roman K
\renewcommand{\L}{\mathrm{L}} % Roman L
\newcommand{\M}{\mathrm{M}}   % Roman M
\newcommand{\N}{\mathrm{N}}   % Roman N
\renewcommand{\O}{\mathrm{O}} % Roman O
\renewcommand{\P}{\mathrm{P}} % Roman P
\newcommand{\Q}{\mathrm{Q}}   % Roman Q
\newcommand{\R}{\mathrm{R}}   % Roman R
\renewcommand{\S}{\mathrm{S}} % Roman S
\newcommand{\T}{\mathrm{T}}   % Roman T
\newcommand{\U}{\mathrm{U}}   % Roman U
\newcommand{\V}{\mathrm{V}}   % Roman V
\newcommand{\W}{\mathrm{W}}   % Roman W
\newcommand{\X}{\mathrm{X}}   % Roman X
\newcommand{\Y}{\mathrm{Y}}   % Roman Y
\newcommand{\Z}{\mathrm{Z}}   % Roman Z

\renewcommand{\a}{\mathrm{a}} % Roman a
\renewcommand{\b}{\mathrm{b}} % Roman b
\renewcommand{\c}{\mathrm{c}} % Roman c
\renewcommand{\d}{\mathrm{d}} % Roman d
\newcommand{\e}{\mathrm{e}}   % Roman e
\newcommand{\f}{\mathrm{f}}   % Roman f
\newcommand{\g}{\mathrm{g}}   % Roman g
\newcommand{\h}{\mathrm{h}}   % Roman h
\renewcommand{\i}{\mathrm{i}} % Roman i
\renewcommand{\j}{\mathrm{j}} % Roman j
\renewcommand{\k}{\mathrm{k}} % Roman k
\renewcommand{\l}{\mathrm{l}} % Roman l
\newcommand{\m}{\mathrm{m}}   % Roman m
\renewcommand{\n}{\mathrm{n}} % Roman n
\renewcommand{\o}{\mathrm{o}} % Roman o
\newcommand{\p}{\mathrm{p}}   % Roman p
\newcommand{\q}{\mathrm{q}}   % Roman q
\renewcommand{\r}{\mathrm{r}} % Roman r
\newcommand{\s}{\mathrm{s}}   % Roman s
\renewcommand{\t}{\mathrm{t}} % Roman t
\renewcommand{\u}{\mathrm{u}} % Roman u
\renewcommand{\v}{\mathrm{v}} % Roman v
\newcommand{\w}{\mathrm{w}}   % Roman w
\newcommand{\x}{\mathrm{x}}   % Roman x
\newcommand{\y}{\mathrm{y}}   % Roman y
\newcommand{\z}{\mathrm{z}}   % Roman z

% Tikz

\tikzset{
  arrow symbol/.style={"#1" description, allow upside down, auto=false, draw=none, sloped},
  subset/.style={arrow symbol={\subset}},
  cong/.style={arrow symbol={\cong}}
}

% Fancy header

\pagestyle{fancy}
\lhead{\module}
\rhead{\nouppercase{\leftmark}}

% Make title

\title{\module}
\author{Lectured by \lecturer \\ Typed by David Kurniadi Angdinata}
\date{\term}

\begin{document}

% Title page
\maketitle
\cover
\vfill
\begin{abstract}
\noindent\syllabus
\end{abstract}

\pagebreak

% Contents page
\tableofcontents

\pagebreak

% Document page
\setcounter{section}{-1}

\section{Introduction}

\lecture{1}{Thursday}{04/10/18}

The module is concerned with some of the foundational issues of mathematics. In propositional and predicate logic, we analyse the way in which we reason formally about mathematical structures. In set theory, we will look at the ZFC axioms and use these to develop the notion of cardinality. These topics have applications to other areas of mathematics. Formal logic has applications via model theory and ZFC provides an essential toolkit for handling infinite objects.

The first level of the subject is propositional logic. We look at the way simple statements or propositions can be built into more complicated ones using connectives and make precise how the truth or falsity of the component statements influences the truth or falsity of the compound statement. This is done using truth tables and can be useful for testing the validity of various forms of reasoning. It provides a way of analysing deductions. We then move on to a completely symbolic process of deduction and describe the formal deduction system for propositional calculus. The statements we consider, propositional formulas, are regarded as strings of symbols and we give rules for deducing a new formula from a given collection of formulas. We want these deduction rules to have the property that anything that could be deduced using truth tables, so by considering truth or falsity of the various statements, can be deduced in this formal way, and vice versa. This is the soundness and completeness of our formal system.

The next level of the subject is predicate logic. This is what is needed to analyse real mathematics and the extra ingredient is the use of quantifiers. We introduce the notion of a first-order structure, which is general enough to include many of the algebraic objects you come across in mathematics. We then have to be precise about the expressions, formulas, which make statements about these structures, and give a precise definition of what it means for a particular formula to be true in a structure. This is quite intricate, and the clever part is in getting the definitions right, but it corresponds to ordinary mathematical usage. Once this is done, we set up a formal deduction system for predicate logic. This parallels what we did for propositional logic, but is much harder. Nevertheless, the end result is the same. The formulas which are produced by our formal deduction system, the theorems, are precisely the formulas which are true in all first-order structures. This is G\"odel's completeness theorem.

Set theory provides the basic foundations and the language in which most of modern mathematics can be expressed, as well as the means for discussing the various notions of sizes of infinity. For example, although the set of natural numbers, the set of integers and the set of real numbers are all infinite, there is a very natural sense in which the first two have the same size, whereas the third is strictly bigger. This is expressed properly in the notion of cardinality. To avoid paradoxes and inconsistencies, we have to be careful about what collections of objects we allow to be called sets. This is done by the Zermelo-Fraenkel axioms, which essentially tell us how we are allowed to create new sets out of old ones. Of course, having laid down these quite rigid rules, we have to show that they are sufficiently flexible to allow us to talk about everyday objects of mathematics. There are also situations in mathematics where an extra axiom is needed, the axiom of choice. For example without this axiom, or rather, method of construction of a new set from given ones, we cannot show that every vector space has a basis. But it also has some slightly counterintuitive consequences, and we shall also look at some of these.

The lecture notes should be fairly self-contained, but the following books might also be of use. You might find that the notation which they use differs form that used in the lectures. You will be able to find various lecture notes on the internet. Some will be good, others not so good.
\begin{enumerate}
\item P Johnstone, Notes on logic and set theory, 1987
\item P J Cameron, Sets, logic and categories, 1999
\item A G Hamilton, Logic for mathematicians, 1988
\item R Cori and D Lascar, Mathematical logic: a course with exercises parts I and II, 2001
\item K Hrbaček and T Jech, Introduction to set theory 3rd edition, 1999
\end{enumerate}
$ 1 $ is very concise, but covers a surprising amount. $ 2 $ is friendlier, but skips some of the harder material. $ 4 $ is quite comprehensive and also available in the original French. $ 3 $ is useful for the logic part and $ 5 $ is a very nice introduction to set theory.

\pagebreak

\section{Propositional logic}

Let $ p $ be 'Mr Jones is happy' and $ q $ be 'Mrs Jones is unhappy'. Then 'if Mr Jones is happy, then Mrs Jones is unhappy, and if Mrs Jones is unhappy, then Mr Jones is unhappy, so Mr Jones is unhappy' is
$$ \impb{\andb{\impb{p}{q}}{\impb{q}{\notb{p}}}}{\notb{p}}. $$

\subsection{Propositional formulas}

A \textbf{proposition} is a statement that is either \textbf{True} $ \rb{T} $ or \textbf{False} $ \rb{F} $. The following are \textbf{truth table rules}.

\begin{definition}
\label{def:1.1.1}
Represent propositions symbolically using \textbf{propositional variables}.
$$ p, \quad q, \quad \dots \qquad p_1, \quad p_2, \quad \dots $$
Combine basic propositions into others using \textbf{connectives}.
\begin{itemize}
\item \textbf{negation (not)}.
\begin{align*}
\notb{p} \ \text{has value} \ F \qquad & \iff \qquad p \ \text{has value} \ T. \\
\intertext{\item \textbf{conjunction (and)}.}
\andb{p}{q} \ \text{has value} \ T \qquad & \iff \qquad p \ \text{and} \ q \ \text{both have value} \ T. \\
\intertext{\item \textbf{disjunction (or)}.}
\orb{p}{q} \ \text{has value} \ T \qquad & \iff \qquad \text{at least one of} \ p \ \text{and} \ q \ \text{has value} \ T. \\
\intertext{\item \textbf{implication (implies)}.}
\impb{p}{q} \ \text{has value} \ F \qquad & \iff \qquad p \ \text{has value} \ T \ \text{and} \ q \ \text{has value} \ F. \\
\intertext{\item \textbf{biconditional (iff)}.}
\iffb{p}{q} \ \text{has value} \ T \qquad & \iff \qquad p \ \text{and} \ q \ \text{have the same value}.
\end{align*}
\end{itemize}
The following is a \textbf{truth table}.
$$
\begin{array}{cc|cccc}
p & q & \andb{p}{q} & \orb{p}{q} & \impb{p}{q} & \iffb{p}{q} \\
\hline
T & T & T & T & T & T \\
T & F & F & T & F & F \\
F & T & F & T & T & F \\
F & F & F & F & T & T
\end{array}
$$
\end{definition}

\begin{definition}
\label{def:1.1.2}
A \textbf{propositional formula} is obtained in the following way.
\begin{itemize}
\item Any propositional variable is a formula.
\item If $ \phi $ and $ \psi $ are formulas, then so are
$$ \notb{\phi}, \qquad \andb{\phi}{\psi}, \qquad \orb{\phi}{\psi}, \qquad \impb{\phi}{\psi}, \qquad \iffb{\phi}{\psi}. $$
\item Any formula arises in this way.
\end{itemize}
\end{definition}

\pagebreak

\begin{example*}
\hfill
\begin{itemize}
\item Formulas.
$$ p_1 \qquad p_2 \qquad \notb{p_1} \qquad \impb{p_1}{\notb{p_2}} \qquad \impb{\impb{p_1}{\notb{p_2}}}{p_2} $$
\item Not formulas.
$$ p_1 \land p_2 \quad \rb{\text{missing brackets}} \qquad )( \neg p_1 \quad \rb{\text{not well-formed}} $$
\end{itemize}
\end{example*}

Because of the brackets, every formula is either a propositional variable or is built from shorter formulas in a unique way. Arguments are often proved by induction, on length of the formula, or the number of connectives in the formula.

\begin{definition}
Let $ n \in \N $.
\begin{itemize}
\item A \textbf{truth function} of $ n $ variables is a function $ f : \cb{T, F}^n \to \cb{T, F} $, where
$$ \cb{T, F}^n = \cb{\rb{x_1, \dots, x_n} \mid x_i \in \cb{T, F}}. $$
(Exercise: how many truth functions of $ n $ variables are there?)
\item Suppose $ \phi $ is a formula whose variables are amongst $ p_1, \dots, p_n $. We obtain a truth function
$$ F_\phi : \cb{T, F}^n \to \cb{T, F}, $$
whose value at $ \rb{x_1, \dots, x_n} $ is the truth value of $ \phi $ when $ p_i $ has value $ x_i $, for $ i = 1, \dots, n $, computed using the rules in \ref{def:1.1.1}. $ F_\phi $ is the truth function of $ \phi $.
\end{itemize}
\end{definition}

\begin{example*}
Let
$$ \phi : \impb{\impb{p}{\notb{q}}}{p}. $$
$$
\begin{array}{cc|ccc}
p & q & \notb{q} & \impb{p}{\notb{q}} & \phi \\
\hline
T & T & F & F & T \\
T & F & T & T & T \\
F & T & F & T & F \\
F & F & T & T & F
\end{array}
$$
So, for example, $ F_\phi\rb{T, F} = T $. The following is its \textbf{condensed form}.
$$
\begin{array}{rcrlcl}
((p & \rightarrow & (\neg & q)) & \rightarrow & p) \\
\hline
T & F & F & T & T & T \\
T & T & T & F & T & T \\
F & T & F & T & F & F \\
F & T & T & F & F & F \\
& & & & \phi &
\end{array}
$$
\end{example*}

\lecture{2}{Friday}{05/10/18}

\begin{example*}
What is the truth function of $ \impb{\andb{\impb{p}{q}}{\impb{q}{\notb{p}}}}{\notb{p}} $? Always $ T $.
\end{example*}

\begin{definition}
\hfill
\begin{itemize}
\item A propositional formula $ \phi $ is a \textbf{tautology} if its truth function $ F_\phi $ always has value $ T $.
\item Say that formulas $ \phi, \psi $ are \textbf{logically equivalent (LE)} if they have the same truth function, that is $ F_\phi = F_\psi $.
\end{itemize}
\end{definition}

\begin{remark}
\label{rem:1.1.5}
\hfill
\begin{enumerate}
\item $ \phi, \psi $ are logically equivalent if and only if $ \iffb{\phi}{\psi} $ is a tautology.
\item Suppose $ \phi $ is a formula with variables $ p_1, \dots, p_n $, and $ \phi_1, \dots, \phi_n $ are formulas with variables $ q_1, \dots, q_r $. For each $ i \le n $, substitute $ \phi_i $ in place of $ p_i $ in $ \phi $. Then the result is a formula $ \theta $, and if $ \phi $ is a tautology, then so is $ \theta $.
\end{enumerate}
\end{remark}

\pagebreak

\begin{example}
Check that
$$ \impb{\impb{\notb{p_2}}{\notb{p_1}}}{\impb{p_1}{p_2}} $$
is a tautology. So by \ref{rem:1.1.5}.$ 2 $, if $ \phi_1 $ and $ \phi_2 $ are any formulas, then
$$ \impb{\impb{\notb{\phi_2}}{\notb{\phi_1}}}{\impb{\phi_1}{\phi_2}} $$
is a tautology.
\end{example}

\begin{proof}[Proof of \ref{rem:1.1.5}]
\hfill
\begin{enumerate}
\item Easy.
\item Prove
$$ F_\theta\rb{p_1, \dots, p_r} = F_\phi\rb{F_{\phi_1}\rb{q_1, \dots, q_r}, \dots, F_{\phi_n}\rb{q_1, \dots, q_r}}, $$
by induction on the number of connectives in $ \phi $.
\end{enumerate}
\end{proof}

\begin{example*}
Logically equivalent formulas.
\begin{align*}
\andb{p_1}{\andb{p_2}{p_3}} \ & \text{is logically equivalent to} \ \andb{\andb{p_1}{p_2}}{p_3}. \\
\orb{p_1}{\orb{p_2}{p_3}} \ & \text{is logically equivalent to} \ \orb{\orb{p_1}{p_2}}{p_3}. \\
\orb{p_1}{\andb{p_2}{p_3}} \ & \text{is logically equivalent to} \ \andb{\orb{p_1}{p_2}}{\orb{p_1}{p_3}}. \\
\andb{p_1}{\orb{p_2}{p_3}} \ & \text{is logically equivalent to} \ \orb{\andb{p_1}{p_2}}{\andb{p_1}{p_3}}. \\
\notb{\notb{p_1}} \ & \text{is logically equivalent to} \ p_1. \\
\notb{\andb{p_1}{p_2}} \ & \text{is logically equivalent to} \ \orb{\notb{p_1}}{\notb{p_2}}. \\
\notb{\orb{p_1}{p_2}} \ & \text{is logically equivalent to} \ \andb{\notb{p_1}}{\notb{p_2}}.
\end{align*}
So we usually omit brackets in $ \rb{p_1 \land p_2 \land p_3} $ and $ \rb{p_1 \lor p_2 \lor p_3} $.
\end{example*}

\begin{note*}
By \ref{rem:1.1.5}, we obtain, for formulas $ \phi, \psi, \chi $,
$$ \andb{\phi}{\andb{\psi}{\chi}} \ \text{is logically equivalent to} \ \andb{\andb{\phi}{\psi}}{\chi}, $$
etc.
\end{note*}

\begin{lemma}
There are $ 2^{2^n} $ truth functions of $ n $ variables.
\end{lemma}

\begin{proof}
A truth function is a function
$$ F : \cb{T, F}^n \to \cb{T, F}. $$
$ \abs{\cb{T, F}^n} = 2^n $, and for each $ \overline{x} \in \cb{T, F}^n $, $ F\rb{\overline{x}} \in \cb{T, F} $. Hence the result.
\end{proof}

\pagebreak

\begin{definition}
Say that a set of connectives is \textbf{adequate} if for every $ n \ge 1 $, every truth function of $ n $ variables is the truth function of some formula which involves only connectives from the set, and variables $ p_1, \dots, p_n $.
\end{definition}

\begin{theorem}
\label{thm:1.1.9}
The set $ \cb{\neg, \land, \lor} $ is adequate.
\end{theorem}

\begin{proof}
Let $ G : \cb{T, F}^n \to \cb{T, F} $.
\begin{enumerate}[leftmargin=0.5in, label=Case \arabic*.]
\item $ G\rb{\overline{v}} = F $ for all $ \overline{v} \in \cb{T, F}^n $. Take $ \phi $ to be
$$ \andb{p_1}{\notb{p_1}}. $$
Then $ G = F_\phi $.
\item List the $ \overline{v} \in \cb{T, F}^n $ with $ G\rb{\overline{v}} = T $ as
$$ \overline{v_1}, \dots, \overline{v_r}. $$
Write
$$ \overline{v_i} = \rb{v_{i1}, \dots, v_{in}}, $$
where each $ v_{ij} $ is $ T $ or $ F $. Define
$$ q_{ij} =
\begin{cases}
p_j & v_{ij} = T \\
\notb{p_j} & v_{ij} = F
\end{cases}.
$$
So
$$ q_{ij} \ \text{has value} \ T \qquad \iff \qquad p_j \ \text{has value} \ v_{ij}. $$
Let
$$ \psi_i : \rb{q_{i1} \land \dots \land q_{in}}. $$
Then
$$ F_{\psi_i}\rb{\overline{v}} = T \qquad \iff \qquad \text{each} \ q_{ij} \ \text{has value} \ T \qquad \iff \qquad \overline{v} = \overline{v_i}. $$
Let
$$ \theta : \rb{\psi_1 \lor \dots \lor \psi_r}. $$
Then
$$ F_\theta\rb{\overline{v}} = T \qquad \iff \qquad F_{\psi_i}\rb{\overline{v}} = T \qquad \iff \qquad \overline{v} = \overline{v_i} $$
for some $ i \le r $. Thus
$$ F_\theta\rb{\overline{v}} = T \qquad \iff \qquad G\rb{\overline{v}} = T, $$
that is $ F_\theta = G $.
\end{enumerate}
As $ \phi $ and $ \theta $ were constructed using only $ \neg, \land, \lor $, \ref{thm:1.1.9} follows.
\end{proof}

A formula $ \theta $ as in case $ 2 $ is said to be in \textbf{disjunctive normal form (DNF)}.

\begin{corollary}
Suppose $ \chi $ is a formula whose truth function is not always $ F $. Then $ \chi $ is logically equivalent to a formula in disjunctive normal form.
\end{corollary}

\begin{proof}
Take $ G = F_\chi $, and apply case $ 2 $ of \ref{thm:1.1.9}.
\end{proof}

\begin{example*}
Let
$$ \chi : \impb{\impb{p_1}{p_2}}{\notb{p_2}}. $$
Then
$$ F_\chi\rb{\overline{v}} = T \qquad \iff \qquad \overline{v} \in \cb{\rb{T, F}, \rb{F, F}}. $$
Thus disjunctive normal form is
$$ \orb{\andb{p_1}{\notb{p_2}}}{\andb{\notb{p_1}}{\notb{p_2}}}. $$
\end{example*}

\pagebreak

\begin{corollary}
The following sets of connectives are adequate.
\begin{enumerate}
\item $ \cb{\neg, \lor} $,
\item $ \cb{\neg, \land} $, and
\item $ \cb{\neg, \rightarrow} $.
\end{enumerate}
\end{corollary}

\begin{proof}
\hfill
\begin{enumerate}
\item By \ref{thm:1.1.9}, enough to show that we can express $ \land $ using $ \neg, \lor $.
$$ \andb{p_1}{p_2} \ \text{is logically equivalent to} \ \notb{\orb{\notb{p_1}}{\notb{p_2}}}. $$
\item By \ref{thm:1.1.9}, enough to show that we can express $ \lor $ using $ \neg, \land $.
$$ \orb{p_1}{p_2} \ \text{is logically equivalent to} \ \notb{\andb{\notb{p_1}}{\notb{p_2}}}. $$
\item By \ref{thm:1.1.9}, enough to show that we can express $ \lor $ using $ \neg, \rightarrow $.
$$ \orb{p_1}{p_2} \ \text{is logically equivalent to} \ \impb{\notb{p}}{q}. $$
\end{enumerate}
\end{proof}

\lecture{3}{Monday}{08/10/18}

\begin{example}
The following are not adequate.
\begin{enumerate}
\item $ \cb{\land, \lor} $, and
\item $ \cb{\neg, \leftrightarrow} $.
\end{enumerate}
\end{example}

\begin{proof}
\hfill
\begin{enumerate}
\item If $ \phi $ is built using $ \lor, \land $, then $ F_\phi\rb{T, \dots, T} = T $. Proof by induction on number of connectives.
\item Exercise. Puzzle.
\end{enumerate}
\end{proof}

\begin{example}
The \textbf{NOR} connective $ \downarrow $ has the following truth table.
$$
\begin{array}{cc|c}
p & q & \rb{p \downarrow q} \\
\hline
T & T & F \\
T & F & F \\
F & T & F \\
F & F & T
\end{array}
$$
$ \rb{p \downarrow q} $ is logically equivalent to $ \andb{\notb{p}}{\notb{q}} $. $ \cb{\downarrow} $ is adequate.
\begin{itemize}
\item $ \rb{p \downarrow p} $ is logically equivalent to $ \notb{p} $, and
\item $ \rb{\rb{p \downarrow p} \downarrow \rb{q \downarrow q}} $ is logically equivalent to $ \andb{p}{q} $.
\end{itemize}
So, as $ \cb{\neg, \land} $ is adequate, so is $ \cb{\downarrow} $.
\end{example}

\pagebreak

\subsection{A formal system for propositional logic}

Idea is to try to generate all tautologies from basic assumptions, axioms, using appropriate deduction rules.

\begin{definition}
Very general definition.
\begin{itemize}
\item A \textbf{formal deduction system} $ \Sigma $ has the following ingredients.
\begin{itemize}
\item An \textbf{alphabet} $ A \ne \emptyset $ of symbols.
\item A non-empty subset $ \FF $ of the set of all finite sequences, \textbf{strings}, of elements of $ A $, the \textbf{formulas} of $ \Sigma $.
\item A subset $ \AA \subseteq \FF $ called the \textbf{axioms} of $ \Sigma $.
\item A collection of \textbf{deduction rules}.
\end{itemize}
\item A \textbf{proof} in $ \Sigma $ is a finite sequence of formulas in $ \FF $, $ \phi_1, \dots, \phi_n $ such that each $ \phi_i $ is either an axiom, that is is in $ \AA $, or is obtained from $ \phi_1, \dots, \phi_{i - 1} $ using one of the deduction rules.
\item The last, or any, formula in a proof is a \textbf{theorem} of $ \Sigma $. Write $ \vdash_\Sigma \phi $ for $ \phi $ is a theorem of $ \Sigma $.
\end{itemize}
\end{definition}

\begin{remark*}
\hfill
\begin{itemize}
\item If $ \phi \in \AA $, then $ \vdash_\Sigma \phi $.
\item Should have an algorithm to test whether a string is a formula and whether it is an axiom. Then a computer can systematically generate all possible proofs in $ \Sigma $, and check whether something is a proof. Say $ \Sigma $ is \textbf{recursive} in this case.
\end{itemize}
\end{remark*}

The following is the main example.

\begin{definition}
The formal system $ L $ for propositional logic has the following.
\begin{itemize}
\item Alphabet.
$$
\begin{array}{ll}
\text{Variables} & p_1 \quad p_2 \quad \dots \\
\text{Connectives} & \neg \quad \rightarrow \\
\textbf{Punctuation} & ( \quad )
\end{array}
$$
\item Formulas. $ L $-formulas defined in \ref{def:1.1.2} for $ \neg, \rightarrow $.
\begin{itemize}
\item Any variable $ p_i $ is a formula.
\item If $ \phi, \psi $ are formulas, then so are
$$ \notb{\phi}, \qquad \impb{\phi}{\psi}. $$
\item Any formula arises in this way.
\end{itemize}
\item Axioms. Suppose $ \phi, \psi, \chi $ are $ L $-formulas. The following are axioms of $ L $.
\begin{enumerate}[label=(A\arabic*)]
\item $ \impb{\phi}{\impb{\psi}{\phi}} $,
\item $ \impb{\impb{\phi}{\impb{\psi}{\chi}}}{\impb{\impb{\phi}{\psi}}{\impb{\phi}{\chi}}} $, and
\item $ \impb{\impb{\notb{\psi}}{\notb{\phi}}}{\impb{\phi}{\psi}} $.
\end{enumerate}
\item Deduction rule.
\begin{itemize}
\item[(MP)] \textbf{Modus Ponens}. From $ \phi, \impb{\phi}{\psi} $, deduce $ \psi $.
\end{itemize}
\end{itemize}
\end{definition}

\begin{example}
\label{eg:1.2.3}
Theorem $ 0 $. Suppose $ \phi $ is an $ L $-formula. Then $ \vdash_L \impb{\phi}{\phi} $. Here is a proof in $ L $.
\begin{align*}
1 \qquad & \impb{\phi}{\impb{\impb{\phi}{\phi}}{\phi}} & \rb{\text{A1}} \\
2 \qquad & \impb{\impb{\phi}{\impb{\impb{\phi}{\phi}}{\phi}}}{\impb{\impb{\phi}{\impb{\phi}{\phi}}}{\impb{\phi}{\phi}}} & \rb{\text{A2}} \\
3 \qquad & \impb{\impb{\phi}{\impb{\phi}{\phi}}}{\impb{\phi}{\phi}} & \rb{1, 2, \text{MP}} \\
4 \qquad & \impb{\phi}{\impb{\phi}{\phi}} & \rb{\text{A1}} \\
5 \qquad & \impb{\phi}{\phi} & \rb{3, 4, \text{MP}}
\end{align*}
\end{example}

\pagebreak

\lecture{4}{Thursday}{11/10/18}

\begin{definition}
Suppose $ \Gamma $ is a set of $ L $-formulas.
\begin{itemize}
\item A \textbf{deduction from $ \Gamma $} is a finite sequence of $ L $-formulas $ \phi_1, \dots, \phi_n $ such that each $ \phi_i $ is either an axiom, a formula in $ \Gamma $, or is obtained from previous formulas $ \phi_1, \dots, \phi_{i - 1} $ using the deduction rule MP.
\item Write $ \Gamma \vdash_L \phi $ if there is a deduction from $ \Gamma $ ending in $ \phi $. Say $ \phi $ is a \textbf{consequence} of $ \Gamma $. So $ \emptyset \vdash_L \phi $ is the same as $ \vdash_L \phi $.
\end{itemize}
\end{definition}

\begin{theorem}[Deduction theorem]
\label{thm:1.2.5}
Suppose $ \Gamma $ is a set of $ L $-formulas and $ \phi, \psi $ are $ L $-formulas. Suppose $ \Gamma \cup \cb{\phi} \vdash_L \psi $. Then $ \Gamma \vdash_L \impb{\phi}{\psi} $.
\end{theorem}

\begin{corollary}[Hypothetical syllogism]
Suppose $ \phi, \psi, \chi $ are $ L $-formulas and $ \vdash_L \impb{\phi}{\psi} $ and $ \vdash_L \impb{\psi}{\chi} $. Then $ \vdash_L \impb{\phi}{\chi} $.
\end{corollary}

\begin{proof}
Use deduction theorem with $ \Gamma = \emptyset $. Show that $ \cb{\phi} \vdash_L \chi $. Here is a deduction of $ \chi $ from $ \phi $.
\begin{align*}
1 \qquad & \impb{\phi}{\psi} & \rb{\text{theorem of} \ L} \\
2 \qquad & \impb{\psi}{\chi} & \rb{\text{theorem of} \ L} \\
3 \qquad & \phi & \rb{\text{assumption}} \\
4 \qquad & \psi & \rb{1, 3, \text{MP}} \\
5 \qquad & \chi & \rb{2, 4, \text{MP}}
\end{align*}
Thus $ \cb{\phi} \vdash_L \chi $. By deduction theorem, $ \emptyset \vdash_L \impb{\phi}{\chi} $, that is $ \vdash_L \impb{\phi}{\chi} $.
\end{proof}

\begin{proposition}
\label{prop:1.2.7}
Suppose $ \phi, \psi $ are $ L $-formulas. Then
\begin{enumerate}
\item $ \vdash_L \impb{\notb{\psi}}{\impb{\psi}{\phi}} $,
\item $ \cb{\notb{\psi}, \psi} \vdash_L \phi $, and
\item $ \vdash_L \impb{\impb{\notb{\phi}}{\phi}}{\phi} $.
\end{enumerate}
\end{proposition}

\begin{proof}
\hfill
\begin{enumerate}
\item Problem sheet $ 1 $.
\item By $ 1 $ and MP twice.
\item Suppose $ \chi $ is any formula. Then
$$ \cb{\notb{\phi}, \impb{\notb{\phi}}{\phi}} \vdash_L \chi, $$
by MP and $ 2 $. Let $ \alpha $ be any axiom and let $ \chi $ be $ \notb{\alpha} $. Apply deduction theorem to get
$$ \cb{\impb{\notb{\phi}}{\phi}} \vdash_L \impb{\notb{\phi}}{\notb{\alpha}}. $$
A3 is $ \cb{\impb{\notb{\phi}}{\notb{\alpha}}} \vdash_L \impb{\alpha}{\phi} $. Using this and MP, we get
$$ \cb{\impb{\notb{\phi}}{\phi}} \vdash_L \impb{\alpha}{\phi}. $$
As $ \alpha $ is an axiom, we get, from MP,
$$ \cb{\impb{\notb{\phi}}{\phi}} \vdash_L \phi. $$
Now use deduction theorem to obtain
$$ \vdash_L \impb{\impb{\notb{\phi}}{\phi}}{\phi}. $$
\end{enumerate}
\end{proof}

\pagebreak

\begin{proof}[Proof of \ref{thm:1.2.5}]
Suppose $ \Gamma \cup \cb{\phi} \vdash_L \psi $ using a deduction of length $ n $. Show, by induction on $ n $, that $ \Gamma \vdash_L \impb{\phi}{\psi} $.
\begin{itemize}
\item Base case. $ n = 1 $. In this case, $ \psi $ is either an axiom, or in $ \Gamma $, or is $ \phi $. In the first two cases, $ \Gamma \vdash_L \psi $, a one line deduction. Using the A1 axiom $ \impb{\psi}{\impb{\phi}{\psi}} $ and MP, we obtain $ \Gamma \vdash_L \impb{\phi}{\psi} $. If $ \phi $ is $ \psi $, we have $ \Gamma \vdash_L \impb{\phi}{\phi} $, by \ref{eg:1.2.3}. This finishes the base case.
\item Inductive step. In our deduction of $ \psi $ from $ \Gamma \cup \cb{\phi} $, either
\begin{enumerate}[leftmargin=0.5in, label=Case \arabic*.]
\item $ \psi $ is an axiom, or in $ \Gamma $, or is $ \phi $,
\item or $ \psi $ is obtained from earlier steps using MP.
\end{enumerate}
In case $ 1 $, we argue as in the base case to get $ \Gamma \vdash_L \impb{\phi}{\psi} $. In case $ 2 $, there are formulas $ \chi, \impb{\chi}{\psi} $ earlier in the deduction. We use the inductive hypothesis to get
\begin{equation}
\label{eq:1}
\Gamma \vdash_L \impb{\phi}{\chi}, \qquad \Gamma \vdash_L \impb{\phi}{\impb{\chi}{\psi}}.
\end{equation}
We have the A2 axiom $ \impb{\impb{\phi}{\impb{\psi}{\chi}}}{\impb{\impb{\phi}{\psi}}{\impb{\phi}{\chi}}} $. Using $ \rb{\ref{eq:1}} $, the A2 axiom, and MP twice, we obtain $ \Gamma \vdash_L \impb{\phi}{\chi} $, as required, completing the inductive step.
\end{itemize}
\end{proof}

\lecture{5}{Friday}{12/10/18}

\subsection{Soundness and completeness of $ L $}

\begin{theorem}[Soundness of $ L $]
\label{thm:1.3.1}
Suppose $ \phi $ is a theorem of $ L $. Then $ \phi $ is a tautology.
\end{theorem}

\begin{definition}
A \textbf{propositional valuation} $ v $ is an assignment of truth values to the propositional variables $ p_1, p_2, \dots $. So
$$ v\rb{p_i} \in \cb{T, F}, \qquad i \in \N. $$
\end{definition}

\begin{note*}
Using the truth table rules, this assigns a truth value $ v\rb{\phi} \in \cb{T, F} $ to every $ L $-formula $ \phi $, satisfying $ v\rb{\notb{\phi}} \ne v\rb{\phi} $, etc. See problem sheet $ 2 $.
\end{note*}

\begin{proof}[Proof of \ref{thm:1.3.1}]
By induction on the length of a proof of $ \phi $, it is enough to show
\begin{enumerate}
\item every axiom is a tautology, and
\item MP preserves tautologies, that is if $ \psi, \impb{\psi}{\chi} $ are tautologies, so is $ \chi $.
\end{enumerate}
\begin{enumerate}
\item Use truth tables, or argue as follows. For A2, suppose for a contradiction there is a valuation $ v $ with
$$ v\rb{\impb{\impb{\phi}{\impb{\psi}{\chi}}}{\impb{\impb{\phi}{\psi}}{\impb{\phi}{\chi}}}} = F. $$
Then
\begin{equation}
\label{eq:2}
v\rb{\impb{\phi}{\impb{\psi}{\chi}}} = T,
\end{equation}
and
\begin{equation}
\label{eq:3}
v\rb{\impb{\impb{\phi}{\psi}}{\impb{\phi}{\chi}}} = F.
\end{equation}
By $ \rb{\ref{eq:3}} $,
\begin{equation}
\label{eq:4}
v\rb{\impb{\phi}{\psi}} = T,
\end{equation}
and
\begin{equation}
\label{eq:5}
v\rb{\impb{\phi}{\chi}} = F.
\end{equation}
So by $ \rb{\ref{eq:5}} $, $ v\rb{\phi} = T $ and $ v\rb{\chi} = F $. By $ \rb{\ref{eq:4}} $, $ v\rb{\psi} = T $. This contradicts $ \rb{\ref{eq:2}} $. (Exercise: A1 and A3)
\item If $ v $ is a valuation, and $ v\rb{\psi} = T $ and $ v\rb{\impb{\psi}{\chi}} = T $, then $ v\rb{\chi} = T $.
\end{enumerate}
\end{proof}

\begin{theorem}[Generalisation of soundness]
Suppose $ \Gamma $ is a set of formulas and $ \phi $ a formula with $ \Gamma \vdash_L \phi $. Suppose $ v $ is a valuation with $ v\rb{\psi} = T $ for all $ \psi \in \Gamma $. Then $ v\rb{\phi} = T $.
\end{theorem}

\begin{proof}
Exercise. Same proof.
\end{proof}

\pagebreak

\begin{theorem}[Completeness (adequacy) of $ L $]
\label{thm:1.3.4}
Suppose $ \phi $ is a tautology, that is $ v\rb{\phi} = T $ for every valuation $ v $. Then $ \vdash_L \phi $.
\end{theorem}

\begin{remark}
Steps in proof.
\begin{itemize}
\item If $ v\rb{\phi} = T $ for all valuations $ v $, want to show $ \vdash_L \phi $.
\item Try to prove a generalisation. Suppose that, for every $ v $ with $ v\rb{\Gamma} = T $, that is $ v\rb{\psi} = T $ for all $ \psi \in \Gamma $, we have $ v\rb{\phi} = T $. Then $ \Gamma \vdash_L \phi $.
\item Equivalently, if $ \Gamma \not\vdash_L \phi $, show there is a valuation $ v $ with $ v\rb{\Gamma} = T $ and $ v\rb{\phi} = F $.
\end{itemize}
\end{remark}

\begin{definition}
A set $ \Gamma $ of $ L $-formulas is \textbf{consistent} if there is no $ L $-formula $ \phi $ such that
$$ \Gamma \vdash_L \phi, \qquad \Gamma \vdash_L \notb{\phi}. $$
\end{definition}

\begin{proposition}
\label{prop:1.3.7}
Suppose $ \Gamma $ is a consistent set of $ L $-formulas and $ \Gamma \not\vdash_L \phi $. Then $ \Gamma \cup \cb{\notb{\phi}} $ is consistent.
\end{proposition}

\begin{proof}
Suppose not. So there is some formula $ \psi $ with
\begin{equation}
\label{eq:6}
\Gamma \cup \cb{\notb{\phi}} \vdash_L \psi,
\end{equation}
and
\begin{equation}
\label{eq:7}
\Gamma \cup \cb{\notb{\phi}} \vdash_L \notb{\psi}.
\end{equation}
Apply deduction theorem to $ \rb{\ref{eq:7}} $,
$$ \Gamma \vdash_L \impb{\notb{\phi}}{\notb{\psi}}. $$
By A3 and MP, we obtain
\begin{equation}
\label{eq:8}
\Gamma \vdash_L \impb{\psi}{\phi}.
\end{equation}
By $ \rb{\ref{eq:6}} $, $ \rb{\ref{eq:8}} $, and MP,
$$ \Gamma \cup \cb{\notb{\phi}} \vdash_L \phi. $$
By deduction theorem,
\begin{equation}
\label{eq:9}
\Gamma \vdash_L \impb{\notb{\phi}}{\phi}.
\end{equation}
By \ref{prop:1.2.7}.$ 3 $,
$$ \vdash_L \impb{\impb{\notb{\phi}}{\phi}}{\phi}. $$
So by this, $ \rb{\ref{eq:9}} $, and MP, $ \Gamma \vdash_L \phi $. This contradicts $ \Gamma \not\vdash_L \phi $.
\end{proof}

\begin{proposition}[Lindenbaum lemma]
\label{prop:1.3.8}
Suppose $ \Gamma $ is a consistent set of $ L $-formulas. Then there is a consistent set of formulas $ \Gamma^* \supseteq \Gamma $ such that, for every $ \phi $, either $ \Gamma^* \vdash_L \phi $ or $ \Gamma^* \vdash_L \notb{\phi} $.
\end{proposition}

Sometimes say $ \Gamma^* $ is \textbf{complete}.

\begin{proof}
The set of $ L $-formulas is countable, so we can list the $ L $-formulas as $ \phi_0, \phi_1, \dots $. This is countable, since the alphabet
$$ \neg \quad \rightarrow \qquad ) \quad ( \qquad p_1 \quad p_2 \quad \dots $$
is countable, and the formulas are finite sequences from this alphabet, so only countable many. Define inductively sets of formulas
$$ \Gamma_0 \subseteq \Gamma_1 \subseteq \dots, \qquad \Gamma_0 = \Gamma, \qquad \Gamma_{n + 1} =
\begin{cases}
\Gamma_n & \Gamma_n \vdash_L \phi_n \\
\Gamma_n \cup \cb{\notb{\phi_n}} & \Gamma_n \not\vdash_L \phi_n
\end{cases}.
$$
An easy induction using \ref{prop:1.3.7} shows that each $ \Gamma_i $ is consistent. Then let $ \Gamma^* = \bigcup_{i \in \N} \Gamma_i $. Claim that $ \Gamma^* $ is consistent. If $ \Gamma^* \vdash_L \phi $ and $ \Gamma^* \vdash_L \notb{\phi} $, then as deductions are finite sequence of formulas,
$$ \Gamma_n \vdash_L \phi, \qquad \Gamma_n \vdash_L \notb{\phi} $$
for some $ n \in \N $, a contradiction. Let $ \phi $ be any formula. So $ \phi = \phi_n $ for some $ n $. If $ \Gamma^* \not\vdash_L \phi $, then $ \Gamma_n \not\vdash_L \phi $. So by construction,
$$ \Gamma_{n + 1} \vdash_L \notb{\phi}, $$
as $ \notb{\phi} = \notb{\phi_n} \in \Gamma_{n + 1} $. Thus $ \Gamma^* \vdash_L \notb{\phi} $.
\end{proof}

\pagebreak

\lecture{6}{Monday}{15/10/18}

\begin{lemma}
\label{lem:1.3.9}
Let $ \Gamma^* $ be as above. Then there is a valuation $ v $ such that, for every $ L $-formula $ \phi $,
$$ v\rb{\phi} = T \qquad \iff \qquad \Gamma^* \vdash_L \phi. $$
\end{lemma}

The proof is later.

\begin{corollary}
\label{cor:1.3.10}
Suppose $ \Delta $ is a set of $ L $-formulas which is consistent and $ \Delta \not\vdash_L \phi $. Then there is a valuation $ v $ with $ v\rb{\Delta} = T $ and $ v\rb{\phi} = F $.
\end{corollary}

\begin{proof}
Let
$$ \Gamma = \Delta \cup \cb{\notb{\phi}}. $$
By \ref{prop:1.3.7}, $ \Gamma $ is consistent. By \ref{prop:1.3.8}, there is $ \Gamma^* \supseteq \Gamma $ which is still consistent, and such that, for every $ \chi $, either $ \Gamma^* \vdash_L \chi $ or $ \Gamma^* \vdash_L \notb{\chi} $. By \ref{lem:1.3.9}, there is a valuation $ v $ with
$$ v\rb{\Gamma^*} = T. $$
In particular, $ v\rb{\Delta} = T $ and $ v\rb{\notb{\phi}} = T $. So $ v\rb{\phi} = F $.
\end{proof}

\begin{proof}[Proof of \ref{thm:1.3.4}]
Suppose $ \not\vdash_L \phi $. Apply \ref{cor:1.3.10} with $ \Delta = \emptyset $. This is consistent, by the soundness theorem. There is a valuation $ v $ with $ v\rb{\phi} = F $.
\end{proof}

\begin{proof}[Proof of \ref{lem:1.3.9}]
Let $ \Gamma^* $ be a consistent set of $ L $-formulas such that, for every $ L $-formula $ \phi $, either $ \Gamma^* \vdash_L \phi $ or $ \Gamma^* \vdash_L \notb{\phi} $. Want a valuation $ v $ with $ v\rb{\phi} = T $ for all $ \phi \in \Gamma^* $, that is
$$ v\rb{\phi} = T \qquad \iff \qquad \Gamma^* \vdash_L \phi. $$
Note that, for each variable $ p_i $, either $ \Gamma^* \vdash_L p_i $ or $ \Gamma^* \vdash_L \notb{p_i} $. So let $ v $ be the valuation with
$$ v\rb{p_i} = T \qquad \iff \qquad \Gamma^* \vdash_L p_i. $$
Prove, by induction on the length of $ \phi $, that
$$ v\rb{\phi} = T \qquad \iff \qquad \Gamma^* \vdash_L \phi. $$
\begin{itemize}
\item Base case. $ \phi $ is just a propositional variable. This case is by definition of $ v $.
\item Inductive step.
\begin{enumerate}[leftmargin=0.5in, label=Case \arabic*.]
\item $ \phi $ is $ \notb{\psi} $.
\begin{itemize}
\item[$ \implies $] As $ v $ is a valuation,
\begin{align*}
v\rb{\phi} = T \qquad
& \implies \qquad v\rb{\psi} = F & \text{as} \ v \ \text{is a valuation} \\
& \implies \qquad \Gamma^* \not\vdash_L \psi & \text{by inductive hypothesis} \\
& \implies \qquad \Gamma^* \vdash_L \notb{\psi} & \text{by Lindenbaum property} \\
& \implies \qquad \Gamma^* \vdash_L \phi.
\end{align*}
\item[$ \impliedby $] Conversely suppose $ \Gamma^* \vdash_L \phi $.
\begin{align*}
\Gamma^* \vdash_L \phi \qquad
& \implies \qquad \Gamma^* \not\vdash_L \psi & \text{by consistency} \\
& \implies \qquad v\rb{\psi} = F & \text{by inductive hypothesis} \\
& \implies \qquad v\rb{\notb{\psi}} = T & \text{as} \ v \ \text{is a valuation} \\
& \implies \qquad v\rb{\phi} = T.
\end{align*}
\pagebreak
\end{itemize}
\item $ \phi $ is $ \impb{\psi}{\chi} $.
\begin{itemize}
\item[$ \impliedby $] Suppose $ v\rb{\phi} = F $. Show that $ \Gamma^* \not\vdash_L \phi $. Then
$$ v\rb{\psi} = T, \qquad v\rb{\chi} = F. $$
By inductive hypothesis,
$$ \Gamma^* \vdash_L \psi, \qquad \Gamma^* \not\vdash_L \chi. $$
If $ \Gamma^* \vdash_L \phi $, then using $ \Gamma^* \vdash_L \psi $ and MP, we get $ \Gamma^* \vdash_L \chi $, a contradiction. So $ \Gamma^* \not\vdash_L \phi $.
\item[$ \implies $] Suppose $ \Gamma^* \not\vdash_L \phi $, that is $ \Gamma^* \not\vdash_L \impb{\psi}{\chi} $. Then
\begin{equation}
\label{eq:10}
\Gamma^* \not\vdash_L \chi,
\end{equation}
as $ \vdash_L \impb{\chi}{\impb{\psi}{\chi}} $. Also
\begin{equation}
\label{eq:11}
\Gamma^* \not\vdash_L \notb{\psi},
\end{equation}
as $ \vdash_L \impb{\notb{\psi}}{\impb{\psi}{\chi}} $ by \ref{prop:1.2.7}.$ 1 $. By $ \rb{\ref{eq:10}} $, $ \rb{\ref{eq:11}} $, and inductive hypothesis,
$$ v\rb{\chi} = F, \qquad v\rb{\notb{\psi}} = F, $$
so $ v\rb{\psi} = T $. Thus $ v\rb{\phi} = F $, which does the inductive step.
\end{itemize}
\end{enumerate}
\end{itemize}
\end{proof}

\begin{corollary}
\label{cor:1.3.11}
Suppose $ \Delta $ is a set of $ L $-formulas and $ \phi $ is an $ L $-formula. Then
\begin{enumerate}
\item $ \Delta $ is consistent if and only if there is a valuation $ v $ with $ v\rb{\Delta} = T $, and
\item $ \Delta \vdash_L \phi $ if and only if, for every valuation $ v $ with $ v\rb{\Delta} = T $, we have $ v\rb{\phi} = T $.
\end{enumerate}
\end{corollary}

\begin{proof}
Exercise. Deduce these from the preliminaries to completeness theorem. Warning that, in $ 2 $, do not assume that $ \Delta $ is consistent.
\end{proof}

\begin{theorem}[Compactness theorem for $ L $]
Suppose $ \Delta $ is a set of $ L $-formulas. The following are equivalent.
\begin{enumerate}
\item There is a valuation $ v $ with $ v\rb{\Delta} = T $.
\item For every finite subset $ \Delta_0 \subseteq \Delta $, there is a valuation $ w $ with $ w\rb{\Delta_0} = T $.
\end{enumerate}
\end{theorem}

\begin{proof}
By \ref{cor:1.3.11},
$$ 1 \ \text{holds} \qquad \iff \qquad \Delta \ \text{is consistent}. $$
Similarly,
$$ 2 \ \text{holds} \qquad \iff \qquad \text{every finite subset of} \ \Delta \ \text{is consistent}. $$
But if $ \Delta \vdash_L \psi $ and $ \Delta \vdash_L \notb{\psi} $, then, as deductions are finite, and therefore only involve finitely many formulas in $ \Delta $, for some finite $ \Delta_0 \subseteq \Delta $,
$$ \Delta_0 \vdash_L \psi, \qquad \Delta_0 \vdash_L \notb{\psi}. $$
\end{proof}

\begin{example*}
Exercise. Let
$$ P = \cb{\text{sequences of} \ \cb{T, F}} = \cb{\text{functions} \ f : \N \to \cb{T, F}}. $$
Topologise with basis open sets. For $ a_1, \dots, a_n \in \cb{T, F} $, consider
$$ O\rb{a_1, \dots, a_n} = \cb{\text{all sequences starting} \ a_1, \dots, a_n, \dots}. $$
Use compactness theorem to prove $ P $ is compact.
\end{example*}

\lecture{7}{Thursday}{18/10/18}

Lecture 7 is a problem class.

\pagebreak

\section{Predicate logic}

\lecture{8}{Friday}{19/10/18}

Plan.
\begin{itemize}
\item Semantics.
\begin{itemize}
\item Introduce the mathematical objects, the first-order structures.
\item Introduce the formulas, the first-order languages.
\end{itemize}
\item Syntax.
\begin{itemize}
\item Describe a formal system.
\item Show that its theorems are precisely the formulas true in all structures. This is G\"odel's completeness theorem.
\end{itemize}
\end{itemize}

\subsection{Structures}

\begin{definition}
Suppose $ A $ is a set and $ n \in \N_{\ge 1} $.
\begin{itemize}
\item An \textbf{$ n $-ary relation} on $ A $ is a subset
$$ \overline{R} \subseteq A^n = \cb{\rb{a_1, \dots, a_n} \mid a_i \in A}, $$
\item An \textbf{$ n $-ary function} on $ A $ is a function
$$ \overline{f} : A^n \to A. $$
\end{itemize}
\end{definition}

\begin{example*}
\hfill
\begin{itemize}
\item Ordering $ \le $ on $ \R $ is a binary relation on $ \R $.
\item $ + $ on $ \C $ is a binary function on $ \C $.
\item Even integers as a subset of $ \Z $ is a unary relation on $ \Z $.
\end{itemize}
\end{example*}

\begin{notation*}
If $ \overline{R} \subseteq A^n $ is an $ n $-ary relation and $ a_1, \dots, a_n \in A $, write $ \overline{R}\rb{a_1, \dots, a_n} $ to mean $ \rb{a_1, \dots, a_n} \in \overline{R} $.
\end{notation*}

\begin{definition}
A \textbf{first-order structure} $ \AA $ consists of
\begin{itemize}
\item a non-empty set $ A $, the \textbf{domain} of $ \AA $,
\item a set of \textbf{relations} on $ A $,
$$ \cb{\overline{R_i} \subseteq A^{n_i} \mid i \in I} $$
\item a set of \textbf{functions} on $ A $, and
$$ \cb{\overline{f_j} : A^{m_j} \to A \mid j \in J} $$
\item a set of \textbf{constants}, just elements of $ A $.
$$ \cb{\overline{c_k} \mid k \in K} $$
\end{itemize}
The sets $ I, J, K $ are indexing sets, which can be empty. Usually subsets of $ \N $. The information
$$ \rb{n_i \mid i \in I}, \qquad \rb{m_j \mid j \in J}, \qquad K $$
is called the \textbf{signature} of $ \AA $. Might denote the structure by
\begin{align*}
\AA
& = \ab{A; \rb{\overline{R_i} \mid i \in I}, \rb{\overline{f_j} \mid j \in J}, \rb{\overline{c_k} \mid k \in K}} \\
& = \ab{\text{domain}; \ \text{relations}, \ \text{functions}, \ \text{constants}}.
\end{align*}
\end{definition}

\pagebreak

\begin{example}
\label{eg:2.1.3}
\hfill
\begin{itemize}
\item Orderings.
$$ A \in \cb{\N, \Z, \Q, \R}, \qquad I = \cb{1}, \qquad J = \emptyset, \qquad K = \emptyset $$
$$ \overline{R_1}\rb{a_1, a_2} \qquad \text{to mean} \qquad a_1 < a_2 $$
\item Groups.
$$
\begin{array}{ll}
\overline{R} & \text{the binary relation for equality} \\
\overline{m} & \text{the binary function for multiplication} \\
\overline{i} & \text{the unary function for inversion} \\
\overline{e} & \text{the constant for identity element}
\end{array}
$$
\item Rings.
$$
\begin{array}{ll}
\overline{R} & \text{the binary relation for equality} \\
\overline{m} & \text{the binary function for multiplication} \\
\overline{a} & \text{the binary function for addition} \\
\overline{n} & \text{the binary function for negation} \\
\overline{0} & \text{the constant for zero} \\
\overline{1} & \text{the constant for one}
\end{array}
$$
\item Graphs.
$$
\begin{array}{ll}
\overline{R} & \text{the binary relation for equality} \\
\overline{E} & \text{the binary relation for adjacency}
\end{array}
$$
\end{itemize}
\end{example}

\subsection{First-order languages}

\begin{definition}
\label{def:2.2.1}
A \textbf{first-order language} $ \LL $ has an alphabet of \textbf{symbols} of the following types.
$$
\begin{array}{ll}
\text{Variables} & x_0 \quad x_1 \quad \dots \\
\text{Connectives} & \neg \quad \rightarrow \\
\text{Punctuation} & ( \quad ) \quad , \\
\textbf{Quantifier} & \forall \\
\textbf{Relation symbols} & R_i, \quad i \in I \\
\textbf{Function symbols} & f_j, \quad j \in J \\
\textbf{Constant symbols} & c_k, \quad k \in K
\end{array}
$$
Here $ I, J, K $ are indexing sets, which could have $ J, K = \emptyset $.
\begin{itemize}
\item Each $ R_i $ comes equipped with an arity $ n_i $.
\item Each $ f_j $ comes equipped with an arity $ m_j $.
\end{itemize}
The information
$$ \rb{n_i \mid i \in I}, \qquad \rb{m_j \mid j \in J}, \qquad K $$
is called the signature of $ \LL $.
\end{definition}

\pagebreak

\begin{definition}
A \textbf{term} of $ \LL $ is defined as follows.
\begin{itemize}
\item Any variable is a term.
\item Any constant symbol is a term.
\item If $ f $ is an $ m $-ary function symbol of $ \LL $ and $ t_1, \dots, t_m $ are terms, then
$$ f\rb{t_1, \dots, t_m} $$
is also a term.
\item Any term arises in this way.
\end{itemize}
\end{definition}

\begin{example*}
Suppose $ \LL $ has a binary function symbol $ f $ and constant symbols $ c_1, c_2 $.
\begin{itemize}
\item Some terms.
$$ c_1 \qquad c_2 \qquad x_1 \qquad f\rb{c_1, x_1} \qquad f\rb{f\rb{c_1, x_2}, c_2} \qquad f\rb{x_1, f\rb{f\rb{c_1, x_2}, c_2}} $$
\item Not terms.
$$ ffx_1 \quad \rb{\text{not well-formed}} $$
\end{itemize}
\end{example*}

\lecture{9}{Monday}{22/10/18}

\begin{definition}
Use the previous notation.
\begin{itemize}
\item An \textbf{atomic formula} of $ \LL $ is of the form
$$ R\rb{t_1, \dots, t_n}, $$
where $ R $ is an $ n $-ary relation symbol of $ \LL $ and $ t_1, \dots, t_n $ are terms.
\item The formulas of $ \LL $ are defined as follows.
\begin{itemize}
\item Any atomic formula is a formula.
\item If $ \phi, \psi $ are $ \LL $-formulas, then
$$ \notb{\phi}, \qquad \impb{\phi}{\psi}, \qquad \fab{x}\phi $$
are $ \LL $-formulas, where $ x $ is any variable.
\item Every $ \LL $-formula arises in this way.
\end{itemize}
\end{itemize}
\end{definition}

\begin{example*}
Suppose $ \LL $ has a binary function symbol $ f $, a unary relation symbol $ P $, a binary relation symbol $ R $, and constant symbols $ c_1, c_2 $.
\begin{itemize}
\item Some terms.
$$ x_1 \qquad c_1 \qquad f\rb{x_1, c_1} \qquad f\rb{f\rb{x_1, c_1}, x_2} $$
\item Some atomic formulas.
$$ P\rb{x_1} \qquad R\rb{f\rb{x_1, c_1}, x_2} $$
\item Some formulas.
$$ \fab{x_1}\impb{R\rb{f\rb{x_1, c_1}, x_2}}{P\rb{x_1}} $$
\end{itemize}
\end{example*}

\begin{definition}
Suppose $ \phi, \psi $ are $ \LL $-formulas.
\begin{itemize}
\item $ \teb{x}\phi $ means $ \notb{\fab{x}\notb{\phi}} $.
\item $ \orb{\phi}{\psi} $ means $ \impb{\notb{\phi}}{\psi} $, etc, as in propositional logic.
\end{itemize}
\end{definition}

\pagebreak

\begin{definition}
Suppose $ \LL $ is a first-order language with relation symbols $ R_i $, of arity $ n_i $, for $ i \in I $, function symbols $ f_j $, of arity $ m_j $, for $ j \in J $, and constant symbols $ c_k $ for $ k \in K $. An \textbf{$ \LL $-structure} is a structure
$$ \AA = \ab{A; \rb{\overline{R_i} \mid i \in I}, \rb{\overline{f_j} \mid j \in J}, \rb{\overline{c_k} \mid k \in K}} $$
of the same signature as $ \LL $.
\end{definition}

There is a correspondence between the relation, function, and constant symbols of $ \LL $ and the actual relations, functions, and constants in $ \AA $, and the arities match up. This correspondence
$$ R_i \leftrightsquigarrow \overline{R_i}, \qquad f_j \leftrightsquigarrow \overline{f_j}, \qquad c_k \leftrightsquigarrow \overline{c_k}, $$
or $ \AA $, is called an \textbf{interpretation} of $ \LL $.

\begin{definition}
With the same notation, suppose $ \AA $ is an $ \LL $-structure. A \textbf{valuation} in $ \AA $ is a function $ v $ from the set of terms of $ \LL $ to $ A $ satisfying
\begin{itemize}
\item $ v\rb{c_k} = \overline{c_k} $, and
\item if $ t_1, \dots, t_m $ are terms of $ \LL $ and $ f $ is an $ m $-ary function symbol, then
$$ v\rb{f\rb{t_1, \dots, t_m}} = \overline{f}\rb{v\rb{t_1}, \dots, v\rb{t_m}}, $$
where $ \overline{f} $ is an interpretation of $ f $ in $ \AA $.
\end{itemize}
\end{definition}

\begin{lemma}
Suppose $ \AA $ is an $ \LL $-structure and $ a_0, a_1, \dots \in A $. Then there is a unique valuation $ v $, in $ \AA $, with $ v\rb{x_l} = a_l $, for all $ l \in \N $, where the variables of $ \LL $ are $ x_0, x_1, \dots $.
\end{lemma}

\begin{proof}
By induction on the length of terms. Show that, if we let
\begin{itemize}
\item $ v\rb{x_l} = a_l $ for all $ l \in \N $,
\item $ v\rb{c_k} = \overline{c_k} $ for all $ k \in K $, and
\item $ v\rb{f\rb{t_1, \dots, t_m}} = \overline{f}\rb{v\rb{t_1}, \dots, v\rb{t_m}} $,
\end{itemize}
then $ v $ is a well-defined valuation.
\end{proof}

\begin{example*}
Groups. Signature, as in \ref{eg:2.1.3}, has
\begin{itemize}
\item a binary relation symbol $ R $, for equality,
\item a binary function symbol $ m $, for multiplication,
\item a unary function symbol $ i $, for inversion, and
\item a constant symbol $ e $, for identity element.
\end{itemize}
Let $ G $ be a group and $ g, h \in G $. Let $ v $ be a valuation with $ v\rb{x_0} = g $ and $ v\rb{x_1} = h $. Then
\begin{align*}
v\rb{m\rb{m\rb{x_0, x_1}, i\rb{x_0}}}
& = \overline{m}\rb{v\rb{m\rb{x_0, x_1}}, v\rb{i\rb{x_0}}} \\
& = \overline{m}\rb{\overline{m}\rb{v\rb{x_0}, v\rb{x_1}}, \overline{i}\rb{v\rb{x_0}}} \\
& = \overline{m}\rb{\overline{m}\rb{g, h}, \overline{i}\rb{g}} \\
& = \overline{m}\rb{gh, g^{-1}} \\
& = ghg^{-1}.
\end{align*}
\end{example*}

\begin{definition}
Suppose $ \AA $ is an $ \LL $-structure and $ x_l $ is any variable. Suppose $ v, w $ are valuations in $ \AA $. We say $ v, w $ are \textbf{$ x_l $-equivalent} if $ v\rb{x_m} = w\rb{x_m} $ whenever $ m \ne l $.
\end{definition}

\pagebreak

\begin{definition}
\label{def:2.2.9}
Suppose $ \AA $ is an $ \LL $-structure and $ v $ is a valuation in $ \AA $. Define, for an $ \LL $-formula $ \phi $, what is meant by $ v $ \textbf{satisfies} $ \phi $, in $ \AA $.
\begin{itemize}
\item Atomic formulas. Suppose $ R $ is an $ n $-ary relation symbol and $ t_1, \dots, t_n $ are terms of $ \LL $. Then
$$ v \ \text{satisfies the atomic formula} \ R\rb{t_1, \dots, t_n} \qquad \iff \qquad \overline{R}\rb{v\rb{t_1}, \dots, v\rb{t_n}} \ \text{holds in} \ \AA. $$
\item Suppose $ \phi, \psi $ are $ \LL $-formulas, and we already know about valuations satisfying $ \phi, \psi $.
\begin{align*}
v \ \text{satisfies} \ \notb{\phi} \ \text{in} \ \AA \qquad \iff \qquad & v \ \text{does not satisfy} \ \phi \ \text{in} \ \AA, \\
v \ \text{satisfies} \ \impb{\phi}{\psi} \ \text{in} \ \AA \qquad \iff \qquad & \text{it is not the case that} \ v \ \text{satisfies} \ \phi \ \text{in} \ \AA \\
& \text{and} \ v \ \text{does not satisfy} \ \psi \ \text{in} \ \AA, \\
v \ \text{satisfies} \ \fab{x_l}\phi \ \text{in} \ \AA \qquad \iff \qquad & \text{whenever} \ w \ \text{is a valuation in} \ \AA \ \text{which is} \ x_l \text{-equivalent to} \ v, \\
& \text{then} \ w \ \text{satisfies} \ \phi \ in \ \AA.
\end{align*}
\end{itemize}
\end{definition}

\lecture{10}{Thursday}{25/10/18}

\begin{remark*}
\ref{def:2.2.9} does not work if we allow empty structure.
\end{remark*}

\begin{notation*}
If $ v $ satisfies $ \phi $, write $ v\sb{\phi} = T $. If $ v $ does not satisfy $ \phi $, write $ v\sb{\phi} = F $. If every valuation in $ \AA $ satisfies $ \phi $, say that $ \phi $ is \textbf{true} in $ \AA $ or $ \AA $ is a \textbf{model} of $ \phi $, and write $ \AA \vDash \phi $. If $ \AA \vDash \phi $ for every $ \LL $-structure $ \AA $, we say that $ \phi $ is \textbf{logically valid}, and write $ \vDash \phi $.
\end{notation*}

These are the analogues of tautologies in the propositional logic. Difference is that in propositional logic there is an algorithm to decide whether a given formula is a tautology. There is no such algorithm to decide whether a given $ \LL $-formula is logically valid or not. This is a consequence of G\"odel's incompleteness theorem.

\begin{example}
\hfill
\begin{itemize}
\item Suppose $ \LL $ has a binary relation symbol $ R $. The $ \LL $-formula
$$ \impb{R\rb{x_1, x_2}}{\impb{R\rb{x_2, x_3}}{R\rb{x_1, x_3}}} $$
is true in $ \AA = \ab{\N; R} $, where $ R $ is interpreted as $ < $. If not, there is a valuation $ v $, in $ \AA $, such that $ v $ satisfies $ R\rb{x_1, x_2} $ and $ v $ does not satisfy $ \impb{R\rb{x_2, x_3}}{R\rb{x_1, x_3}} $. So
$$ v\sb{R\rb{x_2, x_3}} = T, \qquad v\sb{R\rb{x_1, x_3}} = F. $$
Let $ v\rb{x_i} = a_i \in \N $. So
$$ a_1 < a_2, \qquad a_2 < a_3, \qquad a_1 \not< a_3. $$
As $ < $ is transitive on $ \N $, this is a contradiction.
\item The same formula is not true in the structure $ \BB $ with domain $ \N $ where we interpret $ R\rb{x_i, x_j} $ as $ x_i \ne x_j $. Take a valuation in $ \BB $ with
$$ v\rb{x_1} = 1 = v\rb{x_3}, \qquad v\rb{x_2} = 2. $$
$ v $ does not satisfy the formula in $ \BB $.
\item Recall that $ \teb{x_1}\phi $ is an abbreviation for $ \notb{\fab{x_1}\notb{\phi}} $. Suppose $ \AA $ is an $ \LL $-structure and $ \phi $ an $ \LL $-formula. Let $ v $ be a valuation in $ \AA $. Claim that then $ v $ satisfies $ \teb{x_1}\phi $ in $ \AA $ if and only if there is a valuation $ w $, which is $ x_1 $-equivalent to $ v $, such that $ w $ satisfies $ \phi $.
\begin{itemize}
\item[$ \implies $] Suppose $ v $ satisfies $ \notb{\fab{x_1}\notb{\phi}} $. Using \ref{def:2.2.9}, $ v $ does not satisfy $ \fab{x_1}\notb{\phi} $. So there is valuation $ w $, $ x_1 $-equivalent to $ v $, such that $ w $ does not satisfy $ \notb{\phi} $. Such a $ w $ satisfies $ \phi $.
\item[$ \impliedby $] Exercise.
\end{itemize}
\end{itemize}
\end{example}

\begin{example}
$$ \fab{x_1}\teb{x_2}R\rb{x_1, x_2} $$
is true in $ \ab{\Z; <} $ and $ \ab{\N; <} $, but not in $ \ab{\N; >} $.
\end{example}

\pagebreak

\begin{example}
Suppose $ \phi $ is any $ \LL $-formula. Then
\begin{enumerate}
\item $ \impb{\teb{x_1}\fab{x_2}\phi}{\fab{x_2}\teb{x_1}\phi} $ is logically valid, and
\item $ \impb{\fab{x_2}\teb{x_1}\phi}{\teb{x_1}\fab{x_2}\phi} $ is not necessarily logically valid.
\end{enumerate}
\end{example}

\begin{proof}
Exercise.
\begin{enumerate}
\item Argument with valuation.
\item Give an example.
\end{enumerate}
\end{proof}

\begin{example*}
Consider the propositional formula
$$ \chi : \impb{p_1}{\impb{p_2}{p_1}}. $$
Suppose $ \LL $ is a first-order language and $ \phi_1, \phi_2 $ are $ \LL $-formulas. Substitute $ \phi_1 $ in place of $ p_1 $, and $ \phi_2 $ in place of $ p_2 $, in $ \chi $. We obtain an $ \LL $-formula
$$ \theta : \impb{\phi_1}{\impb{\phi_2}{\phi_1}}. $$
Check that, as $ \chi $ is a tautology, $ \theta $ is logically valid.
\end{example*}

\begin{definition}
Suppose $ \chi $ is an $ \LL $-formula involving propositional variables $ p_1, \dots, p_n $. Suppose $ \LL $ is a first-order language and $ \phi_1, \dots, \phi_n $ are $ \LL $-formulas. A \textbf{substitution instance} of $ \chi $ is obtained by replacing each $ p_i $ in $ \chi $ by $ \phi_i $, for $ i = 1, \dots, n $.
\end{definition}

Call the result $ \theta $.

\begin{theorem}
\hfill
\begin{itemize}
\item $ \theta $ is an $ \LL $-formula, and
\item if $ \chi $ is a tautology, then $ \theta $ is logically valid.
\end{itemize}
\end{theorem}

\lecture{11}{Friday}{26/10/18}

\begin{proof}
Take an $ \LL $-structure $ \AA $ and a valuation $ v $ in $ \AA $. Use this to define a propositional valuation $ w $ with
$$ w\rb{p_i} = v\sb{\phi_i}, \qquad i \le n. $$
Then prove, by induction on the number of connectives in $ \chi $, that
$$ w\rb{\chi} = v\sb{\theta}. $$
In particular, if $ \chi $ is a tautology, then $ v\sb{\theta} = T $. (Exercise) For example, in the inductive step, consider
$$ \chi : \impb{\alpha}{\beta}. $$
So $ \theta $ is $ \impb{\theta_1}{\theta_2} $, where $ \theta_1 $ is obtained from $ \alpha $ and $ \theta_2 $ is obtained from $ \beta $. By inductive hypothesis,
$$ w\rb{\alpha} = v\sb{\theta_1}, \qquad w\rb{\beta} = v\sb{\theta_2}. $$
So $ w\impb{\alpha}{\beta} = v\sb{\impb{\theta_1}{\theta_2}} $, etc.
\end{proof}

\begin{note*}
Not all logically valid formulas arise in this way.
\end{note*}

\begin{example*}
$$ \impb{\teb{x_2}\fab{x_1}\phi}{\fab{x_1}\teb{x_2}\phi}. $$
\end{example*}

\pagebreak

\subsection{Bound and free variables in formulas}

\begin{definition}
Suppose $ \phi, \psi $ are $ \LL $-formulas and $ \fab{x_i}\phi $ occurs as a subformula of $ \psi $, that is $ \psi $ is
$$ \dots \quad \fab{x_i}\phi \quad \dots. $$
\begin{itemize}
\item We say that $ \phi $ is the \textbf{scope} of that quantifier $ \fab{x_i} $ here in $ \psi $. An occurrence of a variable $ x_j $ in $ \psi $ is \textbf{bound} if it is in the scope of a quantifier $ \fab{x_j} $ in $ \psi $, or it is the $ x_j $ here in $ \fab{x_j} $.
\item Otherwise, it is a \textbf{free} occurrence, of $ x_j $. Variables having a free occurrence in $ \psi $ are called \textbf{free variables} of $ \psi $.
\item A formula with no free variables is called a \textbf{closed formula} or a \textbf{sentence}, of $ \LL $.
\end{itemize}
\end{definition}

\begin{example*}
$$ \psi_1 : \impb{R_1\rb{\overbrace{x_1, x_2}^{\text{free}}}}{\fab{x_3}\underbrace{R_2\rb{\overbrace{x_1}^{\text{free}}, \overbrace{x_3}^{\text{bound}}}}_{\text{scope of} \ \fab{x_3}}} $$
$$ \psi_2 : \impb{\fab{x_1}\underbrace{R_1\rb{\overbrace{x_1}^{\text{bound}}, \overbrace{x_2}^{\text{free}}}}_{\text{scope of} \ \fab{x_1}}}{R_2\rb{\overbrace{x_1, x_2}^{\text{free}}}} \qquad \psi_{2'} : \fab{x_1}\underbrace{\impb{R_1\rb{\overbrace{x_1}^{\text{bound}}, \overbrace{x_2}^{\text{free}}}}{R_2\rb{\overbrace{x_1}^{\text{bound}}, \overbrace{x_2}^{\text{free}}}}}_{\text{scope of} \ \fab{x_1}} $$
$$ \psi_3 : \impb{\teb{x_1}\underbrace{R_1\rb{\overbrace{x_1}^{\text{bound}}, \overbrace{x_2}^{\text{free}}}}_{\text{scope of} \ \teb{x_1}}}{\fab{x_2}\underbrace{R_2\rb{\overbrace{x_2}^{\text{bound}}, \overbrace{x_3}^{\text{free}}}}_{\text{scope of} \ \fab{x_2}}} $$
\end{example*}

\begin{definition}
If $ \psi $ is an $ \LL $-formula with free variables amongst $ x_1, \dots, x_n $, we might write
$$ \psi\rb{x_1, \dots, x_n}, $$
instead of $ \psi $. If $ t_1, \dots, t_n $ are terms, by
$$ \psi\rb{t_1, \dots, t_n}, $$
we mean the $ \LL $-formula obtained by replacing each free occurrence of $ x_i $ in $ \psi $ by $ t_i $.
\end{definition}

\begin{example*}
Let $ t_1 $ be $ f_1\rb{x_1} $, $ t_2 $ be $ f_2\rb{x_1, x_2} $, and
$$ \psi\rb{x_1, x_2} : \impb{\fab{x_1}R\rb{x_1, \overbrace{x_2}^{\text{free}}}}{\fab{x_3}R\rb{\overbrace{x_1}^{\text{free}}, \overbrace{x_2}^{\text{free}}, x_3}}. $$
So
$$ \psi\rb{t_1, t_2} : \impb{\fab{x_1}R_1\rb{x_1, f_2\rb{x_1, x_2}}}{\fab{x_3}R_2\rb{f_1\rb{x_1}, f_2\rb{x_1, x_2}, x_3}}. $$
\end{example*}

\pagebreak

\begin{theorem}
\label{thm:2.3.3}
Suppose $ \phi $ is a closed $ \LL $-formula and $ \AA $ is an $ \LL $-structure. Then either $ \AA \vDash \phi $ or $ \AA \vDash \notb{\phi} $. More generally, if $ \phi $ has free variables amongst $ x_1, \dots, x_n $, and $ v, w $ are valuations in $ \AA $ with $ v\rb{x_i} = w\rb{x_i} $, for $ i = 1, \dots, n $, then
$$ v\sb{\phi} = T \qquad \iff \qquad w\sb{\phi} = T. $$
Allow $ n = 0 $ here, for no free variables.
\end{theorem}

\begin{proof}
Note that the first statement follows from the generalisation. If $ \phi $ has no free variables, then, for any valuations $ v, w $ in $ \AA $, they agree on the free variables of $ \phi $, so
$$ v\sb{\phi} = w\sb{\phi}. $$
Prove the generalisation by induction on the number of connectives and quantifiers in $ \phi $.
\begin{itemize}
\item Base case. $ \phi $ is atomic. Let
$$ \phi : R\rb{t_1, \dots, t_m}, $$
where $ t_j $ are terms. The $ t_j $ only involve variables amongst $ x_1, \dots, x_n $. As $ v $ and $ w $ agree on these variables,
$$ v\rb{t_j} = w\rb{t_j}. $$
So
$$ v\sb{R\rb{t_1, \dots, t_m}} = T \qquad \iff \qquad \overline{R}\rb{v\rb{t_1}, \dots, v\rb{t_m}} \qquad \iff \qquad w\sb{R\rb{t_1, \dots, t_m}} = T. $$
\item Inductive step. $ \phi $ is $ \notb{\psi} $, $ \impb{\psi}{\chi} $, or $ \fab{x_i}\psi $. (Exercise: first two cases) Suppose $ \phi $ is $ \fab{x_i}\psi $. Suppose $ v\sb{\phi} = F $. By \ref{def:2.2.9}, there is a valuation $ v' $, $ x_i $-equivalent to $ v $, with $ v'\sb{\psi} = F $. The free variables of $ \psi $ are amongst $ x_1, \dots, x_n, x_i $. Let $ w' $ be the valuation $ x_i $-equivalent to $ w $ with
$$ w'\rb{x_i} = v'\rb{x_i}. $$
Then $ v', w' $ agree on the free variables of $ \psi $. By inductive hypothesis,
$$ v'\sb{\psi} = w'\sb{\psi}, $$
so $ w'\sb{\psi} = F $. As $ w' $ is $ x_i $-equivalent to $ w $, we obtain $ w\sb{\fab{x_i}\psi} = F $.
\end{itemize}
\end{proof}

\lecture{12}{Monday}{29/10/18}

\begin{remark}
If $ \AA $ is an $ \LL $-structure and $ \psi\rb{x_1, \dots, x_n} $ an $ \LL $-formula, whose free variables are amongst $ x_1, \dots, x_n $, and $ a_1, \dots, a_n \in A $, for domain $ A $, then we write
$$ \AA \vDash \psi\rb{a_1, \dots, a_n}, $$
to mean $ v\sb{\psi} = T $ for every valuation $ v $ in $ \AA $ with $ v\rb{x_i} = a_i $, for $ i = 1, \dots, n $.
\end{remark}

\begin{remark*}
By the proof of \ref{thm:2.3.3}, this holds if $ v\sb{\psi} = T $ for some such valuation.
\end{remark*}

\begin{example*}
A warning example, where $ \AA \vDash \fab{x_1}\phi\rb{x_1} $, but we have term $ t $, and a valuation $ v $ in $ \AA $, with $ v\sb{\phi\rb{t}} = F $. Let
$$ \phi\rb{x_1} : \impb{\fab{x_2}\underbrace{R\rb{\overbrace{x_1}^{\text{free}}, x_2}}_{\text{scope of} \ \fab{x_2}}}{S\rb{\overbrace{x_1}^{\text{free}}}}. $$
Let $ t_1 $ be $ x_2 $. Then $ \phi\rb{t_1} $ is
$$ \impb{\fab{x_2}R\rb{x_2, x_2}}{S\rb{x_2}}. $$
Suppose $ \AA = \ab{\N; \le, = 0} $.
\begin{itemize}
\item Domain is $ \N = \cb{0, 1, \dots} $,
\item $ R\rb{x_1, x_2} $ is interpreted as $ x_1 \le x_2 $, and
\item $ S\rb{x_1} $ is interpreted as $ x_1 = 0 $.
\end{itemize}
So $ \AA \vDash \fab{x_1}\phi\rb{x_1} $. But if we choose a valuation $ v\rb{x_2} = 1 $, then $ v\sb{\phi\rb{t_1}} = F $ in $ \AA $.
\end{example*}

\pagebreak

\begin{definition}
Let $ \phi $ be an $ \LL $-formula, $ x_i $ a variable, and $ t $ an $ \LL $-term. We say \textbf{$ t $ is free for $ x_i $ in $ \phi $} if there is no variable $ x_j $ in $ t $, such that $ x_i $ has a free occurrence within the scope of a quantifier $ \fab{x_j} $ in $ \phi $.
\end{definition}

\begin{example*}
Exercise. Let $ t = f\rb{x_3, x_2, x_5} $. Let
\begin{align*}
\phi_1 & : \impb{\impb{\fab{x_2}R\rb{x_2, x_4}}{K\rb{x_1}}}{\fab{x_1}R\rb{x_1, x_1}}, \\
\phi_2 & : \impb{\fab{x_2}\impb{R\rb{x_2, x_4}}{\fab{x_1}K\rb{x_1}}}{\fab{x_2}R\rb{x_1, x_1}}.
\end{align*}
For which $ t $ is $ t $ free for $ x_1 $?
\end{example*}

\begin{theorem}
\label{thm:2.3.6}
Suppose $ \phi\rb{x_1} $ is an $ \LL $-formula, possibly with other free variables. Let $ t $ be a term free for $ x_1 $ in $ \phi $, then
$$ \vDash \impb{\fab{x_1}\phi\rb{x_1}}{\phi\rb{t}}. $$
In particular, if $ \AA $ is an $ \LL $-structure with $ \AA \vDash \fab{x_1}\phi\rb{x_1} $, then $ \AA \vDash \phi\rb{t} $.
\end{theorem}

\begin{lemma}
\label{lem:2.3.7}
With this notation, suppose $ \AA $ is an $ \LL $-structure and $ v $ is a valuation in $ \AA $. Let $ v' $ be the valuation in $ \AA $ which is $ x_1 $-equivalent to $ v $, with $ v'\rb{x_1} = v\rb{t} $. Then
$$ v'\sb{\phi\rb{x_1}} = T \qquad \iff \qquad v\sb{\phi\rb{t}} = T. $$
\end{lemma}

\begin{proof}[Proof of \ref{thm:2.3.6}]
Suppose $ v $ is a valuation with $ v\sb{\phi\rb{t}} = F $. Show that $ v\sb{\fab{x_1}\phi\rb{x_1}} = F $, then
$$ v\sb{\impb{\fab{x_1}\phi\rb{x_1}}{\phi\rb{t}}} = T. $$
Take $ v' $, $ x_1 $-equivalent to $ v $, and
$$ v'\rb{x_1} = v\rb{t}. $$
Then by \ref{lem:2.3.7},
$$ v'\sb{\phi\rb{x_1}} = F. $$
Thus $ v\sb{\fab{x_1}\phi\rb{x_1}} = F $.
\end{proof}

\begin{proof}[Proof of \ref{lem:2.3.7}]
Appendix A.1.
\end{proof}

\subsection{The formal system $ K_\LL $}

\begin{definition}
Suppose $ \LL $ is a first-order language. The formal system $ K_\LL $ has, as formulas, $ \LL $-formulas, and the following.
\begin{itemize}
\item Axioms. For $ \LL $-formulas $ \phi, \chi, \psi $,
\begin{enumerate}[label=(A\arabic*)]
\item $ \impb{\phi}{\impb{\psi}{\phi}} $,
\item $ \impb{\impb{\phi}{\impb{\psi}{\chi}}}{\impb{\impb{\phi}{\psi}}{\impb{\phi}{\chi}}} $,
\item $ \impb{\impb{\notb{\phi}}{\notb{\psi}}}{\impb{\psi}{\phi}} $,
\end{enumerate}
\begin{enumerate}[label=(K\arabic*)]
\item $ \impb{\fab{x_i}\phi\rb{x_i}}{\phi\rb{t}} $, where $ t $ is a term free for $ x_i $ in $ \phi $, and $ \phi $ can have other free variables, and
\item $ \impb{\fab{x_i}\impb{\phi}{\psi}}{\impb{\phi}{\fab{x_i}\psi}} $, if $ x_i $ is not free in $ \phi $.
\end{enumerate}
\item Deduction rules.
\begin{itemize}
\item[(MP)] Modus Ponens. From $ \phi, \impb{\phi}{\psi} $, deduce $ \psi $.
\item[(Gen)] \textbf{Generalisation}. From $ \phi $, deduce $ \fab{x_i}\phi $.
\end{itemize}
\end{itemize}
A proof in $ K_\LL $ is a finite sequence of $ \LL $-formulas, each of which is an axiom, or deduced from previous formulas in the proof using a rule of deduction. A theorem of $ K_\LL $ is the last formula in some proof. Write $ \vdash_{K_\LL} \phi $ for $ \phi $ is a theorem in $ K_\LL $.
\end{definition}

\begin{note*}
Books do not always use $ K_\LL $, that is they write $ \vdash \phi $.
\end{note*}

\pagebreak

\begin{definition}
Suppose $ \Sigma $ is a set of $ \LL $-formulas and $ \psi $ an $ \LL $-formula. A deduction of $ \psi $ from $ \Sigma $ is a finite sequence of formulas, ending with $ \psi $, each of which is one of
\begin{itemize}
\item an axiom,
\item a formula in $ \Sigma $, or
\item obtained from earlier formulas in the deduction using MP or Gen, with the restriction that, when Gen is applied, it does not involve a variable occurring freely in a formula in $ \Sigma $.
\end{itemize}
Write $ \Sigma \vdash_{K_\LL} \psi $ if there is a deduction from $ \Sigma $ to $ \psi $.
\end{definition}

\lecture{13}{Thursday}{01/11/18}

Lecture 13 is a problem class.

\lecture{14}{Friday}{02/11/18}

\begin{remark}
\hfill
\begin{itemize}
\item If $ \Sigma $ consists of closed formulas, do not need to worry about the restriction on Gen.
\item $ \phi \vdash_{K_\LL} \fab{x_1}\phi $.
\end{itemize}
\end{remark}

\begin{theorem}
\label{thm:2.4.4}
Suppose $ \phi $ is an $ \LL $-formula which is a substitution instance of a tautology in propositional logic. Then $ \vdash_{K_\LL} \phi $.
\end{theorem}

\begin{example*}
For an $ \LL $-formula $ \phi $,
$$ \impb{\notb{\notb{\phi}}}{\phi}, $$
as this is a substitution instance of
$$ \impb{\notb{\notb{p_1}}}{p_1}. $$
\end{example*}

\begin{proof}
There is a tautology $ \chi $ with propositional variables $ p_1, \dots, p_n $ and $ \LL $-formulas $ \psi_1, \dots, \psi_n $, such that $ \phi $ is obtained from $ \chi $ by substituting $ \psi_i $ for $ p_i $, for $ i = 1, \dots, n $. By completeness of propositional logic in \ref{thm:1.3.4}, there is a proof in $ L $ of $ \chi $,
$$ \chi_1, \dots, \chi_r, $$
where each $ \chi_i $ is a propositional formula, that is in $ L $, and $ \chi_r = \chi $. If we substitute $ \psi_1, \dots, \psi_n $ for $ p_1, \dots, p_n $ in all $ \chi_j $, we obtain a sequence of $ \LL $-formulas,
$$ \phi_1, \dots, \phi_r, $$
which is a proof of $ \phi = \phi_r $ in $ K_\LL $.
\end{proof}

\begin{theorem}[Soundness of $ K_\LL $]
If $ \vdash_{K_\LL} \phi $, then $ \vDash \phi $, that is it is logically valid.
\end{theorem}

\begin{proof}
Like in the proof for $ L $, we need to show
\begin{enumerate}
\item axioms are logically valid, and
\item deduction rules preserve logical validity.
\end{enumerate}
\begin{enumerate}
\item A1, A2, A3 are substitution instances of propositional tautologies, by \ref{def:2.2.1}, so are logically valid by \ref{thm:2.4.4}. K1 is logically valid by \ref{thm:2.3.6}. K2 is $ \impb{\fab{x_i}\impb{\phi}{\psi}}{\impb{\phi}{\fab{x_i}\psi}} $, if $ x_i $ is not free in $ \phi $. Suppose we have valuation $ v $ such that $ v\sb{\impb{\phi}{\fab{x_i}\psi}} = F $. So
$$ v\sb{\phi} = T, \qquad v\sb{\fab{x_i}\psi} = F. $$
So there is a valuation $ v' $, $ x_i $-equivalent to $ v $, with $ v'\sb{\psi} = F $. $ v $ and $ v' $ agree on all variables free in $ \phi $. So by \ref{thm:2.3.3},
$$ v\sb{\phi} = v'\sb{\phi} = T, $$
so $ v'\sb{\impb{\phi}{\psi}} = F $. Thus $ v\sb{\fab{x_i}\impb{\phi}{\psi}} = F $, so $ v\sb{\text{K2}} = T $.
\item For MP, if $ \vDash \phi $ and $ \vDash \impb{\phi}{\psi} $, then $ \vDash \psi $. For Gen, if $ \vDash \phi $, then $ \vDash \fab{x_i}\phi $. (Exercise: proof)
\end{enumerate}
\end{proof}

\pagebreak

\begin{example}
\label{eg:2.4.6}
Suppose $ \Sigma \vdash_{K_\LL} \psi $, then for every valuation $ v $ with $ v\sb{\sigma} = T $ for all $ \sigma \in \Sigma $, we have $ v\sb{\psi} = T $.
\end{example}

\begin{proof}
Exercise.
\end{proof}

\begin{corollary}
\label{cor:2.4.7}
There is no $ \LL $-formula $ \phi $ with $ \vdash_{K_\LL} \phi $ and $ \vdash_{K_\LL} \notb{\phi} $.
\end{corollary}

\begin{theorem}[Deduction theorem]
Suppose $ \LL $ is a first-order language, $ \Sigma $ is a set of $ \LL $-formulas, and $ \phi, \psi $ are $ \LL $-formulas. Then if $ \Sigma \cup \cb{\phi} \vdash_{K_\LL} \psi $, then $ \Sigma \vdash_{K_\LL} \impb{\phi}{\psi} $.
\end{theorem}

\begin{proof}
Follows proof of deduction theorem for $ L $ in \ref{thm:1.2.5}, by induction on the length of the deduction.
\begin{itemize}
\item Base case. One line deduction. Argue exactly as in \ref{thm:1.2.5}. Note that $ \vdash_{K_\LL} \impb{\phi}{\phi} $ by \ref{thm:2.4.4}.
\item Inductive step. Suppose $ \psi $ follows from earlier formulas in the deduction using MP or Gen. MP is exactly as in \ref{thm:1.2.5}. For Gen, suppose $ \psi $ is obtained using Gen, then $ \psi $ is $ \fab{x_i}\theta $ and
$$ \Sigma \cup \cb{\phi} \vdash_{K_\LL} \theta, $$
and $ x_i $ is not free in any formula in $ \Sigma \cup \cb{\phi} $. By induction, we have
$$ \Sigma \vdash_{K_\LL} \impb{\phi}{\theta}. $$
By K2,
$$ \Sigma \vdash_{K_\LL} \impb{\fab{x_i}\impb{\phi}{\theta}}{\impb{\phi}{\fab{x_i}\theta}}. $$
By Gen,
$$ \Sigma \vdash_{K_\LL} \fab{x_i}\impb{\phi}{\theta}, $$
for $ x_i $ not free in any formula in $ \Sigma $. So by MP, we get $ \Sigma \vdash_{K_\LL} \impb{\phi}{\fab{x_i}\theta} $, which is $ \Sigma \vdash_{K_\LL} \impb{\phi}{\psi} $.
\end{itemize}
\end{proof}

\lecture{15}{Monday}{05/11/18}

\subsection{G\"odel's completeness theorem}

\begin{definition}
A set $ \Sigma $ of $ \LL $-formulas is consistent if there is no formula $ \phi $ with
$$ \Sigma \vdash_{K_\LL} \phi, \qquad \Sigma \vdash_{K_\LL} \notb{\phi}. $$
\end{definition}

By soundness, or \ref{cor:2.4.7}, $ \emptyset $ is consistent, so $ K_\LL $ is consistent.

\begin{remark*}
If $ \Sigma $ is inconsistent, then
$$ \Sigma \vdash_{K_\LL} \chi, $$
for any $ \LL $-formula $ \chi $.
\end{remark*}

Recall that a closed $ \LL $-formula is one without free variables, sometimes called a sentence of $ \LL $. Show that, if $ \Sigma $ is a set of closed $ \LL $-formulas which is consistent, then there is an $ \LL $-structure $ \AA $ with $ \AA \vDash \sigma $ for all $ \sigma \in \Sigma $. For a simplification, suppose that $ \LL $ is countable, that is the variables are $ x_0, x_1, \dots $, and there are countably many relation, function, and constant symbols. So we can enumerate the $ \LL $-formulas, or any subset thereof, as a list indexed by $ \N $. Enumerate the closed $ \LL $-formulas as
$$ \psi_0, \psi_1, \dots \in \cb{\psi_i : i \in \N}. $$

\begin{proposition}
\label{prop:2.5.2}
Suppose $ \Sigma $ is a consistent set of closed $ \LL $-formulas and $ \phi $ is a closed $ \LL $-formula.
\begin{enumerate}
\item Compare \ref{prop:1.3.7}. If $ \Sigma \not\vdash_{K_\LL} \phi $, then $ \Sigma \cup \cb{\notb{\phi}} $ is consistent.
\item Compare the Lindenbaum lemma in \ref{prop:1.3.8}. There is a consistent set $ \Sigma^* \supseteq \Sigma $ of closed $ \LL $-formulas such that, for every closed $ \LL $-formula $ \psi $, either $ \Sigma^* \vdash_{K_\LL} \psi $ or $ \Sigma^* \vdash_{K_\LL} \notb{\psi} $.
\end{enumerate}
\end{proposition}

\begin{proof}
\hfill
\begin{enumerate}
\item As in \ref{prop:1.3.7}, use deduction theorem and $ \vdash_{K_\LL} \impb{\impb{\notb{\phi}}{\phi}}{\phi} $.
\item Use $ 1 $ and the enumeration $ \psi_0, \psi_1, \dots $ of the closed $ \LL $-formulas.
\end{enumerate}
\end{proof}

\pagebreak

\begin{theorem}
\label{thm:2.5.3}
Suppose $ \LL $ is a countable first-order language and $ \Sigma $ is a consistent set of closed $ \LL $-formulas. Then there is a countable $ \LL $-structure $ \AA $ such that $ \AA \vDash \Sigma $, that is $ \AA \vDash \sigma $ for all $ \sigma \in \Sigma $.
\end{theorem}

\begin{theorem}
\label{thm:2.5.4}
Let $ \Sigma $ be a set of closed $ \LL $-formulas and $ \phi $ a closed $ \LL $-formula. If every model of $ \Sigma $ is a model of $ \phi $, then $ \Sigma \vdash_{K_\LL} \phi $. That is, if $ \AA \vDash \Sigma $, or $ \AA \vDash \sigma $ for all $ \sigma \in \Sigma $, then $ \AA \vDash \phi $, then $ \Sigma \vdash_{K_\LL} \phi $.
\end{theorem}

\begin{notation*}
$ \Sigma \vDash \phi $, if every model of $ \Sigma $ is a model of $ \phi $. Then $ \Sigma \vDash \phi $ implies $ \Sigma \vdash_{K_\LL} \phi $. Converse is soundness.
\end{notation*}

\begin{proof}
May assume $ \Sigma $ is consistent. Otherwise, everything is a consequence of $ \Sigma $. By assumption, there is no model of $ \Sigma \cup \cb{\notb{\phi}} $. So by \ref{thm:2.5.3}, $ \Sigma \cup \cb{\notb{\phi}} $ is inconsistent. So by \ref{prop:2.5.2}.$ 1 $, $ \Sigma \vdash_{K_\LL} \phi $.
\end{proof}

\begin{theorem}[G\"odel's completeness theorem for $ K_\LL $]
If $ \phi $ is an $ \LL $-formula with $ \vDash \phi $, then $ \phi $ is a theorem of $ K_\LL $, that is $ \vdash_{K_\LL} \phi $.
\end{theorem}

\begin{proof}
If $ \phi $ is closed this follows from \ref{thm:2.5.4} with $ \Sigma = \emptyset $. Suppose $ \phi $ has free variables amongst $ x_1, \dots, x_n $, and consider the closed formula
$$ \psi : \fab{x_1} \dots \fab{x_n}\phi. $$
As $ \vDash \phi $, we obtain $ \vDash \psi $. So, by the closed case, $ \vdash_{K_\LL} \psi $, that is
\begin{equation}
\label{eq:13}
\vdash_{K_\LL} \fab{x_1} \dots \fab{x_n}\phi.
\end{equation}
If $ \theta $ is any formula, then
$$ \vdash_{K_\LL} \impb{\fab{x_i}\theta}{\theta}, $$
by the K1 axiom. So from $ \rb{\ref{eq:13}} $ and this fact, and MP, applied $ n $ times, we obtain $ \vdash_{K_\LL} \phi $.
\end{proof}

\begin{corollary}[Compactness theorem for $ K_\LL $]
\label{cor:2.5.6}
Suppose $ \Sigma $ is a set of closed $ \LL $-formulas and every finite subset of $ \Sigma $ has a model. Then $ \Sigma $ has a model.
\end{corollary}

\begin{proof}
Suppose $ \Sigma $ has no model. By \ref{thm:2.5.3}, $ \Sigma $ is inconsistent, so there is a formula $ \phi $ with
$$ \Sigma \vdash_{K_\LL} \phi, \qquad \Sigma \vdash_{K_\LL} \notb{\phi}. $$ Deductions only involve finitely many formulas in $ \Sigma $. So there is a finite $ \Sigma_0 \subseteq \Sigma $ with
$$ \Sigma_0 \vdash_{K_\LL} \phi, \qquad \Sigma_0 \vdash_{K_\LL} \notb{\phi}. $$
But then $ \Sigma_0 $ is inconsistent, so has no model, a contradiction.
\end{proof}

\lecture{16}{Thursday}{08/11/18}

\begin{proof}[Proof of \ref{thm:2.5.3}]
Appendix A.2.
\end{proof}

\begin{example*}
Exercise. Think about this where $ \Sigma $ consists of the group axioms. What is $ \AA $? Is it a group?
\end{example*}

\lecture{17}{Friday}{09/11/18}

\subsection{Equality}

\begin{example*}
In the language of groups, have a binary relation symbol $ E\rb{x_1, x_2} $ for equality $ x_1 = x_2 $.
\end{example*}

\begin{definition}
Suppose $ \LL^E $ is a first-order language with a distinguished binary relation symbol $ E $.
\begin{itemize}
\item An $ \LL^E $-structure in which $ E $ is interpreted as equality $ = $ is a \textbf{normal} $ \LL^E $-structure.
\item The following are the \textbf{axioms of equality}, $ \Sigma_E $.
\begin{itemize}
\item $ \fab{x_1}E\rb{x_1, x_1} $.
\item $ \fab{x_1}\fab{x_2}\impb{E\rb{x_1, x_2}}{E\rb{x_2, x_1}} $.
\item $ \fab{x_1}\fab{x_2}\fab{x_3}\impb{E\rb{x_1, x_2}}{\impb{E\rb{x_2, x_3}}{E\rb{x_1, x_3}}} $.
\item For each $ n $-ary relation symbol $ R $ of $ \LL^E $,
$$ \fab{x_1 \dots x_n}\fab{y_1 \dots y_n}\impb{\rb{R\rb{x_1, \dots, x_n} \land E\rb{x_1, y_1} \land \dots \land E\rb{x_n, y_n}}}{R\rb{y_1, \dots, y_n}}. $$
\item For each $ m $-ary function symbol $ f $ of $ \LL^E $,
$$ \fab{x_1 \dots x_m}\fab{y_1 \dots y_m}\impb{\rb{E\rb{x_1, y_1} \land \dots \land E\rb{x_m, y_m}}}{E\rb{f\rb{x_1, \dots, x_m}, f\rb{y_1, \dots, y_m}}}. $$
\end{itemize}
\end{itemize}
\end{definition}

\pagebreak

\begin{remark}
\hfill
\begin{itemize}
\item If $ \AA $ is a normal $ \LL^E $-structure, then $ \AA \vDash \Sigma_E $.
\item Suppose $ \AA = \ab{A; \overline{E}, \dots} $ is an $ \LL^E $-structure and $ \AA \vDash \Sigma_E $. Then $ \overline{E} $ is an equivalence relation on $ A $. Denote, for $ a \in A $,
$$ \widehat{a} = \cb{b \in A \mid \overline{E}\rb{a, b} \ \text{holds}}, $$
the equivalence class of $ a $. Let
$$ \widehat{A} = \cb{\widehat{a} \mid a \in A}. $$
Make $ \widehat{A} $ into an $ \LL^E $-structure $ \widehat{\AA} $.
\begin{itemize}
\item If $ R $ is an $ n $-ary relation symbol and $ \widehat{a_1}, \dots, \widehat{a_n} \in \widehat{A} $, then say
$$ \overline{R}\rb{\widehat{a_1}, \dots, \widehat{a_n}} \ \text{holds in} \ \widehat{\AA} \qquad \iff \qquad \overline{R}\rb{a_1, \dots, a_n} \ \text{holds in} \ \AA. $$
This is well-defined by $ \Sigma_E $.
\item Similarly, if $ f $ is an $ m $-ary function symbol and $ \widehat{a_1}, \dots, \widehat{a_m} \in \widehat{A} $, let
$$ \overline{f}\rb{\widehat{a_1}, \dots, \widehat{a_m}} = \widehat{\overline{f}\rb{a_1, \dots, a_m}}. $$
This is also well-defined by $ \Sigma_E $.
\item If $ c $ is a constant symbol, then interpret $ c $ as $ \widehat{\overline{c}} $ in $ \widehat{\AA} $, where $ \overline{c} $ is the interpretation in $ \AA $.
\end{itemize}
\end{itemize}
\end{remark}

\begin{note*}
In $ \widehat{\AA} $,
$$ \overline{E}\rb{\widehat{a_1}, \widehat{a_2}} \qquad \iff \qquad \overline{E}\rb{a_1, a_2} \ \text{in} \ \AA \qquad \iff \qquad \widehat{a_1} = \widehat{a_2}. $$
So $ \widehat{\AA} $ is a normal $ \LL^E $-structure.
\end{note*}

\begin{lemma}
\label{lem:2.6.3}
Suppose $ \AA $ is an $ \LL^E $-structure with $ \AA \vDash \Sigma_E $. Let $ v $ be a valuation in $ \AA $. Let $ \widehat{\AA} $ be as given above. Let $ \widehat{v} $ be the valuation in $ \widehat{\AA} $ with
$$ \widehat{v}\rb{x_i} = \widehat{v\rb{x_i}}. $$
Then for every $ \LL^E $-formula $ \phi $,
$$ \widehat{v} \ \text{satisfies} \ \phi \ \text{in} \ \widehat{\AA} \qquad \iff \qquad v \ \text{satisfies} \ \phi \ \text{in} \ \AA. $$
In particular, if $ \phi $ is closed, then
$$ \AA \vDash \phi \qquad \iff \qquad \widehat{\AA} \vDash \phi. $$
\end{lemma}

\begin{note*}
If $ t $ is any term, then $ \widehat{v}\rb{t} = \widehat{v\rb{t}} $, by definition of $ \overline{f} $ on the structure $ \widehat{\AA} $.
\end{note*}

\begin{proof}[Proof of \ref{lem:2.6.3}]
Appendix A.3.
\end{proof}

\begin{lemma}
\label{lem:2.6.4}
Suppose $ \Delta $ is a set of closed $ \LL^E $-formulas. Then $ \Delta $ has a \textbf{normal model}, that is a normal $ \LL^E $-structure $ \BB $ with $ \BB \vDash \sigma $ for all $ \sigma \in \Delta $, if and only if $ \Delta \cup \Sigma_E $ has a model.
\end{lemma}

\begin{proof}
\hfill
\begin{itemize}
\item[$ \implies $] Trivial, as $ \Sigma_E $ holds in a normal $ \LL^E $-structure.
\item[$ \impliedby $] If $ \AA \vDash \Delta \cup \Sigma_E $, then by \ref{lem:2.6.3}, $ \widehat{\AA} \vDash \Delta $, and $ \widehat{\AA} $ is a normal $ \LL^E $-structure.
\end{itemize}
\end{proof}

\begin{theorem}[Compactness theorem for normal models]
Suppose $ \LL^E $ is a countable language with equality, and $ \Delta $ is a set of closed $ \LL^E $-formulas such that every finite subset of $ \Delta $ has a normal model. Then $ \Delta $ has a normal model.
\end{theorem}

\begin{proof}
Every normal $ \LL^E $-structure is a model of $ \Sigma_E $, so every finite subset of $ \Delta \cup \Sigma_E $ has a model. By \ref{cor:2.5.6}, $ \Delta \cup \Sigma_E $ has a model $ \AA $. Then by \ref{lem:2.6.3} or \ref{lem:2.6.4}, $ \widehat{\AA} $ is a normal model of $ \Delta $.
\end{proof}

\begin{notation*}
From now on, write $ \LL^= $ instead of $ \LL^E $, and $ x_1 = x_2 $ instead of $ E\rb{x_1, x_2} $, etc.
\end{notation*}

\pagebreak

\lecture{18}{Monday}{12/11/18}

\begin{theorem}[Countable downward L\"owenheim-Skolem theorem]
\label{thm:2.6.6}
Suppose $ \LL^= $ is a countable first-order language with equality, and $ \BB $ is a normal $ \LL^= $ structure. Then there is a countable normal $ \LL^= $-structure $ \AA $ such that, for every closed $ \LL^= $-formula $ \phi $,
$$ \BB \vDash \phi \qquad \iff \qquad \AA \vDash \phi. $$
\end{theorem}

\begin{example*}
Let $ \BB = \ab{\R; +, \cdot, 0, 1, \le, \exp\rb{}} $. What is $ \AA $?
\end{example*}

\begin{proof}
Let
$$ \Sigma = \cb{\text{closed} \ \phi \mid \BB \vDash \phi}, $$
called the \textbf{theory} of $ \BB $. Then $ \Sigma \supseteq \Sigma_E $, the axioms of equality, and $ \Sigma $ is consistent. By \ref{thm:2.5.3}, $ \Sigma $ has a countable model $ \CC $. Then $ \widehat{\CC} $ is a countable normal model of $ \Sigma $ by \ref{lem:2.6.3}. So if $ \phi $ is closed and $ \BB \vDash \phi $, then $ \widehat{\CC} \vDash \phi $. Conversely, if $ \phi $ is closed and $ \BB \not\vDash \phi $, then $ \BB \vDash \notb{\phi} $, by \ref{thm:2.3.3}, so $ \widehat{\CC} \vDash \notb{\phi} $, so $ \widehat{\CC} \not\vDash \phi $. Take
$$ \AA = \widehat{\CC}. $$
\end{proof}

\subsection{Examples and applications}

Let $ \LL^= $ be a first-order language with equality and a binary relation symbol $ \le $.

\begin{definition}
\hfill
\begin{itemize}
\item A \textbf{linear order} $ \AA = \ab{A; \le_A} $ is a normal model of
\begin{align*}
\phi_1 & : \fab{x_1}\fab{x_2}\iffb{\andb{\leb{x_1}{x_2}}{\leb{x_2}{x_1}}}{\eqb{x_1}{x_2}}, \\
\phi_2 & : \fab{x_1}\fab{x_2}\fab{x_3}\impb{\andb{\leb{x_1}{x_2}}{\leb{x_2}{x_3}}}{\leb{x_1}{x_3}}, \\
\phi_3 & : \fab{x_1}\fab{x_2}\orb{\leb{x_1}{x_2}}{\leb{x_2}{x_1}}.
\intertext{\item It is \textbf{dense} if also}
\phi_4 & : \fab{x_1}\fab{x_2}\teb{x_3}\impb{\ltb{x_1}{x_2}}{\andb{\ltb{x_1}{x_3}}{\ltb{x_3}{x_2}}},
\intertext{where $ \ltb{x_1}{x_2} $ is an abbreviation for $ \andb{\leb{x_1}{x_2}}{\neb{x_1}{x_2}} $.}
\intertext{\item It is \textbf{without endpoints} if}
\phi_5 & : \fab{x_1}\teb{x_2}\ltb{x_1}{x_2}, \\
\phi_6 & : \fab{x_1}\teb{x_2}\ltb{x_2}{x_1}.
\end{align*}
\item Let
$$ \Delta = \cb{\phi_1, \dots, \phi_6}. $$
\begin{itemize}
\item $ \QQ = \ab{\Q; \le} $ is a normal model of $ \Delta $.
\item $ \RR = \ab{\R; \le} $ is also a model of $ \Delta $.
\end{itemize}
\end{itemize}
\end{definition}

Will prove the following.

\begin{theorem}
\label{thm:2.7.2}
\hfill
\begin{enumerate}
\item For every closed $ \LL^= $-formula $ \phi $,
$$ \QQ \vDash \phi \qquad \iff \qquad \RR \vDash \phi. $$
\item There is an algorithm which decides, given a closed $ \LL^= $-formula $ \phi $, whether $ \QQ \vDash \phi $ or $ \QQ \not\vDash \phi $, that is $ \QQ \vDash \notb{\phi} $, by \ref{thm:2.3.3}.
\end{enumerate}
\end{theorem}

\pagebreak

\begin{definition}
\label{def:2.7.3}
Week 5 problem class.
\begin{enumerate}
\item Linear orders $ \AA = \ab{A; \le_A} $ and $ \BB = \ab{B; \le_B} $ are \textbf{isomorphic} if there is a bijection $ \alpha : A \to B $ such that for all $ a, a' \in A $,
$$ a \le_A a' \qquad \iff \qquad \alpha\rb{a} \le_B \alpha\rb{a'}. $$
\item If $ \AA, \BB $ are isomorphic and $ \phi $ is closed, then
$$ \AA \vDash \phi \qquad \iff \qquad \BB \vDash \phi. $$
\end{enumerate}
\end{definition}

\begin{theorem}[Cantor]
\label{thm:2.7.4}
If $ \AA, \BB $ are countable dense linear orders without endpoints, then $ \AA, \BB $ are isomorphic.
\end{theorem}

\begin{lemma}[Los-Vaught test]
\label{lem:2.7.5}
Let $ \Sigma = \Sigma_E \cup \Delta $. Then for every closed $ \LL^= $-formula $ \phi $, we have either $ \Sigma \vdash_{K_{\LL^=}} \phi $ or $ \Sigma \vdash_{K_{\LL^=}} \notb{\phi} $. Say $ \Sigma $ is complete.
\end{lemma}

\begin{proof}
Suppose not. Then as $ \Sigma $ has a model, it is consistent, so we can use \ref{prop:2.5.2} to get that
$$ \Sigma_1 = \Sigma \cup \cb{\notb{\phi}}, \qquad \Sigma_2 = \Sigma \cup \cb{\notb{\notb{\phi}}} $$
are consistent. So $ \Sigma \cup \cb{\phi} $ is consistent. By \ref{thm:2.5.3}, \ref{thm:2.6.6}, it follows that $ \Sigma_1, \Sigma_2 $ have countable normal models $ \AA_1, \AA_2 $. So $ \AA_1, \AA_2 $ are countable dense linear orders without endpoints, and
$$ \AA_1 \vDash \notb{\phi}, \qquad \AA_2 \vDash \phi. $$
This contradicts \ref{thm:2.7.4} and \ref{def:2.7.3}.$ 2 $.
\end{proof}

\begin{proof}[Proof of \ref{thm:2.7.2}]
\hfill
\begin{enumerate}
\item Show that
$$ \QQ \vDash \phi \qquad \iff \qquad \Sigma \vdash_{K_{\LL^=}} \phi. $$
\begin{itemize}
\item[$ \impliedby $] As $ \QQ \vDash \Sigma $, this is \ref{eg:2.4.6}.
\item[$ \implies $] If $ \Sigma \not\vdash_{K_{\LL^=}} \phi $, then by \ref{lem:2.7.5}, $ \Sigma \vdash_{K_{\LL^=}} \notb{\phi} $. So $ \QQ \vDash \notb{\phi} $, so $ \QQ \not\vDash \phi $.
\end{itemize}
Similarly,
$$ \RR \vDash \phi \qquad \iff \qquad \Sigma \vdash_{K_{\LL^=}} \phi. $$
So
$$ \RR \vDash \phi \qquad \iff \qquad \Sigma \vdash_{K_{\LL^=}} \phi \qquad \iff \qquad \QQ \vDash \phi. $$
\item $ \Sigma $ is a \textbf{recursively enumerable} set of formulas. That is, we can write an algorithm to systematically generate the formulas in $ \Sigma $. Note that the set of axioms for $ K_\LL $ is also recursively enumerable. So the set of deductions from $ \Sigma $ is recursively enumerable. Therefore the set of consequences of $ \Sigma $ is recursively enumerable. Method is to run the algorithm which generates all consequences of $ \Sigma $. By $ 1 $, at some point, we will see either $ \theta $ or $ \notb{\theta} $. At this point, the method stops.
\end{enumerate}
\end{proof}

\lecture{19}{Thursday}{15/11/18}

Lecture 19 is a problem class.

\lecture{20}{Friday}{16/11/18}

\begin{note*}
\hfill
\begin{itemize}
\item Depends on
\begin{itemize}
\item the completeness theorem, and
\item the axioms $ \Delta $ for $ \QQ $.
\end{itemize}
\item Works for some other structures, but can have better algorithms.
\item But, there is no such algorithm for
$$ \ab{\N; +, \cdot, 0}. $$
This is G\"odel's incompleteness theorem.
\end{itemize}
\end{note*}

\pagebreak

\section{Set theory}

\setcounter{subsection}{-1}

\subsection{Basic set theory}

\begin{itemize}
\item Extensionality. Sets $ A, B $ are \textbf{equal} if and only if
$$ \fab{x}\iffb{\inb{x}{A}}{\inb{x}{B}}. $$
\item \textbf{Natural numbers} $ \N = \cb{0, 1, \dots} $. Think of
$$ 0 = \emptyset, \qquad \dots, \qquad n + 1 = \cb{0, \dots, n}, \qquad \dots. $$
\begin{itemize}
\item Note that, for $ m, n \in \N $,
$$ m < n \qquad \iff \qquad m \in n \qquad \iff \qquad m \subsetneq n. $$
\end{itemize}
\item Ordered pairs. The \textbf{ordered pair} $ \rb{x, y} $ is the set
$$ \cb{\cb{x}, \cb{x, y}}. $$
\begin{itemize}
\item For example, for any $ x, y, z, w $,
$$ \rb{x, y} = \rb{z, w} \qquad \iff \qquad \andb{\eqb{x}{z}}{\eqb{y}{w}}. $$
\item If $ A, B $ are sets, then
$$ A \times B = \cb{\rb{a, b} \mid a \in A, \ b \in B}, $$
$$ A^0 = \cb{\emptyset}, \qquad A^1 = A, \qquad A^2 = A \times A, \qquad \dots, \qquad A^{n + 1} = A^n \times A, \qquad \dots, $$
$$ \bigcup_{n \in \N} A^n = \cb{\text{finite sequences of elements of} \ A}. $$
\end{itemize}
\item Functions. Think of a \textbf{function} $ f : A \to B $ as a subset of $ A \times B $.
\begin{itemize}
\item $ A = dom\rb{f} $ is the \textbf{domain}, and
\item $ B = ran\rb{f} $ is the \textbf{range}.
\item If $ X \subseteq A $,
$$ f\sb{X} = \cb{f\rb{a} \mid a \in X} \subseteq B. $$
\item Set of functions from $ A $ to $ B $ is
$$ B^A \subseteq \PP\rb{A \times B}, $$
where $ \PP $ is the powerset, that is the set of all subsets.
\end{itemize}
\end{itemize}

\subsection{Cardinality}

\begin{definition}
Sets $ A, B $ are \textbf{equinumerous}, or of the \textbf{same cardinality}, if there is a bijection
$$ f : A \to B. $$
Write $ A \approx B $ or $ \abs{A} = \abs{B} $.
\end{definition}

\begin{definition}
\hfill
\begin{itemize}
\item A set $ A $ is \textbf{finite} if it is equinumerous with some element $ n = \cb{0, \dots, n - 1} $ of $ \N $.
\item A set $ A $ is \textbf{countably infinite} if it is equinumerous with $ \N $.
\item \textbf{Countable} is finite or countably infinite.
\end{itemize}
\end{definition}

\pagebreak

\begin{remark}
Basic facts.
\begin{itemize}
\item Every subset of a countable set is countable.
\item A set $ A $ is countable if and only if there is an injective function
$$ f : A \to \N. $$
\item If $ A, B $ are countable, then
$$ A \times B $$
is countable.
\item If $ A_0, A_1, \dots $ are countable, then
$$ \bigcup_{i \in \N} A_i $$
is countable. Proof uses axiom of choice.
\end{itemize}
\end{remark}

\begin{example*}
\hfill
\begin{itemize}
\item $ \Q $ is countable.
\item $ \bigcup_{n \in \N} A^n $ is countable if $ A $ is countable.
\item $ \R $ is not countable, by Cantor.
\end{itemize}
\end{example*}

If $ X $ is any set, then $ \PP\rb{X} $ is the set of all subsets of $ X $.

\begin{theorem}[Cantor]
\label{thm:3.1.4}
There is no surjective function
$$ f : X \to \PP\rb{X}. $$
\end{theorem}

\begin{proof}
Suppose there is such an $ f $. Let
$$ Y = \cb{y \in X \mid y \notin f\rb{y}}. $$
There is $ z \in X $ with $ f\rb{z} = Y $.
\begin{itemize}
\item If $ z \in Y $, then $ z \notin f\rb{z} = Y $, a contradiction.
\item If $ z \notin Y $, then $ z \notin f\rb{z} $, so $ z \in Y $, a contradiction.
\end{itemize}
\end{proof}

\lecture{21}{Monday}{19/11/18}

\begin{definition}
For sets $ A, B $, write $ \abs{A} \le \abs{B} $, or $ A \le B $, if there is an injective function
$$ f : A \to B. $$
\end{definition}

\begin{note*}
$ \abs{X} \le \abs{\PP\rb{X}} $, by using
$$ x \mapsto \cb{x}. $$
So, as $ \abs{X} \ne \abs{\PP\rb{X}} $, by \ref{thm:3.1.4}, we have
$$ \abs{X} < \abs{\PP\rb{X}}. $$
\end{note*}

\begin{example*}
If $ \abs{A} \le \abs{B} $ and $ \abs{B} \le \abs{C} $, then $ \abs{A} \le \abs{C} $.
\end{example*}

\begin{theorem}[Schr\"oder-Bernstein]
\label{thm:3.1.6}
Suppose $ A, B $ are sets, and
$$ f : A \to B, \qquad g : B \to A $$
are injective functions. Then $ A \approx B $. That is, if $ \abs{A} \le \abs{B} $ and $ \abs{B} \le \abs{A} $, then $ \abs{A} = \abs{B} $.
\end{theorem}

\pagebreak

\begin{proof}
Let $ h = g \circ f : A \to A $. Let $ A_0 = A \setminus g\sb{B} $, and for $ n > 0 $, let $ A_n = h\sb{A_{n - 1}} $. Let
$$ A^* = \bigcup_{n \in \N} A_n, \qquad B^* = f\sb{A^*}. $$
Note that $ h\sb{A^*} \subseteq A^* $, so $ g\sb{B^*} = h\sb{A^*} \subseteq A^* $. Claim that
$$ g\sb{B \setminus B^*} = A \setminus A^*. $$
Once we have this, $ f $ gives a bijection $ A^* \to B^* $, and $ g $ gives a bijection $ B \setminus B^* \to A \setminus A^* $. So
$$ k\rb{a} =
\begin{cases}
f\rb{a} & a \in A^* \\
g^{-1}\rb{a} & a \in A \setminus A^*
\end{cases}
$$
is a bijection.
\begin{itemize}
\item Let $ a \in A \setminus A^* $. As $ a \notin A_0 $, there is $ b \in B $ with $ g\rb{b} = a $. Then $ b \notin B^* $, as
$$ b \in B^* \qquad \implies \qquad b \in f\sb{A^*} \qquad \implies \qquad g\rb{b} \in h\sb{A^*} \subseteq A^* \qquad \implies \qquad a \in A^*, $$
a contradiction. Thus $ g\sb{B \setminus B^*} \supseteq A \setminus A^* $.
\item Let $ b \in B $. Suppose $ g\rb{b} \in A^* $. Show that $ b \in B^* $. As $ g\rb{b} \notin A_0 = A \setminus g\sb{B} $, we have $ g\rb{b} \in A_n $ for some $ n > 0 $. So $ g\rb{b} = h\rb{a} $, for some $ a \in A_{n - 1} $. So $ g\rb{b} = g\rb{f\rb{a}} $. Thus $ b = f\rb{a} $ for some $ a \in A^* $. Thus $ b \in f\sb{A^*} = B^* $.
\end{itemize}
\end{proof}

\begin{example}
The following sets are equinumerous.
\begin{align*}
S_1 & = \cb{\text{set of all sequences of zeroes and ones}} = \cb{0, 1}^\N \\
S_2 & = \R \\
S_3 & = \PP\rb{\N} \\
S_4 & = \PP\rb{\N \times \N} \\
S_5 & = \cb{\text{set of all sequences of natural numbers}} = \N^\N
\end{align*}
\end{example}

\begin{proof}
Find injective functions
$$ f_{i, j} : S_i \to S_j, \qquad i, j = 1, \dots, 5. $$
Then use \ref{thm:3.1.6}.
\begin{itemize}
\item As $ \N \times \N \approx \N $, we get $ S_3 \approx S_4 $.
\item Also $ S_1 \subseteq S_5 \subseteq S_4 $.
\item $ f_{3, 1} : \PP\rb{\N} \to \cb{0, 1}^\N $.
$$ X \subseteq \N \mapsto \rb{a_n}_{n \in \N}, \qquad a_n =
\begin{cases}
0 & n \notin X \\
1 & n \in X
\end{cases}
$$
\item $ f_{1, 2} : \cb{0, 1}^\N \to \R $.
$$ \rb{a_n}_{n \in \N} \mapsto 0 . a_1a_2 \dots $$
\item $ f_{2, 5} : \R \to \N^\N $.
$$ n . m_1m_2 \dots \mapsto
\begin{cases}
\rb{0, n, m_1, m_2, \dots} & n \ge 0 \\
\rb{1, -n, m_1, m_2, \dots} & n < 0
\end{cases}
$$
\end{itemize}
\end{proof}

Questions.
\begin{itemize}
\item If $ A, B $ are sets, do we have $ \abs{A} \le \abs{B} $ or $ \abs{B} \le \abs{A} $? If we assume axiom of choice, yes.
\item Is there $ X \subseteq \R $ with $ \abs{\N} < \abs{X} < \abs{\R} $? \textbf{Continuum hypothesis} says no.
\end{itemize}

\pagebreak

\subsection{Axioms for set theory}

\textbf{Zermelo-Fraenkel axioms (ZF)} say how we are allowed to build sets. All can be expressed in a first-order language, with equality, using a single binary relation symbol $ \in $. Have to avoid \textbf{Russell paradox}. Let
$$ S = \cb{x \mid x \ \text{is a set and} \ x \notin x}. $$
If this is a set, is $ S \in S $?
$$ \teb{S}\fab{x}\iffb{\inb{x}{S}}{\nib{x}{x}} $$
leads to inconsistency.

\begin{axiom}[Axiom of extensionality]
Two sets are equal if and only if they have the same elements.
$$ \fab{x}\fab{y}\iffb{\eqb{x}{y}}{\fab{z}\iffb{\inb{z}{x}}{\inb{z}{y}}} $$
\end{axiom}

\begin{axiom}[Empty set axiom]
$$ \teb{x}\fab{y}\nib{y}{x} $$
There is a unique set $ x $ with this property, $ \emptyset $.
\end{axiom}

\lecture{22}{Thursday}{22/11/18}

\begin{axiom}[Pairing axiom]
Given sets $ x, y $, then we can form $ z = \cb{x, y} $.
$$ \fab{x}\fab{y}\teb{z}\fab{w}\iffb{\inb{w}{z}}{\orb{\eqb{w}{x}}{\eqb{w}{y}}} $$
\end{axiom}

\begin{remark*}
\hfill
\begin{itemize}
\item Use twice to form
$$ \rb{x, y} = \cb{\cb{x}, \cb{x, y}}. $$
\item Use to produce
$$ 0 = \emptyset, \qquad 1 = \cb{\emptyset}, \qquad 2 = \cb{0, 1}. $$
\end{itemize}
\end{remark*}

\begin{axiom}[Union axiom]
For any set $ A $, there is a set $ B = \bigcup A $.
$$ \fab{A}\teb{B}\fab{x}\iffb{\inb{x}{B}}{\teb{z}\andb{\inb{z}{A}}{\inb{x}{z}}} $$
So
$$ B = \bigcup \cb{z \mid z \in A}. $$
If $ A = \cb{x, y} $, then $ B = x \cup y $.
\end{axiom}

\begin{example*}
$$ 3 = \cb{0, 1, 2} = \cb{0, 1} \cup \cb{2}. $$
\end{example*}

\begin{axiom}[Power set axiom]
For any set $ A $, there is a set $ \PP\rb{A} $ whose elements are the subsets of $ A $.
$$ \fab{A}\teb{B}\fab{z}\iffb{\inb{z}{B}}{\subb{z}{A}}, $$
where $ z \subseteq A $ means
$$ \fab{y}\impb{\inb{y}{z}}{\inb{y}{A}}. $$
\end{axiom}

\pagebreak

\begin{axiom}[Axiom scheme of specification]
Suppose $ P\rb{x, y_1, \dots, y_r} $ is a formula in our language. Then we have an axiom
$$ \fab{A}\fab{y_1} \dots \fab{y_r}\teb{B}\fab{x}\iffb{\inb{x}{B}}{\andb{\inb{x}{A}}{P\rb{x, y_1, \dots, y_r}}}. $$
So this guarantees we can form the subset of $ A $,
$$ B = \cb{x \in A \mid P\rb{x, y_1, \dots, y_r} \ \text{holds}}, $$
for all given sets $ A, y_1, \dots, y_r $.
\end{axiom}

\begin{example*}
\hfill
\begin{itemize}
\item Let $ C $ be any non-empty set and $ A \in C $. Then
$$ \bigcap C = \cb{x \in A \mid P\rb{x, C} = \fab{z}\impb{\inb{z}{C}}{\inb{x}{z}}}. $$
\item
$$ A \times B = \cb{w \in \PP\rb{\PP\rb{A \cup B}} \mid \teb{a}\teb{b}\rb{\inb{a}{A} \land \inb{b}{B} \land \eqb{w}{\cb{\cb{a}, \cb{a, b}}}}}. $$
\end{itemize}
\end{example*}

\begin{definition}
For a set $ a $, the \textbf{successor} of $ a $ is
$$ a^\dagger = a \cup \cb{a}. $$
A set $ A $ is \textbf{inductive} if
$$ \andb{\inb{\emptyset}{A}}{\fab{x}\impb{\inb{x}{A}}{\inb{x^\dagger}{A}}}. $$
\end{definition}

\begin{example*}
$$ 2 = \cb{0, 1} = \cb{0} \cup \cb{1} = 1^\dagger. $$
\end{example*}

\begin{axiom}[Axiom of infinity]
$$ \teb{A}\andb{\inb{\emptyset}{A}}{\fab{x}\impb{\inb{x}{A}}{\inb{x^\dagger}{A}}} $$
\end{axiom}

\begin{definition}
Let $ A $ be an inductive set. We can form, using specification, the set
$$ \N = \cb{x \in A \mid \text{if} \ B \ \text{is an inductive set, then} \ x \in B}. $$
Informally, this is the intersection of all inductive sets. This does not depend on the choice of $ A $. Also denote this by $ \omega $.
\end{definition}

\begin{theorem}
\label{thm:3.2.3}
\hfill
\begin{enumerate}
\item $ \N $ is an inductive set, and if $ B $ is an inductive set, then $ \N \subseteq B $.
\item Proof by induction works. Suppose $ P\rb{x} $ is a property of sets, that is a formula, such that
\begin{itemize}
\item $ P\rb{\emptyset} $ holds, and
\item for every $ k \in \N $, if $ P\rb{k} $ holds, then $ P\rb{k^\dagger} $ holds.
\end{itemize}
Then $ P\rb{n} $ holds for all $ n \in \N $.
\end{enumerate}
\end{theorem}

\begin{proof}
\hfill
\begin{enumerate}
\item Exercise. Use definition.
\item Consider $ B \subseteq \N $ given by
$$ B = \cb{x \in \N \mid P\rb{x} \ \text{holds}}. $$
(a), (b) say that $ B $ is an inductive set. So by $ 1 $, $ \N \subseteq B $. Thus $ \N = B $.
\end{enumerate}
\end{proof}

Could develop arithmetic in $ \N $, using $ n^\dagger $ as $ n + 1 $, etc, and in $ \Z, \Q, \R $, in the usual way, using ZF1-7.

\begin{example*}
Exercise. Hard. For $ m, n \in \N $, write $ m \le n $ to mean
$$ \orb{\eqb{m}{n}}{\inb{m}{n}}. $$
This is a well ordering on $ \N $.
\end{example*}

\pagebreak

\subsection{Well orderings}

\begin{definition}
A linear ordering $ \ab{A; \le} $ is a \textbf{well ordering}, or a \textbf{woset}, if every non-empty subset of $ A $ has a least element.
$$ \fab{X}\impb{\andb{\subb{X}{A}}{\neb{X}{\emptyset}}}{\teb{x}\andb{\inb{x}{X}}{\fab{y}\impb{\inb{y}{X}}{\inb{x}{y}}}} $$
\end{definition}

\begin{example*}
\hfill
\begin{itemize}
\item $ \ab{\Z; \le} $ is not a well ordering.
\item $ \ab{\N; \le} $ is a well ordering.
\end{itemize}
\end{example*}

\lecture{23}{Friday}{23/11/18}

Suppose $ \AA_1 = \ab{A_1; \le_1} $ and $ \AA_2 = \ab{A_2; \le_2} $ are linearly ordered sets.

\begin{definition}
\hfill
\begin{itemize}
\item Say $ \AA_1, \AA_2 $ are \textbf{similar}, or isomorphic, if there is a bijection
$$ \alpha : A_1 \to A_2, $$
with for all $ a, b \in A_1 $,
$$ a <_1 b \qquad \iff \qquad \alpha\rb{a} <_2 \alpha\rb{b}. $$
Write $ \AA_1 \simeq \AA_2 $. Say $ \alpha $ is a \textbf{similarity}.
\item If
$$ a <_1 b \qquad \implies \qquad \alpha\rb{a} <_2 \alpha\rb{b}, $$
say $ \alpha $ is \textbf{order-preserving}.
\end{itemize}
\end{definition}

\begin{definition}
\hfill
\begin{itemize}
\item The \textbf{reverse-lexicographic product}
$$ \AA_1 \times \AA_2 = \ab{A_1 \times A_2; \le}, $$
where
$$ \rb{a_1, a_2} \le \rb{a_1', a_2'} \qquad \iff \qquad a_2 <_2 a_2' \ \text{or} \ \rb{\rb{a_2 = a_2'} \ \text{and} \ \rb{a_1 \le_1 a_1'}}. $$
In $ \AA_2 $, replace each element by a copy of $ \AA_1 $.
\item \textbf{Sum}. Regard $ A_1, A_2 $ as disjoint, by replacing them by similar orderings on disjoint sets, such as
$$ \cb{\rb{a, 0} : a \in A_1}, \qquad \cb{\rb{b, 1} \mid b \in A_2}, $$
and define
$$ \AA_1 + \AA_2 = \ab{A_1 \cup A_2; \le}, $$
where $ \le $ is the union of $ \le_1, \le_2 $, and
$$ a_1 \le a_2, \qquad a_1 \in A_1, \qquad a_2 \in A_2. $$
\end{itemize}
\end{definition}

\begin{example*}
\hfill
\begin{itemize}
\item $ \N + \N $.
$$
\begin{array}{ccccccccccc}
\N & & & \qquad & 0 & 1 & \dots & \quad & & & \\
& & \N & \qquad & & & & \quad & 0 & 1 & \dots \\
\N & + & \N & \qquad & 0 & 1 & \dots & \quad & 0 & 1 & \dots
\end{array}
$$
\item $ \cb{0, 1} \times \N $.
$$
\begin{array}{ccccccccccccccc}
& & \N & \qquad & & 0 & & \quad & & 1 & & \quad & & \dots & \\
\cb{0, 1} & \times & \N & \qquad & 0 & & 1 & \quad & 0 & & 1 & \quad & \dots & & \dots
\end{array}
$$
Thus $ \cb{0, 1} \times \N \simeq \N $.
\item $ \N \times \cb{0, 1} $.
$$
\begin{array}{ccccccccccc}
& & \cb{0, 1} & \qquad & & 0 & & \quad & & 1 & \\
\N & \times & \cb{0, 1} & \qquad & 0 & 1 & \dots & \quad & 0 & 1 & \dots
\end{array}
$$
Thus $ \N \times \cb{0, 1} \simeq \N + \N $.
\end{itemize}
\end{example*}

\pagebreak

\begin{lemma}
\hfill
\begin{itemize}
\item $ \AA_1 + \AA_2, \AA_1 \times \AA_2 $ are linearly ordered sets.
\item If $ \AA_1, \AA_2 $ are well orderings, then so are $ \AA_1 + \AA_2 $ and $ \AA_1 \times A_2 $.
\end{itemize}
\end{lemma}

\begin{proof}
Exercise. For example, $ \AA_1 \times \AA_2 $ is a well ordering. Let
$$ \emptyset \ne X \subsetneq A_1 \times A_2. $$
Let
$$ Y = \cb{b \in A_2 \mid \text{there is} \ a \in A_1 \ \text{with} \ \rb{a, b} \in X} \subseteq A_2. $$
Let $ y $ be the least element of $ Y $. Let
$$ Z = \cb{z \in A_1 \mid \rb{z, y} \in X}. $$
This has a least element $ x $. Then $ \rb{x, y} $ is the least element of $ X $.
\end{proof}

\begin{definition}
\hfill
\begin{itemize}
\item Suppose $ \AA = \ab{A; \le} $ is a linearly ordered set. A subset $ X \subseteq A $ is an \textbf{initial segment} of $ A $ if
$$ \fab{x \in X}\fab{a \in A}\impb{\ltb{a}{x}}{\inb{a}{X}} $$
\item It is \textbf{proper} if $ X \ne A $.
\end{itemize}
\end{definition}

\begin{example*}
Let $ b \in A $.
\begin{itemize}
\item A proper initial segment is
$$ A\sb{b} = \cb{x \in A \mid x < b}. $$
\item An initial segment is
$$ A\sb{\le b} = \cb{x \in A \mid x \le b}. $$
\end{itemize}
\end{example*}

\begin{lemma}
\label{lem:3.3.6}
If $ \AA = \ab{A; \le} $ is a well ordered set, then every proper initial segment $ X $ is of the form $ A\sb{b} $ for some $ b \in A $.
\end{lemma}

\begin{remark*}
Not true for linear orders in general.
\end{remark*}

\begin{example*}
$$ \cb{x \in \Q \mid x < \pi} \subseteq \Q $$
is not of the form $ \Q\sb{b} $ for some $ b \in \Q $.
\end{example*}

\begin{proof}
Let $ b $ be the minimal element of $ A \setminus X $.
\end{proof}

\begin{theorem}
\label{thm:3.3.7}
Suppose $ \AA_1 = \ab{A_1; \le_1} $ and $ \AA_2 = \ab{A_2; \le_2} $ are well ordered sets. Then exactly one of the following holds.
\begin{enumerate}
\item $ \AA_1, \AA_2 $ are similar,
\item $ \AA_1 $ is similar to a proper initial segment of $ \AA_2 $, or
\item $ \AA_2 $ is similar to a proper initial segment of $ \AA_1 $.
\end{enumerate}
In each case, the similarity is unique.
\end{theorem}

\pagebreak

\begin{proof}
\hfill
\begin{itemize}
\item Uniqueness. Suppose we have order preserving
$$ \alpha, \beta : A_1 \to A_2, $$
whose images are initial segments of $ A_2 $. Show that $ \alpha = \beta $.
\begin{enumerate}[leftmargin=0.5in, label=Step \arabic*.]
\item Check that, if $ b \in A_1 $, then
$$ \alpha\rb{A_1\sb{b}} = A_2\sb{\alpha\rb{b}}. $$
\item If $ \alpha \ne \beta $, take $ b \in A_1 $ minimal with $ \alpha\rb{b} \ne \beta\rb{b} $, so
$$ \alpha \mid_{A_1\sb{b}} = \beta \mid_{A_1\sb{b}}. $$
By step $ 1 $,
$$ A_2\sb{\alpha\rb{b}} = A_2\sb{\beta\rb{b}}. $$
Conclude that $ \alpha\rb{b} = \beta\rb{b} $, a contradiction.
\end{enumerate}
This shows
\begin{itemize}
\item by taking $ A_1 = A_2 $ and $ \alpha $ the identity, obtain that $ A_1 $ is not similar to a proper initial segment of itself, and
\item it follows that we cannot have two of $ 1 $, $ 2 $, $ 3 $ holding.
\end{itemize}
\item Existence. Suppose $ A_2 $ is not similar to an initial segment of $ A_1 $. Show that $ A_1 $ is similar to a proper initial segment of $ A_2 $. Look at
$$ C = \cb{c \in A_1 \mid \text{there is a similarity between} \ A_1\sb{\le c} \ \text{and an initial segment of} \ A_2}. $$
If $ c \in C $, there is a unique
$$ \alpha_c : A_1\sb{\le c} \to A_2, $$
with image an initial segment, by uniqueness part. Note that
\begin{itemize}
\item $ C $ is an initial segment of $ A_1 $,
\item if $ c_1 < c_2 \in C $, then $ \alpha_{c_1} $ is the restriction of $ \alpha_{c_2} $ to $ A_1\sb{\le c_1} $, and
\item let
$$ \alpha = \bigcup \cb{\alpha_c \mid c \in C}, $$
then $ \alpha $ is a similarity between $ C $ and an initial segment of $ A_2 $.
\end{itemize}
If $ C = A_1 $, done. Suppose $ C \ne A_1 $. Let $ a $ be the minimal element of $ A_1 \setminus C $.
$$ \alpha\rb{C} \ne A_2, $$
otherwise $ A_2 $ is similar to $ C $, so $ 3 $ holds. So $ \alpha\rb{C} = A_2\sb{b} $ for some $ b \in A_2 $. Can extend $ \alpha $ by sending $ a $ to $ b $, and get
$$ A_1\sb{\le a} \simeq A_2\sb{\le b}. $$
Thus $ a \in C $, a contradiction.
\end{itemize}
\end{proof}

\lecture{24}{Monday}{26/11/18}

\begin{remark*}
In the notation of \ref{thm:3.3.7}, we have, in particular, an injective function
$$ A_1 \to A_2, $$
for cases $ 1 $ and $ 2 $, or an injective function
$$ A_2 \to A_1. $$
So either $ \abs{A_1} \le \abs{A_2} $ or $ \abs{A_2} \le \abs{A_1} $.
\end{remark*}

Axiom of choice gives that any set $ A $ can be well ordered. Putting this together with \ref{thm:3.3.7} gives that, for any sets $ A_1, A_2 $, $ \abs{A_1} \le \abs{A_2} $ or $ \abs{A_2} \le \abs{A_1} $.

\pagebreak

\subsection{Ordinals}

We now define some very particular well ordered sets which generalise the natural numbers,
$$ 0 = \emptyset, \qquad 1 = 0^\dagger = \cb{\emptyset}, \qquad 2 = 1^\dagger = \cb{0, 1}, \qquad 3 = 2^\dagger = \cb{0, 1, 2}, \qquad \dots. $$
to the transfinite.

\begin{definition}
\hfill
\begin{itemize}
\item A set $ X $ is a \textbf{transitive set} if every element of $ X $ is also a subset of $ X $. That is, if $ y \in x \in X $, then $ y \in X $.
\item A set $ \alpha $ is an \textbf{ordinal} if
\begin{itemize}
\item $ \alpha $ is a transitive set, and
\item the relation $ < $ on $ \alpha $ given by, for $ x, y \in \alpha $,
$$ x < y \qquad \iff \qquad x \in y $$
is a strict well ordering on $ \alpha $.
\end{itemize}
\end{itemize}
\end{definition}

\begin{note*}
As part of the definition, we have $ \alpha \notin \alpha $ for an ordinal $ \alpha $.
\end{note*}

It is standard to use lower case Greek letters for ordinals. The ordering on an ordinal $ \alpha $ is sometimes denoted by $ \in_\alpha $.

\begin{lemma}
\label{lem:3.4.2}
If $ \alpha $ is an ordinal, then so is $ \alpha^\dagger = \alpha \cup \cb{\alpha} $.
\end{lemma}

\begin{proof}
As $ \alpha $ is a transitive set, then so is $ \alpha^\dagger = \alpha \cup \cb{\alpha} $. Moreover, the ordering $ \in $ restricted to $ \alpha^\dagger $ is an ordering. We have the ordering on $ \alpha $ with the extra element $ \alpha $ added on the end as the greatest element. This is a well ordering, as the ordering on $ \alpha $ is a well ordering.
\end{proof}

We also denote the ordinal $ \alpha^\dagger $ by $ \alpha + 1 $. It is of course similar to the ordered set $ \alpha + 1 $ as defined previously.

\begin{corollary}
If $ n \in \omega $, then $ n $ is an ordinal.
\end{corollary}

\begin{proof}
As $ \emptyset $ is an ordinal, this follows by induction in \ref{thm:3.2.3} using \ref{lem:3.4.2} above.
\end{proof}

\begin{proposition}
\label{prop:3.4.4}
\hfill
\begin{enumerate}
\item If $ \alpha $ is an ordinal, then $ \alpha \notin \alpha $.
\item If $ \alpha $ is an ordinal and $ \beta \in \alpha $, then $ \beta $ is an ordinal.
\item If $ \alpha, \beta $ are ordinals and $ \alpha \subsetneq \beta $, then $ \alpha \in \beta $.
\item If $ \alpha $ is an ordinal, then $ \alpha = \cb{\beta \mid \beta \ \text{is an ordinal and} \ \beta \in \alpha} $.
\end{enumerate}
\end{proposition}

\begin{proof}
\hfill
\begin{enumerate}
\item Otherwise, $ \alpha $ has an element $ x = \alpha $ with $ x \in x $. But in terms of the ordering on $ \alpha $, this says $ x < x $, which is impossible.
\item First show $ \beta $ is a transitive set. So let $ x \in \beta $ and $ y \in x $. We want $ y \in \beta $. But we have $ y \in x \in \beta \in \alpha $. Using twice that $ \alpha $ is a transitive set, we get that $ x, y \in \alpha $. So, as the restriction of $ \in $ to $ \alpha $ is an ordering, we get $ y \in \beta $. As $ \alpha $ is a transitive set, $ \in_\beta $ is simply the restriction of $ \in_\alpha $ to $ \beta $, so this is a well ordering.
\item Note that, as $ \alpha \subsetneq \beta $, $ \beta \setminus \alpha $ is non-empty, so has a least element $ \gamma $, under $ \in_\beta $. We aim to show that $ \alpha = \gamma $. Think about $ \alpha $ and $ \beta $ as natural numbers to see why this is plausible. If $ x \in \gamma $, then $ x \in \beta $, so $ x \in \alpha $, as $ \gamma $ is the least element of $ \beta \setminus \alpha $. Thus $ \gamma \subseteq \alpha $. Conversely suppose there is some $ y \in \alpha \setminus \gamma $. Then, as $ \in $ is an ordering on $ \beta $ and $ y \notin \gamma $, either $ y = \gamma $ or $ \gamma \in y $. But in either case, we get $ \gamma \in \alpha $, as $ \alpha $ is a transitive set. A contradiction. So $ \alpha \subseteq \gamma $.
\item Trivial, by $ 2 $.
\end{enumerate}
\end{proof}

\pagebreak

\begin{definition}
If $ \alpha, \beta $ are ordinals, we write $ \alpha < \beta $ to indicate that $ \alpha \in \beta $, and $ \alpha \le \beta $ has the obvious meaning.
\end{definition}

\begin{note*}
By \ref{prop:3.4.4} above,
$$ \alpha \le \beta \qquad \iff \qquad \alpha \subseteq \beta. $$
\end{note*}

The following theorems can be paraphrased informally as saying that the class of all ordinals is well ordered by $ \le $.

\begin{theorem}
\label{thm:3.4.6}
Suppose $ \alpha, \beta, \gamma $ are ordinals.
\begin{enumerate}
\item If $ \alpha < \beta $ and $ \beta < \gamma $, then $ \alpha < \gamma $.
\item If $ \alpha \le \beta $ and $ \beta \le \alpha $, then $ \alpha = \beta $.
\item Exactly one of
$$ \alpha < \beta, \qquad \alpha = \beta, \qquad \beta < \alpha $$
holds.
\item If $ X $ is a non-empty set of ordinals, then $ X $ has a least element $ \delta $, and moreover,
$$ \delta = \bigcap X. $$
\end{enumerate}
\end{theorem}

\begin{proof}
\hfill
\begin{enumerate}
\item If $ \alpha \in \beta \in \gamma $, then $ \alpha \in \gamma $, as $ \gamma $ is a transitive set.
\item Otherwise, we have $ \alpha \in \beta $ and $ \beta \in \gamma $, whence $ \alpha \in \alpha $, contradicting \ref{prop:3.4.4}.
\item Certainly we already know that no more than one of these holds. Consider $ \alpha \cap \beta $. This is also an ordinal, by $ 4 $. (Exercise) Show that, if $ \alpha \ne \beta $, then $ \alpha \subsetneq \beta $ or $ \beta \subsetneq \alpha $. If, for example $ \alpha \cap \beta = \alpha $, then $ \alpha \subseteq \beta $ and so $ \alpha \le \beta $. So for a contradiction, suppose $ \alpha \cap \beta $ is a proper subset of both $ \alpha $ and $ \beta $. Thus, by \ref{prop:3.4.4}.$ 3 $,
$$ \alpha \cap \beta \in \alpha, \qquad \alpha \cap \beta \in \beta. $$
So $ \alpha \cap \beta \in \alpha \cap \beta $, a contradiction, all using $ \alpha \cap \beta $ is an ordinal.
\item Let $ \alpha \in X $ be any element and consider $ \alpha \cap X $. So this is the set of elements of $ X $ which are less than $ \alpha $. If this is empty, then $ \alpha $ is the least element of $ X $. Otherwise, it is a non-empty subset of $ \alpha $, so has a least element $ \beta $. It is then obvious that $ \beta $ is the least element of $ X $. If $ \delta $ is the least element of $ X $, then $ \delta \subseteq \eta $ for all $ \eta \in X $. It follows that $ \delta = \bigcap X $.
\end{enumerate}
\end{proof}

\begin{corollary}
\label{cor:3.4.7}
\hfill
\begin{enumerate}
\item If $ X $ is a non-empty set of ordinals, then $ \bigcup X $ is an ordinal.
\item $ \omega $ is an ordinal.
\end{enumerate}
\end{corollary}

\begin{proof}
\hfill
\begin{enumerate}
\item By \ref{thm:3.4.6}, $ X $ is well ordered by $ \in $. So it is enough to show that $ \bigcup X $ is a transitive set. If $ y \in z \in \bigcup X $, there is $ x \in X $ with $ z \in x $. So $ y \in z \in x $, and therefore $ y \in x $. So $ y \in \bigcup X $, as required.
\item Check that $ \bigcup \omega = \omega $, and use $ 1 $.
\end{enumerate}
\end{proof}

\pagebreak

So now we can form ordinals
$$ \omega^\dagger = \cb{0, 1, \dots, \omega}, \qquad \rb{\omega^\dagger}^\dagger = \cb{0, 1, \dots, \omega, \omega^+}, \qquad \dots. $$
Just as the natural numbers provided us with canonical examples of finite sets, we will now show that ordinals do the same thing for all well ordered sets.

\begin{theorem}
\label{thm:3.4.8}
If $ \ab{A; \le} $ is a well ordered set, then there is a unique ordinal $ \alpha $ which is similar to $ \ab{A; \le} $.
\end{theorem}

\begin{proof}
\hfill
\begin{itemize}
\item Uniqueness. Suppose $ \ab{A; \le} $ is similar to ordinals $ \alpha, \beta $. Then $ \alpha $ and $ \beta $ are similar. Without loss of generality, may assume $ \alpha \le \beta $, so if $ \alpha \ne \beta $, we have that $ \alpha < \beta $. $ \alpha $ is therefore a proper initial segment of $ \beta $, and so $ \beta $ is similar to a proper initial segment of itself, which contradicts \ref{thm:3.3.7}. This establishes the uniqueness part of the statement.
\item Existence. Now we show that there is some ordinal similar to $ A $. Let
$$ X = \cb{x \in A \mid \text{the initial segment} \ A\sb{x} \ \text{is similar to an ordinal}}. $$
For example, if $ A \ne \emptyset $ and $ x $ is the least element of $ A $, then $ A\sb{x} $ is empty, and the empty set is an ordinal, so $ x \in X $. Note that, by uniqueness, if $ x \in X $, then $ A\sb{x} $ is similar to a unique ordinal $ \alpha_x $. Let
$$ S = \cb{\alpha_x \mid x \in X}. $$
So $ S $ is a set of ordinals. In fact, $ S $ is an ordinal. To see this, note that we need to only show that it is a transitive set, that is, if $ \beta \in \alpha_x \in S $, then $ \beta \in S $. Let $ \alpha_x \in S $ and let
$$ \pi : A\sb{x} \to \alpha_x $$
be a similarity. If $ \beta \in \alpha_x $, let
$$ y = \pi^{-1}\rb{\beta}. $$
Then the restriction of $ \pi $ to $ A\sb{y} $ gives a similarity
$$ A\sb{y} \to \cb{\delta \in \alpha_x \mid \delta < \beta} = \beta. $$
So $ \beta = \alpha_y \in S $. So now write $ \alpha $ instead of $ S $. By what we already did, this argument also shows that $ X $ is an initial segment of $ A $. If we knew that $ A = X $, then we would be done, as the function
$$ \function{X}{\alpha}{x}{\alpha_x} $$
is easily seen to be a similarity $ A \to \alpha $. So suppose that $ X \ne A $, so $ X $ is a proper initial segment of $ A $. By \ref{lem:3.3.6}, we have that $ X = A\sb{z} $ for some $ z \in A \setminus X $. Know that $ x \mapsto \alpha_x $ gives a similarity $ X \to \alpha $. But then
$$ z \in X = A\sb{z}, $$
by definition of $ X $, a contradiction. So $ X = A $ and $ A \simeq \alpha $.
\end{itemize}
\end{proof}

\lecture{25}{Thursday}{29/11/18}

There is a problem with this proof. Why is $ S $ a set? No combination of the axioms we have given so far actually justifies the existence of such a set. There is no alternative but to assume another axiom about how we are allowed to create new sets from old ones.

\pagebreak

\begin{definition}
\label{def:3.4.9}
Suppose
$$ F\rb{x, y, z_1, \dots, z_r} $$
is a property of sets, expressible in our first-order language such that, whenever $ s_1, \dots, s_r $ are sets, for every set $ b $, there is a unique set $ a $ where
$$ F\rb{a, b, s_1, \dots, s_r} $$
holds.
\begin{itemize}
\item $ F $ is called an \textbf{operation on sets},
\item with $ z_1, \dots, z_r $ being \textbf{parameter variables},
\item the $ s_1, \dots, s_r $ are referred to as the \textbf{parameters} of the operation, and
\item $ F\rb{x, y, s_1, \dots, s_r} $ gives us a \textbf{function on sets} $ b \mapsto a $.
\end{itemize}
\end{definition}

\begin{example*}
\hfill
\begin{itemize}
\item No parameter variables.
$$ F\rb{a, b} \ \text{says} \ a \ \text{is the power set of} \ b. $$
\item One parameter $ c $.
$$ F\rb{a, b, c} \ \text{says} \ a \ \text{is the set of functions from} \ b \ \text{to} \ c. $$
\end{itemize}
\end{example*}

\begin{axiom}[Axiom of replacement]
Suppose
$$ F\rb{x, y, z_1, \dots, z_r} $$
is an operation on sets as above, and $ s_1, \dots, s_r $ are sets. Then for any set $ B $, there is a set $ A $ such that
$$ A = \cb{a \mid F\rb{a, b, s_1, \dots, s_r} \ \text{holds for some} \ b \in B}. $$
\end{axiom}

So here, the set $ A $ is obtained from $ B $ by replacing each $ b \in B $ by the corresponding set $ a $ for which
$$ F\rb{a, b, s_1, \dots, s_r} $$
holds. If we allow ourselves this axiom, then the set $ S $ in the above proof of \ref{thm:3.4.8} is justified. Obtain $ S $ from $ X $ using the operation $ F\rb{a, b} $ to say,
$$ \text{either} \ b \ \text{is a well ordered set similar to the ordinal} \ a, \ \text{or} \ b \ \text{is not a well ordered set and} \ a = \emptyset. $$
Then apply this to
$$ B = \cb{A\sb{x} \mid x \in X}, $$
to get
$$ S = \cb{x \mid F\rb{x, A\sb{y}} \ \text{holds for some} \ y \in X}. $$

\subsection{Transfinite induction}

\begin{theorem}[Transfinite induction]
\label{thm:3.5.1}
Suppose $ P\rb{x} $ is a property of sets. Assume that, for all ordinals $ \alpha $,
\begin{equation}
\label{eq:16}
\text{if} \ P\rb{\beta} \ \text{holds for all ordinals} \ \beta < \alpha, \ \text{then} \ P\rb{\alpha} \ \text{holds}.
\end{equation}
Then $ P\rb{\gamma} $ holds for all ordinals $ \gamma $.
\end{theorem}

\begin{proof}
Note that, if $ \alpha = 0 = \emptyset $, then $ P\rb{\beta} $ holds for all $ \beta < \alpha $, vacuously. So by $ \rb{\ref{eq:16}} $, $ P\rb{0} $ holds. Suppose for a contradiction that there is some ordinal $ \gamma $, such that $ P\rb{\gamma} $ does not hold. There is a least such $ \gamma $, by applying \ref{thm:3.4.6}.$ 4 $ to
$$ \cb{\beta \mid \beta \le \gamma, \ P\rb{\beta} \ \text{does not hold}}. $$
Call it $ \alpha $. Thus, for an ordinal $ \beta < \alpha $, $ P\rb{\beta} $ holds. By $ \rb{\ref{eq:16}} $, $ P\rb{\alpha} $ holds, a contradiction.
\end{proof}

\pagebreak

\begin{theorem}
\label{thm:3.5.2}
Suppose $ \alpha $ is an infinite ordinal, that is $ \omega \le \alpha $. Then $ \alpha \approx \alpha \times \alpha $.
\end{theorem}

\begin{corollary}
\label{cor:3.5.3}
\hfill
\begin{enumerate}
\item If $ \ab{A; \le} $ is an infinite well ordered set, then $ \abs{A} = \abs{A \times A} $.
\item Assuming axiom of choice, if $ A $ is any infinite set, then $ \abs{A} = \abs{A \times A} $.
\end{enumerate}
\end{corollary}

\begin{proof}
\hfill
\begin{enumerate}
\item By \ref{thm:3.4.8}, there is an ordinal $ \alpha $ with $ \alpha \approx A $. So $ \abs{A} = \abs{A \times A} $ follows from \ref{thm:3.5.2}.
\item By axiom of choice, $ A $ can be well ordered, so $ 2 $ follows from $ 1 $.
\end{enumerate}
\end{proof}

$ 2 $ is called the \textbf{fundamental theorem of cardinal arithmetic}.

\lecture{26}{Friday}{30/11/18}

\begin{proof}[Proof of \ref{thm:3.5.2}]
\hfill
\begin{itemize}
\item Assume that, if $ \omega \ne \beta < \alpha $, then $ \beta \approx \beta \times \beta $, the transfinite inductive hypothesis. Deduce $ \alpha \approx \alpha \times \alpha $.
\item May assume that, if $ \beta < \alpha $, then $ \abs{\beta} < \abs{\alpha} $. Implies $ \abs{\beta^\dagger} < \abs{\alpha} $.
\item Enough to show $ \abs{\alpha \times \alpha} \le \abs{\alpha} $, by \ref{thm:3.1.6}.
\end{itemize}
\begin{enumerate}[leftmargin=0.5in, label=Step \arabic*.]
\item Suppose we have a well ordering $ \le $ of $ A = \alpha \times \alpha $ such that, for all $ x \in A $,
$$ \abs{A\sb{x} = \cb{y \in A \mid y \le x}} < \abs{\alpha}. $$
Then prove that $ \abs{\alpha \times \alpha} = \abs{\alpha} $. By \ref{thm:3.4.6}, there is an ordinal $ \gamma $ which is similar to $ \ab{A; \le} $. Let
$$ f : \gamma \to A $$
be the similarity. Show that $ \gamma \subseteq \alpha $. Thus $ \abs{\gamma} \le \abs{\alpha} $, so $ \abs{A} \le \abs{\alpha} $. Let $ \eta \in \gamma $, so $ \eta < \gamma $. As $ f $ is a similarity, it gives a bijection
$$ \eta = \cb{\delta \in \gamma \mid \delta < \eta} \to A\sb{f\rb{\eta}}. $$
So
$$ \abs{\eta} = \abs{A\sb{f\rb{\eta}}} < \abs{\alpha}. $$
Thus $ \eta < \alpha $. Otherwise $ \alpha \le \eta $, so $ \alpha \subseteq \eta $, and then $ \abs{\alpha} \le \abs{\eta} $, a contradiction. Thus $ \eta \in \alpha $.
\item Find an ordering $ \le $ on $ \alpha \times \alpha = A $ as in step $ 1 $. For $ \lambda < \alpha $, let
$$ A_\lambda = \cb{\rb{\theta, \zeta} \in \alpha \times \alpha \mid \max\rb{\theta, \zeta} = \lambda}. $$
Define $ \le $ on $ A $ by
$$ \rb{\theta', \zeta'} < \rb{\theta, \zeta} \qquad \iff \qquad
\begin{cases}
\max\rb{\theta', \zeta'} < \max\rb{\theta, \zeta} \\
\max\rb{\theta', \zeta'} = \lambda = \max\rb{\theta, \zeta} \ \text{and}
\begin{cases}
\zeta = \zeta' = \lambda \ \text{and} \ \theta' < \theta \\
\zeta' < \zeta = \lambda \\
\theta = \theta' = \lambda \ \text{and} \ \zeta' < \zeta
\end{cases}
\end{cases}.
$$
Check that $ \le $ is a well ordering on $ A $. Show that property in step $ 1 $ holds. Let $ x = \rb{\theta, \zeta} \in A $. Let $ \lambda = \max\rb{\theta, \zeta} $. May assume $ \lambda \ge \omega $. Let $ \mu = \lambda^\dagger $. So $ \mu < \alpha $, and by inductive hypothesis, $ \abs{\mu \times \mu} = \abs{\mu} < \abs{\alpha} $.
$$ \cb{y \in A \mid y < x} \subseteq \cb{\rb{\theta', \zeta'} \in A \mid \max\rb{\theta', \zeta'} \le \lambda} \subseteq \mu \times \mu. $$
Thus $ \abs{A\sb{x}} \le \abs{\mu \times \mu} < \abs{\alpha} $.
\end{enumerate}
\end{proof}

\pagebreak

\subsection{Transfinite recursion}

Recall the notion of an operation $ F $ defined on sets. We have some property $ F\rb{x, y} $ of sets such that, for every set $ b $, there is a unique set $ a $ such that $ F\rb{a, b} $ holds.

\begin{notation*}
We denote, by $ F\rb{b} $, the result of applying $ F $ to the set $ b $ here, by \ref{def:3.4.9}.
\end{notation*}

\begin{note*}
If $ A $ is any set, then by the axiom of replacement, we have a function with domain $ A $ with $ a \mapsto F\rb{a} $ for all $ a \in A $. We refer to this as
the restriction of $ F $ to $ A $ and denote it by $ F \upharpoonright A $.
\end{note*}

In transfinite recursion, we want to define an operation $ G $ on all ordinals $ \alpha $, in such a way that sets $ G\rb{\alpha} $ is obtained from the values of sets $ G\rb{\beta} $ for ordinals $ \beta < \alpha $. Recall that
$$ \alpha = \cb{\beta \mid \beta < \alpha}, $$
so another way of saying this is that $ G\rb{\alpha} $ is determined by
$$ G \upharpoonright \alpha : \alpha \to \cb{G\rb{\beta} \mid \beta < \alpha}, $$
which is a set, by the axiom of replacement.
The way we construct $ G\rb{\alpha} $ from the previous values is controlled by applying some operation
$$ F : G\rb{0}, \cdots, G\rb{\beta}, \dots \to G\rb{\alpha}, $$

\begin{theorem}[Transfinite recursion]
\label{thm:3.6.1}
Let $ F $ be an operation. Then there is an operation $ G $ such that, for all ordinals $ \alpha $, we have
$$ G\rb{\alpha} = F\rb{G \upharpoonright \alpha}. $$
If $ G' $ is another such operation, then $ G\rb{\alpha} = G'\rb{\alpha} $ for all ordinals $ \alpha $.
\end{theorem}

We cannot say that $ G $ is unique here, as we have no control over what it does on sets which are not ordinals, and it is usually not relevant. We might say that $ G $ is uniquely determined on ordinals. In practice, we usually do not write $ F $ down explicitly as a first-order formula.

\begin{remark}
An application is the Lindenbaum lemma, comparing \ref{prop:1.3.8} and \ref{prop:2.5.2}. Suppose $ \LL $ is a first-order language whose alphabet of symbols can be well ordered. Suppose $ \Sigma $ is a consistent set of closed $ \LL $-formulas. Then there is a consistent set $ \Sigma^* \supseteq \Sigma $ of closed $ \LL $-formulas such that, for every closed $ \LL $-formula $ \psi $, either $ \Sigma^* \vdash_{K_\LL} \psi $ or $ \Sigma^* \vdash_{K_\LL} \notb{\psi} $.
\end{remark}

\begin{proof}
The set of $ \LL $-formulas can be well ordered. Any subset of a well ordered set is well ordered, so the set of closed $ \LL $-formulas is well ordered. Any well ordered set is similar to some ordinal, by \ref{thm:3.4.8}. So we can write the set of closed $ \LL $-formulas as
$$ \cb{\phi_\alpha \mid \alpha < \lambda}, $$
for some ordinal $ \lambda $. Define, for each ordinal $ \alpha $, a consistent set $ G\rb{\alpha} \supseteq \Sigma $ of closed $ \LL $-formulas
$$ G\rb{\alpha} =
\begin{cases}
\Sigma \cup \bigcup_{\beta < \alpha} G\rb{\beta} & \alpha < \lambda, \ \Sigma \cup \bigcup_{\beta < \alpha} G\rb{\beta} \vdash_{K_\LL} \phi_\alpha \\
\Sigma \cup \bigcup_{\beta < \alpha} G\rb{\beta} \cup \cb{\notb{\phi_\alpha}} & \alpha < \lambda, \ \Sigma \cup \bigcup_{\beta < \alpha} G\rb{\beta} \not\vdash_{K_\LL} \phi_\alpha \\
\Sigma \cup \bigcup_{\beta < \lambda} G\rb{\beta} & \alpha \ge \lambda
\end{cases}.
$$
Let $ \Sigma^* = G\rb{\lambda} $. Proof that it works is just as in the countable case.
\end{proof}

\lecture{27}{Monday}{03/12/18}

\begin{remark}
Can use similar arguments at other points in the proof of the completeness theorem, by adding witnesses in \ref{thm:2.5.3}, to get that completeness theorem, compactness theorem, etc. holds.
\end{remark}

\begin{proof}[Proof of \ref{thm:3.6.1}]
Appendix A.4.
\end{proof}

\subsection{Axiom of regularity}

\begin{axiom}[Axiom of regularity (foundation)]
$$ \fab{x}\impb{\neb{x}{\emptyset}}{\teb{a}\andb{\inb{a}{x}}{\eqb{a \cap x}{\emptyset}}} $$
\end{axiom}

In particular, there is no set $ b $ with $ b \in b $. Consider $ x = \cb{b} $. By ZF9, $ b \cap \cb{b} \ne \emptyset $, so $ b \notin b $. We will not use this. Axioms ZF1-9 is \textbf{Zermelo-Fraenkel set theory}.

\pagebreak

\section{Axiom of choice and consequences}

\subsection{Statement and well ordering principle}

\begin{definition}[Axiom of choice]
Suppose $ A $ is a set of non-empty sets. Then there is a function
$$ f : A \to \bigcup A, $$
with $ f\rb{a} \in a $ for every $ a \in A $.
\end{definition}

Axioms ZF1-9 and axiom of choice (AC) is \textbf{ZFC}.

\begin{example}
Suppose $ X $ is any non-empty set. Let
$$ A = \PP\rb{X} \setminus \cb{\emptyset}, $$
the non-empty subsets of $ X $. By axiom of choice, there is a function
$$ f : A \to X, $$
such that $ f\rb{Y} \in Y $ for every $ \emptyset \ne Y \subseteq X $. Such an $ f $ is called a \textbf{choice function}, on the subsets of $ X $.
\end{example}

\begin{note*}
If $ \ab{X; \le} $ is a well ordered set, we do not need axiom of choice to get a choice function on $ X $. Define, for $ \emptyset \ne Y \subseteq X $, $ f\rb{Y} = \min\rb{Y} $.
$$ f = \cb{\rb{Y, a} \in \rb{\PP\rb{X} \setminus \cb{\emptyset}} \times X \mid a = \min\rb{Y}} $$
exists, just by axiom of specification. 
\end{note*}

\begin{theorem}
\label{thm:4.1.3}
Suppose $ X $ is a non-empty set and
$$ f : \PP\rb{X} \setminus \cb{\emptyset} \to X $$
is a choice function. Then there is a well ordering $ \le $ of $ X $, that is a well ordered set $ \ab{X; \le} $.
\end{theorem}

Proof needs Hartogs' lemma.

\begin{lemma}[Hartogs' lemma]
\label{lem:4.1.4}
For any set $ X $, there is an ordinal $ \alpha $ such that there is no injective function
$$ h : \alpha \to X. $$
\end{lemma}

\begin{proof}[Proof of \ref{thm:4.1.3}]
Let $ \alpha $ be such an ordinal. Consider
$$ \widetilde{X} = X \cup \cb{\infty}, $$
where $ \infty $ is some set with $ \infty \notin X $. Use transfinite recursion and the choice function $ f $ to define an operation $ G $, for an ordinal $ \gamma $,
$$ G\rb{\gamma} =
\begin{cases}
f\rb{X \setminus \cb{G\rb{\beta} \mid \beta < \gamma}} & X \setminus \cb{G\rb{\beta} \mid \beta < \gamma} \ne \emptyset \\
\infty & \text{otherwise}
\end{cases}.
$$
Note that, if $ \infty \notin Im\rb{G \upharpoonright \gamma} $, then $ G \upharpoonright \gamma $ is an injective function
$$ \gamma \to X. $$
By Hartogs' lemma, there is some ordinal $ \alpha $ with $ G\rb{\alpha} = \infty $. Take the least such ordinal and call it $ \beta $. Then
$$ g = G \upharpoonright \beta : \beta \to X $$
is injective and surjective, as $ G\rb{\beta} = \infty $. So $ g : \beta \to X $ is a bijection. Define $ \le $ on $ X $ by
$$ x_1 \le x_2 \qquad \iff \qquad g^{-1}\rb{x_1} \le g^{-1}\rb{x_2}, $$
where $ \le $ is the ordering $ \in $ on $ \beta $.
\end{proof}

\pagebreak

\begin{proof}[Proof of \ref{lem:4.1.4}]
Given $ A $ any set, find an ordinal $ \alpha $ with no injective function $ \alpha \to A $. Consider
$$ X = \cb{\ab{Y, \le_Y} \mid Y \subseteq A, \ \le_Y \ \text{is a well order on} \ Y}. $$
This is a set, by specification. Let
$$ S = \cb{\beta \mid \beta \ \text{is an ordinal similar to some} \ \ab{Y, \le_Y} \in X}, $$
using \ref{thm:3.4.8} and axiom of replacement to get that $ S $ is a set. So
$$ S = \cb{\beta \mid \beta \ \text{is an ordinal and there is an injective function} \ \beta \to \alpha}. $$
Let $ \sigma = \bigcup S $. This is an ordinal, and $ \beta \le \sigma $ for all $ \beta \in S $, by \ref{prop:3.4.4} and \ref{cor:3.4.7}. Let $ \alpha = \sigma^\dagger $. So $ \alpha $ is an ordinal, and for all $ \beta \in S $,
$$ \beta \le \sigma < \alpha. $$
So $ \alpha \notin S $. Thus there is no injective function $ \alpha \to A $.
\end{proof}

\lecture{28}{Thursday}{06/12/18}

Lecture 28 is a problem class.

\lecture{29}{Friday}{07/12/18}

\begin{corollary}
Assuming ZF, axiom of choice is equivalent to the following \textbf{well ordering principle (WO)}.
$$ \text{If} \ A \ \text{is any set, there is} \ \le_A \subseteq A \times A \ \text{such that} \ \ab{A; \le_A} \ \text{is a well ordered set}. $$
That is,
$$ \text{ZF} \vdash \iffb{\text{axiom of choice}}{\text{well ordering principle}}. $$
\end{corollary}

\begin{proof}
\hfill
\begin{itemize}
\item[$ \implies $] \ref{thm:4.1.3}.
\item[$ \impliedby $] If $ A $ is a set of non-empty sets, let $ B = \bigcup A $. By well ordering principle, there is a well ordering $ \le_B $ of $ B $. Then define
$$ f : A \to \bigcup A, $$
by $ f\rb{a} $ as the least element of $ a $, with respect to $ b $.
\end{itemize}
\end{proof}

\begin{corollary}
\label{cor:4.1.6}
Assume ZFC.
\begin{enumerate}
\item If $ A $ is any set, there is an ordinal $ \alpha $ with $ \alpha \approx A $.
\item If $ A, B $ are sets, then one of $ \abs{A} \le \abs{B} $ or $ \abs{B} \le \abs{A} $ holds.
\item Fundamental theorem of cardinal arithmetic. If $ A $ is any infinite set, then $ \abs{A \times A} = \abs{A} $.
\end{enumerate}
\end{corollary}

\begin{proof}
\hfill
\begin{enumerate}
\item \ref{cor:3.5.3}.
\item By remark after proof of \ref{thm:3.3.7}.
\item \ref{cor:3.5.3}.
\end{enumerate}
\end{proof}

\begin{lemma}
\label{lem:4.1.7}
Assuming ZFC, for sets $ A, B $, with $ A \ne \emptyset $, we have
$$ \abs{A} \le \abs{B} \qquad \iff \qquad \text{there is a surjective function} \ h : B \to A. $$
\end{lemma}

\begin{proof}
Problem class.
\end{proof}

\pagebreak

\subsection{Cardinals and cardinal arithmetic}

Assume ZFC.

\begin{definition}
An ordinal $ \alpha $ is a \textbf{cardinal} if it is not equinumerous with any $ \beta < \alpha $.
\end{definition}

\begin{example*}
\hfill
\begin{itemize}
\item If $ n \in \omega $, then $ n $ is a cardinal.
\item $ \omega $ is a cardinal.
\item If $ \beta $ is an infinite ordinal, then $ \beta^\dagger $ is not a cardinal, as $ \beta \approx \beta^\dagger $.
\end{itemize}
\end{example*}

\begin{lemma}
\hfill
\begin{enumerate}
\item Suppose $ \gamma $ is any ordinal. There is a unique cardinal $ \alpha $ with $ \alpha \approx \gamma $.
\item Suppose $ A $ is any set. There is a unique cardinal $ \alpha $ such that $ A \approx \alpha $.
\end{enumerate}
\end{lemma}

\begin{proof}
\hfill
\begin{enumerate}
\item Take the least element of $ \cb{\beta \le \gamma \mid \beta \approx \gamma} $.
\item By \ref{cor:4.1.6}, there is an ordinal $ \gamma $ with $ \gamma \approx A $. Use $ 1 $.
\end{enumerate}
\end{proof}

\begin{definition}
The unique cardinal equinumerous with $ A $ is called the \textbf{cardinality} of $ A $. Denote by $ card\rb{A} $ or $ \abs{A} $.
\end{definition}

\begin{remark*}
$$ \text{There is an injective function} \ A \to B \qquad \iff \qquad card\rb{A} \le card\rb{B}. $$
(Exercise) So ok use $ \abs{A} $.
\end{remark*}

If $ \alpha $ is an ordinal, then
$$ \alpha \ \text{is a cardinal} \qquad \iff \qquad \abs{\alpha} = \alpha. $$

\begin{example*}
If $ A $ is a countably infinite set, then $ \abs{A} = \omega $.
\end{example*}

\begin{definition}
Suppose $ A, B $ are disjoint sets with $ \abs{A} = \kappa $ and $ \abs{B} = \lambda $. Let
\begin{itemize}
\item $ \kappa + \lambda = \abs{A \cup B} $, and
\item $ \kappa \cdot \lambda = \abs{A \times B} $.
\end{itemize}
\end{definition}

\begin{remark*}
Does not depend on choice of $ A, B $.
\end{remark*}

\begin{theorem}
\label{thm:4.2.5}
Suppose $ \kappa, \lambda $ are cardinals, $ \kappa \le \lambda $, and $ \lambda $ infinite. Then
\begin{enumerate}
\item $ \kappa \cdot \lambda = \lambda $, if $ \kappa \ne 0 $, and
\item $ \kappa + \lambda = \lambda $.
\end{enumerate}
\end{theorem}

\begin{proof}
\hfill
\begin{enumerate}
\item As $ \kappa \subseteq \lambda $, we have $ \kappa \times \lambda \subseteq \lambda \times \lambda $. So
$$ \kappa \cdot \lambda \le \abs{\lambda \times \lambda} = \abs{\lambda} = \lambda, $$
by the fundamental theorem of cardinal arithmetic, and $ \lambda $ is a cardinal. On the other hand, as $ \kappa \ne 0 $,
$$ \lambda = \abs{\lambda} = \abs{\cb{0} \times \lambda} \le \abs{\kappa \times \lambda} = \kappa \cdot \lambda. $$
\item By $ 1 $,
$$ \lambda \le \kappa + \lambda \le \lambda + \lambda = \cb{0, 1} \cdot \lambda = 2 \cdot \lambda = \lambda. $$
So $ \lambda = \kappa + \lambda $.
\end{enumerate}
\end{proof}

\pagebreak

\begin{theorem}
\label{thm:4.2.6}
Suppose $ A $ is an infinite set of cardinality $ \lambda $. Suppose each element of $ A $ is a set of cardinality at most $ \kappa $. Then
$$ \abs{\bigcup A} \le \lambda \cdot \kappa. $$
\end{theorem}

\begin{proof}
Can assume $ \emptyset \notin A $. For each $ a \in A $, the set
$$ S_a = \cb{\text{surjective functions} \ \kappa \to a} $$
is non-empty, by \ref{lem:4.1.7}. Assuming axiom of choice, there is a function
$$ \function[F]{A}{\bigcup_{a \in A} S_a}{a}{\kappa \to \alpha}, $$
which is a surjective function. Denote this by $ F_a $. Let $ h : \lambda \to A $ be a bijection. Now define
$$ \function[G]{\lambda \times \kappa}{\bigcup A}{\rb{\alpha, \beta}}{F_{h\rb{\alpha}}\rb{\beta}}. $$
This is surjective. So by \ref{lem:4.1.7}, $ \abs{\lambda \times \kappa} \ge \abs{\bigcup A} $, that is $ \abs{\bigcup A} \le \lambda \cdot \kappa $.
\end{proof}

\begin{example}
\hfill
\begin{enumerate}
\item Suppose $ A $ is an infinite set, so $ \abs{A} = \lambda \ge \omega $. Let
$$ S = \cb{\text{finite sequences of elements of} \ A} = \bigcup_{n \in \omega} A^n. $$
Then $ \abs{S} = \abs{A} = \lambda $.
\item Consider $ \R $ as a vector space over $ \Q $. Suppose $ B \subseteq \R $, such that the linear span of $ B $, over $ \Q $, is $ \R $. Then $ \abs{B} = \abs{\R} $.
\end{enumerate}
\end{example}

\begin{proof}
\hfill
\begin{enumerate}
\item For $ n \in \N $, $ \abs{A}^n = \abs{A} $, by repeated use of the fundamental theorem of cardinal arithmetic. By \ref{thm:4.2.6},
$$ \abs{S} \le \omega \cdot \abs{A} = \abs{\omega} \cdot \abs{A} = \lambda, $$
by \ref{thm:4.2.5}.$ 1 $. $ A \subseteq S $, so $ \abs{S} \ge \lambda $.
\item Obviously $ \abs{B} \le \abs{\R} $, and $ B $ is infinite. If $ x \in \R $, there are
$$ n \in \N, \qquad q_0, \dots, q_n \in \Q, \qquad b_0, \dots, b_n \in B, \qquad x = \sum_{i = 0}^n q_ib_i. $$
Let
$$ S_1 = \cb{\text{finite sequences from} \ \Q}, \qquad S_2 = \cb{\text{finite sequences from} \ B}. $$
Consider
$$ S_1 \times S_2 \supseteq T = \cb{\rb{\rb{q_0, \dots, q_n}, \rb{b_0, \dots, b_n}} \mid q_i \in \Q, \ b_i \in B}. $$
We have a surjection
$$ \function{T}{\R}{\rb{\rb{q_0, \dots, q_n}, \rb{b_0, \dots, b_n}}}{x = \sum_{i = 0}^n q_ib_i}. $$
By $ 1 $, $ \abs{S_1} = \abs{\Q} = \omega $ and $ \abs{S_2} = \abs{B} $. By \ref{lem:4.1.7},
$$ \abs{\R} \le \abs{T} \le \abs{S_1 \times S_2} = \omega \cdot \abs{B} = \max\rb{\omega, \abs{B}}, $$
by \ref{thm:4.2.5}. As $ \abs{\R} > \omega $, we get $ \abs{B} = \abs{\R} $.
\end{enumerate}
\end{proof}

\pagebreak

\subsection{Zorn's lemma}

\lecture{30}{Monday}{10/12/18}

A \textbf{partially ordered set (poset)} $ \ab{A; \le} $ satisfies
\begin{align*}
& \fab{x \in A}\fab{y \in A}\fab{z \in A}\impb{\andb{\leb{x}{y}}{\leb{y}{z}}}{\leb{x}{z}} \\
\land & \fab{x \in A}\fab{y \in A}\impb{\andb{\leb{x}{y}}{\leb{y}{x}}}{\eqb{x}{y}} \\
\land & \fab{x \in A}\leb{x}{x}.
\end{align*}

\begin{example*}
$ \ab{\PP\rb{X}; \subseteq} $.
\end{example*}

A \textbf{chain} $ C $ in a partially ordered set $ \ab{A; \le} $ is a subset $ C \subseteq A $ such that
$$ \fab{x \in C}\fab{y \in C}\orb{\leb{x}{y}}{\leb{y}{x}}. $$
An \textbf{upper bound} of $ C $ in $ A $ is $ a \in A $ such that
$$ \fab{x \in C}\leb{x}{a}. $$

\begin{example*}
If $ C \subseteq \PP\rb{X} $ is a chain, then $ \bigcup C $ is an upper bound for $ C $ in $ \PP\rb{X} $.
\end{example*}

\begin{definition}[Zorn's lemma]
Suppose $ \ab{A; \le} $ is a non-empty partially ordered set, in which every chain has an upper bound in $ A $. Then $ A $ has a maximal element. That is,
$$ \teb{z \in A}\fab{x \in A}\impb{\leb{z}{x}}{\eqb{x}{z}}. $$
\end{definition}

\begin{theorem}
\label{thm:4.3.2}
$$ \text{ZF} \vdash \iffb{\text{axiom of choice}}{\text{Zorn's lemma}}. $$
\begin{itemize}
\item Assuming ZFC, then Zorn's lemma holds.
\item Assuming ZF and Zorn's lemma, then axiom of choice holds.
\end{itemize}
\end{theorem}

\begin{example}
Assume ZFC. Suppose $ V $ is a vector space over a field $ F $. Then $ V $ has a basis over $ F $.
\end{example}

\begin{proof}
Use Zorn's lemma. Let
$$ A = \cb{\text{linearly independent subsets of} \ V, \ \text{ordered by} \ \subseteq}. $$
Claim that, if $ C $ is a chain in $ A $, then $ \bigcup C \in A $. Must show that, if $ y_1, \dots, y_n \in \bigcup C $, then $ y_1, \dots, y_n $ are linearly independent. Let $ C_1, \dots, C_n \in C $, such that $ y_i \in C_i $. As $ C $ is a chain, there is some $ j \le n $ with $ C_i \subseteq C_j $ for all $ i \le n $. But then $ y_1, \dots, y_n \in C_j $, and so $ y_1, \dots, y_n $ are linearly independent. So in $ \ab{A; \subseteq} $, every chain has an upper bound. So by Zorn's lemma, there is a maximal element $ B $ of $ A $. So $ B $ is a linearly independent set, and if $ v \in V \setminus B $, then $ B \cup \cb{v} $ is not linearly independent. It follows that $ v $ is a linear combination of elements of $ B $. So span of $ B $ is $ V $.
\end{proof}

\begin{proof}[Proof of \ref{thm:4.3.2}]
\hfill
\begin{itemize}
\item[$ \implies $] Let
$$ f : \PP \rb{A} \setminus \cb{\emptyset} \to A $$
be a choice function. Suppose for a contradiction that $ A $ has no maximal element. Let $ C \subseteq A $ be a chain. By assumption, there is some $ y \in A $ with $ c \le y $ for all $ c \in C $. So there is $ z > y $ in $ A $, and therefore $ z > c $ for all $ c \in C $. Use transfinite recursion to define an operation $ G $ such that, for all ordinals $ \alpha $, $ G\rb{\alpha} \in A $ and
$$ G\rb{\alpha} = f\rb{\cb{z \in A \mid z > G\rb{\beta} \ \text{for all} \ \beta < \alpha}}. $$
So
$$ G\rb{0} < \dots < G\rb{\beta} < \dots. $$
That is, for all ordinals $ \beta < \alpha $,
$$ G\rb{\beta} < G\rb{\alpha}. $$
By Hartogs' lemma, or by cardinality, this is impossible.
\end{itemize}
\end{proof}

\pagebreak

\subsection{Postscript}

Assume ZFC. For a first-order language $ \LL $, the cardinality of $ \LL $ is the cardinality of the set of $ \LL $-formulas, which is the cardinality of the alphabet. Assuming ZFC, the following hold for $ \LL $ of arbitrary cardinality.
\begin{itemize}
\item G\"odel's completeness theorem, and results leading to it. Compare 3.6.
\item Compactness theorem, and compactness theorem for normal models in $ \LL^= $.
\end{itemize}

\pagebreak

\appendix

\section{Proofs}

\subsection{Lemma \ref{lem:2.3.7}}

\begin{proof}[Proof of \ref{lem:2.3.7}]
This is by induction on the number of connectives and quantifiers in $ \phi $.
\begin{itemize}
\item Base case. $ \phi $ is an atomic formula $ R\rb{u_1, \dots, u_m} $, where $ R $ is an $ m $-ary relation symbol and $ u_1, \dots, u_m $ are terms. Let $ u^*_i $ be the result of substituting $ t $ for $ x_1 $ in $ u_i $. Then by induction on the length of the terms, each $ u_i^* $ is a term and $ v'\rb{u_i} = v\rb{u_i^*} $. Moreover, $ \phi\rb{t} $ is $ R\rb{u_1^*, \dots, u_m^*} $. Then
\begin{align*}
v'\sb{\phi\rb{x_1}} = T \qquad
& \iff \qquad \AA \vDash R\rb{v'\rb{u_1}, \dots, v'\rb{u_m}} \\
& \iff \qquad \AA \vDash R\rb{v\rb{u_1^*}, \dots, v\rb{u_m^*}} \\
& \iff \qquad v\sb{\phi\rb{t}} = T.
\end{align*}
\item Inductive step. There are three cases.
\begin{enumerate}[leftmargin=0.5in, label=Case \arabic*.]
\item $ \phi $ is $ \notb{\psi} $.
\item $ \phi $ is $ \impb{\psi}{\chi} $.
\item $ \phi $ is $ \fab{x_i}\psi $.
\end{enumerate}
(Exercise: the first two cases) We do the third. We can assume that $ i \ne 1 $. Otherwise, $ x_1 $ is not free in $ \phi $, and $ \phi\rb{t} $ is just $ \phi $. \ref{lem:2.3.7} then follows from \ref{thm:2.3.3}. Note also that, as $ t $ is free for $ x_1 $ in $ \fab{x_i}\psi $, it follows that $ t $ is free for $ x_1 $ in $ \psi $, and $ x_i $ is not a variable in $ t $. Suppose first that $ v'\sb{\phi\rb{x_1}} = F $. We show that $ v\sb{\phi\rb{t}} = F $. By \ref{def:2.2.9}, there is a valuation $ w' $ which is $ x_i $-equivalent to $ v' $ with $ w'\sb{\psi\rb{x_1}} = F $. Note that, as $ i \ne 1 $,
\begin{equation}
\label{eq:12}
w'\rb{x_1} = v'\rb{x_1} = v\rb{t}.
\end{equation}
Define a valuation $ w $ by
$$ w\rb{x_j} =
\begin{cases}
v\rb{x_j} & j \ne 1, i \\
w'\rb{x_i} & j = i \\
v\rb{x_1} & j = 1
\end{cases}.
$$
So $ w $ is $ x_1 $-equivalent to $ w' $ and $ x_i $-equivalent to $ v $, noting that $ v, v' $ are $ x_i $-equivalent, and $ w, v' $ are $ x_i $-equivalent. As $ x_i $ does not occur in $ t $, we have, by \ref{thm:2.3.3} and $ \rb{\ref{eq:12}} $,
$$ w\rb{t} = v\rb{t} = w'\rb{x_1}. $$
We can now apply the inductive hypothesis to $ w $, $ w' $, and $ \psi $. We obtain that
$$ w\sb{\psi\rb{t}} = w'\sb{\psi\rb{x_1}} = F. $$
As $ w, v $ are $ x_i $-equivalent, it follows that
$$ v\sb{\fab{x_i}\psi\rb{t}} = F. $$
So $ v\sb{\phi\rb{t}} = F $, as required. We now prove the converse direction, since we cannot argue by symmetry here. So suppose $ v\sb{\phi\rb{t}} = F $. There is a valuation $ w $ which is $ x_i $-equivalent to $ v $ with $ w\sb{\psi\rb{t}} = F $. Let $ w' $ be the valuation $ x_1 $-equivalent to $ w $ with
$$ w'\rb{x_1} = w\rb{t} = v\rb{t} = v'\rb{x_1}. $$
The fact that $ w\rb{t} = v\rb{t} $ is as before. By the inductive hypothesis,
$$ w'\sb{\psi\rb{x_1}} = w\rb{\psi\rb{t}} = F. $$
As $ w' $ is $ x_i $-equivalent to $ v' $, we have
$$ v'\sb{\fab{x_i}\psi\rb{x_1}} = F. $$
So $ v'\sb{\phi\rb{x_1}} = F $. This completes the inductive step.
\end{itemize}
\end{proof}

\pagebreak

\subsection{Theorem \ref{thm:2.5.3}}

\begin{proof}[Sketch of proof of \ref{thm:2.5.3}]
Series of steps. Notation is cumulative.
\begin{enumerate}[leftmargin=0.5in, label=Step \arabic*.]
\item Let $ b_0, b_1, \dots $ be new constant symbols. Form $ \LL^+ $ by adding these to the symbols of $ \LL $. Regard $ \Sigma $ as a set of $ \LL^+ $-formulas. Check that $ \Sigma $ is still consistent, in the formal system $ K_{\LL^+} $. Note that $ \LL^+ $ is still a countable language.
\item Adding witnesses. Claim that there is a consistent set of closed $ \LL^+ $-formulas $ \Sigma_\infty \supseteq \Sigma $ such that, for every $ \LL^+ $-formula $ \theta\rb{x_i} $ with one free variable, there is some $ b_j $ with
$$ \Sigma_\infty \vdash_{K_{\LL^+}} \impb{\notb{\fab{x_i}\theta\rb{x_i}}}{\notb{\theta\rb{b_j}}}. $$
Think of $ \theta\rb{x_i} $ as $ \notb{\chi\rb{x_i}} $. Then this formula is essentially
$$ \impb{\teb{x_i}\chi\rb{x_i}}{\chi\rb{b_j}}, $$
so $ b_j $ witnesses the existence of $ x_i $ satisfying $ \chi $.
\item By the Lindenbaum lemma in \ref{prop:2.5.2}, there is a consistent set $ \Sigma^* \supseteq \Sigma_\infty $ of closed $ \LL^+ $-formulas such that, for every closed $ \phi $, either $ \Sigma^* \vdash_{K_{\LL^+}} \phi $ or $ \Sigma^* \vdash_{K_{\LL^+}} \notb{\phi} $.
\item Building a structure. Let
$$ A = \cb{\overline{t} \mid t \ \text{is a closed term of} \ \LL^+}. $$
Note that
\begin{itemize}
\item a term is closed if it only involves constant symbols and function symbols, and no variables,
\item use the $ \overline{\cdot} $ to distinguish when we are thinking of a term as an element of $ A $, and
\item as $ \LL^+ $ is countable, $ A $ is countable.
\end{itemize}
Make $ A $ into an $ \LL^+ $ structure.
\begin{enumerate}
\item Each constant symbol $ c $ of $ \LL^+ $ is interpreted as $ \overline{c} \in A $.
\item Suppose $ R $ is an $ n $-ary relation symbol. Define the relation $ \overline{R} \subseteq A^n $ by
$$ \rb{\overline{t_1}, \dots, \overline{t_n}} \in \overline{R} \qquad \iff \qquad \Sigma^* \vdash_{K_{\LL^+}} R\rb{t_1, \dots, t_n}, $$
where $ R\rb{t_1, \dots, t_n} $ is a closed atomic $ \LL^+ $-formula and $ t_1, \dots, t_n $ are closed $ \LL^+ $-terms.
\item Suppose $ f $ is an $ m $-ary function symbol. Define a function $ \overline{f} : A^m \to A $ by
$$ \overline{f}\rb{\overline{t_1}, \dots, \overline{t_m}} = \overline{f\rb{t_1, \dots, t_m}}, $$
for closed terms $ t_1, \dots, t_m $.
\end{enumerate}
Call this structure $ \AA $. Note that, if $ v $ is a valuation in $ \AA $ and $ t $ is a closed term, then $ v\rb{t} = \overline{t} $, by (a) and (c) here.
\pagebreak
\item Main lemma. Claim that, for every closed $ \LL^+ $-formula $ \phi $,
\begin{equation}
\label{eq:14}
\Sigma^* \vdash_{K_{\LL^+}} \phi \qquad \iff \qquad \AA \vDash \phi.
\end{equation}
Proof by induction on number of connectives and quantifiers in $ \phi $.
\begin{itemize}
\item Base case. $ \phi $ is atomic, that is $ \phi $ is $ R\rb{t_1, \dots, t_n} $, for some closed terms $ t_i $, and relation symbol $ R $. $ \rb{\ref{eq:14}} $ holds by (b) in definition of $ \AA $.
\item Inductive step. Assume $ \rb{\ref{eq:14}} $ holds for closed formulas involving fewer connectives and quantifiers.
\begin{enumerate}[leftmargin=0.5in, label=Case \arabic*.]
\item $ \phi $ is $ \notb{\psi} $.
\item $ \phi $ is $ \impb{\psi}{\chi} $.
\item $ \phi $ is $ \fab{x_i}\psi $.
\end{enumerate}
In cases $ 1 $ and $ 2 $, $ \psi, \chi $ are closed. So $ \rb{\ref{eq:14}} $ holds for these.
\begin{enumerate}[leftmargin=0.5in, label=Case \arabic*.]
\item $ \phi $ is $ \notb{\psi} $.
\begin{align*}
\AA \vDash \phi \qquad
& \iff \qquad \AA \not\vDash \psi & \text{by \ref{thm:2.3.3}} \\
& \iff \qquad \Sigma^* \not\vdash_{K_{\LL^+}} \psi & \text{by inductive hypothesis} \ \rb{\ref{eq:14}} \\
& \iff \qquad \Sigma^* \vdash_{K_{\LL^+}} \notb{\psi} & \text{by step} \ 3.
\end{align*}
\item Exercise.
\item $ \phi $ is $ \fab{x_i}\psi $.
\begin{enumerate}[leftmargin=0.5in, label=Case 3\alph*.]
\item $ x_i $ is not free in $ \psi $. So $ \psi $ is closed and we can use inductive hypothesis.
\item $ x_i $ is free in $ \psi $. So $ \psi\rb{x_i} $ has a single free variable.
\begin{itemize}
\item[$ \impliedby $] Suppose for a contradiction that $ \AA \vDash \phi $ and $ \Sigma^* \not\vdash_{K_{\LL^+}} \phi $. Then by step $ 3 $,
$$ \Sigma^* \vdash_{K_{\LL^+}} \notb{\phi}. $$
By step $ 2 $,
$$ \Sigma^* \vdash_{K_{\LL^+}} \impb{\notb{\fab{x_i}\psi\rb{x_i}}}{\notb{\psi\rb{b_j}}} $$
for some constant symbol $ b_j $. That is,
$$ \Sigma^* \vdash_{K_{\LL^+}} \impb{\notb{\phi}}{\notb{\psi\rb{b_j}}}. $$
So
$$ \Sigma^* \vdash_{K_{\LL^+}} \notb{\psi\rb{b_j}}. $$
$ \notb{\psi\rb{b_j}} $ is closed and, by case $ 1 $, $ \rb{\ref{eq:14}} $ applies. We obtain
\begin{equation}
\label{eq:15}
\AA \vDash \notb{\psi\rb{b_j}}.
\end{equation}
This contradicts $ \AA \vDash \fab{x_i}\psi $. Take a valuation $ v $ in $ \AA $ with $ v\rb{x_i} = \overline{b_j} $, then $ v $ does not satisfy $ \psi $, by $ \rb{\ref{eq:15}} $.
\end{itemize}
\end{enumerate}
\end{enumerate}
\end{itemize}
\end{enumerate}
\end{proof}

\pagebreak

\subsection{Lemma \ref{lem:2.6.3}}

\begin{proof}[Proof of \ref{lem:2.6.3}]
The result \ref{lem:2.6.3} is proved by induction on the number of connectives and quantifiers in $ \phi $.
\begin{itemize}
\item Base case. $ \phi $ is an atomic formula $ R\rb{t_1, \dots, t_n} $, where $ R $ is an $ n $-ary relation symbol and $ t_1, \dots, t_n $ are terms. Then
\begin{align*}
v\sb{\phi} = T \qquad
& \iff \qquad \overline{R}\rb{v\rb{t_1}, \dots, v\rb{t_n}} \ \text{holds in} \ \AA \\
& \iff \qquad \overline{R}\rb{\widehat{v\rb{t_1}}, \dots, \widehat{v\rb{t_n}}} \ \text{holds in} \ \AA & \text{by definition of} \ \overline{R} \ \text{in} \ \widehat{\AA} \\
& \iff \qquad \overline{R}\rb{\widehat{v}\rb{t_1}, \dots, \widehat{v}\rb{t_n}} \ \text{holds in} \ \AA \\
& \iff \qquad \widehat{v}\sb{\phi} = T,
\end{align*}
as required.
\item Inductive step.
\begin{enumerate}[leftmargin=0.5in, label=Case \arabic*.]
\item $ \phi $ is $ \notb{\psi} $. Exercise.
\item $ \phi $ is $ \impb{\theta}{\chi} $. Exercise.
\item $ \phi $ is $ \fab{x_i}\psi $.
\begin{itemize}
\item[$ \implies $] If $ v\sb{\fab{x_i}\psi} = F $, there is a $ v' $, $ x_i $-equivalent to $ v $, with $ v'\sb{\psi} = F $. Then $ \widehat{v'} $ is $ x_i $-equivalent to $ \widehat{v} $, and by the inductive hypothesis,
$$ \widehat{v'}\sb{\psi} = F. $$
So $ \widehat{v}\sb{\fab{x_i}\psi} = F $.
\item[$ \impliedby $] Suppose $ \widehat{v}\sb{\fab{x_i}\psi} = F $. So there is a valuation $ w $ in $ \widehat{\AA} $ which is $ x_i $-equivalent to $ \widehat{v} $ and $ w\sb{\psi} = F $. There is a valuation $ v' $ in $ \AA $, $ x_i $-equivalent to $ v $, with $ \widehat{v'} = w $. We just change $ v\rb{x_i} $, so
$$ \widehat{v'\rb{x_i}} = w\rb{x_i}. $$
Then $ v'\sb{\psi} = F $, by inductive hypothesis. So $ v\sb{\fab{x_i}\psi} = F $.
\end{itemize}
\end{enumerate}
\end{itemize}
\end{proof}

\pagebreak

\subsection{Theorem \ref{thm:3.6.1}}

\begin{proof}[Proof of \ref{thm:3.6.1}]
Suppose $ \alpha $ is an ordinal and $ t $ is a function with domain $ \alpha^\dagger $ such that, for all $ \beta \le \alpha $, we have that $ t\rb{\beta} = F\rb{t \upharpoonright \beta} $. For the purposes of the proof, we say that $ t $ is a \textbf{computation of length $ \alpha $ based on $ F $}. Note that, in this case, if $ \gamma \le \alpha $, then
$$ t \upharpoonright \cb{\beta \mid \beta \le \gamma} $$
is a computation of length $ \gamma $ based on $ F $. The main thing to prove is the claim that, for each ordinal $ \alpha $, there is a unique computation $ t $ based on $ F $ of length $ \alpha $. Once we have this, consider the property $ G\rb{x, y} $ given by
$$ \text{either} \ x \ \text{is an ordinal and} \ y = t\rb{x}, \ \text{or} \ x \ \text{is not an ordinal and} \ y = \emptyset, $$
where $ t $ is a computation based on $ F $ of length $ x $. According to the claim, $ G\rb{x, y} $ defines an operation $ G $, with
$$ G\rb{a} = b \qquad \iff \qquad G\rb{a, b} \ \text{holds}. $$
Moreover, by definition, we have that, if $ \alpha $ is an ordinal, then $ G\rb{\alpha} = t\rb{\alpha} $ for the unique computation $ t $ of length $ \alpha $. It follows, by the uniqueness, and the fact that the restriction of a computation to a successor initial segment is a computation, that $ G\rb{\beta} = t\rb{\beta} $ for all $ \beta \le \alpha $. So
$$ t \upharpoonright \alpha = G \upharpoonright \alpha, $$
and we have $ G\rb{\alpha} = F\rb{G \upharpoonright \alpha} $, as required. To prove the claim, we use transfinite induction, in \ref{thm:3.5.1}. So assume that, for all $ \beta < \alpha $, there is a unique computation $ t_\beta $ of length $ \beta $, based on $ F $. We must show that there is a unique computation of length $ \alpha $. This comes in two parts, existence and uniqueness.
\begin{itemize}
\item Existence. Note that, if $ \gamma < \beta < \alpha $, then the uniqueness part of the inductive hypothesis implies that $ t_\gamma $ must be the restriction of $ t_\beta $ to $ \gamma^\dagger $. So
$$ s = \bigcup_{\beta < \alpha} t_\beta $$
is a function with domain $ \alpha $, and
$$ s\rb{\beta} = t_\beta\rb{\beta} = F\rb{s \upharpoonright \beta}, $$
for all $ \beta < \alpha $. We extend $ s $ to a function $ t $ with domain $ \alpha^\dagger $ by setting $ t\rb{\alpha} = F\rb{s} $. Then for all $ \beta \le \alpha $, we have that $ t\rb{\beta} = F\rb{t \upharpoonright \beta} $. So $ t $ is a computation of length $ \alpha $ based on $ F $.
\item Uniqueness. Suppose $ t, t' $ are computations of length $ \alpha $. We must show $ t\rb{\beta} = t'\rb{\beta} $ for all $ \beta \le \alpha $. We do this by transfinite induction on $ \beta $. Indeed, if $ t\rb{\gamma} = t'\rb{\gamma} $ for all $ \gamma < \beta $, then
$$ t\rb{\beta} = F\rb{t \upharpoonright \beta} = F\rb{t' \upharpoonright \beta} = t'\rb{\beta}. $$
\end{itemize}
So this establishes the inductive step, and proves the claim. As we have seen, \ref{thm:3.6.1} then follows.
\end{proof}

There is a reformulation which is convenient in practice. First, some terminology. We say that an ordinal $ \alpha $ is a \textbf{successor ordinal} if there is an ordinal $ \beta $ such that $ \alpha = \beta^\dagger $. Otherwise it is a \textbf{limit ordinal}. It is easy to see that an ordinal $ \alpha $ is a limit ordinal if and only if
$$ \alpha = \bigcup \cb{\beta \mid \beta < \alpha}. $$

\begin{example*}
$ 0 $ and $ \omega $ are limit ordinals.
\end{example*}

\begin{theorem}[Definition by transfinite recursion]
Let $ F_1, F_2 $ be operations on sets and $ a_0 $ a set. Then there is an operation $ G $ such that, for all ordinals $ \alpha $,
\begin{itemize}
\item $ G\rb{\emptyset} = a_0 $,
\item $ G\rb{\alpha^\dagger} = F_1\rb{G \upharpoonright \alpha^\dagger} $, and
\item $ G\rb{\alpha} = F_2\rb{G \upharpoonright \alpha} $ if $ \alpha $ is a limit ordinal.
\end{itemize}
The operation $ G $ is uniquely determined on ordinals.
\end{theorem}

The second version of transfinite recursion is obtained simply by combining $ F_1 $ and $ F_2 $ into a suitable property $ F $. We omit the details.

\end{document}