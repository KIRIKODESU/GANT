\def\module{M3P15 Algebraic Number Theory}
\def\lecturer{Dr Ana Caraiani}
\def\term{Spring 2019}

\def\thm{subsection}

\documentclass{article}

% Packages

\usepackage{amssymb}
\usepackage{amsthm}
\usepackage[UKenglish]{babel}
\usepackage{commath}
\usepackage{enumitem}
\usepackage{etoolbox}
\usepackage{fancyhdr}
\usepackage[margin=1in]{geometry}
\usepackage{graphicx}
\usepackage[hidelinks]{hyperref}
\usepackage[utf8]{inputenc}
\usepackage{listings}
\usepackage{mathtools}
\usepackage{stmaryrd}
\usepackage{tikz-cd}
\usepackage{csquotes}

% Formatting

\addto\captionsUKenglish{\renewcommand{\abstractname}{Syllabus}}
\delimitershortfall5pt
\ifx\thm\undefined\newtheorem{n}{}\else\newtheorem{n}{}[\thm]\fi
\newcommand\newoperator[1]{\ifcsdef{#1}{\cslet{#1}{\relax}}{}\csdef{#1}{\operatorname{#1}}}
\setlength{\parindent}{0cm}

% Environments

\theoremstyle{plain}
\newtheorem{algorithm}[n]{Algorithm}
\newtheorem*{algorithm*}{Algorithm}
\newtheorem{algorithm**}{Algorithm}
\newtheorem{conjecture}[n]{Conjecture}
\newtheorem*{conjecture*}{Conjecture}
\newtheorem{conjecture**}{Conjecture}
\newtheorem{corollary}[n]{Corollary}
\newtheorem*{corollary*}{Corollary}
\newtheorem{corollary**}{Corollary}
\newtheorem{lemma}[n]{Lemma}
\newtheorem*{lemma*}{Lemma}
\newtheorem{lemma**}{Lemma}
\newtheorem{proposition}[n]{Proposition}
\newtheorem*{proposition*}{Proposition}
\newtheorem{proposition**}{Proposition}
\newtheorem{theorem}[n]{Theorem}
\newtheorem*{theorem*}{Theorem}
\newtheorem{theorem**}{Theorem}

\theoremstyle{definition}
\newtheorem{aim}[n]{Aim}
\newtheorem*{aim*}{Aim}
\newtheorem{aim**}{Aim}
\newtheorem{axiom}[n]{Axiom}
\newtheorem*{axiom*}{Axiom}
\newtheorem{axiom**}{Axiom}
\newtheorem{condition}[n]{Condition}
\newtheorem*{condition*}{Condition}
\newtheorem{condition**}{Condition}
\newtheorem{definition}[n]{Definition}
\newtheorem*{definition*}{Definition}
\newtheorem{definition**}{Definition}
\newtheorem{example}[n]{Example}
\newtheorem*{example*}{Example}
\newtheorem{example**}{Example}
\newtheorem{exercise}[n]{Exercise}
\newtheorem*{exercise*}{Exercise}
\newtheorem{exercise**}{Exercise}
\newtheorem{fact}[n]{Fact}
\newtheorem*{fact*}{Fact}
\newtheorem{fact**}{Fact}
\newtheorem{goal}[n]{Goal}
\newtheorem*{goal*}{Goal}
\newtheorem{goal**}{Goal}
\newtheorem{law}[n]{Law}
\newtheorem*{law*}{Law}
\newtheorem{law**}{Law}
\newtheorem{plan}[n]{Plan}
\newtheorem*{plan*}{Plan}
\newtheorem{plan**}{Plan}
\newtheorem{problem}[n]{Problem}
\newtheorem*{problem*}{Problem}
\newtheorem{problem**}{Problem}
\newtheorem{question}[n]{Question}
\newtheorem*{question*}{Question}
\newtheorem{question**}{Question}
\newtheorem{warning}[n]{Warning}
\newtheorem*{warning*}{Warning}
\newtheorem{warning**}{Warning}
\newtheorem{acknowledgements}[n]{Acknowledgements}
\newtheorem*{acknowledgements*}{Acknowledgements}
\newtheorem{acknowledgements**}{Acknowledgements}
\newtheorem{annotations}[n]{Annotations}
\newtheorem*{annotations*}{Annotations}
\newtheorem{annotations**}{Annotations}
\newtheorem{assumption}[n]{Assumption}
\newtheorem*{assumption*}{Assumption}
\newtheorem{assumption**}{Assumption}
\newtheorem{conclusion}[n]{Conclusion}
\newtheorem*{conclusion*}{Conclusion}
\newtheorem{conclusion**}{Conclusion}
\newtheorem{claim}[n]{Claim}
\newtheorem*{claim*}{Claim}
\newtheorem{claim**}{Claim}
\newtheorem{notation}[n]{Notation}
\newtheorem*{notation*}{Notation}
\newtheorem{notation**}{Notation}
\newtheorem{note}[n]{Note}
\newtheorem*{note*}{Note}
\newtheorem{note**}{Note}
\newtheorem{remark}[n]{Remark}
\newtheorem*{remark*}{Remark}
\newtheorem{remark**}{Remark}

% Lectures

\newcommand{\lecture}[3]{ % Lecture
  \marginpar{
    Lecture #1 \\
    #2 \\
    #3
  }
}

% Blackboard

\renewcommand{\AA}{\mathbb{A}} % Blackboard A
\newcommand{\BB}{\mathbb{B}}   % Blackboard B
\newcommand{\CC}{\mathbb{C}}   % Blackboard C
\newcommand{\DD}{\mathbb{D}}   % Blackboard D
\newcommand{\EE}{\mathbb{E}}   % Blackboard E
\newcommand{\FF}{\mathbb{F}}   % Blackboard F
\newcommand{\GG}{\mathbb{G}}   % Blackboard G
\newcommand{\HH}{\mathbb{H}}   % Blackboard H
\newcommand{\II}{\mathbb{I}}   % Blackboard I
\newcommand{\JJ}{\mathbb{J}}   % Blackboard J
\newcommand{\KK}{\mathbb{K}}   % Blackboard K
\newcommand{\LL}{\mathbb{L}}   % Blackboard L
\newcommand{\MM}{\mathbb{M}}   % Blackboard M
\newcommand{\NN}{\mathbb{N}}   % Blackboard N
\newcommand{\OO}{\mathbb{O}}   % Blackboard O
\newcommand{\PP}{\mathbb{P}}   % Blackboard P
\newcommand{\QQ}{\mathbb{Q}}   % Blackboard Q
\newcommand{\RR}{\mathbb{R}}   % Blackboard R
\renewcommand{\SS}{\mathbb{S}} % Blackboard S
\newcommand{\TT}{\mathbb{T}}   % Blackboard T
\newcommand{\UU}{\mathbb{U}}   % Blackboard U
\newcommand{\VV}{\mathbb{V}}   % Blackboard V
\newcommand{\WW}{\mathbb{W}}   % Blackboard W
\newcommand{\XX}{\mathbb{X}}   % Blackboard X
\newcommand{\YY}{\mathbb{Y}}   % Blackboard Y
\newcommand{\ZZ}{\mathbb{Z}}   % Blackboard Z

% Brackets

\renewcommand{\eval}[1]{\left. #1 \right|}          % Evaluation
\newcommand{\br}{\del}                              % Brackets
\newcommand{\abr}[1]{\left\langle #1 \right\rangle} % Angle brackets
\newcommand{\fbr}[1]{\left\lfloor #1 \right\rfloor} % Floor brackets
\newcommand{\lbr}[1]{\left\lfloor #1 \right\rfloor} % Ceiling brackets
\newcommand{\st}{\ \middle| \ }                     % Such that

% Calligraphic

\newcommand{\AAA}{\mathcal{A}} % Calligraphic A
\newcommand{\BBB}{\mathcal{B}} % Calligraphic B
\newcommand{\CCC}{\mathcal{C}} % Calligraphic C
\newcommand{\DDD}{\mathcal{D}} % Calligraphic D
\newcommand{\EEE}{\mathcal{E}} % Calligraphic E
\newcommand{\FFF}{\mathcal{F}} % Calligraphic F
\newcommand{\GGG}{\mathcal{G}} % Calligraphic G
\newcommand{\HHH}{\mathcal{H}} % Calligraphic H
\newcommand{\III}{\mathcal{I}} % Calligraphic I
\newcommand{\JJJ}{\mathcal{J}} % Calligraphic J
\newcommand{\KKK}{\mathcal{K}} % Calligraphic K
\newcommand{\LLL}{\mathcal{L}} % Calligraphic L
\newcommand{\MMM}{\mathcal{M}} % Calligraphic M
\newcommand{\NNN}{\mathcal{N}} % Calligraphic N
\newcommand{\OOO}{\mathcal{O}} % Calligraphic O
\newcommand{\PPP}{\mathcal{P}} % Calligraphic P
\newcommand{\QQQ}{\mathcal{Q}} % Calligraphic Q
\newcommand{\RRR}{\mathcal{R}} % Calligraphic R
\newcommand{\SSS}{\mathcal{S}} % Calligraphic S
\newcommand{\TTT}{\mathcal{T}} % Calligraphic T
\newcommand{\UUU}{\mathcal{U}} % Calligraphic U
\newcommand{\VVV}{\mathcal{V}} % Calligraphic V
\newcommand{\WWW}{\mathcal{W}} % Calligraphic W
\newcommand{\XXX}{\mathcal{X}} % Calligraphic X
\newcommand{\YYY}{\mathcal{Y}} % Calligraphic Y
\newcommand{\ZZZ}{\mathcal{Z}} % Calligraphic Z

% Fraktur

\newcommand{\aaa}{\mathfrak{a}}   % Fraktur a
\newcommand{\bbb}{\mathfrak{b}}   % Fraktur b
\newcommand{\ccc}{\mathfrak{c}}   % Fraktur c
\newcommand{\ddd}{\mathfrak{d}}   % Fraktur d
\newcommand{\eee}{\mathfrak{e}}   % Fraktur e
\newcommand{\fff}{\mathfrak{f}}   % Fraktur f
\renewcommand{\ggg}{\mathfrak{g}} % Fraktur g
\newcommand{\hhh}{\mathfrak{h}}   % Fraktur h
\newcommand{\iii}{\mathfrak{i}}   % Fraktur i
\newcommand{\jjj}{\mathfrak{j}}   % Fraktur j
\newcommand{\kkk}{\mathfrak{k}}   % Fraktur k
\renewcommand{\lll}{\mathfrak{l}} % Fraktur l
\newcommand{\mmm}{\mathfrak{m}}   % Fraktur m
\newcommand{\nnn}{\mathfrak{n}}   % Fraktur n
\newcommand{\ooo}{\mathfrak{o}}   % Fraktur o
\newcommand{\ppp}{\mathfrak{p}}   % Fraktur p
\newcommand{\qqq}{\mathfrak{q}}   % Fraktur q
\newcommand{\rrr}{\mathfrak{r}}   % Fraktur r
\newcommand{\sss}{\mathfrak{s}}   % Fraktur s
\newcommand{\ttt}{\mathfrak{t}}   % Fraktur t
\newcommand{\uuu}{\mathfrak{u}}   % Fraktur u
\newcommand{\vvv}{\mathfrak{v}}   % Fraktur v
\newcommand{\www}{\mathfrak{w}}   % Fraktur w
\newcommand{\xxx}{\mathfrak{x}}   % Fraktur x
\newcommand{\yyy}{\mathfrak{y}}   % Fraktur y
\newcommand{\zzz}{\mathfrak{z}}   % Fraktur z

% Geometry

\newcommand{\CP}{\mathbb{CP}}                                              % Complex projective space
\newcommand{\iintd}[4]{\iint_{#1} \, #2 \, \dif #3 \, \dif #4}             % Double integral
\newcommand{\RP}{\mathbb{RP}}                                              % Real projective space
\newcommand{\intd}[4]{\int_{#1}^{#2} \, #3 \, \dif #4}                     % Single integral
\newcommand{\iiintd}[5]{\iint_{#1} \, #2 \, \dif #3 \, \dif #4 \, \dif #5} % Triple integral

% Logic

\newcommand{\iffb}[2]{\br{#1 \leftrightarrow #2}} % Biconditional
\newcommand{\andb}[2]{\br{#1 \land #2}}           % Conjunction
\newcommand{\orb}[2]{\br{#1 \lor #2}}             % Disjunction
\newcommand{\nib}[2]{\br{#1 \notin #2}}           % Element of
\newcommand{\eqb}[2]{\br{#1 = #2}}                % Equal to
\newcommand{\teb}[1]{\br{\exists #1}}             % Existential quantifier
\newcommand{\impb}[2]{\br{#1 \rightarrow #2}}     % Implication
\newcommand{\ltb}[2]{\br{#1 < #2}}                % Less than
\newcommand{\leb}[2]{\br{#1 \le #2}}              % Less than or equal to
\newcommand{\notb}[1]{\br{\neg #1}}               % Negation
\newcommand{\inb}[2]{\br{#1 \in #2}}              % Not element of
\newcommand{\neb}[2]{\br{#1 \ne #2}}              % Not equal to
\newcommand{\subb}[2]{\br{#1 \subseteq #2}}       % Subset
\newcommand{\fab}[1]{\br{\forall #1}}             % Universal quantifier

% Maps

\newcommand{\bijection}[7][]{    % Bijection
  \ifx &#1&
    \begin{array}{rcl}
      #2 & \longleftrightarrow & #3 \\
      #4 & \longmapsto         & #5 \\
      #6 & \longmapsfrom       & #7
    \end{array}
  \else
    \begin{array}{ccrcl}
      #1 & : & #2 & \longrightarrow & #3 \\
         &   & #4 & \longmapsto     & #5 \\
         &   & #6 & \longmapsfrom   & #7
    \end{array}
  \fi
}
\newcommand{\birational}[7][]{   % Birational map
  \ifx &#1&
    \begin{array}{rcl}
      #2 & \dashrightarrow & #3 \\
      #4 & \longmapsto     & #5 \\
      #6 & \longmapsfrom   & #7
    \end{array}
  \else
    \begin{array}{ccrcl}
      #1 & : & #2 & \dashrightarrow & #3 \\
         &   & #4 & \longmapsto     & #5 \\
         &   & #6 & \longmapsfrom   & #7
    \end{array}
  \fi
}
\newcommand{\correspondence}[2]{ % Correspondence
  \cbr{
    \begin{array}{c}
      #1
    \end{array}
  }
  \qquad
  \leftrightsquigarrow
  \qquad
  \cbr{
    \begin{array}{c}
      #2
    \end{array}
  }
}
\newcommand{\function}[5][]{     % Function
  \ifx &#1&
    \begin{array}{rcl}
      #2 & \longrightarrow & #3 \\
      #4 & \longmapsto     & #5
    \end{array}
  \else
    \begin{array}{ccrcl}
      #1 & : & #2 & \longrightarrow & #3 \\
         &   & #4 & \longmapsto     & #5
    \end{array}
  \fi
}
\newcommand{\functions}[7][]{    % Functions
  \ifx &#1&
    \begin{array}{rcl}
      #2 & \longrightarrow & #3 \\
      #4 & \longmapsto     & #5 \\
      #6 & \longmapsto     & #7
    \end{array}
  \else
    \begin{array}{ccrcl}
      #1 & : & #2 & \longrightarrow & #3 \\
         &   & #4 & \longmapsto     & #5 \\
         &   & #6 & \longmapsto     & #7
    \end{array}
  \fi
}
\newcommand{\rational}[5][]{     % Rational map
  \ifx &#1&
    \begin{array}{rcl}
      #2 & \dashrightarrow & #3 \\
      #4 & \longmapsto     & #5
    \end{array}
  \else
    \begin{array}{ccrcl}
      #1 & : & #2 & \dashrightarrow & #3 \\
         &   & #4 & \longmapsto     & #5
    \end{array}
  \fi
}

% Matrices

\newcommand{\onebytwo}[2]{      % One by two matrix
  \begin{pmatrix}
    #1 & #2
  \end{pmatrix}
}
\newcommand{\onebythree}[3]{    % One by three matrix
  \begin{pmatrix}
    #1 & #2 & #3
  \end{pmatrix}
}
\newcommand{\twobyone}[2]{      % Two by one matrix
  \begin{pmatrix}
    #1 \\
    #2
  \end{pmatrix}
}
\newcommand{\twobytwo}[4]{      % Two by two matrix
  \begin{pmatrix}
    #1 & #2 \\
    #3 & #4
  \end{pmatrix}
}
\newcommand{\threebyone}[3]{    % Three by one matrix
  \begin{pmatrix}
    #1 \\
    #2 \\
    #3
  \end{pmatrix}
}
\newcommand{\threebythree}[9]{  % Three by three matrix
  \begin{pmatrix}
    #1 & #2 & #3 \\
    #4 & #5 & #6 \\
    #7 & #8 & #9
  \end{pmatrix}
}
\newcommand{\twobytwosmall}[4]{ % Two by two small matrix
  \begin{psmallmatrix}
    #1 & #2 \\
    #3 & #4
  \end{psmallmatrix}
}

% Number theory

\renewcommand{\symbol}[2]{\br{\tfrac{#1}{#2}}} % Power residue symbol
\newcommand{\unit}[1]{\br{\ZZ / #1\ZZ}^\times} % Unit group

% Operators

\newoperator{ab}    % Abelian
\newoperator{AG}    % Affine geometry
\newoperator{alg}   % Algebraic
\newoperator{Ann}   % Annihilator
\newoperator{area}  % Area
\newoperator{Aut}   % Automorphism
\newoperator{card}  % Cardinality
\newoperator{ch}    % Characteristic
\newoperator{Cl}    % Class
\newoperator{Coker} % Cokernel
\newoperator{col}   % Column
\newoperator{Corr}  % Correspondence
\newoperator{diam}  % Diameter
\newoperator{Disc}  % Discriminant
\newoperator{dom}   % Domain
\newoperator{Eig}   % Eigenvalue
\newoperator{Em}    % Embedding
\newoperator{End}   % Endomorphism
\newoperator{fin}   % Finite
\newoperator{Fix}   % Fixed
\newoperator{Frac}  % Fraction
\newoperator{Frob}  % Frobenius
\newoperator{Fun}   % Function
\newoperator{Gal}   % Galois
\newoperator{GL}    % General linear
\newoperator{Ham}   % Hamming
\newoperator{Homeo} % Homeomorphism
\newoperator{Hom}   % Homomorphism
\newoperator{id}    % Identity
\newoperator{Im}    % Image
\newoperator{Ind}   % Index
\newoperator{Ker}   % Kernel
\newoperator{lcm}   % Least common multiple
\newoperator{Mat}   % Matrix
\newoperator{mult}  % Multiplicity
\newoperator{new}   % New
\newoperator{Nm}    % Norm
\newoperator{old}   % Old
\newoperator{op}    % Opposite
\newoperator{ord}   % Order
\newoperator{Pay}   % Payley
\newoperator{PG}    % Projective geometry
\newoperator{PGL}   % Projective general linear
\newoperator{PSL}   % Projective special linear
\newoperator{rad}   % Radical
\newoperator{ran}   % Range
\newoperator{Res}   % Residue
\newoperator{rk}    % Rank
\newoperator{Re}    % Real
\newoperator{row}   % Row
\newoperator{sgn}   % Sign
\newoperator{Sing}  % Singular
\newoperator{SK}    % Skeleton
\newoperator{sp}    % Span
\newoperator{SL}    % Special linear
\newoperator{SO}    % Special orthogonal
\newoperator{Spec}  % Spectrum
\newoperator{Stab}  % Stabiliser
\newoperator{star}  % Star
\newoperator{srg}   % Strongly regular graph
\newoperator{supp}  % Support
\newoperator{Sym}   % Symmetric
\newoperator{tors}  % Torsion
\newoperator{Tr}    % Trace
\newoperator{vol}   % Volume
\newoperator{wt}    % Weight

% Roman

\newcommand{\A}{\mathrm{A}}   % Roman A
\newcommand{\B}{\mathrm{B}}   % Roman B
\newcommand{\C}{\mathrm{C}}   % Roman C
\newcommand{\D}{\mathrm{D}}   % Roman D
\newcommand{\E}{\mathrm{E}}   % Roman E
\newcommand{\F}{\mathrm{F}}   % Roman F
\newcommand{\G}{\mathrm{G}}   % Roman G
\renewcommand{\H}{\mathrm{H}} % Roman H
\newcommand{\I}{\mathrm{I}}   % Roman I
\newcommand{\J}{\mathrm{J}}   % Roman J
\newcommand{\K}{\mathrm{K}}   % Roman K
\renewcommand{\L}{\mathrm{L}} % Roman L
\newcommand{\M}{\mathrm{M}}   % Roman M
\newcommand{\N}{\mathrm{N}}   % Roman N
\renewcommand{\O}{\mathrm{O}} % Roman O
\renewcommand{\P}{\mathrm{P}} % Roman P
\newcommand{\Q}{\mathrm{Q}}   % Roman Q
\newcommand{\R}{\mathrm{R}}   % Roman R
\renewcommand{\S}{\mathrm{S}} % Roman S
\newcommand{\T}{\mathrm{T}}   % Roman T
\newcommand{\U}{\mathrm{U}}   % Roman U
\newcommand{\V}{\mathrm{V}}   % Roman V
\newcommand{\W}{\mathrm{W}}   % Roman W
\newcommand{\X}{\mathrm{X}}   % Roman X
\newcommand{\Y}{\mathrm{Y}}   % Roman Y
\newcommand{\Z}{\mathrm{Z}}   % Roman Z

\renewcommand{\a}{\mathrm{a}} % Roman a
\renewcommand{\b}{\mathrm{b}} % Roman b
\renewcommand{\c}{\mathrm{c}} % Roman c
\renewcommand{\d}{\mathrm{d}} % Roman d
\newcommand{\e}{\mathrm{e}}   % Roman e
\newcommand{\f}{\mathrm{f}}   % Roman f
\newcommand{\g}{\mathrm{g}}   % Roman g
\newcommand{\h}{\mathrm{h}}   % Roman h
\renewcommand{\i}{\mathrm{i}} % Roman i
\renewcommand{\j}{\mathrm{j}} % Roman j
\renewcommand{\k}{\mathrm{k}} % Roman k
\renewcommand{\l}{\mathrm{l}} % Roman l
\newcommand{\m}{\mathrm{m}}   % Roman m
\renewcommand{\n}{\mathrm{n}} % Roman n
\renewcommand{\o}{\mathrm{o}} % Roman o
\newcommand{\p}{\mathrm{p}}   % Roman p
\newcommand{\q}{\mathrm{q}}   % Roman q
\renewcommand{\r}{\mathrm{r}} % Roman r
\newcommand{\s}{\mathrm{s}}   % Roman s
\renewcommand{\t}{\mathrm{t}} % Roman t
\renewcommand{\u}{\mathrm{u}} % Roman u
\renewcommand{\v}{\mathrm{v}} % Roman v
\newcommand{\w}{\mathrm{w}}   % Roman w
\newcommand{\x}{\mathrm{x}}   % Roman x
\newcommand{\y}{\mathrm{y}}   % Roman y
\newcommand{\z}{\mathrm{z}}   % Roman z

% Tikz

\tikzset{
  arrow symbol/.style={"#1" description, allow upside down, auto=false, draw=none, sloped},
  subset/.style={arrow symbol={\subset}},
  cong/.style={arrow symbol={\cong}}
}

% Fancy header

\pagestyle{fancy}
\lhead{\module}
\rhead{\nouppercase{\leftmark}}

% Make title

\title{\module}
\author{Lectured by \lecturer \\ Typed by David Kurniadi Angdinata}
\date{\term}

\begin{document}

% Title page
\maketitle
\cover
\vfill
\begin{abstract}
\noindent\syllabus
\end{abstract}

\pagebreak

% Contents page
\tableofcontents

\pagebreak

% Document page
\setcounter{section}{-1}

\section{Motivation and overview}

\lecture{1}{Friday}{11/01/19}

The goal of this course will be to introduce algebraic number theory, specifically the arithmetic of finite extensions of $ \Q $, with an emphasis on quadratic extensions as a rich source of examples. We will start with some motivation and then review the necessary background from ring theory. We will then discuss unique factorisation domains, principal ideal domains and Euclidean domains. These tools will be enough to study Gaussian integers and Eisenstein integers in-depth. To understand more general number fields, we will need some more commutative algebra. We will discuss the structure theorem for finitely generated abelian groups and the notion of integral closure. We will also introduce norms, traces, and discriminants. We will show that rings of integers in number fields are Dedekind domains and we will state and prove unique factorisation for Dedekind domains. We will then study the splitting of prime ideals in quadratic fields. We will define the class group and prove that it is always finite. We will end with a discussion of the groups of units. For quadratic fields, a good reference with many examples is $ 2 $. Another reference we will use is $ 1 $.
\begin{enumerate}
\item P Samuel, Algebraic theory of numbers, 1970
\item M Trifkovic, Algebraic theory of quadratic numbers, 2013
\end{enumerate}

Algebraic number theory developed from
\begin{itemize}
\item trying to generalise known properties of integers, such as unique factorisation, to finite extensions of $ \Q $,
\item trying to solve Diophantine equations in a systematic way. For example, Fermat's equation
$$ x^n + y^n = z^n, \qquad n \ge 2, \qquad x, y, z \in \Z. $$
\end{itemize}

Let $ n \in \Z_{\ge 0} $. A question is when can we write $ n $ as
$$ n = a^2 + b^2, \qquad a, b \in \Z? $$
Some observations.
\begin{itemize}
\item If $ n = a_1^2 + b_1^2 $, $ m = a_2^2 + b_2^2 $,
$$ m \cdot n = \rb{a_1a_2 + b_1b_2}^2 + \rb{a_1b_2 - a_2b_1}^2. $$
\item Every $ n \ge 0 $ can be written as a product
$$ n = p_1^{k_1} \dots p_r^{k_r}, \qquad k_i \in \Z_{\ge 1}, $$
where $ p_i $ are prime numbers. Irreducibles are such that only divisors are $ 1 $ and $ p_i $. Primes are such that $ p_i \mid mn $ gives $ p_i \mid m $ or $ p_i \mid n $. Irreducibles and primes are equivalent in $ \Z $.
\item Only care about $ p_i $ with odd exponent.
\end{itemize}
When can we write
$$ p = a^2 + b^2, \qquad a, b \in \Z, $$
where $ p $ is prime? An observation is that
$$ p = 2, 5, 13, 17, 29, 37, \dots $$
is ok, and
$$ p \ne 3, 7, 11, 19, 23, \dots $$
is not ok. A conjecture is if $ p \equiv 3 \mod 4 $, then $ p \ne a^2 + b^2 $, otherwise this is ok.

\begin{theorem}
If $ p \equiv 3 \mod 4 $ then $ p \ne a^2 + b^2 $.
\end{theorem}

\begin{proof}
$ a^2 + b^2 \equiv 0 \mod p $ and $ a, b \not\equiv 0 \mod p $ if and only if
$$ \rb{\dfrac{a}{b}}^2 \equiv -1 \mod p, $$
if and only if $ \jacobi{-1}{p} = 1 $, so $ p \equiv 3 \mod 4 $.
\end{proof}

\begin{remark*}
Proof tells us that $ n \ne a^2 + b^2 $ whenever $ n $ has a prime factor $ p_i \equiv 3 \mod 4 $ with odd exponent $ k_i $ for $ i = 1, \dots, r $. If every $ p \equiv 1 \mod 4 $ is of the form $ p = a^2 + b^2 $, then we understand the general case,
$$ n = a^2 + b^2 \qquad \iff \qquad \forall p_i \mid n, \ p_i \equiv 3 \mod 4, \ k_i \in 2\Z. $$
\end{remark*}

\begin{theorem}
If $ p \equiv 1 \mod 4 $ then
$$ p = a^2 + b^2, \qquad a, b \in \Z. $$
\end{theorem}

Factorisation in $ \Z\sb{i} $ for $ i^2 = 1 $ is $ p = a^2 + b^2 = \rb{a + bi}\rb{a - bi} $ for $ a, b \in \Z $.
$$ \Z\sb{i} = \Z \oplus \Z i = \cb{a + bi \mid a, b \in \Z} $$
is the subring of \textbf{Gaussian integers} in $ \Q\rb{i} / \Q $, an extension $ \Q\sb{x} / \rb{x^2 + 1} $ of $ \Q $ of degree two, a quadratic field. We will understand prime factorisation in $ \Z\sb{i} $, and in more general finite extensions of $ \Q $.

\begin{theorem}[Unique factorisation in $ \Z $]
\label{thm:uniquefactorisationinz}
Any $ n \in \Z \setminus \cb{0, \pm 1} $ can be written uniquely as a product of primes, up to permuting the prime factors or changing their signs.
\end{theorem}

\begin{proposition}[Division algorithm]
Given $ a, b \in \Z $, $ b \ne 0 $, there exist $ q, r \in \Z $ such that $ a = qb + r $ such that $ 0 \le r < \abs{b} $.
\end{proposition}

\begin{proposition}[Euclid's algorithm]
Let $ a, b \in \Z $, $ ab \ne 0 $. There exist a greatest common divisor $ \gcd\rb{a, b} \mid a $ and $ \gcd\rb{a, b} \mid b $, and $ r, s \in \Z $ such that $ ar + bs = \gcd\rb{a, b} $.
\end{proposition}

\begin{proof}
Consider $ I = \cb{ma + nb \mid m, n \in \Z} $. $ \gcd\rb{a, b} $ will be the smallest positive element of $ I $.
\end{proof}

Let $ I \subseteq \Z $ be the ideal of $ \Z $ generated by $ a, b $. Proof of Euclid's algorithm shows $ I $ is generated by $ \gcd\rb{a, b} $. In fact, every ideal of $ \Z $ is generated by one element, that is it is \textbf{principal}.

\begin{proposition}[Euclid's lemma]
If $ p \in \Z $ is prime, then
$$ p \mid ab, \qquad a, b \in \Z \qquad \implies \qquad p \mid a \ \text{or} \ p \mid b. $$
\end{proposition}

\begin{proof}[Proof of Theorem \ref{thm:uniquefactorisationinz}]
\hfill
\begin{itemize}
\item All $ n \in \Z $ has a prime divisor by taking $ p \in \Z_{\ge 2} $, the smallest divisor of $ n $.
\item Prime factorisation exists. Let $ n $ be the smallest integer which does not have one.
\item Uniqueness. $ n = p_2 \dots p_n = q_2 \dots q_? $ Euclid's lemma gives $ p_1 \mid q_1 $, up to reordering, so $ p_1 = \pm q_1 $, and continue.
\end{itemize}
\end{proof}

\pagebreak

\section{Rings}

\lecture{2}{Monday}{14/01/19}

\subsection{Commutative rings}

\begin{definition}
A \textbf{ring} is commutative and with unity. A \textbf{unit} in a ring $ R $ is an element $ a \in R $ such that there exists $ b \in R $ with $ a \cdot b = 1 $.
\begin{itemize}
\item The set of units forms a group under multiplication, denoted by $ R^\times $.
\item If $ b \in R $ exists such that $ ab = 1 $ then $ b $ is unique.
\end{itemize}
If $ R \setminus \cb{0} = R^\times $, then $ R $ is a \textbf{field}.
\end{definition}

\begin{example*}
\hfill
\begin{itemize}
\item $ \Z^\times = \cb{\pm 1} $.
\item $ \Q^\times = \Q \setminus \cb{0} $.
\item $ \Z\sb{\sqrt{2}}^\times \supseteq \cb{\pm 1, \epsilon^n} $, where $ \epsilon = 1 + \sqrt{2} $.
$$ \Z\sb{\sqrt{2}} = \cb{a + b\sqrt{2} \mid a, b \in \Z}. $$
$ \rb{1 + \sqrt{2}}\rb{-1 + \sqrt{2}} = 2 - 1 = 1 $. $ \epsilon^n = \epsilon^m $ for $ n, m \in \Z $ and $ n \ge m $ if and only if $ \epsilon^{n - m} = 1 $.
\end{itemize}
\end{example*}

\begin{definition}
Let $ R $ be a ring. An \textbf{ideal} $ I \subseteq R $ is an additive subgroup, so $ x, y \in I $ gives $ x + y \in I $, which absorbs multiplication. If $ x \in I $ and $ a \in R $ then $ ax \in I $.
\end{definition}

\begin{fact*}
If $ \phi : R \to S $ a ring homomorphism then $ Ker\rb{\phi} \subseteq R $ is an ideal. Conversely, if $ I \subseteq R $ is an ideal, can define
$$ \dfrac{R}{I} = \dfrac{R}{\sim} $$
as the set of equivalence classes modulo $ I $, that is $ a + I $ for $ a \in R $, via $ a \sim b $ for $ a, b \in R $ if $ a - b \in I $.
\end{fact*}

\begin{proposition}
$ R / I $ has ring structure induced by
\begin{align*}
\rb{a + I} + \rb{b + I} & = \rb{a + b} + I, \\
\rb{a + I} \cdot \rb{b + I} & = \rb{a \cdot b} + I,
\end{align*}
and a canonical surjective ring homomorphism
$$ \function{R}{\dfrac{R}{I}}{a}{a + I}. $$
\end{proposition}

Check that $ a - a' \in I $ and $ b - b' \in I $ gives
$$ \rb{a + b} - \rb{a' + b'} = \rb{a - a'} + \rb{b - b'} \in I, $$
$$ ab - a'b' = a\rb{b - b'} + b'\rb{a - a'} \in I. $$

\begin{theorem}[First isomorphism theorem for rings]
\label{thm:firstisomorphism}
Let $ \phi : R \to S $ be a ring homomorphism. Then we have a canonical ring isomorphism
$$ \function{\dfrac{R}{Ker\rb{\phi}}}{\phi\rb{R} \subset S}{r + Ker\rb{\phi}}{\phi\rb{r}}, $$
for $ r \in R $.
\end{theorem}

\begin{example*}
Let $ R = \Z\sb{\sqrt{5}} = \cb{a + b\sqrt{5} \mid a, b \in \Z} $.
\begin{itemize}
\item Let $ I $ be the ideal $ 11\Z \oplus \rb{4 - \sqrt{5}}\Z $. A question is what is $ R / I $? Claim that
$$ \dfrac{R}{I} \cong \dfrac{\Z}{11\Z} = \F_{11}, $$
the finite field with $ 11 $ elements. Write down $ \phi : R \twoheadrightarrow \Z / 11\Z $ such that $ Ker\rb{\phi} = I $, then result follows from Theorem \ref{thm:firstisomorphism}. Such a $ \phi $ would have to satisfy
$$ \phi\rb{4 - \sqrt{5}} = 0, \qquad \phi\rb{11} = 0. $$
$ \phi\rb{\sqrt{5}} = \phi\rb{4} = 4 \mod 11 $.
$$ \function[\phi]{\Z \oplus \Z\sb{\sqrt{5}}}{\dfrac{\Z}{11\Z}}{\sqrt{5}}{4}. $$
Still have to check that
$$ 16 = \phi\rb{5}^2 = \phi\rb{\sqrt{5}^2} = \phi\rb{5} = 5 \mod 11. $$
Ok because $ 16 \equiv 5 \mod 11 $.
\item What can we say about $ R / J $, where
$$ J = \ab{9, 4 - \sqrt{5}} = 9R + \rb{4 - \sqrt{5}}R $$
is generated over $ R $? $ R / J $ is trivial and $ \ab{9, 4 - \sqrt{5}} = R $.
\end{itemize}
\end{example*}

\begin{definition}
\hfill
\begin{itemize}
\item If $ I, J $ are ideals in a ring $ R $, we say that $ I $ \textbf{divides} $ J $ if $ J \subseteq I $.
\item We can form ideals
\begin{align*}
I \cap J & = \cb{r \mid r \in I, \ r \in J}, \\
I + J & = \cb{r + s \mid r \in I, \ s \in J}, \\
I \cdot J & = \cb{\sum_{i = 1}^n r_is_i \ \Bigg| \ r_i \in I, \ s_i \in J, \ i = 1, \dots, n}.
\end{align*}
\item $ I, J $ are said to be \textbf{relatively prime} if $ I + J = R $.
\end{itemize}
\end{definition}

\begin{theorem}[Chinese remainder theorem]
\label{thm:chineseremainder}
Let $ I, J $ be two relatively prime ideals of $ R $. Then
$$ \dfrac{R}{IJ} \cong \dfrac{R}{I} \times \dfrac{R}{J}. $$
\end{theorem}

\begin{remark*}
If $ R = \Z $, all ideals are principal and Theorem \ref{thm:chineseremainder} specialises to usual Chinese remainder theorem.
\end{remark*}

\begin{proof}
Find surjective ring homomorphism
$$ \function{R}{\dfrac{R}{I} \times \dfrac{R}{J}}{r}{\rb{r \mod I, r \mod J}}, $$
with kernel $ I \cdot J $.
\end{proof}

\begin{definition}
A ring $ R $ is \textbf{Noetherian} if it satisfies the \textbf{ascending chain condition} on ideals, that is any infinite sequence of ideals
$$ I_1 \subseteq I_2 \subseteq \dots $$
stabilises.
\end{definition}

\begin{example*}
$ \Z $ and $ \Z\sb{x} $ are Noetherian. $ \Z\sb{x_1, x_2, \dots} $ is not Noetherian.
\end{example*}

\pagebreak

\subsection{Integral domains}

\lecture{3}{Tuesday}{15/01/19}

\begin{definition}
A ring $ R $ is an \textbf{integral domain (ID)} if $ ab = 0 $ for $ a, b \in R $ gives $ a = 0 $ or $ b = 0 $.
\end{definition}

\begin{example*}
\hfill
\begin{itemize}
\item $ \Z $ and $ \Z\sb{\sqrt{5}} $ are IDs.
\item $ \Z\sb{\sqrt{5}} / \ab{4 - \sqrt{5}} = \Z / 11\Z = \F_{11} $, since $ I = 11\Z \oplus \rb{4 - \sqrt{5}}\Z = \rb{4 - \sqrt{5}} \cdot \Z\sb{\sqrt{5}} $, because $ 11 = \rb{4 - \sqrt{5}}\rb{4 + \sqrt{5}} = 16 - 5 $. Thus
$$ \dfrac{\Z\sb{\sqrt{5}}}{\ab{11}} \cong \dfrac{\Z\sb{\sqrt{5}}}{\ab{4 - \sqrt{5}}} \times \dfrac{\Z\sb{\sqrt{5}}}{\ab{4 - \sqrt{5}}} = \F_{11} \times \F_{11}, $$
which is no longer an ID.
\end{itemize}
\end{example*}

\begin{remark*}
An ideal $ \pp \subsetneq R $ is \textbf{prime} if $ ab \in \pp $ gives $ a \in \pp $ or $ b \in \pp $. $ \rb{a + \pp}\rb{b + \pp} = 0 $ in $ R / \pp $ gives $ a + \pp = 0 $, that is $ a \in \pp $, or $ b + \pp = 0 $, that is $ b \in \pp $. This is equivalent to asking that $ R / \pp $ is an ID.
\end{remark*}

IDs are well-suited to studying divisibility. $ a \mid b $ in $ R $ if there exists $ c $ such that $ ac = b $.

\begin{lemma}
Let $ R $ be an ID. If $ a \mid b $ and $ b \mid a $, then there exist $ c, d \in R^\times $ such that $ ac = b $ and $ bd = a $.
\end{lemma}

\begin{proof}
$ a \mid b $ gives there exists $ c $ such that $ ac = b $ and $ b \mid a $ gives there exists $ d $ such that $ bd = a $ for $ c, d \in R $. $ acd = bd = a $ if and only if $ a\rb{cd - 1} = 0 $. $ R $ is an ID gives $ a = 0 $ or $ cd = 1 $. If $ a = 0 $, then $ b = 0 $, so $ c = d = 1 $.
\end{proof}

\begin{definition}
Let $ R $ be an ID.
\begin{itemize}
\item We say $ a \in R $ is \textbf{irreducible} if
\begin{itemize}
\item $ a $ is not a unit, and
\item $ a = bc $ for $ b, c \in R $ then either $ b $ or $ c $ is in $ R^\times $.
\end{itemize}
\item We say $ a \in R $ is \textbf{prime} if
\begin{itemize}
\item $ a $ is not a unit, and
\item $ a \mid bc $ gives $ a \mid b $ or $ a \mid c $.
\end{itemize}
\end{itemize}
\end{definition}

$ \ab{0} $ is prime if and only if $ R $ is an ID.

\begin{remark*}
Over $ \Z $, these two notions are equivalent, but not in general. If $ R $ is an ID and $ a \in R \setminus \cb{0} $ is prime, then $ a $ is irreducible.
\end{remark*}

\begin{proof}
Let $ b, c \in R $ be such that $ a = bc $, so $ b \mid a $ and $ c \mid a $. Because $ a $ is prime, $ a = bc $ gives $ a \mid b $ or $ a \mid c $. Say $ a \mid b $ happens. There exists $ d \in R^\times $ such that $ a = bd $. $ a = bc $ gives $ b\rb{d - c} = 0 $. $ b \ne 0 $, because $ a \ne 0 $, so $ d = c $, that is $ c $ is a unit.
\end{proof}

\begin{remark*}
If $ a \in R \setminus \cb{0} $ is irreducible, $ a $ does not have to be prime.
\end{remark*}

\begin{example*}
$ \Z\sb{\sqrt{-5}} = \cb{a + b\sqrt{-5} \mid a, b \in \Z} $ is the ring of integers of $ \Q\rb{\sqrt{-5}} $, an extension of $ \Q $ of degree two, a subring of $ \C $. $ 6 = 2 \cdot 3 = \rb{1 + \sqrt{-5}} \cdot \rb{1 - \sqrt{-5}} $. Claim that these are two factorisations of $ 6 $ into irreducible elements.
\begin{itemize}
\item $ 2 $ is irreducible. Why? Assume $ 2 = \alpha\beta $ for $ \alpha, \beta \in \Z\sb{\sqrt{-5}} $. Goal is that $ \alpha $ or $ \beta $ is a unit. We will use
$$ \function[N]{\Z\sb{\sqrt{-5}}}{\Z_{\ge 0}}{a + \sqrt{-5}b}{\rb{a + \sqrt{-5}b}\rb{a - \sqrt{-5}b} = a^2 + 5b^2}, $$
which is multiplicative. $ N\rb{2} = 4 = N\rb{\alpha}N\rb{\beta} $. If $ N\rb{\alpha} = 1 $, then $ \alpha $ is a unit. $ N\rb{\alpha} = N\rb{\beta} = 2 $ gives $ a^2 + 5b^2 = 2 $, which has no solutions, a contradiction.
\item $ 2 $ and $ 1 + \sqrt{-5} $ do not differ by units, since $ N\rb{2} = 4 $ and $ N\rb{1 + \sqrt{-5}} = 6 $.
\end{itemize}
Upshot is that $ 2 $ is irreducible in $ \Z\sb{\sqrt{-5}} $ but not prime.
\end{example*}

\pagebreak

\subsection{Unique factorisation domains}

Let $ R $ be an ID. We define an equivalence relation $ \sim $ on $ R $ by $ a \sim b $ if $ a \mid b $ and $ b \mid a $, or there exist $ c, d \in R^\times $ such that $ a = bc $ and $ b = da $.

\begin{definition}
An ID $ R $ has \textbf{unique factorisation} if for all $ a \in R \setminus \cb{0} $ there is a factorisation $ a = u \cdot p_1 \cdot \dots \cdot p_r $, where $ u \in R^\times $ and the $ p_i $ are irreducible. This is unique in the sense that, if there exists another factorisation $ v \cdot q_1 \cdot \dots \cdot q_s $, where $ v \in R^\times $ and the $ q_i $ are irreducible, then $ r = s $, and up to reordering $ p_i \sim q_i $, for $ i = 1, \dots, r = s $. An ID with this property is called an \textbf{unique factorisation domain (UFD)},
\end{definition}

\begin{example*}
$ \Z $, but not $ \Z\sb{\sqrt{-5}} $.
\end{example*}

\begin{lemma}
If $ R $ is a UFD, then $ p \in R \setminus \cb{0} $ is irreducible gives $ p $ is prime.
\end{lemma}

\begin{proof}
Exercise.
\end{proof}

\begin{theorem}
\label{thm:ufd}
Let $ R $ be an ID. The following conditions are equivalent.
\begin{itemize}
\item $ R $ is a UFD.
\item $ R $ satisfies ascending chain condition for principal ideals, that is every infinite sequence
$$ \ab{a_1} \subseteq \ab{a_2} \subseteq \dots $$
stabilises after finitely many steps, and every irreducible in $ R $ is prime.
\end{itemize}
\end{theorem}

If $ R $ is a UFD, can define $ d\rb{a} \in \Z_{\ge 0} $ as the number of irreducible factorisations of $ a $. $ d\rb{a} = 0 $ if and only if $ a \in R^\times $ is a unit.

\begin{lemma}
Let $ R $ be a UFD and $ a \mid b $ for $ a, b \in R $. Then
\begin{itemize}
\item $ d\rb{a} \le d\rb{b} $, and
\item $ b \mid a $ if and only if $ d\rb{a} = d\rb{b} $.
\end{itemize}
\end{lemma}

\lecture{4}{Friday}{18/01/19}

\begin{proof}
Let $ a = u \cdot p_1 \cdot \dots \cdot p_{d\rb{a}} $ and $ b = v \cdot q_1 \cdot \dots \cdot q_{d\rb{b}} $. $ a \mid b $ gives $ b = a \cdot c $ for $ c \in R \setminus \cb{0} $. Let $ c = w \cdot r_1 \cdot \dots \cdot r_{d\rb{c}} $.
$$ v \cdot q_1 \cdot \dots \cdot q_{d\rb{b}} = u \cdot w \cdot p_1 \cdot \dots \cdot p_{d\rb{a}} \cdot r_1 \cdot \dots \cdot r_{d\rb{c}}. $$
Uniqueness of factorisation gives $ d\rb{b} = d\rb{a} + d\rb{c} $, so $ d\rb{b} \ge d\rb{a} $. Equality if and only if $ d\rb{c} = 0 $ if and only if $ c $ is a unit, if and only if $ b \mid a $.
\end{proof}

\begin{proof}[Proof of Theorem \ref{thm:ufd}]
\hfill
\begin{itemize}
\item[$ \implies $] Assume $ R $ is a UFD. Irreducibles are prime. Let
$$ \ab{a_1} \subseteq \ab{a_2} \subseteq \dots \qquad \implies \qquad \dots \mid a_2 \mid a_1 \qquad \implies \qquad d\rb{a_1} \ge \dots \ge 0. $$
This sequence stabilises after finitely many steps. There exists $ n $ such that
$$ d\rb{a_n} = d\rb{a_{n + 1}} = \dots \qquad \implies \qquad a_n \sim a_{n + 1} \sim \dots \qquad \implies \qquad \ab{a_n} = \ab{a_{n + 1}} = \dots. $$
\item[$ \impliedby $] For all $ a \in R \setminus \cb{0} $, claim that $ a $ has a factorisation into irreducibles. If $ a_1 = a $, irreducible. Otherwise $ a = b \cdot c $ for $ b, c \in R \setminus \cb{0} $ not units. If both irreducible, done. If not, say $ b $ not irreducible, $ a_2 = b $. $ a = bc $ for $ c $ not a unit gives $ \ab{a} \subsetneq \ab{b} $. Redoing the process here,
$$ \ab{a_1} \subsetneq \ab{a_2} \subsetneq \dots. $$
By ascending chain condition, this process terminates, getting a contradiction, so $ a $ has factorisation into irreducibles. The factorisation of $ a $ is unique, up to units and reordering. Let
$$ a = u \cdot p_1 \cdot \dots \cdot p_r = v \cdot q_1 \cdot \dots \cdot q_s. $$
$ p_1 $ irreducible gives $ p_1 $ is prime, so $ p_1 \mid q_i $ for some $ i $, where $ q_i $ is irreducible, so $ p_1 \sim q_i $. Cancel out $ p_1, q_i $ and repeat.
\end{itemize}
\end{proof}

\begin{remark*}
$ \Z\sb{\sqrt{-5}} $ is not a UFD because $ 2 $ is irreducible but not prime.
\end{remark*}

\pagebreak

\subsection{Principal ideal domains}

\begin{definition}
An ID $ R $ is a \textbf{principal ideal domain (PID)} if every ideal of $ R $ is principal.
\end{definition}

\begin{example*}
\hfill
\begin{itemize}
\item Fields.
\item $ \Z $ follows from Euclid's algorithm.
\end{itemize}
\end{example*}

\begin{theorem}
\label{thm:pid}
A PID $ R $ is a UFD.
\end{theorem}

\begin{proof}
Check two characterising properties.
\begin{itemize}
\item Ascending chain condition. Let
$$ \ab{a_1} \subseteq \ab{a_2} \subseteq \dots. $$
Consider
$$ I = \bigcup_{n = 1}^\infty \ab{a_n}. $$
Claim that $ I $ is an ideal of $ R $. Say $ x \in I $ and $ r \in R $. Want $ rx \in I $. There exists $ n \in \Z_{\ge 1} $ such that $ x \in \ab{a_n} $, so $ rx \in \ab{a_n} $ and $ rx \in I $. Say $ x, y \in I $. Then $ x \in \ab{a_n} $ for $ n \in \Z_{\ge 1} $ and $ y \in \ab{a_m} $ for $ m \in \Z_{\ge 1} $. If $ m \ge n $ then $ x \in \ab{a_m} $, so $ x + y \in \ab{a_m} $ gives $ x + y \in I $. Otherwise $ y \in \ab{a_n} $, so $ x + y \in \ab{a_n} $ gives $ x + y \in I $. Hence $ I \subseteq R $ is an ideal, so $ I $ is principal, that is there exists $ a \in R $ such that $ I = \ab{a} $. There exists $ n \in \Z_{\ge 1} $ such that $ a \in \ab{a_n} $. Have inclusions
$$ \ab{a} \subseteq \ab{a_n} \subseteq \ab{a_m} \subseteq \ab{a}. $$
All inclusions are equalities, so $ \ab{a_m} = \ab{a_n} $ for all $ m \ge n $.
\item Exercise: irreducibles are prime.
\end{itemize}
\end{proof}

\begin{remark*}
\hfill
\begin{itemize}
\item $ \Z\sb{\sqrt{-5}} $ is not a PID. Follows from Theorem \ref{thm:pid} and failure of unique factorisation. $ \ab{2, 1 + \sqrt{-5}} $ is not a principal ideal. (Exercise: check this)
\item A UFD that is not a PID. $ \Q\sb{x, y} $ is a UFD but $ \ab{x, y} $ is not principal. $ \Z\sb{x} $ is a UFD but $ \ab{2, x} $ is not principal.
\end{itemize}
\end{remark*}

\pagebreak

\subsection{Euclidean domains}

\begin{definition}
\hfill
\begin{itemize}
\item A \textbf{Euclidean norm} on an ID $ R $ is a function $ \phi : R \setminus \cb{0} \to \Z_{\ge 1} $ such that for all $ a, b \in R \setminus \cb{0} $ there exist $ q, r \in R $ such that $ a = qb + r $ and
\begin{itemize}
\item either $ r = 0 $,
\item or $ \phi\rb{r} < \phi\rb{b} $.
\end{itemize}
\item An ID that admits a Euclidean norm is called a \textbf{Euclidean domain}.
\end{itemize}
\end{definition}

Sometimes, add condition
\begin{equation}
\label{eq:1}
\phi\rb{ab} \ge \phi\rb{b}.
\end{equation}
If $ \phi $ is a Euclidean norm as in definition, can use $ \phi $ to construct $ \psi $ Euclidean norm satisfying $ \rb{\ref{eq:1}} $.

\begin{theorem}
If $ R $ is a Euclidean domain, then $ R $ is a PID, so $ R $ is a UFD.
\end{theorem}

\lecture{5}{Monday}{21/01/19}

\begin{proof}
Let $ I \subseteq R $ be an ideal. Assume $ I \ne \ab{0} $. Goal is that $ I $ is generated by one element $ a \in R \setminus \cb{0} $. Let $ 0 \ne a \in I $ be an element such that $ \phi\rb{a} $ is minimal along the values of $ \phi $ on $ I $. $ \ab{a} \subseteq I $. We will show that we have an equality. Let $ b \in I \setminus \ab{a} $. Apply property of $ \phi $ to $ b $ and $ a $, $ b = qa + r $. $ r \ne 0 $, otherwise $ a \mid b $ gives $ b \in \ab{a} $. $ r = b - qa \in I $ but $ \phi\rb{r} < \phi\rb{a} $, a contradiction.
\end{proof}

\begin{example*}
\hfill
\begin{itemize}
\item $ \Z $, with Euclidean norm
$$ \function{\Z \setminus \cb{0}}{\Z_{\ge 1}}{n}{\abs{n}}. $$
\item Gaussian integers. $ \Z\sb{i} $, with Euclidean norm given by restriction to $ \Z\sb{i} \subset \C $ of complex absolute value
$$ \function{\Z^2 \setminus \cb{\rb{0, 0}}}{\Z_{\ge 1}}{a + ib}{\rb{a + ib}\rb{a - ib} = a^2 + b^2}. $$
\item Eisenstein integers. Let $ 1 \ne \omega \in \C $ be a primitive cube root of unity, so $ \omega = \tfrac{-1 + \sqrt{-3}}{2} $. The subring
$$ \Z\sb{\omega} = \cb{a + b\omega \mid a, b \in \Z} \subset \C $$
is Euclidean, with Euclidean norm given by
$$ \function{\Z^2 \setminus \cb{\rb{0, 0}}}{\Z_{\ge 1}}{a + b\omega}{a^2 - ab + b^2}. $$
\end{itemize}
\end{example*}

\begin{remark*}
\hfill
\begin{itemize}
\item In all these examples, norm is multiplicative. This does not have to hold true, such as $ \Q\sb{x} $, with Euclidean norm $ f \mapsto \deg\rb{f} $.
\item There are PIDs that do not admit a Euclidean norm, such as $ \Z\sb{\tfrac{1 + \sqrt{-19}}{2}} $.
\end{itemize}
\end{remark*}

\subsection{Summary of rings}

$$ \cb{\text{commutative rings}} \supsetneq \cb{\text{IDs}} \supsetneq \cb{\text{UFDs}} \supsetneq \cb{\text{PIDs}} \supsetneq \cb{\text{Euclidean domains}}. $$
\begin{itemize}
\item $ \Q\sb{x, y} / xy $, $ \Z / 6\Z $, $ \F_3\sb{x} / x^2 $ are commutative rings but not IDs.
\item $ \Z\sb{\sqrt{-5}} $ is an ID but not a UFD, since $ 6 = 2 \cdot 3 = \rb{1 + \sqrt{-5}} \cdot \rb{1 - \sqrt{-5}} $.
\item $ \Z\sb{x} $ is a UFD but not a PID, since $ \ab{2, x} $ is not principal.
\item $ \Z\sb{\tfrac{1 + \sqrt{-19}}{2}} $ is a PID but not a Euclidean domain.
\end{itemize}

\pagebreak

\section{Gaussian and Eisenstein integers}

\subsection{Gaussian integers}

The \textbf{Gaussian integers} are
$$ \Z\sb{i} = \cb{a + bi \mid a, b \in \Z} \subset \Q\rb{i} \subset \C. $$
We will crucially use the norm
$$ \function[N]{\Z\sb{i}}{\Z_{\ge 0}}{a + bi}{\rb{a + bi}\rb{a - bi} = a^2 + b^2}, $$
which is not the same as the Euclidean norm.

\begin{note*}
This is multiplicative.
\end{note*}

\begin{proposition}
If $ u \in \Z\sb{i}^\times $ then $ N\rb{u} = 1 $.
\end{proposition}

\begin{proof}
$ N \mid_{\Z\sb{i} \setminus \cb{0}}\rb{u} \ge 1 $, $ N $ is multiplicative, and $ N\rb{1} = 1 $. $ uv = 1 $ gives $ N\rb{u} \cdot N\rb{v} = 1 $, so $ N\rb{u} \ge 1 $ and $ N\rb{v} \ge 1 $ gives $ N\rb{u} = N\rb{v} = 1 $.
\end{proof}

$ N\rb{u} = u \cdot \overline{u} = 1 $. $ u = a + bi \in \Z\sb{i}^\times $ if and only if $ a^2 + b^2 = 1 $, if and only if $ \rb{a, b} = \rb{\pm 1, 0} $, that is $ u = \pm 1 $, or $ \rb{a, b} = \rb{0, \pm 1} $, that is $ u = \pm i $.

\begin{remark*}
$ \Z\sb{i}^\times \cong \rb{\Z / 4\Z, +} $ as groups.
\end{remark*}

\begin{proposition}
Given $ \alpha, \beta \in \Z\sb{i} $, $ \beta \ne 0 $, there exist $ \kappa, \lambda \in \Z\sb{i} $ such that $ \alpha = \kappa\beta + \lambda $ and either $ \lambda = 0 $ or $ N\rb{\lambda} < N\rb{\beta} $, so $ N $ is Euclidean and $ \Z\sb{i} $ has unique factorisation.
\end{proposition}

\begin{proof}
$ \Z\sb{i} \subset \C $ is a lattice. $ \alpha = \kappa\beta + \lambda $ if and only if $ \alpha / \beta = \kappa + \lambda / \beta $ in $ \C $ for $ \beta \ne 0 $. $ \alpha / \beta \in \C $ lands inside one of the unit squares in the lattice spanned by $ \Z\sb{i} $. Open unit discs centred at the vertices of the unit square cover the entire square. $ \alpha / \beta $ is in the unit disc centred at $ \kappa $ if and only if $ N\rb{\alpha / \beta - \kappa} < 1 $. Let $ \lambda / \beta = \alpha / \beta - \kappa $ and $ N\rb{\lambda / \beta} < 1 $ if and only if $ N\rb{\lambda} < N\rb{\beta} $. Choose $ \kappa $ to be one vertex such that $ \alpha / \beta $ is in the open unit disc centred at $ \kappa $. $ \lambda = \beta\rb{\alpha / \beta - \kappa} $ gives $ N\rb{\lambda} < N\rb{\beta} $.
\end{proof}

\begin{lemma}[Special case of quadratic reciprocity]
If $ p $ is an odd prime, then $ -1 $ is a square mod $ p $ if and only if $ p \equiv 1 \mod 4 $.
\end{lemma}

The following is the decomposition of primes in $ \Z\sb{i} $.
\begin{itemize}
\item $ 2 = \rb{1 + i}\rb{1 - i} = \rb{-i}\rb{1 + i}^2 $. Notice that $ i\rb{1 + i} = i - 1 = -\rb{1 - i} $. Up to units in $ \Z\sb{i}^\times $ these prime factors are the same. This is a \textbf{ramified} prime.
\item $ p \equiv 1 \mod 4 $. $ p = \rb{a + bi}\rb{a - bi} $, which are distinct primes in $ \Z\sb{i} $. This is a \textbf{split} prime.
\item $ p \equiv 3 \mod 4 $. $ p $ stays prime. If not, $ a + bi \mid p $, so $ N\rb{a + bi} \mid N\rb{p} = p^2 $ gives $ N\rb{a + bi} = p = a^2 + b^2 $, which cannot happen. This is an \textbf{inert} prime.
\end{itemize}

\lecture{6}{Tuesday}{22/01/19}

Quadratic reciprocity gives that there exists $ n \in \Z $ such that $ p \mid n^2 + 1 = \rb{n + i}\rb{n - i} $. Assume $ p $ stays prime, or irreducible, in $ \Z\sb{i} $, so $ p \mid n + i $ or $ p \mid n - i $. By conjugating, we see $ p \mid n + i $ if and only if $ p \mid n - i $, so $ p \mid \rb{n + i} - \rb{n - i} = 2i $. Taking $ N $, see $ N\rb{p} = p^2 \nmid 4 = N\rb{2i} $, a contradiction.

\begin{theorem}
$ n \in \Z_{> 0} $ is of the form $ n = a^2 + b^2 $ for $ a, b \in \Z $ if and only if for all $ p \mid n $ such that $ p \equiv 3 \mod 4 $ the exponent of $ p $ in $ n $ is even.
\end{theorem}

\begin{theorem}
The only solutions to the Diophantine equation $ x^2 + 1 = y^3 $ are $ x = 0 $ and $ y = 1 $.
\end{theorem}

\begin{proof}
$ \rb{x + i}\rb{x - i} = y^3 $. Are $ x + i, x - i $ coprime in $ \Z\sb{i} $? If $ \pp $ is a prime of $ \Z\sb{i} $ dividing both, then $ \pp \mid 2i $, that is $ N\rb{\pp} \mid 4 $, so $ 2 \mid y $ gives $ 8 \mid y^3 $. But $ x^3 + 1 \equiv 1, 2, 5 \mod 8 $ gives $ \gcd\rb{x + i, x - i} = 1 $, so
$$
\begin{cases}
x + i = uz^3 = \rb{uz}^3 \\
x - i = \overline{u}\overline{z}^3 = \rb{\overline{u}\overline{z}}^3
\end{cases},
$$
for $ u \in \cb{\pm 1, \pm i} $. $ x - i = \rb{a + bi}^3 = a^3 - b^3i + 3a^2bi - 3ab^2 $ for some $ a, b \in \Z $. Looking at coefficients of $ i $, $ 1 = 3a^2b - b^3 $, so $ a = 0 $ and $ b = -1 $. Plugging this back in we get $ x = 0 $ and $ y = 1 $.
\end{proof}

\pagebreak

\subsection{Eisenstein integers}

The \textbf{Eisenstein integers} are $ \Z\sb{\omega} $ for $ \omega = \tfrac{-1 + \sqrt{-3}}{2} $. This is a subring of $ \C $, since
$$ \rb{a + b\omega}\rb{c + d\omega} = ac + \rb{ad + bc}\omega + bd\omega^2 = \rb{ac - bd} + \rb{ad + bc - bd}\omega. $$
What is $ \Z\sb{\sqrt{-3}} \subset \Z\sb{\omega} $? Both are subrings of $ \Q\rb{\sqrt{-3}} = \Q\sb{x} / \ab{x^3 + 3} $.
\begin{itemize}
\item In $ \Z\sb{\sqrt{-3}} $, $ 4 = 2 \cdot 2 = \rb{1 + \sqrt{-3}} \cdot \rb{1 - \sqrt{-3}} $, where $ 2, 1 + \sqrt{-3}, 1 - \sqrt{-3} $ are all irreducible.
\item $ \pi = \tfrac{1 + \sqrt{-3}}{2} $ is a unit in $ \Z\sb{\omega} $ and $ \pi^6 = 1 $, but $ \pi \notin \Z\sb{\sqrt{-3}} $.
\item $ \Z\sb{\sqrt{-3}} $ is not integrally closed in $ \Q\rb{\sqrt{-3}} $, but $ \Z\sb{\omega} $ is its integral closure and it is integrally closed in $ \Q\rb{\sqrt{-3}} $.
\item $ \omega^2 + \omega + 1 = 0 $, so $ \omega $ is an algebraic integer in $ \Z\sb{\omega} \setminus \Z\sb{\sqrt{-3}} $.
\end{itemize}

\begin{proposition}
If $ u \in \Z\sb{\omega}^\times $ then $ N\rb{u} = 1 $, where
$$ \function[N]{\Z\sb{\omega}}{\Z}{a + b\omega}{\rb{a + b\omega}\rb{a + b\overline{\omega}} = a^2 - ab + b^2}. $$
\end{proposition}

\begin{proof}
Multiplicative because it is the restriction of $ z \in \C \mapsto \abs{z}^2 $ to $ \Z\sb{\omega} $. Holds true in any imaginary quadratic field $ \Q\rb{\sqrt{-d}} $.
\end{proof}

$ a^2 - ab + b^2 = 1 $ if and only if $ \rb{a, b} = \rb{\pm 1, 0} $, that is $ u = \pm 1 $, or $ \rb{a, b} = \rb{0, \pm 1} $, that is $ u = \pm \omega $, or $ \rb{a, b} = \pm\rb{1, 1} $, that is $ u = \pm\rb{1 + \omega} = \pm\pi $.

\begin{remark*}
$ \Z\sb{\omega}^\times \cong \rb{\Z / 6\Z, +} $.
\end{remark*}

\begin{theorem}
$ \Z\sb{\omega} $ is a Euclidean domain, with Euclidean norm given by $ N\rb{a + b\omega} = a^2 - ab + b^2 $.
\end{theorem}

\begin{proof}
Let $ \alpha, \beta \in \Z\sb{\omega} \setminus \cb{0} $. There exists $ \kappa, \lambda \in \Z\sb{\omega} $ such that $ \alpha = \kappa\beta + \lambda $ and $ N\rb{\lambda} < N\rb{\beta} $. Use geometric proof. $ \Z\sb{\omega} = \cb{a + b\omega \mid a, b \in \Z} \subset \C $ is tiled by parallelograms of $ \C $, which are translates of a parallelogram at $ \pi $. Want to take $ \kappa $ to be a vertex of a parallelogram such that $ N\rb{\kappa - \alpha / \beta} < 1 $. Parallelogram covered by interior of unit discs centred at lattice points, so ok. Let $ \lambda = \beta\rb{\alpha / \beta - \kappa} $, so $ N\rb{\lambda} / N\rb{\beta} < 1 $.
\end{proof}

\lecture{7}{Friday}{25/01/19}

Lecture 7 is a problem class.

\lecture{8}{Monday}{28/01/19}

\begin{lemma}[Special case of quadratic reciprocity]
If $ p \ne 3 $ is an odd prime, then $ -3 $ is a square mod $ p $ if and only if $ p \equiv 1 \mod 3 $.
\end{lemma}

The following is the decomposition of primes in $ \Z\sb{\omega} $.
\begin{itemize}
\item $ 3 $ ramifies. $ 3 = -\rb{\sqrt{-3}}^2 $, which is irreducible in $ \Z\sb{\omega} $.
\item $ p \equiv 2 \mod 3 $ stays inert in $ \Z\sb{\omega} $. Because $ N $ is multiplicative and $ p $ cannot be written as $ a^2 - ab + b^2 $ with $ a, b \in \Z $.
\item $ p \equiv 1 \mod 3 $ splits as a product of distinct prime factors $ \pp, \overline{\pp} \in \Z\sb{\omega} $. $ p $ divides $ a^2 - ab + b^2 $ with $ a, b \in \Z $ and $ p \nmid a, b $, so $ p $ divides $ \rb{2a - b}^2 + 3b^2 $. Take $ z \in \Z $ odd such that $ z^2 \equiv -3 \mod p $, and let $ b = 1 \in \Z $ and $ a = \rb{z + 1} / 2 \in \Z $. To show that $ p $ splits in $ \Z\sb{\omega} $, let $ p \mid a^2 - a + 1 = \rb{a + \omega}\rb{a + \overline{\omega}} $ for $ z \in \Z $. Using unique factorisation, $ p \mid a + \omega $ or $ p \mid a + \overline{\omega} $. In fact, since $ a + \omega, a + \overline{\omega} $ are complex conjugates, $ p \mid a + \omega $ and $ p \mid a + \overline{\omega} $, so $ p \mid \omega - \overline{\omega} = \tfrac{-1 + \sqrt{-3}}{2} - \tfrac{-1 - \sqrt{-3}}{2} = \sqrt{-3} $. But $ \rb{3, p} = 1 $, a contradiction. Thus $ p = \pp\overline{\pp} = N\rb{\pp} $. Check that $ \pp / \overline{\pp} \ne u \in \Z\sb{\omega}^\times $, (Exercise) so $ p $ splits.
\end{itemize}

\begin{remark*}
These three possible behaviours have to do with the structure of $ \Z\sb{\omega} / \ab{p} $.
\begin{itemize}
\item If this is a field, $ p $ is inert.
\item If this is of the form $ \F_1 \times \F_2 $, $ p $ is split.
\item If this is of the form $ \F\sb{\epsilon} / \ab{\epsilon^2} $, $ p $ is ramified.
\end{itemize}
\end{remark*}

\pagebreak

\subsection{Other Euclidean domains}

\begin{itemize}
\item $ \Z\sb{\omega} $ and $ \Z\sb{i} $ are norm Euclidean. Using geometric proof, $ \Z\sb{i}, \Z\sb{\omega} \subset \C $ are lattices.
\item $ \Z\sb{\sqrt{-5}} $ is not a UFD, so not Euclidean, since
$$ 6 = 2 \cdot 3 = \rb{1 + \sqrt{-5}}\rb{1 - \sqrt{-5}}. $$
What goes wrong if we try to adapt geometric proof from $ \Z\sb{i}, \Z\sb{\omega} $? Unit discs do not cover all of the area of $ \C $.
\item The ring of integers $ \OO_7, \OO_{11} \subset \Q\rb{\sqrt{-7}} $ both are norm Euclidean. Adopt proof from $ \Z\sb{i}, \Z\sb{\omega} $.
\item It is hard to tell which fields are Euclidean and which are not. For example, $ \Z\sb{\tfrac{1 + \sqrt{-19}}{2}} $ is not Euclidean but is a PID and a UFD.
\item Among real quadratic fields, $ \Z\sb{\sqrt{2}} $ is Euclidean. The same geometric proof will not work because $ \Z\sb{\sqrt{2}} \subset \R $. (Exercise: $ \Z\sb{\sqrt{2}} $ is dense in $ \R $) We do have a geometric way to think about this.
$$ \dfrac{\Q\rb{\sqrt{2}}}{\Q} = \cb{a + b\sqrt{2} \mid a, b \in \Q} $$
is a two-dimensional $ \Q $-vector space.
$$ \function[\sigma]{\Q\rb{\sqrt{2}}}{\Q\rb{-\sqrt{2}}}{a + b\sqrt{2}}{a - b\sqrt{2}} $$
is a field automorphism that preserves $ \Q $.
$$
\begin{array}{rcl}
\Z\sb{\sqrt{2}} \subset \Q\rb{\sqrt{2}} & \hookrightarrow & \R^2 \\
a + b\sqrt{2} & \mapsto & \rb{a + b\sqrt{2}, a - b\sqrt{2}} \\
1 & \mapsto & \theta_1 = \rb{1, 1} \\
\sqrt{2} & \mapsto & \theta_2 = \rb{\sqrt{2}, -\sqrt{2}}
\end{array}.
$$
$ \theta_1, \theta_2 $ generate a lattice in $ \R^2 $. Can do a geometric proof in this, but use $ N\rb{x, y} = x \cdot y $ and areas under hyperbolas.
\end{itemize}

\pagebreak

\section{Structure theorem for finitely generated abelian groups}

\lecture{9}{Tuesday}{29/01/19}

Useful for describing the ring of integers $ \OO_K \subset K $ for a finite extension $ K / \Q $ and $ \OO^\times $, the group of units in $ \OO_K $, by Dirichlet's unit theorem.

\subsection{Modules}

\begin{definition}
Let $ R $ be a ring. An \textbf{$ R $-module} $ M $ is a set, together with
\begin{itemize}
\item an additive structure on $ M $
$$ m_1, m_2 \in M \qquad \implies \qquad m_1 + m_2 \in M, $$
\item an action of $ R $ on $ M $,
$$ \function{R \times M}{M}{\rb{r, m}}{rm} $$
satisfying
\begin{itemize}
\item $ r\rb{m_1 + m_2} = rm_1 + rm_2 $,
\item $ 1 \cdot m = m $,
\item $ r_1\rb{r_2m} = \rb{r_1r_2}m $,
\item $ \rb{r_1 + r_2}m = r_1m + r_2m $, and
\item $ 0 \cdot m = 0 $.
\end{itemize}
\end{itemize}
\end{definition}

\begin{note*}
\hfill
\begin{itemize}
\item If $ R $ is a field, then an $ R $-module is just an $ R $-vector space.
\item If $ R = \Z $, a $ \Z $-module $ M $ is an abelian group.
\end{itemize}
\end{note*}

\begin{definition}
A \textbf{free $ \Z $-module of rank $ n $} is a $ \Z $-module $ M $ which has a basis $ \rb{e_1, \dots, e_n} $ such that all $ m \in M $ can be written uniquely as $ a_1e_1 + \dots + a_ne_n $ for $ a_1, \dots, a_n \in \Z $.
\end{definition}

\begin{example*}
\hfill
\begin{itemize}
\item $ \Z $ is a free $ \Z $-module of rank one.
\item $ \Z^n = \cb{\rb{a_1, \dots, a_n} \mid a_i \in \Z} $ is a free $ \Z $-module of rank $ n $. Any free $ \Z $-module of rank $ n $ is isomorphic to $ \Z^n $, so there exists $ \phi : M \xrightarrow{\sim} \Z^n $, where $ M $ is free of rank $ n $, such that $ \phi\rb{m_1 + m_2} = \phi\rb{m_1} + \phi\rb{m_2} $, and $ \phi\rb{nm} = n\phi\rb{m} $ that is redundant once you respect addition.
\item $ \Z / 3\Z $ is not free because $ 3 \cdot 1 = 0 $. $ 0 $ is not written uniquely in terms of basis.
\item Any finite abelian group is a $ \Z $-module, but not free.
\item $ \Q = \cb{r / s \mid \rb{r, s} = 1, \ r, s \in \Z, \ s > 0} $ is a $ \Z $-module but it is not free of finite rank. Assume that $ \Q $ was free of rank $ n $, for some $ n \in \Z_{\ge 0} $. Let $ e_1 = r_1 / s_1, \dots, e_n = r_n / s_n $ be a basis. Then
$$ \dfrac{1}{s_1 \cdot \dots \cdot s_n + 1} \notin e_1\Z \oplus \dots \oplus e_n\Z, $$
a contradiction. Alternatively, prove that $ e_1 $ and $ e_2 $ are linearly dependent over $ \Z $, so rank would have to be one, and argue as above.
\item $ \Z\sb{i}, \Z\sb{\omega}, \Z\sb{\sqrt{-5}} $ are free $ \Z $-modules of rank two. We will later see that the ring of integers $ \OO_K \subset K $ is a free $ \Z $-module of rank equal to the rank of $ K / \Q $, or $ \dim_\Q\rb{K} $.
\end{itemize}
\end{example*}

\pagebreak

\subsection{Weak structure theorem}

\begin{theorem}[Structure theorem, weak form]
Let $ M $ be a free $ \Z $-module of finite rank $ n $, and $ M' \subseteq M $ be a $ \Z $-submodule. Then $ M' $ is free of rank $ m \le n $.
\end{theorem}

\begin{proof}
We will prove this by induction on $ n = rk\rb{M} $. We have a basis $ e_1, \dots, e_n $ of $ M $ and projections
$$ \function[p_i]{M}{\Z}{a_1e_1 + \dots + a_ne_n}{a_i}, $$
which are module homomorphisms. If $ p_i\rb{M'} = 0 $ for every $ i = 1, \dots, n $, then $ M' = 0 $. If $ M' \ne 0 $, we can assume without loss of generality that $ p_1\rb{M'} \ne 0 $.
\begin{itemize}
\item $ p_1\rb{M'} $ will be an ideal of $ \Z $, therefore it will be a principal ideal. There exists $ x \in M' $ such that $ p_1\rb{M'} = \ab{p_1\rb{x}} $.
\item $ N = Ker\rb{p_1} \hookrightarrow M $ is a submodule of $ M $ free of rank $ n - 1 $ because it is generated by $ e_2, \dots, e_n $.
\end{itemize}
Consider $ N' = N \cap M' $, a submodule of $ N, M', M $. We have an isomorphism of $ \Z $-modules
$$ N' \oplus x\Z = \cb{n' + n \cdot x \mid n' \in N', \ n \in \Z} \cong M'. $$
(Exercise: prove) $ N $ is free of rank $ n - 1 $, so by induction hypothesis $ N' $ is free of rank $ m' \le n - 1 $. $ M' $ is free of rank $ m' + 1 \le n $. Have a basis $ \rb{e_1', \dots, e_m', x} $ for $ M' $, where $ \rb{e_1, \dots, e_m'} $ is a basis for $ N' $.
\end{proof}

\subsection{Strong structure theorem}

\begin{theorem}[Structure theorem, strong form]
\label{thm:structure}
Let $ M $ be a free $ \Z $-module of rank $ n $. Let $ M' \subseteq M $ be a submodule. Then there exist
\begin{itemize}
\item a basis $ \rb{e_1, \dots, e_n} $ of $ M $, and
\item $ a_1, \dots, a_q \in \Z \setminus \cb{0} $ for $ q \le n $ such that $ M' $ has a basis $ \rb{a_1e_1, \dots, a_qe_q} $ and such that $ a_1 \mid \dots \mid a_q $.
\end{itemize}
\end{theorem}

\begin{corollary}
\label{cor:structure}
Let $ G $ be a finitely generated abelian group. Then there exist $ a_1, \dots, a_n \in \Z $ such that $ a_1 \mid \dots \mid a_n $ and
$$ G \cong \dfrac{\Z}{a_1\Z} \times \dots \times \dfrac{\Z}{a_n\Z}. $$
\end{corollary}

\lecture{10}{Friday}{01/02/19}

\begin{remark*}
In Corollary \ref{cor:structure}, we are allowing $ a_i = 0 $ for some $ i \in \cb{1, \dots, n} $.
\end{remark*}

\begin{proof}
Consider $ e_1, \dots, e_n $, the generators of $ G $, and let $ M $ be the free $ \Z $-module spanned by $ e_1, \dots, e_n $. $ \phi : M \twoheadrightarrow G $ is a surjective $ \Z $-module homomorphism. Have isomorphism of $ \Z $-modules $ M / M' \subseteq G $, induced by $ \phi $. Theorem \ref{thm:structure} gives
$$ M = e_1\Z \oplus \dots \oplus e_n\Z, \qquad M' = a_1e_1\Z \oplus \dots \oplus a_qe_q\Z, $$
where $ a_{q + 1} = \dots = a_n = 0 $. Thus
$$ \dfrac{M}{M'} \cong \dfrac{\Z}{a_1\Z} \times \dots \times \dfrac{\Z}{a_q\Z} \times \dfrac{\Z}{a_{q + 1}\Z} \times \dots \times \dfrac{\Z}{a_n\Z} \cong \dfrac{\Z}{a_1\Z} \times \dots \times \dfrac{\Z}{a_q\Z} \times \Z \times \dots \times \Z. $$
\end{proof}

\begin{lemma}
\label{lem:1}
Let $ M $ be a free $ \Z $-module of rank $ n $, and $ x \in M $. Let $ p_1 : M \to \Z $ and $ p_2 : M \to \Z $. There exists a homomorphism $ M \to \Z $ such that $ p\rb{x} \mid p_1\rb{x} $ and $ p\rb{x} \mid p_2\rb{x} $.
\end{lemma}

\begin{proof}
Find $ a, b \in \Z $ such that $ \gcd\rb{p_1\rb{x}, p_2\rb{x}} = ap_1\rb{x} + bp_2\rb{x} $, by Euclid's algorithm. Define $ p = ap_1 + bp_2 $.
\end{proof}

\begin{lemma}
\label{lem:2}
Let $ R $ be a PID. Let $ S $ be a set of ideals of $ R $. There exists an ideal $ I \in S' $ such that $ I \subseteq J $ and $ J \in S' $ gives $ I = J $, that is such that $ I $ is maximal with respect to inclusion.
\end{lemma}

\begin{proof}
We will use the ascending chain condition, ok because $ R $ is a PID, to argue by contradiction. If Lemma \ref{lem:2} were not true, would set
$$ I_1 \subsetneq I_2 \subsetneq \dots, $$
with $ I_i \in S $, a contradiction.
\end{proof}

\begin{lemma}
\label{lem:3}
Let $ M $ be a free $ \Z $-module of rank $ n $, and $ x \in M $. Then there exists a homomorphism $ p : M \to \Z $ such that $ p\rb{x} \mid q\rb{x} $ for every $ q : M \to \Z $.
\end{lemma}

\begin{proof}
Look at the set of all ideals $ \ab{q\rb{x}} \subseteq \Z $. Applying Lemma \ref{lem:2}, there exists $ \ab{p\rb{x}} \subseteq \Z $, where $ p : M \to \Z $, which is maximal with respect to inclusion. Want $ p\rb{x} \mid q\rb{x} $ for all $ q : M \to \Z $. Applying Lemma \ref{lem:1} to $ p, q $ gives $ r : M \to \Z $ such that $ r\rb{x} \mid q\rb{x} $ and $ r\rb{x} \mid p\rb{x} $, so $ \ab{r\rb{x}} \supseteq \ab{p\rb{x}} $. Because $ p $ is maximal, have equality $ r\rb{x} \sim p\rb{x} $, so $ p\rb{x} \mid q\rb{x} $.
\end{proof}

\begin{proof}[Proof of Theorem \ref{thm:structure}]
Argue by induction on $ n = rk_\Z\rb{M} $.
\begin{itemize}
\item Let $ M' \subseteq M $. If $ p : M \to \Z $, then $ p\rb{M'} \subseteq M $. Choose $ p $ such that $ p\rb{M'} $ is maximal with respect to inclusion among all $ q : M \to \Z $.
\item What is $ p\rb{M} \subseteq \Z $? We have $ p\rb{M} = a\Z $ for $ a \in \Z \setminus \cb{0} $. If $ a \ne \pm 1 $, could define $ p'\rb{x} = p\rb{x} / a $ for all $ x \in M $. $ p'\rb{M'} \supsetneq p\rb{M'} $ contradicts the maximality of $ p $ with respect to $ M' $. Thus $ p\rb{M} = \Z $.
\item Let $ N = Ker\rb{p} \subseteq M $, where $ p : M \twoheadrightarrow \Z $. $ N $ is free of rank $ n - 1 $.
\item Let $ N' = M' \cap N $ be a submodule, where
$$
\begin{tikzcd}
N' \arrow[hookrightarrow]{r} \arrow[hookrightarrow]{d} & N \arrow[hookrightarrow]{d} \\
M' \arrow[hookrightarrow]{r} \arrow[twoheadrightarrow]{d} & M \arrow[twoheadrightarrow]{d} \\
p\rb{M'} \arrow[hookrightarrow]{r} & \Z
\end{tikzcd}.
$$
\item Apply induction hypothesis to $ \rb{N, N'} $.
\item $ N $ has a basis $ \rb{e_2, \dots, e_n} $. Can complete $ \rb{e_2, \dots, e_n} $ to a basis for $ M $. Choose $ e_1' \in M $ such that $ p\rb{e_1'} = 1 $. Have a basis $ \rb{e_1', e_2, \dots, e_n} $ of $ M $.
\item There exist $ a_2, \dots, a_q \in \Z \setminus \cb{0} $ such that $ N' $ has a basis $ \rb{a_2e_2, \dots, a_qe_q} $ and $ a_2 \mid \dots \mid a_q $. $ M' \twoheadrightarrow p\rb{M'} = a_1\Z $ for $ a_1 \in \Z \setminus \cb{0} $, assuming $ M' \ne 0 $. Choose $ x \in M' $ such that $ p\rb{x} = a_1 $. We may assume $ p\rb{x} \mid q\rb{x} $ for every $ q : M \to \Z $. Look at $ \rb{e_1', e_2, \dots, e_n} $, a basis of $ M $. Let $ p_i : M \to \Z $ be the projection onto the $ i $-th coordinate. $ a_1 \mid p_i\rb{x} $ for all $ i = 1, \dots, n $, by maximality property of $ p $ and $ p\rb{x} = a_1 $. Can find a basis $ \rb{e_1, \dots, e_n} $ of $ M $ such that $ x = a_1e_1 $, where
$$ e_1 = e_1' + \dfrac{p_2\rb{x}}{a_1}e_2 + \dots + \dfrac{p_n\rb{x}}{a_1}e_n, $$
and $ p_2\rb{x} / a_1, \dots, p_n\rb{x} / a_1 \in \Z $.
\item Left to prove that $ a_1 \mid a_2 $. Let $ d = \rb{a_1, a_2} = b_1a_1 + b_2a_2 $. There exists $ d : M \to \Z $ such that $ d\rb{x} = b_1p_1\rb{x} + b_2p_2\rb{x} $, where $ p_1\rb{x} = p\rb{x} = a_1 $ and $ p_2\rb{x} = a_2 $. This will contradict maximality of $ p_1 = p $.
\end{itemize}
\end{proof}

\pagebreak

\subsection{Torsion}

\lecture{11}{Monday}{04/02/19}

\begin{definition}
\hfill
\begin{itemize}
\item If $ M $ is a $ \Z $-module, an element $ x \in M \setminus \cb{0} $ is called a \textbf{torsion element} if there exists $ a \in \Z \setminus \cb{0} $ such that $ ax = 0 $.
\item We say that a $ \Z $-module $ M $ is \textbf{torsion-free} if it does not contain torsion elements, that is if $ ax = 0 $ for $ a \in \Z $ and $ x \in M $, then $ a = 0 $ or $ x = 0 $.
\end{itemize}
\end{definition}

\begin{example*}
\hfill
\begin{itemize}
\item If $ G $ is any finite group, all elements of $ G $ are torsion.
\item If $ M $ is a free $ \Z $-module, then $ M $ is torsion-free, such as $ \Z^n $ for $ n \in \Z_{\ge 1} $.
\item $ \Q $ is torsion-free, even though it is not free of finite rank.
\end{itemize}
\end{example*}

\begin{proposition}
If $ M $ is a finitely generated $ \Z $-module and $ M $ is torsion-free, then $ M $ is free of finite rank.
\end{proposition}

\begin{proof}
Use structure theorem.
$$ M \cong \dfrac{\Z}{a_1\Z} \times \dots \times \dfrac{\Z}{a_n\Z}, $$
with $ a_1, \dots, a_n \in \Z $ satisfying $ a_1 \mid \dots \mid a_n $. Want $ a_1 = \dots = a_n = 0 $. If not, there exists $ a_i \ne 0 $ such that all $ x \in \Z / a_i\Z \setminus \cb{0} $ are torsion elements, so $ M $ cannot be a torsion-free, a contradiction.
\end{proof}

\pagebreak

\section{Integral closure}

\subsection{Integral elements}

\begin{definition}
An element $ x \in \C $ is called
\begin{itemize}
\item an \textbf{algebraic number} if it satisfies an equation
$$ x^n + a_{n - 1}x^{n - 1} + \dots + a_0 = 0, $$
with all $ a_i \in \Q $, and
\item an \textbf{algebraic integer} if $ a_i \in \Z $.
\end{itemize}
\end{definition}

\begin{example*}
\hfill
\begin{itemize}
\item $ x = i $ is an algebraic integer, since $ x^2 + 1 = 0 $.
\item $ x = \sqrt{2} $ is an algebraic integer, since $ x^2 - 2 = 0 $.
\item $ x = \sqrt{2} + i $ is an algebraic integer, since
$$ x - \sqrt{2} = i \qquad \implies \qquad x^2 - 2\sqrt{2}x + 3 = 0 \qquad \implies \qquad x^4 - 2x^2 + 9 = 0. $$
\end{itemize}
\end{example*}

In general, sum of product of algebraic integers are algebraic integers.

\begin{definition}
Let $ R $ be a ring and $ A \subseteq R $. An element $ x \in R $ is said to be \textbf{integral} over $ A $ if there exists a monic polynomial equation
$$ x^n + a_{n - 1}x^{n - 1} + \dots + a_0 = 0, $$
with $ a_i \in A $ for $ i = 0, \dots, n - 1 $.
\end{definition}

\begin{theorem}
\label{thm:integralsubring}
Let $ R $ be an ID and $ A \subseteq R $ a subring. Then if $ a, b \in R $ are integral over $ A $, so are $ a + b, a - b, ab $.
\end{theorem}

\begin{lemma}
\label{lem:cayleyhamilton}
Let $ R $ be an ID. Let
$$ M = \rb{a_{ij}}_{1 \le i, j \le n} \in M_n\rb{R} $$
be an $ n \times n $ matrix with coefficients in $ R $. Assume $ v = \rb{v_1, \dots, v_n} \in R^n $ for $ x \in R $ such that $ Mv = x \cdot v $, that is $ v $ is an eigenvector of $ M $ with eigenvalue $ x $. Let $ P \in R\sb{X} $ be the characteristic polynomial of $ M $. Then $ P\rb{x} = 0 $, that is $ x $ is a root of $ P $.
\end{lemma}

\begin{proof}
$$ P\rb{X} = \det\rb{X \cdot I_n - M} = X^n + a_{n - 1}X^{n - 1} + \dots + a_0 $$
is monic of degree $ n $ with coefficients in $ R $. Cayley-Hamilton theorem gives
$$ M^n + a_{n - 1}M^{n - 1} + \dots + a_0I_n = 0_n \in R^n \qquad \implies \qquad M^nv + a_{n - 1}M^{n - 1}v + \dots + a_0I_nv = 0_n \in R^n. $$
Since $ Mv = x \cdot v $, we get
$$ x^n \cdot v + a_{n - 1}x^{n - 1} \cdot v + \dots + a_0 \cdot v = 0_n \in R^n \qquad \implies \qquad \rb{x^n + a_{n - 1}x^{n - 1} + \dots + a_0} \cdot v = 0_n \in R^n. $$
$ v \ne 0 $ gives
$$ x^n + a_{n - 1}x^{n - 1} + \dots + a_0 = 0 \in R, $$
so $ x $ is a root of $ P $.
\end{proof}

\begin{proof}[Proof of Theorem \ref{thm:integralsubring}]
Let $ x = a + b $. Proof is similar for $ a - b $ and $ ab $. $ a $ is integral over $ A $, so there exists a polynomial $ f \in A\sb{x} $ for $ n = \deg\rb{f} $ such that
\begin{equation}
\label{eq:2}
f\rb{a} = a^n + a_{n - 1}a^{n - 1} + \dots + a_0 = 0,
\end{equation}
so $ a^n = -a_{n - 1}a^{n - 1} - \dots - a_0 $ is in the $ A $-linear span of $ a^{n - 1}, \dots, 1 $. Similarly for $ b $, there exists $ g \in A\sb{x} $ for $ m = \deg\rb{g} $ such that
\begin{equation}
\label{eq:3}
g\rb{b} = b^m + b_{m - 1}b^{m - 1} + \dots + b_0 = 0,
\end{equation}
so $ b^m $ is in the $ A $-linear span of $ b^{m - 1}, \dots, 1 $.
$$ \rb{a + b} \cdot a^ib^j = a^{i + 1}b^j + a^ib^{j + 1}, $$
for $ i = 0, \dots, n - 1 $ and $ j = 0, \dots, m - 1 $. If $ i + 1 = n $ use equation $ \rb{\ref{eq:2}} $. If $ j + 1 = m $ use equation $ \rb{\ref{eq:3}} $. Then $ \rb{a + b} \cdot a^ib^j $ is an $ A $-linear combination of $ a^kb^l $ for $ k \in \cb{0, \dots, n - 1} $ and $ l \in \cb{0, \dots, m - 1} $. Consider
$$ v = \threebyone{1}{\vdots}{a^{n - 1}b^{m - 1}} \in R^{m \cdot n}. $$
$ \rb{a + b} \cdot v = M \cdot v $ for some $ n \cdot m \times n \cdot m $ matrix $ M \in M_{n \cdot m}\rb{A} $. Lemma \ref{lem:cayleyhamilton} gives that $ a + b $ is a root of $ \det\rb{I_{n \cdot m}X - M} \in A\sb{X} $, that is $ a + b $ is integral over $ A $.
\end{proof}

\begin{corollary}
If $ R $ is an integral domain and $ A \subseteq R $. Then the set $ A' = \cb{x \in R \mid x \ \text{integral over} \ A} $ is a subring of $ R $, containing $ A $. $ A' $ is the \textbf{integral closure} of $ A $ in $ R $.
\end{corollary}

\begin{definition}
\hfill
\begin{itemize}
\item Let $ R $ be an ID with field of fractions $ K $. The \textbf{integral closure} of $ R $ is the integral closure of $ R $ in $ K $.
\item We say $ R $ is \textbf{integrally closed} if $ R $ is the integral closure of $ R $.
\end{itemize}
\end{definition}

\begin{example*}
\hfill
\begin{itemize}
\item $ \Z $ and $ \Z\sb{\omega} $ are integrally closed.
\item $ \Z\sb{x} $ is integrally closed.
\item $ R = \Z\sb{\sqrt{-3}} $ is not integrally closed. If $ \omega = \tfrac{-1 + \sqrt{-3}}{2} $ then $ \omega \in K = \Q\rb{\sqrt{-3}} $ and $ \omega^2 + \omega + 1 = 0 $, so $ \omega $ is integral over $ \Z\sb{\sqrt{-3}} $ but not in $ \Z\sb{\sqrt{-3}} $. Integral closure of $ \Z\sb{\sqrt{-3}} $ is $ \Z\sb{\omega} $, the Eisenstein integers.
\item $ \Q\sb{x, y} / \ab{x^2 - y^3} $ is not integrally closed. $ t = x / y \in Frac\rb{\Q\sb{x, y} / \ab{x^2 - y^3}} $ satisfies monic polynomial equations $ t^2 - y = 0 $ and $ t^3 - x = 0 $.
\end{itemize}
\end{example*}

\lecture{12}{Tuesday}{05/02/19}

\begin{proposition}
Let $ R $ be a UFD. Then $ R $ is integrally closed.
\end{proposition}

\begin{proof}
Let $ K = Frac\rb{R} $. Let $ x \in K $ be integral over $ R $. Want $ x \in R $. $ x $ satisfies a monic polynomial equation
\begin{equation}
\label{eq:4}
x^n + a_{n - 1}x^{n - 1} + \dots + a_0 = 0,
\end{equation}
for $ a_{n - 1}, \dots, a_0 \in R $. Write $ x = a / b $ with $ a, b \in R \setminus \cb{0} $. Can we ensure that $ a, b $ have no irreducible factor in common? Yes. Among all possible representations $ x = a / b $, choose the one for which $ d\rb{b} \in \Z_{\ge 0} $, the number of irreducible factors of $ b $, is the smallest. If $ \pp \mid a $ and $ \pp \mid b $ then $ a' = a / \pp $ and $ b' = b / \pp $, so $ x = a / b = a' / b' $, where $ d\rb{b'} = d\rb{b} - 1 $, a contradiction. Multiply $ \rb{\ref{eq:4}} $ by $ b^n $,
$$ a^n + a_{n - 1}a^{n - 1}b + \dots + a_0b^n = 0 \qquad \implies \qquad b\rb{a_{n - 1}a^{n - 1}b + \dots + a_0} = -a^n, $$
so $ b \mid a^n $, but $ \gcd\rb{a, b} = 1 $. Thus $ b \in R^\times $ is a unit, so $ x = a / b \in R $.
\end{proof}

\begin{theorem}
Let $ R \subset S $ be an inclusion of IDs. Let $ R' $ be the integral closure of $ R $ in $ S $. Then $ R' $ is integrally closed in $ S $.
\end{theorem}

\begin{example*}
Let $ \Z \subset R $, where $ R / \Q $ is a finite extension. Let $ \OO_K $ be the integral closure of $ \Z $ in $ K $, the ring of integers of $ K $. Then $ \OO_K $ is integrally closed. Applies to $ \Z\sb{i}, \Z\sb{\omega}, \Z\sb{\sqrt{-5}} $.
\end{example*}

\pagebreak

\subsection{Ring of integers in number fields}

A \textbf{number field} $ K $ is a field containing $ \Q $ such that $ \dim_\Q\rb{K} $ is finite. Any finite field extension of $ \Q $ is a number field. The \textbf{degree} of the number field is by definition $ \dim_\Q\rb{K} $. A \textbf{quadratic field} is an extension of $ \Q $ of degree two. The \textbf{ring of integers} $ \OO_K \subset K $ is the integral closure of $ \Z $ in $ K $.

\begin{lemma}
Every quadratic field $ K / \Q $ is of the form $ \Q\rb{\sqrt{d}} = \cb{a + b\sqrt{d} \mid a, b \in \Q} $ for some square-free $ d \in \Z $.
\end{lemma}

\begin{proof}
Let $ x \in K \setminus \Q $. Then $ \ab{1, x} $ is a $ \Q $-basis of $ K $. $ x^2 + \alpha x + \beta \in K $ for $ \alpha, \beta \in \Q $.
$$ x = \dfrac{\alpha \pm \sqrt{\alpha^2 + 4\beta}}{2}. $$
$ d = \alpha^2 + 4\beta \in \Q $, so $ K = \Q\rb{\sqrt{d}} $. Multiplying $ d $ by $ n^2 $, for all $ n \in \Z $, $ \Q\rb{\sqrt{d}} = \Q\rb{\sqrt{dn^2}} $, so can assume $ d \in \Z $. Similarly, can assume $ d $ is square-free. Thus $ \ab{1, \sqrt{d}} $ is a basis over $ \Q $.
\end{proof}

\begin{remark*}
If $ d < 0 $, $ \Q\rb{\sqrt{d}} $ is called an \textbf{imaginary quadratic field}. If $ d > 0 $, $ \Q\rb{\sqrt{d}} $ is called a \textbf{real quadratic field}.
\end{remark*}

\begin{theorem}
Let $ K = \Q\rb{\sqrt{d}} $ with $ d \in \Z $ square-free. Note that $ d \not\equiv 0 \mod 4 $.
\begin{enumerate}
\item If $ d \equiv 2, 3 \mod 4 $ then
$$ \OO_K = \Z\sb{\sqrt{d}} = \cb{a + b\sqrt{d} \mid a, b \in \Z}. $$
\item If $ d \equiv 1 \mod 4 $ then
$$ \OO_K = \cb{\tfrac{u + v\sqrt{d}}{2} \mid u, v \in \Z, \ u \equiv v \mod 2} \supsetneq \Z\sb{\sqrt{d}}. $$
In this case $ \OO_K $ is the $ \Z $-linear span of $ 1 $ and $ \tfrac{1 + \sqrt{d}}{2} $.
\end{enumerate}
\end{theorem}

\begin{example*}
\hfill
\begin{enumerate}
\item $ \Z\sb{i}, \Z\sb{\sqrt{-2}}, \Z\sb{\sqrt{2}}, \Z\sb{\sqrt{-5}} $.
\item $ \Z\sb{\omega}, \Z\sb{\tfrac{1 + \sqrt{5}}{2}} $.
\end{enumerate}
\end{example*}

\begin{proof}
Let $ \OO_K $ be the integral closure of $ \Z $ in $ \Q\rb{\sqrt{d}} = K $. Let $ x = a + b\sqrt{d} $ for $ a, b \in \Q $. Assume $ x $ is an algebraic integer. Let
$$ x = a + b\sqrt{d} \mapsto \overline{x} = a - b\sqrt{d}. $$
$ x, \overline{x} $ satisfy the same polynomial equation with $ \Z $ coefficients, so $ \overline{x} = a - b\sqrt{d} $ is also an algebraic integer.
\begin{itemize}
\item $ x\overline{x} = \rb{a + b\sqrt{d}}\rb{a - b\sqrt{d}} = a^2 - b^2d \in \Q $ is an algebraic integer, so $ a^2 - b^2d \in \Z $.
\item $ x - \overline{x} = 2b\sqrt{d} $, so $ 4b^2d \in \Z $ gives $ 2b \in \Z $, because $ d $ is square-free.
\item $ x + \overline{x} = 2a $, so $ 2a \in \Z $.
\end{itemize}
Let $ a = u / 2 $ and $ b = v / 2 $.
\begin{enumerate}
\item If $ d \equiv 2, 3 \mod 4 $,
$$ a^2 - b^2d = \dfrac{u^2 - v^2d}{4} \in \Z \qquad \implies \qquad 4 \mid u^2, v^2 \qquad \implies \qquad 2 \mid u, v \qquad \implies \qquad a, b \in \Z. $$
\item If $ d \equiv 1 \mod 4 $,
$$ a^2 - db^2 = \dfrac{u^2 - v^2d}{4} \in \Z \qquad \implies \qquad 4 \mid u^2 - dv^2 \qquad \implies \qquad u \equiv v \mod 2. $$
\end{enumerate}
\end{proof}

\lecture{13}{Friday}{08/02/19}

Lecture 13 is a problem class.

\pagebreak

\section{$ \OO_K $ is a lattice}

\lecture{14}{Monday}{11/02/19}

\subsection{Trace and norm}

Let $ K / \Q $ be a quadratic field. The conjugate is
$$ \function{K}{K}{\alpha = a + b\sqrt{d}}{\overline{\alpha} = a - b\sqrt{d}}. $$
Then
$$ \function[Tr]{K}{\Q}{\alpha}{\alpha + \overline{\alpha}}, \qquad \function[Nm]{K}{\Q}{\alpha}{\alpha \cdot \overline{\alpha}}, $$
and $ Tr : \OO_K \to \Z $ and $ Nm : \OO_K \to \Z $. $ \OO_K $ is a free $ \Z $-module of rank $ 2 $. Goal is to discuss trace and norm for general number fields. Motivation is that $ \OO_K $ is a free $ \Z $-module of rank $ \deg\rb{K / \Q} $.

\begin{proposition}
\label{prop:embeddings}
Let $ F \subseteq \C $ be a subfield. Let $ K / F $ be a finite extension of degree $ n $. Then there exist exactly $ n $ embeddings $ \sigma : K \hookrightarrow \C $ such that $ \sigma \mid_F = id_F $.
\end{proposition}

\begin{proof}
Assume first that $ K = F\rb{x} $, where $ x $ is a root of a minimal polynomial $ P\rb{t} \in F\sb{t} $. $ P $ has degree $ n $, since $ x^n $ is an $ F $-linear combination of $ 1, \dots, x^{n - 1} $. $ P $ has $ n $ distinct roots in $ \C $. Let $ \alpha $ be a root of $ P\rb{t} $ in $ \C $. This determines
$$ \function[\sigma]{K}{\C}{x}{\sigma\rb{x} = \alpha}, $$
where $ \sigma \mid_F = id_F $. Conversely, if $ \sigma : K \hookrightarrow \C $ such that $ \sigma \mid_F = id_F $, $ \sigma\rb{P\rb{t}} = P\rb{t} $ and $ \sigma\rb{x} $ is some root of $ P\rb{t} $ in $ \C $. In general, use induction on $ \deg\rb{K / F} = n $.
\begin{itemize}
\item $ n = 1 $ is ok. $ K = F $, so only one embedding.
\item $ n > 1 $. Choose $ x \in K \setminus F $. $ K / F\rb{x} / F $, so apply induction hypothesis on $ \deg\rb{K / F\rb{x}} < \deg\rb{K / F} $.
$$ n = \deg\rb{K / F} = \deg\rb{K / F\rb{x}} \cdot \deg\rb{F\rb{x} / F} = k \cdot m. $$
Have $ m $ embeddings $ \tau : F\rb{x} \hookrightarrow \C $ such that $ \tau \mid_F = id_F $. By induction, have $ k $ embeddings $ \sigma : K \hookrightarrow \C $ such that $ \sigma \mid_{F\rb{x}} = \tau $. Overall, have $ n = k \cdot m $ embeddings $ K \hookrightarrow \C $ which are $ id_F $ on $ F $.
\end{itemize}
\end{proof}

\begin{notation*}
Let $ e\rb{K / F} $ denote the set of embeddings as in Proposition \ref{prop:embeddings}.
\end{notation*}

Let $ x \in K $. Think of
$$ \function{K}{K}{y}{x \cdot y} $$
as an $ F $-linear transformation on $ K $. Let $ char_{K / F}\rb{x} $ denote the characteristic polynomial of multiplication by $ x $ in $ K $. $ char_{K / F}\rb{x} \in F\sb{t} $ has degree $ n = \sb{K : F} $.

\begin{example*}
Let $ K / \Q $ be quadratic and $ x = \sqrt{d} $. Then
$$ a + b\sqrt{d} \mapsto x \cdot \rb{a + b\sqrt{d}} = a\sqrt{d} + bd. $$
If $ K \cong \Q^2 $, then
$$ x = \twobytwo{0}{d}{1}{0}, \qquad char_{K / \Q}\rb{x} = t^2 - d = \rb{t - \sqrt{d}}\rb{t + \sqrt{d}}. $$
\end{example*}

\begin{proposition}
Let $ K / F $ be a finite extension of degree $ n $. Then
$$ char_{K / F}\rb{x} = \prod_{\sigma \in e\rb{K / F}} \rb{t - \sigma\rb{x}} \in F\sb{t}, $$
for all $ x \in K $.
\end{proposition}

\begin{proof}
First assume $ K = F\rb{x} $. Then the right hand side is just the minimal polynomial $ P\rb{t} \in F\sb{t} $ of $ x $. For any root $ \alpha $ of $ P\rb{t} $, $ char_{K / F}\rb{x}\rb{\alpha} = 0 $, since
$$ \function{K}{\C}{x}{\alpha} $$
has an $ F $-basis given by $ 1, \dots, \alpha^{n - 1} $, and multiplication by $ \alpha $ shifts this. Every root of $ P\rb{t} $ is also a root of $ char_{K / F}\rb{x} $, and they are both monic polynomials of degree $ n $, so $ P\rb{t} = char_{K / F}\rb{x} $. In general, $ K / F\rb{x} / F $. Choose a basis $ e_1, \dots, e_m $ of $ K $ over $ F\rb{x} $. For any $ i = 1, \dots, m $ multiplication by $ x $ leaves $ e_iF\rb{x} \subset K $ stable and has characteristic polynomial equal to
$$ \prod_{\sigma \in e\rb{F\rb{x} / F}} \rb{t - \sigma\rb{x}}, $$
where $ e_iF\rb{x} \subset K $ is an $ F $-vector subspace of dimension $ \deg\rb{F\rb{x} / F} $. Thus
$$ char_{K / F}\rb{x} = \prod_{\sigma \in e\rb{F\rb{x} / F}} \rb{t - \sigma\rb{x}}^m = \prod_{\sigma \in e\rb{F\rb{x} / F}} \rb{\prod_{\tau \in e\rb{K / F\rb{x}}, \ \tau_{F\rb{x}} = \sigma} \rb{t - \tau\rb{x}}}. $$
\end{proof}

\lecture{15}{Tuesday}{12/02/19}

\begin{definition}
$ Tr : K \to F $ is the trace of multiplication by $ x $ and $ Nm : K \to F $ is the determinant of multiplication by $ x $. These are coefficients of $ char_{K / F}\rb{x} $.
\end{definition}

\begin{theorem}
\label{thm:integralclosure}
Let $ R \subseteq F $ be an integrally closed domain. Let $ S $ be the integral closure of $ R $ in $ K $. Then if $ x \in S $, $ char_{K / F}\rb{x} \in R\sb{t} $.
\end{theorem}

\begin{corollary}
Let $ K, F, S, R $ as in Theorem \ref{thm:integralclosure}. We have $ Tr : S \to R $ and $ Nm : S \to R $.
\end{corollary}

\begin{example*}
Let $ K / \Q $ be quadratic. Then $ Tr : \OO_K \to \Z $ and $ Nm : \OO_K \to \Z $.
\end{example*}

\begin{proof}[Proof of Theorem \ref{thm:integralclosure}]
Let $ x \in S $. Is
$$ char_{K / F}\rb{x} = \prod_{\sigma \in e\rb{K / F}} \rb{t - \sigma\rb{x}} \in R\sb{t}? $$
Let $ L $ be the \textbf{composite} of extensions $ \sigma\rb{K} \subseteq \C $, the smallest field extension of $ F $ containing all $ \sigma\rb{K} $. Let $ T $ be the integral closure of $ R $ in $ L $.
$$
\begin{tikzcd}
T \arrow[subset]{r} & L \\
S \arrow[hookrightarrow]{u} \arrow[subset]{r} & K \arrow[hookrightarrow]{u} \\
R \arrow[hookrightarrow]{u} \arrow[subset]{r} & F \arrow[hookrightarrow]{u}
\end{tikzcd}.
$$
For all $ \sigma \in e\rb{K / F} $, $ \sigma\rb{x} $ is a root of the minimal polynomial $ P\rb{t} \in F\sb{t} $ of $ x $ over $ F $, and $ x \in S $ gives $ P\rb{t} \in R\sb{t} $, so $ \sigma\rb{x} \in T $. The coefficients of $ char_{K / F}\rb{x} $ are symmetric polynomials in the $ \sigma\rb{x} $,
$$ \sum_{\sigma \in e\rb{K / F}} \sigma\rb{x}, \sum_{\sigma, \sigma' \in e\rb{K / F}} \sigma\rb{x}\sigma'\rb{x}, \dots \in T, $$
therefore they are elements of $ T $. Upshot is that $ char_{K / F}\rb{x} \in \rb{F \cap T}\sb{t} = R\sb{t} $, since $ F \cap T $ is the integral closure of $ R $ in $ F $, which is $ R $.
\end{proof}

\begin{corollary}
If $ K / \Q $ is a finite extension, so $ F = \Q $, and $ \OO_K \subset K $ is the ring of integers, so $ R = \Z $. Then $ Tr : \OO_K \to \Z $ and $ Nm : \OO_K \to \Z $.
\end{corollary}

\pagebreak

\subsection{Bilinear forms}

\begin{definition}
Let $ V $ be a finite dimensional $ \Q $-vector space. A function
$$ \function[\ab{,}]{V \times V}{\Q}{\rb{v, w}}{\ab{v, w}} $$
is
\begin{itemize}
\item \textbf{$ \Q $-bilinear} if it is $ \Q $-linear as a function of $ v $ and $ \Q $-linear as a function of $ w $,
\item \textbf{symmetric} if $ \ab{v, w} = \ab{w, v} $, and
\item \textbf{non-degenerate} if for all $ v \in V $ such that $ v \ne 0 $, there exists $ w \in V $ such that $ \ab{v, w} \ne 0 $.
\end{itemize}
\end{definition}

\begin{example*}
\hfill
\begin{itemize}
\item Let $ V = \Q $.
$$ \function{V \times V}{\Q}{\rb{v, w}}{0} $$
is symmetric and bilinear.
\item Let $ V = \Q^2 $.
$$ \function{V \times V}{\Q}{\rb{v, w}}{\ab{v, w} = v \cdot w = v\twobytwo{1}{0}{0}{1}w^t} $$
is the inner product, which is non-degenerate.
\item Let $ K / \Q $ be quadratic.
$$ \function[Tr_{K / \Q}\rb{,}]{K \times K}{\Q}{\rb{x, y}}{Tr_{K / \Q}\rb{x \cdot y} \in \Q} $$
is
\begin{itemize}
\item symmetric, because $ x \cdot y = y \cdot x $, that is multiplication in $ K $ is commutative,
\item non-degenerate, because for all $ x \in K^\times $, take $ y = x^{-1} $,
$$ Tr_{K / \Q}\rb{x, y} = Tr_{K / \Q}\rb{xx^{-1}} = Tr_{K / \Q}\rb{1} = 2 \ne 0, $$
\item bilinear, because $ Tr_{K / \Q} $ is $ \Q $-linear.
\end{itemize}
\item Let $ K = \Q\rb{i} $, $ x = a + bi $, and $ y = c + di $.
$$ Tr_{\Q\rb{i} / \Q}\rb{x, y} = Tr_{\Q\rb{i} / \Q}\rb{\rb{a + bi}\rb{c + di}} = Tr_{\Q\rb{i} / \Q}\rb{ac - bd + ibc + iad} = 2\rb{ac - bd}, $$
so $ x, y \in \Z\sb{i} $ gives $ Tr_{\Q\rb{i} / \Q}\rb{x, y} \in \Z $.
\end{itemize}
\end{example*}

\pagebreak

\subsection{Lattices}

\lecture{16}{Friday}{15/02/19}

\begin{definition}
Let $ V $ be a finite dimensional $ \Q $-vector space. A \textbf{free $ \Z $-lattice}, or \textbf{lattice}, in $ V $ is a $ \Z $-submodule $ M \subseteq V $ that is free of rank $ \dim_\Q\rb{V} $.
\end{definition}

\begin{example*}
\hfill
\begin{itemize}
\item $ \Q\rb{\sqrt{-3}} \supset \Z\sb{\sqrt{-3}}, \Z\sb{2\sqrt{-3}}, \Z\sb{\omega / 2} $ are lattices.
\item $ \Z, \sqrt{-3}\Z $ are not lattices.
\end{itemize}
\end{example*}

\begin{lemma}
Let $ M \subseteq V $ be a lattice. If $ e_1, \dots, e_n $ is a $ \Z $-basis for $ M $ then $ e_1, \dots, e_n $ is a $ \Q $-basis for $ V $.
\end{lemma}

\begin{proof}
Notice that $ \dim_\Q\rb{V} = n $, since $ rk_\Z\rb{M} = n $. If $ e_1, \dots, e_n $ are $ \Q $-linearly independent then $ e_1, \dots, e_n $ generate $ W \subseteq V $ with $ \dim_\Q\rb{W} = n = \dim_\Q\rb{V} $, so $ W = V $. Assume there exist $ a_1, \dots, a_n \in \Q $ such that
$$ a_1e_1 + \dots + a_ne_n = 0. $$
Multiply this equation by the product of the denominators of the $ a_i $, which is not zero,
$$ a_1'e_1 + \dots + a_n'e_n = 0, $$
where $ a_1', \dots, a_n' \in \Z $, so $ a_1' = \dots = a_n' = 0 $. Thus $ a_1 = \dots = a_n = 0 $.
\end{proof}

Let $ M \subseteq V / \Q $ be a lattice. Let $ \ab{,} $ be a non-degenerate symmetric bilinear form on $ V $. Define
$$ M^V = \cb{w \in V \mid \ab{v, w} \in \Z \ \text{for all} \ v \in M}. $$

\begin{proposition}
\label{prop:duallattice}
$ M^V \subseteq V $ is also a lattice.
\end{proposition}

\begin{example*}
Let $ K = \Q\rb{\sqrt{-3}} $, $ Tr_{K / \Q}\rb{,} $, and $ M = \Z\sb{\sqrt{-3}} $. Then
$$ Tr_{K / \Q}\rb{a + b\sqrt{-3}, c + d\sqrt{-3}} = Tr_{K / \Q}\rb{ac - 3bd + \sqrt{-3}\rb{ad + bc}} = 2\rb{ac - 3bd}. $$
\begin{itemize}
\item $ \ab{1, c + d\sqrt{-3}} = 2c \in \Z $.
\item $ \ab{\sqrt{-3}, c + d\sqrt{-3}} = -6d \in \Z $.
\end{itemize}
Thus
$$ M^V = \cb{c + d\sqrt{-3} \mid c \in \tfrac{1}{2}\Z, \ d \in \tfrac{1}{6}\Z} = \ab{\tfrac{1}{2}, \tfrac{\sqrt{-3}}{6}} \supseteq \Z\sb{\omega} \supseteq M. $$
$ 1^V = 1 / 2 $ and $ \rb{\sqrt{-3}}^V = \sqrt{-3} / 6 $.
\end{example*}

\begin{proof}
Want that $ M^V \subseteq V $ is a lattice. Let $ e_1, \dots, e_n $ be a $ \Z $-basis of $ M $, so a $ \Q $-basis of $ V $. Given $ \ab{,} : V \times V \to \Q $ define $ e_1^V, \dots, e_n^V $ to be the dual basis to $ e_1, \dots, e_n $,
$$ \ab{e_i, e_j^V} =
\begin{cases}
1 & i = j \\
0 & i \ne j
\end{cases}.
$$
Claim that $ e_1^V, \dots, e_n^V $ is a $ \Z $-basis for $ M^V $.
\begin{itemize}
\item $ e_1^V, \dots, e_n^V $ are $ \Z $-linearly independent because $ \Q $-linearly independent.
$$ w = a_1e_1^V + \dots + a_ne_n^V = 0, $$
where $ a_i \in \Z \subseteq \Q $, so $ a_i = \ab{e_i, w} = 0 $.
\item $ e_i^V \in M^V $, by using definition of $ M^V $. Want $ \ab{v, e_i^V} \in \Z $ for all $ v \in M $. $ v \in M $ gives
$$ v = b_1e_1 + \dots + b_ne_n, $$
where $ b_i \in \Z $, so $ \ab{v, e_i^V} = b_i \in \Z $.
\item For all $ w \in M^V \subseteq V $,
$$ w = c_1e_1^V + \dots + c_ne_n^V, $$
where $ c_i \in \Z $. Can do this with $ c_i \in \Q $ for $ i = 1, \dots, n $. Need to show they are in $ \Z $. Have $ \ab{e_i, w} = c_i $ and $ w \in M^V $, so $ c_i \in \Z $.
\end{itemize}
\end{proof}

\pagebreak

\subsection{Main result}

\begin{theorem}
Let $ K / \Q $ be a number field of degree $ n $, with ring of integers $ \OO_K $. Then $ \OO_K $ is a lattice in $ K $.
\end{theorem}

\begin{proof}
Idea is
\begin{enumerate}
\item find lattice $ M \subseteq \OO_K $, and
\item show $ M^V \supseteq \OO_K $, the dual with respect to $ Tr_{K / \Q}\rb{,} $.
\end{enumerate}
By structure theorem,
$$ M^V \supseteq \OO_K \supseteq M, $$
so
$$ rk\rb{M} \le rk\rb{\OO_K} \le rk\rb{M^V}. $$
\begin{enumerate}
\item We can find $ n $ $ \Q $-linearly independent algebraic numbers $ e_1, \dots, e_n \in K $, because $ \dim_\Q\rb{K} = n $, so any $ \Q $-basis of $ K $ will work. $ e_i $ is an algebraic number, but may not be an algebraic integer.
$$ e_i^n + \alpha_1e_i^{n - 1} + \dots + \alpha_{n - 1} = 0, $$
for $ \alpha_1, \dots, \alpha_{n - 1} \in \Q $. Multiply equation by $ n $-th power $ A^n $ of denominators of $ \alpha_i $. Let $ e_i' = Ae_i $, so
$$ \rb{Ae_i}^n + \rb{A\alpha_1}\rb{Ae_i}^{n - 1} + \dots + \rb{A^n\alpha_{n - 1}} = 0, $$
for $ A\alpha_1, \dots, A^n\alpha_{n - 1} \in \Z $. Can assume $ e_i \in \OO_K $, that is algebraic integers. Let $ M \subseteq \OO_K $ be the $ \Z $-span of $ e_1, \dots, e_n $.
\item $ M^V \subseteq K $ is a lattice by Proposition \ref{prop:duallattice}. Show that $ \alpha \in \OO_K \subseteq M^V $. For all $ \beta \in M $, $ Tr_{K / \Q}\rb{\beta, \alpha} = Tr_{K / \Q}\rb{\alpha \cdot \beta} \in \Z $, since we know $ \alpha \cdot \beta \in \OO_K $ and $ Tr_{K / \Q} \mid_{\OO_K} : \OO_K \to \Z \subset \Q $.
\end{enumerate}
\end{proof}

\pagebreak

\section{$ \OO_K $ is a Dedekind domain}

\lecture{17}{Monday}{18/02/19}

Goal is to discuss Dedekind domains. A number field $ K / \Q $ gives the ring of integers $ \OO_K $, which is not usually a UFD.
\begin{itemize}
\item We will show that unique factorisation of ideals holds in Dedekind domains and $ \OO_K $ is a Dedekind domain.
\item We will introduce the ideal class group, which measures how far $ \OO_K $ is from being a PID or UFD.
\end{itemize}

\subsection{Dedekind domains}

Recall that $ \mm \subsetneq R $ is a maximal ideal if for all $ \mm \subseteq \nn \subseteq R $ either $ \nn = \mm $ or $ \nn = R $.

\begin{definition}
A ring $ R $ is called a \textbf{Dedekind domain} if $ R $ is an integrally closed Noetherian domain and every non-zero proper prime ideal of $ R $ is a maximal ideal.
\end{definition}

\begin{proposition}
\label{prop:dedekinddomain}
If $ R $ is a PID, then $ R $ is a Dedekind domain.
\end{proposition}

\begin{lemma}
\label{lem:dedekinddomain}
An element $ a \in R \setminus \cb{0} $ is irreducible if and only if $ \ab{a} $ is a maximal ideal among principal ideals.
\end{lemma}

\begin{proof}
$ a = bc $ is irreducible if and only if
$$
\begin{cases}
b \in R^\times, \ a \mid c \\
c \in R^\times, \ a \mid b
\end{cases}
\qquad \iff \qquad
\begin{cases}
\ab{b} = R, \ \ab{c} = \ab{a} \\
\ab{c} = R, \ \ab{b} = \ab{a}
\end{cases}.
$$
\begin{itemize}
\item[$ \implies $] Assume $ \ab{a} \subseteq \ab{b} \subseteq R $. $ b \mid a $ gives $ b \in R^\times $, so $ \ab{b} = R $, or $ a \mid b $, so $ \ab{b} = \ab{a} $. Thus $ a $ is irreducible.
\item[$ \impliedby $] Assume $ a = bc $. $ b \mid a $ gives $ R \supseteq \ab{b} \supseteq \ab{a} $, so either $ \ab{b} = R $ if and only if $ b \in R^\times $, or $ \ab{b} = \ab{a} $ if and only if $ a \mid b $ and $ c \in R^\times $.
\end{itemize}
\end{proof}

\begin{proof}[Proof of Proposition \ref{prop:dedekinddomain}]
$ R $ is a PID gives $ R $ is an integrally closed Noetherian domain. Let $ a \in R \setminus \cb{0} $ be such that $ \ab{a} $ is prime, if and only if $ a $ is prime, so $ a $ is irreducible. Lemma \ref{lem:dedekinddomain} gives that $ \ab{a} $ is maximal.
\end{proof}

\begin{example*}
\hfill
\begin{itemize}
\item $ \Z, \Z\sb{i}, \Z\sb{\sqrt{2}}, \Z\sb{\omega} $ are PIDs.
\item $ \Z\sb{\sqrt{-3}} $ is not a Dedekind domain because it is not integrally closed.
\item $ \Z\sb{\sqrt{-5}}, \Q\sb{x} $ are Dedekind domains.
\item $ \Z\sb{x} $ is not a Dedekind domain, since $ 0 \subsetneq \ab{x} \subsetneq \ab{2, x} $ are prime but $ \ab{x} $ is not zero or maximal.
\end{itemize}
\end{example*}

\subsection{Ideal norms}

\begin{definition}
Let $ K $ be a number field, with ring of integers $ \OO_K $ and a prime ideal $ 0 \ne \nn \subseteq \OO_K $. Then $ \OO_K / \nn $ is finite and define $ Nm\rb{\nn} = \#\OO_K / \nn $.
\end{definition}

If $ 0 \ne a \in \nn \cap \Z $, then $ \nn \supseteq \ab{a} $ and
$$ \dfrac{\OO_K}{\ab{a}} = \rb{\dfrac{\Z}{a\Z}}^{\deg\rb{K / \Q}} $$
is finite, so $ \OO_K / \ab{a} \twoheadrightarrow \OO_K / \nn $ and
$$ \#\dfrac{\OO_K}{\nn} \le \#\dfrac{\OO_K}{\ab{a}} $$
is finite.

\begin{example*}
\hfill
\begin{itemize}
\item Let $ \ab{p} \subsetneq \Z $ for $ p $ prime. Then
$$ Nm\rb{\ab{p}} = \#\dfrac{\Z}{\ab{p}} = \#\dfrac{\Z}{p\Z} = p. $$
\item Let $ a \in \Z \hookrightarrow \OO_K $. Then
$$ Nm\rb{\ab{a}} = \#\dfrac{\OO_K}{\ab{a}} = a^{\deg\rb{K / \Q}} = Nm_{K / \Q}\rb{a}. $$
If $ x \in \OO_K $, then $ Nm\rb{\ab{x}} = Nm_{K / \Q}\rb{x} $.
\item Let $ \OO_K = \Z\sb{\sqrt{-5}} \supsetneq \ab{2, 1 + \sqrt{-5}} $. Then $ Nm\rb{\ab{2, 1 + \sqrt{-5}}} = 2 $.
\begin{itemize}
\item
$$ \dfrac{\OO_K}{\ab{2, 1 + \sqrt{-5}}} = \dfrac{\Z\sb{x}}{\ab{x^2 + 5, 2, 1 + x}} = \dfrac{\Z}{2\Z}, $$
\item Alternatively, use structure theorem of finitely generated abelian groups. Have $ \rb{e_1, e_2} $ a basis for $ \OO_K $ over $ \Z $ and $ a_1, a_2 \in \Z \setminus \cb{0} $ such that $ a_1 \mid a_2 $ and $ \rb{a_1e_1, a_2e_2} $ a basis for $ \ab{2, 1 + \sqrt{-5}} $.
\begin{itemize}
\item $ \OO_K $ is generated by $ 1 + \sqrt{-5}, 1 $, and
\item $ \ab{2, 1 + \sqrt{-5}} $ is generated by $ 1 + \sqrt{-5}, 2 $,
\end{itemize}
so $ a_1 = 1 $ and $ a_2 = 2 $. Thus $ Nm\rb{\ab{2, 1 + \sqrt{-5}}} = a_1 \cdot a_2 $.
\item Alternatively, in the standard basis of $ \OO_K $, where $ e_1 = 1 $ and $ e_2 = \sqrt{-5} $,
$$ \ab{2, 1 + \sqrt{-5}} = \twobytwo{2}{0}{1}{1} \cdot \OO_K. $$
Then
$$ Nm\rb{\ab{2, 1 + \sqrt{-5}}} = \abs{\det\twobytwo{2}{0}{1}{1}} $$
is the absolute value of the determinant.
\end{itemize}
\end{itemize}
\end{example*}

\lecture{18}{Tuesday}{19/02/19}

\begin{remark*}
$ Nm\rb{\nn} \in \nn $ because $ Nm\rb{\nn} $ in $ \OO_K / \nn $ is equal to zero, because the order of a finite group is always equal to zero in that finite group.
\end{remark*}

\subsection{Ring of integers}

Goal is to show that $ \OO_K $ is a Dedekind domain, that is it is an integrally closed Noetherian domain and non-zero prime ideals are maximal ideals. It is an integrally closed domain.

\begin{proposition}
The ring of integers $ \OO_K $ in a number field $ K $ is Noetherian.
\end{proposition}

\begin{proof}
Assume that
$$ \aa_1 \subseteq \dots \subseteq \aa_n \subseteq \dots $$
is an ascending sequence of ideals in $ \OO_K $.
$$ Nm\rb{\aa_1} \ge \dots \ge Nm\rb{\aa_n} \ge \dots, $$
since
$$ \OO_K / \aa_1 \twoheadrightarrow \dots \twoheadrightarrow \OO_K / \aa_n \twoheadrightarrow \dots. $$
This must stabilise, so
$$ Nm\rb{\aa_n} = Nm\rb{\aa_{n + 1}} = \dots. $$
$ \aa_n \subseteq \aa_{n + 1} $ gives $ \OO_K / \aa_n \twoheadrightarrow \OO_K / \aa_{n + 1} $. Equality of norms gives that this must be a bijection, so $ \aa_n = \aa_{n + 1} $. (Exercise: $ \aa \subseteq \bb \subseteq \OO_K $ such that $ Nm\rb{\aa} = Nm\rb{\bb} $ gives $ \aa = \bb $)
\end{proof}

\begin{lemma}
Let $ R $ be an integral domain which is also a finite set. Then $ R $ is a field.
\end{lemma}

\begin{proof}
Let $ x \in R \setminus \cb{0} $. Look at $ x, x^2, \dots $. Since $ R $ is finite, we must have $ x^n = x^m $ for some $ n, m \in \Z_{\ge 1} $. If $ n > m $ then $ x^{n - m} = 1 $, so $ x \in R^\times $. Thus $ R $ is a field. Alternatively,
$$ \function{R}{R}{y}{x \cdot y} $$
is injective because $ R $ is an integral domain and bijective because $ R $ is a finite set.
\end{proof}

\begin{lemma}
Let $ \pp $ be a prime ideal in $ \OO_K $. Then $ \pp $ must be a maximal ideal.
\end{lemma}

\begin{proof}
$ \OO_K / \pp $ is an integral domain and a finite set of cardinality $ Nm\rb{\pp} $.
\end{proof}

\begin{remark*}
Let $ \pp \ne 0 $ be a prime ideal of $ \OO_K $. Then $ Nm\rb{\pp} = p^r $ for some $ p \in \Z $ prime and $ r \in \Z_{\ge 1} $ because of the classification of finite fields.
\end{remark*}

\begin{remark*}
$$
\begin{tikzcd}[row sep=tiny, column sep=tiny]
& & \cb{\text{UFDs}} \arrow[subset]{dr} & & & \\
\dots \arrow[subset]{r} & \cb{\text{PIDs}} \arrow[subset]{ur} \arrow[subset]{dr} & \ne & \cb{\text{integrally closed domains}} \arrow[subset]{r} & \cb{\text{IDs}} \arrow[subset]{r} & \dots \\
& & \cb{\text{Dedekind domains}} \arrow[subset]{ur} & & &
\end{tikzcd}.
$$
\begin{itemize}
\item $ \Z\sb{\sqrt{-5}} $ is a Dedekind domain but not a UFD.
\item $ \Z\sb{x} $ is a UFD but not a Dedekind domain.
\end{itemize}
\end{remark*}

\subsection{Properties of ideal norms}

\begin{proposition}
Let $ I \subseteq J \subseteq \OO_K $ be a sequence of ideals. Then $ Nm\rb{J} \mid Nm\rb{I} $.
\end{proposition}

\begin{proof}
Let $ \phi : \OO_K / I \twoheadrightarrow \OO_K / J $ be a ring homomorphism. Then
$$ \dfrac{\dfrac{\OO_K}{I}}{Ker\rb{\phi}} \cong \dfrac{\OO_K}{J}, $$
so
$$ Nm\rb{I} = Nm\rb{J} \cdot \#Ker\rb{\phi}. $$
Thus $ Nm\rb{J} \mid Nm\rb{I} $.
\end{proof}

\begin{example*}
Show that the ideals $ I = \ab{11, 3 + \sqrt{31}} $ and $ J = \ab{6, 1 + \sqrt{31}} $ in $ \Z\sb{\sqrt{31}} $ are relatively prime, that is $ I + J = \Z\sb{\sqrt{31}} $, if and only if $ Nm\rb{I + J} = 1 $.
$$ Nm\rb{I} = \abs{\det\twobytwo{11}{0}{3}{1}} = 11, \qquad Nm\rb{I} = \abs{\det\twobytwo{6}{0}{1}{1}} = 6. $$
$ I + J \subset I $, so $ Nm\rb{I + J} \mid Nm\rb{I} = 11 $, and $ I + J \subset J $, so $ Nm\rb{I + J} \mid Nm\rb{J} = 6 $. Thus $ Nm\rb{I + J} = 1 $.
\end{example*}

Let $ a \in \OO_K $. Structure theorem of finite generated abelian groups gives that if $ e_1, \dots, e_n $ is a basis for $ \OO_K $, where $ n = \deg\rb{K / \Q} $, then $ a_1e_1, \dots, a_ne_n $ is a basis for $ \ab{a} $. Then
$$ \dfrac{\OO_K}{\ab{a}} = \dfrac{\Z}{a_1\Z} \times \dots \times \dfrac{\Z}{a_n\Z}, $$
so
$$ Nm\rb{\ab{a}} = a_1 \cdot \dots \cdot a_n = \det\rb{a} = Nm_{K / \Q}\rb{a}. $$

\lecture{19}{Friday}{22/02/19}

Lecture 19 is a problem class.

\pagebreak

\section{Unique factorisation in a Dedekind domain}

\lecture{20}{Monday}{25/02/19}

\subsection{Fractional ideals}

Goal is to show that unique factorisation of ideals into prime ideals always holds in a Dedekind domain.

\begin{definition}
Let $ R $ be an integral domain, with fraction field $ K $. A \textbf{fractional ideal} $ \nn $ of $ R $ is an $ R $-submodule of $ K $, that is
\begin{itemize}
\item an additive subgroup, so $ x \in \nn $ and $ y \in \nn $ gives $ x + y \in \nn $, and
\item stable under multiplication by $ R $, so $ x \in \nn $ and $ r \in R $ gives $ rx \in \nn $,
\end{itemize}
such that there exists $ a \in R $ such that $ a\nn = \cb{ax \mid x \in \nn} \subseteq R $.
\end{definition}

\begin{example*}
\hfill
\begin{itemize}
\item Any ideal of $ R $ is also a fractional ideal with $ a = 1 $. Conversely, if $ \nn $ is a fractional ideal and $ \nn \subseteq R $, then $ \nn $ is an integral ideal.
\item Let $ R = \Z \hookrightarrow \Q $. Then the fractional ideals of $ \Q $ are $ \tfrac{p}{q} \cdot \Z $ for $ p, q \in \Z $ and $ q \ne 0 $. $ \Q $ is not a fractional ideal, but it is a $ \Z $-module.
\end{itemize}
\end{example*}

We can multiply fractional ideals of $ R $ by
$$ \mm \cdot \nn = \cb{\sum_{i = 1}^k x_i \cdot y_i \ \Bigg| \ x_i \in \mm, \ y_i \in \nn, \ k \in \Z_{\ge 0}}. $$

\begin{lemma}
If $ \mm, \nn $ are fractional ideals of $ R $ then $ \mm \cdot \nn $ is also a fractional ideal of $ R $.
\end{lemma}

\begin{proof}
$ \mm \cdot \nn $ is additive. $ \mm \cdot \nn $ is stable under multiplication by $ r \in R $, since
$$ r\rb{\sum_{i = 1}^k x_i \cdot y_i} = \sum_{i = 1}^k rx_i \cdot y_i \in \mm \cdot \nn. $$
There exists $ a, b \in R $ such that $ a\mm \subseteq R $ and $ b\nn \subseteq R $, so $ ab \cdot \mm \cdot \nn \subseteq R $.
\end{proof}

Multiplication of fractional ideals is commutative, so $ \mm \cdot \nn = \nn \cdot \mm $, is associative, and has unit $ R \cdot \mm = \mm \cdot R = \mm $. If $ R $ is a Dedekind domain, we will show that every fractional ideal has a multiplicative inverse, that is given a fractional ideal $ \mm $ of $ R $, there exists a fractional ideal $ \mm^{-1} $ of $ R $ such that $ \mm \cdot \mm^{-1} = R $.

\begin{example*}
$ \Z\sb{x} $ is not a Dedekind domain, since $ \ab{2, x} $ does not have an inverse with respect to multiplication.
\end{example*}

\begin{theorem}
\label{thm:group}
Let $ R $ be a Dedekind domain. The set of non-zero fractional ideals of $ R $ forms a commutative group under multiplication.
\end{theorem}

To prove Theorem \ref{thm:group}, need some preliminary results.

\begin{lemma}
\label{lem:primeideal}
Let $ \pp \subsetneq R $ be a prime ideal in an integral domain. Assume $ \pp \supseteq \aa_1\aa_2 $ for ideals $ \aa_1, \aa_2 \subseteq R $. Then $ \pp \supseteq \aa_1 $ or $ \pp \supseteq \aa_2 $.
\end{lemma}

\begin{proof}
If $ \pp \nsupseteq \aa_1 $ there exists $ x \in \aa_1 \setminus \pp $, so $ x\aa_2 \subseteq \pp $. Let $ y \in \aa_2 $. Then $ xy \in \pp $ and $ x \notin \pp $ gives $ y \in \pp $, since $ \pp $ is prime, so $ \aa_2 \subseteq \pp $.
\end{proof}

\begin{remark*}
Same Lemma \ref{lem:primeideal} holds if $ \pp \supseteq \aa_1, \dots, \aa_n $ for $ n \in \Z_{\ge 1} $ then $ \pp \supseteq \aa_i $ for some $ i $.
\end{remark*}

\begin{lemma}
\label{lem:primeproduct}
If $ I \subseteq R $ is a non-zero ideal of a Noetherian domain, there exists $ \pp_1, \dots, \pp_n \subseteq R $ non-zero prime ideals not necessarily distinct, such that $ I \supseteq \pp_1 \dots \pp_n $ for some finite $ n \in \Z $.
\end{lemma}

\begin{proof}
Assume Lemma \ref{lem:primeproduct} is false. Choose $ I $ for which condition in Lemma \ref{lem:primeproduct} fails and such that $ I $ is maximal with respect to inclusion, that is for all $ J \supseteq I $, $ J $ satisfies the condition. $ I $ is not prime, otherwise it would satisfy the condition. There exists $ a, b \in R $ such that $ a \notin I $ and $ b \notin I $ but $ ab \in I $. Look at $ I \subsetneq I + bR $ and $ I \subsetneq I + aR $, both satisfying the condition. Let $ I + bR \supseteq \pp_1 \dots \pp_r $ and $ I + aR \supseteq \qq_1 \dots \qq_s $.
Thus
$$ \pp_1 \dots \pp_r \cdot \qq_1 \dots \qq_s \subseteq \rb{I + bR} \cdot \rb{I + aR} \subseteq I. $$
\end{proof}

\begin{proposition}
\label{prop:inverse}
Let $ R $ be a Dedekind domain. Let $ \mm \subsetneq R $ be a non-zero prime ideal, if and only if maximal ideal of $ R $. Define
$$ \mm^{-1} = \cb{a \in K \mid a\mm \subseteq R}. $$
Then $ \mm^{-1} $ is a fractional ideal of $ R $ and $ \mm \cdot \mm^{-1} = R $.
\end{proposition}

\lecture{21}{Tuesday}{26/02/19}

\begin{proof}
$ \mm^{-1} $ is a fractional ideal.
\begin{itemize}
\item $ \mm^{-1} $ is an additive subgroup of $ K $.
$$
\begin{cases}
x \in \mm^{-1} \quad \implies \quad x \cdot \mm \subseteq R \\
y \in \mm^{-1} \quad \implies \quad y \cdot \mm \subseteq R
\end{cases}
\quad \implies \quad \rb{x + y}\mm \subseteq x\mm + y\mm \subseteq R \quad \implies \quad x + y \in \mm^{-1}.
$$
\item $ \mm^{-1} $ is stable under multiplication by $ R $.
$$ x \in \mm^{-1} \qquad \implies \qquad x \cdot \mm \subseteq R \qquad \implies \qquad rx \cdot \mm \subseteq R \qquad \implies \qquad rx \in \mm^{-1}. $$
\item Let $ a \in \mm $ such that $ a \ne 0 $. Then $ a\mm^{-1} \subseteq R $ by definition of $ \mm^{-1} $.
\end{itemize}
$ \mm = \mm \cdot 1 \subseteq \mm \cdot \mm^{-1} \subseteq R $ is automatic by definition. Since $ \mm $ is maximal, either $ \mm = \mm \cdot \mm^{-1} $ or $ \mm \cdot \mm^{-1} = R $. Assume $ \mm = \mm \cdot \mm^{-1} $ and get a contradiction. Take $ x \in \mm^{-1} $.
$$ \dots \subseteq x^n \cdot \mm \subseteq \dots \subseteq x \cdot \mm \subseteq \mm \subsetneq R, $$
so $ x^n \in \mm^{-1} $ for all $ n \in \Z_{\ge 1} $. Let $ R\sb{x} $ be the subring of $ K $ generated by $ x $, so $ R\sb{x} \subseteq \mm^{-1} $. Let $ a \in R \setminus \cb{0} $ be such that $ a\mm^{-1} \subseteq R $. In particular $ aR\sb{x} \subseteq R $ is an integral ideal of $ R $. $ R $ is Noetherian, so $ aR\sb{x} $ is generated over $ R $ by finitely many elements $ y_1, \dots, y_k \in R $. $ x \cdot y_i \in aR\sb{x} $ gives
$$ x \cdot \threebyone{y_1}{\vdots}{y_k} = A\threebyone{y_1}{\vdots}{y_k}, \qquad A = \threebythree{a_{11}}{\dots}{a_{1k}}{\vdots}{\ddots}{\vdots}{a_{k1}}{\dots}{a_{kk}} \in M_k\rb{R}, $$
a $ k \times k $ matrix, so $ x $ is an eigenvalue of $ A $ with eigenvector $ \rb{y_1, \dots, y_k}^t $. Thus $ x $ is a root of $ \det\rb{tI_k - A} $, so $ x $ is integral over $ R $. All $ x \in \mm^{-1} $ satisfy $ x \in R $ because $ R $ is integrally closed, so $ \mm^{-1} = R $. Choose $ a \in \mm \setminus \cb{0} $.
$$ \mm \supsetneq \ab{a} \supseteq \pp_1 \cdot \dots \cdot \pp_k, $$
where $ \pp_i $ are prime, by Lemma \ref{lem:primeproduct}. Lemma \ref{lem:primeideal} from last time gives that $ \mm $ must contain $ \pp_i $ for some $ i = 1, \dots, k $. Assume $ \mm \supseteq \pp_1 \ne 0 $, so $ \mm = \pp_1 $. Choose $ k $ such that it is minimal among all possible $ \pp_1 \cdot \dots \cdot \pp_k $ contained in $ \ab{a} $, so $ \pp_2 \cdot \dots \cdot \pp_k \nsubseteq \ab{a} $, since $ \pp_2 \cdot \dots \cdot \pp_k $ has length $ k - 1 $. Choose $ 0 \ne b \in \pp_2 \cdot \dots \pp_k \setminus \ab{a} $. Then $ b / a \in K $ such that
\begin{enumerate}
\item $ b / a \in R $, and
\item $ b / a \in \mm^{-1} $
\end{enumerate}
gives a contradiction.
\begin{enumerate}
\item $ b \in aR = \ab{a} $ is false, so $ b / a \notin R $.
\item $ b / a \in \mm^{-1} $ if $ b / a \cdot \mm \subseteq R $, or equivalently
$$ b \cdot \mm \in \pp_1 \cdot \dots \cdot \pp_k \subseteq \ab{a}. $$
\end{enumerate}
$ \mm^{-1} \ne R $ gives $ \mm \cdot \mm^{-1} \ne R $, so $ \mm \cdot \mm^{-1} = R $, which is done.
\end{proof}

\pagebreak

\subsection{Unique factorisation of ideals}

\begin{theorem}[Unique factorisation of ideals]
\label{thm:uniquefactorisationofideals}
\hfill
\begin{enumerate}
\item Any non-zero fractional ideal $ I $ of $ R $ can be written as
$$ I = \pp_1^{k_1} \cdot \dots \cdot \pp_r^{k_r}, $$
where $ \pp_1 $ are non-zero prime ideals and $ k_i \in \Z \setminus \cb{0} $.
\item This factorisation is unique up to order of the $ \pp_i $.
\end{enumerate}
\end{theorem}

\begin{remark*}
Given $ I $,
$$ I^{-1} = \pp_1^{-k_1} \cdot \dots \cdot \pp_r^{-k_r}. $$
Get $ I \cdot I^{-1} = R $, so any non-zero fractional ideal has a multiplicative inverse.
\end{remark*}

\begin{proof}
\hfill
\begin{enumerate}
\item There exists $ a \in R \setminus \cb{0} $ such that $ aI \subseteq R $ is an integral ideal so we may assume $ I $ is an integral ideal. Assume Theorem \ref{thm:uniquefactorisationofideals} is false. Because $ R $ is Noetherian, there exists $ I $ such that $ I $ does not admit a factorisation and $ I $ is maximal with this property. Let $ \mm \supseteq I $ be a maximal ideal. Then $ J = \mm^{-1} \cdot I \subseteq R $, so $ J $ is an integral ideal. $ J $ admits a factorisation because
$$ \mm^{-1} \cdot I = J \supsetneq I = 1 \cdot I, $$
since if $ \mm^{-1} \cdot I = I $, argue as in proof of Proposition \ref{prop:inverse}. Thus $ I = \mm \cdot J $, a contradiction.
\item Let
$$ \pp_1^{k_1} \cdot \dots \cdot \pp_r^{k_r} = \qq_1^{k_1} \cdot \dots \cdot \qq_r^{k_r}. $$
We may assume all $ k_i, l_j > 0 $. $ \pp_1 \supseteq \qq_1^{l_1} \cdot \dots \cdot \qq_s^{l_s} $, so $ \pp_1 \supseteq \qq_1 $. Both maximal, so $ \pp_1 = \qq_1 $.
\end{enumerate}
\end{proof}

\lecture{22}{Friday}{01/03/19}

\subsection{Ideal class group}

\begin{definition}
Let $ R $ be a Dedekind domain. The \textbf{ideal class group} of $ R $ is the quotient of the group of all non-zero fractional ideals by the subgroup of principal fractional ideals. This is denoted by $ Cl\rb{R} $.
\end{definition}

If $ K $ is a number field, then $ \OO_K $ is a Dedekind domain and $ Cl\rb{K} = \OO_K $.

\begin{example*}
\hfill
\begin{itemize}
\item $ Cl\rb{K} = 1 $ if $ \OO_K $ is a PID or UFD. For example if $ K = \Q\rb{i}, \Q\rb{\omega}, \Q\rb{\sqrt{-11}} $.
\item $ Cl\rb{K} \ne 1 $ if $ K = \Q\rb{\sqrt{-5}}, \Q\rb{\sqrt{-6}} $. $ \ab{2, 1 + \sqrt{-5}} $ is not principal, and
$$ \ab{2, 1 + \sqrt{-5}}^2 = \ab{4, 2 + 2\sqrt{-5}, -4 + 2\sqrt{-5}} = \ab{2}, $$
so $ \Z / 2\Z $ is a subgroup of $ Cl\rb{K} $. Claim that $ \Z / 2\Z \cong Cl\rb{K} $. For example
$$ \ab{3, 1 - \sqrt{-5}} = \ab{a} \cdot \ab{2, 1 + \sqrt{-5}}, $$
where $ a = \rb{1 - \sqrt{-5}} / 2 \in \Q\rb{\sqrt{-5}}^\times $ because
\begin{itemize}
\item $ \rb{1 + \sqrt{-5}}a = \rb{1 + \sqrt{-5}}\rb{1 - \sqrt{-5}} / 2 = 3 $, and
\item $ 2a = 1 - \sqrt{-5} $.
\end{itemize}
\end{itemize}
\end{example*}

\begin{theorem}
If $ K $ is a number field, then $ Cl\rb{K} $ is finite.
\end{theorem}

Proof relies on an explicit bound. Every \textbf{ideal class}, a fractional ideal of $ \OO_K $ up to multiplication by principal ideals, contains a representative of norm less than an explicit bound depending on $ K $.

\pagebreak

\section{Arithmetic in quadratic fields}

\subsection{Discriminant}

Let $ K $ be a number field. Recall $ \OO_K $ is a free $ \Z $-module of rank $ n = \deg\rb{K / \Q} $. Choose $ e_1, \dots, e_n $, a $ \Z $-basis of $ \Q $.
$$ \function[Tr_{K / \Q}\rb{,}]{K \times K}{\Q}{\rb{v, w}}{Tr_{K / \Q}\rb{v \cdot w}} $$
is symmetric, $ \Q $-bilinear, and non-degenerate. The \textbf{discriminant} of $ K $ is
$$ Disc\rb{K} = \det\rb{Tr_{K / \Q}\rb{e_i, e_j}_{i, j = 1, \dots, n}}. $$

\begin{remark*}
$ Disc\rb{K} \ne 0 $ and $ Disc\rb{K} \in \Z $.
\end{remark*}

\begin{lemma}
$ Disc\rb{K} $ is independent of choice of $ \Z $-basis $ e_1, \dots, e_n $.
\end{lemma}

\begin{proof}
If $ f_1, \dots, f_n $ is another $ \Z $-basis, there exists $ A \in M_n\rb{\Z} $ such that
$$ A\threebyone{e_1}{\vdots}{e_n} = \threebyone{f_1}{\vdots}{f_n}. $$
$ \det\rb{A} = \pm 1 $ and
$$ A^t \cdot \rb{Tr_{K / \Q}\rb{e_i, e_j}_{i, j}} \cdot A = \rb{Tr_{K / \Q}\rb{f_i, f_j}_{i, j}}, $$
so
$$ \det\rb{Tr_{K / \Q}\rb{e_i, e_j}_{i, j}} = \det\rb{Tr_{K / \Q}\rb{e_i, e_j}_{i, j}} \cdot \rb{\det\rb{A}}^2 = \det\rb{Tr_{K / \Q}\rb{f_i, f_j}_{i, j}}. $$
\end{proof}

\begin{example*}
Let $ K / \Q $ be quadratic, so $ K = \Q\rb{\sqrt{d}} $. Then
$$ Disc\rb{K} =
\begin{cases}
4d & d \equiv 2, 3 \mod 4, \ \OO_K = \ab{1, \sqrt{d}} \\
d & d \equiv 1 \mod 4, \ \OO_K = \ab{1, \tfrac{1 + \sqrt{d}}{2}}
\end{cases},
$$
since
$$ \twobytwo{Tr_{K / \Q}\rb{1}}{Tr_{K / \Q}\rb{\sqrt{d}}}{Tr_{K / \Q}\rb{\sqrt{d}}}{Tr_{K / \Q}\rb{d}} = \twobytwo{2}{0}{0}{2d}, \qquad \twobytwo{Tr_{K / \Q}\rb{1}}{Tr_{K / \Q}\rb{\tfrac{1 + \sqrt{d}}{2}}}{Tr_{K / \Q}\rb{\tfrac{1 + \sqrt{d}}{2}}}{Tr_{K / \Q}\rb{\tfrac{1 + d + 2\sqrt{d}}{4}}} = \twobytwo{2}{1}{1}{\tfrac{d + 1}{2}}. $$
\end{example*}

\pagebreak

\subsection{Criterion for decomposition of primes}

Let $ p \in \Z $ be a rational prime. How does $ \ab{p} $ decompose, or factor, into prime ideals of $ \OO_K $?
\begin{itemize}
\item If $ \OO_K / \ab{p} \cong \F_p\sb{x} / x^2 $ we say $ \ab{p} $ ramifies in $ \OO_K $. Then $ \ab{p} = \pp^2 $, where $ \pp $ is a prime ideal of $ \OO_K $.
\item If $ \OO_K / \ab{p} \cong \F_{\pp_1} \times \F_{\pp_2} $ we say $ \ab{p} $ splits in $ \OO_K $. Then $ \ab{p} = \pp_1 \cdot \pp_2 $, where $ \pp_1 \ne \pp_2 $ are distinct prime ideals of $ \OO_K $.
\item If $ \OO_K / \ab{p} \cong \F_{p^2} $ we say $ \ab{p} $ is inert in $ \OO_K $. Then $ \ab{p} $ is a prime ideal of $ \OO_K $.
\end{itemize}
To see that these are the only possibilities when $ K / \Q $ is quadratic, key observation is $ \#\rb{\OO_K / \ab{p}} = p^2 $, so $ \OO_K / \ab{p} $ is an $ \F_p $-vector space of dimension two. The following is the criterion.

\lecture{23}{Monday}{04/03/19}

\begin{itemize}
\item $ \ab{p} $ is ramified if and only if $ p \mid Disc\rb{K} $. More precisely,
\begin{itemize}
\item $ p \mid 4d $ if $ d \equiv 2, 3 \mod 4 $, and
\item $ p \mid d $ if $ d \equiv 1 \mod 4 $.
\end{itemize}
\item If $ d \equiv 1 \mod 4 $, then $ 2 $ is
\begin{itemize}
\item inert if $ d \equiv 5 \mod 8 $, and
\item split if $ d \equiv 1 \mod 8 $.
\end{itemize}
\item If $ p \ne 2 $ and $ p \nmid Disc\rb{K} $, then $ p $ is
\begin{itemize}
\item split if $ d $ is a quadratic residue mod $ p $, and
\item inert if $ d $ is not a quadratic residue mod $ p $.
\end{itemize}
Choose standard basis $ 1, \delta $ of $ \OO_K $.
$$ \delta =
\begin{cases}
\sqrt{d} & d \equiv 2, 3 \mod 4 \\
\tfrac{1 + \sqrt{d}}{2} & d \equiv 1 \mod 4
\end{cases}.
$$
Minimal polynomial equation of $ \delta $ is
$$ f\rb{x} =
\begin{cases}
x^2 - d & d \equiv 2, 3 \mod 4 \\
x^2 - x - \tfrac{d - 1}{4} & d \equiv 1 \mod 4
\end{cases}.
$$
Compute
$$ \dfrac{\OO_K}{\ab{p}} = \dfrac{\Z\sb{x}}{\ab{f\rb{x}, p}} = \dfrac{\F_p\sb{x}}{\ab{f\rb{x}}}. $$
$ p $ is inert if and only if $ \F_p\sb{x} / \ab{f\rb{x}} \cong \F_{p^2} $, if and only if $ f\rb{x} = 0 $.
\begin{itemize}
\item If $ d \equiv 2, 3 \mod 4 $, then $ x^2 - d = 0 $ has no solutions if and only if $ d $ is not a quadratic residue mod $ p $.
\item If $ d \equiv 1 \mod 4 $, then $ x^2 - x - \tfrac{d - 1}{4} = 0 $ has no solutions if and only if $ d $ is not a quadratic residue mod $ p $.
\end{itemize}
\end{itemize}

\pagebreak

\section{Finiteness of ideal class group}

Goal is if $ K $ is an imaginary quadratic field, so $ d < 0 $, then $ Cl\rb{K} $ is finite. Use that every ideal class in $ Cl\rb{K} $ contains an integral ideal of norm less than explicit bound depending on $ K $. Uses geometry of numbers and Minkowski's theorem.

\subsection{Standard form of ideals}

Let $ K $ be a quadratic field and $ \OO_K = \Z \oplus \Z\delta $ be the ring of integers, where
$$ \delta =
\begin{cases}
\sqrt{d} & d \equiv 2, 3 \mod 4 \\
\tfrac{1 + \sqrt{d}}{2} & d \equiv 1 \mod 4
\end{cases}.
$$

\begin{proposition}
Let $ I \subseteq \OO_K $ be an integral ideal. There exists $ a, b, d \in \Z $ such that the following hold.
\begin{itemize}
\item $ I = d\rb{\Z a + \Z\rb{-b + \delta}} $.
\item $ a \mid Nm_{K / \Q}\rb{-b + \delta} $.
\end{itemize}
\end{proposition}

Conversely, every subset of $ \OO_K $ of this form is an integral ideal. The expression
$$ I = d\rb{\Z a + \Z\rb{-b + \delta}} $$
is called the \textbf{standard form} of $ I $.

\begin{proof}
Let $ K $ be imaginary quadratic. $ \OO_K $ is a lattice $ \Lambda_0 $ generated by $ 1, \delta $ in $ K $. $ I $ is a lattice $ \Lambda \subseteq \Lambda_0 $. There exists $ \gamma \in M_2\rb{\Z} $ such that $ \Lambda = \gamma \cdot \Lambda_0 $. Can assume
$$ \gamma = \twobytwo{a'}{b'}{0}{d}, $$
for $ a', b', d \in \Z $. $ I = \Z a' + \Z\rb{b' + d\delta} $ is an ideal and $ a' \in I $ gives $ a'\delta = a'b + \rb{b' + d\delta}a \in I $ for $ a, b \in \Z $, so
$$ a'\rb{\delta - b} = \rb{b' + d\delta}a. $$
The coefficients of $ \delta $ gives $ a' = da $, and the coefficients of $ 1 $ gives $ -a'b = ab' $, so $ b' = -db $. Left to check that $ a \mid Nm_{K / \Q}\rb{-b + \delta} $. For simplicity assume $ d = 1 $. $ I $ is an ideal and $ -b + \delta \in I $ gives $ \delta\rb{-b + \delta} = af + \rb{-b + \delta}e \in I $ for $ e, f \in \Z $, so
$$ \rb{e - \delta}\rb{-b + \delta} = -af \in \Z. $$
If $ e - \delta = -b + \overline{\delta} $ the product would be $ \rb{-b + \overline{\delta}}\rb{-b + \delta} = Nm_{K / \Q}\rb{-b + \delta} = -af \in \Z $, if and only if $ a \mid Nm_{K / \Q}\rb{-b + \delta} $. Conversely,
$$ -af = \rb{e - \delta}\rb{-b + \delta} = \rb{e - \delta + b - \overline{\delta} + \rb{-b + \overline{\delta}}}\rb{-b + \delta} = \rb{e - \delta + b - \overline{\delta}}\rb{-b + \delta} + Nm_{K / \Q}\rb{-b + \delta}. $$
Then $ -af \in \Z $, $ Nm_{K / \Q}\rb{-b + \delta} \in \Z $, $ e - \delta + b - \overline{\delta} \in \Z $, and $ -b + \delta \in \OO_K \setminus \Z $, so $ e - \delta + b - \overline{\delta} = 0 $.
\end{proof}

An observation is that if $ I = d\rb{\Z a + \Z\rb{-b + \delta}} $ is in standard form, $ Nm\rb{I} = d^2 \cdot a $, since
$$ \det\twobytwo{a}{-b}{0}{1} = 0. $$

\lecture{24}{Tuesday}{05/03/19}

\begin{example*}
Let
$$ I = \ab{2 + i} = \rb{2 + i}\Z + \rb{2 + i}i\Z = \rb{2 + i}\Z + \rb{2i - 1}\Z = \twobytwo{2}{-1}{1}{2} \cdot \Z\sb{i} \subsetneq \Z\sb{i}. $$
Column-reducing,
$$ \twobytwo{2}{-1}{1}{2} \sim \twobytwo{2}{-5}{1}{0} \sim \twobytwo{5}{2}{0}{1}. $$
Standard form is $ I = 5\Z + \rb{2 + i}\Z $. (Exercise: check that $ 5 = Nm_{\Z\sb{i} / \Z}\rb{2 + i} $)
\end{example*}

Let $ K $ be an imaginary quadratic field and
$$ J = q\rb{\Z a + \Z\rb{-b + \delta}} \subsetneq K, $$
for $ a, b \in \Z $ and $ q \in \Q^\times $, be a fractional ideal of $ \OO_K $. $ Nm\rb{J} = q^2 \cdot a $, by extending norm from ideals to fractional ideals multiplicatively. Standard form of $ J $ has fundamental parallelogram with vertices
$$ 0, \ qa, \ q\rb{-b + \delta}, \ qa + q\rb{-b + \delta}. $$
Let $ A\rb{J} $ be the area of the fundamental parallelogram of $ J $.

\begin{proposition}
$$ A\rb{J} = \dfrac{Nm\rb{J} \cdot \sqrt{Disc\rb{K}}}{2}. $$
\end{proposition}

\begin{proof}
$$ A\rb{J} = qa \cdot q\abs{Im\rb{\delta}} = q^2 \cdot a \cdot \abs{Im\rb{\delta}} = \dfrac{q^2 \cdot a \cdot \sqrt{Disc\rb{K}}}{2} = \dfrac{Nm\rb{J} \cdot \sqrt{Disc\rb{K}}}{2}. $$
\end{proof}

\subsection{Minkowski's theorem}

If $ J $ is a fractional ideal of $ K $, goal is to show that there exists $ \alpha \in J $ such that $ Nm\rb{\alpha} < C_K \cdot Nm\rb{J} $, where $ C_K $ is an explicit bound in terms of $ K $, so $ Nm\rb{\alpha \cdot J^{-1}} < C_K $. Up to multiplication by a principal ideal, can get $ Nm\rb{J^{-1}} < C_K $. Idea is to use Minkowski's theorem. Have a lattice $ \Lambda \subset \R^2 $ and a nice region $ S \subset \R^2 $. Let $ A\rb{\Lambda} $ be the area of the fundamental parallelogram of $ \Lambda $.

\begin{theorem}[Minkowski's theorem]
If $ A\rb{S} > 4A\rb{\Lambda} $ then $ S $ contains a non-zero lattice point, that is $ S \cap \Lambda \ne \emptyset $.
\end{theorem}

$ S $ is \textbf{nice}
\begin{itemize}
\item if $ x \in S $ then $ -x \in S $, and
\item $ S $ is convex, that is if $ x, y \in S $ then the segment $ \sb{x, y} \subset S $.
\end{itemize}

\begin{example*}
Let $ S $ be the closed or open disc of radius $ r > 2 / \sqrt{\pi} $ and $ \Lambda = \Z\sb{i} \subset \C $. Then $ A\rb{S} = \pi r^2 > 4 = 4A\rb{\Lambda} $.
\end{example*}

\begin{proof}
Consider all parallelograms of $ 2\Lambda $ that intersect $ S $. Translate elements of $ 2\Lambda $ until they all overlap.
$$ A\rb{2\Lambda} = 4A\rb{\Lambda} < A\rb{S} = A\rb{S_1} + \dots + A\rb{S_n}. $$
There exists $ S_i, S_j $ for $ i \ne j $ such that $ S_i \cap S_j \ne \emptyset $ translated. There exists $ x \in S \cap S_i $ and $ y \in S \cap S_j $ for $ i \ne j $ such that $ x - y \in 2\Lambda $. Claim that $ \rb{x - y} / 2 \in S \cap \Lambda $.
\begin{itemize}
\item $ x - y \in 2\Lambda $ gives $ \rb{x - y} / 2 \in \Lambda $.
\item $ \rb{x - y} / 2 \in S $ because $ y \in S $ gives $ -y \in S $, and $ x \in S $ gives the midpoint $ \rb{x - y} / 2 \in S $.
\end{itemize}
\end{proof}

\begin{theorem}
Let $ K $ be an imaginary quadratic field. Then every ideal class in $ Cl\rb{K} $ contains a representative $ I $ such that $ I $ is an integral ideal of norm less than
$$ \dfrac{2\sqrt{Disc\rb{K}}}{\pi}. $$
\end{theorem}

\begin{proof}
Back to $ J = q\rb{\Z a + \Z\rb{-b + \delta}} $ a fractional ideal of $ \OO_K $. Let $ S $ be the disc centred at the origin of radius
$$ r = \sqrt{\dfrac{2\sqrt{Disc\rb{K}} \cdot Nm\rb{J}}{\pi}} + \epsilon. $$
Then
$$ A\rb{S} > \pi r^2 = 2\sqrt{Disc\rb{K}} \cdot Nm\rb{J} = 4A\rb{J}. $$
Minkowski gives that there exists $ \alpha \in J $ such that $ \alpha \ne 0 $ and
$$ Nm_{K / \Q}\rb{\alpha} = \abs{\alpha}^2 < \dfrac{2\sqrt{Disc\rb{K}} \cdot Nm\rb{J}}{\pi} + \epsilon. $$
Thus there exists $ \alpha \in J $ such that
$$ Nm_{K / \Q}\rb{\alpha} < \dfrac{2\sqrt{Disc\rb{K}}}{\pi} \cdot Nm\rb{J}. $$
\end{proof}

(Exercise: prove $ Cl\rb{\Q\rb{\sqrt{-5}}} = \Z / 2\Z $)

\lecture{25}{Friday}{08/03/19}

Lecture 25 is a problem class.

\end{document}