\def\module{M4P61 Infinite Groups}
\def\lecturer{Dr Isabel M\"uller}
\def\term{Autumn 2019}
\def\cover{}
\def\syllabus{}
\def\thm{subsection}

\documentclass{article}

% Packages

\usepackage{amssymb}
\usepackage{amsthm}
\usepackage[UKenglish]{babel}
\usepackage{commath}
\usepackage{enumitem}
\usepackage{etoolbox}
\usepackage{fancyhdr}
\usepackage[margin=1in]{geometry}
\usepackage{graphicx}
\usepackage[hidelinks]{hyperref}
\usepackage[utf8]{inputenc}
\usepackage{listings}
\usepackage{mathtools}
\usepackage{stmaryrd}
\usepackage{tikz-cd}
\usepackage{csquotes}

% Formatting

\addto\captionsUKenglish{\renewcommand{\abstractname}{Syllabus}}
\delimitershortfall5pt
\ifx\thm\undefined\newtheorem{n}{}\else\newtheorem{n}{}[\thm]\fi
\newcommand\newoperator[1]{\ifcsdef{#1}{\cslet{#1}{\relax}}{}\csdef{#1}{\operatorname{#1}}}
\setlength{\parindent}{0cm}

% Environments

\theoremstyle{plain}
\newtheorem{algorithm}[n]{Algorithm}
\newtheorem*{algorithm*}{Algorithm}
\newtheorem{algorithm**}{Algorithm}
\newtheorem{conjecture}[n]{Conjecture}
\newtheorem*{conjecture*}{Conjecture}
\newtheorem{conjecture**}{Conjecture}
\newtheorem{corollary}[n]{Corollary}
\newtheorem*{corollary*}{Corollary}
\newtheorem{corollary**}{Corollary}
\newtheorem{lemma}[n]{Lemma}
\newtheorem*{lemma*}{Lemma}
\newtheorem{lemma**}{Lemma}
\newtheorem{proposition}[n]{Proposition}
\newtheorem*{proposition*}{Proposition}
\newtheorem{proposition**}{Proposition}
\newtheorem{theorem}[n]{Theorem}
\newtheorem*{theorem*}{Theorem}
\newtheorem{theorem**}{Theorem}

\theoremstyle{definition}
\newtheorem{aim}[n]{Aim}
\newtheorem*{aim*}{Aim}
\newtheorem{aim**}{Aim}
\newtheorem{axiom}[n]{Axiom}
\newtheorem*{axiom*}{Axiom}
\newtheorem{axiom**}{Axiom}
\newtheorem{condition}[n]{Condition}
\newtheorem*{condition*}{Condition}
\newtheorem{condition**}{Condition}
\newtheorem{definition}[n]{Definition}
\newtheorem*{definition*}{Definition}
\newtheorem{definition**}{Definition}
\newtheorem{example}[n]{Example}
\newtheorem*{example*}{Example}
\newtheorem{example**}{Example}
\newtheorem{exercise}[n]{Exercise}
\newtheorem*{exercise*}{Exercise}
\newtheorem{exercise**}{Exercise}
\newtheorem{fact}[n]{Fact}
\newtheorem*{fact*}{Fact}
\newtheorem{fact**}{Fact}
\newtheorem{goal}[n]{Goal}
\newtheorem*{goal*}{Goal}
\newtheorem{goal**}{Goal}
\newtheorem{law}[n]{Law}
\newtheorem*{law*}{Law}
\newtheorem{law**}{Law}
\newtheorem{plan}[n]{Plan}
\newtheorem*{plan*}{Plan}
\newtheorem{plan**}{Plan}
\newtheorem{problem}[n]{Problem}
\newtheorem*{problem*}{Problem}
\newtheorem{problem**}{Problem}
\newtheorem{question}[n]{Question}
\newtheorem*{question*}{Question}
\newtheorem{question**}{Question}
\newtheorem{warning}[n]{Warning}
\newtheorem*{warning*}{Warning}
\newtheorem{warning**}{Warning}
\newtheorem{acknowledgements}[n]{Acknowledgements}
\newtheorem*{acknowledgements*}{Acknowledgements}
\newtheorem{acknowledgements**}{Acknowledgements}
\newtheorem{annotations}[n]{Annotations}
\newtheorem*{annotations*}{Annotations}
\newtheorem{annotations**}{Annotations}
\newtheorem{assumption}[n]{Assumption}
\newtheorem*{assumption*}{Assumption}
\newtheorem{assumption**}{Assumption}
\newtheorem{conclusion}[n]{Conclusion}
\newtheorem*{conclusion*}{Conclusion}
\newtheorem{conclusion**}{Conclusion}
\newtheorem{claim}[n]{Claim}
\newtheorem*{claim*}{Claim}
\newtheorem{claim**}{Claim}
\newtheorem{notation}[n]{Notation}
\newtheorem*{notation*}{Notation}
\newtheorem{notation**}{Notation}
\newtheorem{note}[n]{Note}
\newtheorem*{note*}{Note}
\newtheorem{note**}{Note}
\newtheorem{remark}[n]{Remark}
\newtheorem*{remark*}{Remark}
\newtheorem{remark**}{Remark}

% Lectures

\newcommand{\lecture}[3]{ % Lecture
  \marginpar{
    Lecture #1 \\
    #2 \\
    #3
  }
}

% Blackboard

\renewcommand{\AA}{\mathbb{A}} % Blackboard A
\newcommand{\BB}{\mathbb{B}}   % Blackboard B
\newcommand{\CC}{\mathbb{C}}   % Blackboard C
\newcommand{\DD}{\mathbb{D}}   % Blackboard D
\newcommand{\EE}{\mathbb{E}}   % Blackboard E
\newcommand{\FF}{\mathbb{F}}   % Blackboard F
\newcommand{\GG}{\mathbb{G}}   % Blackboard G
\newcommand{\HH}{\mathbb{H}}   % Blackboard H
\newcommand{\II}{\mathbb{I}}   % Blackboard I
\newcommand{\JJ}{\mathbb{J}}   % Blackboard J
\newcommand{\KK}{\mathbb{K}}   % Blackboard K
\newcommand{\LL}{\mathbb{L}}   % Blackboard L
\newcommand{\MM}{\mathbb{M}}   % Blackboard M
\newcommand{\NN}{\mathbb{N}}   % Blackboard N
\newcommand{\OO}{\mathbb{O}}   % Blackboard O
\newcommand{\PP}{\mathbb{P}}   % Blackboard P
\newcommand{\QQ}{\mathbb{Q}}   % Blackboard Q
\newcommand{\RR}{\mathbb{R}}   % Blackboard R
\renewcommand{\SS}{\mathbb{S}} % Blackboard S
\newcommand{\TT}{\mathbb{T}}   % Blackboard T
\newcommand{\UU}{\mathbb{U}}   % Blackboard U
\newcommand{\VV}{\mathbb{V}}   % Blackboard V
\newcommand{\WW}{\mathbb{W}}   % Blackboard W
\newcommand{\XX}{\mathbb{X}}   % Blackboard X
\newcommand{\YY}{\mathbb{Y}}   % Blackboard Y
\newcommand{\ZZ}{\mathbb{Z}}   % Blackboard Z

% Brackets

\renewcommand{\eval}[1]{\left. #1 \right|}          % Evaluation
\newcommand{\br}{\del}                              % Brackets
\newcommand{\abr}[1]{\left\langle #1 \right\rangle} % Angle brackets
\newcommand{\fbr}[1]{\left\lfloor #1 \right\rfloor} % Floor brackets
\newcommand{\lbr}[1]{\left\lfloor #1 \right\rfloor} % Ceiling brackets
\newcommand{\st}{\ \middle| \ }                     % Such that

% Calligraphic

\newcommand{\AAA}{\mathcal{A}} % Calligraphic A
\newcommand{\BBB}{\mathcal{B}} % Calligraphic B
\newcommand{\CCC}{\mathcal{C}} % Calligraphic C
\newcommand{\DDD}{\mathcal{D}} % Calligraphic D
\newcommand{\EEE}{\mathcal{E}} % Calligraphic E
\newcommand{\FFF}{\mathcal{F}} % Calligraphic F
\newcommand{\GGG}{\mathcal{G}} % Calligraphic G
\newcommand{\HHH}{\mathcal{H}} % Calligraphic H
\newcommand{\III}{\mathcal{I}} % Calligraphic I
\newcommand{\JJJ}{\mathcal{J}} % Calligraphic J
\newcommand{\KKK}{\mathcal{K}} % Calligraphic K
\newcommand{\LLL}{\mathcal{L}} % Calligraphic L
\newcommand{\MMM}{\mathcal{M}} % Calligraphic M
\newcommand{\NNN}{\mathcal{N}} % Calligraphic N
\newcommand{\OOO}{\mathcal{O}} % Calligraphic O
\newcommand{\PPP}{\mathcal{P}} % Calligraphic P
\newcommand{\QQQ}{\mathcal{Q}} % Calligraphic Q
\newcommand{\RRR}{\mathcal{R}} % Calligraphic R
\newcommand{\SSS}{\mathcal{S}} % Calligraphic S
\newcommand{\TTT}{\mathcal{T}} % Calligraphic T
\newcommand{\UUU}{\mathcal{U}} % Calligraphic U
\newcommand{\VVV}{\mathcal{V}} % Calligraphic V
\newcommand{\WWW}{\mathcal{W}} % Calligraphic W
\newcommand{\XXX}{\mathcal{X}} % Calligraphic X
\newcommand{\YYY}{\mathcal{Y}} % Calligraphic Y
\newcommand{\ZZZ}{\mathcal{Z}} % Calligraphic Z

% Fraktur

\newcommand{\aaa}{\mathfrak{a}}   % Fraktur a
\newcommand{\bbb}{\mathfrak{b}}   % Fraktur b
\newcommand{\ccc}{\mathfrak{c}}   % Fraktur c
\newcommand{\ddd}{\mathfrak{d}}   % Fraktur d
\newcommand{\eee}{\mathfrak{e}}   % Fraktur e
\newcommand{\fff}{\mathfrak{f}}   % Fraktur f
\renewcommand{\ggg}{\mathfrak{g}} % Fraktur g
\newcommand{\hhh}{\mathfrak{h}}   % Fraktur h
\newcommand{\iii}{\mathfrak{i}}   % Fraktur i
\newcommand{\jjj}{\mathfrak{j}}   % Fraktur j
\newcommand{\kkk}{\mathfrak{k}}   % Fraktur k
\renewcommand{\lll}{\mathfrak{l}} % Fraktur l
\newcommand{\mmm}{\mathfrak{m}}   % Fraktur m
\newcommand{\nnn}{\mathfrak{n}}   % Fraktur n
\newcommand{\ooo}{\mathfrak{o}}   % Fraktur o
\newcommand{\ppp}{\mathfrak{p}}   % Fraktur p
\newcommand{\qqq}{\mathfrak{q}}   % Fraktur q
\newcommand{\rrr}{\mathfrak{r}}   % Fraktur r
\newcommand{\sss}{\mathfrak{s}}   % Fraktur s
\newcommand{\ttt}{\mathfrak{t}}   % Fraktur t
\newcommand{\uuu}{\mathfrak{u}}   % Fraktur u
\newcommand{\vvv}{\mathfrak{v}}   % Fraktur v
\newcommand{\www}{\mathfrak{w}}   % Fraktur w
\newcommand{\xxx}{\mathfrak{x}}   % Fraktur x
\newcommand{\yyy}{\mathfrak{y}}   % Fraktur y
\newcommand{\zzz}{\mathfrak{z}}   % Fraktur z

% Geometry

\newcommand{\CP}{\mathbb{CP}}                                              % Complex projective space
\newcommand{\iintd}[4]{\iint_{#1} \, #2 \, \dif #3 \, \dif #4}             % Double integral
\newcommand{\RP}{\mathbb{RP}}                                              % Real projective space
\newcommand{\intd}[4]{\int_{#1}^{#2} \, #3 \, \dif #4}                     % Single integral
\newcommand{\iiintd}[5]{\iint_{#1} \, #2 \, \dif #3 \, \dif #4 \, \dif #5} % Triple integral

% Logic

\newcommand{\iffb}[2]{\br{#1 \leftrightarrow #2}} % Biconditional
\newcommand{\andb}[2]{\br{#1 \land #2}}           % Conjunction
\newcommand{\orb}[2]{\br{#1 \lor #2}}             % Disjunction
\newcommand{\nib}[2]{\br{#1 \notin #2}}           % Element of
\newcommand{\eqb}[2]{\br{#1 = #2}}                % Equal to
\newcommand{\teb}[1]{\br{\exists #1}}             % Existential quantifier
\newcommand{\impb}[2]{\br{#1 \rightarrow #2}}     % Implication
\newcommand{\ltb}[2]{\br{#1 < #2}}                % Less than
\newcommand{\leb}[2]{\br{#1 \le #2}}              % Less than or equal to
\newcommand{\notb}[1]{\br{\neg #1}}               % Negation
\newcommand{\inb}[2]{\br{#1 \in #2}}              % Not element of
\newcommand{\neb}[2]{\br{#1 \ne #2}}              % Not equal to
\newcommand{\subb}[2]{\br{#1 \subseteq #2}}       % Subset
\newcommand{\fab}[1]{\br{\forall #1}}             % Universal quantifier

% Maps

\newcommand{\bijection}[7][]{    % Bijection
  \ifx &#1&
    \begin{array}{rcl}
      #2 & \longleftrightarrow & #3 \\
      #4 & \longmapsto         & #5 \\
      #6 & \longmapsfrom       & #7
    \end{array}
  \else
    \begin{array}{ccrcl}
      #1 & : & #2 & \longrightarrow & #3 \\
         &   & #4 & \longmapsto     & #5 \\
         &   & #6 & \longmapsfrom   & #7
    \end{array}
  \fi
}
\newcommand{\birational}[7][]{   % Birational map
  \ifx &#1&
    \begin{array}{rcl}
      #2 & \dashrightarrow & #3 \\
      #4 & \longmapsto     & #5 \\
      #6 & \longmapsfrom   & #7
    \end{array}
  \else
    \begin{array}{ccrcl}
      #1 & : & #2 & \dashrightarrow & #3 \\
         &   & #4 & \longmapsto     & #5 \\
         &   & #6 & \longmapsfrom   & #7
    \end{array}
  \fi
}
\newcommand{\correspondence}[2]{ % Correspondence
  \cbr{
    \begin{array}{c}
      #1
    \end{array}
  }
  \qquad
  \leftrightsquigarrow
  \qquad
  \cbr{
    \begin{array}{c}
      #2
    \end{array}
  }
}
\newcommand{\function}[5][]{     % Function
  \ifx &#1&
    \begin{array}{rcl}
      #2 & \longrightarrow & #3 \\
      #4 & \longmapsto     & #5
    \end{array}
  \else
    \begin{array}{ccrcl}
      #1 & : & #2 & \longrightarrow & #3 \\
         &   & #4 & \longmapsto     & #5
    \end{array}
  \fi
}
\newcommand{\functions}[7][]{    % Functions
  \ifx &#1&
    \begin{array}{rcl}
      #2 & \longrightarrow & #3 \\
      #4 & \longmapsto     & #5 \\
      #6 & \longmapsto     & #7
    \end{array}
  \else
    \begin{array}{ccrcl}
      #1 & : & #2 & \longrightarrow & #3 \\
         &   & #4 & \longmapsto     & #5 \\
         &   & #6 & \longmapsto     & #7
    \end{array}
  \fi
}
\newcommand{\rational}[5][]{     % Rational map
  \ifx &#1&
    \begin{array}{rcl}
      #2 & \dashrightarrow & #3 \\
      #4 & \longmapsto     & #5
    \end{array}
  \else
    \begin{array}{ccrcl}
      #1 & : & #2 & \dashrightarrow & #3 \\
         &   & #4 & \longmapsto     & #5
    \end{array}
  \fi
}

% Matrices

\newcommand{\onebytwo}[2]{      % One by two matrix
  \begin{pmatrix}
    #1 & #2
  \end{pmatrix}
}
\newcommand{\onebythree}[3]{    % One by three matrix
  \begin{pmatrix}
    #1 & #2 & #3
  \end{pmatrix}
}
\newcommand{\twobyone}[2]{      % Two by one matrix
  \begin{pmatrix}
    #1 \\
    #2
  \end{pmatrix}
}
\newcommand{\twobytwo}[4]{      % Two by two matrix
  \begin{pmatrix}
    #1 & #2 \\
    #3 & #4
  \end{pmatrix}
}
\newcommand{\threebyone}[3]{    % Three by one matrix
  \begin{pmatrix}
    #1 \\
    #2 \\
    #3
  \end{pmatrix}
}
\newcommand{\threebythree}[9]{  % Three by three matrix
  \begin{pmatrix}
    #1 & #2 & #3 \\
    #4 & #5 & #6 \\
    #7 & #8 & #9
  \end{pmatrix}
}
\newcommand{\twobytwosmall}[4]{ % Two by two small matrix
  \begin{psmallmatrix}
    #1 & #2 \\
    #3 & #4
  \end{psmallmatrix}
}

% Number theory

\renewcommand{\symbol}[2]{\br{\tfrac{#1}{#2}}} % Power residue symbol
\newcommand{\unit}[1]{\br{\ZZ / #1\ZZ}^\times} % Unit group

% Operators

\newoperator{ab}    % Abelian
\newoperator{AG}    % Affine geometry
\newoperator{alg}   % Algebraic
\newoperator{Ann}   % Annihilator
\newoperator{area}  % Area
\newoperator{Aut}   % Automorphism
\newoperator{card}  % Cardinality
\newoperator{ch}    % Characteristic
\newoperator{Cl}    % Class
\newoperator{Coker} % Cokernel
\newoperator{col}   % Column
\newoperator{Corr}  % Correspondence
\newoperator{diam}  % Diameter
\newoperator{Disc}  % Discriminant
\newoperator{dom}   % Domain
\newoperator{Eig}   % Eigenvalue
\newoperator{Em}    % Embedding
\newoperator{End}   % Endomorphism
\newoperator{fin}   % Finite
\newoperator{Fix}   % Fixed
\newoperator{Frac}  % Fraction
\newoperator{Frob}  % Frobenius
\newoperator{Fun}   % Function
\newoperator{Gal}   % Galois
\newoperator{GL}    % General linear
\newoperator{Ham}   % Hamming
\newoperator{Homeo} % Homeomorphism
\newoperator{Hom}   % Homomorphism
\newoperator{id}    % Identity
\newoperator{Im}    % Image
\newoperator{Ind}   % Index
\newoperator{Ker}   % Kernel
\newoperator{lcm}   % Least common multiple
\newoperator{Mat}   % Matrix
\newoperator{mult}  % Multiplicity
\newoperator{new}   % New
\newoperator{Nm}    % Norm
\newoperator{old}   % Old
\newoperator{op}    % Opposite
\newoperator{ord}   % Order
\newoperator{Pay}   % Payley
\newoperator{PG}    % Projective geometry
\newoperator{PGL}   % Projective general linear
\newoperator{PSL}   % Projective special linear
\newoperator{rad}   % Radical
\newoperator{ran}   % Range
\newoperator{Res}   % Residue
\newoperator{rk}    % Rank
\newoperator{Re}    % Real
\newoperator{row}   % Row
\newoperator{sgn}   % Sign
\newoperator{Sing}  % Singular
\newoperator{SK}    % Skeleton
\newoperator{sp}    % Span
\newoperator{SL}    % Special linear
\newoperator{SO}    % Special orthogonal
\newoperator{Spec}  % Spectrum
\newoperator{Stab}  % Stabiliser
\newoperator{star}  % Star
\newoperator{srg}   % Strongly regular graph
\newoperator{supp}  % Support
\newoperator{Sym}   % Symmetric
\newoperator{tors}  % Torsion
\newoperator{Tr}    % Trace
\newoperator{vol}   % Volume
\newoperator{wt}    % Weight

% Roman

\newcommand{\A}{\mathrm{A}}   % Roman A
\newcommand{\B}{\mathrm{B}}   % Roman B
\newcommand{\C}{\mathrm{C}}   % Roman C
\newcommand{\D}{\mathrm{D}}   % Roman D
\newcommand{\E}{\mathrm{E}}   % Roman E
\newcommand{\F}{\mathrm{F}}   % Roman F
\newcommand{\G}{\mathrm{G}}   % Roman G
\renewcommand{\H}{\mathrm{H}} % Roman H
\newcommand{\I}{\mathrm{I}}   % Roman I
\newcommand{\J}{\mathrm{J}}   % Roman J
\newcommand{\K}{\mathrm{K}}   % Roman K
\renewcommand{\L}{\mathrm{L}} % Roman L
\newcommand{\M}{\mathrm{M}}   % Roman M
\newcommand{\N}{\mathrm{N}}   % Roman N
\renewcommand{\O}{\mathrm{O}} % Roman O
\renewcommand{\P}{\mathrm{P}} % Roman P
\newcommand{\Q}{\mathrm{Q}}   % Roman Q
\newcommand{\R}{\mathrm{R}}   % Roman R
\renewcommand{\S}{\mathrm{S}} % Roman S
\newcommand{\T}{\mathrm{T}}   % Roman T
\newcommand{\U}{\mathrm{U}}   % Roman U
\newcommand{\V}{\mathrm{V}}   % Roman V
\newcommand{\W}{\mathrm{W}}   % Roman W
\newcommand{\X}{\mathrm{X}}   % Roman X
\newcommand{\Y}{\mathrm{Y}}   % Roman Y
\newcommand{\Z}{\mathrm{Z}}   % Roman Z

\renewcommand{\a}{\mathrm{a}} % Roman a
\renewcommand{\b}{\mathrm{b}} % Roman b
\renewcommand{\c}{\mathrm{c}} % Roman c
\renewcommand{\d}{\mathrm{d}} % Roman d
\newcommand{\e}{\mathrm{e}}   % Roman e
\newcommand{\f}{\mathrm{f}}   % Roman f
\newcommand{\g}{\mathrm{g}}   % Roman g
\newcommand{\h}{\mathrm{h}}   % Roman h
\renewcommand{\i}{\mathrm{i}} % Roman i
\renewcommand{\j}{\mathrm{j}} % Roman j
\renewcommand{\k}{\mathrm{k}} % Roman k
\renewcommand{\l}{\mathrm{l}} % Roman l
\newcommand{\m}{\mathrm{m}}   % Roman m
\renewcommand{\n}{\mathrm{n}} % Roman n
\renewcommand{\o}{\mathrm{o}} % Roman o
\newcommand{\p}{\mathrm{p}}   % Roman p
\newcommand{\q}{\mathrm{q}}   % Roman q
\renewcommand{\r}{\mathrm{r}} % Roman r
\newcommand{\s}{\mathrm{s}}   % Roman s
\renewcommand{\t}{\mathrm{t}} % Roman t
\renewcommand{\u}{\mathrm{u}} % Roman u
\renewcommand{\v}{\mathrm{v}} % Roman v
\newcommand{\w}{\mathrm{w}}   % Roman w
\newcommand{\x}{\mathrm{x}}   % Roman x
\newcommand{\y}{\mathrm{y}}   % Roman y
\newcommand{\z}{\mathrm{z}}   % Roman z

% Tikz

\tikzset{
  arrow symbol/.style={"#1" description, allow upside down, auto=false, draw=none, sloped},
  subset/.style={arrow symbol={\subset}},
  cong/.style={arrow symbol={\cong}}
}

% Fancy header

\pagestyle{fancy}
\lhead{\module}
\rhead{\nouppercase{\leftmark}}

% Make title

\title{\module}
\author{Lectured by \lecturer \\ Typed by David Kurniadi Angdinata}
\date{\term}

\begin{document}

% Title page
\maketitle
\cover
\vfill
\begin{abstract}
\noindent\syllabus
\end{abstract}

\pagebreak

% Contents page
\tableofcontents

\pagebreak

% Document page
\setcounter{section}{-1}

\section{Introduction}

\lecture{1}{Thursday}{03/10/19}

Groups are ubiquitous throughout almost all areas in mathematics and many areas in physics. They arise naturally as the symmetries of classical mathematical objects, that is bijective maps which preserve the structure of the object studied. Well known groups include $ \SSS_n $, the group of symmetries of a set of size $ n $, or $ \DDD_n $, the group of symmetries of a regular $ n $-gon. From linear algebra we also know $ \GL_n\br{\RR} $, the group of all invertible linear transformations of the vector space $ \RR^n $, and $ \O\br{n} $, its subgroup of isometries. Historically, groups appeared for the first time in the work of Galois, when he tried to understand solutions of polynomial equations by studying the group of symmetries of their roots. He was the first to use the word group in the modern sense and that work dates back to 1829, when he was 18 years old. Another main contribution to the study of groups in mathematics came from Felix Klein's Erlangen program in 1872, in which he aimed to understand and classify euclidean, affine, projective, etc geometries by studying their group of symmetries. A huge milestone in the study of groups has been the classification of finite simple groups, which is a result based on the accumulated work of more than 100 authors on tens of thousands of pages published between 1954 and 2004.

This course will focus on infinite groups. More specifically, we will aim to study and understand groups by their actions on geometric objects. In that sense, we can consider the course program as an inverse of Klein's Erlangen program. This area of mathematics goes back to the 1980s, hence is comparably new, and is nowadays wider known as geometric group theory. The two leading questions will roughly be the following.
\begin{itemize}
\item If we know that a given group $ G $ admits an action with properties $ P $ on a space of type $ T $, what
does this tell me about the group $ G $ itself?
\item Assume we are given a group $ G $. Does it act on a given space $ T $ with properties $ P $?
\end{itemize}

Our main goal in the first part will be the fundamental theorem of Bass-Serre theory, which states that a group acting on a tree is the fundamental group of a graph of groups. We first will introduce the notion of graphs in the sense of Serre and study group actions on these graphs. Afterwards, we will introduce free groups as the universal object in the class of groups and study how groups can arise as fundamental groups of graphs. We will see that groups can be presented by giving a set of generators accompanied with a set of relators and point out advantages and disadvantages of this viewpoint on groups.

In the second chapter, we will learn how to construct new groups out of given data via free products, free amalgamated products and HNN extensions. The counterpoint to this, that is the question on whether a given group decomposes into the amalgamated product or HNN extension of other groups, will be of special interest and we will approach it by understanding their actions on trees. This second part concludes with the introduction of graphs of groups and the fundamental theorem of Bass-Serre theory.

In the last part, we will investigate the word problem and its solvability in specific classes of groups. The word problem asks if two words on the generators of some group $ G $ represent the same element in it. Even for finitely presentable groups, the word problem is not always solvable, that is decidable. We will get to know Hopfian and residually finite groups as examples of classes in which the words problem actually is solvable. If time permits, we will conclude the lecture with an introduction into hyperbolic groups.

The following are reading material.
\begin{itemize}
\item R C Lyndon and P E Schupp, Combinatorial group theory, 2001
\item P de la Harpe, Topics in geometric group theory, 2000
\item O Bogopolski, Introduction to group theory, 2008
\item J Rotman, An introduction to the theory of groups, 1995
\item W Magnus, A Karrass, and D Solitar, Combinatorial group theory, 2005
\item D Robinson, A course in the theory of groups, 1993
\end{itemize}

\pagebreak

\section{Geometric group theory}

\subsection{Bass-Serre graphs}

\begin{definition}
A \textbf{graph} $ X $ is a tuple consisting of a set of vertices $ X^0 $, a set of edges $ X^1 $, together with functions $ \alpha, \omega : X^1 \to X^0 $ and $ \overline{\cdot} : X^1 \to X^1 $, such that $ \overline{\overline{e}} = e $ and $ \alpha\br{\overline{e}} = \omega\br{e} $ for every $ e \in X^1 $. We call $ \alpha\br{e} $ the \textbf{initial vertex}, $ \omega\br{e} $ the \textbf{terminal vertex}, and $ \overline{e} $ the \textbf{inverse vertex}.
\end{definition}

A convention is that unless otherwise specified, we identify edges $ e $ and $ e' $ if $ \alpha\br{e} = \alpha\br{e'} $ and $ \omega\br{e} = \omega\br{e'} $. The following are translations of notions.
\begin{itemize}
\item A subgraph is an \textbf{induced} subgraph.
\item A graph homomorphism $ \phi $ from $ X $ to $ Y $ is a mapping from $ X^i \to Y^i $ for $ i = 0, 1 $ such that $ \phi\br{\alpha\br{e}} = \alpha\br{\phi\br{e}} $ and $ \phi\br{\overline{e}} = \overline{\phi\br{e}} $.
\item Given $ x \in X^0 $, then we call the set $ \cbr{e \st \alpha\br{e} = x} $ the \textbf{star} of $ x $, or $ \star x $. The cardinality of $ \star x $ is called the \textbf{valency} of $ x $.
\item A homomorphism $ \phi : X \to Y $ is \textbf{locally injective} if and only if its restriction to $ \star x $ is injective for all $ x \in X^0 $.
\item An \textbf{orientation} of $ X $ is a choice of vertices $ X_+^1 \subseteq X^1 $ which picks exactly one of each pair $ \cbr{e, \overline{e}} $.
\end{itemize}

\begin{example}
\hfill
\begin{itemize}
\item Fix $ n \in \NN_{\ge 1} $ for $ n \ne 2 $. Set
$$ \CCC_n^0 = \cbr{0, \dots, n - 1}, \qquad \CCC_n^1 = \cbr{e_i, \overline{e_i} \st i < n}, \qquad \omega\br{e_i} = \alpha\br{e_{i + 1}} = i + 1 \mod n_i, \qquad i < n. $$
Then $ \CCC_1 $ is
$$
\begin{tikzpicture}
\fill (0, 0) circle (0.1) node[left]{$ 0 $};
\draw [->] (0, 0) arc (180:-170:0.5);
\draw (1, 0) node[right]{$ e_0 = \overline{e_0} $};
\end{tikzpicture}.
$$
\item $ \CCC_\infty $ is given by
$$ \CCC_\infty^0 = \ZZ, \qquad \CCC_\infty^1 = \cbr{e_i, \overline{e_i} \st i \in \ZZ}, \qquad \omega\br{e_i} = \alpha\br{e_{i + 1}}. $$
Then $ \CCC_\infty $ is
$$
\begin{tikzpicture}
\fill (0, 0) circle (0.1) node[above]{$ 0 $};
\fill (1, 0) circle (0.1) node[above]{$ 1 $};
\draw [dashed] (-2, 0) to (-1, 0);
\draw [->] (-1, 0) to node[above]{$ e_{-1} $} (-0.1, 0);
\draw [->] (0, 0) to node[above]{$ e_0 $} (0.9, 0);
\draw [->] (1, 0) to node[above]{$ e_1 $} (1.9, 0);
\draw [dashed] (2, 0) -- (3, 0);
\end{tikzpicture}.
$$
\end{itemize}
The graphs $ \CCC_n $ and $ \CCC_\infty $ for $ n \ne 2 $ are called \textbf{circuits}.
\end{example}

\lecture{2}{Tuesday}{08/10/19}

A sequence $ p = e_1 \dots e_n $ with $ e_i \in X^1 $ is called a \textbf{path} from $ \alpha\br{e_1} = x_0 $ to $ x_n = \omega\br{e_n} $ if and only if $ \omega\br{e_i} = \alpha\br{e_{i + 1}} $ for all $ i < n $. We consider vertices to be paths of length zero. A path is called \textbf{reduced} if $ \overline{e_i} \ne e_{i + 1} $. If $ p $ is a path, then $ p^{-1} = \overline{e_n} \dots \overline{e_1} $ is called its \textbf{inverse path}. A path is called a \textbf{closed path} if $ \omega\br{e_n} = \alpha\br{e_1} $.

\begin{note*}
\hfill
\begin{itemize}
\item If we have a path $ p $ given, we can naturally consider it to be a subgraph via
$$ X_p^0 = \cbr{\alpha\br{e_i} \st i < n} \cup \br{\omega\br{e_n}}, \qquad X_p^1 = \cbr{e_1, \dots, e_n, \overline{e_1}, \dots, \overline{e_n}}. $$
\item If $ p = e_1 \dots e_n $ is closed, then a permutation of the form
$$ e_{i + 1} \dots e_ne_1 \dots e_i $$
is called a \textbf{cyclic permutation}. $ p $ is called \textbf{cyclically reduced} if every cyclic permutation is reduced.
\end{itemize}
\end{note*}

\pagebreak

\begin{exercise}
\hfill
\begin{itemize}
\item Let $ \phi : X \to Y $ be a morphism of graphs. Then $ \phi $ is locally injective if and only if the image of any reduced path is reduced.
\item If $ p $ is closed and reduced, then it contains a circuit as a substructure.
\end{itemize}
\end{exercise}

If $ p = e_1 \dots e_n $ and $ q = f_1 \dots f_m $ such that $ \omega\br{e_n} = \alpha\br{f_1} $ then we denote by
$$ pq = e_1 \dots e_nf_1 \dots f_m. $$
A graph $ X $ is \textbf{connected} if for any $ x, y \in X^0 $ there is a path from $ x $ to $ y $. A connected graph without circuits is called a \textbf{tree}.

\begin{exercise}
\hfill
\begin{itemize}
\item $ X $ is a tree if and only if for all $ x, y \in X^0 $ there is a unique reduced path from $ x $ to $ y $.
\item If $ X $ is connected and $ T $ is a tree, then any $ \phi : X \to T $ locally injective is already injective and $ X $ is a tree.
\end{itemize}
\end{exercise}

\begin{lemma}
Let $ X $ be a connected graph and $ T \subseteq X $ a maximal subtree of $ X $, then $ T^0 = X^0 $.
\end{lemma}

\begin{proof}
Otherwise, there is some $ x \in X^0 \setminus T^0 $. As $ X $ is connected, there is some path $ p $ starting in $ T $, ending in $ x $. As $ x \notin T^0 $, there exists an edge in $ p $ such that $ \alpha\br{e} \in T^0 $ and $ \omega\br{e} \notin T^0 $. But then
$$ T' = \br{T^0 \cup \cbr{\omega\br{e}}, T^1 \cup \cbr{e}} $$
is again a tree, a contradiction.
\end{proof}

Such a tree $ T $ is called a \textbf{spanning tree} for $ X $.

\subsection{Cayley graphs}

\begin{definition}
Let $ G $ be a group and $ X $ a graph. We say that $ G $ \textbf{acts} on $ X $ if and only if it acts on $ X^0 $ and $ X^1 $ as sets, such that
\begin{itemize}
\item $ g \cdot \alpha\br{e} = \alpha\br{g \cdot e} $, and
\item $ g \cdot \overline{e} = \overline{g \cdot e} $.
\end{itemize}
\end{definition}

\begin{note*}
This just means that
$$ \function[\phi_g]{X^0}{X^0}{x}{g \cdot x} $$
is a morphism of graphs for any $ g \in G $.
\end{note*}

\begin{notation*}
$ gh $ is multiplication and $ g \cdot h $ is action.
\end{notation*}

\begin{remark*}
Given $ G $ and $ X $ arbitrary, then $ G $ acts on $ X $ by $ g \cdot x = x $ and $ g \cdot e = e $. Hence we will ask for nice properties of the action.
\end{remark*}

\begin{definition}
Assume $ G $ acts on a graph $ X $. Then we say that $ G $ \textbf{acts without inversion of edges}, if $ g \cdot e \ne \overline{e} $ for all $ e \in X^1 $. We say that $ G $ \textbf{acts freely} on $ X $, if $ g \cdot x = x $ if and only if $ g = e_G $.
\end{definition}

\begin{definition}
Let $ G $ be a group and $ S \subseteq G \setminus \cbr{e_G} $.
\begin{itemize}
\item We say that $ S $ \textbf{generates} $ G $, or $ G $ is \textbf{generated} by $ S $, if there is no proper subgroup of $ G $ containing $ S $. That is, the smallest subgroup $ H $ containing $ S $ equals $ G $.
\item If $ S $ has some property $ P $, then we say that $ G $ is \textbf{$ P $-ly generated}. For example, if $ S $ is finite, then $ G $ is finitely generated.
\item If $ P $ is a property of subgroups, then $ S $ \textbf{$ P $-ly generates} $ G $, if the smallest subgroup of $ G $ containing $ S $ with property $ P $, is already $ G $. For example, if the smallest normal subgroup of $ G $ containing $ S $ is already $ G $, then $ S $ normally generates $ G $.
\end{itemize}
\end{definition}

\begin{example}
$ \br{\ZZ, +} $ is generated by $ \cbr{1} $ or $ \cbr{-1} $ or $ \cbr{-1, 1} $ or $ \cbr{2, 3} $ or $ \ZZ \setminus \cbr{0} $.
\end{example}

\pagebreak

\begin{example}
Let $ G $ be an infinite simple group. Then it is normally generated by any $ g \in G \setminus \cbr{e_G} $. A question is can it be generated by $ g $? No. $ G $ is cyclic and simple if and only if $ G = \ZZ / p\ZZ $ for $ p $ prime. \footnote{Exercise} $ A_\infty $ is an infinite simple group.
\end{example}

\lecture{3}{Wednesday}{09/10/19}

\begin{definition}
Assume $ G $ is a group and $ S \subseteq G \setminus \cbr{e_G} $. Then we define the graph $ \Gamma\br{G, S} $ via
\begin{itemize}
\item the vertex set is $ \Gamma\br{G, S}^0 = G $,
\item the set of positive edges is $ \Gamma\br{G, S}_+^1 = G \times S $,
\item for $ e $ an edge, we have $ \alpha\br{\br{g, s}} = g $ and $ \omega\br{\br{g, s}} = gs $, and
\item the inverse of $ \br{g, s} $ is $ \overline{\br{g, s}} = \br{gs, s^{-1}} $, where
$$ S^{-1} = \cbr{s^{-1} \st s \in S} $$
is a set of new formal symbols. Thus $ \br{g, s^{-1}} \notin G \times S $, even if as elements $ s^{-1} = s' \in S $. If $ s = s^{-1} $, this avoids troubles.
\end{itemize}
We consider $ \Gamma\br{G, S} $ to be a labelled graph, where the label of $ \br{g, s} $ is $ s $.
\end{definition}

\begin{exercise}
\hfill
\begin{itemize}
\item $ \Gamma\br{G, S} $ is connected if and only if $ S $ is a generating set for $ G $.
\item Otherwise set $ H = \abr{S} \subsetneq G $. How does $ H $ relate to $ \Gamma\br{G, S} $?
\end{itemize}
\end{exercise}

\begin{definition}
If $ G $ is a group and $ S \subseteq G \setminus \cbr{e_G} $ generates $ G $, then $ \Gamma\br{G, S} $ is called the \textbf{Cayley graph} of $ G $ with respect to $ S $.
\end{definition}

\begin{exercise}
Given $ S $ a connected graph. Is there a group $ G $ and $ S \subseteq G \setminus \cbr{e_G} $ such that $ X \cong \Gamma\br{G, S} $, where $ S $ is a generating set?
\end{exercise}

\begin{example}
\hfill
\begin{itemize}
\item Recall $ \CCC_n $ and $ \CCC_\infty $. Then
$$ \CCC_n \cong \Gamma\br{\ZZ / n\ZZ, \cbr{1}}, \qquad \CCC_\infty \cong \Gamma\br{\ZZ, \cbr{1}}. $$
\item Careful. Cayley graphs depend heavily on the choice of $ S $. It is not always easy to determine whether it is cyclic. Consider
$$ \Gamma_1 = \Gamma\br{\SSS_3, \cbr{\br{123}, \br{12}}}, \qquad \Gamma_2 = \Gamma\br{\ZZ / 6\ZZ, \cbr{2, 3}}. $$
Then
$$
\begin{array}{ccc}
\begin{tikzpicture}
\fill (0, 0) circle (0.1);
\fill (0, -1) circle (0.1);
\fill (-0.5, -2) circle (0.1);
\fill (0.5, -2) circle (0.1);
\fill (-1.5, -2.5) circle (0.1);
\fill (1.5, -2.5) circle (0.1);
\draw [->] (0, 0) to [bend left=45] node[right]{$ \br{12} $} (0.1, -1);
\draw [->] (0, 0) to [bend right=45] node[left]{$ \br{123} $} (-1.5, -2.4);
\draw [->] (0, -1) to [bend left=45] node[left]{$ \br{12} $} (-0.1, 0);
\draw [->] (0, -1) to [bend left=45] node[right]{$ \br{123} $} (0.5, -1.9);
\draw [->] (-0.5, -2) to [bend left=45] node[left]{$ \br{123} $} (-0.1, -1);
\draw [->] (-0.5, -2) to [bend left=45] node[below]{$ \br{12} $} (-1.4, -2.5);
\draw [->] (0.5, -2) to [bend left=45] node[below]{$ \br{123} $} (-0.5, -2.1);
\draw [->] (0.5, -2) to [bend left=45] node[above]{$ \br{12} $} (1.5, -2.4);
\draw [->] (-1.5, -2.5) to [bend left=45] node[above]{$ \br{12} $} (-0.6, -2);
\draw [->] (-1.5, -2.5) to [bend right=45] node[below]{$ \br{123} $} (1.5, -2.6);
\draw [->] (1.5, -2.5) to [bend right=45] node[right]{$ \br{123} $} (0.1, 0);
\draw [->] (1.5, -2.5) to [bend left=45] node[below]{$ \br{12} $} (0.5, -2.1);
\end{tikzpicture}
& \qquad &
\begin{tikzpicture}
\fill (0, 0) circle (0.1);
\fill (0, -1) circle (0.1);
\fill (-0.5, -2) circle (0.1);
\fill (0.5, -2) circle (0.1);
\fill (-1.5, -2.5) circle (0.1);
\fill (1.5, -2.5) circle (0.1);
\draw [->] (0, 0) to [bend left=45] node[right]{$ 3 $} (0.1, -1);
\draw [->] (0, 0) to [bend left=45] node[right]{$ 2 $} (1.5, -2.4);
\draw [->] (0, -1) to [bend left=45] node[left]{$ 3 $} (-0.1, 0);
\draw [->] (0, -1) to [bend left=45] node[right]{$ 2 $} (0.5, -1.9);
\draw [->] (-0.5, -2) to [bend left=45] node[left]{$ 2 $} (-0.1, -1);
\draw [->] (-0.5, -2) to [bend left=45] node[below]{$ 3 $} (-1.4, -2.5);
\draw [->] (0.5, -2) to [bend left=45] node[below]{$ 2 $} (-0.5, -2.1);
\draw [->] (0.5, -2) to [bend left=45] node[above]{$ 3 $} (1.5, -2.4);
\draw [->] (-1.5, -2.5) to [bend left=45] node[left]{$ 2 $} (-0.1, 0);
\draw [->] (-1.5, -2.5) to [bend left=45] node[above]{$ 3 $} (-0.6, -2);
\draw [->] (1.5, -2.5) to [bend left=45] node[below]{$ 3 $} (0.5, -2.1);
\draw [->] (1.5, -2.5) to [bend left=45] node[below]{$ 2 $} (-1.5, -2.6);
\end{tikzpicture}
\\
\Gamma_1 & & \Gamma_2
\end{array}.
$$
Given $ \Gamma_i $, is the group abelian? A group is abelian if and only if all its generators commute, that is $ ab = ba $. For $ \Gamma_2 $, if $ a = 2 $ and $ b = 3 $, then $ \br{2}\br{3} = \br{3}\br{2} $.
\end{itemize}
\end{example}

\pagebreak

\begin{lemma}
\label{lem:1.2.11}
Every group $ G $ acts on its Cayley graph by left multiplication. The multiplication is free, label-preserving, and without inversion of edges. Furthermore, every $ \phi_g $ is a label-preserving automorphism of $ \Gamma\br{G, S} $.
\end{lemma}

\begin{proof}
Define the action via $ h \cdot g = hg $ for all $ h \in G $ and all $ g \in \Gamma\br{G, S}^0 $, and $ h \cdot \br{g, s} = \br{hg, s} $. One checks easily that this defines an action. It is obviously label-preserving and hence without inversion of edges, as positive and negative edges have disjoint label sets. Now, if $ h \cdot g = g $, then $ hg = g $ and this $ h = e_G $. Hence the action is free. Clearly, $ \phi_h $ is injective, as $ \phi_h\br{g_1} = \phi_h\br{g_2} $ if and only if $ hg_1 = hg_2 $ if and only if $ g_1 = g_2 $. For surjectivity, note that $ g = hh^{-1}g $ and hence $ g = h \cdot \br{h^{-1}g} = \phi_h\br{h^{-1}g} $.
\end{proof}

\lecture{4}{Thursday}{10/10/19}

\begin{lemma}
Let $ G $ be some group and $ S \subseteq G \setminus \cbr{e_G} $ a generating set. Denote by $ \Aut_L \Gamma\br{G, S} $ the label-preserving automorphism group of its Cayley graph. Then
$$ G \cong \Aut_L \Gamma\br{G, S}. $$
\end{lemma}

\begin{proof}
By \ref{lem:1.2.11} we know that
$$ \function[\Phi]{G}{\Aut_L \Gamma\br{G, S}}{h}{\phi_h}. $$
One easily checks that this is a group homomorphism. If $ \phi_h = \phi_g $, then in particular they agree on the vertex $ e_G $, that is $ h = \phi_h\br{e_G} = \phi_g\br{e_G} = g $, so $ g = h $ and $ \Phi $ is injective. Now consider $ \phi \in \Aut_L \Gamma\br{G, S} $ arbitrary. We claim that $ \phi = \phi_h $ with $ h = \phi\br{e_G} $. As $ \phi $ is label-preserving and every vertex has exactly one outgoing and one incoming edge with label $ s $, we know that $ \phi\br{\br{g, s}} = \br{\phi\br{g}, s} $. Hence
$$ \phi\br{\omega\br{\br{g, s}}} = \omega\br{\phi\br{\br{g, s}}} = \omega\br{\br{\phi\br{g}, s}} = \phi\br{g}s. $$
As $ \Gamma\br{G, S} $ is connected, we get that two label-preserving automorphisms agree if and only if they agree on one vertex. Now, $ \phi\br{e_G} = h = \phi_h\br{e_G} $, so $ \phi = \phi_h $.
\end{proof}

\begin{example}
The group of all automorphisms of $ \CCC_n $ is called the \textbf{dihedral group} and denoted by $ \DDD_n $. Note that every such automorphism is uniquely determined by its image on $ e_0 $. Hence if we consider $ \alpha\br{e_0} = e_1 $ and $ \beta\br{e_0} = \overline{e_{n - 1}} $, then
$$ \DDD_n = \cbr{a^k, a^kb \st k < n}, \qquad \DDD_\infty = \cbr{a^k, a^kb \st k \in \ZZ}. $$
\end{example}

\begin{exercise}
\hfill
\begin{itemize}
\item Draw the Cayley graphs of $ \DDD_n $ with respect to $ S = \cbr{a, b} $.
\item Prove that $ \DDD_3 \cong \SSS_3 $.
\item Determine the axis of the reflection and the representation $ a^k $ and $ a^kb $ for given $ \phi $ just by using $ \omega\br{\phi\br{e_0}} $ and $ \alpha\br{\phi\br{e_0}} $.
\end{itemize}
\end{exercise}

\subsection{Words and paths}

\begin{note*}
If for some group element $ g $, both $ g = s_1 $ and $ g^{-1} = s_2 $ are in $ S $, then we distinguish the edges $ e_1 = \br{e_G, s_1} $ and $ e_2 = \br{e_G, s_2^{-1}} $ even though $ \alpha\br{e_1} = e_G = \alpha\br{e_2} $ and $ \omega\br{e_1} = s_1 = g = s_2^{-1} = \omega\br{e_2} $.
\end{note*}

\begin{definition}
Let $ S $ be any set. We say that $ w $ is a \textbf{word} on $ S $ if and only if it is a finite sequence of the form
$$ w = s_1^{\epsilon_1} \dots s_n^{\epsilon_n}, \qquad s_i \in S, \qquad \epsilon_i = -1, 1. $$
We call $ S $ an \textbf{alphabet} and elements of $ S $ are \textbf{letters}. If $ S \subseteq G $, then every word in $ S $ considered as a product, defines some group element. We write
$$ w \underset{G}{=} s_1^{\epsilon_1} \dots s_n^{\epsilon_n} \underset{G}{=} g, $$
and we say that $ w $ \textbf{represents} $ G $.
\end{definition}

\begin{example}
Consider $ \ZZ $ with $ S = \cbr{s_0 = -1, s_1 = 1} $. Then $ w_1 = s_0s_1 \ne s_1^{-1}s_1 = w_2 $ but $ w_1 \underset{G}{=} w_2 $.
\end{example}

\pagebreak

\begin{remark}
If $ S $ is a generating set for $ G $, then for every $ g \in G $, every word in $ S $ corresponds to a unique path $ p_w\br{g} $ in the Cayley graph starting at $ g $ and ending at $ gh $, where $ h \underset{G}{=} w $.
\end{remark}

\begin{example}
Let $ \ZZ \times \ZZ $ and $ S = \cbr{a = \br{1, 0}, b = \br{0, 1}} $. Consider
$$ w_1 = aabbab^{-1}, \qquad w_2 = baaa, \qquad w_3 = aba^{-1}a^{-1}. $$
Then $ w_1 \underset{G}{=} w_2 $ and $ w_3 \underset{G}{=} ba^{-1} \underset{G}{=} a^{-1}b $.
\end{example}

\begin{definition}
A word $ w = s_1^{\epsilon_1} \dots s_n^{\epsilon_n} $ on $ S $ is called \textbf{reduced} if and only if $ s_i = s_{i + 1} $ implies that $ \epsilon_i = \epsilon_{i + 1} $.
\end{definition}

Consider $ s \in G $ with $ s^2 = 1 $. Then $ s \underset{G}{=} s^{-1} $ and $ w = ss^{-1} $ is not reduced. But $ w' = ss $ is reduced.

\subsection{Free groups}

\lecture{5}{Tuesday}{15/10/19}

\begin{fact}[Tits alternative]
If $ G $ is an infinite linear group, then either it is virtually solvable, that is there is a finite index subgroup which is solvable, or it contains a non-abelian free group as a subgroup.
\end{fact}

\begin{definition}[Free groups I]
Let $ G $ be a group and $ S \subseteq G \setminus \cbr{e_G} $ be any subset of $ G $. Then $ G $ is called \textbf{free on $ S $}, or a \textbf{free group with basis $ S $}, if and only if every element of $ G $ can be represented uniquely as a reduced word on $ S $.
\end{definition}

\begin{remark*}
This implies that $ S $ generates $ G $.
\end{remark*}

\begin{example*}
Let $ G $ be finite. Then for all $ s \in S $ there exists $ n \in \NN_{> 0} $ such that $ s^n \underset{G}{=} e_G $, so not unique.
\end{example*}

\begin{exercise}[Free groups II]
A group $ G $ is \textbf{free on $ S \subseteq G $} if and only if $ \Gamma\br{G, S} $ of $ G $ with respect to $ S $ is a tree and $ S \cap S^{-1} = \emptyset $, considered as an intersection in $ G $.
\end{exercise}

\begin{example}
Consider the subgroup $ F \subseteq \SL_2\br{\ZZ} $ generated by
$$ A = \twobytwo{1}{2}{0}{1}, \qquad B = \twobytwo{1}{0}{2}{1}. $$
Then $ F $ is free on $ S = \cbr{A, B} $. First note that
$$ A^n = \twobytwo{1}{2n}{0}{1}, \qquad B^n = \twobytwo{1}{0}{2n}{1}. $$
Clearly, $ F $ acts on $ \RR^2 $. Set
$$ U = \cbr{\br{x, y} \st \abs{x} < \abs{y}}, \qquad V = \cbr{\br{x, y} \st \abs{x} > \abs{y}}. $$
Then
$$ \twobytwo{1}{2n}{0}{1}\twobyone{x}{y} = \twobyone{x + 2ny}{y}, $$
so $ A^n\br{U} \subseteq V $ for all $ n \ge 1 $. Similarly, $ B^n\br{V} \subseteq V $. Assume
$$ w = A^{k_0}B^{l_0} \dots A^{k_{n - 1}}B^{k_{n - 1}}A^{k_n}, \qquad \abs{k_i}, \abs{l_i} > 0, \qquad k_0, l_0, k_n \ge 0 $$
is an arbitrary word on $ S $. Assume $ w \underset{G}{=} e_G = I $. Now, if $ \abs{k_n}, \abs{k_0} > 0 $ then $ w\br{U} \subseteq V $. But $ U \cap V = \emptyset $, a contradiction. Otherwise consider $ w' $, the word which arises by conjugating by a high enough power of $ A $, so $ w' = A^NwA^{-N} $. Then $ w' $ is of the above form. But $ w' \underset{G}{=} e_G $ if and only if $ w \underset{G}{=} e_G $, a contradiction.
\end{example}

\begin{remark*}
This proof generalises to the so-called ping-pong lemma, telling when two subgroups $ \abr{A} $ and $ \abr{B} $ appear as a free product $ \abr{A} * \abr{B} $.
\end{remark*}

\lecture{6}{Wednesday}{16/10/19}

Lecture 6 is a problem class.

\pagebreak

\lecture{7}{Thursday}{17/10/19}

\begin{proposition}[Free groups III]
Let $ S $ be an arbitrary non-empty set. Then there is a \textbf{group $ \F\br{S} $ which is free on $ S $}.
\end{proposition}

\begin{proof}
\hfill
\begin{itemize}
\item Set $ S^\pm = \cbr{s^\epsilon \st \epsilon = -1, 1} $. If $ s_1 = s_2^{-1} \in S^\pm $, identify $ s_1^{-1} = \br{s_2^{-1}}^{-1} = s_2 $. Set $ \F'\br{S} $ to be the set of all words on $ S $. As usual, we denote the empty word by $ \epsilon $. Further, for two words $ u_1 $ and $ u_2 $ given by $ u_i = s_{i, 1}^{\epsilon_{i, 1}} \dots s_{i, n_i}^{\epsilon_{i, n_i}} $ let $ u_1u_2 = s_{1, 1}^{\epsilon_{1, 1}} \dots s_{1, n_1}^{\epsilon_{1, n_1}}s_{2, 1}^{\epsilon_{2, 1}} \dots s_{2, n_2}^{\epsilon_{2, n_2}} $. Note that this is not a group, since no inverses.
\item We will define an equivalence relation on $ \F'\br{S} $. Say that $ u \sim v $ if and only if there exists a finite sequence of words such that $ u = u_0, \dots, u_n = v $ and each $ u_{i + 1} $ arises from $ u_i $ by inserting or deleting a subword of the form $ ss^{-1} $ for $ s \in S^\pm $. We say that $ u $ is reduced, if it does not contain a subword $ ss^{-1} $.
\item Claim that every equivalence class contains exactly one reduced word. Assume $ u \sim v $. Then there exists $ u = u_0, \dots, u_n = v $. Choose this sequence such that $ \sum_{i = 0}^n \abs{u_i} $ is minimal, where $ \abs{u_i} $ denotes the word length of $ u_i $. As $ u $ and $ v $ are reduced, we know that $ \abs{u_0} < \abs{u_1} $ and $ \abs{u_n} < \abs{u_{n - 1}} $. Then there exists $ 0 < i < n $ such that $ \abs{u_{i - 1}} < \abs{u_i} $ and $ \abs{u_{i + 1}} < \abs{u_i} $. Say, $ u_i $ arises from $ u_{i - 1} $ by adding $ ss^{-1} $ and $ u_{i + 1} $ from $ u_i $ by deleting $ tt^{-1} $. Now either $ ss^{-1} $ and $ tt^{-1} $ are disjoint in $ u_i $, then replace the sequence $ u_{i - 1}u_iu_{i + 1} $ by $ u_{i - 1}u_i'u_{i + 1} $ where $ u_i' $ arises from $ u_{i - 1} $ by deleting $ tt^{-1} $, or not, then cancelling the subsequence $ u_iu_{i + 1} $ still gives a connecting sequence from $ u $ to $ v $. In both cases we obtain a sequence of smaller length, a contradiction.
\item Denote by $ \sbr{u} $ the class $ u / \sim $. We set $ \sbr{u}\sbr{v} = \sbr{uv} $. This is clearly independent of choice, that is if $ u' \sim u $, then $ u'v \sim uv $. Hence associativity is clear. Also, if $ w = s_1^{\epsilon_1} \dots s_n^{\epsilon_n} $, then $ \sbr{w} = \sbr{s_1^{\epsilon_1}} \dots \sbr{s_n^{\epsilon_n}} $. Hence $ \sbr{S} = \cbr{\sbr{s} \st s \in S} $ generates $ \F\br{S} $. By the claim, every word has a unique reduced representation in $ \sbr{S} $. Hence $ \F\br{S} $ is free on $ \sbr{S} $.
\end{itemize}
\end{proof}

\begin{proposition}[Free groups IV]
\label{prop:1.4.6}
Assume $ F $ is a group and $ S \subseteq F $ is any set. Then $ F $ is a \textbf{free group with basis $ S $} if and only if $ F $ is the universal object in the class of groups with respect to $ S $, that is whenever $ G $ is a group and $ f : S \to G $ is any map, there exists a unique group homomorphism $ \phi : F \to G $ extending $ f $, so
$$
\begin{tikzcd}
S \arrow{r}{f} & G \\
F \arrow{u}{\id} \arrow[swap]{ur}{\exists !f} &
\end{tikzcd}.
$$
\end{proposition}

\begin{proof}
\hfill
\begin{itemize}
\item[$ \implies $] Assume $ F $ is free on $ S $ and $ f : S \to G $ is any map. We first prove uniqueness of $ \phi $. By definition, for any $ g \in F $ there exists a unique reduced word on $ S $ such that $ w \underset{F}{=} g $, say $ w = s_1^{\epsilon_1} \dots s_n^{\epsilon_n} $. Then, if $ \phi : S \to G $ is a homomorphism of groups extending $ f $, clearly,
$$ \phi\br{g} = f\br{s_1}^{\epsilon_1} \dots f\br{s_n}^{\epsilon_n}. $$
Hence unique. Now for existence we prove that this actually is a homomorphism of groups, that is $ \phi\br{gh} = \phi\br{g}\phi\br{h} $. Let $ w_g \underset{F}{=} g $ and $ w_h \underset{F}{=} h $ be reduced. Then $ w_gw_h \underset{F}{=} gh $, but maybe not reduced. Say $ w_{gh} $ is the reduced word. If $ w_gw_h $ is not reduced, then $ w_g = w_g^1w_g^2 $ and $ w_h = w_h^1w_h^2 $ such that $ w_g^2 = \br{w_h^1}^{-1} $. Then clearly
$$ \phi\br{gh} = f\br{w_g^1w_h^2} = f\br{w_g^1}f\br{w_g^2}f\br{w_h^1}^{-1}f\br{w_h^2} = \phi\br{g}\phi\br{h}. $$
Hence $ \phi $ is indeed a group homomorphism.

\pagebreak

\item[$ \impliedby $] Let $ F $ be any group and $ S \subseteq F $ such that for any group $ G $ and $ f : S \to G $ there exists a unique homomorphism $ \phi : F \to G $ extending $ f $. First, consider $ G_1 = \abr{S} \le F $. Then the map $ f : S \to F $ sending $ S $ to itself extends to some $ \phi : F \to F $ with image $ G_1 $. But also, the homomorphism $ \id_F : F \to F $ is a homomorphism extending $ f $. By uniqueness, $ \phi = \id_F $ whence $ \abr{S} = G_1 = F $. Hence $ S $ generates $ F $. Next, consider $ G = \F\br{S} $ and $ f : S \to \F\br{S} $ the natural embedding from $ S $ into $ \F\br{S} $. We want to show that any element in $ F $ has a unique reduced representation on $ S $. Indeed, assume $ w = s_1^{\epsilon_1} \dots s_n^{\epsilon_n} $ is reduced for $ n > 0 $, but $ w \underset{F}{=} e_F $. Then any homomorphism $ \phi : F \to \F\br{S} $ extending $ f $ has to send $ w_G $ simultaneously to $ e_{\F\br{S}} $ and to $ w_{\F\br{S}} $, hence $ e_{\F\br{S}} \underset{\F\br{S}}{=} w $, a contradiction.
\end{itemize}
\end{proof}

\lecture{8}{Tuesday}{22/10/19}

\begin{remark}
Note that we also proved that $ F $ is free on $ S \subseteq F $ if and only if $ F = \F\br{S} $. Hence, for every set $ S $ there is a unique, up to isomorphism, group $ F $ which is free on $ S $.
\end{remark}

\begin{lemma}
Every group $ G $ is the quotient of some free group.
\end{lemma}

\begin{proof}
Consider $ S \subseteq G $ some generating set of $ G $. Consider the homomorphism $ \phi : \F\br{S} \to G $. By \ref{prop:1.4.6} this exists. As $ S $ generates $ G $, it is surjective. Now by the isomorphism theorem, we have $ G \cong \F\br{S} / \Ker \phi $.
\end{proof}

\begin{remark}
\label{rem:1.4.9}
Let $ G $ be any group and $ S \subseteq G $ a generating set. Then if $ S $ is infinite then $ \abs{S} = \abs{G} $. Why? Note that $ \abs{\PPP^{\fin}\br{S}} = \abs{S} $ for all infinite sets $ S $, in ZF. As every element corresponds to a finite sequence in $ S $, we get
$$ \abs{S} \le \abs{G} \le \abs{\PPP^{\fin}\br{S}} = \abs{S}. $$
\end{remark}

\begin{definition}
Let $ G $ be an arbitrary group. The \textbf{rank} of $ G $ is the smallest cardinal $ \kappa $ such that $ G $ arises as the quotient of a free group of rank $ \kappa $. By \ref{rem:1.4.9} this is well-defined. We denote it by $ \rk G $.
\end{definition}

\begin{lemma}
Let $ S $ and $ S' $ be any sets. Then $ \F\br{S} \cong \F\br{S'} $ if and only if $ \abs{S} = \abs{S'} $.
\end{lemma}

\begin{proof}
\hfill
\begin{itemize}
\item[$ \implies $] If $ \abs{S} = \abs{S'} $, then there exists a bijection $ f : S \to S' $. Let $ \phi : \F\br{S} \to \F\br{S'} $ be the unique homomorphism extending $ f $, and $ \psi : \F\br{S'} \to \F\br{S} $ the unique homomorphism extending $ f^{-1} $. Then $ \psi \circ \phi : \F\br{S} \to \F\br{S} $ extends $ \id_S : S \to \F\br{S} $. But clearly $ \id_{\F\br{S}} : \F\br{S} \to \F\br{S} $ is another such homomorphism, whence by uniqueness $ \psi \circ \phi = \id_{\F\br{S}} $, whence $ \phi $ is an isomorphism.
\item[$ \impliedby $] Assume $ \F\br{S} \cong \F\br{S'} $. If both $ S $ and $ S' $ are infinite, then
$$ \abs{S} = \abs{\F\br{S}} = \abs{\F\br{S'}} = \abs{S'}, $$
by \ref{rem:1.4.9}. Otherwise assume $ S $ is finite. Set $ G = \ZZ / 2\ZZ $. Recall that any homomorphism from $ \F\br{S} $ to $ G $ is uniquely determined by its image on $ S $. Hence, there is a one to one correspondence between the set of homomorphisms $ \phi : \F\br{S} \to G $ and $ \PPP\br{S} $, via
$$ \phi \mapsto X_\phi = \cbr{s \in S \st \phi\br{s} = 1}. $$
Note, as $ \F\br{S} \cong \F\br{S'} $, there are as many homomorphisms from $ \F\br{S} \to G $ as from $ \F\br{S'} \to G $. Hence
$$ 2^{\abs{S}} = \abs{\Hom\br{\F\br{S}, G}} = \abs{\Hom\br{\F\br{S'}, G}} = 2^{\abs{S'}}, $$
and as $ S $ was finite, $ \abs{S} = \abs{S'} $.
\end{itemize}
\end{proof}

\begin{definition}
If $ G $ is free on $ S $, we call $ \abs{S} $ the \textbf{rank} of $ G $ and say that $ S $ is a basis of $ G $.
\end{definition}

\pagebreak

\subsection{Presentations of groups}

\begin{definition}
Let $ G $ be any group and $ S \subseteq G $ a subset for $ G $. Recall that $ f : S \hookrightarrow G $. Let $ R $ be a set of words over $ S $. We say that $ G $ \textbf{allows the presentation} $ \abr{S \st R} $ if and only if $ S $ generates $ G $ and $ R $ normally generates $ \Ker \phi $. For brevity we write
$$ G = \abr{S \st R}. $$
We call the elements in $ G $ \textbf{generators} of $ G $ and the elements in $ R $ \textbf{relators}.
\end{definition}

\lecture{9}{Wednesday}{23/10/19}

\begin{notation*}
$ \abr{R}^{\F\br{S}} $ denotes the subgroup of $ \F\br{S} $ normally generated by $ R $, so
$$ G \cong \F\br{S} / \abr{R}^{\F\br{S}}. $$
Generally if $ G $ is a group and $ X \subseteq G $ is a subset, then $ \abr{X}^G $ denotes the subgroup normally generated by $ X $.
\end{notation*}

\begin{note*}
Elements of $ R $ are called relators and identities in $ G $ are called \textbf{relations}. Every relator $ r $ codes a relation $ r \underset{G}{=} e_G $.
\end{note*}

\begin{example*}
In $ S = \cbr{a, b} $, $ ab^2a^{-1} $ is a relator and $ ab^2 \underset{G}{=} $ is a relation, and $ ab^2a^{-1} \underset{G}{=} e_G $ if and only if $ ab^2 \underset{G}{=} a $.
\end{example*}

\begin{remark}[Free groups V]
A group $ F $ is \textbf{free on $ S $} if and only if it allows the presentation $ \abr{S \st} $.
\end{remark}

\begin{exercise}
\label{ex:1.5.3}
\hfill
\begin{enumerate}
\item Show that for $ r \in \abr{R}^{\F\br{S}} $ and $ g, h \in \F\br{S} $ is arbitrary, then $ grh \in \abr{R}^{\F\br{S}} $ if and only if $ gh \in \abr{R}^{\F\br{S}} $.
\item Show that $ g \in \F\br{S} $ is in $ \abr{R}^{\F\br{S}} $ if and only if it is of the form
$$ g = \prod_{i = 1}^n u_ir_iu_i^{-1}, \qquad r_i \in R, \qquad u_i \in \F\br{S}. $$
\end{enumerate}
\end{exercise}

\begin{definition}
Let $ G = \abr{S \st R} $. Then we call $ G $ \textbf{finitely presented} if both $ S $ and $ R $ are finite.
\end{definition}

\begin{example}
Claim that the group $ \SSS_3 $ allows the presentation
$$ \abr{x, y \st x^2, y^2, \br{xy}^3}. $$
Let
$$ \function[\phi]{F\br{\cbr{x, y}}}{\SSS_3}{\br{x, y}}{\br{\br{12}, \br{23}}}. $$
Clearly
$$ \phi\br{x^2} = \br{12}^2 = e_{\SSS_3} = \br{23}^2 = \phi\br{y^2}. $$
Also, $ xy = \br{132} $ and hence
$$ \phi\br{\br{xy}^3} = \br{132}^3 = e_{\SSS_3}, $$
hence $ \br{xy}^3 \in \Ker \phi $. Thus, by \ref{ex:1.5.3}.$ 2 $ and as $ \phi $ is a homomorphism, we get
$$ \abr{\cbr{x^2, y^2, \br{xy}^3}}^{\F\br{S}} \subseteq \Ker \phi. $$
Now assume $ w = x^{k_1}y^{l_1} \dots x^{k_n}y^{l_n} $ on $ \cbr{x, y} $ such that $ w \underset{\SSS_3}{=} e_{\SSS_3} $, that is $ w \in \Ker \phi $. By \ref{ex:1.5.3}.$ 1 $, we can cancel all occurences of $ x^{\pm 2}y^{\pm 2} $. We obtain $ w' = xyxy \dots $ such that $ w' \underset{\SSS_3}{=} e_{\SSS_3} $ if and only if $ w' \underset{\SSS_3}{=} e_{\SSS_3} $. Now cancel $ \br{xy}^3 $ and obtain a word of length at most six. Now, note the relator $ \br{xy}^3 $ yields the relation $ xyx \underset{\SSS_3}{=} yxy $. Hence the only words left are of length at most three, that is
$$ \epsilon, x, y, xy, yx, xyx = yxy. $$
Hence, $ w $ is in $ \abr{R}^{\F\br{S}} $.
\end{example}

\begin{exercise}
For any set $ S $ and $ R $ a set of words on $ S $, there exists some $ G $ with $ G = \abr{R \st S} $.
\end{exercise}

\pagebreak

\begin{remark}
\label{rem:1.5.7}
Let $ G $ be a group and $ S \subseteq G $, and let $ R $ be a set of words on $ S $. Then the following are equivalent.
\begin{enumerate}
\item $ G = \abr{S \st R} $.
\item A word $ w $ on $ S $ represents $ e_G $ in $ G $ if and only if $ w \in \abr{R}^{\F\br{S}} $.
\end{enumerate}
\end{remark}

\begin{proposition}
Let $ H $ be a group and $ S \subseteq H $, and $ R \subseteq \F\br{S} $. Then the following are equivalent.
\begin{itemize}
\item $ H = \abr{S \st R} $.
\item For any group $ G $ and $ f : S \to G $, there exists a homomorphism $ \phi : H \to G $ extending $ f $ if and only if $ f\br{r} = e_G $, with if $ r = s_1^{\epsilon_1} \dots s_n^{\epsilon_n} $ then $ f\br{r} = f\br{s_1}^{\epsilon_1} \dots f\br{s_n}^{\epsilon_n} $. In that case, $ \phi $ is unique.
\end{itemize}
\end{proposition}

\begin{proof}
\hfill
\begin{itemize}
\item[$ \implies $] Assume $ H = \abr{S \st R} $ and $ f : S \to G $. Clearly, as $ S $ generates $ H $, any homomorphism $ \phi : H \to G $ extending $ f $ has to send $ h = s_1^{\epsilon_1} \dots s_n^{\epsilon_n} $ to $ f\br{s_1}^{\epsilon_1} \dots f\br{s_n}^{\epsilon_n} $. Hence $ \phi $ is uniquely determined, if it exists. Now, assume that $ \phi $ is a well-defined homomorphism of groups. By Remark \ref{rem:1.5.7}, that $ r \underset{H}{=} e_H $ for any $ r \in R $, whence $ \phi\br{r} = e_G = f\br{r} $. Now assume $ f\br{r} = e_G $ for all $ r \in R $. We want to show that $ \phi $ is well-defined, that is if $ w_1 \underset{H}{=} w_2 $, then $ \phi\br{w_1} \underset{G}{=} \phi\br{w_2} $. Clearly, this amounts to say that any word $ w_1w_2^{-1} $ which represents $ e_H $ is sent to $ e_G $. By \ref{rem:1.5.7}.$ 2 $, $ w $ represents $ e_H $ if and only if $ w \in \abr{R}^{\F\br{S}} $. By assumption $ \phi\br{R} = \cbr{e_G} $ and \ref{ex:1.5.3}.$ 2 $, the claim follows.
\item[$ \impliedby $] Exercise. \footnote{Exercise}
\end{itemize}
\end{proof}

\lecture{10}{Thursday}{24/10/19}

\begin{remark}
We defined $ G = \abr{S \st R} $ through some $ \phi : \F\br{S} \to G $. Note that $ \phi $ and $ R $ do not determine each other uniquely.
\end{remark}

\begin{example*}
\hfill
\begin{itemize}
\item Consider
$$ G = \abr{x \st x^3} = \ZZ_3, $$
and
$$ \function[\phi_1]{\F\br{\cbr{x}}}{G}{x}{1}, \qquad \function[\phi_1]{\F\br{\cbr{x}}}{G}{x}{2}. $$
\item Consider
$$ G = \abr{x, y \st x^2, y^2, \br{xy}^3} = \SSS_3, $$
as we saw, given by
$$ \function[\phi_1]{\F\br{\cbr{x, y}}}{G}{\br{x, y}}{\br{\br{12}, \br{23}}}. $$
An easy calculation shows that
$$ \SSS_3 = \abr{x, y \st x^2, \br{xy}^3, x^{-1}y^{-1}\br{xy}^2}, $$
via the same homomorphism $ \phi $.
\end{itemize}
\end{example*}

It is hard to read even basic algebraic properties from a presentation, such as whether or not a group is finite.

\begin{example*}
$$ \SSS_3 = \abr{x, y \st x^2, y^2, \br{xy}^3}, \ \AAA_4 = \abr{x, y \st x^3, y^3, \br{xy}^2} \ \text{are finite}, \qquad G = \abr{x, y \st x^3, y^3, \br{xy}^3} \ \text{is infinite}. $$
\end{example*}

\pagebreak

\begin{remark}
More generally, a group in which every element has finite order is called \textbf{periodic}. Burnside asked, in 1902, whether every finitely generated periodic group is finite. This generalises to the following. Let $ S $ be any set and let $ w\br{S} $ denote the set of all words on $ S $. Define the groups
$$ \B\br{m, n} = \abr{x_1, \dots, x_m \st w\br{\cbr{x_1, \dots, x_m}}^n}. $$
They are called \textbf{free Burnside groups}. Are the $ \B\br{m, n} $ finite?
\begin{itemize}
\item Burnside proved for $ n = 2, 3 $.
\item Sanov proved in 1940 for $ n = 4 $.
\item Hall proved in 1958 for $ n = 6 $.
\item Novikov and Adian disproved in 1968 for $ m \ge 2 $ and $ n \ge 667 $ odd.
\item $ \B\br{2, 5} = \abr{x, y \st w\br{x, y}^5} $ is open.
\end{itemize}
\end{remark}

\begin{exercise}
Show that all $ \B\br{1, n} $ and $ \B\br{m, n} $ are finite.
\end{exercise}

Why bother? Most groups simply just come through their presentation and problems on groups, such as isomorphism problems and word problems, are best expressed through presentations. A very important class of groups are Coxeter groups. A group $ G $ is \textbf{Coxeter}, if
$$ G = \abr{s_1, \dots, s_n \st s_i^2, \br{s_is_j}^{m_{ij}}, i, j = 1, \dots, n}, \qquad 1 \le m_{ij} \le \infty. $$

\subsection{Group actions on graphs}

Recall the definition of a group acting on a graph. Many statements in Bass-Serre theory start with the assumption that $ G $ acts on $ X $ without inversion of edges.

\begin{definition}
Let $ X $ be a graph. We define its \textbf{barycentric subdivision} $ \BBB\br{X} $ via
\begin{itemize}
\item $ \BBB\br{X}^0 = \cbr{v_e \st e \in X^1} $,
\item $ \BBB\br{X}^1 = \cbr{e_1, e_2 \st e \in X^0} $ such that
$$ \alpha\br{e_1} = \alpha\br{e}, \qquad \omega\br{e_1} = \alpha\br{e_2} = v_e, \qquad \omega\br{e_2} = \omega\br{e}, $$
\item $ \overline{e_1} = \br{\overline{e}}_2 $.
\end{itemize}
\end{definition}

Now, if $ G $ acts on $ X $, we can naturally extend the action via
$$ g \cdot v_e = v_{ge}, \qquad g \cdot e_1 = \br{ge}_1, \qquad g \cdot e_2 = \br{ge}_2. $$

\begin{note*}
The action did not change on $ X^0 $.
\end{note*}

\begin{exercise}
If $ G $ acts on $ X $, then it acts as above on $ \BBB\br{X} $ and this action is without inversion of edges.
\end{exercise}

\begin{definition}
Let $ X $ be a tree. A reduced path is called a \textbf{geodesic}. For $ x, y \in X $, we denote the unique geodesic from $ x $ to $ y $ by $ \sbr{x - y} $. Its length is called the \textbf{distance} between $ x $ and $ y $ and denoted by $ \d\br{x, y} $.
\end{definition}

\begin{note*}
$ \d $ defines a distance function and if $ \sigma \in \Aut X $, then it acts on $ X $ as an isometry.
\end{note*}

\begin{note*}
Further, for $ X_1, X_2 \subseteq X $ disjoint subtrees of $ X $, there is a unique geodesic which starts in $ X_1 $, ends in $ X_2 $, and with all edges outside $ X_1^1 \cup X_2^1 $. We denote this by $ X_1 - X_2 $, and its length by $ \d\br{X_1, X_2} $.
\end{note*}

\pagebreak

\begin{definition}
Let $ X $ be a tree and $ \sigma \in \Aut X $. Then we set
$$ \abs{\sigma} = \min\cbr{\d\br{x, \sigma\br{x}} \st x \in X^0}, $$
and call it the \textbf{displacement length} of $ \sigma $. Let
$$ \widetilde{\sigma_{\min}} = \cbr{x \in X^0 \st \d\br{x, \sigma\br{x}} = \abs{\sigma}}. $$
Then $ \sigma_{\min} $ is the induced subgraph on $ \widetilde{\sigma_{\min}} $. If $ \abs{\sigma} = 0 $, we write $ \sigma_{\min} = \mathring{\sigma} $, and if $ \abs{\sigma} > 0 $, we write $ \sigma_{\min} = \vec{\sigma} $.
\end{definition}

\lecture{11}{Tuesday}{29/10/19}

\begin{definition}
If $ X $ is a tree and $ G $ acting without inversion of edges as a group of automorphisms, we say that $ X $ is a \textbf{$ G $-tree}.
\end{definition}

\begin{proposition}
Let $ X $ be a $ G $-tree, and $ \sigma \in G $.
\begin{enumerate}
\item If $ \abs{\sigma} = 0 $, then $ \mathring{\sigma} $ is a tree. Furthermore, if $ y \in X^0 $, then let $ x $ be the final vertex of $ \sbr{y - \mathring{\sigma}} $. Then
$$ \d\br{y, \sigma\br{y}} = 2\d\br{y, x}, \qquad \sbr{y - \sigma\br{y}} = \sbr{y - x} \cup \sbr{x - \sigma\br{y}}. $$
\item Assume $ \abs{\sigma} > 0 $. Then $ \vec{\sigma} \cong \CCC_\infty $ and $ \sigma $ acts on $ \vec{\sigma} $ by translation of length $ \abs{\sigma} $. If $ y \in X^0 $ is arbitrary and $ x $ is the terminal vertex of $ \sbr{y - \vec{\sigma}} $, then
$$ \d\br{y, \sigma\br{y}} = 2\d\br{y, x} + \abs{\sigma}, \qquad \sbr{y - \sigma\br{y}} = \sbr{y - x} \cup \sbr{x - \sigma\br{x}} \cup \sbr{\sigma\br{x} - \sigma\br{y}}. $$
\end{enumerate}
\end{proposition}

\begin{proof}
\hfill
\begin{enumerate}
\item Let $ \abs{\sigma} = 0 $. Clearly, $ \mathring{\sigma} $ is a tree as for $ x $ and $ y $ fixed by $ \sigma $, clearly $ \sbr{x - y} $ has to be fixed by $ \sigma $, whence $ \sbr{x - y} \in \mathring{\sigma} $. Consider $ y \in X^0 $ arbitrary and $ x $ as described. We prove the statement by induction on $ \d\br{y, \mathring{\sigma}} $. Clearly, if $ \d\br{y, \mathring{\sigma}} = 0 $, then $ y \in \mathring{\sigma} $, and the statement holds, as all geodesics are the empty path. Now $ \d\br{y, \mathring{\sigma}} = n + 1 $. Let $ z $ be the terminal vertex of the first edge in $ \sbr{y - x} $. Then $ \d\br{z, \mathring{\sigma}} = \d\br{z, x} = n $, whence $ \sbr{z - \sigma\br{z}} = \sbr{z - x} \cup \sbr{x - \sigma\br{z}} $. Now $ \sigma\br{z} $ is a neighbour of $ \sigma\br{y} $. Then either $ \sigma\br{y} \notin \sbr{x - \sigma\br{z}} $, then we are done, as $ \d\br{y, \sigma\br{y}} = \d\br{y, z} + \d\br{z, \sigma\br{z}} + \d\br{\sigma\br{z}, \sigma\br{y}} = 2\br{n + 1} = 2\d\br{y, x} $. Otherwise, $ \d\br{\sigma\br{y}, x} = n - 1 $. Clearly, $ \abs{\sigma^{-1}} = 0 $ and $ \mathring{\sigma^{-1}} = \mathring{\sigma} $. Then $ \d\br{\sigma\br{y}, x} = n - 1 < n + 1 = \d\br{x, y} = \d\br{\sigma^{-1}\br{\sigma\br{y}}, x} $. Hence $ \d\br{\sigma\br{y}, \sigma^{-1}\br{\sigma\br{y}}} = 2n > 2\d\br{\sigma\br{y}, x} $, a contradiction.
\item Consider $ x \in \vec{\sigma} $. We want to show that $ \sbr{x - \sigma^2\br{x}} = \sbr{x - \sigma\br{x}} \cup \sbr{\sigma\br{x} - \sigma^2\br{x}} $. Assume for a contradiction that the terminal edge of $ \sbr{x - \sigma\br{x}} $, call it $ e $, is the first edge inverse of $ \sbr{\sigma\br{x} - \sigma^2\br{x}} $. Let $ z $ be the terminal vertex of the first edge in $ \sbr{x - \sigma\br{x}} $. Then clearly, $ \sigma\br{z} $ is the terminal vertex of the first edge in $ \sbr{\sigma\br{x} - \sigma^2\br{x}} $. By assumption, this is the initial vertex of $ e $. But then $ \d\br{z, \sigma\br{z}} = \d\br{x, \sigma\br{x}} - 2 < \abs{\sigma} $, a contradiction. Note that if $ \d\br{x, \sigma\br{x}} = 1 $, then $ \sigma $ would invert the edge $ e $, a contradiction. Hence $ \sbr{x - \sigma^2\br{x}} = \sbr{x - \sigma\br{x}} \cup \sbr{\sigma\br{x} - \sigma^2\br{x}} $. Inductively, this produces a tree
$$ T = \dots - \sigma^{-1}\br{x} - x - \sigma\br{x} - \dots \subseteq \vec{\sigma}, $$
and $ \sigma $ acts on $ T $ by translation with length $ \abs{\sigma} $. Now consider $ y \in X^0 \setminus T $. Note that $ \sigma \in \Aut T $. Thus $ \sbr{\sigma\br{x} - \sigma\br{y}} $ intersects $ T $ trivially. Hence, $ \d\br{y, \sigma\br{y}} = \d\br{y, x} + \d\br{x, \sigma\br{x}} + \d\br{\sigma\br{x}, \sigma\br{y}} = \d\br{y, x} + \abs{\sigma} + \d\br{x, y} = 2\d\br{x, y} + \abs{\sigma} > \abs{\sigma} $, for all $ y \notin T $. Hence $ T = \vec{\sigma} \cong \CCC_\infty $.
\end{enumerate}
\end{proof}

\begin{definition}
Let $ X $ be a $ G $-tree and $ \sigma \in G $. Then
\begin{itemize}
\item if $ \abs{\sigma} = 0 $, then $ \sigma $ is called an \textbf{elliptic element}, and
\item if $ \abs{\sigma} > 0 $, then $ \sigma $ is called a \textbf{hyperbolic element}.
\end{itemize}
\end{definition}

\begin{remark}
If $ G $ does invert an edge, it resembles more a rotation. Indeed, the extended action on $ \BBB\br{X} $ yields that $ \sigma $ is elliptic.
\end{remark}

\lecture{12}{Wednesday}{30/10/19}

Lecture 12 is a problem class.

\end{document}