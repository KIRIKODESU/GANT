\def\module{M4P54 Differential Topology}
\def\lecturer{Prof Paolo Cascini}
\def\term{Spring 2020}
\def\cover{}
\def\syllabus{}
\def\thm{section}

\documentclass{article}

% Packages

\usepackage{amssymb}
\usepackage{amsthm}
\usepackage[UKenglish]{babel}
\usepackage{commath}
\usepackage{enumitem}
\usepackage{etoolbox}
\usepackage{fancyhdr}
\usepackage[margin=1in]{geometry}
\usepackage{graphicx}
\usepackage[hidelinks]{hyperref}
\usepackage[utf8]{inputenc}
\usepackage{listings}
\usepackage{mathtools}
\usepackage{stmaryrd}
\usepackage{tikz-cd}
\usepackage{csquotes}

% Formatting

\addto\captionsUKenglish{\renewcommand{\abstractname}{Syllabus}}
\delimitershortfall5pt
\ifx\thm\undefined\newtheorem{n}{}\else\newtheorem{n}{}[\thm]\fi
\newcommand\newoperator[1]{\ifcsdef{#1}{\cslet{#1}{\relax}}{}\csdef{#1}{\operatorname{#1}}}
\setlength{\parindent}{0cm}

% Environments

\theoremstyle{plain}
\newtheorem{algorithm}[n]{Algorithm}
\newtheorem*{algorithm*}{Algorithm}
\newtheorem{algorithm**}{Algorithm}
\newtheorem{conjecture}[n]{Conjecture}
\newtheorem*{conjecture*}{Conjecture}
\newtheorem{conjecture**}{Conjecture}
\newtheorem{corollary}[n]{Corollary}
\newtheorem*{corollary*}{Corollary}
\newtheorem{corollary**}{Corollary}
\newtheorem{lemma}[n]{Lemma}
\newtheorem*{lemma*}{Lemma}
\newtheorem{lemma**}{Lemma}
\newtheorem{proposition}[n]{Proposition}
\newtheorem*{proposition*}{Proposition}
\newtheorem{proposition**}{Proposition}
\newtheorem{theorem}[n]{Theorem}
\newtheorem*{theorem*}{Theorem}
\newtheorem{theorem**}{Theorem}

\theoremstyle{definition}
\newtheorem{aim}[n]{Aim}
\newtheorem*{aim*}{Aim}
\newtheorem{aim**}{Aim}
\newtheorem{axiom}[n]{Axiom}
\newtheorem*{axiom*}{Axiom}
\newtheorem{axiom**}{Axiom}
\newtheorem{condition}[n]{Condition}
\newtheorem*{condition*}{Condition}
\newtheorem{condition**}{Condition}
\newtheorem{definition}[n]{Definition}
\newtheorem*{definition*}{Definition}
\newtheorem{definition**}{Definition}
\newtheorem{example}[n]{Example}
\newtheorem*{example*}{Example}
\newtheorem{example**}{Example}
\newtheorem{exercise}[n]{Exercise}
\newtheorem*{exercise*}{Exercise}
\newtheorem{exercise**}{Exercise}
\newtheorem{fact}[n]{Fact}
\newtheorem*{fact*}{Fact}
\newtheorem{fact**}{Fact}
\newtheorem{goal}[n]{Goal}
\newtheorem*{goal*}{Goal}
\newtheorem{goal**}{Goal}
\newtheorem{law}[n]{Law}
\newtheorem*{law*}{Law}
\newtheorem{law**}{Law}
\newtheorem{plan}[n]{Plan}
\newtheorem*{plan*}{Plan}
\newtheorem{plan**}{Plan}
\newtheorem{problem}[n]{Problem}
\newtheorem*{problem*}{Problem}
\newtheorem{problem**}{Problem}
\newtheorem{question}[n]{Question}
\newtheorem*{question*}{Question}
\newtheorem{question**}{Question}
\newtheorem{warning}[n]{Warning}
\newtheorem*{warning*}{Warning}
\newtheorem{warning**}{Warning}
\newtheorem{acknowledgements}[n]{Acknowledgements}
\newtheorem*{acknowledgements*}{Acknowledgements}
\newtheorem{acknowledgements**}{Acknowledgements}
\newtheorem{annotations}[n]{Annotations}
\newtheorem*{annotations*}{Annotations}
\newtheorem{annotations**}{Annotations}
\newtheorem{assumption}[n]{Assumption}
\newtheorem*{assumption*}{Assumption}
\newtheorem{assumption**}{Assumption}
\newtheorem{conclusion}[n]{Conclusion}
\newtheorem*{conclusion*}{Conclusion}
\newtheorem{conclusion**}{Conclusion}
\newtheorem{claim}[n]{Claim}
\newtheorem*{claim*}{Claim}
\newtheorem{claim**}{Claim}
\newtheorem{notation}[n]{Notation}
\newtheorem*{notation*}{Notation}
\newtheorem{notation**}{Notation}
\newtheorem{note}[n]{Note}
\newtheorem*{note*}{Note}
\newtheorem{note**}{Note}
\newtheorem{remark}[n]{Remark}
\newtheorem*{remark*}{Remark}
\newtheorem{remark**}{Remark}

% Lectures

\newcommand{\lecture}[3]{ % Lecture
  \marginpar{
    Lecture #1 \\
    #2 \\
    #3
  }
}

% Blackboard

\renewcommand{\AA}{\mathbb{A}} % Blackboard A
\newcommand{\BB}{\mathbb{B}}   % Blackboard B
\newcommand{\CC}{\mathbb{C}}   % Blackboard C
\newcommand{\DD}{\mathbb{D}}   % Blackboard D
\newcommand{\EE}{\mathbb{E}}   % Blackboard E
\newcommand{\FF}{\mathbb{F}}   % Blackboard F
\newcommand{\GG}{\mathbb{G}}   % Blackboard G
\newcommand{\HH}{\mathbb{H}}   % Blackboard H
\newcommand{\II}{\mathbb{I}}   % Blackboard I
\newcommand{\JJ}{\mathbb{J}}   % Blackboard J
\newcommand{\KK}{\mathbb{K}}   % Blackboard K
\newcommand{\LL}{\mathbb{L}}   % Blackboard L
\newcommand{\MM}{\mathbb{M}}   % Blackboard M
\newcommand{\NN}{\mathbb{N}}   % Blackboard N
\newcommand{\OO}{\mathbb{O}}   % Blackboard O
\newcommand{\PP}{\mathbb{P}}   % Blackboard P
\newcommand{\QQ}{\mathbb{Q}}   % Blackboard Q
\newcommand{\RR}{\mathbb{R}}   % Blackboard R
\renewcommand{\SS}{\mathbb{S}} % Blackboard S
\newcommand{\TT}{\mathbb{T}}   % Blackboard T
\newcommand{\UU}{\mathbb{U}}   % Blackboard U
\newcommand{\VV}{\mathbb{V}}   % Blackboard V
\newcommand{\WW}{\mathbb{W}}   % Blackboard W
\newcommand{\XX}{\mathbb{X}}   % Blackboard X
\newcommand{\YY}{\mathbb{Y}}   % Blackboard Y
\newcommand{\ZZ}{\mathbb{Z}}   % Blackboard Z

% Brackets

\renewcommand{\eval}[1]{\left. #1 \right|}          % Evaluation
\newcommand{\br}{\del}                              % Brackets
\newcommand{\abr}[1]{\left\langle #1 \right\rangle} % Angle brackets
\newcommand{\fbr}[1]{\left\lfloor #1 \right\rfloor} % Floor brackets
\newcommand{\lbr}[1]{\left\lfloor #1 \right\rfloor} % Ceiling brackets
\newcommand{\st}{\ \middle| \ }                     % Such that

% Calligraphic

\newcommand{\AAA}{\mathcal{A}} % Calligraphic A
\newcommand{\BBB}{\mathcal{B}} % Calligraphic B
\newcommand{\CCC}{\mathcal{C}} % Calligraphic C
\newcommand{\DDD}{\mathcal{D}} % Calligraphic D
\newcommand{\EEE}{\mathcal{E}} % Calligraphic E
\newcommand{\FFF}{\mathcal{F}} % Calligraphic F
\newcommand{\GGG}{\mathcal{G}} % Calligraphic G
\newcommand{\HHH}{\mathcal{H}} % Calligraphic H
\newcommand{\III}{\mathcal{I}} % Calligraphic I
\newcommand{\JJJ}{\mathcal{J}} % Calligraphic J
\newcommand{\KKK}{\mathcal{K}} % Calligraphic K
\newcommand{\LLL}{\mathcal{L}} % Calligraphic L
\newcommand{\MMM}{\mathcal{M}} % Calligraphic M
\newcommand{\NNN}{\mathcal{N}} % Calligraphic N
\newcommand{\OOO}{\mathcal{O}} % Calligraphic O
\newcommand{\PPP}{\mathcal{P}} % Calligraphic P
\newcommand{\QQQ}{\mathcal{Q}} % Calligraphic Q
\newcommand{\RRR}{\mathcal{R}} % Calligraphic R
\newcommand{\SSS}{\mathcal{S}} % Calligraphic S
\newcommand{\TTT}{\mathcal{T}} % Calligraphic T
\newcommand{\UUU}{\mathcal{U}} % Calligraphic U
\newcommand{\VVV}{\mathcal{V}} % Calligraphic V
\newcommand{\WWW}{\mathcal{W}} % Calligraphic W
\newcommand{\XXX}{\mathcal{X}} % Calligraphic X
\newcommand{\YYY}{\mathcal{Y}} % Calligraphic Y
\newcommand{\ZZZ}{\mathcal{Z}} % Calligraphic Z

% Fraktur

\newcommand{\aaa}{\mathfrak{a}}   % Fraktur a
\newcommand{\bbb}{\mathfrak{b}}   % Fraktur b
\newcommand{\ccc}{\mathfrak{c}}   % Fraktur c
\newcommand{\ddd}{\mathfrak{d}}   % Fraktur d
\newcommand{\eee}{\mathfrak{e}}   % Fraktur e
\newcommand{\fff}{\mathfrak{f}}   % Fraktur f
\renewcommand{\ggg}{\mathfrak{g}} % Fraktur g
\newcommand{\hhh}{\mathfrak{h}}   % Fraktur h
\newcommand{\iii}{\mathfrak{i}}   % Fraktur i
\newcommand{\jjj}{\mathfrak{j}}   % Fraktur j
\newcommand{\kkk}{\mathfrak{k}}   % Fraktur k
\renewcommand{\lll}{\mathfrak{l}} % Fraktur l
\newcommand{\mmm}{\mathfrak{m}}   % Fraktur m
\newcommand{\nnn}{\mathfrak{n}}   % Fraktur n
\newcommand{\ooo}{\mathfrak{o}}   % Fraktur o
\newcommand{\ppp}{\mathfrak{p}}   % Fraktur p
\newcommand{\qqq}{\mathfrak{q}}   % Fraktur q
\newcommand{\rrr}{\mathfrak{r}}   % Fraktur r
\newcommand{\sss}{\mathfrak{s}}   % Fraktur s
\newcommand{\ttt}{\mathfrak{t}}   % Fraktur t
\newcommand{\uuu}{\mathfrak{u}}   % Fraktur u
\newcommand{\vvv}{\mathfrak{v}}   % Fraktur v
\newcommand{\www}{\mathfrak{w}}   % Fraktur w
\newcommand{\xxx}{\mathfrak{x}}   % Fraktur x
\newcommand{\yyy}{\mathfrak{y}}   % Fraktur y
\newcommand{\zzz}{\mathfrak{z}}   % Fraktur z

% Geometry

\newcommand{\CP}{\mathbb{CP}}                                              % Complex projective space
\newcommand{\iintd}[4]{\iint_{#1} \, #2 \, \dif #3 \, \dif #4}             % Double integral
\newcommand{\RP}{\mathbb{RP}}                                              % Real projective space
\newcommand{\intd}[4]{\int_{#1}^{#2} \, #3 \, \dif #4}                     % Single integral
\newcommand{\iiintd}[5]{\iint_{#1} \, #2 \, \dif #3 \, \dif #4 \, \dif #5} % Triple integral

% Logic

\newcommand{\iffb}[2]{\br{#1 \leftrightarrow #2}} % Biconditional
\newcommand{\andb}[2]{\br{#1 \land #2}}           % Conjunction
\newcommand{\orb}[2]{\br{#1 \lor #2}}             % Disjunction
\newcommand{\nib}[2]{\br{#1 \notin #2}}           % Element of
\newcommand{\eqb}[2]{\br{#1 = #2}}                % Equal to
\newcommand{\teb}[1]{\br{\exists #1}}             % Existential quantifier
\newcommand{\impb}[2]{\br{#1 \rightarrow #2}}     % Implication
\newcommand{\ltb}[2]{\br{#1 < #2}}                % Less than
\newcommand{\leb}[2]{\br{#1 \le #2}}              % Less than or equal to
\newcommand{\notb}[1]{\br{\neg #1}}               % Negation
\newcommand{\inb}[2]{\br{#1 \in #2}}              % Not element of
\newcommand{\neb}[2]{\br{#1 \ne #2}}              % Not equal to
\newcommand{\subb}[2]{\br{#1 \subseteq #2}}       % Subset
\newcommand{\fab}[1]{\br{\forall #1}}             % Universal quantifier

% Maps

\newcommand{\bijection}[7][]{    % Bijection
  \ifx &#1&
    \begin{array}{rcl}
      #2 & \longleftrightarrow & #3 \\
      #4 & \longmapsto         & #5 \\
      #6 & \longmapsfrom       & #7
    \end{array}
  \else
    \begin{array}{ccrcl}
      #1 & : & #2 & \longrightarrow & #3 \\
         &   & #4 & \longmapsto     & #5 \\
         &   & #6 & \longmapsfrom   & #7
    \end{array}
  \fi
}
\newcommand{\birational}[7][]{   % Birational map
  \ifx &#1&
    \begin{array}{rcl}
      #2 & \dashrightarrow & #3 \\
      #4 & \longmapsto     & #5 \\
      #6 & \longmapsfrom   & #7
    \end{array}
  \else
    \begin{array}{ccrcl}
      #1 & : & #2 & \dashrightarrow & #3 \\
         &   & #4 & \longmapsto     & #5 \\
         &   & #6 & \longmapsfrom   & #7
    \end{array}
  \fi
}
\newcommand{\correspondence}[2]{ % Correspondence
  \cbr{
    \begin{array}{c}
      #1
    \end{array}
  }
  \qquad
  \leftrightsquigarrow
  \qquad
  \cbr{
    \begin{array}{c}
      #2
    \end{array}
  }
}
\newcommand{\function}[5][]{     % Function
  \ifx &#1&
    \begin{array}{rcl}
      #2 & \longrightarrow & #3 \\
      #4 & \longmapsto     & #5
    \end{array}
  \else
    \begin{array}{ccrcl}
      #1 & : & #2 & \longrightarrow & #3 \\
         &   & #4 & \longmapsto     & #5
    \end{array}
  \fi
}
\newcommand{\functions}[7][]{    % Functions
  \ifx &#1&
    \begin{array}{rcl}
      #2 & \longrightarrow & #3 \\
      #4 & \longmapsto     & #5 \\
      #6 & \longmapsto     & #7
    \end{array}
  \else
    \begin{array}{ccrcl}
      #1 & : & #2 & \longrightarrow & #3 \\
         &   & #4 & \longmapsto     & #5 \\
         &   & #6 & \longmapsto     & #7
    \end{array}
  \fi
}
\newcommand{\rational}[5][]{     % Rational map
  \ifx &#1&
    \begin{array}{rcl}
      #2 & \dashrightarrow & #3 \\
      #4 & \longmapsto     & #5
    \end{array}
  \else
    \begin{array}{ccrcl}
      #1 & : & #2 & \dashrightarrow & #3 \\
         &   & #4 & \longmapsto     & #5
    \end{array}
  \fi
}

% Matrices

\newcommand{\onebytwo}[2]{      % One by two matrix
  \begin{pmatrix}
    #1 & #2
  \end{pmatrix}
}
\newcommand{\onebythree}[3]{    % One by three matrix
  \begin{pmatrix}
    #1 & #2 & #3
  \end{pmatrix}
}
\newcommand{\twobyone}[2]{      % Two by one matrix
  \begin{pmatrix}
    #1 \\
    #2
  \end{pmatrix}
}
\newcommand{\twobytwo}[4]{      % Two by two matrix
  \begin{pmatrix}
    #1 & #2 \\
    #3 & #4
  \end{pmatrix}
}
\newcommand{\threebyone}[3]{    % Three by one matrix
  \begin{pmatrix}
    #1 \\
    #2 \\
    #3
  \end{pmatrix}
}
\newcommand{\threebythree}[9]{  % Three by three matrix
  \begin{pmatrix}
    #1 & #2 & #3 \\
    #4 & #5 & #6 \\
    #7 & #8 & #9
  \end{pmatrix}
}
\newcommand{\twobytwosmall}[4]{ % Two by two small matrix
  \begin{psmallmatrix}
    #1 & #2 \\
    #3 & #4
  \end{psmallmatrix}
}

% Number theory

\renewcommand{\symbol}[2]{\br{\tfrac{#1}{#2}}} % Power residue symbol
\newcommand{\unit}[1]{\br{\ZZ / #1\ZZ}^\times} % Unit group

% Operators

\newoperator{ab}    % Abelian
\newoperator{AG}    % Affine geometry
\newoperator{alg}   % Algebraic
\newoperator{Ann}   % Annihilator
\newoperator{area}  % Area
\newoperator{Aut}   % Automorphism
\newoperator{card}  % Cardinality
\newoperator{ch}    % Characteristic
\newoperator{Cl}    % Class
\newoperator{Coker} % Cokernel
\newoperator{col}   % Column
\newoperator{Corr}  % Correspondence
\newoperator{diam}  % Diameter
\newoperator{Disc}  % Discriminant
\newoperator{dom}   % Domain
\newoperator{Eig}   % Eigenvalue
\newoperator{Em}    % Embedding
\newoperator{End}   % Endomorphism
\newoperator{fin}   % Finite
\newoperator{Fix}   % Fixed
\newoperator{Frac}  % Fraction
\newoperator{Frob}  % Frobenius
\newoperator{Fun}   % Function
\newoperator{Gal}   % Galois
\newoperator{GL}    % General linear
\newoperator{Ham}   % Hamming
\newoperator{Homeo} % Homeomorphism
\newoperator{Hom}   % Homomorphism
\newoperator{id}    % Identity
\newoperator{Im}    % Image
\newoperator{Ind}   % Index
\newoperator{Ker}   % Kernel
\newoperator{lcm}   % Least common multiple
\newoperator{Mat}   % Matrix
\newoperator{mult}  % Multiplicity
\newoperator{new}   % New
\newoperator{Nm}    % Norm
\newoperator{old}   % Old
\newoperator{op}    % Opposite
\newoperator{ord}   % Order
\newoperator{Pay}   % Payley
\newoperator{PG}    % Projective geometry
\newoperator{PGL}   % Projective general linear
\newoperator{PSL}   % Projective special linear
\newoperator{rad}   % Radical
\newoperator{ran}   % Range
\newoperator{Res}   % Residue
\newoperator{rk}    % Rank
\newoperator{Re}    % Real
\newoperator{row}   % Row
\newoperator{sgn}   % Sign
\newoperator{Sing}  % Singular
\newoperator{SK}    % Skeleton
\newoperator{sp}    % Span
\newoperator{SL}    % Special linear
\newoperator{SO}    % Special orthogonal
\newoperator{Spec}  % Spectrum
\newoperator{Stab}  % Stabiliser
\newoperator{star}  % Star
\newoperator{srg}   % Strongly regular graph
\newoperator{supp}  % Support
\newoperator{Sym}   % Symmetric
\newoperator{tors}  % Torsion
\newoperator{Tr}    % Trace
\newoperator{vol}   % Volume
\newoperator{wt}    % Weight

% Roman

\newcommand{\A}{\mathrm{A}}   % Roman A
\newcommand{\B}{\mathrm{B}}   % Roman B
\newcommand{\C}{\mathrm{C}}   % Roman C
\newcommand{\D}{\mathrm{D}}   % Roman D
\newcommand{\E}{\mathrm{E}}   % Roman E
\newcommand{\F}{\mathrm{F}}   % Roman F
\newcommand{\G}{\mathrm{G}}   % Roman G
\renewcommand{\H}{\mathrm{H}} % Roman H
\newcommand{\I}{\mathrm{I}}   % Roman I
\newcommand{\J}{\mathrm{J}}   % Roman J
\newcommand{\K}{\mathrm{K}}   % Roman K
\renewcommand{\L}{\mathrm{L}} % Roman L
\newcommand{\M}{\mathrm{M}}   % Roman M
\newcommand{\N}{\mathrm{N}}   % Roman N
\renewcommand{\O}{\mathrm{O}} % Roman O
\renewcommand{\P}{\mathrm{P}} % Roman P
\newcommand{\Q}{\mathrm{Q}}   % Roman Q
\newcommand{\R}{\mathrm{R}}   % Roman R
\renewcommand{\S}{\mathrm{S}} % Roman S
\newcommand{\T}{\mathrm{T}}   % Roman T
\newcommand{\U}{\mathrm{U}}   % Roman U
\newcommand{\V}{\mathrm{V}}   % Roman V
\newcommand{\W}{\mathrm{W}}   % Roman W
\newcommand{\X}{\mathrm{X}}   % Roman X
\newcommand{\Y}{\mathrm{Y}}   % Roman Y
\newcommand{\Z}{\mathrm{Z}}   % Roman Z

\renewcommand{\a}{\mathrm{a}} % Roman a
\renewcommand{\b}{\mathrm{b}} % Roman b
\renewcommand{\c}{\mathrm{c}} % Roman c
\renewcommand{\d}{\mathrm{d}} % Roman d
\newcommand{\e}{\mathrm{e}}   % Roman e
\newcommand{\f}{\mathrm{f}}   % Roman f
\newcommand{\g}{\mathrm{g}}   % Roman g
\newcommand{\h}{\mathrm{h}}   % Roman h
\renewcommand{\i}{\mathrm{i}} % Roman i
\renewcommand{\j}{\mathrm{j}} % Roman j
\renewcommand{\k}{\mathrm{k}} % Roman k
\renewcommand{\l}{\mathrm{l}} % Roman l
\newcommand{\m}{\mathrm{m}}   % Roman m
\renewcommand{\n}{\mathrm{n}} % Roman n
\renewcommand{\o}{\mathrm{o}} % Roman o
\newcommand{\p}{\mathrm{p}}   % Roman p
\newcommand{\q}{\mathrm{q}}   % Roman q
\renewcommand{\r}{\mathrm{r}} % Roman r
\newcommand{\s}{\mathrm{s}}   % Roman s
\renewcommand{\t}{\mathrm{t}} % Roman t
\renewcommand{\u}{\mathrm{u}} % Roman u
\renewcommand{\v}{\mathrm{v}} % Roman v
\newcommand{\w}{\mathrm{w}}   % Roman w
\newcommand{\x}{\mathrm{x}}   % Roman x
\newcommand{\y}{\mathrm{y}}   % Roman y
\newcommand{\z}{\mathrm{z}}   % Roman z

% Tikz

\tikzset{
  arrow symbol/.style={"#1" description, allow upside down, auto=false, draw=none, sloped},
  subset/.style={arrow symbol={\subset}},
  cong/.style={arrow symbol={\cong}}
}

% Fancy header

\pagestyle{fancy}
\lhead{\module}
\rhead{\nouppercase{\leftmark}}

% Make title

\title{\module}
\author{Lectured by \lecturer \\ Typed by David Kurniadi Angdinata}
\date{\term}

\begin{document}

% Title page
\maketitle
\cover
\vfill
\begin{abstract}
\noindent\syllabus
\end{abstract}

\pagebreak

% Contents page
\tableofcontents

\pagebreak

% Document page
\setcounter{section}{-1}

\section{Introduction}

\lecture{1}{Thursday}{09/01/20}

Differential topology is the study of the topology of a manifold using analysis. The topics are
\begin{itemize}
\item a review of differential forms,
\item de Rham cohomology,
\item Morse theory, and
\item singular homology.
\end{itemize}
The following are references.
\begin{itemize}
\item J M Lee, Introduction to smooth manifolds, 2000
\item L W Tu, Introduction to smooth manifolds, 2008
\item J Milnor, Morse theory, 1960
\item A Banyaga and D Hurtubise, Lectures on Morse homology, 2004
\end{itemize}

\pagebreak

\section{Differential forms on manifolds}

\subsection{Alternating \texorpdfstring{$ p $}{p}-forms on a vector space}

Let $ V $ be a vector space over $ \RR $, and let $ p \ge 0 $. Then $ V^p = V \times \dots \times V $.

\begin{definition}
A multilinear map $ \omega : V^p \to \RR $ is called an \textbf{alternating $ p $-form} if we have
$$ \omega\br{v_{\sigma\br{1}}, \dots, v_{\sigma\br{p}}} = \epsilon\br{\sigma}\omega\br{v_1, \dots, v_p}, \qquad v_1, \dots, v_p \in V \qquad \sigma \in \SSS_p, $$
where $ \SSS_p $ is the group of permutations of $ p $ elements and $ \epsilon\br{\sigma} $ is the signature of $ \sigma $.
\end{definition}

Recall that if $ m $ is the number of transpositions in a decomposition of $ \sigma $, then $ \epsilon\br{\sigma} = \br{-1}^m $, where a \textbf{transposition} is $ \br{a_ia_j} $ for $ a_i \ne a_j $.

\begin{notation}
$ \Lambda^pV^* = \cbr{\text{alternating $ p $-forms} \ \omega \ \text{on} \ V} $ is called the \textbf{$ p $-th exterior power} of $ V $.
\end{notation}

Check that it is a vector space. \footnote{Exercise}

\begin{example}
\hfill
\begin{itemize}
\item $ \Lambda^0V^* = \RR $.
\item $ \Lambda^1V^* = V^* = \Hom\br{V, \RR} $, the \textbf{dual} of $ V $.
\end{itemize}
\end{example}

\begin{definition}
Let $ \omega_1 \in \Lambda^pV^* $ and $ \omega_2 \in \Lambda^qV^* $. We define the \textbf{exterior product} $ \omega_1 \wedge \omega_2 \in \Lambda^{p + q}V^* $ of $ \omega_1 $ and $ \omega_2 $ by
$$ \omega_1 \wedge \omega_2\br{v_1, \dots, v_{p + q}} = \sum_{\sigma \in \SSS_{p, q}} \epsilon\br{\sigma}\omega_1\br{v_{\sigma\br{1}}, \dots, v_{\sigma\br{p}}}\omega_2\br{v_{\sigma\br{p + 1}}, \dots, v_{\sigma\br{p + q}}}, \qquad v_1, \dots, v_{p + q} \in V, $$
where
$$ \SSS_{p, q} = \cbr{\sigma \in \SSS_{p + q} \st \sigma\br{1} < \dots < \sigma\br{p}, \ \sigma\br{p + 1} < \dots < \sigma\br{p + q}}. $$
\end{definition}

\begin{example}
\hfill
\begin{itemize}
\item Assume $ \omega_1, \omega_2 \in \Lambda^1V^* $. Then
$$ \omega_1 \wedge \omega_2\br{v_1, v_2} = \omega_1\br{v_1}\omega_2\br{v_2} - \omega_1\br{v_2}\omega_2\br{v_1}, \qquad v_1, v_2 \in V. $$
\item Assume $ \omega_1, \dots, \omega_p \in \Lambda^1V^* $. Then
$$ \omega_1 \wedge \dots \wedge \omega_p\br{v_1, \dots, v_p} = \det \br{\omega_i\br{v_j}}_{i, j = 1, \dots, p}, \qquad v_1, \dots, v_p \in V. $$
\end{itemize}
\end{example}

\begin{proposition}
\label{prop:1.6}
Let $ \omega_i \in \Lambda^{p_i}V^* $ for $ i = 1, 2, 3 $.
\begin{itemize}
\item Associativity $ \br{\omega_1 \wedge \omega_2} \wedge \omega_3 = \omega_1 \wedge \br{\omega_2 \wedge \omega_3} $.
\item Distributivity $ \omega_1 \wedge \br{\omega_2 + \omega_3} = \omega_1 \wedge \omega_2 + \omega_1 \wedge \omega_3 $, assuming $ p_2 = p_3 $.
\item Supercommutativity $ \omega_1 \wedge \omega_2 = \br{-1}^{p_1 \cdot p_2}\omega_2 \wedge \omega_1 $.
\end{itemize}
\end{proposition}

\begin{definition}
Let $ \Phi : V \to W $ be a linear map between vector spaces over $ \RR $. Let $ \omega \in \Lambda^pW^* $. Then the \textbf{pull-back} $ \Phi^*\omega \in \Lambda^pV^* $ of $ \omega $ is an alternating $ p $-form on $ V $ defined by
$$ \Phi^*\omega\br{v_1, \dots, v_p} = \omega\br{\Phi\br{v_1}, \dots, \Phi\br{v_p}}, \qquad v_1, \dots, v_p \in V. $$
\end{definition}

\pagebreak

\begin{proposition}
\label{prop:1.8}
Given $ \Phi : V \to W $ a linear map,
\begin{itemize}
\item the pull-back
$$ \function[\Phi^*]{\Lambda^pW^*}{\Lambda^pV^*}{\omega}{\Phi^*\omega} $$
is a linear map that preserves exterior products, that is
$$ \Phi^*\br{\omega_1 \wedge \omega_2} = \Phi^*\omega_1 \wedge \Phi^*\omega_2, \qquad \omega_1 \in \Lambda^pW^*, \qquad \omega_2 \in \Lambda^qW^*, $$
\item if $ \Psi : W \to Z $ is linear then
$$ \br{\Psi \circ \Phi}^*\omega = \Phi^*\Psi^*\omega, \qquad \omega \in \Lambda^pZ^*, $$
\item assuming $ V = W $ and $ p = \dim V $, then
$$ \Phi^*\omega = \br{\det \Phi}\omega, \qquad \omega \in \Lambda^pV^*. $$
\end{itemize}
\end{proposition}

\subsection{Differential forms on manifolds}

Let $ M $ be a smooth manifold of dimension $ n $, and let $ x \in M $. Then the tangent space $ \T_xM $ of $ M $ at $ x $ is a vector space of dimension $ n $.

\begin{notation}
Let
$$ \Lambda^p\T_x^*M = \Lambda^p\br{\T_xM}^*. $$
Consider the set
$$ \Lambda^p\T^*M = \bigsqcup_{x \in M} \Lambda^p\T_x^*M, $$
the \textbf{$ p $-th exterior bundle} on $ M $. There exists a morphism $ \pi : \Lambda^p\T^*M \to M $ such that for all $ x \in M $, $ \pi^{-1}\br{x} = \Lambda^p\T_x^*M $, so $ \Lambda^p\T^*M $ is a vector bundle and it is a smooth manifold, and $ \pi $ is a smooth morphism.
\end{notation}

\begin{example}
\hfill
\begin{itemize}
\item $ \Lambda^0\T^*M = M \times \RR $.
\item $ \Lambda^1\T^*M $ is the \textbf{cotangent bundle}, the dual of the tangent bundle.
\end{itemize}
\end{example}

\lecture{2}{Monday}{13/01/20}

\begin{definition}
A \textbf{differential $ p $-form} $ \omega $ on $ M $ is a smooth section of $ \pi $. That is, it is a smooth morphism $ \omega : M \to \Lambda^p\T^*M $ such that $ \pi \circ \omega = \id_M $.
\end{definition}

Thus, $ \omega\br{x} \in \Lambda^p\T_x^*M $.

\begin{notation}
$$ \Omega^p\br{M} = \cbr{\text{differential $ p $-forms} \ \omega \ \text{on} \ M}, \qquad \Omega^\bullet\br{M} = \bigoplus_p \Omega^p\br{M}. $$
\end{notation}

\begin{example}
$$ \Omega^0\br{M} \cong \cbr{f : M \to \RR \ \text{$ \C^\infty $-function}}. $$
\end{example}

\begin{exercise*}
If $ n = \dim M $, then $ \Omega^{n + 1}\br{M} = 0 $.
\end{exercise*}

The algebra is the same as last week.

\begin{definition}
Let $ \omega_1 \in \Omega^p\br{M} $ and $ \omega_2 \in \Omega^q\br{M} $. Then $ \omega_1 \wedge \omega_2 \in \Omega^{p + q}\br{M} $ is defined by
$$ \omega_1 \wedge \omega_2\br{x} = \omega_1\br{x} \wedge \omega_2\br{x} \in \Lambda^{p + q}\T_x^*M, \qquad x \in M. $$
\end{definition}

By Proposition \ref{prop:1.6}, associativity, distributivity, and supercommutativity hold for $ \Omega^p\br{M} $. Let $ F : M \to N $ be a smooth morphism between manifolds. Then for all $ x \in M $, the differential of $ F $ at $ x $ is the linear map
$$ \D F_x : \T_xM \to \T_{F\br{x}}N. $$

\pagebreak

Thus, for all $ p \ge 0 $, we have a natural map, called the \textbf{pull-back},
$$ \function[F_x^*]{\Lambda^p\T_{F\br{x}}^*N}{\Lambda^p\T_x^*M}{\omega\br{v_1, \dots, v_p}}{\omega\br{\D F_x\br{v_1}, \dots, \D F_x\br{v_p}}}, \qquad \omega \in \Lambda^p\T_{F\br{x}}^*N, \qquad v_1, \dots, v_p \in \T_x^*M. $$
Thus, we can define
$$ \function[F^*]{\Omega^p\br{N}}{\Omega^p\br{M}}{\omega\br{x}}{F^*\omega\br{F\br{x}}}, \qquad \omega \in \Omega^p\br{N}. $$
By Proposition \ref{prop:1.8}, the pull-back preserves the exterior product, so
$$ F^*\br{\omega_1 \wedge \omega_2} = F^*\omega_1 \wedge F^*\omega_2. $$
If $ G : N \to P $,
$$ \br{G \circ F}^*\omega = F^*G^*\omega. $$

\subsection{Local description of \texorpdfstring{$ p $}{p}-forms}

Let $ M $ be a manifold of dimension $ n $, let $ x_0 \in M $, let $ \br{U, \phi} $ be a local chart around $ x_0 $, and let $ \br{x_1, \dots, x_n} $ be local coordinates around $ x_0 $. A basis of $ \T_{x_0}M $ is given by
$$ \cbr{\dpd{}{x_1}, \dots, \dpd{}{x_n}}. $$
A basis of $ \T_{x_0}^*M $ is given by
$$ \cbr{\d x_1, \dots, \d x_n}, \qquad \d x_i\br{\dpd{}{x_j}} = \delta_{ij}. $$
A basis of $ \Lambda^p\T_{x_0}^*M $ is
$$ \d x_{i_1} \wedge \dots \wedge \d x_{i_p}, \qquad i_1 < \dots < i_p. $$
Thus, $ \omega \in \Omega^p\br{M} $ is locally given by
$$ \omega\br{x} = \sum_{\abs{I} = p} f_I\br{x} \d x_{i_1} \wedge \dots \wedge \d x_{i_p}, \qquad I = \br{i_1, \dots, i_p}, \qquad i_1 < \dots < i_p, $$
where $ f_I $ is a $ \C^\infty $-function on $ U $ for all $ I $.

\begin{example}
Let $ F : M \to N $ be a smooth morphism between manifolds of dimension $ n $, and let $ \omega \in \Omega^n\br{N} $. Locally,
$$ \omega\br{y} = f\br{y} \d y_1 \wedge \dots \wedge \d y_n, \qquad y \in N, $$
for some $ f \in \C^\infty $. Proposition \ref{prop:1.8} implies that
$$ F^*\omega\br{x} = \br{f \circ F}\br{x}\det \D F_x \d x_1 \wedge \dots \wedge \d x_n, \qquad x \in M, $$
where $ y_i = \p_i \circ F $ and $ \p_i : \RR^n \to \RR $ is the $ i $-th projection.
\end{example}

Let $ f : M \to \RR $ be a smooth function, so $ f \in \Omega^0\br{M} $. Locally, the \textbf{differential} is
$$ \function[\d]{\Omega^0\br{M}}{\Omega^1\br{M}}{f}{\sum_{i = 1}^n \dpd{f}{x_i} \d x_i}. $$
Check that $ \d f \in \Omega^1\br{M} $, so $ \d f $ is a $ 1 $-form on $ M $. Alternatively, $ \d f = f^*\d x $ for $ \d x $ a $ 1 $-form on $ \RR $, or $ \d f\br{X} = X\br{f} $ for any vector field $ X $ on $ M $. More in general, let $ \omega \in \Omega^p\br{M} $. Locally,
$$ \omega = \sum_{\abs{I} = p} f_I \d x_{i_1} \wedge \dots \wedge \d x_{i_p}, \qquad f_I \in \C^\infty, $$
so $ \d\omega \in \Omega^{p + 1}\br{M} $. Then the \textbf{de Rham differential} is
$$ \function[\d]{\Omega^p\br{M}}{\Omega^{p + 1}\br{M}}{\omega}{\sum_{\abs{I} = p} \d f_I \wedge \d x_{i_1} \wedge \dots \wedge \d x_{i_p}}. $$

\pagebreak

\begin{proposition}
\label{prop:1.16}
\hfill
\begin{itemize}
\item The Leibnitz rule
$$ \d\br{\omega_1 \wedge \omega_2} = \d\omega_1 \wedge \omega_2 + \br{-1}^p\omega_1 \wedge \d\omega_2, \qquad w_1 \in \Omega^p\br{M}, \qquad \omega_2 \in \Omega^q\br{M}. $$
\item $ \d^2 = 0 $, that is
$$ \d\br{\d\omega} = 0, \qquad \omega \in \Omega^p\br{M}. $$
\item Let $ F : M \to N $ be a smooth morphism between manifolds. Then
$$ F^*\d\omega = \d\br{F^*\omega}, \qquad \omega \in \Omega^p\br{M} $$
so
$$
\begin{tikzcd}
\Omega^p\br{M} \arrow{r}{\d} & \Omega^{p + 1}\br{M} \\
\Omega^p\br{N} \arrow{u}{F^*} \arrow[swap]{r}{\d} & \Omega^{p + 1}\br{N} \arrow[swap]{u}{F^*}
\end{tikzcd}.
$$
\end{itemize}
\end{proposition}

\lecture{3}{Tuesday}{14/01/20}

\begin{definition}
\hfill
\begin{itemize}
\item $ \omega \in \Omega^p\br{M} $ is \textbf{closed} if $ \d\omega = 0 $.
\item $ \omega \in \Omega^p\br{M} $ is \textbf{exact} if there exists $ \omega' \in \Omega^{p - 1}\br{M} $ such that $ \d\omega' = \omega $.
\end{itemize}
\end{definition}

$ \omega $ is exact implies that $ \omega $ is closed, since if $ \omega = \d\omega' $ then $ \d\omega = \d^2\omega' = 0 $.

\subsection{Integrations on manifolds}

Let $ M $ be a manifold of dimension $ n $, let $ F : M \to M $ be a smooth morphism, and let $ \omega \in \Omega^n\br{M} $. Then
$$ F^*\omega\br{x} = \det \D F_x\omega\br{F\br{x}}. $$
Locally, assume $ \omega = f \d y_1 \wedge \dots \wedge \d y_n $ for some coordinates $ \br{y_1, \dots, y_n} $ and $ f \in \C^\infty $. Let $ \cbr{\br{U_\alpha, \phi_\alpha}} $ be an atlas of $ M $, where $ \phi_\alpha : U_\alpha \to V_\alpha \subset \RR^n $. Then
$$ h_{\alpha\beta} = \phi_\beta \circ \phi_\alpha^{-1} : \phi_\alpha\br{U_\alpha \cap U_\beta} \subset \RR^n \to \phi_\beta\br{U_\alpha \cap U_\beta} \subset \RR^n, $$
such that
$$ h_{\alpha\beta}^*\omega\br{x} = \br{f \circ h_{\alpha\beta}}\br{x}\det \br{\D h_{\alpha\beta}}_x \d x_1 \wedge \dots \wedge \d x_n. $$
Let $ D \subset \RR^n $ be compact such that $ \partial D $ has zero measure, so $ D $ is a domain of integration, let $ f : U \to \RR $ be a $ \C^\infty $-function where $ U \subset \RR^n $ is open such that $ D \subset U $, and let $ h : U \to h\br{U} $ be a diffeomorphism. Then
$$ \intd{h^{-1}\br{D}}{}{f\br{y}}{y_1 \dots \d y_n} = \intd{h^{-1}\br{D}}{}{f\br{y}}{y_1 \wedge \dots \wedge \d y_n} = \intd{D}{}{\br{f \circ h}\br{x}\abs{\det \D h_x}}{x_1 \wedge \dots \wedge \d x_n}. $$
Let us assume that $ \omega = f\br{y}\d y_1 \wedge \dots \wedge \d y_n $ on $ U $. We define
$$ \int_D \omega = \intd{D}{}{f\br{y}}{y_1 \wedge \dots \wedge \d y_n}, \qquad D \subset U. $$

\begin{definition}
Let $ U \subset \RR^n $ be an open set. We define the \textbf{support} of $ \omega $ as
$$ \supp \omega = \overline{\cbr{x \in U \st \omega\br{x} \ne 0}}, \qquad \omega\br{x} \in \Lambda^p\T_x^*U. $$
Then $ \omega $ has \textbf{compact support}, if $ \supp \omega $ is compact.
\end{definition}

\begin{fact*}
Under this assumption, we can define
$$ \int_U \omega = \int_D \omega \in \RR, $$
which is well-defined. Under the same assumption, if $ \phi : V \to U $ is a diffeomorphism, provided that $ \det \D\phi_x > 0 $, since $ \det \D\phi_x \ne 0 $ for all $ x $, then
$$ \int_U \omega = \int_V \phi^*\omega. $$
\end{fact*}

\pagebreak

\subsection{Orientation}

Let $ V $ be a vector space over $ \RR $ of dimension $ n $, and let $ B = \br{b_1, \dots, b_n} \subset V $ and $ B' = \br{b_1', \dots, b_n'} \subset V $ be ordered bases of $ V $. Then $ B $ and $ B' $ have the \textbf{same orientation} if $ \det T > 0 $ where
$$ \function[T]{V}{V}{b_i}{b_i'} $$
is a linear map. Let $ \omega \in \Lambda^nV^* $ for $ \omega \ne 0 $. Then $ B $ and $ B' $ have the same orientation if and only if $ \omega\br{b_1, \dots, b_n} $ has the same sign as $ \omega\br{b_1', \dots, b_n'} $, by Proposition \ref{prop:1.8}. An \textbf{orientation} $ \Lambda $ of $ V $ is a set of all the ordered basis of $ V $ with the same orientation. Let $ \phi : V \to W $ be an isomorphism of vector spaces with fixed orientations $ \Lambda_v $ and $ \Lambda_w $ respectively. We say that $ \phi $ is \textbf{orientation preserving} if an ordered basis of $ V $ induces an ordered basis of $ W $, so $ \Lambda_v $ induces $ \Lambda_w $. Let $ V = \RR^n $, and let $ e_i = \br{0, \dots, 0, 1, 0, \dots, 0} $. Then $ e_1, \dots, e_n $ defines an orientation of $ V $ called \textbf{positive}. Let $ M $ be a manifold. The idea is to find an orientation $ \Lambda_x $ of $ \T_xM $ for all $ x \in M $.
\begin{itemize}[leftmargin=1in]
\item[Special case.] Let $ M = U \subset \RR^n $ be open. There exists a natural isomorphism $ \phi_x : \T_xU \to \RR^n $. Let $ \Lambda_x^+ $ be an orientation on $ \T_xU $ such that $ \phi_x $ is orientation preserving with respect to the positive orientation on $ \RR^n $. Let $ \Lambda^+ = \cbr{\Lambda_x^+} $.
\item[General case.] Let $ \cbr{\br{U_\alpha, \phi_\alpha}} $ be an atlas on $ M $. On $ U_\alpha $, we define the orientation so that $ \br{\D\phi_\alpha}_x : \T_xU_\alpha \to \T_{\phi_\alpha\br{x}}\phi_\alpha\br{U} \subset \RR^n $ is orientation preserving. This is called the positive orientation on the chart $ \br{U_\alpha, \phi_\alpha} $. We define $ \Lambda $ on $ M $, which is a collection of $ \Lambda^+ $ on $ \T_xM $ for all $ x \in M $. Then $ M $ is \textbf{orientable} if there exists an atlas with positive orientation charts. This coincides in assuming that $ \det \D\br{\phi_\beta^{-1} \circ \phi_\alpha} > 0 $ for all $ \alpha $ and $ \beta $.
\end{itemize}

\lecture{4}{Thursday}{16/01/20}

\begin{notation}
For all $ p \ge 0 $,
$$ \Omega_\c^p\br{M} = \cbr{\omega \in \Omega^p\br{M} \st \supp M \ \text{is compact}}. $$
\end{notation}

If $ M $ is compact $ \Omega_\c^p\br{M} = \Omega^p\br{M} $. Let $ \omega \in \Omega_\c^r\br{M} $. Assume $ \supp \omega \subset U $ where $ \br{U, \phi} $ is a chart of $ M $, and $ \phi : U \to \phi\br{U} \subset \RR^n $. Assume also that $ \br{U, \phi} $ is positively oriented. Let $ \phi^{-1} : \phi\br{U} \to U $ such that $ \br{\phi^{-1}}^*\omega \in \Omega_\c^n\br{\phi\br{U}} $, that is $ \supp \br{\phi^{-1}}^*\omega \subset \phi\br{U} $. We define
\begin{equation}
\label{eq:1}
\int_M \omega = \int_{\phi\br{U}} \br{\phi^{-1}}^*\omega.
\end{equation}
We need to show that, under the assumptions above, $ \int_M \omega $ does not depend on $ \br{U, \phi} $. Let $ \br{\overline{U}, \overline{\phi}} $ be also a positively oriented chart such that $ \supp \omega \subset \overline{U} $. We want to show that
$$ \int_{\phi\br{U}} \br{\phi^{-1}}^*\omega = \int_{\overline{\phi}\br{\overline{U}}} \br{\overline{\phi}^{-1}}^*\omega. $$
Let $ \overline{\phi} \circ \phi^{-1} : \phi\br{U \cap \overline{U}} \to \overline{\phi}\br{U \cap \overline{U}} $, so
$$
\begin{tikzcd}
& U \cap \overline{U} \arrow[swap]{dl}{\phi} \arrow{dr}{\overline{\phi}} & \\
\RR^n \supset \phi\br{U \cap \overline{U}} \arrow[swap]{rr}{\overline{\phi} \circ \phi^{-1}} & & \overline{\phi}\br{U \cap \overline{U}} \subset \RR^n
\end{tikzcd}.
$$
Since both charts are positively oriented the determinant of the differential $ \D\br{\overline{\phi} \circ \phi^{-1}} $ is positive, so
\begin{align*}
\int_{\overline{\phi}\br{U}} \br{\overline{\phi}^{-1}}^*\omega
& = \int_{\overline{\phi}\br{U \cap \overline{U}}} \br{\overline{\phi}^{-1}}^*\omega
= \int_{\overline{\phi}\br{U \cap \overline{U}}} \br{\overline{\phi} \circ \phi^{-1}}^*\br{\overline{\phi}^{-1}}^*\omega
= \int_{\overline{\phi}\br{U \cap \overline{U}}} \br{\phi^{-1}}^*\overline{\phi}^*\br{\overline{\phi}^{-1}}^*\omega \\
& = \int_{\overline{\phi}\br{U \cap \overline{U}}} \br{\phi^{-1}}^*\br{\overline{\phi}^{-1} \circ \overline{\phi}}^*\omega
= \int_{\overline{\phi}\br{U \cap \overline{U}}} \br{\phi^{-1}}^*\omega
= \int_{\overline{\phi}\br{U}} \br{\phi^{-1}}^*\omega,
\end{align*}
by a property of the pull-back and since $ \br{\overline{\phi}^{-1}}^*\omega = 0 $ outside $ \overline{\phi}\br{U \cap \overline{U}} $.

\pagebreak

\subsection{Partitions of unity}

\begin{definition}
Let $ M $ be a manifold, and let $ U = \cbr{U_\alpha} $ be an open covering. A \textbf{partition of unity} with respect to $ U $ is a collection of smooth functions $ f_\alpha : M \to \sbr{0, 1} $ such that
\begin{enumerate}
\item $ \supp f_\alpha = \overline{\cbr{x \in M \st f_\alpha\br{x} = 0}} \subset U_\alpha $ for all $ \alpha $,
\item $ \sum_\alpha f_\alpha\br{x} = 1 $ for all $ x \in M $, and
\item for all $ x \in M $, there exists $ U \ni x $ open such that $ \supp f_\alpha \cap U \ne \emptyset $ for only finitely many $ \alpha $.
\end{enumerate}
\end{definition}

\begin{remark*}
$ 3 $ implies that $ 2 $ is a finite sum.
\end{remark*}

\begin{example}
Let
$$ M = \S^1 = \cbr{x \in \RR^2 \st \abs{x} = 1}, \qquad U_1 = \S^1 \setminus \cbr{\br{1, 0}}, \qquad U_2 = \S^1 \setminus \cbr{\br{-1, 0}}, $$
so $ \cbr{U_i} $ is a cover. Let
$$ f_1\br{\cos \theta, \sin \theta} = \dfrac{1}{2} - \dfrac{1}{2}\cos \theta, \qquad f_2\br{\cos \theta, \sin \theta} = \dfrac{1}{2} + \dfrac{1}{2}\cos \theta. $$
Then $ f_i $ is a partition of unity.
\end{example}

\begin{proposition}
Let $ M $ be a manifold, and let $ U = \cbr{U_\alpha} $ be an open covering of $ M $. Then there exists a partition of unity $ f_\alpha $ with respect to $ U $.
\end{proposition}

\begin{proof}
We omit the proof.
\end{proof}

\begin{proposition}
\label{prop:1.23}
Let $ M $ be a manifold, and let $ n = \dim M $. Then $ M $ is orientable if and only if there exists $ \omega \in \Omega^n\br{M} $ which is never vanishing on $ M $, so $ \omega\br{x} \ne 0 $ for all $ x \in M $.
\end{proposition}

$ \omega $ is called a \textbf{volume form} on $ M $.

\begin{proof}
\hfill
\begin{itemize}
\item[$ \impliedby $] Assume $ \omega \in \Omega^n\br{M} $ is a volume form. We want to construct an orientation $ \Lambda $ on $ M $, that is $ \Lambda_x $ on $ \T_xM $ for all $ x \in M $. Given an oriented basis $ v_1, \dots, v_n $ of $ \T_xM $ we say that it is \textbf{positively oriented} if $ \omega\br{x}\br{v_1, \dots, v_n} > 0 $. For all $ x \in M $, we define the orientation $ \Lambda_x $ on $ \T_xM $ by considering the class of positively oriented ordered basis of $ \T_xM $ which is compatible with the choice of an atlas on $ M $. Take any atlas $ \cbr{\br{U_\alpha, \phi_\alpha}} $, where $ \phi_\alpha : U_\alpha \to \RR^n $. On $ U_\alpha $,
$$ \omega = g_\alpha\phi_\alpha^*\d x_1 \wedge \dots \wedge \d x_n. $$
Since $ \omega \ne 0 $, $ g_\alpha > 0 $ or $ g_\alpha < 0 $. If $ g_\alpha < 0 $ then switch $ x_1 $ with $ x_2 $, so $ g_\alpha > 0 $. After this change of coordinates, $ \br{U_\alpha, \phi_\alpha} $ is positively oriented, so $ M $ is orientable.

\lecture{5}{Monday}{20/01/20}

\item[$ \implies $] Assume that $ M $ is orientable, that is there exists an atlas $ \cbr{\br{U_\alpha, \phi_\alpha}} $ of positively oriented charts. On $ U_\alpha $, we consider
$$ \omega_\alpha = \phi_\alpha^*\d x_1 \wedge \dots \wedge \d x_n. $$
Let $ f_\alpha $ be a partition of unity with respect to $ \cbr{U_\alpha} $. Let $ \widetilde{\omega_\alpha} = f_\alpha\omega_\alpha \in \Omega^n\br{U_\alpha} $. We may assume that $ \widetilde{\omega_\alpha} \in \Omega^n\br{M} $ by extending equal to zero outside $ U_\alpha $. We define $ \omega = \sum_\alpha \widetilde{\omega_\alpha} \in \Omega^n\br{M} $. For all $ \alpha $, since $ \sum_\alpha f_\alpha = 1 $ there exists $ \alpha $ such that $ \widetilde{\omega_\alpha} \ne 0 $, so $ \omega \ne 0 $.
\end{itemize}
\end{proof}

Let $ M $ be an orientable manifold of dimension $ n $, and let $ \omega \in \Omega_\c^n\br{M} $. We want to define $ \int_M \omega $. So far we defined for $ \omega $ such that $ \supp \omega \subset U_\alpha $ where $ \br{U_\alpha, \phi_\alpha} $ is a chart.

\begin{definition}
Let $ \cbr{\br{U_\alpha, \phi_\alpha}} $ be a positively oriented atlas on $ M $, and let $ f_\alpha $ be a partition of unity with respect to $ \cbr{U_\alpha} $. Then $ \supp f_\alpha\omega \subset U_\alpha $, so let
$$ \int_M \omega = \sum_\alpha \int_{U_\alpha} f_\alpha\omega. $$
\end{definition}

\pagebreak

\begin{remark}
Note that for each $ \alpha $, we have that the support of $ f_\alpha\omega $ is contained in $ U_\alpha $ and therefore each term of the sum is well-defined as in $ \br{\ref{eq:1}} $. Indeed, we have
$$ \int_M \omega = \sum_\alpha \int_{U_\alpha} f_\alpha\omega = \sum_\alpha \int_{\phi_\alpha\br{U_\alpha}} \br{\phi^{-1}}^*f_\alpha. $$
\end{remark}

\begin{lemma}
$ \int_M \omega $ does not depend on $ \cbr{\br{U_\alpha, \phi_\alpha}} $ and $ f_\alpha $.
\end{lemma}

\begin{proof}
Under the assumption that $ \supp \omega \subset U_\alpha $ then we showed $ \int_{U_\alpha} \omega $ does not depend on $ \br{U_\alpha, \phi_\alpha} $. Let $ \cbr{\br{U_\alpha, \phi_\alpha}} $ and $ \cbr{\br{\overline{U_\alpha}, \overline{\phi_\alpha}}} $ be two atlases with positively oriented charts, and let $ f_\alpha $ and $ \overline{f_\alpha} $ be two partitions of unity with respect to $ \cbr{U_\alpha} $ and $ \cbr{\overline{U_\alpha}} $ respectively. Then $ \sum_\alpha f_\alpha = \sum_\alpha \overline{f_\alpha} = 1 $, so $ \int_M f_\alpha\omega = \sum_\beta \int_M \overline{f_\beta}f_\alpha\omega $. Thus
$$ \int_M \omega = \sum_\alpha \int_M f_\alpha\omega = \sum_{\alpha, \beta} \int_M \overline{f_\beta}f_\alpha\omega = \sum_\beta \int_M \sum_\alpha f_\alpha\overline{f_\beta}\omega = \sum_\beta \int_M \overline{f_\beta}\omega. $$
\end{proof}

\begin{proposition}
Let $ M $ and $ N $ be orientable manifolds of dimension $ n $, and let $ \omega, \eta \in \Omega_\c^n\br{M} $.
\begin{enumerate}
\item Linearity
$$ \int_M \br{a\omega + b\eta} = a\int_M \omega + b\int_M \eta. $$
\item Orientation reversal. Let $ \overline{M} $ be the manifold $ M $ with opposite orientation $ \Lambda^- = \cbr{\Lambda_x^- \st x \in M} $, which is the orientation opposite than the one induced by $ M $ with orientation $ \Lambda $. Then
$$ \int_M \omega = -\int_{\overline{M}} \omega. $$
\item Positivity. Let $ \omega $ be the volume form on $ M $. Then
$$ \int_M \omega > 0. $$
\item Diffeomorphism invariance. Let $ F : N \to M $ be an orientation preserving diffeomorphism. Then
$$ \int_M \omega = \int_N F^*\omega. $$
\end{enumerate}
\end{proposition}

\begin{proof}
\hfill
\begin{enumerate}
\item Exercise. \footnote{Exercise}
\item Exercise. \footnote{Exercise}
\item Choose a positively oriented chart $ \br{U_\alpha, \phi_\alpha} $ on $ U_\alpha $, so
$$ \omega = g_\alpha\phi_\alpha^*\d x_1 \wedge \dots \wedge \d x_n, \qquad g_\alpha > 0. $$
Then $ \int_M \omega = \sum_\alpha \int_{U_\alpha} f_\alpha\omega $ where $ f_\alpha $ is a partition of unity. For all $ x \in M $ there exists $ \alpha $ such that $ x \in U_\alpha $ and $ \int_{U_\alpha} f_\alpha\omega > 0 $, so $ \int_M \omega > 0 $.
\item Let $ \br{U_\alpha, \phi_\alpha} $ be a positively oriented atlas on $ M $. Then $ \br{F^{-1}\br{U_\alpha}, \phi_\alpha \circ F} $ is an atlas on $ N $ which is positively oriented. Let $ f_\alpha $ be a partition of unity with respect to $ \cbr{U_\alpha} $. Then $ f_\alpha \circ F $ is a partition of the unity with respect to $ \cbr{F^{-1}\br{U_\alpha}} $, so
$$ \int_N F^*\omega = \sum_\alpha \int_N \br{f_\alpha \circ F}F^*\omega = \sum_\alpha \int_N F^*\br{f_\alpha\omega} = \sum_\alpha \int_M f_\alpha\omega = \int_M \omega. $$
\end{enumerate}
\end{proof}

\pagebreak

\subsection{Manifolds with boundary}

Denote
$$ \RR_{\ge 0}^n = \br{\RR_{\ge 0}}^n, \qquad \RR_+^n = \cbr{\br{x_1, \dots, x_n} \in \RR^n \st x_n \ge 0}. $$
Let $ U \subset \RR_+^n $ be open, and let $ F : U \to \RR^m $ be a function. Then $ F $ is $ \C^\infty $ if it can be extended to a $ \C^\infty $-function $ \widetilde{F} : \widetilde{U} \to \RR^m $ where $ \widetilde{U} \supset U $ and $ \widetilde{U} $ is open.

\lecture{6}{Tuesday}{21/01/20}

\begin{definition}
A \textbf{manifold with boundary} of dimension $ n $ is a Hausdorff topological space $ M $ such that there exists an open covering $ \cbr{U_\alpha} $, and for all $ \alpha $, there exists a homeomorphism $ \phi_\alpha : U_\alpha \to \RR_+^n = \RR^{n - 1} \times \RR_{\ge 0} $ such that for all $ \alpha $ and $ \beta $,
$$ \phi_\alpha \circ \phi_\beta^{-1} : \phi_\beta\br{U_\alpha \cap U_\beta} \subset \RR_+^n \to \phi_\alpha\br{U_\alpha \cap U_\beta} \subset \RR_+^n $$
is a diffeomorphism, so
$$
\begin{tikzcd}
& U_\alpha \cap U_\beta \arrow[swap]{dl}{\phi_\alpha} \arrow{dr}{\phi_\beta} & \\
\RR_+^n \supset \phi_\alpha\br{U_\alpha \cap U_\beta} \arrow[swap]{rr}{\phi_\alpha \circ \phi_\beta^{-1}} & & \phi_\beta\br{U_\alpha \cap U_\beta} \subset \RR_+^n
\end{tikzcd}.
$$
The \textbf{boundary} of $ M $ is
$$ \partial M = \cbr{x \in M \st \exists \alpha, \ \phi_\alpha\br{x} \in \partial\RR_+^n = \RR^{n - 1} \times \cbr{0}}. $$
Then $ \br{U_\alpha, \phi_\alpha} $ is called a \textbf{chart} and $ \cbr{\br{U_\alpha, \phi_\alpha}} $ is called an \textbf{atlas}.
\end{definition}

\begin{remark}
\hfill
\begin{itemize}
\item $ \partial M $ is closed in $ M $.
\item $ \mathring{M} = M \setminus \partial M $ is a manifold of dimension $ n $.
\end{itemize}
\end{remark}

\begin{example}
\hfill
\begin{itemize}
\item $ M = \sbr{0, 1} $ is a manifold with boundary $ \partial M = \cbr{0, 1} $.
\item The closed disc $ \D = \cbr{x \in \RR^n \st \abs{x} \le 1} $ is a manifold with boundary $ \partial\D = \S^{n - 1} $.
\item $ M = \sbr{0, 1} \times \S^1 $ is a manifold with boundary $ \partial M = \S^1 \sqcup \S^1 $.
\end{itemize}
\end{example}

\begin{remark}
\hfill
\begin{itemize}
\item We can define tangent spaces and differential forms exactly in the same way as usual manifolds.
\item The definition of orientability is the same. If $ M $ is orientable, then $ \partial M $ is also orientable. As a convention, the positive orientation on the boundary of $ \RR_+^n = \RR^{n - 1} \times \cbr{0} $ is given by $ \br{-1}^n \d x_1 \wedge \dots \wedge \d x_{n - 1} $. This induces a positive orientation on $ \partial M $.
\item Also partitions of unity for any open cover $ U_\alpha $ of $ M $ is defined the same way. If $ M $ is orientable, for any manifold with boundary, for all open covering $ U = \cbr{U_\alpha} $, there exists a partition of unity $ f_\alpha $. This implies that if $ \omega \in \Omega_\c^n\br{M} $, then $ \int_M \omega $ is defined the same way for manifolds.
\end{itemize}
\end{remark}

\subsection{Stokes' theorem}

\begin{theorem}[Stokes]
For any manifold with boundary $ M $ of dimension $ n $, and for any $ \omega \in \Omega_\c^{n - 1}\br{M} $ we have
$$ \intd{M}{}{}{\omega} = \int_{\partial M} \omega \in \Omega_\c^n\br{M}. $$
\end{theorem}

\pagebreak

\begin{proof}
Let $ \cbr{\br{U_\alpha, \phi_\alpha}} $ be an atlas, and let $ f_\alpha : M \to \RR $ be a partition of unity with respect to this cover. Then $ \sum_\alpha f_\alpha = 1 $ on $ M $, so
$$ \intd{M}{}{}{\omega} = \intd{M}{}{}{\br{\sum_\alpha f_\alpha\omega}} = \sum_\alpha \intd{M}{}{}{\br{f_\alpha\omega}} = \sum_\alpha \int_{\phi_\alpha\br{U_\alpha}} \br{\phi_\alpha^{-1}}^*\d\br{f_\alpha\omega}. $$
Proposition \ref{prop:1.16} implies that
$$ \br{\phi_\alpha^{-1}}^*\d\br{f_\alpha\omega} = \d\br{\br{\phi_\alpha^{-1}}^*\br{f_\alpha\omega}}. $$
Then $ \br{\phi_\alpha^{-1}}^*\br{f_\alpha\omega} $ is an $ \br{n - 1} $-form on $ \phi_\alpha\br{U_\alpha} $. In coordinates,
$$ \br{\phi_\alpha^{-1}}^*\br{f_\alpha\omega} = \sum_{j = 1}^n \widetilde{f_\alpha}\omega_j \d x_1 \wedge \dots \wedge \widehat{\d x_j} \wedge \dots \wedge \d x_n, $$
where $ \omega_j $ is a smooth function on $ \phi_\alpha\br{U_\alpha} $ and
$$
\begin{tikzcd}
U_\alpha \arrow{r}{\widetilde{\phi_\alpha}} \arrow[swap]{d}{f_\alpha} & \phi_\alpha\br{U_\alpha} \arrow{dl}{\widetilde{f_\alpha}} \\
\sbr{0, 1} &
\end{tikzcd}.
$$
Then
\begin{align*}
\d\br{\br{\phi_\alpha^{-1}}^*\br{f_\alpha\omega}}
& = \d\br{\sum_{j = 1}^n \widetilde{f_\alpha}\omega_j \d x_1 \wedge \dots \wedge \widehat{\d x_j} \wedge \dots \wedge \d x_n} \\
& = \sum_{j = 1}^n \sum_{k = 1}^n \dpd{}{x_k}\br{\widetilde{f_\alpha}\omega_j} \d x_k \wedge \d x_1 \wedge \dots \wedge \widehat{\d x_j} \wedge \dots \wedge \d x_n \\
& = \sum_{j = 1}^n \dpd{}{x_j}\br{\widetilde{f_\alpha}\omega_j} \d x_j \wedge \d x_1 \wedge \dots \wedge \widehat{\d x_j} \wedge \dots \wedge \d x_n \\
& = \sum_{j = 1}^n \br{-1}^{j - 1}\dpd{}{x_j}\br{\widetilde{f_\alpha}\omega_j} \d x_1 \wedge \dots \wedge \d x_n,
\end{align*}
so
$$ \sum_\alpha \intd{\phi_\alpha\br{U_\alpha}}{}{}{\br{\br{\phi_\alpha^{-1}}^*\br{f_\alpha\omega}}} = \sum_\alpha \intd{\RR_+^n}{}{}{\br{\br{\phi_\alpha^{-1}}^*\br{f_\alpha\omega}}}, $$
because $ \widetilde{f_\alpha} = 0 $ outside $ \phi_\alpha\br{U_\alpha} $. Thus
\begin{align*}
\intd{M}{}{}{\omega}
& = \sum_\alpha \intd{\RR_+^n}{}{\sum_{j = 1}^n \br{-1}^{j - 1}\dpd{}{x_j}\br{\widetilde{f_\alpha}\omega_j}}{x_1 \wedge \dots \wedge \d x_n} \\
& = \sum_\alpha \intd{-\infty}{\infty}{\dots \intd{-\infty}{\infty}{\intd{0}{\infty}{\sum_{j = 1}^n \br{-1}^{j - 1}\dpd{}{x_j}\br{\widetilde{f_\alpha}\omega_j}}{x_n}}{x_{n - 1} \dots}}{x_1} \\
& = \sum_\alpha \sum_{j = 1}^n \intd{-\infty}{\infty}{\dots \widehat{\int_{-\infty}^\infty} \dots \intd{-\infty}{\infty}{\intd{0}{\infty}{\br{-1}^{j - 1}\dpd{}{x_j}\eval{\br{\widetilde{f_\alpha}\omega_j}}_{x_n = 0}}{x_n}}{x_{n - 1} \dots \widehat{\d x_j} \dots}}{x_1} \\
& = \sum_\alpha \intd{-\infty}{\infty}{\dots \intd{-\infty}{\infty}{\br{-1}^{n - 1}\eval{\br{\widetilde{f_\alpha}\omega_j}}_{x_n = 0}}{x_{n - 1} \dots}}{x_1},
\end{align*}
since $ \eval{\br{f_\alpha\omega_j}}_{x_n = 0} = 0 $ for $ j = 1, \dots, n - 1 $, so
$$ \intd{M}{}{}{\omega} = \sum_\alpha \intd{-\infty}{\infty}{\dots \intd{-\infty}{\infty}{\br{-1}^{n - 1}\eval{\br{\widetilde{f_\alpha}\omega_j}}_{x_n = 0}}{x_{n - 1} \dots}}{x_1} = \sum_\alpha \int_{\partial U_\alpha} \eval{f_\alpha}_{\partial U_\alpha}\omega = \int_{\partial M} \omega, $$
where $ \partial U_\alpha = U_\alpha \cap \partial M $.
\end{proof}

\pagebreak

\subsection{Applications of Stokes' theorem}

\lecture{7}{Thursday}{23/01/20}

\begin{theorem}[Integration by parts]
Let $ M $ be an orientable $ n $-dimensional manifold with boundary, let $ \omega \in \Omega_\c^p\br{M} $, let $ \eta \in \Omega_\c^{n - p - 1}\br{M} $, and let $ p \in \cbr{0, \dots, n - 1} $. Then
$$ \int_{\partial M} \omega \wedge \eta = \int_M \d\omega \wedge \eta + \br{-1}^p\int_M \omega \wedge \d\eta. $$
\end{theorem}

\begin{proof}
$$ \int_{\partial M} \omega \wedge \eta = \intd{M}{}{}{\br{\omega \wedge \eta}} = \int_M \br{\d\omega \wedge \eta + \br{-1}^p\omega \wedge \d\eta}, $$
by Stokes and the Leibnitz rule.
\end{proof}

\begin{theorem}[Brouwer's fixed point theorem]
Let
$$ D = \cbr{x \in \RR^n \st \abs{x} \le 1}, $$
so
$$ \partial D = \S^{n - 1} = \cbr{x \in \RR^n \st \abs{x} = 1}, $$
and let $ f : D \to D $ be a smooth morphism. Then $ f $ admits a fixed point, that is there exists $ x \in D $ such that $ f\br{x} = x $.
\end{theorem}

\begin{proof}
Assume that $ f\br{x} \ne x $ for all $ x \in D $. For any $ x \in D $, consider the ray starting from $ f\br{x} $ and passing through $ x $. Let $ g\br{x} $ be the point where this ray intersects $ \partial D $ away from $ f\br{x} $. Note that if $ x \in \partial D $ then $ g\br{x} = x $. Then $ g : D \to \partial D $. It is easy to check that $ g $ is smooth. Since $ \partial D = \S^{n - 1} $ is orientable by Proposition \ref{prop:1.23} there exists a volume form $ \omega \in \Omega^{n - 1}\br{\partial D} $, so $ \omega\br{x} \ne 0 $. Since $ \omega \in \Omega^{n - 1}\br{\partial D} $, $ \d\omega \in \Omega^n\br{\partial D} $, which is an $ n $-dimensional manifold, so $ \d\omega = 0 $. Thus
$$ 0 < \int_{\partial D} \omega = \int_{\partial D} g^*\omega = \intd{D}{}{}{g^*\omega} = \int_D g^*\d\omega = 0, $$
by Stokes, a contradiction.
\end{proof}

\begin{example}
Recall any exact form is closed, since $ \d^2 = 0 $. But the opposite is not always true. Let $ M = \RR^2 \setminus \cbr{0} $, and let
$$ \omega = \dfrac{x}{x^2 + y^2} \d y - \dfrac{y}{x^2 + y^2} \d x \in \Omega^1\br{M}. $$
Then $ \omega $ is closed, since
$$ \d\omega = \dpd{}{x}\br{\dfrac{x}{x^2 + y^2}} \d x \wedge \d y - \dpd{}{y}\br{\dfrac{y}{x^2 + y^2}} \d y \wedge \d x = 0. $$
We want to show that $ \omega $ is not exact. Assume that
$$ \omega = \d f, \qquad f \in \Omega^0\br{M} = \cbr{\text{$ \C^\infty $-function}}. $$
In particular $ \omega = \d f $ on $ \S^1 \subset M $. Let
$$ \function[\gamma]{\sbr{0, 2\pi}}{\S^1}{\theta}{\br{\cos \theta, \sin \theta}}. $$
Then
$$ \int_{\S^1} \omega = \int_0^{2\pi} \gamma^*\omega = \int_0^{2\pi} \br{\br{\dfrac{\cos \theta}{\cos^2 \theta + \sin^2 \theta}}\cos \theta \d\theta - \br{\dfrac{\sin \theta}{\cos^2 \theta + \sin^2 \theta}}\br{-\sin \theta} \d\theta} = \intd{0}{2\pi}{}{\theta} = 2\pi, $$
and
$$ \int_{\S^1} \omega = \intd{\S^1}{}{}{f} = \int_{\partial\S^1} f = \int_\emptyset f = 0, $$
so $ \omega $ is not exact.
\end{example}

\pagebreak

\begin{proposition}
Let $ M $ be an orientable manifold of dimension $ n $ without boundary, and let $ \omega \in \Omega_\c^n\br{M} $. Assume $ \omega $ is exact. Then
$$ \int_M \omega = 0. $$
\end{proposition}

\begin{proof}
Easy from Stokes.
\end{proof}

\begin{proposition}
Let $ M $ be an orientable manifold of dimension $ n $ with boundary, and let $ \omega \in \Omega_\c^{n - 1}\br{M} $ be a closed form. Then
$$ \int_{\partial M} \omega = 0. $$
\end{proposition}

\begin{proof}
Easy from Stokes.
\end{proof}

Let $ M $ be an orientable manifold of dimension $ n $, let $ \omega \in \Omega_\c^k\br{M} $, and let $ N \subset M $ be a submanifold of dimension $ k $. We can define
$$ \int_M \omega = \int_N i^*\omega, $$
where $ i : N \hookrightarrow M $ is the inclusion. We will denote
$$ \eval{\omega}_N = i^*\omega \in \Omega_\c^k\br{N}. $$

\begin{proposition}
Let $ M $ be an oriented manifold of dimension $ n $, let $ \omega \in \Omega_\c^k\br{M} $, and let $ S \subset M $ be a compact orientable submanifold of dimension $ k $ such that $ \partial S = \emptyset $ and $ \int_S \omega \ne 0 $. Then
\begin{itemize}
\item $ \omega $ is not exact,
\item $ \eval{\omega}_S $ is not exact, and
\item $ S $ is not the boundary of an orientable manifold $ N \subset M $ of dimension $ k + 1 $.
\end{itemize}
\end{proposition}

\begin{proof}
Exercise. \footnote{Exercise}
\end{proof}

\pagebreak

\section{De Rham cohomology}

\subsection{De Rham cohomology}

\lecture{8}{Monday}{27/01/20}

\begin{definition}
Let $ M $ be a manifold of dimension $ n $, and let $ p \ge 0 $. Then $ \omega_1, \omega_2 \in \Omega^p\br{M} $ are said to be \textbf{cohomologous} if $ \omega_1 - \omega_2 = \d\eta $ where $ \eta \in \Omega^{p - 1}\br{M} $. In particular $ \omega \in \Omega^p\br{M} $ is cohomologous to zero if it is exact. Let
$$ \ZZZ^p\br{M} = \Ker \br{\d : \Omega^p\br{M} \to \Omega^{p + 1}\br{M}} = \cbr{\omega \in \Omega^p\br{M} \st \omega \ \text{is closed}} \subset \Omega^p\br{M}, $$
and let
$$ \BBB^p\br{M} = \Im \br{\d : \Omega^{p - 1}\br{M} \to \Omega^p\br{M}} = \cbr{\omega \in \Omega^p\br{M} \st \omega \ \text{is exact}} \subset \Omega^p\br{M}. $$
\end{definition}

Then $ \BBB^p\br{M} \subset \ZZZ^p\br{M} $ for all $ p \ge 0 $.

\begin{notation*}
If $ p = 0 $, then $ \BBB^0\br{M} = 0 $.
\end{notation*}

\begin{note*}
If $ \omega_1, \omega_2 \in \ZZZ^p\br{M} $ then $ \omega_1 - \omega_2 \in \BBB^p\br{M} $ if and only if $ \omega_1 $ and $ \omega_2 $ are cohomologous.
\end{note*}

\begin{definition}
Denote the \textbf{$ p $-th de Rham cohomology group} as
$$ \H^p\br{M} = \ZZZ^p\br{M} / \BBB^p\br{M} = \cbr{\sbr{\omega} \st \omega \in \ZZZ^p\br{M}}, \qquad p \ge 0. $$
where
$$ \sbr{\omega} = \cbr{\omega' \in \Omega^p\br{M} \ \text{cohomologous to} \ \omega} $$
is the \textbf{de Rham class} of $ \omega $.
\end{definition}

\begin{remark*}
$ \H^p\br{M} $ is a vector space over $ \RR $.
\end{remark*}

\begin{definition}
$ \b_p\br{M} = \dim \H^p\br{M} \in \ZZ_{\ge 0} \cup \cbr{\infty} $ is the \textbf{$ p $-th Betti number} of $ M $.
\end{definition}

\begin{proposition}
If $ M $ is connected then
$$ \H^0\br{M} = \RR, $$
that is $ \b_0\br{M} = 1 $. More in general, $ \b_0\br{M} $ is the number of connected components of $ M $.
\end{proposition}

\begin{proof}
Assume $ M $ is connected. Then $ \BBB^0\br{M} = 0 $, so
\begin{align*}
\H^0\br{M}
& = \ZZZ^0\br{M}
= \cbr{f \in \Omega^0\br{M} \ \text{closed}} \\
& = \cbr{f \in \Omega^0\br{M} \st \text{locally} \ \forall x \in M, \ \dpd{}{x_i}f\br{x} = 0} \\
& = \cbr{f \in \Omega^0\br{M} \ \text{locally constant}}
= \RR.
\end{align*}
\end{proof}

\begin{example*}
Let $ M = \S^1 $. Then $ \H^0\br{M} = \RR $.
\end{example*}

\begin{proposition}
\label{prop:2.5}
Let $ M $ be a manifold of dimension $ n $. Then
$$ \H^p\br{M} = 0, \qquad p \ge n + 1. $$
\end{proposition}

\begin{proof}
Recall $ \Omega^p\br{M} = 0 $ if $ p \ge n + 1 $ because all alternating $ p $-forms for $ p \ge n + 1 $ on an $ n $-dimensional vector space are zero, so $ \ZZZ^p\br{M} = 0 $. Thus $ \H^p\br{M} = 0 $.
\end{proof}

\begin{proposition}
Let $ M $ be a compact orientable manifold of dimension $ n $ without boundary. Then
$$ \H^n\br{M} \ne 0. $$
\end{proposition}

\begin{proof}
$ M $ is orientable, so there exists a volume form $ \omega \in \Omega^n\br{M} = \Omega_\c^n\br{M} $, by Proposition \ref{prop:1.23}. Then $ \omega $ is closed, because $ \d\omega $ is an $ \br{n + 1} $-form on $ M $, so $ \omega \in \ZZZ^n\br{M} $. We want to show that $ \sbr{\omega} \ne 0 $ in $ \H^n\br{M} $. Assume $ \sbr{\omega} = 0 $, so $ \omega $ is exact. Thus $ \omega = \d\eta $ where $ \eta $ is an $ \br{n - 1} $-form on $ M $, so
$$ 0 < \int_M \omega = \int_M \d\eta = \int_{\partial M} \eta = \int_\emptyset \eta = 0, $$
by Stokes, a contradiction.
\end{proof}

\pagebreak

\begin{proposition}
\label{prop:2.7}
Let $ G : M \to N $ be a smooth morphism between manifolds. Then
$$ G^* : \Omega^p\br{N} \to \Omega^p\br{M}, \qquad p \ge 0 $$
takes closed forms of $ N $ to closed forms on $ M $ and exact forms of $ N $ to exact forms on $ M $.
\end{proposition}

\begin{proof}
Proposition \ref{prop:1.16} implies that $ G^*\d = \d G^* $. If $ \omega $ is closed then $ \d G^*\omega = G^*\d\omega = G^*0 = 0 $, so $ G^*\omega $ is closed. If $ \omega = \d\eta $ is exact then $ G^*\omega = \d G^*\eta $ is also exact.
\end{proof}

Thus $ G^* : \ZZZ^p\br{N} \to \ZZZ^p\br{M} $ and $ G^* : \BBB^p\br{N} \to \BBB^p\br{M} $, so there exists a linear map
$$ \function[G^*]{\H^p\br{N}}{\H^p\br{M}}{\sbr{\omega}}{\sbr{G^*\omega}}. $$

\begin{corollary}
Let $ M $ and $ N $ be diffeomorphic manifolds. Then
$$ \H^p\br{M} \cong \H^p\br{N}, \qquad p \ge 0, $$
that is $ \H^p\br{M} $ is a diffeomorphic invariant.
\end{corollary}

\begin{proof}
By Proposition \ref{prop:2.7} there exists $ F^* : \H^p\br{N} \to \H^p\br{M} $ and $ \br{F^{-1}}^* : \H^p\br{M} \to \H^p\br{N} $. By Proposition \ref{prop:1.8},
$$ \br{F^{-1}}^*F^*\omega = \br{F \circ F^{-1}}^*\omega = \id_N^*\omega = \omega, \qquad \omega \in \H^p\br{N} $$
so $ \br{F^{-1}}^* \circ F^* = \id_{\H^p\br{N}} $. Similarly $ F^* \circ \br{F^{-1}}^* = \id_{\H^p\br{M}} $, so $ F^* $ is an isomorphism.
\end{proof}

\subsection{Homotopy invariance}

\begin{definition}
Let $ M_0 $ and $ M_1 $ be manifolds, and let $ f_0, f_1 : M_0 \to M_1 $ be smooth morphisms. Then $ f_0 $ and $ f_1 $ are \textbf{smoothly homotopic equivalent} if there exists a smooth morphism $ H : M_0 \times \sbr{0, 1} \to M_1 $ such that $ H\br{x, 0} = f_0\br{x} $ and $ H\br{x, 1} = f_1\br{x} $ for all $ x \in M_0 $. A \textbf{homotopy} is a smooth morphism $ H : M_0 \times \sbr{0, 1} \to M_1 $ where $ M_0 $ and $ M_1 $ are smooth manifolds.
\end{definition}

\lecture{9}{Tuesday}{28/01/20}

\begin{notation}
Let $ f_t\br{x} = H\br{x, t} $, so $ f_t : M_0 \to M_1 $ is a smooth morphism. Then $ f_0 $ and $ f_1 $ are said to be homotopic equivalent, denoted by $ f_0 \sim f_1 $. Then $ \sim $ is an equivalence. \footnote{Exercise}
\end{notation}

\begin{definition}
$ M_0 $ and $ M_1 $ are \textbf{homotopy equivalent} if there exist smooth morphisms $ f : M_0 \to M_1 $ and $ g : M_1 \to M_0 $ such that $ f \circ g \sim \id_{M_1} $ and $ g \circ f \sim \id_{M_0} $.
\end{definition}

\begin{example}
\label{eg:2.12}
\hfill
\begin{itemize}
\item Let $ M_0 = \RR^n $ and $ M_1 = \cbr{0} $. Then $ M_0 $ and $ M_1 $ are homotopy equivalent. Let
$$ \function[f]{M_0}{M_1}{x}{0}, \qquad \function[g]{M_1}{M_0}{0}{0}. $$
Then
$$ \function[f \circ g]{M_1}{M_1}{0}{0}, $$
so $ f \circ g = \id_{M_1} $, and
$$ \function[g \circ f]{M_0}{M_0}{x}{0}. $$
We want to show that $ g \circ f \sim \id_{M_0} $. Define a smooth morphism
$$ \function[H]{M_0 \times \sbr{0, 1}}{M_0}{\br{x, t}}{tx}. $$
Then $ H\br{x, 0} = 0 = \br{g \circ f}\br{x} $ for all $ x $, and $ H\br{x, 1} = x = \id_{M_0}\br{x} $ for all $ x $, so $ g \circ f \sim \id_{M_0} $. More in general $ M \subset \RR^n $ is called \textbf{convex} if for all $ x, y \in M $ the segment joining $ x $ to $ y $ is contained inside $ M $. If $ M $ is convex then $ M $ is homotopy equivalent to $ M \times \cbr{0} $.

\pagebreak

\item Let $ M_0 = \RR^2 \setminus \cbr{0} $ and $ M_1 = \S^1 $. Then $ M_0 $ and $ M_1 $ are homotopy equivalent. Let
$$ \function[f]{M_0}{M_1}{x}{\dfrac{x}{\abs{x}}}, \qquad \function[g]{M_1}{M_0}{x}{x}. $$
Then
$$ \function[f \circ g]{M_1}{M_1}{x}{x}, $$
so $ f \circ g = \id_{M_1} $, and
$$ \function[g \circ f]{M_0}{M_0}{x}{\dfrac{x}{\abs{x}}}. $$
Let
$$ \function[H]{M_0 \times \sbr{0, 1}}{M_0}{\br{x, t}}{tx + \br{1 - t}\dfrac{x}{\abs{x}}} $$
be smooth. Then $ H\br{x, 0} = x / \abs{x} = \br{g \circ f}\br{x} $ and $ H\br{x, 1} = x = \id_{M_0}\br{x} $, so $ g \circ f \sim \id_{M_0} $.
\end{itemize}
\end{example}

\begin{proposition}
\label{prop:2.13}
Let $ M $ and $ N $ be manifolds, and let $ H : M \times \sbr{0, 1} \to N $ be smooth. Denote
$$ \function[f_t]{M}{N}{x}{H\br{x, t}}, \qquad t \in \sbr{0, 1}. $$
Then $ f_t^* : \H^p\br{N} \to \H^p\br{M} $ does not depend on $ t $ for all $ p \ge 0 $.
\end{proposition}

\begin{proof}
Let $ t_1, t_2 \in \sbr{0, 1} $ such that $ t_1 < t_2 $. The goal is $ f_{t_1}^*\sbr{\eta} = f_{t_2}^*\sbr{\eta} $ for all $ \sbr{\eta} \in \H^p\br{N} $. Let
$$ \function[i_k]{M}{M \times \sbr{0, 1}}{x}{\br{x, t_k}}, \qquad k = 1, 2. $$
Claim that for all $ p $ there exists a linear map $ h : \Omega^p\br{M \times \sbr{t_1, t_2}} \to \Omega^{p - 1}\br{M} $ such that
\begin{equation}
\label{eq:2}
\d\br{h\br{\omega}} + h\br{\d\omega} = i_2^*\omega - i_1^*\omega \in \Omega^p\br{M}, \qquad \omega \in \Omega^p\br{M \times \sbr{0, 1}}.
\end{equation}
\begin{enumerate}[label=Step \arabic*., leftmargin=0.5in]
\item The claim implies the proposition. Let $ \eta \in \Omega^p\br{N} $ be closed, so $ \d\eta = 0 $. Then $ H^*\eta $ is also closed, so let $ \omega = H^*\eta \in \Omega^p\br{M \times \sbr{t_1, t_2}} $. Apply $ h $. Then $ \d\omega = 0 $, so $ \d\br{h\br{\omega}} = i_2^*\omega - i_1^*\omega $ is exact. Thus
$$ f_{t_1}^*\sbr{\eta} = \sbr{f_{t_1}^*\eta} = \sbr{i_1^*H^*\eta} = \sbr{i_1^*\omega} = \sbr{i_2^*\omega} = \sbr{i_2^*H^*\eta} = \sbr{f_{t_2}^*\eta} = f_{t_2}^*\sbr{\eta}, $$
so the proposition follows.

\lecture{10}{Thursday}{30/01/20}

\item The proof of the claim. Let $ \omega \in \Omega^p\br{M \times \sbr{t_1, t_2}} $. Then for all $ \br{x, t} \in M \times \sbr{t_1, t_2} $, $ \omega\br{x, t} $ is an alternating $ p $-form on $ \T_{\br{x, t}}\br{M \times \sbr{t_1, t_2}} $. We want an alternating $ \br{p - 1} $-form $ h\br{\omega}\br{x} $ on $ \T_xM $. Let $ v_1, \dots, v_{p - 1} \in \T_xM $. Then
$$ h\br{\omega}\br{x}\br{v_1, \dots, v_{p - 1}} = \intd{t_1}{t_2}{\omega\br{x, t}\br{\dpd{}{t}, v_1, \dots, v_{p - 1}}}{t} $$
is a $ \br{p - 1} $-form on $ M $, and $ \tpd{}{t} $ is a global vector field. Check $ h $ is linear. \footnote{Exercise} It is enough to prove $ \br{\ref{eq:2}} $ locally. Remark that exactness is not a local property. Fix local coordinates $ \br{x_1, \dots, x_n, t} $ around a point of $ M \times \sbr{0, 1} $. Then
$$ \omega = \sum_{\abs{I} = p} \omega_I + \sum_{\abs{J} = p - 1} \omega_J, \qquad \omega_I = g_I \d x_{i_1} \wedge \dots \wedge \d x_{i_p}, \qquad \omega_J = g_J \d x_{j_1} \wedge \dots \wedge \d x_{j_{p - 1}} \wedge \d t, $$
where $ g_I $ and $ g_J $ are smooth functions. Any term of $ \br{\ref{eq:2}} $ is linear. So we just need to check $ \br{\ref{eq:2}} $ holds for $ \omega_I $ and $ \omega_J $.

\pagebreak

\begin{itemize}
\item[$ \omega_I $.] Let $ \omega = g\br{x, t} \d x_{i_1} \wedge \dots \wedge \d x_{i_p} $. Then
$$ \d\br{h\br{\omega\br{x, t}\br{\dpd{}{t}, v_1, \dots, v_{p - 1}}}} = \d\br{h\br{0}} = 0, $$
and
\begin{align*}
h\br{\d\omega}
& = h\br{\dpd{}{t}g\br{x, t} \d t \wedge \d x_{i_1} \wedge \dots \wedge \d x_{i_p} + \sum_{j = 1}^n \dpd{}{x_j}g\br{x, t} \d x_j \wedge \d x_{i_1} \wedge \dots \wedge \d x_{i_p}} \\
& = \br{\intd{t_1}{t_2}{\dpd{}{t}g\br{x, t}}{t}} \d x_{i_1} \wedge \dots \wedge \d x_{i_p} + 0 \\
& = \br{g\br{x, t_2} - g\br{x, t_1}} \d x_{i_1} \wedge \dots \wedge \d x_{i_p}
= i_2^*\omega - i_1^*\omega,
\end{align*}
so $ \br{\ref{eq:2}} $ holds.
\item[$ \omega_J $.] Let $ \omega = g\br{x, t} \d x_{j_1} \wedge \dots \wedge \d x_{j_{p - 1}} \wedge \d t $. Then
\begin{align*}
\d\br{h\br{\omega}}
& = \br{-1}^{p - 1}\d\br{\br{\intd{t_1}{t_2}{g\br{x, t}}{t}} \d x_{j_1} \wedge \dots \wedge \d x_{j_{p - 1}}} \\
& = \br{-1}^{p - 1}\sum_{j = 1}^n \dpd{}{x_j}\br{\intd{t_1}{t_2}{g\br{x, t}}{t}} \d x_j \wedge \d x_{j_1} \wedge \dots \wedge \d x_{j_{p - 1}} \\
& = \br{-1}^{p - 1}\sum_{j = 1}^n \br{\intd{t_1}{t_2}{\dpd{}{x_j}g\br{x, t}}{t}} \d x_j \wedge \d x_{j_1} \wedge \dots \wedge \d x_{j_{p - 1}},
\end{align*}
and
\begin{align*}
h\br{\d\omega}
& = h\br{\sum_{j = 1}^n \dpd{}{x_j}g\br{x, t} \d x_j \wedge \d x_{j_1} \wedge \dots \wedge \d x_{j_{p - 1}} \wedge \d t + 0} \\
& = \br{-1}^{p - 1}\sum_{j = 1}^n \br{\intd{t_1}{t_2}{\dpd{}{x_j}g\br{x, t}}{t}} \d x_j \wedge \d x_{j_1} \wedge \dots \wedge \d x_{j_{p - 1}}
= -\d\br{h\br{\omega}},
\end{align*}
and $ i_2^*\omega = i_1^*\omega = 0 $, so $ \br{\ref{eq:2}} $ holds.
\end{itemize}
\end{enumerate}
\end{proof}

\begin{corollary}
\label{cor:2.14}
Assume $ M $ and $ N $ are homotopy equivalent. Then there exist isomorphisms
$$ \H^p\br{N} \to \H^p\br{M}, \qquad p \ge 0. $$
\end{corollary}

\begin{proof}
There exist $ f : M \to N $ and $ g : N \to M $ such that $ g \circ f \sim \id_M $ and $ f \circ g \sim \id_N $. By Proposition \ref{prop:2.13} $ \br{g \circ f}^* : \H^p\br{M} \to \H^p\br{M} $ coincides with $ \id_M^* = \id_{\H^p\br{M}} $. Then $ f^* \circ g^* = \br{g \circ f}^* = \id_{\H^p\br{M}} $. Similarly $ g^* \circ f^* = \id_{\H^p\br{N}} $, so $ g^* $ and $ f^* $ are isomorphisms.
\end{proof}

\begin{definition}
Let $ M $ be a manifold. Then $ M $ is \textbf{smoothly contractible} if $ M $ is homotopy equivalent to a point.
\end{definition}

\begin{example*}
$ \RR^n $ is contractible, by Example \ref{eg:2.12}. If $ M \subset \RR^n $ is convex then $ M $ is contractible.
\end{example*}

\begin{theorem}[Poincar\'e lemma]
If $ M $ is a contractible manifold then
$$ \H^p\br{M} = 0, \qquad p \ge 1. $$
\end{theorem}

\begin{proof}
By previous Corollary \ref{cor:2.14}, there exists an isomorphism $ \H^p\br{M} \to \H^p\br{\cbr{\text{point}}} $. Then $ \cbr{\text{point}} $ is a zero-dimensional manifold, so by Proposition \ref{prop:2.5}, $ \H^p\br{\cbr{\text{point}}} = 0 $ for all $ p > 0 $.
\end{proof}

Thus $ \H^p\br{\RR^n} = 0 $ for all $ p > 0 $, so $ \RR^n $ is not diffeomorphic to any compact orientable manifold.

\lecture{11}{Monday}{03/02/20}

\begin{proposition}
Let $ M $ be a manifold, and let $ \omega \in \Omega^p\br{M} $ be a closed $ p $-form for $ p > 0 $. Then for all $ x \in X $, there exists a neighbourhood $ U \ni x $ such that $ \omega $ is exact on $ U $, that is there exists $ \eta \in \Omega^{p - 1}\br{U} $ such that $ \omega = \d\eta $ on $ U $.
\end{proposition}

\begin{proof}
Let $ \br{U, \phi} $ be a chart around $ x $. I may assume that $ V = \phi\br{U} $ is a ball in $ \RR^n $. Then $ U $ is diffeomorphic to $ B = \cbr{z \st \abs{z - z_0} < r} $ for some $ z_0 \in \RR^n $ and $ r > 0 $, so $ \H^p\br{U} \cong \H^p\br{B} $ for all $ p \ge 0 $. Since $ B $ is contractible, $ \H^p\br{B} = 0 $ for all $ p > 0 $. The restriction of $ \omega $ on $ U $ gives a class $ \sbr{\omega} \in \H^p\br{U} = 0 $, so $ \omega $ is cohomologous to zero on $ U $. Thus $ \omega $ is exact on $ U $.
\end{proof}

\begin{definition}
Let $ M $ be a manifold, let $ \gamma : \sbr{0, 1} \to M $ be a continuous or smooth path, and let $ x = \gamma\br{0} $ and $ y = \gamma\br{1} $. A \textbf{homotopy of paths} from $ x $ to $ y $ is a map
$$ \functions[F]{\sbr{0, 1} \times \sbr{0, 1}}{M}{\br{0, t}}{x}{\br{1, t}}{y}. $$
\end{definition}

\begin{proposition}
\label{prop:2.19}
Let $ \gamma_0 $ and $ \gamma_1 $ be homotopic paths on a manifold $ M $, and let $ \omega \in \Omega^1\br{M} $ be closed. Then
$$ \int_0^1 \gamma_0^*\omega = \int_{\gamma_0} \omega = \int_{\gamma_1} \omega. $$
\end{proposition}

\begin{proof}
Lee's introduction to smooth manifolds. The idea is
$$ \int_{\gamma_0} \omega - \int_{\gamma_1} \omega = \int_{\gamma_0 \cup \gamma_1} \omega = \int_{\Im F} \omega = 0, $$
by Stokes' theorem.
\end{proof}

Recall that $ M $ is \textbf{simply connected}, so $ \pi_1\br{M} = 0 $, if any path $ \gamma $ from $ x $ to $ x $ is homotopic equivalent to a point.

\begin{proposition}
Let $ M $ be a simply connected orientable manifold. Then
$$ \H^1\br{M} = 0. $$
\end{proposition}

\begin{proof}
Let $ \omega \in \Omega^1\br{M} $ be a closed form. Then claim that $ \omega $ is exact if and only if $ \int_\gamma \omega = 0 $ for all loops $ \gamma $, that is paths from $ x $ to $ x $.
\begin{itemize}
\item The proof of the claim. Assume that $ \omega = \d f $ is exact for $ f \in \Omega^0\br{M} $. By Proposition \ref{prop:2.19},
$$ \int_\gamma \omega = \int_{\text{trivial loop}} \omega = 0. $$
Assume that $ \int_\gamma \omega = 0 $ for all loops $ \gamma $. Fix $ x $. Let
$$ f\br{y} = \int_x^y \omega. $$
Since $ \int_{\gamma_1 \cup \gamma_2} \omega = 0 $, $ f $ is well-defined, that is it does not depend on the choice of the path. Then $ \d f = \omega $. This can be checked locally, that is in an open set of $ \RR^n $. Here it follows from the fundamental theorem of calculus.
\item The claim implies the proposition. Being simply connected, any loop inside $ M $ is homotopic equivalent to the trivial loop. For all loops $ \gamma $ and for all closed $ \omega $, $ \int_\gamma \omega = 0 $ by Proposition \ref{prop:2.19}, so $ \omega $ is exact. Thus $ \sbr{\omega} = 0 $ in $ \H^1\br{M} $.
\end{itemize}
\end{proof}

\pagebreak

\subsection{Homological algebra}

Let $ C^\bullet $ be a sequence of vector spaces, that is $ C^k $ is a vector space for $ k \in \ZZ $.

\begin{definition}
$ \br{C^\bullet, d^\bullet} $ is a \textbf{cochain complex} if $ C^\bullet $ is a sequence of vector spaces and $ d^\bullet $ is a sequence of linear maps $ d^k : C^k \to C^{k + 1} $ such that the composition $ d^{k + 1} \circ d^k : C^k \to C^{k + 1} \to C^{k + 2} $ is zero for all $ k $.
\end{definition}

\begin{definition}
The elements of
$$ \ZZZ^k\br{C^\bullet, d^\bullet} = \Ker \br{d^k : C^k \to C^{k + 1}} \subset C^k $$
are called \textbf{cocycles}. The elements of
$$ \BBB^k\br{C^\bullet, d^\bullet} = \Im \br{d^k : C^{k - 1} \to C^k} \subset C^k $$
are called \textbf{coboundaries}. Then $ d^{k - 1} \circ d^k = 0 $, so $ \BBB^k \subset \ZZZ^k $. The quotients
$$ \H^k\br{C^\bullet, d^\bullet} = \ZZZ^k\br{C^\bullet, d^\bullet} / \BBB^k\br{C^\bullet, d^\bullet} $$
are the \textbf{$ k $-th cohomology groups} of $ \br{C^\bullet, d^\bullet} $.
\end{definition}

\begin{definition}
Let $ \br{C^\bullet, d^\bullet} $ and $ \br{D^\bullet, d^\bullet} $ be two cochain complexes. A map $ f : \br{C^\bullet, d^\bullet} \to \br{D^\bullet, d^\bullet} $ is a sequence of linear maps $ f^k : C^k \to D^k $ such that $ f^{k + 1} \circ d^k = d^k \circ f^k $ for all $ k $, so
$$
\begin{tikzcd}
\dots \arrow{r} & C^k \arrow{r}{d^k} \arrow{d}{f^k} & C^{k + 1} \arrow{r}{d^{k + 1}} \arrow{d}{f^{k + 1}} & C^{k + 2} \arrow{r} \arrow{d}{f^{k + 2}} & \dots \\
\dots \arrow{r} & D^k \arrow{r}{d^k} & D^{k + 1} \arrow{r}{d^{k + 1}} & D^{k + 2} \arrow{r} & \dots
\end{tikzcd}.
$$
\end{definition}

\begin{proposition}
Let $ f : \br{C^\bullet, d^\bullet} \to \br{D^\bullet, d^\bullet} $ be a map between cochain complexes. Then there exists a natural induced map $ f^k : \H^k\br{C^\bullet, d^\bullet} \to \H^k\br{D^\bullet, d^\bullet} $.
\end{proposition}

\begin{proof}
Let $ \sbr{\omega} \in \H^k\br{C^\bullet, d^\bullet} = \ZZZ^k\br{C^\bullet, d^\bullet} / \BBB^k\br{C^\bullet, d^\bullet} $ for $ \omega \in \ZZZ^k\br{C^\bullet, d^\bullet} $, that is $ d^k\br{\omega} = 0 $. I want to check that $ f^k\br{\omega} \in \ZZZ^k\br{D^\bullet, d^\bullet} $. By definition of maps, $ d^k\br{f^k\br{\omega}} = f^{k + 1}\br{d^k\br{\omega}} = 0 $, so $ \ZZZ^k\br{C^\bullet, d^\bullet} \to \ZZZ^k\br{D^\bullet, d^\bullet} $. Now I need to check that if $ \omega \in \BBB^k\br{C^\bullet, d^\bullet} $ then $ f^k\br{\omega} \in \BBB^k\br{D^\bullet, d^\bullet} $. \footnote{Exercise}
\end{proof}

\end{document}