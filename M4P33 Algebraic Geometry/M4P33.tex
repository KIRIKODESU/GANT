\def\module{M4P33 Algebraic Geometry}
\def\lecturer{Dr Genival Da Silva Jr}
\def\term{Spring 2019}

\def\thm{section}

\documentclass{article}

% Packages

\usepackage{amssymb}
\usepackage{amsthm}
\usepackage[UKenglish]{babel}
\usepackage{commath}
\usepackage{enumitem}
\usepackage{etoolbox}
\usepackage{fancyhdr}
\usepackage[margin=1in]{geometry}
\usepackage{graphicx}
\usepackage[hidelinks]{hyperref}
\usepackage[utf8]{inputenc}
\usepackage{listings}
\usepackage{mathtools}
\usepackage{stmaryrd}
\usepackage{tikz-cd}
\usepackage{csquotes}

% Formatting

\addto\captionsUKenglish{\renewcommand{\abstractname}{Syllabus}}
\delimitershortfall5pt
\ifx\thm\undefined\newtheorem{n}{}\else\newtheorem{n}{}[\thm]\fi
\newcommand\newoperator[1]{\ifcsdef{#1}{\cslet{#1}{\relax}}{}\csdef{#1}{\operatorname{#1}}}
\setlength{\parindent}{0cm}

% Environments

\theoremstyle{plain}
\newtheorem{algorithm}[n]{Algorithm}
\newtheorem*{algorithm*}{Algorithm}
\newtheorem{algorithm**}{Algorithm}
\newtheorem{conjecture}[n]{Conjecture}
\newtheorem*{conjecture*}{Conjecture}
\newtheorem{conjecture**}{Conjecture}
\newtheorem{corollary}[n]{Corollary}
\newtheorem*{corollary*}{Corollary}
\newtheorem{corollary**}{Corollary}
\newtheorem{lemma}[n]{Lemma}
\newtheorem*{lemma*}{Lemma}
\newtheorem{lemma**}{Lemma}
\newtheorem{proposition}[n]{Proposition}
\newtheorem*{proposition*}{Proposition}
\newtheorem{proposition**}{Proposition}
\newtheorem{theorem}[n]{Theorem}
\newtheorem*{theorem*}{Theorem}
\newtheorem{theorem**}{Theorem}

\theoremstyle{definition}
\newtheorem{aim}[n]{Aim}
\newtheorem*{aim*}{Aim}
\newtheorem{aim**}{Aim}
\newtheorem{axiom}[n]{Axiom}
\newtheorem*{axiom*}{Axiom}
\newtheorem{axiom**}{Axiom}
\newtheorem{condition}[n]{Condition}
\newtheorem*{condition*}{Condition}
\newtheorem{condition**}{Condition}
\newtheorem{definition}[n]{Definition}
\newtheorem*{definition*}{Definition}
\newtheorem{definition**}{Definition}
\newtheorem{example}[n]{Example}
\newtheorem*{example*}{Example}
\newtheorem{example**}{Example}
\newtheorem{exercise}[n]{Exercise}
\newtheorem*{exercise*}{Exercise}
\newtheorem{exercise**}{Exercise}
\newtheorem{fact}[n]{Fact}
\newtheorem*{fact*}{Fact}
\newtheorem{fact**}{Fact}
\newtheorem{goal}[n]{Goal}
\newtheorem*{goal*}{Goal}
\newtheorem{goal**}{Goal}
\newtheorem{law}[n]{Law}
\newtheorem*{law*}{Law}
\newtheorem{law**}{Law}
\newtheorem{plan}[n]{Plan}
\newtheorem*{plan*}{Plan}
\newtheorem{plan**}{Plan}
\newtheorem{problem}[n]{Problem}
\newtheorem*{problem*}{Problem}
\newtheorem{problem**}{Problem}
\newtheorem{question}[n]{Question}
\newtheorem*{question*}{Question}
\newtheorem{question**}{Question}
\newtheorem{warning}[n]{Warning}
\newtheorem*{warning*}{Warning}
\newtheorem{warning**}{Warning}
\newtheorem{acknowledgements}[n]{Acknowledgements}
\newtheorem*{acknowledgements*}{Acknowledgements}
\newtheorem{acknowledgements**}{Acknowledgements}
\newtheorem{annotations}[n]{Annotations}
\newtheorem*{annotations*}{Annotations}
\newtheorem{annotations**}{Annotations}
\newtheorem{assumption}[n]{Assumption}
\newtheorem*{assumption*}{Assumption}
\newtheorem{assumption**}{Assumption}
\newtheorem{conclusion}[n]{Conclusion}
\newtheorem*{conclusion*}{Conclusion}
\newtheorem{conclusion**}{Conclusion}
\newtheorem{claim}[n]{Claim}
\newtheorem*{claim*}{Claim}
\newtheorem{claim**}{Claim}
\newtheorem{notation}[n]{Notation}
\newtheorem*{notation*}{Notation}
\newtheorem{notation**}{Notation}
\newtheorem{note}[n]{Note}
\newtheorem*{note*}{Note}
\newtheorem{note**}{Note}
\newtheorem{remark}[n]{Remark}
\newtheorem*{remark*}{Remark}
\newtheorem{remark**}{Remark}

% Lectures

\newcommand{\lecture}[3]{ % Lecture
  \marginpar{
    Lecture #1 \\
    #2 \\
    #3
  }
}

% Blackboard

\renewcommand{\AA}{\mathbb{A}} % Blackboard A
\newcommand{\BB}{\mathbb{B}}   % Blackboard B
\newcommand{\CC}{\mathbb{C}}   % Blackboard C
\newcommand{\DD}{\mathbb{D}}   % Blackboard D
\newcommand{\EE}{\mathbb{E}}   % Blackboard E
\newcommand{\FF}{\mathbb{F}}   % Blackboard F
\newcommand{\GG}{\mathbb{G}}   % Blackboard G
\newcommand{\HH}{\mathbb{H}}   % Blackboard H
\newcommand{\II}{\mathbb{I}}   % Blackboard I
\newcommand{\JJ}{\mathbb{J}}   % Blackboard J
\newcommand{\KK}{\mathbb{K}}   % Blackboard K
\newcommand{\LL}{\mathbb{L}}   % Blackboard L
\newcommand{\MM}{\mathbb{M}}   % Blackboard M
\newcommand{\NN}{\mathbb{N}}   % Blackboard N
\newcommand{\OO}{\mathbb{O}}   % Blackboard O
\newcommand{\PP}{\mathbb{P}}   % Blackboard P
\newcommand{\QQ}{\mathbb{Q}}   % Blackboard Q
\newcommand{\RR}{\mathbb{R}}   % Blackboard R
\renewcommand{\SS}{\mathbb{S}} % Blackboard S
\newcommand{\TT}{\mathbb{T}}   % Blackboard T
\newcommand{\UU}{\mathbb{U}}   % Blackboard U
\newcommand{\VV}{\mathbb{V}}   % Blackboard V
\newcommand{\WW}{\mathbb{W}}   % Blackboard W
\newcommand{\XX}{\mathbb{X}}   % Blackboard X
\newcommand{\YY}{\mathbb{Y}}   % Blackboard Y
\newcommand{\ZZ}{\mathbb{Z}}   % Blackboard Z

% Brackets

\renewcommand{\eval}[1]{\left. #1 \right|}          % Evaluation
\newcommand{\br}{\del}                              % Brackets
\newcommand{\abr}[1]{\left\langle #1 \right\rangle} % Angle brackets
\newcommand{\fbr}[1]{\left\lfloor #1 \right\rfloor} % Floor brackets
\newcommand{\lbr}[1]{\left\lfloor #1 \right\rfloor} % Ceiling brackets
\newcommand{\st}{\ \middle| \ }                     % Such that

% Calligraphic

\newcommand{\AAA}{\mathcal{A}} % Calligraphic A
\newcommand{\BBB}{\mathcal{B}} % Calligraphic B
\newcommand{\CCC}{\mathcal{C}} % Calligraphic C
\newcommand{\DDD}{\mathcal{D}} % Calligraphic D
\newcommand{\EEE}{\mathcal{E}} % Calligraphic E
\newcommand{\FFF}{\mathcal{F}} % Calligraphic F
\newcommand{\GGG}{\mathcal{G}} % Calligraphic G
\newcommand{\HHH}{\mathcal{H}} % Calligraphic H
\newcommand{\III}{\mathcal{I}} % Calligraphic I
\newcommand{\JJJ}{\mathcal{J}} % Calligraphic J
\newcommand{\KKK}{\mathcal{K}} % Calligraphic K
\newcommand{\LLL}{\mathcal{L}} % Calligraphic L
\newcommand{\MMM}{\mathcal{M}} % Calligraphic M
\newcommand{\NNN}{\mathcal{N}} % Calligraphic N
\newcommand{\OOO}{\mathcal{O}} % Calligraphic O
\newcommand{\PPP}{\mathcal{P}} % Calligraphic P
\newcommand{\QQQ}{\mathcal{Q}} % Calligraphic Q
\newcommand{\RRR}{\mathcal{R}} % Calligraphic R
\newcommand{\SSS}{\mathcal{S}} % Calligraphic S
\newcommand{\TTT}{\mathcal{T}} % Calligraphic T
\newcommand{\UUU}{\mathcal{U}} % Calligraphic U
\newcommand{\VVV}{\mathcal{V}} % Calligraphic V
\newcommand{\WWW}{\mathcal{W}} % Calligraphic W
\newcommand{\XXX}{\mathcal{X}} % Calligraphic X
\newcommand{\YYY}{\mathcal{Y}} % Calligraphic Y
\newcommand{\ZZZ}{\mathcal{Z}} % Calligraphic Z

% Fraktur

\newcommand{\aaa}{\mathfrak{a}}   % Fraktur a
\newcommand{\bbb}{\mathfrak{b}}   % Fraktur b
\newcommand{\ccc}{\mathfrak{c}}   % Fraktur c
\newcommand{\ddd}{\mathfrak{d}}   % Fraktur d
\newcommand{\eee}{\mathfrak{e}}   % Fraktur e
\newcommand{\fff}{\mathfrak{f}}   % Fraktur f
\renewcommand{\ggg}{\mathfrak{g}} % Fraktur g
\newcommand{\hhh}{\mathfrak{h}}   % Fraktur h
\newcommand{\iii}{\mathfrak{i}}   % Fraktur i
\newcommand{\jjj}{\mathfrak{j}}   % Fraktur j
\newcommand{\kkk}{\mathfrak{k}}   % Fraktur k
\renewcommand{\lll}{\mathfrak{l}} % Fraktur l
\newcommand{\mmm}{\mathfrak{m}}   % Fraktur m
\newcommand{\nnn}{\mathfrak{n}}   % Fraktur n
\newcommand{\ooo}{\mathfrak{o}}   % Fraktur o
\newcommand{\ppp}{\mathfrak{p}}   % Fraktur p
\newcommand{\qqq}{\mathfrak{q}}   % Fraktur q
\newcommand{\rrr}{\mathfrak{r}}   % Fraktur r
\newcommand{\sss}{\mathfrak{s}}   % Fraktur s
\newcommand{\ttt}{\mathfrak{t}}   % Fraktur t
\newcommand{\uuu}{\mathfrak{u}}   % Fraktur u
\newcommand{\vvv}{\mathfrak{v}}   % Fraktur v
\newcommand{\www}{\mathfrak{w}}   % Fraktur w
\newcommand{\xxx}{\mathfrak{x}}   % Fraktur x
\newcommand{\yyy}{\mathfrak{y}}   % Fraktur y
\newcommand{\zzz}{\mathfrak{z}}   % Fraktur z

% Geometry

\newcommand{\CP}{\mathbb{CP}}                                              % Complex projective space
\newcommand{\iintd}[4]{\iint_{#1} \, #2 \, \dif #3 \, \dif #4}             % Double integral
\newcommand{\RP}{\mathbb{RP}}                                              % Real projective space
\newcommand{\intd}[4]{\int_{#1}^{#2} \, #3 \, \dif #4}                     % Single integral
\newcommand{\iiintd}[5]{\iint_{#1} \, #2 \, \dif #3 \, \dif #4 \, \dif #5} % Triple integral

% Logic

\newcommand{\iffb}[2]{\br{#1 \leftrightarrow #2}} % Biconditional
\newcommand{\andb}[2]{\br{#1 \land #2}}           % Conjunction
\newcommand{\orb}[2]{\br{#1 \lor #2}}             % Disjunction
\newcommand{\nib}[2]{\br{#1 \notin #2}}           % Element of
\newcommand{\eqb}[2]{\br{#1 = #2}}                % Equal to
\newcommand{\teb}[1]{\br{\exists #1}}             % Existential quantifier
\newcommand{\impb}[2]{\br{#1 \rightarrow #2}}     % Implication
\newcommand{\ltb}[2]{\br{#1 < #2}}                % Less than
\newcommand{\leb}[2]{\br{#1 \le #2}}              % Less than or equal to
\newcommand{\notb}[1]{\br{\neg #1}}               % Negation
\newcommand{\inb}[2]{\br{#1 \in #2}}              % Not element of
\newcommand{\neb}[2]{\br{#1 \ne #2}}              % Not equal to
\newcommand{\subb}[2]{\br{#1 \subseteq #2}}       % Subset
\newcommand{\fab}[1]{\br{\forall #1}}             % Universal quantifier

% Maps

\newcommand{\bijection}[7][]{    % Bijection
  \ifx &#1&
    \begin{array}{rcl}
      #2 & \longleftrightarrow & #3 \\
      #4 & \longmapsto         & #5 \\
      #6 & \longmapsfrom       & #7
    \end{array}
  \else
    \begin{array}{ccrcl}
      #1 & : & #2 & \longrightarrow & #3 \\
         &   & #4 & \longmapsto     & #5 \\
         &   & #6 & \longmapsfrom   & #7
    \end{array}
  \fi
}
\newcommand{\birational}[7][]{   % Birational map
  \ifx &#1&
    \begin{array}{rcl}
      #2 & \dashrightarrow & #3 \\
      #4 & \longmapsto     & #5 \\
      #6 & \longmapsfrom   & #7
    \end{array}
  \else
    \begin{array}{ccrcl}
      #1 & : & #2 & \dashrightarrow & #3 \\
         &   & #4 & \longmapsto     & #5 \\
         &   & #6 & \longmapsfrom   & #7
    \end{array}
  \fi
}
\newcommand{\correspondence}[2]{ % Correspondence
  \cbr{
    \begin{array}{c}
      #1
    \end{array}
  }
  \qquad
  \leftrightsquigarrow
  \qquad
  \cbr{
    \begin{array}{c}
      #2
    \end{array}
  }
}
\newcommand{\function}[5][]{     % Function
  \ifx &#1&
    \begin{array}{rcl}
      #2 & \longrightarrow & #3 \\
      #4 & \longmapsto     & #5
    \end{array}
  \else
    \begin{array}{ccrcl}
      #1 & : & #2 & \longrightarrow & #3 \\
         &   & #4 & \longmapsto     & #5
    \end{array}
  \fi
}
\newcommand{\functions}[7][]{    % Functions
  \ifx &#1&
    \begin{array}{rcl}
      #2 & \longrightarrow & #3 \\
      #4 & \longmapsto     & #5 \\
      #6 & \longmapsto     & #7
    \end{array}
  \else
    \begin{array}{ccrcl}
      #1 & : & #2 & \longrightarrow & #3 \\
         &   & #4 & \longmapsto     & #5 \\
         &   & #6 & \longmapsto     & #7
    \end{array}
  \fi
}
\newcommand{\rational}[5][]{     % Rational map
  \ifx &#1&
    \begin{array}{rcl}
      #2 & \dashrightarrow & #3 \\
      #4 & \longmapsto     & #5
    \end{array}
  \else
    \begin{array}{ccrcl}
      #1 & : & #2 & \dashrightarrow & #3 \\
         &   & #4 & \longmapsto     & #5
    \end{array}
  \fi
}

% Matrices

\newcommand{\onebytwo}[2]{      % One by two matrix
  \begin{pmatrix}
    #1 & #2
  \end{pmatrix}
}
\newcommand{\onebythree}[3]{    % One by three matrix
  \begin{pmatrix}
    #1 & #2 & #3
  \end{pmatrix}
}
\newcommand{\twobyone}[2]{      % Two by one matrix
  \begin{pmatrix}
    #1 \\
    #2
  \end{pmatrix}
}
\newcommand{\twobytwo}[4]{      % Two by two matrix
  \begin{pmatrix}
    #1 & #2 \\
    #3 & #4
  \end{pmatrix}
}
\newcommand{\threebyone}[3]{    % Three by one matrix
  \begin{pmatrix}
    #1 \\
    #2 \\
    #3
  \end{pmatrix}
}
\newcommand{\threebythree}[9]{  % Three by three matrix
  \begin{pmatrix}
    #1 & #2 & #3 \\
    #4 & #5 & #6 \\
    #7 & #8 & #9
  \end{pmatrix}
}
\newcommand{\twobytwosmall}[4]{ % Two by two small matrix
  \begin{psmallmatrix}
    #1 & #2 \\
    #3 & #4
  \end{psmallmatrix}
}

% Number theory

\renewcommand{\symbol}[2]{\br{\tfrac{#1}{#2}}} % Power residue symbol
\newcommand{\unit}[1]{\br{\ZZ / #1\ZZ}^\times} % Unit group

% Operators

\newoperator{ab}    % Abelian
\newoperator{AG}    % Affine geometry
\newoperator{alg}   % Algebraic
\newoperator{Ann}   % Annihilator
\newoperator{area}  % Area
\newoperator{Aut}   % Automorphism
\newoperator{card}  % Cardinality
\newoperator{ch}    % Characteristic
\newoperator{Cl}    % Class
\newoperator{Coker} % Cokernel
\newoperator{col}   % Column
\newoperator{Corr}  % Correspondence
\newoperator{diam}  % Diameter
\newoperator{Disc}  % Discriminant
\newoperator{dom}   % Domain
\newoperator{Eig}   % Eigenvalue
\newoperator{Em}    % Embedding
\newoperator{End}   % Endomorphism
\newoperator{fin}   % Finite
\newoperator{Fix}   % Fixed
\newoperator{Frac}  % Fraction
\newoperator{Frob}  % Frobenius
\newoperator{Fun}   % Function
\newoperator{Gal}   % Galois
\newoperator{GL}    % General linear
\newoperator{Ham}   % Hamming
\newoperator{Homeo} % Homeomorphism
\newoperator{Hom}   % Homomorphism
\newoperator{id}    % Identity
\newoperator{Im}    % Image
\newoperator{Ind}   % Index
\newoperator{Ker}   % Kernel
\newoperator{lcm}   % Least common multiple
\newoperator{Mat}   % Matrix
\newoperator{mult}  % Multiplicity
\newoperator{new}   % New
\newoperator{Nm}    % Norm
\newoperator{old}   % Old
\newoperator{op}    % Opposite
\newoperator{ord}   % Order
\newoperator{Pay}   % Payley
\newoperator{PG}    % Projective geometry
\newoperator{PGL}   % Projective general linear
\newoperator{PSL}   % Projective special linear
\newoperator{rad}   % Radical
\newoperator{ran}   % Range
\newoperator{Res}   % Residue
\newoperator{rk}    % Rank
\newoperator{Re}    % Real
\newoperator{row}   % Row
\newoperator{sgn}   % Sign
\newoperator{Sing}  % Singular
\newoperator{SK}    % Skeleton
\newoperator{sp}    % Span
\newoperator{SL}    % Special linear
\newoperator{SO}    % Special orthogonal
\newoperator{Spec}  % Spectrum
\newoperator{Stab}  % Stabiliser
\newoperator{star}  % Star
\newoperator{srg}   % Strongly regular graph
\newoperator{supp}  % Support
\newoperator{Sym}   % Symmetric
\newoperator{tors}  % Torsion
\newoperator{Tr}    % Trace
\newoperator{vol}   % Volume
\newoperator{wt}    % Weight

% Roman

\newcommand{\A}{\mathrm{A}}   % Roman A
\newcommand{\B}{\mathrm{B}}   % Roman B
\newcommand{\C}{\mathrm{C}}   % Roman C
\newcommand{\D}{\mathrm{D}}   % Roman D
\newcommand{\E}{\mathrm{E}}   % Roman E
\newcommand{\F}{\mathrm{F}}   % Roman F
\newcommand{\G}{\mathrm{G}}   % Roman G
\renewcommand{\H}{\mathrm{H}} % Roman H
\newcommand{\I}{\mathrm{I}}   % Roman I
\newcommand{\J}{\mathrm{J}}   % Roman J
\newcommand{\K}{\mathrm{K}}   % Roman K
\renewcommand{\L}{\mathrm{L}} % Roman L
\newcommand{\M}{\mathrm{M}}   % Roman M
\newcommand{\N}{\mathrm{N}}   % Roman N
\renewcommand{\O}{\mathrm{O}} % Roman O
\renewcommand{\P}{\mathrm{P}} % Roman P
\newcommand{\Q}{\mathrm{Q}}   % Roman Q
\newcommand{\R}{\mathrm{R}}   % Roman R
\renewcommand{\S}{\mathrm{S}} % Roman S
\newcommand{\T}{\mathrm{T}}   % Roman T
\newcommand{\U}{\mathrm{U}}   % Roman U
\newcommand{\V}{\mathrm{V}}   % Roman V
\newcommand{\W}{\mathrm{W}}   % Roman W
\newcommand{\X}{\mathrm{X}}   % Roman X
\newcommand{\Y}{\mathrm{Y}}   % Roman Y
\newcommand{\Z}{\mathrm{Z}}   % Roman Z

\renewcommand{\a}{\mathrm{a}} % Roman a
\renewcommand{\b}{\mathrm{b}} % Roman b
\renewcommand{\c}{\mathrm{c}} % Roman c
\renewcommand{\d}{\mathrm{d}} % Roman d
\newcommand{\e}{\mathrm{e}}   % Roman e
\newcommand{\f}{\mathrm{f}}   % Roman f
\newcommand{\g}{\mathrm{g}}   % Roman g
\newcommand{\h}{\mathrm{h}}   % Roman h
\renewcommand{\i}{\mathrm{i}} % Roman i
\renewcommand{\j}{\mathrm{j}} % Roman j
\renewcommand{\k}{\mathrm{k}} % Roman k
\renewcommand{\l}{\mathrm{l}} % Roman l
\newcommand{\m}{\mathrm{m}}   % Roman m
\renewcommand{\n}{\mathrm{n}} % Roman n
\renewcommand{\o}{\mathrm{o}} % Roman o
\newcommand{\p}{\mathrm{p}}   % Roman p
\newcommand{\q}{\mathrm{q}}   % Roman q
\renewcommand{\r}{\mathrm{r}} % Roman r
\newcommand{\s}{\mathrm{s}}   % Roman s
\renewcommand{\t}{\mathrm{t}} % Roman t
\renewcommand{\u}{\mathrm{u}} % Roman u
\renewcommand{\v}{\mathrm{v}} % Roman v
\newcommand{\w}{\mathrm{w}}   % Roman w
\newcommand{\x}{\mathrm{x}}   % Roman x
\newcommand{\y}{\mathrm{y}}   % Roman y
\newcommand{\z}{\mathrm{z}}   % Roman z

% Tikz

\tikzset{
  arrow symbol/.style={"#1" description, allow upside down, auto=false, draw=none, sloped},
  subset/.style={arrow symbol={\subset}},
  cong/.style={arrow symbol={\cong}}
}

% Fancy header

\pagestyle{fancy}
\lhead{\module}
\rhead{\nouppercase{\leftmark}}

% Make title

\title{\module}
\author{Lectured by \lecturer \\ Typed by David Kurniadi Angdinata}
\date{\term}

\begin{document}

% Title page
\maketitle
\cover
\vfill
\begin{abstract}
\noindent\syllabus
\end{abstract}

\pagebreak

% Contents page
\tableofcontents

\pagebreak

% Document page
\setcounter{section}{-1}

\section{Introduction}

\lecture{1}{Friday}{11/01/19}

I will not follow a particular book, but everything I am going to say will be contained in one of the following books.
\begin{itemize}
\item I Shafarevich, Basic algebraic geometry, 1974
\item R Hartshorne, Algebraic geometry, 1977
\item J Harris, Algebraic geometry: a first course, 1922
\end{itemize}

\pagebreak

\section{Affine varieties}

\begin{notation1}
\hfill
\begin{itemize}
\item $ R $ is a commutative ring with unity.
\item $ K $ is a field.
\item $ K\sb{x_1, \dots, x_n} $ is the ring of polynomials in $ n $ variables.
\item $ \A^n $ is $ K^n $ as a set.
\end{itemize}
\end{notation1}

\begin{definition}
Let $ S \subseteq K\sb{x_1, \dots, x_n} $ then
$$ Z\rb{S} = \cb{x \in \A^n \mid \forall f \in S, \ f\rb{x} = 0} $$
is called the \textbf{zero locus} of $ S $. Subsets of $ \A^n $ that are of this form are called \textbf{affine varieties}.
\end{definition}

\begin{remark1}
Some authors call \textbf{algebraic set} the object $ Z\rb{S} $. We will not follow this notation.
\end{remark1}

\begin{example1}
\hfill
\begin{itemize}
\item Single points $ p = \rb{p_1, \dots, p_n} $. $ p = Z\rb{S} $ where
$$ S = \cb{x_1 - p_1, \dots, x_n - p_n}. $$
\item $ \A^n = Z\rb{0} $.
\item $ \emptyset = Z\rb{1} $.
\item Subspaces of $ \A^n = K^n $.
\item If $ X = Z\rb{f_1, \dots, f_n} \subseteq \A^n $ and $ Y = Z\rb{g_1, \dots, g_m} \subseteq \A^n $ are affine varieties then
$$ X \times Y = Z\rb{f_1, \dots, f_n, g_1, \dots, g_m} \subseteq \A^{n + m} $$
is a variety.
\end{itemize}
\end{example1}

\begin{remark1}
If $ S \subseteq K\sb{x_1, \dots, x_n} $ and $ I = \ab{S} $ then $ Z\rb{S} = Z\rb{I} $.
\end{remark1}

\begin{theorem}[Hilbert's basis theorem]
If $ R $ is Noetherian then $ R\sb{x} $ is Noetherian.
\end{theorem}

\begin{corollary}
Every ideal in $ K\sb{x_1, \dots, x_n} $ is finitely generated.
\end{corollary}

\begin{definition}
Let $ X \subseteq \A^n $ then
$$ I\rb{X} = \cb{f \in K\sb{x_1, \dots, x_n} \mid \forall x \in X, \ f\rb{x} = 0}. $$
\end{definition}

\begin{example1}
$ I\rb{p} = I\rb{\rb{p_1, \dots, p_n}} = \ab{x_1 - p_1, \dots, x_n - p_n} $.
\end{example1}

Goal is
$$
\begin{array}{rcl}
\cb{\text{affine varieties in} \ \A^n} & \leftrightarrow & \cb{\text{ideals of} \ K\sb{x_1, \dots, x_n}} \\
X & \mapsto & I\rb{X} \\
Z\rb{J} & \mapsfrom & J
\end{array}.
$$
$ Z\rb{I\rb{X}} = X $ but $ I\rb{Z\rb{J}} \supseteq J $.

\begin{example1}
$ J = \ab{x^2 + 1} \subseteq \R\sb{x} = I\rb{\emptyset} = I\rb{Z\rb{x^2 + 1}} $.
\end{example1}

\begin{proposition}
\hfill
\begin{itemize}
\item If $ X \subseteq Y $ then $ I\rb{Y} \subseteq I\rb{X} $. If $ I \subseteq J $ then $ Z\rb{J} \subseteq Z\rb{I} $.
\item $ X \subseteq Z\rb{I\rb{X}} $ and $ S \subseteq I\rb{Z\rb{S}} $.
\item If $ X $ is affine then $ Z\rb{J\rb{X}} = X $. If $ X = Z\rb{S} $ then take $ Z $ of $ S \subseteq I\rb{Z\rb{S}} $.
\end{itemize}
\end{proposition}

\begin{example1}
Let $ J \subseteq \C\sb{x} $. $ J = \ab{f} $, where $ f = \rb{x - x_1}^{k_1} \dots \rb{x - x_n}^{k_n} $.
\end{example1}

\begin{definition}
Let $ I \subseteq K\sb{x_1, \dots, x_n} $ be an ideal.
$$ I \subseteq \sqrt{I} = \cb{f \in K\sb{x_1, \dots, x_n} \mid \exists n \in \N, \ f^n \in I}. $$
If $ \sqrt{I} = I $, we say $ I $ is a \textbf{radical ideal}.
\end{definition}

(Exercise: $ \sqrt{I} $ is an ideal, $ I \subseteq \sqrt{I} $, and $ \sqrt{I} = \bigcap_{p \ \text{prime}} p $)

\begin{theorem}[Hilbert's Nullstellensatz]
$ I\rb{Z\rb{J}} = \sqrt{J} $. If $ \sqrt{J} = J $ then
$$
\begin{array}{rcl}
\cb{\text{affine varieties}} & \leftrightarrow & \cb{\text{radical ideals}} \\
X & \mapsto & I\rb{X} \\
Z\rb{J} & \mapsfrom & J
\end{array}.
$$
\end{theorem}

\lecture{2}{Monday}{14/01/19}

\begin{proposition}
\hfill
\begin{enumerate}
\item $ Z\rb{S} \cup Z\rb{T} = Z\rb{ST} $.
\item $ \bigcap_i Z\rb{S_i} = Z\rb{\bigcup_i S_i} $.
\item $ Z\rb{0} = \A^n $ and $ Z\rb{1} = \emptyset $.
\end{enumerate}
\end{proposition}

\begin{proof}
\hfill
\begin{itemize}
\item[1.] If $ p \in Z\rb{S} \cup Z\rb{T} $, then $ f\rb{p} = 0 $ for $ f \in S $ or $ f \in T $, so $ f\rb{x} = 0 $ for $ f \in ST $, where
$$ ST = \cb{\sum_{i \in I, \ I \ \text{finite}} s_it_i} \subseteq S \cap T, $$
with equality if $ S + T = R $. If $ p \in Z\rb{ST} $, there exists $ f $ such that $ f\rb{p} = 0 $ for $ f \in S $ or $ f\rb{p} = 0 $ for $ f \in T $, so $ p \in Z\rb{S} \cup Z\rb{T} $.
\end{itemize}
\end{proof}

\begin{definition}
The \textbf{Zariski topology} on $ \A^n $ is the topology generated by closed sets of the form $ Z\rb{S} $. By the above proposition this is a topology.
\end{definition}

\begin{example1}
$ \A^1 $ is not Hausdorff.
\end{example1}

\begin{definition}
A topological space $ X $ is \textbf{irreducible} if it cannot be expressed as a union $ X = A \cup B $, where $ A $ and $ B $ are proper and closed subsets. $ \emptyset $ is not considered irreducible.
\end{definition}

\begin{example1}
$ \A^1 $.
\end{example1}

\begin{example1}
Any non-empty open set of irreducible $ X $ is dense and irreducible. Suppose $ A $ is open then $ X = A^c \cup \overline{A} $. Since $ X $ is irreducible then $ A^c = X $, a contradiction, or $ \overline{A} = X $. Suppose $ A $ is reducible. Let $ A = \rb{A \cap B} \cup \rb{A \cap C} $, where $ B $ and $ C $ are closed. Then $ X = A^c \cup \rb{B \cup C} $. $ A^c = X $ or $ B \cup C = X $, which are contradictions.
\end{example1}

\begin{example1}
If $ A $ is irreducible then $ \overline{A} $ is also irreducible. Suppose $ \overline{A} $ is not irreducible. $ \overline{A} = \rb{\overline{A} \cap B} \cup \rb{\overline{A} \cap C} $. Take $ \bigcap A $, $ A = \rb{A \cap B} \cup \rb{A \cap C} $, a contradiction.
\end{example1}

\begin{definition}
An affine variety is \textbf{irreducible} if it is irreducible as a topological space.
\end{definition}

\begin{remark1}
A \textbf{quasi-affine variety} is an open set of an affine variety.
\end{remark1}

\begin{proposition}
\hfill
\begin{enumerate}
\item $ I\rb{X \cup Y} = I\rb{X} \cap I\rb{Y} $.
\item $ Z\rb{I\rb{X}} = \overline{X} $ for any $ X \subseteq \A^n $.
\end{enumerate}
\end{proposition}

\begin{proof}
\hfill
\begin{enumerate}
\item If $ f \in I\rb{X \cup Y} $ then $ f\rb{p} = 0 $ for all $ p \in X \cup Y $, so $ f \in I\rb{X} $ and $ f \in I\rb{Y} $.
\item We know that $ X \subseteq Z\rb{I\rb{X}} $ hence $ \overline{X} \subseteq Z\rb{I\rb{X}} $. Now, let $ Y $ be a closed set containing $ X $, that is $ X \subseteq Y $. Then
$$ I\rb{Y} \subset I\rb{X} \qquad \implies \qquad Z\rb{I\rb{X}} \subset Z\rb{I\rb{X}} = Y, $$
so any closed set containing $ Y $ contains $ Z\rb{I\rb{X}} $.
\end{enumerate}
\end{proof}

\begin{proposition}
$ X $ is irreducible if and only if $ I\rb{X} $ is prime.
\end{proposition}

\begin{proof}
\hfill
\begin{itemize}
\item[$ \implies $] Let $ f, g \in I\rb{X} $.
$$ X \subseteq Z\rb{fg} = Z\rb{f} \cup Z\rb{g} \qquad \implies \qquad X = \rb{X \cap Z\rb{f}} \cup \rb{X \cap Z\rb{g}}. $$
$ Z\rb{f} \subseteq X $, so $ f \in I\rb{X} $, or $ Z\rb{g} \subseteq X $, so $ g \in I\rb{X} $.
\item[$ \impliedby $] Exercise.
\end{itemize}
\end{proof}

\begin{example1}
$ \A^n $.
\end{example1}

\begin{definition}
If $ X \subseteq \A^n $, the \textbf{coordinate ring} of $ X $ is
$$ A\rb{X} = \dfrac{K\sb{x_1, \dots, x_n}}{I\rb{X}}. $$
\end{definition}

\lecture{3}{Tuesday}{15/01/19}

\begin{example1}
Let $ f \in K\sb{x_1, \dots, x_n} $ be irreducible. If $ n = 3 $, $ Z\rb{f} $ is a surface. If $ n = 2 $, $ Z\rb{f} $ is a curve.
\end{example1}

\begin{example1}
Let $ y - x^2 \in K\sb{x, y} $. Then
$$ \function{A\rb{X} = \dfrac{K\sb{x, y}}{\ab{y - x^2}} \cong K\sb{x, x^2}}{K\sb{x}}{\sum_{i, j} a_{ij}x^ix^{2j} = \sum_{i, j} a_{ij}x^{2j + i}}{\sum_n b_nx^n}. $$
\end{example1}

\begin{example1}
Let $ xy - 1 \in K\sb{x, y} $. Then
$$ A\rb{X} = \dfrac{K\sb{x, y}}{\ab{xy - 1}} \cong K\sb{x, \dfrac{1}{x}}. $$
$ A\rb{X} $ cannot be $ K\sb{x} $.
\end{example1}

\begin{definition}
A \textbf{Noetherian} topological space $ X $ is a topological space such that if
$$ C_1 \supseteq C_2 \supseteq \dots $$
is a decreasing chain of closed sets then there is a $ k $ such that $ C_k = C_{k + 1} = \dots $.
\end{definition}

\begin{example1}
$ \A^n $. Recall that if $ A \subset B $ then $ I\rb{B} \subset I\rb{A} $. So using the definition above,
$$ I\rb{C_1} \subseteq I\rb{C_2} \subseteq \dots. $$
Since $ K\sb{x_1, \dots, x_n} $ is Noetherian then $ I\rb{C_i} $ stabilises. So $ I\rb{C_k} = I\rb{C_{k + 1}} = \dots $, but taking $ Z $, we recover $ C_k $ so $ C_k $ stabilises as well.
\end{example1}

\begin{theorem}
If $ X $ is Noetherian then any non-empty closed subset can be expressed as a finite union of irreducible closed sets $ X = Y_1 \cup \dots \cup Y_n $. Moreover, if we require that $ Y_i \subseteq Y_j $ then this expression is unique.
\end{theorem}

\begin{proof}
Let $ C $ be the collection of closed sets that do not satisfy that property. Let $ Y $ be a minimum closed inside $ C $, in particular $ Y $ is reducible, so $ Y = Y' \cup Y'' $, for $ Y', Y'' $ closed. Hence $ Y', Y'' \not\subset C $, so they can be expressed as a finite union of irreducibles, a contradiction. If $ Y_i \not\subset Y_j $, then suppose
$$ Y_1 \cup \dots \cup Y_n = X_1 \cup \dots \cup X_n. $$
Then $ Y_1 \subset X_1 \cup X_n $, in particular $ Y_1 = \bigcup_j \rb{Y_1 \cap X_j} $, so there is a $ j $ such that $ Y_1 \cap X_j = Y_1 $, so $ Y_1 \subset X_j $. We can assume $ j = 1 $ and repeat the same argument to find that $ Y_1 = X_1 $, so consider $ \overline{Y \setminus Y_1} = Y_2 \cup \dots \cup Y_n $. But
$$ Y_2 \cup \dots \cup Y_n = X_2 \cup \dots \cup X_n, $$
and the result follows by induction.
\end{proof}

\begin{corollary}
Any affine variety in $ \A^n $ can be expressed equally as a union of irreducible algebraic varieties.
\end{corollary}

\begin{definition}
The \textbf{dimension} of a topological space is the supremum of $ n $ where
$$ Y_0 \subset \dots \subset Y_n $$
is a sequence of irreducible closed sets.
\end{definition}

\begin{example1}
Dimension of $ \A^1 $ is one.
\end{example1}

\begin{definition}
Let $ A $ be a ring and $ \mathfrak{p} $ be a prime ideal, then the \textbf{height} of $ \mathfrak{p} $ is the supremum of $ n $ where
$$ \mathfrak{p}_1 \subset \dots \subset \mathfrak{p}_n \subset \mathfrak{p}, $$
where $ \mathfrak{p}_i $ are prime. The \textbf{Krull dimension} of $ A $ is
$$ \sup_{\mathfrak{p} \ \text{prime}} height\rb{\mathfrak{p}}. $$
\end{definition}

\begin{proposition}
If $ Y $ is affine then $ \dim\rb{Y} = \dim\rb{A\rb{Y}} $.
\end{proposition}

\begin{proof}
Let $ C $ be a closed and irreducible set $ C \subset Y $, then $ I\rb{C} \supset I\rb{Y} $, then $ I\rb{C} $ is prime.
\end{proof}

\begin{proposition}
Let $ K $ be a field and $ B $ be an integral domain which is a finitely generated algebra, then
\begin{itemize}
\item $ \dim\rb{B} $ is the transcendence degree of $ K\rb{B} $ over $ K $, and
\item if $ \mathfrak{p} \subseteq B $ is prime, then
$$ height\rb{\mathfrak{p}} + \dim\rb{\dfrac{B}{\mathfrak{p}}} = \dim\rb{B}. $$
\end{itemize}
\end{proposition}

\begin{proof}
Atiyah Macdonald chapter 11.
\end{proof}

\begin{proposition}[Krull Hauptidealsatz]
Let $ A $ be a Noetherian ring and $ f \in A $ not a zero divisor and not a unit. Then every prime ideal containing $ f $ has height one.
\end{proposition}

\begin{proof}
Atiyah Macdonald page 122.
\end{proof}

\lecture{4}{Friday}{18/01/19}

\begin{proposition}
A Noetherian integral domain $ A $ is a UFD if and only if every prime ideal $ I $ of height one is principal.
\end{proposition}

\begin{theorem}
An irreducible variety $ Y \subseteq \A^n $ has dimension $ n - 1 $ if and only if $ Y = Z\rb{f} $ where $ f $ is an irreducible polynomial in $ K\sb{x_1, \dots, x_n} $.
\end{theorem}

\begin{proof}
\hfill
\begin{itemize}
\item[$ \implies $] If $ Y $ has dimension $ n - 1 $ then $ I\rb{Y} $ has height one, by the above proposition $ I\rb{Y} = \ab{f} $, so $ Y = Z\rb{f} $.
\item[$ \impliedby $] Let $ I = I\rb{Y} $ then $ I $ is prime, by the Krull Hauptidealsatz we have that $ I $ has height one, so $ \dim\rb{Y} = n - 1 $.
\end{itemize}
\end{proof}

\section{Projective varieties}

\begin{definition}
The \textbf{projective space} $ \P^n $ is defined as
$$ \P^n = \dfrac{\A^{n + 1} \setminus \cb{0}}{\cb{x \sim \lambda x \mid \lambda \in K^*}}. $$
A point in $ \P^n $ is written as $ \sb{a_0 : \dots : a_n} = \overline{\rb{a_0, \dots, a_n}} $.
\end{definition}

\begin{definition}
A \textbf{graded ring} $ R $ is a ring together with a decomposition
$$ R = \bigoplus_{d > 0} R_d, $$
where $ R_d $ are abelian groups and $ R_k \cdot R_t \subseteq R_{k + t} $.
\end{definition}

\begin{example1}
$ K\sb{x_0, \dots, x_n} $ is a graded ring, where $ R_d $ are monomials of degree $ d $.
\end{example1}

\begin{notation1}
Let $ A $ be $ K\sb{x_0, \dots, x_n} $ without the grading and $ S $ be $ K\sb{x_0, \dots, x_n} $ as a graded ring.
\end{notation1}

\begin{definition}
An ideal $ I \subseteq S $ is \textbf{homogeneous} if
$$ I = \bigoplus_{d \ge 0} \rb{I \cap S_d}. $$
If $ f = f_0 + \dots + f_d $, then $ f_i \in I $.
\end{definition}

\begin{remark1}
$ I $ is homogeneous if and only if $ I = \ab{f_0, \dots, f_n} $, where $ f_i $ are homogeneous.
\end{remark1}

\begin{lemma}
If $ I, J $ are homogeneous then
\begin{enumerate}
\item $ I + J $ is homogeneous,
\item $ IJ $ is homogeneous,
\item $ I \cap J $ is homogeneous, and
\item $ \sqrt{I} $ is homogeneous.
\end{enumerate}
\end{lemma}

\begin{proof}
\hfill
\begin{itemize}
\item[4.] Let $ f = f_0 + \dots + f_d \in \sqrt{I} $ then
$$ f^n = \rb{f_0 + \dots + f_d}^n = f_d^n + \dots \in I \qquad \implies \qquad f_d^n \in I \qquad \implies \qquad f_d \in \sqrt{I}, $$
so $ f - f_d \in \sqrt{I} $, by induction $ f_i \in \sqrt{I} $.
\end{itemize}
\end{proof}

\begin{definition}
If $ f $ is homogeneous of degree $ k $ then $ f\rb{\lambda \cdot x} = \lambda^k \cdot f\rb{x} $, in particular $ f\rb{x} = 0 $ if and only if $ f\rb{\lambda \cdot x} = 0 $, so it makes sense to define
$$ Z\rb{f} = \cb{x \in \P^n \mid f\rb{x} = 0}. $$
More generally, if $ I \subseteq S $ is a homogeneous ideal then
$$ Z\rb{I} = \cb{x \in \P^n \mid f \in I \ \text{homogeneous}, \ f\rb{x} = 0}. $$
\end{definition}

\begin{definition}
A subset $ X \subseteq \P^n $ is called a \textbf{projective variety} if $ X = Z\rb{T} $ for some homogeneous ideal $ T $.
\end{definition}

\begin{proposition}
\hfill
\begin{itemize}
\item $ Z\rb{S} \cup Z\rb{T} = Z\rb{ST} $.
\item $ \bigcap_\alpha Z\rb{S_\alpha} = Z\rb{\bigcup_\alpha S_\alpha} $.
\item $ Z\rb{0} = \P^n $ and $ Z\rb{1} = \emptyset $.
\end{itemize}
\end{proposition}

\begin{definition}
We define the \textbf{Zariski topology} on $ \P^n $ by taking closed sets to be $ Z\rb{T} $ for some $ T $.
\end{definition}

\begin{definition}
\hfill
\begin{itemize}
\item A projective variety is \textbf{irreducible} if it is an irreducible topological space.
\item An open subset of a projective variety is called a \textbf{quasi-projective variety}.
\item The \textbf{dimension} of a projective variety is its dimension as a topological space.
\item If $ T \subseteq S $ then
$$ I\rb{T} = \ab{f \in S \mid f \ \text{homogeneous}, \ \forall p \in T, \ f\rb{p} = 0}. $$
\end{itemize}
\end{definition}

\begin{definition}
If $ X $ is a projective variety the \textbf{homogeneous coordinate ring} is
$$ S\rb{X} = \dfrac{S}{I\rb{X}}. $$
\end{definition}

\begin{definition}
If $ f \in S $ is linear and homogeneous, we call $ Z\rb{f} $ a \textbf{hyperplane}.
\end{definition}

Let
$$ \function[\phi_i]{U_i}{\A^n}{\sb{a_0 : \dots : a_n}}{\rb{\dfrac{a_0}{a_i}, \dots, \dfrac{a_n}{a_i}}}. $$

\begin{proposition}
$ \phi_i $ is a homeomorphism in the Zariski topology.
\end{proposition}

\end{document}