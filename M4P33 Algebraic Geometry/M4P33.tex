\def\module{M4P33 Algebraic Geometry}
\def\lecturer{Dr Genival Da Silva Jr}
\def\term{Spring 2019}

\def\thm{section}

\documentclass{article}

% Packages

\usepackage{amssymb}
\usepackage{amsthm}
\usepackage[UKenglish]{babel}
\usepackage{commath}
\usepackage{enumitem}
\usepackage{etoolbox}
\usepackage{fancyhdr}
\usepackage[margin=1in]{geometry}
\usepackage{graphicx}
\usepackage[hidelinks]{hyperref}
\usepackage[utf8]{inputenc}
\usepackage{listings}
\usepackage{mathtools}
\usepackage{stmaryrd}
\usepackage{tikz-cd}
\usepackage{csquotes}

% Formatting

\addto\captionsUKenglish{\renewcommand{\abstractname}{Syllabus}}
\delimitershortfall5pt
\ifx\thm\undefined\newtheorem{n}{}\else\newtheorem{n}{}[\thm]\fi
\newcommand\newoperator[1]{\ifcsdef{#1}{\cslet{#1}{\relax}}{}\csdef{#1}{\operatorname{#1}}}
\setlength{\parindent}{0cm}

% Environments

\theoremstyle{plain}
\newtheorem{algorithm}[n]{Algorithm}
\newtheorem*{algorithm*}{Algorithm}
\newtheorem{algorithm**}{Algorithm}
\newtheorem{conjecture}[n]{Conjecture}
\newtheorem*{conjecture*}{Conjecture}
\newtheorem{conjecture**}{Conjecture}
\newtheorem{corollary}[n]{Corollary}
\newtheorem*{corollary*}{Corollary}
\newtheorem{corollary**}{Corollary}
\newtheorem{lemma}[n]{Lemma}
\newtheorem*{lemma*}{Lemma}
\newtheorem{lemma**}{Lemma}
\newtheorem{proposition}[n]{Proposition}
\newtheorem*{proposition*}{Proposition}
\newtheorem{proposition**}{Proposition}
\newtheorem{theorem}[n]{Theorem}
\newtheorem*{theorem*}{Theorem}
\newtheorem{theorem**}{Theorem}

\theoremstyle{definition}
\newtheorem{aim}[n]{Aim}
\newtheorem*{aim*}{Aim}
\newtheorem{aim**}{Aim}
\newtheorem{axiom}[n]{Axiom}
\newtheorem*{axiom*}{Axiom}
\newtheorem{axiom**}{Axiom}
\newtheorem{condition}[n]{Condition}
\newtheorem*{condition*}{Condition}
\newtheorem{condition**}{Condition}
\newtheorem{definition}[n]{Definition}
\newtheorem*{definition*}{Definition}
\newtheorem{definition**}{Definition}
\newtheorem{example}[n]{Example}
\newtheorem*{example*}{Example}
\newtheorem{example**}{Example}
\newtheorem{exercise}[n]{Exercise}
\newtheorem*{exercise*}{Exercise}
\newtheorem{exercise**}{Exercise}
\newtheorem{fact}[n]{Fact}
\newtheorem*{fact*}{Fact}
\newtheorem{fact**}{Fact}
\newtheorem{goal}[n]{Goal}
\newtheorem*{goal*}{Goal}
\newtheorem{goal**}{Goal}
\newtheorem{law}[n]{Law}
\newtheorem*{law*}{Law}
\newtheorem{law**}{Law}
\newtheorem{plan}[n]{Plan}
\newtheorem*{plan*}{Plan}
\newtheorem{plan**}{Plan}
\newtheorem{problem}[n]{Problem}
\newtheorem*{problem*}{Problem}
\newtheorem{problem**}{Problem}
\newtheorem{question}[n]{Question}
\newtheorem*{question*}{Question}
\newtheorem{question**}{Question}
\newtheorem{warning}[n]{Warning}
\newtheorem*{warning*}{Warning}
\newtheorem{warning**}{Warning}
\newtheorem{acknowledgements}[n]{Acknowledgements}
\newtheorem*{acknowledgements*}{Acknowledgements}
\newtheorem{acknowledgements**}{Acknowledgements}
\newtheorem{annotations}[n]{Annotations}
\newtheorem*{annotations*}{Annotations}
\newtheorem{annotations**}{Annotations}
\newtheorem{assumption}[n]{Assumption}
\newtheorem*{assumption*}{Assumption}
\newtheorem{assumption**}{Assumption}
\newtheorem{conclusion}[n]{Conclusion}
\newtheorem*{conclusion*}{Conclusion}
\newtheorem{conclusion**}{Conclusion}
\newtheorem{claim}[n]{Claim}
\newtheorem*{claim*}{Claim}
\newtheorem{claim**}{Claim}
\newtheorem{notation}[n]{Notation}
\newtheorem*{notation*}{Notation}
\newtheorem{notation**}{Notation}
\newtheorem{note}[n]{Note}
\newtheorem*{note*}{Note}
\newtheorem{note**}{Note}
\newtheorem{remark}[n]{Remark}
\newtheorem*{remark*}{Remark}
\newtheorem{remark**}{Remark}

% Lectures

\newcommand{\lecture}[3]{ % Lecture
  \marginpar{
    Lecture #1 \\
    #2 \\
    #3
  }
}

% Blackboard

\renewcommand{\AA}{\mathbb{A}} % Blackboard A
\newcommand{\BB}{\mathbb{B}}   % Blackboard B
\newcommand{\CC}{\mathbb{C}}   % Blackboard C
\newcommand{\DD}{\mathbb{D}}   % Blackboard D
\newcommand{\EE}{\mathbb{E}}   % Blackboard E
\newcommand{\FF}{\mathbb{F}}   % Blackboard F
\newcommand{\GG}{\mathbb{G}}   % Blackboard G
\newcommand{\HH}{\mathbb{H}}   % Blackboard H
\newcommand{\II}{\mathbb{I}}   % Blackboard I
\newcommand{\JJ}{\mathbb{J}}   % Blackboard J
\newcommand{\KK}{\mathbb{K}}   % Blackboard K
\newcommand{\LL}{\mathbb{L}}   % Blackboard L
\newcommand{\MM}{\mathbb{M}}   % Blackboard M
\newcommand{\NN}{\mathbb{N}}   % Blackboard N
\newcommand{\OO}{\mathbb{O}}   % Blackboard O
\newcommand{\PP}{\mathbb{P}}   % Blackboard P
\newcommand{\QQ}{\mathbb{Q}}   % Blackboard Q
\newcommand{\RR}{\mathbb{R}}   % Blackboard R
\renewcommand{\SS}{\mathbb{S}} % Blackboard S
\newcommand{\TT}{\mathbb{T}}   % Blackboard T
\newcommand{\UU}{\mathbb{U}}   % Blackboard U
\newcommand{\VV}{\mathbb{V}}   % Blackboard V
\newcommand{\WW}{\mathbb{W}}   % Blackboard W
\newcommand{\XX}{\mathbb{X}}   % Blackboard X
\newcommand{\YY}{\mathbb{Y}}   % Blackboard Y
\newcommand{\ZZ}{\mathbb{Z}}   % Blackboard Z

% Brackets

\renewcommand{\eval}[1]{\left. #1 \right|}          % Evaluation
\newcommand{\br}{\del}                              % Brackets
\newcommand{\abr}[1]{\left\langle #1 \right\rangle} % Angle brackets
\newcommand{\fbr}[1]{\left\lfloor #1 \right\rfloor} % Floor brackets
\newcommand{\lbr}[1]{\left\lfloor #1 \right\rfloor} % Ceiling brackets
\newcommand{\st}{\ \middle| \ }                     % Such that

% Calligraphic

\newcommand{\AAA}{\mathcal{A}} % Calligraphic A
\newcommand{\BBB}{\mathcal{B}} % Calligraphic B
\newcommand{\CCC}{\mathcal{C}} % Calligraphic C
\newcommand{\DDD}{\mathcal{D}} % Calligraphic D
\newcommand{\EEE}{\mathcal{E}} % Calligraphic E
\newcommand{\FFF}{\mathcal{F}} % Calligraphic F
\newcommand{\GGG}{\mathcal{G}} % Calligraphic G
\newcommand{\HHH}{\mathcal{H}} % Calligraphic H
\newcommand{\III}{\mathcal{I}} % Calligraphic I
\newcommand{\JJJ}{\mathcal{J}} % Calligraphic J
\newcommand{\KKK}{\mathcal{K}} % Calligraphic K
\newcommand{\LLL}{\mathcal{L}} % Calligraphic L
\newcommand{\MMM}{\mathcal{M}} % Calligraphic M
\newcommand{\NNN}{\mathcal{N}} % Calligraphic N
\newcommand{\OOO}{\mathcal{O}} % Calligraphic O
\newcommand{\PPP}{\mathcal{P}} % Calligraphic P
\newcommand{\QQQ}{\mathcal{Q}} % Calligraphic Q
\newcommand{\RRR}{\mathcal{R}} % Calligraphic R
\newcommand{\SSS}{\mathcal{S}} % Calligraphic S
\newcommand{\TTT}{\mathcal{T}} % Calligraphic T
\newcommand{\UUU}{\mathcal{U}} % Calligraphic U
\newcommand{\VVV}{\mathcal{V}} % Calligraphic V
\newcommand{\WWW}{\mathcal{W}} % Calligraphic W
\newcommand{\XXX}{\mathcal{X}} % Calligraphic X
\newcommand{\YYY}{\mathcal{Y}} % Calligraphic Y
\newcommand{\ZZZ}{\mathcal{Z}} % Calligraphic Z

% Fraktur

\newcommand{\aaa}{\mathfrak{a}}   % Fraktur a
\newcommand{\bbb}{\mathfrak{b}}   % Fraktur b
\newcommand{\ccc}{\mathfrak{c}}   % Fraktur c
\newcommand{\ddd}{\mathfrak{d}}   % Fraktur d
\newcommand{\eee}{\mathfrak{e}}   % Fraktur e
\newcommand{\fff}{\mathfrak{f}}   % Fraktur f
\renewcommand{\ggg}{\mathfrak{g}} % Fraktur g
\newcommand{\hhh}{\mathfrak{h}}   % Fraktur h
\newcommand{\iii}{\mathfrak{i}}   % Fraktur i
\newcommand{\jjj}{\mathfrak{j}}   % Fraktur j
\newcommand{\kkk}{\mathfrak{k}}   % Fraktur k
\renewcommand{\lll}{\mathfrak{l}} % Fraktur l
\newcommand{\mmm}{\mathfrak{m}}   % Fraktur m
\newcommand{\nnn}{\mathfrak{n}}   % Fraktur n
\newcommand{\ooo}{\mathfrak{o}}   % Fraktur o
\newcommand{\ppp}{\mathfrak{p}}   % Fraktur p
\newcommand{\qqq}{\mathfrak{q}}   % Fraktur q
\newcommand{\rrr}{\mathfrak{r}}   % Fraktur r
\newcommand{\sss}{\mathfrak{s}}   % Fraktur s
\newcommand{\ttt}{\mathfrak{t}}   % Fraktur t
\newcommand{\uuu}{\mathfrak{u}}   % Fraktur u
\newcommand{\vvv}{\mathfrak{v}}   % Fraktur v
\newcommand{\www}{\mathfrak{w}}   % Fraktur w
\newcommand{\xxx}{\mathfrak{x}}   % Fraktur x
\newcommand{\yyy}{\mathfrak{y}}   % Fraktur y
\newcommand{\zzz}{\mathfrak{z}}   % Fraktur z

% Geometry

\newcommand{\CP}{\mathbb{CP}}                                              % Complex projective space
\newcommand{\iintd}[4]{\iint_{#1} \, #2 \, \dif #3 \, \dif #4}             % Double integral
\newcommand{\RP}{\mathbb{RP}}                                              % Real projective space
\newcommand{\intd}[4]{\int_{#1}^{#2} \, #3 \, \dif #4}                     % Single integral
\newcommand{\iiintd}[5]{\iint_{#1} \, #2 \, \dif #3 \, \dif #4 \, \dif #5} % Triple integral

% Logic

\newcommand{\iffb}[2]{\br{#1 \leftrightarrow #2}} % Biconditional
\newcommand{\andb}[2]{\br{#1 \land #2}}           % Conjunction
\newcommand{\orb}[2]{\br{#1 \lor #2}}             % Disjunction
\newcommand{\nib}[2]{\br{#1 \notin #2}}           % Element of
\newcommand{\eqb}[2]{\br{#1 = #2}}                % Equal to
\newcommand{\teb}[1]{\br{\exists #1}}             % Existential quantifier
\newcommand{\impb}[2]{\br{#1 \rightarrow #2}}     % Implication
\newcommand{\ltb}[2]{\br{#1 < #2}}                % Less than
\newcommand{\leb}[2]{\br{#1 \le #2}}              % Less than or equal to
\newcommand{\notb}[1]{\br{\neg #1}}               % Negation
\newcommand{\inb}[2]{\br{#1 \in #2}}              % Not element of
\newcommand{\neb}[2]{\br{#1 \ne #2}}              % Not equal to
\newcommand{\subb}[2]{\br{#1 \subseteq #2}}       % Subset
\newcommand{\fab}[1]{\br{\forall #1}}             % Universal quantifier

% Maps

\newcommand{\bijection}[7][]{    % Bijection
  \ifx &#1&
    \begin{array}{rcl}
      #2 & \longleftrightarrow & #3 \\
      #4 & \longmapsto         & #5 \\
      #6 & \longmapsfrom       & #7
    \end{array}
  \else
    \begin{array}{ccrcl}
      #1 & : & #2 & \longrightarrow & #3 \\
         &   & #4 & \longmapsto     & #5 \\
         &   & #6 & \longmapsfrom   & #7
    \end{array}
  \fi
}
\newcommand{\birational}[7][]{   % Birational map
  \ifx &#1&
    \begin{array}{rcl}
      #2 & \dashrightarrow & #3 \\
      #4 & \longmapsto     & #5 \\
      #6 & \longmapsfrom   & #7
    \end{array}
  \else
    \begin{array}{ccrcl}
      #1 & : & #2 & \dashrightarrow & #3 \\
         &   & #4 & \longmapsto     & #5 \\
         &   & #6 & \longmapsfrom   & #7
    \end{array}
  \fi
}
\newcommand{\correspondence}[2]{ % Correspondence
  \cbr{
    \begin{array}{c}
      #1
    \end{array}
  }
  \qquad
  \leftrightsquigarrow
  \qquad
  \cbr{
    \begin{array}{c}
      #2
    \end{array}
  }
}
\newcommand{\function}[5][]{     % Function
  \ifx &#1&
    \begin{array}{rcl}
      #2 & \longrightarrow & #3 \\
      #4 & \longmapsto     & #5
    \end{array}
  \else
    \begin{array}{ccrcl}
      #1 & : & #2 & \longrightarrow & #3 \\
         &   & #4 & \longmapsto     & #5
    \end{array}
  \fi
}
\newcommand{\functions}[7][]{    % Functions
  \ifx &#1&
    \begin{array}{rcl}
      #2 & \longrightarrow & #3 \\
      #4 & \longmapsto     & #5 \\
      #6 & \longmapsto     & #7
    \end{array}
  \else
    \begin{array}{ccrcl}
      #1 & : & #2 & \longrightarrow & #3 \\
         &   & #4 & \longmapsto     & #5 \\
         &   & #6 & \longmapsto     & #7
    \end{array}
  \fi
}
\newcommand{\rational}[5][]{     % Rational map
  \ifx &#1&
    \begin{array}{rcl}
      #2 & \dashrightarrow & #3 \\
      #4 & \longmapsto     & #5
    \end{array}
  \else
    \begin{array}{ccrcl}
      #1 & : & #2 & \dashrightarrow & #3 \\
         &   & #4 & \longmapsto     & #5
    \end{array}
  \fi
}

% Matrices

\newcommand{\onebytwo}[2]{      % One by two matrix
  \begin{pmatrix}
    #1 & #2
  \end{pmatrix}
}
\newcommand{\onebythree}[3]{    % One by three matrix
  \begin{pmatrix}
    #1 & #2 & #3
  \end{pmatrix}
}
\newcommand{\twobyone}[2]{      % Two by one matrix
  \begin{pmatrix}
    #1 \\
    #2
  \end{pmatrix}
}
\newcommand{\twobytwo}[4]{      % Two by two matrix
  \begin{pmatrix}
    #1 & #2 \\
    #3 & #4
  \end{pmatrix}
}
\newcommand{\threebyone}[3]{    % Three by one matrix
  \begin{pmatrix}
    #1 \\
    #2 \\
    #3
  \end{pmatrix}
}
\newcommand{\threebythree}[9]{  % Three by three matrix
  \begin{pmatrix}
    #1 & #2 & #3 \\
    #4 & #5 & #6 \\
    #7 & #8 & #9
  \end{pmatrix}
}
\newcommand{\twobytwosmall}[4]{ % Two by two small matrix
  \begin{psmallmatrix}
    #1 & #2 \\
    #3 & #4
  \end{psmallmatrix}
}

% Number theory

\renewcommand{\symbol}[2]{\br{\tfrac{#1}{#2}}} % Power residue symbol
\newcommand{\unit}[1]{\br{\ZZ / #1\ZZ}^\times} % Unit group

% Operators

\newoperator{ab}    % Abelian
\newoperator{AG}    % Affine geometry
\newoperator{alg}   % Algebraic
\newoperator{Ann}   % Annihilator
\newoperator{area}  % Area
\newoperator{Aut}   % Automorphism
\newoperator{card}  % Cardinality
\newoperator{ch}    % Characteristic
\newoperator{Cl}    % Class
\newoperator{Coker} % Cokernel
\newoperator{col}   % Column
\newoperator{Corr}  % Correspondence
\newoperator{diam}  % Diameter
\newoperator{Disc}  % Discriminant
\newoperator{dom}   % Domain
\newoperator{Eig}   % Eigenvalue
\newoperator{Em}    % Embedding
\newoperator{End}   % Endomorphism
\newoperator{fin}   % Finite
\newoperator{Fix}   % Fixed
\newoperator{Frac}  % Fraction
\newoperator{Frob}  % Frobenius
\newoperator{Fun}   % Function
\newoperator{Gal}   % Galois
\newoperator{GL}    % General linear
\newoperator{Ham}   % Hamming
\newoperator{Homeo} % Homeomorphism
\newoperator{Hom}   % Homomorphism
\newoperator{id}    % Identity
\newoperator{Im}    % Image
\newoperator{Ind}   % Index
\newoperator{Ker}   % Kernel
\newoperator{lcm}   % Least common multiple
\newoperator{Mat}   % Matrix
\newoperator{mult}  % Multiplicity
\newoperator{new}   % New
\newoperator{Nm}    % Norm
\newoperator{old}   % Old
\newoperator{op}    % Opposite
\newoperator{ord}   % Order
\newoperator{Pay}   % Payley
\newoperator{PG}    % Projective geometry
\newoperator{PGL}   % Projective general linear
\newoperator{PSL}   % Projective special linear
\newoperator{rad}   % Radical
\newoperator{ran}   % Range
\newoperator{Res}   % Residue
\newoperator{rk}    % Rank
\newoperator{Re}    % Real
\newoperator{row}   % Row
\newoperator{sgn}   % Sign
\newoperator{Sing}  % Singular
\newoperator{SK}    % Skeleton
\newoperator{sp}    % Span
\newoperator{SL}    % Special linear
\newoperator{SO}    % Special orthogonal
\newoperator{Spec}  % Spectrum
\newoperator{Stab}  % Stabiliser
\newoperator{star}  % Star
\newoperator{srg}   % Strongly regular graph
\newoperator{supp}  % Support
\newoperator{Sym}   % Symmetric
\newoperator{tors}  % Torsion
\newoperator{Tr}    % Trace
\newoperator{vol}   % Volume
\newoperator{wt}    % Weight

% Roman

\newcommand{\A}{\mathrm{A}}   % Roman A
\newcommand{\B}{\mathrm{B}}   % Roman B
\newcommand{\C}{\mathrm{C}}   % Roman C
\newcommand{\D}{\mathrm{D}}   % Roman D
\newcommand{\E}{\mathrm{E}}   % Roman E
\newcommand{\F}{\mathrm{F}}   % Roman F
\newcommand{\G}{\mathrm{G}}   % Roman G
\renewcommand{\H}{\mathrm{H}} % Roman H
\newcommand{\I}{\mathrm{I}}   % Roman I
\newcommand{\J}{\mathrm{J}}   % Roman J
\newcommand{\K}{\mathrm{K}}   % Roman K
\renewcommand{\L}{\mathrm{L}} % Roman L
\newcommand{\M}{\mathrm{M}}   % Roman M
\newcommand{\N}{\mathrm{N}}   % Roman N
\renewcommand{\O}{\mathrm{O}} % Roman O
\renewcommand{\P}{\mathrm{P}} % Roman P
\newcommand{\Q}{\mathrm{Q}}   % Roman Q
\newcommand{\R}{\mathrm{R}}   % Roman R
\renewcommand{\S}{\mathrm{S}} % Roman S
\newcommand{\T}{\mathrm{T}}   % Roman T
\newcommand{\U}{\mathrm{U}}   % Roman U
\newcommand{\V}{\mathrm{V}}   % Roman V
\newcommand{\W}{\mathrm{W}}   % Roman W
\newcommand{\X}{\mathrm{X}}   % Roman X
\newcommand{\Y}{\mathrm{Y}}   % Roman Y
\newcommand{\Z}{\mathrm{Z}}   % Roman Z

\renewcommand{\a}{\mathrm{a}} % Roman a
\renewcommand{\b}{\mathrm{b}} % Roman b
\renewcommand{\c}{\mathrm{c}} % Roman c
\renewcommand{\d}{\mathrm{d}} % Roman d
\newcommand{\e}{\mathrm{e}}   % Roman e
\newcommand{\f}{\mathrm{f}}   % Roman f
\newcommand{\g}{\mathrm{g}}   % Roman g
\newcommand{\h}{\mathrm{h}}   % Roman h
\renewcommand{\i}{\mathrm{i}} % Roman i
\renewcommand{\j}{\mathrm{j}} % Roman j
\renewcommand{\k}{\mathrm{k}} % Roman k
\renewcommand{\l}{\mathrm{l}} % Roman l
\newcommand{\m}{\mathrm{m}}   % Roman m
\renewcommand{\n}{\mathrm{n}} % Roman n
\renewcommand{\o}{\mathrm{o}} % Roman o
\newcommand{\p}{\mathrm{p}}   % Roman p
\newcommand{\q}{\mathrm{q}}   % Roman q
\renewcommand{\r}{\mathrm{r}} % Roman r
\newcommand{\s}{\mathrm{s}}   % Roman s
\renewcommand{\t}{\mathrm{t}} % Roman t
\renewcommand{\u}{\mathrm{u}} % Roman u
\renewcommand{\v}{\mathrm{v}} % Roman v
\newcommand{\w}{\mathrm{w}}   % Roman w
\newcommand{\x}{\mathrm{x}}   % Roman x
\newcommand{\y}{\mathrm{y}}   % Roman y
\newcommand{\z}{\mathrm{z}}   % Roman z

% Tikz

\tikzset{
  arrow symbol/.style={"#1" description, allow upside down, auto=false, draw=none, sloped},
  subset/.style={arrow symbol={\subset}},
  cong/.style={arrow symbol={\cong}}
}

% Fancy header

\pagestyle{fancy}
\lhead{\module}
\rhead{\nouppercase{\leftmark}}

% Make title

\title{\module}
\author{Lectured by \lecturer \\ Typed by David Kurniadi Angdinata}
\date{\term}

\begin{document}

% Title page
\maketitle
\cover
\vfill
\begin{abstract}
\noindent\syllabus
\end{abstract}

\pagebreak

% Contents page
\tableofcontents

\pagebreak

% Document page
\setcounter{section}{-1}

\section{Introduction}

\lecture{1}{Friday}{11/01/19}

I will not follow a particular book, but everything I am going to say will be contained in one of the following books.
\begin{itemize}
\item I Shafarevich, Basic algebraic geometry, 1974
\item R Hartshorne, Algebraic geometry, 1977
\item J Harris, Algebraic geometry: a first course, 1922
\end{itemize}

\pagebreak

\section{Affine varieties}

\begin{notation}
\hfill
\begin{itemize}
\item $ R $ is a commutative ring with unity.
\item $ K $ is a field.
\item $ K\sb{x_1, \dots, x_n} $ is the ring of polynomials in $ n $ variables.
\item $ \A^n $ is $ K^n $ as a set.
\end{itemize}
\end{notation}

\begin{definition}
Let $ S \subseteq K\sb{x_1, \dots, x_n} $ then
$$ Z\rb{S} = \cb{x \in \A^n \mid \forall f \in S, \ f\rb{x} = 0} $$
is called the \textbf{zero locus} of $ S $. Subsets of $ \A^n $ that are of this form are called \textbf{affine varieties}.
\end{definition}

\begin{remark}
Some authors call \textbf{algebraic set} the object $ Z\rb{S} $. We will not follow this notation.
\end{remark}

\begin{example}
\hfill
\begin{itemize}
\item Single points $ p = \rb{p_1, \dots, p_n} $. $ p = Z\rb{S} $ where $ S = \cb{x_1 - p_1, \dots, x_n - p_n} $.
\item $ \A^n = Z\rb{0} $.
\item $ \emptyset = Z\rb{1} $.
\item Subspaces of $ \A^n = K^n $.
\item If $ X = Z\rb{f_1, \dots, f_n} \subseteq \A^n $ and $ Y = Z\rb{g_1, \dots, g_m} \subseteq \A^n $ are affine varieties then
$$ X \times Y = Z\rb{f_1, \dots, f_n, g_1, \dots, g_m} \subseteq \A^{n + m} $$
is a variety.
\end{itemize}
\end{example}

\begin{remark}
If $ S \subseteq K\sb{x_1, \dots, x_n} $ and $ I = \ab{S} $ then $ Z\rb{S} = Z\rb{I} $.
\end{remark}

\begin{theorem}[Hilbert's basis theorem]
If $ R $ is Noetherian then $ R\sb{x} $ is Noetherian.
\end{theorem}

\begin{corollary}
Every ideal in $ K\sb{x_1, \dots, x_n} $ is finitely generated.
\end{corollary}

\begin{definition}
Let $ X \subseteq \A^n $ then
$$ I\rb{X} = \cb{f \in K\sb{x_1, \dots, x_n} \mid \forall x \in X, \ f\rb{x} = 0}. $$
\end{definition}

\begin{example}
$ I\rb{p} = I\rb{\rb{p_1, \dots, p_n}} = \ab{x_1 - p_1, \dots, x_n - p_n} $.
\end{example}

Goal is
$$
\begin{array}{rcl}
\cb{\text{affine varieties in} \ \A^n} & \leftrightarrow & \cb{\text{ideals of} \ K\sb{x_1, \dots, x_n}} \\
X & \mapsto & I\rb{X} \\
Z\rb{J} & \mapsfrom & J
\end{array}.
$$
$ Z\rb{I\rb{X}} = X $ but $ I\rb{Z\rb{J}} \supseteq J $.

\begin{example}
$ J = \ab{x^2 + 1} \subseteq \R\sb{x} = I\rb{\emptyset} = I\rb{Z\rb{x^2 + 1}} $.
\end{example}

\begin{proposition}
\hfill
\begin{itemize}
\item If $ X \subseteq Y $ then $ I\rb{Y} \subseteq I\rb{X} $. If $ I \subseteq J $ then $ Z\rb{J} \subseteq Z\rb{I} $.
\item $ X \subseteq Z\rb{I\rb{X}} $ and $ S \subseteq I\rb{Z\rb{S}} $.
\item If $ X $ is affine then $ Z\rb{J\rb{X}} = X $. If $ X = Z\rb{S} $ then take $ Z $ of $ S \subseteq I\rb{Z\rb{S}} $.
\end{itemize}
\end{proposition}

\begin{example}
Let $ J \subseteq \C\sb{x} $. $ J = \ab{f} $, where $ f = \rb{x - x_1}^{k_1} \dots \rb{x - x_n}^{k_n} $.
\end{example}

\begin{definition}
Let $ I \subseteq K\sb{x_1, \dots, x_n} $ be an ideal.
$$ I \subseteq \sqrt{I} = \cb{f \in K\sb{x_1, \dots, x_n} \mid \exists n \in \N, \ f^n \in I}. $$
If $ \sqrt{I} = I $, we say $ I $ is a \textbf{radical ideal}. (Exercise: $ \sqrt{I} $ is an ideal, $ I \subseteq \sqrt{I} $, and $ \sqrt{I} = \bigcap_{p \ \text{prime}} p $)
\end{definition}

\begin{theorem}[Hilbert's Nullstellensatz]
$ I\rb{Z\rb{J}} = \sqrt{J} $. If $ \sqrt{J} = J $ then
$$
\begin{array}{rcl}
\cb{\text{affine varieties}} & \leftrightarrow & \cb{\text{radical ideals}} \\
X & \mapsto & I\rb{X} \\
Z\rb{J} & \mapsfrom & J
\end{array}.
$$
\end{theorem}

\lecture{2}{Monday}{14/01/19}

\begin{proposition}
\hfill
\begin{enumerate}
\item $ Z\rb{S} \cup Z\rb{T} = Z\rb{ST} $.
\item $ \bigcap_i Z\rb{S_i} = Z\rb{\bigcup_i S_i} $.
\item $ Z\rb{0} = \A^n $ and $ Z\rb{1} = \emptyset $.
\end{enumerate}
\end{proposition}

\begin{proof}
\hfill
\begin{itemize}
\item[1.] If $ p \in Z\rb{S} \cup Z\rb{T} $, then $ f\rb{p} = 0 $ for $ f \in S $ or $ f \in T $, so $ f\rb{x} = 0 $ for $ f \in ST $, where
$$ ST = \cb{\sum_{i \in I, \ I \ \text{finite}} s_it_i} \subseteq S \cap T, $$
with equality if $ S + T = R $. If $ p \in Z\rb{ST} $, there exists $ f $ such that $ f\rb{p} = 0 $ for $ f \in S $ or $ f\rb{p} = 0 $ for $ f \in T $, so $ p \in Z\rb{S} \cup Z\rb{T} $.
\end{itemize}
\end{proof}

\begin{definition}
The \textbf{Zariski topology} on $ \A^n $ is the topology generated by closed sets of the form $ Z\rb{S} $. By the above proposition this is a topology.
\end{definition}

\begin{example}
$ \A^1 $ is not Hausdorff.
\end{example}

\begin{definition}
A topological space $ X $ is \textbf{irreducible} if it cannot be expressed as a union $ X = A \cup B $, where $ A $ and $ B $ are proper and closed subsets. $ \emptyset $ is not considered irreducible.
\end{definition}

\begin{example}
$ \A^1 $.
\end{example}

\begin{example}
Any non-empty open set of irreducible $ X $ is dense and irreducible. Suppose $ A $ is open then $ X = A^c \cup \overline{A} $. Since $ X $ is irreducible then $ A^c = X $, a contradiction, or $ \overline{A} = X $. Suppose $ A $ is reducible. Let $ A = \rb{A \cap B} \cup \rb{A \cap C} $, where $ B $ and $ C $ are closed. Then $ X = A^c \cup \rb{B \cup C} $. $ A^c = X $ or $ B \cup C = X $, which are contradictions.
\end{example}

\begin{example}
If $ A $ is irreducible then $ \overline{A} $ is also irreducible. Suppose $ \overline{A} $ is not irreducible. $ \overline{A} = \rb{\overline{A} \cap B} \cup \rb{\overline{A} \cap C} $. Take $ \bigcap A $, $ A = \rb{A \cap B} \cup \rb{A \cap C} $, a contradiction.
\end{example}

\begin{definition}
An affine variety is \textbf{irreducible} if it is irreducible as a topological space.
\end{definition}

\begin{remark}
A \textbf{quasi-affine variety} is an open set of an affine variety.
\end{remark}

\begin{proposition}
\hfill
\begin{enumerate}
\item $ I\rb{X \cup Y} = I\rb{X} \cap I\rb{Y} $.
\item $ Z\rb{I\rb{X}} = \overline{X} $ for any $ X \subseteq \A^n $.
\end{enumerate}
\end{proposition}

\begin{proof}
\hfill
\begin{enumerate}
\item If $ f \in I\rb{X \cup Y} $ then $ f\rb{p} = 0 $ for all $ p \in X \cup Y $, so $ f \in I\rb{X} $ and $ f \in I\rb{Y} $.
\item We know that $ X \subseteq Z\rb{I\rb{X}} $ hence $ \overline{X} \subseteq Z\rb{I\rb{X}} $. Now, let $ Y $ be a closed set containing $ X $, that is $ X \subseteq Y $. Then
$$ I\rb{Y} \subset I\rb{X} \qquad \implies \qquad Z\rb{I\rb{X}} \subset Z\rb{I\rb{X}} = Y, $$
so any closed set containing $ Y $ contains $ Z\rb{I\rb{X}} $.
\end{enumerate}
\end{proof}

\begin{proposition}
$ X $ is irreducible if and only if $ I\rb{X} $ is prime.
\end{proposition}

\begin{proof}
\hfill
\begin{itemize}
\item[$ \implies $] Let $ f, g \in I\rb{X} $.
$$ X \subseteq Z\rb{fg} = Z\rb{f} \cup Z\rb{g} \qquad \implies \qquad X = \rb{X \cap Z\rb{f}} \cup \rb{X \cap Z\rb{g}}. $$
$ Z\rb{f} \subseteq X $, so $ f \in I\rb{X} $, or $ Z\rb{g} \subseteq X $, so $ g \in I\rb{X} $.
\item[$ \impliedby $] Exercise.
\end{itemize}
\end{proof}

\begin{example}
$ \A^n $.
\end{example}

\begin{definition}
If $ X \subseteq \A^n $, the \textbf{coordinate ring} of $ X $ is
$$ A\rb{X} = \dfrac{A}{I\rb{X}} = \dfrac{K\sb{x_1, \dots, x_n}}{I\rb{X}}. $$
\end{definition}

\lecture{3}{Tuesday}{15/01/19}

\begin{example}
Let $ f \in K\sb{x_1, \dots, x_n} $ be irreducible. If $ n = 3 $, $ Z\rb{f} $ is a surface. If $ n = 2 $, $ Z\rb{f} $ is a curve.
\end{example}

\begin{example}
Let $ y - x^2 \in K\sb{x, y} $. Then
$$ \function{A\rb{X} = \dfrac{K\sb{x, y}}{\ab{y - x^2}} \cong K\sb{x, x^2}}{K\sb{x}}{\sum_{i, j} a_{ij}x^ix^{2j} = \sum_{i, j} a_{ij}x^{2j + i}}{\sum_n b_nx^n}. $$
\end{example}

\begin{example}
Let $ xy - 1 \in K\sb{x, y} $. Then
$$ A\rb{X} = \dfrac{K\sb{x, y}}{\ab{xy - 1}} \cong K\sb{x, \dfrac{1}{x}}. $$
$ A\rb{X} $ cannot be $ K\sb{x} $.
\end{example}

\begin{definition}
A \textbf{Noetherian} topological space $ X $ is a topological space such that if
$$ C_1 \supseteq C_2 \supseteq \dots $$
is a decreasing chain of closed sets then there is a $ k $ such that $ C_k = C_{k + 1} = \dots $.
\end{definition}

\begin{example}
$ \A^n $. Recall that if $ A \subset B $ then $ I\rb{B} \subset I\rb{A} $. So using the definition above,
$$ I\rb{C_1} \subseteq I\rb{C_2} \subseteq \dots. $$
Since $ K\sb{x_1, \dots, x_n} $ is Noetherian then $ I\rb{C_i} $ stabilises. So $ I\rb{C_k} = I\rb{C_{k + 1}} = \dots $, but taking $ Z $, we recover $ C_k $ so $ C_k $ stabilises as well.
\end{example}

\begin{theorem}
If $ X $ is Noetherian then any non-empty closed subset can be expressed as a finite union of irreducible closed sets $ X = Y_1 \cup \dots \cup Y_n $. Moreover, if we require that $ Y_i \subseteq Y_j $ then this expression is unique.
\end{theorem}

\begin{proof}
Let $ C $ be the collection of closed sets that do not satisfy that property. Let $ Y $ be a minimum closed inside $ C $, in particular $ Y $ is reducible, so $ Y = Y' \cup Y'' $, for $ Y', Y'' $ closed. Hence $ Y', Y'' \not\subset C $, so they can be expressed as a finite union of irreducibles, a contradiction. If $ Y_i \not\subset Y_j $, then suppose
$$ Y_1 \cup \dots \cup Y_n = X_1 \cup \dots \cup X_n. $$
Then $ Y_1 \subset X_1 \cup X_n $, in particular $ Y_1 = \bigcup_j \rb{Y_1 \cap X_j} $, so there is a $ j $ such that $ Y_1 \cap X_j = Y_1 $, so $ Y_1 \subset X_j $. We can assume $ j = 1 $ and repeat the same argument to find that $ Y_1 = X_1 $, so consider $ \overline{Y \setminus Y_1} = Y_2 \cup \dots \cup Y_n $. But
$$ Y_2 \cup \dots \cup Y_n = X_2 \cup \dots \cup X_n, $$
and the result follows by induction.
\end{proof}

\begin{corollary}
Any affine variety in $ \A^n $ can be expressed equally as a union of irreducible algebraic varieties.
\end{corollary}

\begin{definition}
The \textbf{dimension} of a topological space is the supremum of $ n $ where
$$ Y_0 \subset \dots \subset Y_n $$
is a sequence of irreducible closed sets.
\end{definition}

\begin{example}
Dimension of $ \A^1 $ is one.
\end{example}

\begin{definition}
Let $ A $ be a ring and $ \pp $ be a prime ideal, then the \textbf{height} of $ \pp $ is the supremum of $ n $ where
$$ \pp_1 \subset \dots \subset \pp_n \subset \pp, $$
where $ \pp_i $ are prime. The \textbf{Krull dimension} of $ A $ is
$$ \sup_{\pp \ \text{prime}} height\rb{\pp}. $$
\end{definition}

\begin{proposition}
If $ Y $ is affine then $ \dim\rb{Y} = \dim\rb{A\rb{Y}} $.
\end{proposition}

\begin{proof}
Let $ C $ be a closed and irreducible set $ C \subset Y $, then $ I\rb{C} \supset I\rb{Y} $, then $ I\rb{C} $ is prime.
\end{proof}

\begin{proposition}
Let $ K $ be a field and $ B $ be an integral domain which is a finitely generated algebra, then
\begin{itemize}
\item $ \dim\rb{B} $ is the transcendence degree of $ K\rb{B} $ over $ K $, and
\item if $ \pp \subseteq B $ is prime, then
$$ height\rb{\pp} + \dim\rb{\dfrac{B}{\pp}} = \dim\rb{B}. $$
\end{itemize}
\end{proposition}

\begin{proof}
Atiyah Macdonald chapter 11.
\end{proof}

\begin{proposition}[Krull Hauptidealsatz]
Let $ A $ be a Noetherian ring and $ f \in A $ not a zero divisor and not a unit. Then every prime ideal containing $ f $ has height one.
\end{proposition}

\begin{proof}
Atiyah Macdonald page 122.
\end{proof}

\lecture{4}{Friday}{18/01/19}

\begin{proposition}
A Noetherian integral domain $ A $ is a UFD if and only if every prime ideal $ I $ of height one is principal.
\end{proposition}

\begin{theorem}
An irreducible variety $ Y \subseteq \A^n $ has dimension $ n - 1 $ if and only if $ Y = Z\rb{f} $ where $ f $ is an irreducible polynomial in $ K\sb{x_1, \dots, x_n} $.
\end{theorem}

\begin{proof}
\hfill
\begin{itemize}
\item[$ \implies $] If $ Y $ has dimension $ n - 1 $ then $ I\rb{Y} $ has height one, by the above proposition $ I\rb{Y} = \ab{f} $, so $ Y = Z\rb{f} $.
\item[$ \impliedby $] Let $ I = I\rb{Y} $ then $ I $ is prime, by the Krull Hauptidealsatz we have that $ I $ has height one, so $ \dim\rb{Y} = n - 1 $.
\end{itemize}
\end{proof}

\section{Projective varieties}

\begin{definition}
The \textbf{projective space} $ \P^n $ is defined as
$$ \P^n = \dfrac{\A^{n + 1} \setminus \cb{0}}{\cb{x \sim \lambda x \mid \lambda \in K^*}}. $$
A point in $ \P^n $ is written as $ \sb{a_0 : \dots : a_n} = \overline{\rb{a_0, \dots, a_n}} $.
\end{definition}

\begin{definition}
A \textbf{graded ring} $ R $ is a ring together with a decomposition
$$ R = \bigoplus_{d > 0} R_d, $$
where $ R_d $ are abelian groups and $ R_k \cdot R_t \subseteq R_{k + t} $.
\end{definition}

\begin{example}
$ K\sb{x_0, \dots, x_n} $ is a graded ring, where $ R_d $ are monomials of degree $ d $.
\end{example}

\begin{notation}
Let $ A $ be $ K\sb{x_0, \dots, x_n} $ without the grading and $ S $ be $ K\sb{x_0, \dots, x_n} $ as a graded ring.
\end{notation}

\begin{definition}
An ideal $ I \subseteq S $ is \textbf{homogeneous} if
$$ I = \bigoplus_{d \ge 0} \rb{I \cap S_d}. $$
If $ f = f_0 + \dots + f_d $, then $ f_i \in I $.
\end{definition}

\begin{remark}
$ I $ is homogeneous if and only if $ I = \ab{f_0, \dots, f_n} $, where $ f_i $ are homogeneous.
\end{remark}

\begin{lemma}
If $ I, J $ are homogeneous then
\begin{enumerate}
\item $ I + J $ is homogeneous,
\item $ IJ $ is homogeneous,
\item $ I \cap J $ is homogeneous, and
\item $ \sqrt{I} $ is homogeneous.
\end{enumerate}
\end{lemma}

\begin{proof}
\hfill
\begin{itemize}
\item[4.] Let $ f = f_0 + \dots + f_d \in \sqrt{I} $ then
$$ f^n = \rb{f_0 + \dots + f_d}^n = f_d^n + \dots \in I \qquad \implies \qquad f_d^n \in I \qquad \implies \qquad f_d \in \sqrt{I}, $$
so $ f - f_d \in \sqrt{I} $, by induction $ f_i \in \sqrt{I} $.
\end{itemize}
\end{proof}

\begin{definition}
If $ f $ is homogeneous of degree $ k $ then
$$ f\rb{\lambda \cdot x} = \lambda^k \cdot f\rb{x}, $$
in particular $ f\rb{x} = 0 $ if and only if $ f\rb{\lambda \cdot x} = 0 $, so it makes sense to define
$$ Z\rb{f} = \cb{x \in \P^n \mid f\rb{x} = 0}. $$
More generally, if $ I \subseteq S $ is a homogeneous ideal then
$$ Z\rb{I} = \cb{x \in \P^n \mid f \in I \ \text{homogeneous}, \ f\rb{x} = 0}. $$
\end{definition}

\begin{definition}
A subset $ X \subseteq \P^n $ is called a \textbf{projective variety} if $ X = Z\rb{T} $ for some homogeneous ideal $ T $.
\end{definition}

\begin{proposition}
\hfill
\begin{itemize}
\item $ Z\rb{S} \cup Z\rb{T} = Z\rb{ST} $.
\item $ \bigcap_\alpha Z\rb{S_\alpha} = Z\rb{\bigcup_\alpha S_\alpha} $.
\item $ Z\rb{0} = \P^n $ and $ Z\rb{1} = \emptyset $.
\end{itemize}
\end{proposition}

\begin{definition}
We define the \textbf{Zariski topology} on $ \P^n $ by taking closed sets to be $ Z\rb{T} $ for some $ T $.
\end{definition}

\begin{definition}
\hfill
\begin{itemize}
\item A projective variety is \textbf{irreducible} if it is an irreducible topological space.
\item An open subset of a projective variety is called a \textbf{quasi-projective variety}.
\item The \textbf{dimension} of a projective variety is its dimension as a topological space.
\item If $ T \subseteq S $ then
$$ I\rb{T} = \ab{f \in S \mid f \ \text{homogeneous}, \ \forall p \in T, \ f\rb{p} = 0}. $$
\end{itemize}
\end{definition}

\begin{definition}
If $ X $ is a projective variety the \textbf{homogeneous coordinate ring} is
$$ S\rb{X} = \dfrac{S}{I\rb{X}}. $$
\end{definition}

\begin{definition}
If $ f \in S $ is linear and homogeneous, we call $ Z\rb{f} $ a \textbf{hyperplane}.
\end{definition}

\lecture{5}{Monday}{21/01/19}

\begin{proposition}
$$ \function[\phi_i]{U_i = \P^n \setminus Z\rb{x_i}}{\A^n}{\sb{x_0 : \dots : x_n}}{\rb{\dfrac{x_0}{x_i}, \dots, \dfrac{x_n}{x_i}}} $$
is a homeomorphism in the Zariski topology.
\end{proposition}

\begin{proof}
Let $ \phi = \phi_0 $ and $ U = U_0 $, let $ C \subseteq \A^n $ be a closed set then we claim that $ \phi^{-1}\rb{C} $ is closed. Indeed, let $ C = Z\rb{S} $, then $ \phi^{-1}\rb{C} = Z\rb{S'} \cup U $ where
$$ S' = \cb{x_0^d \cdot f\rb{\dfrac{x_1}{x_0}, \dots, \dfrac{x_n}{x_0}} \ \Bigg| \ f \in S}. $$
Similarly, let $ A \subseteq U $ is closed, we claim that $ \phi\rb{A} $ is closed. Let $ \overline{A} $ be its closure in $ \P^n $, then $ \overline{A} = Z\rb{B} $, so $ \phi\rb{A} = Z\rb{B'} $ where
$$ B' = \cb{f\rb{1, x_1, \dots, x_n} \mid f \in B}. $$
So we conclude that $ \phi $ is a homeomorphism.
\end{proof}

\begin{note*}
$ \ab{1} = S $ and $ \ab{x_0, \dots, x_n} \subsetneq S $ map to $ \emptyset $ under $ Z $. So in order to have a one-to-one correspondence we need the following.
\begin{itemize}
\item $ Z\rb{I} = \emptyset $ if and only if $ \sqrt{I} \supseteq \ab{x_0, \dots, x_n} $. If we consider $ Z\rb{I} $ in $ \A^{n + 1} $, note that $ x \in Z\rb{I} $ if and only if $ \lambda x \in Z\rb{I} $. So $ Z\rb{I} = \emptyset $ if and only if $ Z\rb{I} \subseteq \cb{0} $. So $ \sqrt{I} \supseteq \ab{x_0, \dots, x_n} $.
\item $ I\rb{Z\rb{J}} = \sqrt{J} $ if $ Z\rb{J} \ne \emptyset $, since $ I\rb{Z\rb{J}} = I\rb{Z_a\rb{J}} = \sqrt{J} $.
\end{itemize}
\end{note*}

\begin{corollary}
$$ \correspondence{\text{projective varieties}}{\text{homogeneous radical ideals not} \ \ab{x_0, \dots, x_n}}, $$
$$ \correspondence{\text{irreducible projective varieties}}{\text{homogeneous radical prime ideals}}. $$
\end{corollary}

\begin{example}
$ \P^n $ is irreducible.
\end{example}

\begin{proposition}
\hfill
\begin{itemize}
\item $ \P^n $ is Noetherian, that is satisfies the descending chain condition.
\item Every projective variety can be written as a unique union of irreducible projective varieties. We call \textbf{irreducible components} the irreducible varieties in that decomposition.
\end{itemize}
\end{proposition}

\begin{theorem}
Let $ Y \subseteq \P^n $ be an irreducible projective variety. Then
$$ \dim\rb{S\rb{Y}} = \dim\rb{Y} + 1. $$
\end{theorem}

\begin{proof}
Let
$$ \function[\phi_i]{U = \P^n \setminus Z\rb{x_i}}{\A^n}{\sb{x_0 : \dots : x_n}}{\rb{\dfrac{x_0}{x_i}, \dots, \dfrac{x_n}{x_i}}}, $$
and $ Y_i = \phi_i\rb{Y \cap U_i} $. Let
$$ \function{K\sb{x_1, \dots, x_n}}{\rb{S\rb{Y}_{x_i}}_0}{f\rb{x_1, \dots, x_n}}{\dfrac{x_i^{\partial f}f\rb{\dfrac{x_1}{x_i}, \dots, \dfrac{x_n}{x_i}}}{x_i^{\partial f}}}, $$
then
$$ A\rb{Y_i} = \dfrac{K\sb{x_1, \dots, x_n}}{I\rb{Y_i}} \cong \rb{S\rb{Y}_{x_i}}_0, $$
moreover $ S\rb{Y}_{x_i} \cong A\rb{Y_i}\sb{x_i, x_i^{-1}} $. So
$$ \dim\rb{S\rb{Y}} = \dim\rb{S\rb{Y}_{x_i}} = \dim\rb{A\rb{Y_i}\sb{x_i, x_i^{-1}}} = tra\rb{K\rb{Y_i}\rb{x_i}} = \dim\rb{Y_i} + 1. $$
Therefore if $ Y_i \ne \emptyset $, $ \dim\rb{Y_i} = \dim\rb{S\rb{Y}} - 1 $ for all $ i $, but since $ U_i $ cover $ Y $ we have $ \dim\rb{Y} = \max\cb{\dim\rb{Y_i}} $. (Exercise: if $ \cb{U_n}_n $ is a finite cover of a topological space $ Y $ then $ \dim\rb{Y} = \max\cb{\dim\rb{Y_i}} $) Since $ \dim\rb{Y_i} $ are the same if $ Y_i \ne \emptyset $, we conclude that $ \dim\rb{Y} = \dim\rb{Y_d} $ for some $ d $.
\end{proof}

\lecture{6}{Tuesday}{22/01/19}

\begin{proposition}
Every Noetherian topological space is compact.
\end{proposition}

\begin{proof}
Let $ X $ be a Noetherian topological space and let $ \cb{U_n} $ be a cover of $ X $. So consider $ C $, the collection of the union of finitely many open sets of $ \cb{U_n} $. Since $ X $ is Noetherian $ C $ has a maximum element, say $ U_1 \cup \dots \cup U_n $. If $ U_1 \cup \dots \cup U_n \subsetneq X $ then there is $ x \in X $ not in the union, and we can find another $ U_{\alpha_0} \ni x $. But then
$$ U_1 \cup \dots \cup U_n \cup U_{\alpha_0} \supsetneq U_1 \cup \dots \cup U_n, $$
a contradiction. So $ X = U_1 \cup \dots \cup U_n $.
\end{proof}

\begin{corollary}
$ \P^n $, $ \A^n $, affine varieties, and projective varieties are all compact in the Zariski topology.
\end{corollary}

\begin{definition}
A variety $ X $ is \textbf{complete} if for any other variety $ Y $, the projection $ X \times Y \to Y $ is closed.
\end{definition}

\begin{example}
$ \P^n $ is complete. $ \A^n $ is not complete.
\end{example}

\pagebreak

\section{Morphisms}

\begin{definition}
Suppose $ Y $ is a quasi-affine variety and $ p \in Y $. We say that a function $ f : Y \to \A^1 $ is \textbf{regular} at $ p $ if there are $ g, h \in K\sb{x_1, \dots, x_n} $ and $ U \ni p $ such that $ f = g / h $ in $ U $ with $ h \ne 0 $. A function is \textbf{regular} if it is regular for every $ p \in Y $.
\end{definition}

\begin{example}
Local is not global. Let $ X = Z\rb{x_1x_4 - x_2x_3} \subseteq \A^4 $ and $ U = X \setminus Z\rb{x_2, x_4} $. Then
$$ \function[\phi]{U}{\A^1}{\rb{x_1, x_2, x_3, x_4}}{
\begin{cases}
\dfrac{x_1}{x_2} & x_2 \ne 0 \\
\dfrac{x_3}{x_4} & x_4 \ne 0
\end{cases}
} $$
is a regular function.
\end{example}

\begin{definition}
Let $ Y $ be a quasi-projective variety, $ f : Y \to \A^1 $, and $ p \in Y $. We say that $ f $ is \textbf{regular} at $ p $ if there are $ g, h $ homogeneous polynomials of the same degree and an open set $ U \ni p $ such that $ f = g / h $ on $ U $ and $ h \ne 0 $.
\end{definition}

\begin{lemma}
A regular function is continuous.
\end{lemma}

\begin{proof}
It is enough to show that $ f^{-1}\rb{p} $ is closed. Since $ f $ is regular $ f = g / h $ on some neighbourhood $ U $, then $ f^{-1}\rb{p} \cap U = Z\rb{g - ph} \cap U $.
\end{proof}

\begin{remark}
If $ X $ is irreducible then $ f = g $ on $ U \subseteq X $, then $ f = g $ on $ X $. Because the set where $ f - g = 0 $ is closed and dense.
\end{remark}

\begin{definition}
We will use the term \textbf{variety} to denote an affine, quasi-affine, projective, or quasi-projective variety.
\end{definition}

\begin{definition}
A \textbf{morphism} is $ f : X \to Y $ if $ f $ is continuous and for every $ U \subseteq Y $ and every function $ g : U \to \A^1 $ the composition $ g \circ f $ is regular.
\end{definition}

\begin{remark}
\hfill
\begin{itemize}
\item Let $ f : X \to Y $ and $ g : Y \to Z $ then the composition $ g \circ f $ of these two morphisms is the composition of $ f $ and $ g $ as functions.
\item A morphism $ f : X \to Y $ is an \textbf{isomorphism} if there is a morphism $ g : Y \to X $ such that $ f \circ g = id $ and $ g \circ f = id $.
\end{itemize}
\end{remark}

\begin{definition}
Let $ X $ be a variety. Denote the set of all regular functions of $ X $ by $ \OO\rb{X} $. If $ p \in X $ the \textbf{local ring} at $ p \in X $ is
$$ \OO_p = \lim_{\xrightarrow[U \ni p]{}}\rb{\OO\rb{U}}. $$
An element of $ \OO_p $ is a pair $ \rb{U, f} $, where $ p \in U $ and $ f $ is regular at $ p $, moreover $ \rb{U, f} \sim \rb{V, g} $ if $ f = g $ on $ U \cap V $.
\end{definition}

\lecture{7}{Friday}{25/01/19}

\begin{definition}
Let $ Y $ be an irreducible variety, the \textbf{function field} $ K\rb{Y} $ of $ Y $ is the field whose elements are pairs $ \rb{U, f} $ where $ U $ is open and $ f $ is regular on $ U $, and
$$ \rb{U, f} + \rb{V, g} = \rb{U \cap V, f + g}. $$
\end{definition}

\begin{remark}
\hfill
\begin{itemize}
\item $ K\rb{Y} $ is indeed a field for if $ \rb{U, f} \ne 0 $ then $ U^{-1} = U \setminus Z\rb{f} $, so $ \rb{U^{-1}, 1 / f} $ is the inverse to $ \rb{U, f} $.
\item $ K\rb{Y} $ is the quotient field of $ A\rb{Y} $ or $ S\rb{Y} $.
\item $ \OO\rb{Y} \hookrightarrow \OO_p \hookrightarrow K\rb{Y} $ for all $ p \in Y $.
\end{itemize}
\end{remark}

\begin{theorem}
If $ Y \subseteq \A^n $ is an irreducible affine variety with coordinate ring $ A\rb{Y} $ then
\begin{enumerate}
\item $ \OO\rb{Y} = A\rb{Y} $,
\item for all $ p \in Y $, if $ \mm_p = \cb{f \in A\rb{Y} \mid f\rb{p} = 0} $ then we have a one-to-one correspondence
$$ \correspondence{\text{points of} \ Y}{\text{maximal ideals of} \ A\rb{Y}}, $$
\item for all $ p \in Y $, $ \OO_p \cong A\rb{Y}_{\mm_p} $ and $ \dim\rb{\OO_p} = \dim\rb{Y} $, and
\item $ K\rb{Y} $ is the quotient field of $ A\rb{Y} $.
\end{enumerate}
\end{theorem}

\begin{proof}
\hfill
\begin{enumerate}
\item Notice that there is a natural map $ A \to \OO\rb{Y} $ with kernel $ I\rb{Y} $, so there is an injection $ A\rb{Y} \hookrightarrow \OO\rb{Y} $, that is
$$ A\rb{Y} \subseteq \OO\rb{Y} \subseteq \bigcap_{p \in Y} \OO_p = \bigcap_{\mm_p} A\rb{Y}_{\mm_p} = A\rb{Y}, $$
so $ A\rb{Y} = \OO\rb{Y} $.
\item We know that points of $ Y $ correspond to maximal ideals $ \mm_p \supseteq I\rb{Y} $. Taking the quotient, we get maximal ideals inside $ A\rb{Y} $.
\item There is a natural map $ A\rb{Y}_{\mm_p} \to \OO_p $, which is injective by $ \alpha : A\rb{Y} \hookrightarrow \OO\rb{Y} $, and it is surjective by definition of $ \OO_p $. Moreover,
$$ \dim\rb{\OO_p} = \dim\rb{A_p}_{\mm_p} = height\rb{\mm_p} = \dim\rb{Y}. $$
\item The quotient field of $ A\rb{Y} $ is the quotient field of $ \OO_p $ for all $ p $, by $ 3 $, which is $ K\rb{Y} $ by definition.
\end{enumerate}
\end{proof}

\begin{theorem}
Let $ Y \subseteq \P^n $ be irreducible and projective. Then
\begin{enumerate}
\item $ \OO\rb{Y} = K $,
\item for all $ p \in Y $, $ \mm_p $ as before, $ \OO_p \cong \rb{S\rb{Y}_{\mm_p}}_0 $, and
\item $ K\rb{Y} \cong \rb{S\rb{Y}_{\rb{0}}}_0 $.
\end{enumerate}
\end{theorem}

\begin{proof}
Recall that
$$ \function[\phi_i]{U_i = \P^n \setminus Z\rb{x_i}}{\A^n}{\sb{x_0 : \dots : x_n}}{\rb{\dfrac{x_0}{x_i}, \dots, \dfrac{x_n}{x_i}}} $$
gives $ \phi_i^* : A\rb{Y_i} \cong \rb{S\rb{Y}_{x_i}}_0 $ and $ Y_i = \phi_i\rb{Y \cap U_i} $.
\begin{enumerate}
\item $ K \subseteq \OO\rb{Y} $. Take $ f \in \OO\rb{Y} $, so $ f $ is regular at each $ Y_i $, but $ \OO\rb{Y_i} \cong A\rb{Y_i} $, also by $ \phi_i^* $, $ A\rb{Y_i} \cong \rb{S\rb{Y}_{x_i}}_0 $. Thus $ f = g_i / x_i^{n_i} $, where $ n_i = \deg\rb{g_i} $, in particular $ x_i^{n_i}f \in S\rb{Y}_{n_i} $. Now, set $ N \ge \sum_i n_i $, then $ S\rb{Y}_N \cdot f \subseteq S\rb{Y}_N $, so we can iterate this process to obtain $ S\rb{Y}_N \cdot f^q \subseteq S\rb{Y}_N $. In particular $ x_0^Nf \subset S $, hence $ S\rb{Y}\sb{f} $ is contained in $ x_0^{-N}S\rb{Y} $. Therefore $ f $ is integral since $ S\rb{Y}\sb{f} $ is finitely generated. There are $ a_i \in S $ such that
$$ f^k + a_1f^{k - 1} + \dots + a_k = 0. $$
Since $ f $ is homogeneous of degree zero we can take the constant terms of $ a_i $ and still have an equation, hence $ a_i \in K $.
\item Let $ p \in Y $, then $ p \in Y_i $, by the previous theorem we know that $ \OO_p \cong A\rb{Y_i}_{\mm_p} $. By $ \phi_i^* $, $ \OO_p \cong \rb{\rb{S\rb{Y}_{x_i}}_{\mm_p}}_0 $, but since $ x_i \notin \mm_p $, hence $ \OO_p \cong \rb{S\rb{Y}_{\mm_p}}_0 $.
\item Recall that the quotient field of $ Y $ is $ K\rb{Y} = K\rb{Y_i} $, but $ K\rb{Y_i} $ is the quotient field of the coordinate ring $ A\rb{Y_i} $, by $ \phi_i^* $, this is $ \rb{S\rb{Y}_{\rb{0}}}_0 $.
\end{enumerate}
\end{proof}

\lecture{8}{Monday}{28/01/19}

\begin{proposition}
Let $ X $ be an irreducible variety and $ Y $ be an irreducible affine variety, then we have a bijection
$$ \alpha : Hom\rb{X, Y} \xrightarrow{\sim} Hom\rb{A\rb{Y}, \OO\rb{X}}, $$
the set of morphisms from $ X $ to $ Y $ to the set of $ K $-algebra homomorphisms.
\end{proposition}

\begin{proof}
Given a morphism $ \phi : X \to Y $, by definition of morphism, $ \phi $ takes regular functions at $ Y $ to regular functions at $ X $. So if $ f \in A\rb{Y} $ then $ \phi \circ f \in \OO\rb{X} $. Conversely, let $ h : A\rb{Y} \to \OO\rb{X} $ be a homomorphism of $ K $-algebras. Recall that $ A\rb{Y} = A / I\rb{Y} = k\sb{x_1, \dots, x_n} / I\rb{Y} $. Take $ \overline{x_i} \in A\rb{Y} $ and let $ y_i = h\rb{\overline{x_i}} \in \OO\rb{X} $ and define
$$ \function[\psi]{X}{\A^n}{p}{\rb{y_1\rb{p}, \dots, y_n\rb{p}}}. $$
We claim that $ Im\rb{\psi} \subseteq Y $, but since $ Y = Z\rb{I\rb{Y}} $, it is enough to show that if $ f \in I\rb{Y} $ then $ f\rb{\psi\rb{p}} = 0 $.
$$ f\rb{\psi\rb{p}} = f\rb{y_1\rb{p}, \dots, y_n\rb{p}} = f\rb{h\rb{\overline{x_1}\rb{p}}, \dots, h\rb{\overline{x_n}\rb{p}}} = h\rb{f\rb{x_1, \dots, x_n}}\rb{p} = 0. $$
\end{proof}

\begin{lemma}
If $ X, Y $ are as before then $ \psi : X \to Y $ is a morphism if and only if $ \psi_i = x_i \circ \psi $ are regular functions.
\end{lemma}

\begin{proof}
Suppose $ \psi_i $ are regular functions, then if $ p $ is a polynomial $ p \circ \psi $ is regular, but since regular functions are quotients of polynomials, we conclude that $ f \circ \psi $ is regular for any regular function $ f $.
\end{proof}

\begin{corollary}
If $ X, Y $ are affine then $ X \cong Y $ if and only if $ A\rb{X} \cong A\rb{Y} $.
\end{corollary}

\begin{corollary}
The correspondence $ X \mapsto A\rb{X} $ induces an arrow reversing correspondence between the category of affine varieties and the category of $ K $-integral domains.
\end{corollary}

\lecture{9}{Tuesday}{29/01/19}

Lecture 9 is a problem class.

\lecture{10}{Friday}{01/02/19}

Lecture 10 is a problem class.

\pagebreak

\section{Rational maps}

\lecture{11}{Monday}{04/02/19}

\begin{definition}
Let $ X, Y $ be varieties. A \textbf{rational map} $ f : X \dashrightarrow Y $ is a pair $ \rb{U, f_U} $ where $ U \subseteq X $ is open and $ f_U $ is a morphism on $ U $ and we identify $ \rb{U, f_U} \sim \rb{V, g_V} $ if $ f_U = g_V $ on $ U \cap V $.
\end{definition}

\begin{lemma}
If $ X, Y $ are varieties and $ \phi, \psi : X \to Y $ such that $ \phi = \psi $ on $ U \subseteq X $, then $ \phi = \psi $ on $ X $.
\end{lemma}

\begin{proof}
We can assume that $ Y \subseteq \P^n $ for some $ n $, and hence we reduce to the case where $ Y = \P^n $. So the product is $ \phi \times \psi : X \to \P^n \times \P^n $. Let $ \Delta \subseteq \P^n \times \P^n = Z\rb{x_iy_j - x_jy_i} $. Since $ \phi = \psi $ on $ U $, $ \rb{\phi \times \psi}\rb{U} \subseteq A $, so $ \rb{\phi \times \psi}\rb{\overline{U}} = \rb{\phi \times \psi}\rb{X} \subseteq \Delta $.
\end{proof}

\begin{definition}
\hfill
\begin{itemize}
\item A \textbf{dominant rational map} is a rational map $ f : X \dashrightarrow Y $, such that $ f_U\rb{U} $ is dense for some, and hence all, $ \rb{U, f_U} $.
\item A \textbf{birational map} is a dominant rational map $ f : X \dashrightarrow Y $ such that $ f $ admits an inverse $ g : Y \dashrightarrow X $.
\end{itemize}
\end{definition}

\begin{theorem}
For any two varieties $ X, Y $ we have a correspondence
$$ \correspondence{\text{dominant rational maps} \ f : X \to Y}{K \text{-algebra homomorphisms} \ K\rb{Y} \to K\rb{X}}. $$
\end{theorem}

\begin{proof}
Given a rational map $ f : X \dashrightarrow Y $ and let $ g \in K\rb{Y} $. Let $ f_U $ be a representative of $ f $ then we have that if $ \rb{V, g} = g $, $ g \circ f_U \in K\rb{X} $. Since we can cover $ Y $ using affine varieties, we can assume $ Y $ is affine then $ K\rb{Y} = K\rb{A\rb{Y}} $. If we start with a homomorphism $ \theta : K\rb{Y} \to K\rb{X} $, let $ y_1, \dots, y_n \in A\rb{Y} $ be the generators of $ A\rb{Y} $, then $ \theta\rb{y_i} \in K\rb{X} $. We can find $ U $ such that $ \theta\rb{y_i} $ are regular at $ U $. Then this induces a map $ A\rb{Y} \to \OO\rb{U} $. But then we have a morphism $ U \to Y $, and moreover this is the inverse of the map we defined previously.
\end{proof}

\begin{definition}
\hfill
\begin{itemize}
\item A field extension $ L / K $ is \textbf{separably generated} if there is a transcendence basis $ \cb{x_i} $ for $ L / K $ such that $ L $ is a separable algebraic extension of $ K\rb{\cb{x_i}} $.
\item Primitive element theorem. If $ L / K $ is finite and separable then $ L / K\rb{\alpha} $ for some $ \alpha \in L $. If $ L $ is infinite and $ \beta_1, \dots, \beta_n $ are generators for $ L / K $ then $ \alpha = c_1\beta_1 + \dots + c_n\beta_n $ for $ c_i \in K $.
\item If $ K $ is perfect, any finitely generated extension $ L / K $ is separably generated.
\end{itemize}
\end{definition}

\begin{theorem}
Any variety $ X $ of dimension $ n $ is birational to a hypersurface $ Y \subseteq \P^{n + 1} $.
\end{theorem}

\begin{proof}
Since $ K\rb{X} = K $ is finitely generated, by the theorem above it is separably generated. So we can find a transcendence basis $ x_1, \dots, x_n \in K $ such that $ K / k\rb{x_1, \dots, x_n} $ is finite and separable. By the primitive element theorem, $ K = k\rb{x_1, \dots, x_n, y} $ for some $ y $ which is  algebraic over $ k\rb{x_1, \dots, x_n} $, so $ y $ is the solution of a polynomial equation $ f $ in $ k\rb{x_1, \dots, x_n} $. In particular if we clear denominators we get a polynomial $ f\rb{x_1, \dots, x_n, y} $ in $ \A^{n + 1} $, by taking $ Z\rb{f} $ we get a hypersurface and taking its projective closure we get a hypersurface in $ \P^n $.
\end{proof}

\lecture{12}{Tuesday}{05/02/19}

\begin{corollary}
The following are equivalent.
\begin{itemize}
\item $ F : X \dashrightarrow Y $ is birational.
\item There exist $ U, V $ such that $ F : U \to V $ is an isomorphism.
\item $ K\rb{Y} \cong K\rb{X} $.
\end{itemize}
\end{corollary}

\begin{definition}
The \textbf{blow-up} of $ \A^n $ at the origin $ 0 $, denoted by $ \widetilde{\A^n} $, is $ Z\rb{x_iy_j - x_jy_i} \subseteq \A^n \times \P^{n - 1} $.
$$
\begin{tikzcd}
\widetilde{\A^n} \arrow[hookrightarrow]{r} \arrow[swap]{dr}{\pi} & \A^n \times \P^{n - 1} \arrow{d}{\pi_1 : \rb{x, y} \mapsto x} \\
& \A^n
\end{tikzcd}.
$$
\end{definition}

\begin{proposition}
\hfill
\begin{enumerate}
\item Let $ P \in \A^n $, if $ P \ne 0 $ then $ \pi^{-1}\rb{P} $ is a single point, and $ \widetilde{\A^n} \setminus \pi^{-1}\rb{0} \cong \A^n \setminus \cb{0} $.
\item $ \pi^{-1}\rb{0} \cong \P^{n - 1} $.
\item Points of $ \pi^{-1}\rb{0} $ are in one-to-one correspondence with the set of lines through the origin.
\item $ \widetilde{\A^n} $ is irreducible.
\end{enumerate}
\end{proposition}

\begin{proof}
\hfill
\begin{enumerate}
\item If $ P \ne 0 $ then $ y_j = x_jy_i/x_i $ and this is true for every $ j $, so this gives a unique point in $ \P^{n - 1} $.
\item Obvious.
\item A line through the origin is given by $ x_i = ta_i $ for $ t \ne 0 $. Taking $ \pi^{-1} $ of this line we get $ x_i = ta_i $ and $ y_i = ta_i = a_i $. In other words if $ x \ne 0 $, $ \pi^{-1}\rb{X} = \rb{X, \sb{X}} $.
\item $ \widetilde{\A^n} \setminus \pi^{-1}\rb{0} \cong \A^n \setminus \cb{0} $ is dense and irreducible, by $ 3 $.
\end{enumerate}
\end{proof}

\begin{definition}
If $ Y \ni 0 $ is a closed subvariety of $ \A^n $ we define the \textbf{blow-up} of $ Y $ at $ 0 $ by $ \widetilde{Y} = \overline{\pi^{-1}\rb{Y \setminus \cb{0}}} $. More generally, we can blow-up any point by taking an affine change of coordinates. We also get a birational map $ \pi : \widetilde{Y} \to Y $.
\end{definition}

\begin{example}
Let $ Y = Z\rb{y^2 - x^2\rb{x + 1}} $. The equations of the blow-up are
$$
\begin{cases}
y^2 = x^2\rb{x + 1} \\
xu = yt
\end{cases},
$$
where $ \sb{t : u} \in \P^1 $. Suppose $ t \ne 0 $.
$$
\begin{cases}
y^2 = x^2\rb{x + 1} \\
y = xu
\end{cases}
\qquad \implies \qquad \rb{xu}^2 = x^2\rb{x + 1} \qquad \implies \qquad x^2\rb{u^2 - x - 1} = 0.
$$
\end{example}

\begin{example}
Let $ y^2 = x^3 $.
$$
\begin{cases}
y^2 = x^3 \\
y = xu
\end{cases}
\qquad \implies \qquad \rb{xu}^2 = x^3 \qquad \implies \qquad x^2\rb{u^2 - x} = 0.
$$
\end{example}

\pagebreak

\section{Nonsingular varieties}

\lecture{13}{Friday}{08/02/19}

\begin{definition}
Let $ Y \subseteq \A^n $ be an affine variety of dimension $ r $, and suppose $ I\rb{Y} = \ab{f_1, \dots, f_k} $. $ Y $ is \textbf{nonsingular} at $ P \in Y $ if $ rank\rb{\tfrac{\partial f_i\rb{P}}{\partial x_j}} = n - r $. $ Y $ is \textbf{nonsingular} if it is nonsingular at every $ P \in Y $.
\end{definition}

\begin{example}
Let $ x^2 = x^4 + y^4 \subseteq \A^2 $, so $ f = x^2 - x^4 - y^4 $.
$$ \dfrac{\partial f}{\partial x} = 2x - 4x^3 = 0 \qquad \implies \qquad x\rb{1 - 2x^2} = 0 \qquad \implies \qquad x = 0 \ \text{or} \ 2x^2 = 1, $$
$$ \dfrac{\partial f}{\partial y} = -9y^3 = 0 \qquad \implies \qquad y = 0 \qquad \implies \qquad x^2 = x^4 \qquad \implies \qquad x = 0 \ \text{or} \ x^2 = 1, $$
so $ Sing\rb{Y} = \cb{\rb{0, 0}} $.
\end{example}

\begin{example}
Let $ Y = Z\rb{f} = Z\rb{y^2 - x^3} $.
$$ \dfrac{\partial f}{\partial x} = -3x^2 = 0, \qquad \dfrac{\partial f}{\partial y} = 2y = 0, $$
so $ Sing\rb{Y} = \cb{\rb{0, 0}} $.
\end{example}

\begin{definition}
Let $ A $ be a Noetherian local ring with maximal ideal $ \mm $, and residue field $ A / \mm = K $. $ A $ is a \textbf{regular local ring} if $ \dim_K\rb{\mm / \mm^2} = \dim\rb{A} $.
\end{definition}

\begin{note*}
$ \rb{\mm / \mm^2}^* $ is called the \textbf{Zariski-tangent space}.
\end{note*}

Claim that $ \mm / \mm^2 $ is a $ K $-vector space for $ K = A / \mm $.

\begin{theorem}
Let $ Y \subseteq \A^n $ be an affine variety. Then $ Y $ is nonsingular at $ P $ if and only if $ \OO_P $ is a regular local ring.
\end{theorem}

\begin{proof}
Let $ P = \rb{a_1, \dots, a_n} \in Y $ with corresponding maximal ideal $ I_P = \ab{x_1 - a_1, \dots, x_n - a_n} $. We define a map
$$ \function[\theta_P]{A = K\sb{x_1, \dots, x_n}}{K^n}{f}{\rb{\dfrac{\partial f\rb{P}}{\partial x_1}, \dots, \dfrac{\partial f\rb{P}}{\partial x_n}}}. $$
Note that $ \theta\rb{\rb{x_i - a_i}\rb{x_j - a_j}} = 0 $, hence $ \theta_P\rb{I_P^2} = 0 $, in particular we have an isomorphism $ I_P / I_P^2 \cong K^n $. By the isomorphism, if $ \alpha = I\rb{Y} = \ab{f_1, \dots, f_t} $ then the rank of $ \tfrac{\partial f_i\rb{P}}{\partial x_j} $ corresponds to the dimension of $ \alpha $ under the isomorphism, which is $ \overline{\alpha} $ in $ I_P / I_P^2 $, $ \rb{\alpha + I_P} / I_P^2 $. Now $ \OO_P = \rb{A / \alpha}_{I_P} $. If $ \mm = \rb{I_P + \alpha} / \alpha $ then $ \mm^2 = \rb{I_P^2 + \alpha} / \alpha $, so $ \mm / \mm^2 = I_P / \rb{I_P^2 + \alpha} $. So
$$ r = \dim\rb{\dfrac{\mm}{\mm^2}} = \dim\rb{\dfrac{I_P}{I_P^2 + \alpha}} = \dim\rb{\dfrac{I_P}{I_P^2}} - \dim\rb{\dfrac{I_P^2 + \alpha}{I_P^2}} = n - rank\rb{\dfrac{\partial f_i}{\partial x_j}}. $$
So $ \OO_P $ is regular if and only if $ rank\rb{\tfrac{\partial f_i}{\partial x_j}} = n - r $.
\end{proof}

\begin{definition}
Let $ X $ be a variety. $ X $ is \textbf{nonsingular} at $ P $ if $ \OO_P $ is a regular local ring.
\end{definition}

\begin{theorem}
Let $ Y $ be a variety. Then $ Sing\rb{Y} $ is a proper and closed set. The set of nonsingular points of $ Y $ is open and dense.
\end{theorem}

\begin{proof}
Prove that $ Sing\rb{Y} $ is closed, first. We know that the rank of the Jacobian is at most $ n - r $, therefore the singular points occurs when the rank is less than $ n - r $, which is to say that $ Sing\rb{Y} $ is given by the vanishing of the $ \rb{n - r} \times \rb{n - r} $ minors of $ \tfrac{\partial f_i}{\partial x_j} $ and $ I\rb{Y} $, hence is closed. To prove that it is proper $ Sing\rb{Y} \subsetneq Y $.
\end{proof}

\lecture{14}{Monday}{11/02/19}

Lecture 14 is a problem class.

\lecture{15}{Tuesday}{12/02/19}

Lecture 15 is a problem class.

\pagebreak

\section{Intersections in projective space}

\lecture{16}{Friday}{15/02/19}

\begin{theorem}
Let $ Y, Z \subseteq \A^n $ be varieties, with $ \dim\rb{Y} = r $ and $ \dim\rb{Z} = s $ then every irreducible component has dimension at least $ r + s - n $.
\end{theorem}

\begin{proof}
Suppose $ Z $ is a hypersurface. Then if $ Y \subseteq Z $ the theorem holds, and if $ Y \nsubseteq Z $ the theorem is true by homework $ 1 $. Let $ Z $ be general. Consider the diagonal in $ \A^{2n} $ given by the image of the isomorphism $ P \mapsto P \times P $, then $ Y \cap Z $ corresponds to $ \rb{Y \times Z} \cap \Delta $. Recall that
$$ \Delta = Z\rb{x_1 - y_1} \cap \dots \cap Z\rb{x_n - y_n}, $$
by the first case $ n $ times we have that each irreducible component has dimension $ \rb{r + s} - n - 2n = r + s - n $.
\end{proof}

\begin{theorem}
Let $ Y, Z \subseteq \P^n $ be varieties, where $ \dim\rb{Y} = r $ and $ \dim\rb{Z} = s $, then each irreducible component of $ Y \cap Z $ has dimension at least $ r + s - n $. Moreover, if $ r + s - n \ge 0 $ then $ Y \cap Z \ne \emptyset $.
\end{theorem}

\begin{proof}
Take the affine cone of $ Y $ and $ Z $, $ C\rb{Y} $ and $ C\rb{Z} $, since $ 0 \in C\rb{Y} \cap C\rb{Z} $ we apply the previous theorem to get
$$ \rb{r + 1} + \rb{s + 1} - \rb{n + 1} = r + s - n + 1, $$
so therefore $ Y \cap Z \ne \emptyset $.
\end{proof}

\begin{definition}
A \textbf{numerical polynomial} is a polynomial $ f \in \Q\sb{x} $ such that $ f\rb{n} \in \Z $ for $ n \gg 0 $, for $ n $ sufficiently large.
\end{definition}

\begin{theorem}
\hfill
\begin{enumerate}
\item If $ f \in \Q\sb{x} $ is a numerical polynomial then there are $ c_0, \dots, c_r \in \Z $ such that
$$ f\rb{x} = c_0\twobyone{x}{r} + \dots + c_r\twobyone{x}{0}. $$
\item If for $ n \gg 0 $ $ \Delta f = f\rb{n + 1} - f\rb{n} = q $ and $ q $ is a numerical polynomial, then there exists $ p $ such that for $ n \gg 0 $ $ p\rb{n} = f\rb{n} $.
\end{enumerate}
\end{theorem}

\begin{proof}
\hfill
\begin{enumerate}
\item By linear algebra we can find $ c_0, \dots, c_r \in \Q $ such that
$$ f\rb{x} = c_0\twobyone{x}{r} + \dots + c_r\twobyone{x}{0}, $$
then
$$ \Delta f = c_0\twobyone{x}{r - 1} + \dots + c_{r - 1}\twobyone{x}{0}. $$
By induction on the degree of $ f $ we have that $ c_0, \dots, c_{r - 1} \in \Z $, but since $ f\rb{n} \in \Z $ for $ n \gg 0 $ then $ c_r \in \Z $.
\item If
$$ q = c_0\twobyone{x}{r} + \dots + c_r\twobyone{x}{0}, $$
set
$$ p = c_0\twobyone{x}{r + 1} + \dots + c_r\twobyone{x}{1}. $$
$ \Delta p = q $ gives $ \Delta \rb{f - p}\rb{n} = 0 $.
\end{enumerate}
\end{proof}

\begin{definition}
\hfill
\begin{itemize}
\item Let $ S $ be a graded ring. A graded $ S $-module is a module $ M $ with a decomposition
$$ M = \bigoplus_{d \in \Z} M_d, $$
such that $ S_k \cdot M_d \subseteq M_{d + k} $.
\item Let $ l \in \Z $. The twisted module $ M\rb{l} $ is the graded $ S $-module given by $ M\rb{l}_k = M_{l + k} $.
\item $ Ann\rb{M} = \cb{x \in S \mid xM = 0} $.
\end{itemize}
\end{definition}

\begin{theorem}
Let $ M $ be a finitely generated graded $ S $-module. Then there is a filtration
$$ 0 = M^0 \subseteq \dots \subseteq M^r = M, $$
such that $ M^i / M^{i - 1} \cong \rb{S / \pp_i}\rb{l} $ for some $ \pp_i $ prime ideals and $ l_i \in \Z $, such that
\begin{itemize}
\item prime $ \pp \supseteq Ann\rb{M} $ if and only if $ \pp \subseteq \pp_i $, that is $ \pp_i $ are minimal primes of $ M $, and
\item for each minimal prime $ \pp $ of $ M $ the number of times $ \pp $ appears in the set $ \cb{\pp_1, \dots, \pp_r} $ is $ len_{S_\pp}\rb{M_\pp} $.
\end{itemize}
\end{theorem}

\end{document}