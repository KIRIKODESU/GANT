\def\module{M4P33 Algebraic Geometry}
\def\lecturer{Dr Genival Da Silva Jr}
\def\term{Spring 2019}
\def\cover{
$$
\begin{tikzpicture}[scale=0.9]
\draw (0, 16) to (15, 16) node[right]{$ l $};
\draw (0, 1) to (15, 1) node[right]{$ m $};
\draw (1, 0) to (1, 17) node[above]{$ l_1 $};
\draw (0, 14) to node[below right]{$ l_1' $} (3, 17);
\draw (0, 3) to node[above right]{$ l_1'' $} (3, 0);
\draw (4, 0) to (4, 17) node[above]{$ l_2 $};
\draw (3, 14) to node[below right]{$ l_2' $} (6, 17);
\draw (3, 3) to node[above right]{$ l_2'' $} (6, 0);
\draw (7, 0) to (7, 17) node[above]{$ l_3 $};
\draw (6, 14) to node[below right]{$ l_3' $} (9, 17);
\draw (6, 3) to node[above right]{$ l_3'' $} (9, 0);
\draw (10, 0) to (10, 17) node[above]{$ l_4 $};
\draw (9, 14) to node[below right]{$ l_4' $} (12, 17);
\draw (9, 3) to node[above right]{$ l_4'' $} (12, 0);
\draw (13, 0) to (13, 17) node[above]{$ l_5 $};
\draw (12, 14) to node[below right]{$ l_5' $} (15, 17);
\draw (12, 3) to node[above right]{$ l_5'' $} (15, 0);
\draw (0, 13) to (8, 13) node[right]{$ m_0 $};
\draw (0, 12) to (6.5, 12);
\draw (7.5, 12) to (11, 12) node[right]{$ m_1 $};
\draw (0, 11) to (6.5, 11);
\draw (7.5, 11) to (9.5, 11);
\draw (10.5, 11) to (14, 11) node[right]{$ m_2 $};
\draw (0, 10) to (3.5, 10);
\draw (4.5, 10) to (11, 10) node[right]{$ m_3 $};
\draw (0, 9) to (3.5, 9);
\draw (4.5, 9) to (9.5, 9);
\draw (10.5, 9) to (14, 9) node[right]{$ m_4 $};
\draw (0, 8) to (3.5, 8);
\draw (4.5, 8) to (6.5, 8);
\draw (7.5, 8) to (14, 8) node[right]{$ m_5 $};
\draw (3, 7) to (11, 7) node[right]{$ m_6 $};
\draw (3, 6) to (9.5, 6);
\draw (10.5, 6) to (14, 6) node[right]{$ m_7 $};
\draw (3, 5) to (6.5, 5);
\draw (7.5, 5) to (14, 5) node[right]{$ m_8 $};
\draw (6, 4) to (14, 4) node[right]{$ m_9 $};
\end{tikzpicture}
$$
}
\def\syllabus{Affine varieties. Projective varieties. Morphisms. Rational maps. Nonsingular varieties. Intersections in projective space. The $ 27 $ lines on a cubic surface. Grassmannians. Divisors on curves. Elliptic curves.}
\def\thm{section}

\documentclass{article}

% Packages

\usepackage{amssymb}
\usepackage{amsthm}
\usepackage[UKenglish]{babel}
\usepackage{commath}
\usepackage{enumitem}
\usepackage{etoolbox}
\usepackage{fancyhdr}
\usepackage[margin=1in]{geometry}
\usepackage{graphicx}
\usepackage[hidelinks]{hyperref}
\usepackage[utf8]{inputenc}
\usepackage{listings}
\usepackage{mathtools}
\usepackage{stmaryrd}
\usepackage{tikz-cd}
\usepackage{csquotes}

% Formatting

\addto\captionsUKenglish{\renewcommand{\abstractname}{Syllabus}}
\delimitershortfall5pt
\ifx\thm\undefined\newtheorem{n}{}\else\newtheorem{n}{}[\thm]\fi
\newcommand\newoperator[1]{\ifcsdef{#1}{\cslet{#1}{\relax}}{}\csdef{#1}{\operatorname{#1}}}
\setlength{\parindent}{0cm}

% Environments

\theoremstyle{plain}
\newtheorem{algorithm}[n]{Algorithm}
\newtheorem*{algorithm*}{Algorithm}
\newtheorem{algorithm**}{Algorithm}
\newtheorem{conjecture}[n]{Conjecture}
\newtheorem*{conjecture*}{Conjecture}
\newtheorem{conjecture**}{Conjecture}
\newtheorem{corollary}[n]{Corollary}
\newtheorem*{corollary*}{Corollary}
\newtheorem{corollary**}{Corollary}
\newtheorem{lemma}[n]{Lemma}
\newtheorem*{lemma*}{Lemma}
\newtheorem{lemma**}{Lemma}
\newtheorem{proposition}[n]{Proposition}
\newtheorem*{proposition*}{Proposition}
\newtheorem{proposition**}{Proposition}
\newtheorem{theorem}[n]{Theorem}
\newtheorem*{theorem*}{Theorem}
\newtheorem{theorem**}{Theorem}

\theoremstyle{definition}
\newtheorem{aim}[n]{Aim}
\newtheorem*{aim*}{Aim}
\newtheorem{aim**}{Aim}
\newtheorem{axiom}[n]{Axiom}
\newtheorem*{axiom*}{Axiom}
\newtheorem{axiom**}{Axiom}
\newtheorem{condition}[n]{Condition}
\newtheorem*{condition*}{Condition}
\newtheorem{condition**}{Condition}
\newtheorem{definition}[n]{Definition}
\newtheorem*{definition*}{Definition}
\newtheorem{definition**}{Definition}
\newtheorem{example}[n]{Example}
\newtheorem*{example*}{Example}
\newtheorem{example**}{Example}
\newtheorem{exercise}[n]{Exercise}
\newtheorem*{exercise*}{Exercise}
\newtheorem{exercise**}{Exercise}
\newtheorem{fact}[n]{Fact}
\newtheorem*{fact*}{Fact}
\newtheorem{fact**}{Fact}
\newtheorem{goal}[n]{Goal}
\newtheorem*{goal*}{Goal}
\newtheorem{goal**}{Goal}
\newtheorem{law}[n]{Law}
\newtheorem*{law*}{Law}
\newtheorem{law**}{Law}
\newtheorem{plan}[n]{Plan}
\newtheorem*{plan*}{Plan}
\newtheorem{plan**}{Plan}
\newtheorem{problem}[n]{Problem}
\newtheorem*{problem*}{Problem}
\newtheorem{problem**}{Problem}
\newtheorem{question}[n]{Question}
\newtheorem*{question*}{Question}
\newtheorem{question**}{Question}
\newtheorem{warning}[n]{Warning}
\newtheorem*{warning*}{Warning}
\newtheorem{warning**}{Warning}
\newtheorem{acknowledgements}[n]{Acknowledgements}
\newtheorem*{acknowledgements*}{Acknowledgements}
\newtheorem{acknowledgements**}{Acknowledgements}
\newtheorem{annotations}[n]{Annotations}
\newtheorem*{annotations*}{Annotations}
\newtheorem{annotations**}{Annotations}
\newtheorem{assumption}[n]{Assumption}
\newtheorem*{assumption*}{Assumption}
\newtheorem{assumption**}{Assumption}
\newtheorem{conclusion}[n]{Conclusion}
\newtheorem*{conclusion*}{Conclusion}
\newtheorem{conclusion**}{Conclusion}
\newtheorem{claim}[n]{Claim}
\newtheorem*{claim*}{Claim}
\newtheorem{claim**}{Claim}
\newtheorem{notation}[n]{Notation}
\newtheorem*{notation*}{Notation}
\newtheorem{notation**}{Notation}
\newtheorem{note}[n]{Note}
\newtheorem*{note*}{Note}
\newtheorem{note**}{Note}
\newtheorem{remark}[n]{Remark}
\newtheorem*{remark*}{Remark}
\newtheorem{remark**}{Remark}

% Lectures

\newcommand{\lecture}[3]{ % Lecture
  \marginpar{
    Lecture #1 \\
    #2 \\
    #3
  }
}

% Blackboard

\renewcommand{\AA}{\mathbb{A}} % Blackboard A
\newcommand{\BB}{\mathbb{B}}   % Blackboard B
\newcommand{\CC}{\mathbb{C}}   % Blackboard C
\newcommand{\DD}{\mathbb{D}}   % Blackboard D
\newcommand{\EE}{\mathbb{E}}   % Blackboard E
\newcommand{\FF}{\mathbb{F}}   % Blackboard F
\newcommand{\GG}{\mathbb{G}}   % Blackboard G
\newcommand{\HH}{\mathbb{H}}   % Blackboard H
\newcommand{\II}{\mathbb{I}}   % Blackboard I
\newcommand{\JJ}{\mathbb{J}}   % Blackboard J
\newcommand{\KK}{\mathbb{K}}   % Blackboard K
\newcommand{\LL}{\mathbb{L}}   % Blackboard L
\newcommand{\MM}{\mathbb{M}}   % Blackboard M
\newcommand{\NN}{\mathbb{N}}   % Blackboard N
\newcommand{\OO}{\mathbb{O}}   % Blackboard O
\newcommand{\PP}{\mathbb{P}}   % Blackboard P
\newcommand{\QQ}{\mathbb{Q}}   % Blackboard Q
\newcommand{\RR}{\mathbb{R}}   % Blackboard R
\renewcommand{\SS}{\mathbb{S}} % Blackboard S
\newcommand{\TT}{\mathbb{T}}   % Blackboard T
\newcommand{\UU}{\mathbb{U}}   % Blackboard U
\newcommand{\VV}{\mathbb{V}}   % Blackboard V
\newcommand{\WW}{\mathbb{W}}   % Blackboard W
\newcommand{\XX}{\mathbb{X}}   % Blackboard X
\newcommand{\YY}{\mathbb{Y}}   % Blackboard Y
\newcommand{\ZZ}{\mathbb{Z}}   % Blackboard Z

% Brackets

\renewcommand{\eval}[1]{\left. #1 \right|}          % Evaluation
\newcommand{\br}{\del}                              % Brackets
\newcommand{\abr}[1]{\left\langle #1 \right\rangle} % Angle brackets
\newcommand{\fbr}[1]{\left\lfloor #1 \right\rfloor} % Floor brackets
\newcommand{\lbr}[1]{\left\lfloor #1 \right\rfloor} % Ceiling brackets
\newcommand{\st}{\ \middle| \ }                     % Such that

% Calligraphic

\newcommand{\AAA}{\mathcal{A}} % Calligraphic A
\newcommand{\BBB}{\mathcal{B}} % Calligraphic B
\newcommand{\CCC}{\mathcal{C}} % Calligraphic C
\newcommand{\DDD}{\mathcal{D}} % Calligraphic D
\newcommand{\EEE}{\mathcal{E}} % Calligraphic E
\newcommand{\FFF}{\mathcal{F}} % Calligraphic F
\newcommand{\GGG}{\mathcal{G}} % Calligraphic G
\newcommand{\HHH}{\mathcal{H}} % Calligraphic H
\newcommand{\III}{\mathcal{I}} % Calligraphic I
\newcommand{\JJJ}{\mathcal{J}} % Calligraphic J
\newcommand{\KKK}{\mathcal{K}} % Calligraphic K
\newcommand{\LLL}{\mathcal{L}} % Calligraphic L
\newcommand{\MMM}{\mathcal{M}} % Calligraphic M
\newcommand{\NNN}{\mathcal{N}} % Calligraphic N
\newcommand{\OOO}{\mathcal{O}} % Calligraphic O
\newcommand{\PPP}{\mathcal{P}} % Calligraphic P
\newcommand{\QQQ}{\mathcal{Q}} % Calligraphic Q
\newcommand{\RRR}{\mathcal{R}} % Calligraphic R
\newcommand{\SSS}{\mathcal{S}} % Calligraphic S
\newcommand{\TTT}{\mathcal{T}} % Calligraphic T
\newcommand{\UUU}{\mathcal{U}} % Calligraphic U
\newcommand{\VVV}{\mathcal{V}} % Calligraphic V
\newcommand{\WWW}{\mathcal{W}} % Calligraphic W
\newcommand{\XXX}{\mathcal{X}} % Calligraphic X
\newcommand{\YYY}{\mathcal{Y}} % Calligraphic Y
\newcommand{\ZZZ}{\mathcal{Z}} % Calligraphic Z

% Fraktur

\newcommand{\aaa}{\mathfrak{a}}   % Fraktur a
\newcommand{\bbb}{\mathfrak{b}}   % Fraktur b
\newcommand{\ccc}{\mathfrak{c}}   % Fraktur c
\newcommand{\ddd}{\mathfrak{d}}   % Fraktur d
\newcommand{\eee}{\mathfrak{e}}   % Fraktur e
\newcommand{\fff}{\mathfrak{f}}   % Fraktur f
\renewcommand{\ggg}{\mathfrak{g}} % Fraktur g
\newcommand{\hhh}{\mathfrak{h}}   % Fraktur h
\newcommand{\iii}{\mathfrak{i}}   % Fraktur i
\newcommand{\jjj}{\mathfrak{j}}   % Fraktur j
\newcommand{\kkk}{\mathfrak{k}}   % Fraktur k
\renewcommand{\lll}{\mathfrak{l}} % Fraktur l
\newcommand{\mmm}{\mathfrak{m}}   % Fraktur m
\newcommand{\nnn}{\mathfrak{n}}   % Fraktur n
\newcommand{\ooo}{\mathfrak{o}}   % Fraktur o
\newcommand{\ppp}{\mathfrak{p}}   % Fraktur p
\newcommand{\qqq}{\mathfrak{q}}   % Fraktur q
\newcommand{\rrr}{\mathfrak{r}}   % Fraktur r
\newcommand{\sss}{\mathfrak{s}}   % Fraktur s
\newcommand{\ttt}{\mathfrak{t}}   % Fraktur t
\newcommand{\uuu}{\mathfrak{u}}   % Fraktur u
\newcommand{\vvv}{\mathfrak{v}}   % Fraktur v
\newcommand{\www}{\mathfrak{w}}   % Fraktur w
\newcommand{\xxx}{\mathfrak{x}}   % Fraktur x
\newcommand{\yyy}{\mathfrak{y}}   % Fraktur y
\newcommand{\zzz}{\mathfrak{z}}   % Fraktur z

% Geometry

\newcommand{\CP}{\mathbb{CP}}                                              % Complex projective space
\newcommand{\iintd}[4]{\iint_{#1} \, #2 \, \dif #3 \, \dif #4}             % Double integral
\newcommand{\RP}{\mathbb{RP}}                                              % Real projective space
\newcommand{\intd}[4]{\int_{#1}^{#2} \, #3 \, \dif #4}                     % Single integral
\newcommand{\iiintd}[5]{\iint_{#1} \, #2 \, \dif #3 \, \dif #4 \, \dif #5} % Triple integral

% Logic

\newcommand{\iffb}[2]{\br{#1 \leftrightarrow #2}} % Biconditional
\newcommand{\andb}[2]{\br{#1 \land #2}}           % Conjunction
\newcommand{\orb}[2]{\br{#1 \lor #2}}             % Disjunction
\newcommand{\nib}[2]{\br{#1 \notin #2}}           % Element of
\newcommand{\eqb}[2]{\br{#1 = #2}}                % Equal to
\newcommand{\teb}[1]{\br{\exists #1}}             % Existential quantifier
\newcommand{\impb}[2]{\br{#1 \rightarrow #2}}     % Implication
\newcommand{\ltb}[2]{\br{#1 < #2}}                % Less than
\newcommand{\leb}[2]{\br{#1 \le #2}}              % Less than or equal to
\newcommand{\notb}[1]{\br{\neg #1}}               % Negation
\newcommand{\inb}[2]{\br{#1 \in #2}}              % Not element of
\newcommand{\neb}[2]{\br{#1 \ne #2}}              % Not equal to
\newcommand{\subb}[2]{\br{#1 \subseteq #2}}       % Subset
\newcommand{\fab}[1]{\br{\forall #1}}             % Universal quantifier

% Maps

\newcommand{\bijection}[7][]{    % Bijection
  \ifx &#1&
    \begin{array}{rcl}
      #2 & \longleftrightarrow & #3 \\
      #4 & \longmapsto         & #5 \\
      #6 & \longmapsfrom       & #7
    \end{array}
  \else
    \begin{array}{ccrcl}
      #1 & : & #2 & \longrightarrow & #3 \\
         &   & #4 & \longmapsto     & #5 \\
         &   & #6 & \longmapsfrom   & #7
    \end{array}
  \fi
}
\newcommand{\birational}[7][]{   % Birational map
  \ifx &#1&
    \begin{array}{rcl}
      #2 & \dashrightarrow & #3 \\
      #4 & \longmapsto     & #5 \\
      #6 & \longmapsfrom   & #7
    \end{array}
  \else
    \begin{array}{ccrcl}
      #1 & : & #2 & \dashrightarrow & #3 \\
         &   & #4 & \longmapsto     & #5 \\
         &   & #6 & \longmapsfrom   & #7
    \end{array}
  \fi
}
\newcommand{\correspondence}[2]{ % Correspondence
  \cbr{
    \begin{array}{c}
      #1
    \end{array}
  }
  \qquad
  \leftrightsquigarrow
  \qquad
  \cbr{
    \begin{array}{c}
      #2
    \end{array}
  }
}
\newcommand{\function}[5][]{     % Function
  \ifx &#1&
    \begin{array}{rcl}
      #2 & \longrightarrow & #3 \\
      #4 & \longmapsto     & #5
    \end{array}
  \else
    \begin{array}{ccrcl}
      #1 & : & #2 & \longrightarrow & #3 \\
         &   & #4 & \longmapsto     & #5
    \end{array}
  \fi
}
\newcommand{\functions}[7][]{    % Functions
  \ifx &#1&
    \begin{array}{rcl}
      #2 & \longrightarrow & #3 \\
      #4 & \longmapsto     & #5 \\
      #6 & \longmapsto     & #7
    \end{array}
  \else
    \begin{array}{ccrcl}
      #1 & : & #2 & \longrightarrow & #3 \\
         &   & #4 & \longmapsto     & #5 \\
         &   & #6 & \longmapsto     & #7
    \end{array}
  \fi
}
\newcommand{\rational}[5][]{     % Rational map
  \ifx &#1&
    \begin{array}{rcl}
      #2 & \dashrightarrow & #3 \\
      #4 & \longmapsto     & #5
    \end{array}
  \else
    \begin{array}{ccrcl}
      #1 & : & #2 & \dashrightarrow & #3 \\
         &   & #4 & \longmapsto     & #5
    \end{array}
  \fi
}

% Matrices

\newcommand{\onebytwo}[2]{      % One by two matrix
  \begin{pmatrix}
    #1 & #2
  \end{pmatrix}
}
\newcommand{\onebythree}[3]{    % One by three matrix
  \begin{pmatrix}
    #1 & #2 & #3
  \end{pmatrix}
}
\newcommand{\twobyone}[2]{      % Two by one matrix
  \begin{pmatrix}
    #1 \\
    #2
  \end{pmatrix}
}
\newcommand{\twobytwo}[4]{      % Two by two matrix
  \begin{pmatrix}
    #1 & #2 \\
    #3 & #4
  \end{pmatrix}
}
\newcommand{\threebyone}[3]{    % Three by one matrix
  \begin{pmatrix}
    #1 \\
    #2 \\
    #3
  \end{pmatrix}
}
\newcommand{\threebythree}[9]{  % Three by three matrix
  \begin{pmatrix}
    #1 & #2 & #3 \\
    #4 & #5 & #6 \\
    #7 & #8 & #9
  \end{pmatrix}
}
\newcommand{\twobytwosmall}[4]{ % Two by two small matrix
  \begin{psmallmatrix}
    #1 & #2 \\
    #3 & #4
  \end{psmallmatrix}
}

% Number theory

\renewcommand{\symbol}[2]{\br{\tfrac{#1}{#2}}} % Power residue symbol
\newcommand{\unit}[1]{\br{\ZZ / #1\ZZ}^\times} % Unit group

% Operators

\newoperator{ab}    % Abelian
\newoperator{AG}    % Affine geometry
\newoperator{alg}   % Algebraic
\newoperator{Ann}   % Annihilator
\newoperator{area}  % Area
\newoperator{Aut}   % Automorphism
\newoperator{card}  % Cardinality
\newoperator{ch}    % Characteristic
\newoperator{Cl}    % Class
\newoperator{Coker} % Cokernel
\newoperator{col}   % Column
\newoperator{Corr}  % Correspondence
\newoperator{diam}  % Diameter
\newoperator{Disc}  % Discriminant
\newoperator{dom}   % Domain
\newoperator{Eig}   % Eigenvalue
\newoperator{Em}    % Embedding
\newoperator{End}   % Endomorphism
\newoperator{fin}   % Finite
\newoperator{Fix}   % Fixed
\newoperator{Frac}  % Fraction
\newoperator{Frob}  % Frobenius
\newoperator{Fun}   % Function
\newoperator{Gal}   % Galois
\newoperator{GL}    % General linear
\newoperator{Ham}   % Hamming
\newoperator{Homeo} % Homeomorphism
\newoperator{Hom}   % Homomorphism
\newoperator{id}    % Identity
\newoperator{Im}    % Image
\newoperator{Ind}   % Index
\newoperator{Ker}   % Kernel
\newoperator{lcm}   % Least common multiple
\newoperator{Mat}   % Matrix
\newoperator{mult}  % Multiplicity
\newoperator{new}   % New
\newoperator{Nm}    % Norm
\newoperator{old}   % Old
\newoperator{op}    % Opposite
\newoperator{ord}   % Order
\newoperator{Pay}   % Payley
\newoperator{PG}    % Projective geometry
\newoperator{PGL}   % Projective general linear
\newoperator{PSL}   % Projective special linear
\newoperator{rad}   % Radical
\newoperator{ran}   % Range
\newoperator{Res}   % Residue
\newoperator{rk}    % Rank
\newoperator{Re}    % Real
\newoperator{row}   % Row
\newoperator{sgn}   % Sign
\newoperator{Sing}  % Singular
\newoperator{SK}    % Skeleton
\newoperator{sp}    % Span
\newoperator{SL}    % Special linear
\newoperator{SO}    % Special orthogonal
\newoperator{Spec}  % Spectrum
\newoperator{Stab}  % Stabiliser
\newoperator{star}  % Star
\newoperator{srg}   % Strongly regular graph
\newoperator{supp}  % Support
\newoperator{Sym}   % Symmetric
\newoperator{tors}  % Torsion
\newoperator{Tr}    % Trace
\newoperator{vol}   % Volume
\newoperator{wt}    % Weight

% Roman

\newcommand{\A}{\mathrm{A}}   % Roman A
\newcommand{\B}{\mathrm{B}}   % Roman B
\newcommand{\C}{\mathrm{C}}   % Roman C
\newcommand{\D}{\mathrm{D}}   % Roman D
\newcommand{\E}{\mathrm{E}}   % Roman E
\newcommand{\F}{\mathrm{F}}   % Roman F
\newcommand{\G}{\mathrm{G}}   % Roman G
\renewcommand{\H}{\mathrm{H}} % Roman H
\newcommand{\I}{\mathrm{I}}   % Roman I
\newcommand{\J}{\mathrm{J}}   % Roman J
\newcommand{\K}{\mathrm{K}}   % Roman K
\renewcommand{\L}{\mathrm{L}} % Roman L
\newcommand{\M}{\mathrm{M}}   % Roman M
\newcommand{\N}{\mathrm{N}}   % Roman N
\renewcommand{\O}{\mathrm{O}} % Roman O
\renewcommand{\P}{\mathrm{P}} % Roman P
\newcommand{\Q}{\mathrm{Q}}   % Roman Q
\newcommand{\R}{\mathrm{R}}   % Roman R
\renewcommand{\S}{\mathrm{S}} % Roman S
\newcommand{\T}{\mathrm{T}}   % Roman T
\newcommand{\U}{\mathrm{U}}   % Roman U
\newcommand{\V}{\mathrm{V}}   % Roman V
\newcommand{\W}{\mathrm{W}}   % Roman W
\newcommand{\X}{\mathrm{X}}   % Roman X
\newcommand{\Y}{\mathrm{Y}}   % Roman Y
\newcommand{\Z}{\mathrm{Z}}   % Roman Z

\renewcommand{\a}{\mathrm{a}} % Roman a
\renewcommand{\b}{\mathrm{b}} % Roman b
\renewcommand{\c}{\mathrm{c}} % Roman c
\renewcommand{\d}{\mathrm{d}} % Roman d
\newcommand{\e}{\mathrm{e}}   % Roman e
\newcommand{\f}{\mathrm{f}}   % Roman f
\newcommand{\g}{\mathrm{g}}   % Roman g
\newcommand{\h}{\mathrm{h}}   % Roman h
\renewcommand{\i}{\mathrm{i}} % Roman i
\renewcommand{\j}{\mathrm{j}} % Roman j
\renewcommand{\k}{\mathrm{k}} % Roman k
\renewcommand{\l}{\mathrm{l}} % Roman l
\newcommand{\m}{\mathrm{m}}   % Roman m
\renewcommand{\n}{\mathrm{n}} % Roman n
\renewcommand{\o}{\mathrm{o}} % Roman o
\newcommand{\p}{\mathrm{p}}   % Roman p
\newcommand{\q}{\mathrm{q}}   % Roman q
\renewcommand{\r}{\mathrm{r}} % Roman r
\newcommand{\s}{\mathrm{s}}   % Roman s
\renewcommand{\t}{\mathrm{t}} % Roman t
\renewcommand{\u}{\mathrm{u}} % Roman u
\renewcommand{\v}{\mathrm{v}} % Roman v
\newcommand{\w}{\mathrm{w}}   % Roman w
\newcommand{\x}{\mathrm{x}}   % Roman x
\newcommand{\y}{\mathrm{y}}   % Roman y
\newcommand{\z}{\mathrm{z}}   % Roman z

% Tikz

\tikzset{
  arrow symbol/.style={"#1" description, allow upside down, auto=false, draw=none, sloped},
  subset/.style={arrow symbol={\subset}},
  cong/.style={arrow symbol={\cong}}
}

% Fancy header

\pagestyle{fancy}
\lhead{\module}
\rhead{\nouppercase{\leftmark}}

% Make title

\title{\module}
\author{Lectured by \lecturer \\ Typed by David Kurniadi Angdinata}
\date{\term}

\begin{document}

% Title page
\maketitle
\cover
\vfill
\begin{abstract}
\noindent\syllabus
\end{abstract}

\pagebreak

% Contents page
\tableofcontents

\pagebreak

% Document page
\setcounter{section}{-1}

\section{Introduction}

\lecture{1}{Friday}{11/01/19}

I will not follow a particular book, but everything I am going to say will be contained in one of the following books.
\begin{itemize}
\item I Shafarevich, Basic algebraic geometry, 1974
\item R Hartshorne, Algebraic geometry, 1977
\item J Harris, Algebraic geometry: a first course, 1922
\item M Reid, Undergraduate algebraic geometry, 1988
\end{itemize}

\pagebreak

\section{Affine varieties}

\begin{notation}
\hfill
\begin{itemize}
\item $ R $ is a commutative ring with unity.
\item $ K $ is a field.
\item $ K\sbr{x_1, \dots, x_n} $ is the ring of polynomials in $ n $ variables.
\item $ \A^n $ is $ K^n $ as a set.
\end{itemize}
\end{notation}

\begin{definition}
Let $ S \subseteq K\sbr{x_1, \dots, x_n} $ then
$$ Z\br{S} = \cbr{x \in \A^n \mid \forall f \in S, \ f\br{x} = 0} $$
is called the \textbf{zero locus} of $ S $. Subsets of $ \A^n $ that are of this form are called \textbf{affine varieties}.
\end{definition}

\begin{remark}
Some authors call \textbf{algebraic set} the object $ Z\br{S} $. We will not follow this notation.
\end{remark}

\begin{example}
\hfill
\begin{itemize}
\item Single points $ p = \br{p_1, \dots, p_n} $. $ p = Z\br{S} $ where $ S = \cbr{x_1 - p_1, \dots, x_n - p_n} $.
\item $ \A^n = Z\br{0} $.
\item $ \emptyset = Z\br{1} $.
\item Subspaces of $ \A^n = K^n $.
\item If $ X = Z\br{f_1, \dots, f_n} \subseteq \A^n $ and $ Y = Z\br{g_1, \dots, g_m} \subseteq \A^n $ are affine varieties then
$$ X \times Y = Z\br{f_1, \dots, f_n, g_1, \dots, g_m} \subseteq \A^{n + m} $$
is a variety.
\end{itemize}
\end{example}

\begin{remark}
If $ S \subseteq K\sbr{x_1, \dots, x_n} $ and $ I = \abr{S} $ then $ Z\br{S} = Z\br{I} $.
\end{remark}

\begin{theorem}[Hilbert's basis theorem]
If $ R $ is Noetherian then $ R\sbr{x} $ is Noetherian.
\end{theorem}

\begin{corollary}
Every ideal in $ K\sbr{x_1, \dots, x_n} $ is finitely generated.
\end{corollary}

\begin{definition}
Let $ X \subseteq \A^n $ then
$$ I\br{X} = \cbr{f \in K\sbr{x_1, \dots, x_n} \mid \forall x \in X, \ f\br{x} = 0}. $$
\end{definition}

\begin{example}
$ I\br{p} = I\br{\br{p_1, \dots, p_n}} = \abr{x_1 - p_1, \dots, x_n - p_n} $.
\end{example}

The goal is
$$
\begin{array}{rcl}
\cbr{\text{affine varieties in} \ \A^n} & \longleftrightarrow & \cbr{\text{ideals of} \ K\sbr{x_1, \dots, x_n}} \\
X & \longmapsto & I\br{X} \\
Z\br{J} & \longmapsfrom & J
\end{array}.
$$
$ Z\br{I\br{X}} = X $ but $ I\br{Z\br{J}} \supseteq J $.

\begin{example}
$ J = \abr{x^2 + 1} \subseteq \R\sbr{x} = I\br{\emptyset} = I\br{Z\br{x^2 + 1}} $.
\end{example}

\begin{proposition}
\hfill
\begin{itemize}
\item If $ X \subseteq Y $ then $ I\br{Y} \subseteq I\br{X} $. If $ I \subseteq J $ then $ Z\br{J} \subseteq Z\br{I} $.
\item $ X \subseteq Z\br{I\br{X}} $ and $ S \subseteq I\br{Z\br{S}} $.
\item If $ X $ is affine then $ Z\br{J\br{X}} = X $. If $ X = Z\br{S} $ then take $ Z $ of $ S \subseteq I\br{Z\br{S}} $.
\end{itemize}
\end{proposition}

\pagebreak

\begin{example}
Let $ J \subseteq \C\sbr{x} $. $ J = \abr{f} $, where $ f = \br{x - x_1}^{k_1} \dots \br{x - x_n}^{k_n} $.
\end{example}

\begin{definition}
Let $ I \subseteq K\sbr{x_1, \dots, x_n} $ be an ideal.
$$ I \subseteq \sqrt{I} = \cbr{f \in K\sbr{x_1, \dots, x_n} \mid \exists n \in \N, \ f^n \in I}. $$
If $ \sqrt{I} = I $, we say $ I $ is a \textbf{radical ideal}. (Exercise: $ \sqrt{I} $ is an ideal, $ I \subseteq \sqrt{I} $, and $ \sqrt{I} = \bigcap_{p \ \text{prime}} p $)
\end{definition}

\begin{theorem}[Hilbert's Nullstellensatz]
$ I\br{Z\br{J}} = \sqrt{J} $. If $ \sqrt{J} = J $ then
$$
\begin{array}{rcl}
\cbr{\text{affine varieties}} & \longleftrightarrow & \cbr{\text{radical ideals}} \\
X & \longmapsto & I\br{X} \\
Z\br{J} & \longmapsfrom & J
\end{array}.
$$
\end{theorem}

\lecture{2}{Monday}{14/01/19}

\begin{proposition}
\hfill
\begin{enumerate}
\item $ Z\br{S} \cup Z\br{T} = Z\br{ST} $.
\item $ \bigcap_i Z\br{S_i} = Z\br{\bigcup_i S_i} $.
\item $ Z\br{0} = \A^n $ and $ Z\br{1} = \emptyset $.
\end{enumerate}
\end{proposition}

\begin{proof}
\hfill
\begin{itemize}
\item[1.] If $ p \in Z\br{S} \cup Z\br{T} $, then $ f\br{p} = 0 $ for $ f \in S $ or $ f \in T $, so $ f\br{x} = 0 $ for $ f \in ST $, where
$$ ST = \cbr{\sum_{i \in I, \ I \ \text{finite}} s_it_i} \subseteq S \cap T, $$
with equality if $ S + T = R $. If $ p \in Z\br{ST} $, there exists $ f $ such that $ f\br{p} = 0 $ for $ f \in S $ or $ f\br{p} = 0 $ for $ f \in T $, so $ p \in Z\br{S} \cup Z\br{T} $.
\end{itemize}
\end{proof}

\begin{definition}
The \textbf{Zariski topology} on $ \A^n $ is the topology generated by closed sets of the form $ Z\br{S} $. By the above proposition this is a topology.
\end{definition}

\begin{example}
$ \A^1 $ is not Hausdorff.
\end{example}

\begin{definition}
A topological space $ X $ is \textbf{irreducible} if it cannot be expressed as a union $ X = A \cup B $, where $ A $ and $ B $ are proper and closed subsets. $ \emptyset $ is not considered irreducible.
\end{definition}

\begin{example}
$ \A^1 $.
\end{example}

\begin{example}
Any non-empty open set of irreducible $ X $ is dense and irreducible. Suppose $ A $ is open then $ X = A^c \cup \overline{A} $. Since $ X $ is irreducible then $ A^c = X $, a contradiction, or $ \overline{A} = X $. Suppose $ A $ is reducible. Let $ A = \br{A \cap B} \cup \br{A \cap C} $, where $ B $ and $ C $ are closed. Then $ X = A^c \cup \br{B \cup C} $. $ A^c = X $ or $ B \cup C = X $, which are contradictions.
\end{example}

\begin{example}
If $ A $ is irreducible then $ \overline{A} $ is also irreducible. Suppose $ \overline{A} $ is not irreducible. $ \overline{A} = \br{\overline{A} \cap B} \cup \br{\overline{A} \cap C} $. Take $ \bigcap A $, $ A = \br{A \cap B} \cup \br{A \cap C} $, a contradiction.
\end{example}

\begin{definition}
An affine variety is \textbf{irreducible} if it is irreducible as a topological space.
\end{definition}

\begin{remark}
A \textbf{quasi-affine variety} is an open set of an affine variety.
\end{remark}

\pagebreak

\begin{proposition}
\hfill
\begin{enumerate}
\item $ I\br{X \cup Y} = I\br{X} \cap I\br{Y} $.
\item $ Z\br{I\br{X}} = \overline{X} $ for any $ X \subseteq \A^n $.
\end{enumerate}
\end{proposition}

\begin{proof}
\hfill
\begin{enumerate}
\item If $ f \in I\br{X \cup Y} $ then $ f\br{p} = 0 $ for all $ p \in X \cup Y $, so $ f \in I\br{X} $ and $ f \in I\br{Y} $.
\item We know that $ X \subseteq Z\br{I\br{X}} $ hence $ \overline{X} \subseteq Z\br{I\br{X}} $. Now, let $ Y $ be a closed set containing $ X $, that is $ X \subseteq Y $. Then
$$ I\br{Y} \subset I\br{X} \qquad \implies \qquad Z\br{I\br{X}} \subset Z\br{I\br{X}} = Y, $$
so any closed set containing $ Y $ contains $ Z\br{I\br{X}} $.
\end{enumerate}
\end{proof}

\begin{proposition}
$ X $ is irreducible if and only if $ I\br{X} $ is prime.
\end{proposition}

\begin{proof}
\hfill
\begin{itemize}
\item[$ \implies $] Let $ f, g \in I\br{X} $.
$$ X \subseteq Z\br{fg} = Z\br{f} \cup Z\br{g} \qquad \implies \qquad X = \br{X \cap Z\br{f}} \cup \br{X \cap Z\br{g}}. $$
$ Z\br{f} \subseteq X $, so $ f \in I\br{X} $, or $ Z\br{g} \subseteq X $, so $ g \in I\br{X} $.
\item[$ \impliedby $] Exercise.
\end{itemize}
\end{proof}

\begin{example}
$ \A^n $.
\end{example}

\begin{definition}
If $ X \subseteq \A^n $, the \textbf{coordinate ring} of $ X $ is
$$ A\br{X} = \dfrac{A}{I\br{X}} = \dfrac{K\sbr{x_1, \dots, x_n}}{I\br{X}}. $$
\end{definition}

\lecture{3}{Tuesday}{15/01/19}

\begin{example}
Let $ f \in K\sbr{x_1, \dots, x_n} $ be irreducible. If $ n = 3 $, $ Z\br{f} $ is a surface. If $ n = 2 $, $ Z\br{f} $ is a curve.
\end{example}

\begin{example}
Let $ y - x^2 \in K\sbr{x, y} $. Then
$$ \function{A\br{X} = \dfrac{K\sbr{x, y}}{\abr{y - x^2}} \cong K\sbr{x, x^2}}{K\sbr{x}}{\sum_{i, j} a_{ij}x^ix^{2j} = \sum_{i, j} a_{ij}x^{2j + i}}{\sum_n b_nx^n}. $$
\end{example}

\begin{example}
Let $ xy - 1 \in K\sbr{x, y} $. Then
$$ A\br{X} = \dfrac{K\sbr{x, y}}{\abr{xy - 1}} \cong K\sbr{x, \dfrac{1}{x}}. $$
$ A\br{X} $ cannot be $ K\sbr{x} $.
\end{example}

\begin{definition}
A \textbf{Noetherian} topological space $ X $ is a topological space such that if
$$ C_1 \supseteq C_2 \supseteq \dots $$
is a decreasing chain of closed sets then there is a $ k $ such that $ C_k = C_{k + 1} = \dots $.
\end{definition}

\begin{example}
$ \A^n $. Recall that if $ A \subset B $ then $ I\br{B} \subset I\br{A} $. So using the definition above,
$$ I\br{C_1} \subseteq I\br{C_2} \subseteq \dots. $$
Since $ K\sbr{x_1, \dots, x_n} $ is Noetherian then $ I\br{C_i} $ stabilises. So $ I\br{C_k} = I\br{C_{k + 1}} = \dots $, but taking $ Z $, we recover $ C_k $ so $ C_k $ stabilises as well.
\end{example}

\pagebreak

\begin{theorem}
If $ X $ is Noetherian then any non-empty closed subset can be expressed as a finite union of irreducible closed sets $ X = Y_1 \cup \dots \cup Y_n $. Moreover, if we require that $ Y_i \subseteq Y_j $ then this expression is unique.
\end{theorem}

\begin{proof}
Let $ C $ be the collection of closed sets that do not satisfy that property. Let $ Y $ be a minimum closed inside $ C $, in particular $ Y $ is reducible, so $ Y = Y' \cup Y'' $, for $ Y', Y'' $ closed. Hence $ Y', Y'' \not\subset C $, so they can be expressed as a finite union of irreducibles, a contradiction. If $ Y_i \not\subset Y_j $, then suppose
$$ Y_1 \cup \dots \cup Y_n = X_1 \cup \dots \cup X_n. $$
Then $ Y_1 \subset X_1 \cup X_n $, in particular $ Y_1 = \bigcup_j \br{Y_1 \cap X_j} $, so there is a $ j $ such that $ Y_1 \cap X_j = Y_1 $, so $ Y_1 \subset X_j $. We can assume $ j = 1 $ and repeat the same argument to find that $ Y_1 = X_1 $, so consider $ \overline{Y \setminus Y_1} = Y_2 \cup \dots \cup Y_n $. But
$$ Y_2 \cup \dots \cup Y_n = X_2 \cup \dots \cup X_n, $$
and the result follows by induction.
\end{proof}

\begin{corollary}
Any affine variety in $ \A^n $ can be expressed equally as a union of irreducible algebraic varieties.
\end{corollary}

\begin{definition}
The \textbf{dimension} of a topological space is the supremum of $ n $ where
$$ Y_0 \subset \dots \subset Y_n $$
is a sequence of irreducible closed sets.
\end{definition}

\begin{example}
Dimension of $ \A^1 $ is one.
\end{example}

\begin{definition}
Let $ A $ be a ring and $ \pp $ be a prime ideal, then the \textbf{height} of $ \pp $ is the supremum of $ n $ where
$$ \pp_1 \subset \dots \subset \pp_n \subset \pp, $$
where $ \pp_i $ are prime. The \textbf{Krull dimension} of $ A $ is
$$ \sup_{\pp \ \text{prime}} height\br{\pp}. $$
\end{definition}

\begin{proposition}
If $ Y $ is affine then $ \dim\br{Y} = \dim\br{A\br{Y}} $.
\end{proposition}

\begin{proof}
Let $ C $ be a closed and irreducible set $ C \subset Y $, then $ I\br{C} \supset I\br{Y} $, then $ I\br{C} $ is prime.
\end{proof}

\begin{proposition}
Let $ K $ be a field and $ B $ be an integral domain which is a finitely generated algebra, then
\begin{itemize}
\item $ \dim\br{B} $ is the transcendence degree of $ K\br{B} $ over $ K $, and
\item if $ \pp \subseteq B $ is prime, then
$$ height\br{\pp} + \dim\br{\dfrac{B}{\pp}} = \dim\br{B}. $$
\end{itemize}
\end{proposition}

\begin{proof}
Atiyah Macdonald chapter 11.
\end{proof}

\begin{proposition}[Krull Hauptidealsatz]
Let $ A $ be a Noetherian ring and $ f \in A $ not a zero divisor and not a unit. Then every prime ideal containing $ f $ has height one.
\end{proposition}

\begin{proof}
Atiyah Macdonald page 122.
\end{proof}

\lecture{4}{Friday}{18/01/19}

\begin{proposition}
A Noetherian integral domain $ A $ is a UFD if and only if every prime ideal $ I $ of height one is principal.
\end{proposition}

\begin{theorem}
An irreducible variety $ Y \subseteq \A^n $ has dimension $ n - 1 $ if and only if $ Y = Z\br{f} $ where $ f $ is an irreducible polynomial in $ K\sbr{x_1, \dots, x_n} $.
\end{theorem}

\begin{proof}
\hfill
\begin{itemize}
\item[$ \implies $] If $ Y $ has dimension $ n - 1 $ then $ I\br{Y} $ has height one, by the above proposition $ I\br{Y} = \abr{f} $, so $ Y = Z\br{f} $.
\item[$ \impliedby $] Let $ I = I\br{Y} $ then $ I $ is prime, by the Krull Hauptidealsatz we have that $ I $ has height one, so $ \dim\br{Y} = n - 1 $.
\end{itemize}
\end{proof}

\pagebreak

\section{Projective varieties}

\begin{definition}
The \textbf{projective space} $ \P^n $ is defined as
$$ \P^n = \dfrac{\A^{n + 1} \setminus \cbr{0}}{\cbr{x \sim \lambda x \mid \lambda \in K^*}}. $$
A point in $ \P^n $ is written as $ \sbr{a_0 : \dots : a_n} = \overline{\br{a_0, \dots, a_n}} $.
\end{definition}

\begin{definition}
A \textbf{graded ring} $ R $ is a ring together with a decomposition
$$ R = \bigoplus_{d > 0} R_d, $$
where $ R_d $ are abelian groups and $ R_k \cdot R_t \subseteq R_{k + t} $.
\end{definition}

\begin{example}
$ K\sbr{x_0, \dots, x_n} $ is a graded ring, where $ R_d $ are monomials of degree $ d $.
\end{example}

\begin{notation}
Let $ A $ be $ K\sbr{x_0, \dots, x_n} $ without the grading and $ S $ be $ K\sbr{x_0, \dots, x_n} $ as a graded ring.
\end{notation}

\begin{definition}
An ideal $ I \subseteq S $ is \textbf{homogeneous} if
$$ I = \bigoplus_{d \ge 0} \br{I \cap S_d}. $$
If $ f = f_0 + \dots + f_d $, then $ f_i \in I $.
\end{definition}

\begin{remark}
$ I $ is homogeneous if and only if $ I = \abr{f_0, \dots, f_n} $, where $ f_i $ are homogeneous.
\end{remark}

\begin{lemma}
If $ I, J $ are homogeneous then
\begin{enumerate}
\item $ I + J $ is homogeneous,
\item $ IJ $ is homogeneous,
\item $ I \cap J $ is homogeneous, and
\item $ \sqrt{I} $ is homogeneous.
\end{enumerate}
\end{lemma}

\begin{proof}
\hfill
\begin{itemize}
\item[4.] Let $ f = f_0 + \dots + f_d \in \sqrt{I} $ then
$$ f^n = \br{f_0 + \dots + f_d}^n = f_d^n + \dots \in I \qquad \implies \qquad f_d^n \in I \qquad \implies \qquad f_d \in \sqrt{I}, $$
so $ f - f_d \in \sqrt{I} $, by induction $ f_i \in \sqrt{I} $.
\end{itemize}
\end{proof}

\begin{definition}
If $ f $ is homogeneous of degree $ k $ then
$$ f\br{\lambda \cdot x} = \lambda^k \cdot f\br{x}, $$
in particular $ f\br{x} = 0 $ if and only if $ f\br{\lambda \cdot x} = 0 $, so it makes sense to define
$$ Z\br{f} = \cbr{x \in \P^n \mid f\br{x} = 0}. $$
More generally, if $ I \subseteq S $ is a homogeneous ideal then
$$ Z\br{I} = \cbr{x \in \P^n \mid f \in I \ \text{homogeneous}, \ f\br{x} = 0}. $$
\end{definition}

\pagebreak

\begin{definition}
A subset $ X \subseteq \P^n $ is called a \textbf{projective variety} if $ X = Z\br{T} $ for some homogeneous ideal $ T $.
\end{definition}

\begin{proposition}
\hfill
\begin{itemize}
\item $ Z\br{S} \cup Z\br{T} = Z\br{ST} $.
\item $ \bigcap_\alpha Z\br{S_\alpha} = Z\br{\bigcup_\alpha S_\alpha} $.
\item $ Z\br{0} = \P^n $ and $ Z\br{1} = \emptyset $.
\end{itemize}
\end{proposition}

\begin{definition}
We define the \textbf{Zariski topology} on $ \P^n $ by taking closed sets to be $ Z\br{T} $ for some $ T $.
\end{definition}

\begin{definition}
\hfill
\begin{itemize}
\item A projective variety is \textbf{irreducible} if it is an irreducible topological space.
\item An open subset of a projective variety is called a \textbf{quasi-projective variety}.
\item The \textbf{dimension} of a projective variety is its dimension as a topological space.
\item If $ T \subseteq S $ then
$$ I\br{T} = \abr{f \in S \mid f \ \text{homogeneous}, \ \forall p \in T, \ f\br{p} = 0}. $$
\end{itemize}
\end{definition}

\begin{definition}
If $ X $ is a projective variety the \textbf{homogeneous coordinate ring} is
$$ S\br{X} = \dfrac{S}{I\br{X}}. $$
\end{definition}

\begin{definition}
If $ f \in S $ is linear and homogeneous, we call $ Z\br{f} $ a \textbf{hyperplane}.
\end{definition}

\lecture{5}{Monday}{21/01/19}

\begin{proposition}
$$ \function[\phi_i]{U_i = \P^n \setminus Z\br{x_i}}{\A^n}{\sbr{x_0 : \dots : x_n}}{\br{\dfrac{x_0}{x_i}, \dots, \dfrac{x_n}{x_i}}} $$
is a homeomorphism in the Zariski topology.
\end{proposition}

\begin{proof}
Let $ \phi = \phi_0 $ and $ U = U_0 $, let $ C \subseteq \A^n $ be a closed set then we claim that $ \phi^{-1}\br{C} $ is closed. Indeed, let $ C = Z\br{S} $, then $ \phi^{-1}\br{C} = Z\br{S'} \cup U $ where
$$ S' = \cbr{x_0^d \cdot f\br{\dfrac{x_1}{x_0}, \dots, \dfrac{x_n}{x_0}} \ \Bigg| \ f \in S}. $$
Similarly, let $ A \subseteq U $ is closed, we claim that $ \phi\br{A} $ is closed. Let $ \overline{A} $ be its closure in $ \P^n $, then $ \overline{A} = Z\br{B} $, so $ \phi\br{A} = Z\br{B'} $ where
$$ B' = \cbr{f\br{1, x_1, \dots, x_n} \mid f \in B}. $$
So we conclude that $ \phi $ is a homeomorphism.
\end{proof}

\begin{note*}
$ \abr{1} = S $ and $ \abr{x_0, \dots, x_n} \subsetneq S $ map to $ \emptyset $ under $ Z $. So in order to have a one-to-one correspondence we need the following.
\begin{itemize}
\item $ Z\br{I} = \emptyset $ if and only if $ \sqrt{I} \supseteq \abr{x_0, \dots, x_n} $. If we consider $ Z\br{I} $ in $ \A^{n + 1} $, note that $ x \in Z\br{I} $ if and only if $ \lambda x \in Z\br{I} $. So $ Z\br{I} = \emptyset $ if and only if $ Z\br{I} \subseteq \cbr{0} $. So $ \sqrt{I} \supseteq \abr{x_0, \dots, x_n} $.
\item $ I\br{Z\br{J}} = \sqrt{J} $ if $ Z\br{J} \ne \emptyset $, since $ I\br{Z\br{J}} = I\br{Z_a\br{J}} = \sqrt{J} $.
\end{itemize}
\end{note*}

\pagebreak

\begin{corollary}
$$ \correspondence{\text{projective varieties}}{\text{homogeneous radical ideals not} \ \abr{x_0, \dots, x_n}}, $$
$$ \correspondence{\text{irreducible projective varieties}}{\text{homogeneous radical prime ideals}}. $$
\end{corollary}

\begin{example}
$ \P^n $ is irreducible.
\end{example}

\begin{proposition}
\hfill
\begin{itemize}
\item $ \P^n $ is Noetherian, that is satisfies the descending chain condition.
\item Every projective variety can be written as a unique union of irreducible projective varieties. We call \textbf{irreducible components} the irreducible varieties in that decomposition.
\end{itemize}
\end{proposition}

\begin{theorem}
Let $ Y \subseteq \P^n $ be an irreducible projective variety. Then
$$ \dim\br{S\br{Y}} = \dim\br{Y} + 1. $$
\end{theorem}

\begin{proof}
Let
$$ \function[\phi_i]{U = \P^n \setminus Z\br{x_i}}{\A^n}{\sbr{x_0 : \dots : x_n}}{\br{\dfrac{x_0}{x_i}, \dots, \dfrac{x_n}{x_i}}}, $$
and $ Y_i = \phi_i\br{Y \cap U_i} $. Let
$$ \function{K\sbr{x_1, \dots, x_n}}{\br{S\br{Y}_{x_i}}_0}{f\br{x_1, \dots, x_n}}{\dfrac{x_i^{\partial f}f\br{\dfrac{x_1}{x_i}, \dots, \dfrac{x_n}{x_i}}}{x_i^{\partial f}}}, $$
then
$$ A\br{Y_i} = \dfrac{K\sbr{x_1, \dots, x_n}}{I\br{Y_i}} \cong \br{S\br{Y}_{x_i}}_0, $$
moreover $ S\br{Y}_{x_i} \cong A\br{Y_i}\sbr{x_i, x_i^{-1}} $. So
$$ \dim\br{S\br{Y}} = \dim\br{S\br{Y}_{x_i}} = \dim\br{A\br{Y_i}\sbr{x_i, x_i^{-1}}} = tra\br{K\br{Y_i}\br{x_i}} = \dim\br{Y_i} + 1. $$
Therefore if $ Y_i \ne \emptyset $, $ \dim\br{Y_i} = \dim\br{S\br{Y}} - 1 $ for all $ i $, but since $ U_i $ cover $ Y $ we have $ \dim\br{Y} = \max\cbr{\dim\br{Y_i}} $. (Exercise: if $ \cbr{U_n}_n $ is a finite cover of a topological space $ Y $ then $ \dim\br{Y} = \max\cbr{\dim\br{Y_i}} $) Since $ \dim\br{Y_i} $ are the same if $ Y_i \ne \emptyset $, we conclude that $ \dim\br{Y} = \dim\br{Y_d} $ for some $ d $.
\end{proof}

\lecture{6}{Tuesday}{22/01/19}

\begin{proposition}
Every Noetherian topological space is compact.
\end{proposition}

\begin{proof}
Let $ X $ be a Noetherian topological space and let $ \cbr{U_n} $ be a cover of $ X $. So consider $ C $, the collection of the union of finitely many open sets of $ \cbr{U_n} $. Since $ X $ is Noetherian $ C $ has a maximum element, say $ U_1 \cup \dots \cup U_n $. If $ U_1 \cup \dots \cup U_n \subsetneq X $ then there is $ x \in X $ not in the union, and we can find another $ U_{\alpha_0} \ni x $. But then
$$ U_1 \cup \dots \cup U_n \cup U_{\alpha_0} \supsetneq U_1 \cup \dots \cup U_n, $$
a contradiction. So $ X = U_1 \cup \dots \cup U_n $.
\end{proof}

\begin{corollary}
$ \P^n $, $ \A^n $, affine varieties, and projective varieties are all compact in the Zariski topology.
\end{corollary}

\begin{definition}
A variety $ X $ is \textbf{complete} if for any other variety $ Y $, the projection $ X \times Y \to Y $ is closed.
\end{definition}

\begin{example}
$ \P^n $ is complete. $ \A^n $ is not complete.
\end{example}

\pagebreak

\section{Morphisms}

\begin{definition}
Suppose $ Y $ is a quasi-affine variety and $ p \in Y $. We say that a function $ f : Y \to \A^1 $ is \textbf{regular} at $ p $ if there are $ g, h \in K\sbr{x_1, \dots, x_n} $ and $ U \ni p $ such that $ f = g / h $ in $ U $ with $ h \ne 0 $. A function is \textbf{regular} if it is regular for every $ p \in Y $.
\end{definition}

\begin{example}
Local is not global. Let $ X = Z\br{x_1x_4 - x_2x_3} \subseteq \A^4 $ and $ U = X \setminus Z\br{x_2, x_4} $. Then
$$ \function[\phi]{U}{\A^1}{\br{x_1, x_2, x_3, x_4}}{
\begin{cases}
\dfrac{x_1}{x_2} & x_2 \ne 0 \\
\dfrac{x_3}{x_4} & x_4 \ne 0
\end{cases}
} $$
is a regular function.
\end{example}

\begin{definition}
Let $ Y $ be a quasi-projective variety, $ f : Y \to \A^1 $, and $ p \in Y $. We say that $ f $ is \textbf{regular} at $ p $ if there are $ g, h $ homogeneous polynomials of the same degree and an open set $ U \ni p $ such that $ f = g / h $ on $ U $ and $ h \ne 0 $.
\end{definition}

\begin{lemma}
A regular function is continuous.
\end{lemma}

\begin{proof}
It is enough to show that $ f^{-1}\br{p} $ is closed. Since $ f $ is regular $ f = g / h $ on some neighbourhood $ U $, then
$$ f^{-1}\br{p} \cap U = Z\br{g - ph} \cap U. $$
\end{proof}

\begin{remark}
If $ X $ is irreducible then $ f = g $ on $ U \subseteq X $, then $ f = g $ on $ X $. Because the set where $ f - g = 0 $ is closed and dense.
\end{remark}

\begin{definition}
We will use the term \textbf{variety} to denote an affine, quasi-affine, projective, or quasi-projective variety.
\end{definition}

\begin{definition}
A \textbf{morphism} is $ f : X \to Y $ if $ f $ is continuous and for every $ U \subseteq Y $ and every function $ g : U \to \A^1 $ the composition $ g \circ f $ is regular.
\end{definition}

\begin{remark}
\hfill
\begin{itemize}
\item Let $ f : X \to Y $ and $ g : Y \to Z $ then the composition $ g \circ f $ of these two morphisms is the composition of $ f $ and $ g $ as functions.
\item A morphism $ f : X \to Y $ is an \textbf{isomorphism} if there is a morphism $ g : Y \to X $ such that $ f \circ g = id $ and $ g \circ f = id $.
\end{itemize}
\end{remark}

\begin{definition}
Let $ X $ be a variety. Denote the set of all regular functions of $ X $ by $ \OO\br{X} $. If $ p \in X $ the \textbf{local ring} at $ p \in X $ is
$$ \OO_p = \lim_{\xrightarrow[U \ni p]{}}\br{\OO\br{U}}. $$
An element of $ \OO_p $ is a pair $ \br{U, f} $, where $ p \in U $ and $ f $ is regular at $ p $, moreover $ \br{U, f} \sim \br{V, g} $ if $ f = g $ on $ U \cap V $.
\end{definition}

\pagebreak

\lecture{7}{Friday}{25/01/19}

\begin{definition}
Let $ Y $ be an irreducible variety, the \textbf{function field} $ K\br{Y} $ of $ Y $ is the field whose elements are pairs $ \br{U, f} $ where $ U $ is open and $ f $ is regular on $ U $, and
$$ \br{U, f} + \br{V, g} = \br{U \cap V, f + g}. $$
\end{definition}

\begin{remark}
\hfill
\begin{itemize}
\item $ K\br{Y} $ is indeed a field for if $ \br{U, f} \ne 0 $ then $ U^{-1} = U \setminus Z\br{f} $, so $ \br{U^{-1}, 1 / f} $ is the inverse to $ \br{U, f} $.
\item $ K\br{Y} $ is the quotient field of $ A\br{Y} $ or $ S\br{Y} $.
\item $ \OO\br{Y} \hookrightarrow \OO_p \hookrightarrow K\br{Y} $ for all $ p \in Y $.
\end{itemize}
\end{remark}

\begin{theorem}
If $ Y \subseteq \A^n $ is an irreducible affine variety with coordinate ring $ A\br{Y} $ then
\begin{enumerate}
\item $ \OO\br{Y} = A\br{Y} $,
\item for all $ p \in Y $, if $ \mm_p = \cbr{f \in A\br{Y} \mid f\br{p} = 0} $ then we have a one-to-one correspondence
$$ \correspondence{\text{points of} \ Y}{\text{maximal ideals of} \ A\br{Y}}, $$
\item for all $ p \in Y $, $ \OO_p \cong A\br{Y}_{\mm_p} $ and $ \dim\br{\OO_p} = \dim\br{Y} $, and
\item $ K\br{Y} $ is the quotient field of $ A\br{Y} $.
\end{enumerate}
\end{theorem}

\begin{proof}
\hfill
\begin{enumerate}
\item Notice that there is a natural map $ A \to \OO\br{Y} $ with kernel $ I\br{Y} $, so there is an injection $ A\br{Y} \hookrightarrow \OO\br{Y} $, that is
$$ A\br{Y} \subseteq \OO\br{Y} \subseteq \bigcap_{p \in Y} \OO_p = \bigcap_{\mm_p} A\br{Y}_{\mm_p} = A\br{Y}, $$
so $ A\br{Y} = \OO\br{Y} $.
\item We know that points of $ Y $ correspond to maximal ideals $ \mm_p \supseteq I\br{Y} $. Taking the quotient, we get maximal ideals inside $ A\br{Y} $.
\item There is a natural map $ A\br{Y}_{\mm_p} \to \OO_p $, which is injective by $ \alpha : A\br{Y} \hookrightarrow \OO\br{Y} $, and it is surjective by definition of $ \OO_p $. Moreover,
$$ \dim\br{\OO_p} = \dim\br{A_p}_{\mm_p} = height\br{\mm_p} = \dim\br{Y}. $$
\item The quotient field of $ A\br{Y} $ is the quotient field of $ \OO_p $ for all $ p $, by $ 3 $, which is $ K\br{Y} $ by definition.
\end{enumerate}
\end{proof}

\begin{theorem}
Let $ Y \subseteq \P^n $ be irreducible and projective. Then
\begin{enumerate}
\item $ \OO\br{Y} = K $,
\item for all $ p \in Y $, $ \mm_p $ as before, $ \OO_p \cong \br{S\br{Y}_{\mm_p}}_0 $, and
\item $ K\br{Y} \cong \br{S\br{Y}_{\br{0}}}_0 $.
\end{enumerate}
\end{theorem}

\pagebreak

\begin{proof}
Recall that
$$ \function[\phi_i]{U_i = \P^n \setminus Z\br{x_i}}{\A^n}{\sbr{x_0 : \dots : x_n}}{\br{\dfrac{x_0}{x_i}, \dots, \dfrac{x_n}{x_i}}} $$
gives $ \phi_i^* : A\br{Y_i} \cong \br{S\br{Y}_{x_i}}_0 $ and $ Y_i = \phi_i\br{Y \cap U_i} $.
\begin{enumerate}
\item $ K \subseteq \OO\br{Y} $. Take $ f \in \OO\br{Y} $, so $ f $ is regular at each $ Y_i $, but $ \OO\br{Y_i} \cong A\br{Y_i} $, also by $ \phi_i^* $, $ A\br{Y_i} \cong \br{S\br{Y}_{x_i}}_0 $. Thus $ f = g_i / x_i^{n_i} $, where $ n_i = \deg\br{g_i} $, in particular $ x_i^{n_i}f \in S\br{Y}_{n_i} $. Now, set $ N \ge \sum_i n_i $, then $ S\br{Y}_N \cdot f \subseteq S\br{Y}_N $, so we can iterate this process to obtain $ S\br{Y}_N \cdot f^q \subseteq S\br{Y}_N $. In particular $ x_0^Nf \subset S $, hence $ S\br{Y}\sbr{f} $ is contained in $ x_0^{-N}S\br{Y} $. Therefore $ f $ is integral since $ S\br{Y}\sbr{f} $ is finitely generated. There are $ a_i \in S $ such that
$$ f^k + a_1f^{k - 1} + \dots + a_k = 0. $$
Since $ f $ is homogeneous of degree zero we can take the constant terms of $ a_i $ and still have an equation, hence $ a_i \in K $.
\item Let $ p \in Y $, then $ p \in Y_i $, by the previous theorem we know that $ \OO_p \cong A\br{Y_i}_{\mm_p} $. By $ \phi_i^* $, $ \OO_p \cong \br{\br{S\br{Y}_{x_i}}_{\mm_p}}_0 $, but since $ x_i \notin \mm_p $, hence $ \OO_p \cong \br{S\br{Y}_{\mm_p}}_0 $.
\item Recall that the quotient field of $ Y $ is $ K\br{Y} = K\br{Y_i} $, but $ K\br{Y_i} $ is the quotient field of the coordinate ring $ A\br{Y_i} $, by $ \phi_i^* $, this is $ \br{S\br{Y}_{\br{0}}}_0 $.
\end{enumerate}
\end{proof}

\lecture{8}{Monday}{28/01/19}

\begin{proposition}
Let $ X $ be an irreducible variety and $ Y $ be an irreducible affine variety, then we have a bijection
$$ \alpha : Hom\br{X, Y} \xrightarrow{\sim} Hom\br{A\br{Y}, \OO\br{X}}, $$
the set of morphisms from $ X $ to $ Y $ to the set of $ K $-algebra homomorphisms.
\end{proposition}

\begin{proof}
Given a morphism $ \phi : X \to Y $, by definition of morphism, $ \phi $ takes regular functions at $ Y $ to regular functions at $ X $. So if $ f \in A\br{Y} $ then $ \phi \circ f \in \OO\br{X} $. Conversely, let $ h : A\br{Y} \to \OO\br{X} $ be a homomorphism of $ K $-algebras. Recall that
$$ A\br{Y} = \dfrac{A}{I\br{Y}} = \dfrac{k\sbr{x_1, \dots, x_n}}{I\br{Y}}. $$
Take $ \overline{x_i} \in A\br{Y} $ and let $ y_i = h\br{\overline{x_i}} \in \OO\br{X} $ and define
$$ \function[\psi]{X}{\A^n}{p}{\br{y_1\br{p}, \dots, y_n\br{p}}}. $$
We claim that $ Im\br{\psi} \subseteq Y $, but since $ Y = Z\br{I\br{Y}} $, it is enough to show that if $ f \in I\br{Y} $ then $ f\br{\psi\br{p}} = 0 $.
$$ f\br{\psi\br{p}} = f\br{y_1\br{p}, \dots, y_n\br{p}} = f\br{h\br{\overline{x_1}\br{p}}, \dots, h\br{\overline{x_n}\br{p}}} = h\br{f\br{x_1, \dots, x_n}}\br{p} = 0. $$
\end{proof}

\begin{lemma}
If $ X, Y $ are as before then $ \psi : X \to Y $ is a morphism if and only if $ \psi_i = x_i \circ \psi $ are regular functions.
\end{lemma}

\begin{proof}
Suppose $ \psi_i $ are regular functions, then if $ p $ is a polynomial $ p \circ \psi $ is regular, but since regular functions are quotients of polynomials, we conclude that $ f \circ \psi $ is regular for any regular function $ f $.
\end{proof}

\begin{corollary}
If $ X, Y $ are affine then $ X \cong Y $ if and only if $ A\br{X} \cong A\br{Y} $.
\end{corollary}

\begin{corollary}
The correspondence $ X \mapsto A\br{X} $ induces an arrow reversing correspondence between the category of affine varieties and the category of $ K $-integral domains.
\end{corollary}

\lecture{9}{Tuesday}{29/01/19}

Lecture 9 is a problem class.

\lecture{10}{Friday}{01/02/19}

Lecture 10 is a problem class.

\pagebreak

\section{Rational maps}

\lecture{11}{Monday}{04/02/19}

\begin{definition}
Let $ X, Y $ be varieties. A \textbf{rational map} $ f : X \dashrightarrow Y $ is a pair $ \br{U, f_U} $ where $ U \subseteq X $ is open and $ f_U $ is a morphism on $ U $ and we identify $ \br{U, f_U} \sim \br{V, g_V} $ if $ f_U = g_V $ on $ U \cap V $.
\end{definition}

\begin{lemma}
If $ X, Y $ are varieties and $ \phi, \psi : X \to Y $ such that $ \phi = \psi $ on $ U \subseteq X $, then $ \phi = \psi $ on $ X $.
\end{lemma}

\begin{proof}
We can assume that $ Y \subseteq \P^n $ for some $ n $, and hence we reduce to the case where $ Y = \P^n $. So the product is $ \phi \times \psi : X \to \P^n \times \P^n $. Let $ \Delta \subseteq \P^n \times \P^n = Z\br{x_iy_j - x_jy_i} $. Since $ \phi = \psi $ on $ U $, $ \br{\phi \times \psi}\br{U} \subseteq A $, so $ \br{\phi \times \psi}\br{\overline{U}} = \br{\phi \times \psi}\br{X} \subseteq \Delta $.
\end{proof}

\begin{definition}
\hfill
\begin{itemize}
\item A \textbf{dominant rational map} is a rational map $ f : X \dashrightarrow Y $, such that $ f_U\br{U} $ is dense for some, and hence all, $ \br{U, f_U} $.
\item A \textbf{birational map} is a dominant rational map $ f : X \dashrightarrow Y $ such that $ f $ admits an inverse $ g : Y \dashrightarrow X $.
\end{itemize}
\end{definition}

\begin{theorem}
For any two varieties $ X, Y $ we have a correspondence
$$ \correspondence{\text{dominant rational maps} \ f : X \to Y}{K \text{-algebra homomorphisms} \ K\br{Y} \to K\br{X}}. $$
\end{theorem}

\begin{proof}
Given a rational map $ f : X \dashrightarrow Y $ and let $ g \in K\br{Y} $. Let $ f_U $ be a representative of $ f $ then we have that if $ \br{V, g} = g $, $ g \circ f_U \in K\br{X} $. Since we can cover $ Y $ using affine varieties, we can assume $ Y $ is affine then $ K\br{Y} = K\br{A\br{Y}} $. If we start with a homomorphism $ \theta : K\br{Y} \to K\br{X} $, let $ y_1, \dots, y_n \in A\br{Y} $ be the generators of $ A\br{Y} $, then $ \theta\br{y_i} \in K\br{X} $. We can find $ U $ such that $ \theta\br{y_i} $ are regular at $ U $. Then this induces a map $ A\br{Y} \to \OO\br{U} $. But then we have a morphism $ U \to Y $, and moreover this is the inverse of the map we defined previously.
\end{proof}

\begin{definition}
\hfill
\begin{itemize}
\item A field extension $ L / K $ is \textbf{separably generated} if there is a transcendence basis $ \cbr{x_i} $ for $ L / K $ such that $ L $ is a separable algebraic extension of $ K\br{\cbr{x_i}} $.
\item Primitive element theorem. If $ L / K $ is finite and separable then $ L / K\br{\alpha} $ for some $ \alpha \in L $. If $ L $ is infinite and $ \beta_1, \dots, \beta_n $ are generators for $ L / K $ then $ \alpha = c_1\beta_1 + \dots + c_n\beta_n $ for $ c_i \in K $.
\item If $ K $ is perfect, any finitely generated extension $ L / K $ is separably generated.
\end{itemize}
\end{definition}

\begin{theorem}
Any variety $ X $ of dimension $ n $ is birational to a hypersurface $ Y \subseteq \P^{n + 1} $.
\end{theorem}

\begin{proof}
Since $ K\br{X} = K $ is finitely generated, by the theorem above it is separably generated. So we can find a transcendence basis $ x_1, \dots, x_n \in K $ such that $ K / k\br{x_1, \dots, x_n} $ is finite and separable. By the primitive element theorem, $ K = k\br{x_1, \dots, x_n, y} $ for some $ y $ which is algebraic over $ k\br{x_1, \dots, x_n} $, so $ y $ is the solution of a polynomial equation $ f $ in $ k\br{x_1, \dots, x_n} $. In particular if we clear denominators we get a polynomial $ f\br{x_1, \dots, x_n, y} $ in $ \A^{n + 1} $, by taking $ Z\br{f} $ we get a hypersurface and taking its projective closure we get a hypersurface in $ \P^n $.
\end{proof}

\lecture{12}{Tuesday}{05/02/19}

\begin{corollary}
The following are equivalent.
\begin{itemize}
\item $ F : X \dashrightarrow Y $ is birational.
\item There exist $ U, V $ such that $ F : U \to V $ is an isomorphism.
\item $ K\br{Y} \cong K\br{X} $.
\end{itemize}
\end{corollary}

\pagebreak

\begin{definition}
The \textbf{blow-up} of $ \A^n $ at the origin $ 0 $, denoted by $ \widetilde{\A^n} $, is
$$ Z\br{x_iy_j - x_jy_i} \subseteq \A^n \times \P^{n - 1}. $$
$$
\begin{tikzcd}
\widetilde{\A^n} \arrow[hookrightarrow]{r} \arrow[swap]{dr}{\pi} & \A^n \times \P^{n - 1} \arrow{d}{\br{x, y} \mapsto x} \\
& \A^n
\end{tikzcd}.
$$
\end{definition}

\begin{proposition}
\hfill
\begin{enumerate}
\item Let $ P \in \A^n $, if $ P \ne 0 $ then $ \pi^{-1}\br{P} $ is a single point, and $ \widetilde{\A^n} \setminus \pi^{-1}\br{0} \cong \A^n \setminus \cbr{0} $.
\item $ \pi^{-1}\br{0} \cong \P^{n - 1} $.
\item Points of $ \pi^{-1}\br{0} $ are in one-to-one correspondence with the set of lines through the origin.
\item $ \widetilde{\A^n} $ is irreducible.
\end{enumerate}
\end{proposition}

\begin{proof}
\hfill
\begin{enumerate}
\item If $ P \ne 0 $ then $ y_j = x_jy_i/x_i $ and this is true for every $ j $, so this gives a unique point in $ \P^{n - 1} $.
\item Obvious.
\item A line through the origin is given by $ x_i = ta_i $ for $ t \ne 0 $. Taking $ \pi^{-1} $ of this line we get $ x_i = ta_i $ and $ y_i = ta_i = a_i $. In other words if $ x \ne 0 $, $ \pi^{-1}\br{X} = \br{X, \sbr{X}} $.
\item $ \widetilde{\A^n} \setminus \pi^{-1}\br{0} \cong \A^n \setminus \cbr{0} $ is dense and irreducible, by $ 3 $.
\end{enumerate}
\end{proof}

\begin{definition}
If $ Y \ni 0 $ is a closed subvariety of $ \A^n $ we define the \textbf{blow-up} of $ Y $ at $ 0 $ by
$$ \widetilde{Y} = \overline{\pi^{-1}\br{Y \setminus \cbr{0}}}. $$
More generally, we can blow-up any point by taking an affine change of coordinates. We also get a birational map $ \pi : \widetilde{Y} \to Y $.
\end{definition}

\begin{example}
Let $ Y = Z\br{y^2 - x^2\br{x + 1}} $. The equations of the blow-up are
$$
\begin{cases}
y^2 = x^2\br{x + 1} \\
xu = yt
\end{cases},
$$
where $ \sbr{t : u} \in \P^1 $. Suppose $ t \ne 0 $.
$$
\begin{cases}
y^2 = x^2\br{x + 1} \\
y = xu
\end{cases}
\qquad \implies \qquad \br{xu}^2 = x^2\br{x + 1} \qquad \implies \qquad x^2\br{u^2 - x - 1} = 0.
$$
\end{example}

\begin{example}
Let $ y^2 = x^3 $.
$$
\begin{cases}
y^2 = x^3 \\
y = xu
\end{cases}
\qquad \implies \qquad \br{xu}^2 = x^3 \qquad \implies \qquad x^2\br{u^2 - x} = 0.
$$
\end{example}

\pagebreak

\section{Nonsingular varieties}

\lecture{13}{Friday}{08/02/19}

\begin{definition}
Let $ Y \subseteq \A^n $ be an affine variety of dimension $ r $, and suppose $ I\br{Y} = \abr{f_1, \dots, f_k} $. $ Y $ is \textbf{nonsingular} at $ P \in Y $ if $ rank\br{\tpd{f_i\br{P}}{x_j}} = n - r $. $ Y $ is \textbf{nonsingular} if it is nonsingular at every $ P \in Y $.
\end{definition}

\begin{example}
Let $ x^2 = x^4 + y^4 \subseteq \A^2 $, so $ f = x^2 - x^4 - y^4 $.
$$ \dpd{f}{x} = 2x - 4x^3 = 0 \qquad \implies \qquad x\br{1 - 2x^2} = 0 \qquad \implies \qquad x = 0 \ \text{or} \ 2x^2 = 1, $$
$$ \dpd{f}{y} = -9y^3 = 0 \qquad \implies \qquad y = 0 \qquad \implies \qquad x^2 = x^4 \qquad \implies \qquad x = 0 \ \text{or} \ x^2 = 1, $$
so $ Sing\br{Y} = \cbr{\br{0, 0}} $.
\end{example}

\begin{example}
Let $ Y = Z\br{f} = Z\br{y^2 - x^3} $.
$$ \dpd{f}{x} = -3x^2 = 0, \qquad \dpd{f}{y} = 2y = 0, $$
so $ Sing\br{Y} = \cbr{\br{0, 0}} $.
\end{example}

\begin{definition}
Let $ A $ be a Noetherian local ring with maximal ideal $ \mm $, and residue field $ A / \mm = K $. $ A $ is a \textbf{regular local ring} if $ \dim_K\br{\mm / \mm^2} = \dim\br{A} $.
\end{definition}

\begin{note*}
$ \br{\mm / \mm^2}^* $ is called the \textbf{Zariski-tangent space}.
\end{note*}

Claim that $ \mm / \mm^2 $ is a $ K $-vector space for $ K = A / \mm $.

\begin{theorem}
Let $ Y \subseteq \A^n $ be an affine variety. Then $ Y $ is nonsingular at $ P $ if and only if $ \OO_P $ is a regular local ring.
\end{theorem}

\begin{proof}
Let $ P = \br{a_1, \dots, a_n} \in Y $ with corresponding maximal ideal $ I_P = \abr{x_1 - a_1, \dots, x_n - a_n} $. We define a map
$$ \function[\theta_P]{A = K\sbr{x_1, \dots, x_n}}{K^n}{f}{\br{\dpd{f\br{P}}{x_1}, \dots, \dpd{f\br{P}}{x_n}}}. $$
Note that $ \theta\br{\br{x_i - a_i}\br{x_j - a_j}} = 0 $, hence $ \theta_P\br{I_P^2} = 0 $, in particular we have an isomorphism $ I_P / I_P^2 \cong K^n $. By the isomorphism, if $ \alpha = I\br{Y} = \abr{f_1, \dots, f_t} $ then the rank of $ \tpd{f_i\br{P}}{x_j} $ corresponds to the dimension of $ \alpha $ under the isomorphism, which is $ \overline{\alpha} $ in $ I_P / I_P^2 $, $ \br{\alpha + I_P} / I_P^2 $. Now $ \OO_P = \br{A / \alpha}_{I_P} $. If $ \mm = \br{I_P + \alpha} / \alpha $ then $ \mm^2 = \br{I_P^2 + \alpha} / \alpha $, so $ \mm / \mm^2 = I_P / \br{I_P^2 + \alpha} $. So
$$ r = \dim\br{\dfrac{\mm}{\mm^2}} = \dim\br{\dfrac{I_P}{I_P^2 + \alpha}} = \dim\br{\dfrac{I_P}{I_P^2}} - \dim\br{\dfrac{I_P^2 + \alpha}{I_P^2}} = n - rank\br{\dpd{f_i}{x_j}}. $$
So $ \OO_P $ is regular if and only if $ rank\br{\tpd{f_i}{x_j}} = n - r $.
\end{proof}

\begin{definition}
Let $ X $ be a variety. $ X $ is \textbf{nonsingular} at $ P $ if $ \OO_P $ is a regular local ring.
\end{definition}

\begin{theorem}
Let $ Y $ be a variety. Then $ Sing\br{Y} $ is a proper and closed set. The set of nonsingular points of $ Y $ is open and dense.
\end{theorem}

\begin{proof}
Prove that $ Sing\br{Y} $ is closed, first. We know that the rank of the Jacobian is at most $ n - r $, therefore the singular points occurs when the rank is less than $ n - r $, which is to say that $ Sing\br{Y} $ is given by the vanishing of the $ \br{n - r} \times \br{n - r} $ minors of $ \tpd{f_i}{x_j} $ and $ I\br{Y} $, hence is closed. To prove that it is proper $ Sing\br{Y} \subsetneq Y $.
\end{proof}

\lecture{14}{Monday}{11/02/19}

Lecture 14 is a problem class.

\lecture{15}{Tuesday}{12/02/19}

Lecture 15 is a problem class.

\pagebreak

\section{Intersections in projective space}

\lecture{16}{Friday}{15/02/19}

\begin{theorem}
Let $ Y, Z \subseteq \A^n $ be varieties, with $ \dim\br{Y} = r $ and $ \dim\br{Z} = s $ then every irreducible component has dimension at least $ r + s - n $.
\end{theorem}

\begin{proof}
Suppose $ Z $ is a hypersurface. Then if $ Y \subseteq Z $ the theorem holds, and if $ Y \nsubseteq Z $ the theorem is true by homework $ 1 $. Let $ Z $ be general. Consider the diagonal in $ \A^{2n} $ given by the image of the isomorphism $ P \mapsto P \times P $, then $ Y \cap Z $ corresponds to $ \br{Y \times Z} \cap \Delta $. Recall that
$$ \Delta = Z\br{x_1 - y_1} \cap \dots \cap Z\br{x_n - y_n}, $$
by the first case $ n $ times we have that each irreducible component has dimension
$$ \br{r + s} - n - 2n = r + s - n. $$
\end{proof}

\begin{theorem}
Let $ Y, Z \subseteq \P^n $ be varieties, where $ \dim\br{Y} = r $ and $ \dim\br{Z} = s $, then each irreducible component of $ Y \cap Z $ has dimension at least $ r + s - n $. Moreover, if $ r + s - n \ge 0 $ then $ Y \cap Z \ne \emptyset $.
\end{theorem}

\begin{proof}
Take the affine cone of $ Y $ and $ Z $, $ C\br{Y} $ and $ C\br{Z} $, since $ 0 \in C\br{Y} \cap C\br{Z} $ we apply the previous theorem to get
$$ \br{r + 1} + \br{s + 1} - \br{n + 1} = r + s - n + 1, $$
so therefore $ Y \cap Z \ne \emptyset $.
\end{proof}

\begin{definition}
A \textbf{numerical polynomial} is a polynomial $ f \in \Q\sbr{x} $ such that $ f\br{n} \in \Z $ for $ n \g 0 $, for $ n $ sufficiently large.
\end{definition}

\begin{theorem}
\hfill
\begin{enumerate}
\item If $ f \in \Q\sbr{x} $ is a numerical polynomial then there are $ c_0, \dots, c_r \in \Z $ such that
$$ f\br{x} = c_0\twobyone{x}{r} + \dots + c_r\twobyone{x}{0}. $$
\item If for $ n \g 0 $, $ \Delta f = f\br{n + 1} - f\br{n} = q $ and $ q $ is a numerical polynomial, then there exists $ p $ such that for $ n \g 0 $, $ p\br{n} = f\br{n} $.
\end{enumerate}
\end{theorem}

\begin{proof}
\hfill
\begin{enumerate}
\item By linear algebra we can find $ c_0, \dots, c_r \in \Q $ such that
$$ f\br{x} = c_0\twobyone{x}{r} + \dots + c_r\twobyone{x}{0}, $$
then
$$ \Delta f = c_0\twobyone{x}{r - 1} + \dots + c_{r - 1}\twobyone{x}{0}. $$
By induction on the degree of $ f $ we have that $ c_0, \dots, c_{r - 1} \in \Z $, but since $ f\br{n} \in \Z $ for $ n \g 0 $ then $ c_r \in \Z $.
\item If
$$ q = c_0\twobyone{x}{r} + \dots + c_r\twobyone{x}{0}, $$
set
$$ p = c_0\twobyone{x}{r + 1} + \dots + c_r\twobyone{x}{1}. $$
$ \Delta p = q $ gives $ \Delta \br{f - p}\br{n} = 0 $.
\end{enumerate}
\end{proof}

\pagebreak

\begin{definition}
\hfill
\begin{itemize}
\item Let $ S $ be a graded ring. A \textbf{graded $ S $-module} is a module $ M $ with a decomposition
$$ M = \bigoplus_{d \in \Z} M_d, $$
such that $ S_k \cdot M_d \subseteq M_{d + k} $.
\item Let $ l \in \Z $. The \textbf{twisted module} $ M\br{l} $ is the graded $ S $-module given by $ M\br{l}_k = M_{l + k} $.
\item $ Ann\br{M} = \cbr{x \in S \mid xM = 0} $.
\end{itemize}
\end{definition}

\begin{theorem}
Let $ M $ be a finitely generated graded $ S $-module. Then there is a filtration
$$ 0 = M^0 \subseteq \dots \subseteq M^r = M, $$
such that $ M^i / M^{i - 1} \cong \br{S / \pp_i}\br{l} $ for some $ \pp_i $ prime ideals and $ l_i \in \Z $, such that
\begin{itemize}
\item prime $ \pp \supseteq Ann\br{M} $ if and only if $ \pp \subseteq \pp_i $, that is $ \pp_i $ are minimal primes of $ M $, and
\item for each minimal prime $ \pp $ of $ M $ the number of times $ \pp $ appears in the set $ \cbr{\pp_1, \dots, \pp_r} $ is $ len_{S_\pp}\br{M_\pp} $.
\end{itemize}
\end{theorem}

\lecture{17}{Monday}{18/02/19}

\begin{definition}
Let $ \pp $ be a minimal prime of a graded $ S $-module $ M $. Then the \textbf{multiplicity} of $ M $ at $ \pp $ is $ len_{S_\pp}\br{M_\pp} $.
\end{definition}

\begin{definition}
Let $ M $ be a graded $ S = K\sbr{x_1, \dots, x_n} $-module. The \textbf{Hilbert function} of $ M $ is $ \phi_M\br{l} = \dim_K\br{M_l} $.
\end{definition}

\begin{theorem}
Let $ M $ be a graded $ S = K\sbr{x_1, \dots, x_n} $-module. Then for $ n \g 0 $, there is a unique polynomial $ P_M \in \Q\sbr{x} $ such that $ \phi_M\br{n} = P_M\br{n} $. $ P_M $ is called the \textbf{Hilbert polynomial}. It is a polynomial of degree $ \dim\br{Z\br{Ann\br{M}}} $.
\end{theorem}

\begin{proof}
By the previous theorem, $ M $ has a filtration
$$ 0 = M^0 \subseteq \dots \subseteq M^r = M, $$
such that $ M^i / M^{i - 1} $ is of the form $ \br{S / \pp_i}\br{l_i} $. Without loss of generality we can assume $ M = S / \pp $, since $ l_i $ amounts to a translation $ z \mapsto z + l_i $. If $ \pp = \abr{x_0, \dots, x_n} $ then $ S / \pp \cong K $, in particular $ \phi_M\br{l_i} = 0 $ if $ l_i > 0 $, but then take $ P_M = 0 $. We can assume $ \dim\br{0} = -1 $ and $ \dim\br{\emptyset} = -1 $. Suppose $ \pp \ne \abr{x_0, \dots, x_n} $. Then there is $ x_i \notin \pp $ and consider the short exact sequence
$$ 0 \to M \xrightarrow{x_i} M \to \dfrac{M}{x_iM} = M'' \to 0. $$
Taking Hilbert function we get that
$$ \phi_{M''}\br{l} = \phi_M\br{l} - \phi_M\br{l - 1} = \Delta\phi_M\br{l - 1}. $$
Note that $ Ann\br{M''} = Ann\br{M} \cup \cbr{x_i} $, so $ Z\br{Ann\br{M''}} = Z\br{\pp} \cap Z\br{x_i} $. Note that
$$ \dim\br{Ann\br{M''}} = \dim\br{Z\br{\pp}} - 1, $$
so we apply induction over $ \dim\br{Ann\br{M}} $. Thus $ \phi_{M''} $ agrees with a polynomial $ P_{M''}\br{n} $ for $ n \g 0 $ but then $ \Delta\phi_M = P_{M''} $ for $ n \g 0 $, so $ \phi_M $ agrees with a polynomial of degree
$$ \dim\br{Ann\br{M''}} + 1 = \dim\br{Z\br{\pp}}. $$
\end{proof}

\pagebreak

\begin{definition}
If $ Y \subseteq \P^n $ of dimension $ r $, the \textbf{Hilbert polynomial} of $ Y $ is the Hilbert polynomial of $ S\br{Y} $. The degree of $ Y $ is $ r! $ times the leading coefficient of $ P_Y $.
\end{definition}

\begin{theorem}
\hfill
\begin{enumerate}
\item If $ Y \ne \emptyset $, then $ \deg\br{Y} \in \Z_{> 0} $.
\item $ \deg\br{\P^n} = 1 $.
\item If $ Y = Y_1 \cup Y_2 $ with $ \dim\br{Y_i} = r $ and $ \dim\br{Y_1 \cap Y_2} < r $ then $ \deg\br{Y} = \deg\br{Y_1} + \deg\br{Y_2} $.
\item If $ H $ is a hypersurface generated by $ f $ then $ \deg\br{H} = \deg\br{f} $.
\end{enumerate}
\end{theorem}

\begin{proof}
\hfill
\begin{enumerate}
\item Obvious.
\item
$$ \phi_{\P^n}\br{z} = \twobyone{z + n}{n} = \dfrac{1}{n!}\br{z} \dots \br{n + 1} = \dfrac{1}{n!}z^n + \dots. $$
\item Let $ I = I\br{Y} $, $ I_1 = I\br{Y_1} $, and $ I_2 = I\br{Y_2} $. Consider the short exact sequence
$$ 0 \to \dfrac{S}{I} \to \dfrac{S}{I_1} \oplus \dfrac{S}{I_2} \to \dfrac{S}{I_1 + I_2} \to 0. $$
Taking Hilbert function,
$$ \phi_{\tfrac{S}{I_1 + I_2}} = \phi_{\tfrac{S}{I_1} \oplus \tfrac{S}{I_2}} - \phi_{\tfrac{S}{I}}. $$
Since $ Z\br{I_1 + I_2} = Y_1 \cap Y_2 $ and $ \dim\br{Y_1 \cap Y_2} < r $ we have that $ \phi_{S / I_1 \oplus S / I_2} $ and $ \phi_{S / I} $ have the same leading coefficients, hence $ \deg\br{Y} = \deg\br{Y_1} + \deg\br{Y_2} $.
\item Suppose $ \deg\br{f} = d $ then consider the short exact sequence
$$ 0 \to S\br{-d} \xrightarrow{f}{S} \to \dfrac{S}{\abr{f}} \to 0. $$
Taking Hilbert functions,
$$ \phi_{\tfrac{S}{\abr{f}}}\br{z} = \phi_S\br{z} - \phi_S\br{z - d} = \twobyone{z + n}{n} - \twobyone{z - d + n}{n} = \dfrac{d}{\br{n - 1}!}z^{n - 1} + \dots. $$
\end{enumerate}
\end{proof}

\pagebreak

\lecture{18}{Tuesday}{19/02/19}

Let $ Y \subseteq \P^n $ be a projective variety and $ H $ a hypersurface then $ Y \cap H = Z_1 \cup \dots \cup Z_k $, where each $ Z_j $ has dimension $ r - 1 = \dim\br{Y} - 1 $. Suppose $ I\br{Z_j} = \pp_j $, then each $ \pp_j $ is a minimal prime of $ S / \br{I_Y + I_H} $, then the \textbf{intersection multiplicity} $ i\br{Y, H; Z_j} $ is the multiplicity of $ S / \br{I_Y + I_H} $ at $ \pp_j $.

\begin{theorem}
Let $ Y \subseteq \P^n $ be a variety and $ H $ a hypersurface such that $ Y \nsubseteq H $. If $ Y \cap H = Z_1 \cup \dots \cup Z_k $ then
$$ \sum_{j = 1}^k i\br{Y, H; Z_j}\deg\br{Z_j} = \deg\br{Y}\deg\br{H}. $$
\end{theorem}

\begin{corollary}[B\'ezout's theorem]
If $ Y, H \subseteq \P^2 $ are curves and $ Y \cap H = \cbr{P_1, \dots, P_k} $ then
$$ \sum_{j = 1}^k i\br{Y, H; P_j} = \deg\br{Y}\deg\br{H}. $$
\end{corollary}

\begin{proof}
Supppose $ H $ is generated by $ f $, where $ \deg\br{f} = d $, and let $ I = I\br{Y} $.
$$ 0 \to \br{\dfrac{S}{I}}\br{-d} \xrightarrow{f} \dfrac{S}{I} \to \dfrac{S}{I + I_H} \to 0. $$
Taking Hilbert polynomials we get
$$ \phi_{\tfrac{S}{I_1 + I_2}}\br{z} = \phi_{\tfrac{S}{I_Y}}\br{z} + \phi_{\tfrac{S}{I_Y}}\br{z - d}. $$
Let $ \deg\br{Y} = e $, then the right hand side is
$$ \dfrac{e}{r!}z^r + \dots - \br{\dfrac{e}{r!}\br{z - d}^r + \dots} = \dfrac{de}{\br{r - 1}!}z^{r - 1} + \dots. $$
Now on the left hand side, by the structure theorem, there is a filtration
$$ 0 = M^0 \subseteq \dots \subseteq M^s = M, $$
where $ M = S / \br{I_Y + I_H} $. Then
$$ P_M = \sum_{i = 1}^s P_i = \sum_{i = 1}^s P_{\tfrac{M^i}{M^{i - 1}}}, $$
where each $ M^i / M^{i - 1} = \br{S / \pp_i}\br{l_i} $. Since we want to compare the leading coefficient from this with the one from the right hand side, we only care about the $ P_i $'s with degree $ r - 1 $. So the $ \pp_j = I\br{Z_j} $ and the leading term is
$$ \dfrac{\sum_{j = 1}^k i\br{Y, H; Z_j}\deg\br{Z_j}}{\br{r - 1}!} + \dots. $$
\end{proof}

\pagebreak

\section{The \texorpdfstring{$ 27 $}{27} lines on a cubic surface}

\lecture{19}{Friday}{22/02/19}

\begin{theorem}
Let $ S \subseteq \P^3 $ be a nonsingular cubic surface given by a polynomial $ f\br{x, y, z, t} $. Then $ S $ has exactly $ 27 $ lines.
\end{theorem}

We start with a lemma.

\begin{lemma}
\hfill
\begin{enumerate}
\item Given a point $ p \in S $ then there are at most three lines through $ p $. If there are two or three they must be spheres.
\item Every plane $ \pi $ intersect $ S $ in
\begin{itemize}
\item an irreducible cubic,
\item a conic and a line, or
\item three distinct lines.
\end{itemize}
\end{enumerate}
\end{lemma}

\begin{proof}
\hfill
\begin{enumerate}
\item $ l \subseteq S $ gives $ T_pl = l \subseteq T_pS $, by $ 2 $, $ T_pS $ intersect $ S $ in at most three lines.
\item We have to prove that there are no multiple lines in the intersection $ S \cap \pi $. Changing coordinates if necessary, we can suppose $ \pi = \cbr{f = 0} $ and $ l = \cbr{z = 0} $ is the line in the intersection.
$$ f = z^2 \cdot a\br{x, y, z, t} + t \cdot b\br{x, y, z, t}. $$
Claim that $ S $ is singular at $ z = t = b = 0 $.
$$ Jac\br{f} = \onebytwo{z^2a_x + tb_x & z^2a_y + tb_y}{2za + z^2a_z + tb_z & z^2a_t + b + tb_t}. $$
Since $ S $ is smooth there are no multiple lines.
\end{enumerate}
\end{proof}

\begin{lemma}
$ S $ has a line.
\end{lemma}

\begin{proof}
\hfill
\begin{itemize}
\item Let $ P \in S $ and consider $ T_PS $. Then $ T_PS $ intersects $ S $ in a plane cubic $ C = S \cap T_PS $ which is singular at $ P $. Otherwise we are done. Then $ C $ has to be a nodal or a cuspidal curve. So assume that $ C $ is a cuspidal curve, and change coordinates if necessary, assume that $ P = \sbr{0 : 0 : 1 : 0} $ and $ T_PS = \cbr{t = 0} $. So the equation of $ f $ has the shape
$$ f = x^2z - y^3 + gt, $$
for some $ g $ of homogeneous degree two.
\item We consider the point $ P_\alpha = \sbr{1 : \alpha : \alpha^3 : 0} \in C \subset S $, consider the plane $ x = 0 $ and the line $ P_\alpha Q $ in $ \P^3 $ passing through $ P_\alpha $ and intersecting this plane $ x = 0 $ at $ Q = \br{0, y, z, t} $. The line through $ P_\alpha $ and $ Q $ is $ \lambda P_\alpha + \mu Q $ and it lies inside $ S $ if
$$ f\br{\lambda P_\alpha + \mu Q} = 0. $$
After expanding this we have
$$ P_\alpha Q \subset S \qquad \iff \qquad A\br{y, z, t} = B\br{y, z, t} = C\br{y, z, t} = 0, $$
for $ A, B, C $ to be determined. There is a polynomial $ R\br{\alpha} $ of degree $ 27 $ such that $ R\br{\alpha} = 0 $ if and only if $ A = B = C $ have a common zero.

\pagebreak

\lecture{20}{Monday}{25/02/19}

\item Let $ f\br{x, y, z, t} $ be a polynomial, then the \textbf{polar form} of $ f $ is
$$ f_1\br{x, y, z, t, x', y', z', t'} = \dpd{f}{x} \cdot x' + \dpd{f}{y} \cdot y' + \dpd{f}{z} \cdot z' + \dpd{f}{t} \cdot t', $$
where $ P = \br{x, y, z, t} $ and $ Q = \br{x', y', z', t'} $. Then
$$ f\br{\lambda P + \mu Q} = \lambda^3f\br{P} + \lambda^2\mu f_1\br{P, Q} + \lambda\mu^2f_1\br{Q, P} + \mu^3f\br{Q}. $$
The polar form of $ f = x^2z - y^3 + gt $ is
$$ f_1 = 2xzx' - 3y^2y' + x^2z' + g\br{x, y, z, t}t' + tg_1, $$
where $ g_1 $ is the polar form of $ g $. Recall $ P_\alpha = \br{1, \alpha, \alpha^2, 0} $ and $ Q = \br{0, y, z, t} $, so
$$ \cbr{f\br{\lambda P + \mu Q} = 0} = PQ \subseteq S \qquad \iff \qquad f\br{P} = f_1\br{P, Q} = f_1\br{Q, P} = f\br{Q} = 0. $$
Thus
$$
\begin{cases}
A = z - 3\alpha^2y + g\br{1, \alpha, \alpha^3, 0}t \\
B = -3\alpha y^2 + g_1\br{1, \alpha, \alpha^3, 0, 0, y, z, t}t \\
C = -y^3 + g\br{0, y, z, t}t
\end{cases}.
$$
\item Note that
$$ g\br{1, \alpha, \alpha^3, 0} = a^6 + \dots. $$
If $ l = 0 $,
$$ z = 3\alpha^2y + g\br{P}t = 3\alpha^2y + \sbr{a^6}t. $$
Applying this to $ B = 0 $ we have
$$ B = -3\alpha y^2 + g_1\br{1, \alpha, \alpha^3, 0, 0, y, 3\alpha^2y - \sbr{a^6}t, t}t = b_0y^2 + b_1yt + b_2t^2, $$
where
$$ b_0 = -3\alpha, \qquad b_1 = 6\alpha^5 + \dots, \qquad b_2 = -2\alpha^9 + \dots. $$
Substituting $ z $ in $ C $ we get
$$ C = c_0y^3 + c_1y^2t + c_2yt^2 + c_3t^3, $$
where
$$ c_0 = -1, \qquad c_1 = 9\alpha^4 + \dots, \qquad c_2 = -6\alpha^8 + \dots, \qquad c_3 = \alpha^{12} + \dots. $$
By Sylvester theorem $ B $ and $ C $ have a common zero if and only if
$$ \det
\begin{pmatrix}
-3\alpha & 6\alpha^5 & -2\alpha^9 & & \\
& -3\alpha & 6\alpha^5 & -2\alpha^9 & \\
& & -3\alpha & 6\alpha^5 & -2\alpha^9 \\
-1 & 9\alpha^4 & -6\alpha^8 & \alpha^{12} & \\
& -1 & 9\alpha^4 & -6\alpha^8 & \alpha^{12}
\end{pmatrix}
= 0. $$
if and only if
$$ \alpha^{27}\det
\begin{pmatrix}
-3 & 6 & -2 & & \\
& -3 & 6 & -2 & \\
& & -3 & 6 & -2 \\
-1 & 9 & -6 & 1 & \\
& -1 & 9 & -6 & 1
\end{pmatrix}
= \alpha^{27} + \dots = 0. $$
This concludes the proof that $ S $ has a line because we know that the matrix has at least one root and for each root we get a value of $ \alpha $ such that the line $ P_\alpha Q \subseteq S $.
\end{itemize}
\end{proof}

\pagebreak

\lecture{21}{Tuesday}{26/02/19}

\begin{proposition}
Let $ l $ be a line in $ S $, then there are five pairs of lines $ \br{l_i, l_i'} $ intersecting $ l $ such that
\begin{itemize}
\item $ l \cup l_i \cup l_i' $ is coplanar, and
\item $ \br{l_i \cup l_i'} \cap \br{l_j \cup l_j'} = \emptyset $.
\end{itemize}
\end{proposition}

\begin{proof}
Given any plane $ \Pi \subseteq \P^3 $, if $ \Pi $ contains a line $ l $ of $ S $ then $ \Pi \cap S $ is $ l $ and a conic. $ l $ is given by $ z = t = 0 $.
$$ f = Ax^2 + Bxy + Cy^2 + Dx + Ey + F, \qquad A, B, C, D, E, F \in K\sbr{z, t}. $$
We want to prove that there are exactly five planes $ \Pi_i $ such that $ \eval{f}_{\Pi_i} $ is a singular conic. The conic given by $ f $ is singular if and only if
$$ \Delta = \det\threebythree{A}{B}{D}{B}{C}{E}{D}{E}{F} = 4ACF + BDE - AE^2 - B^2F - CD^2 = 0. $$
$ \Delta $ is four times the usual determinant if $ char\br{K} \ne 2 $. Notice that $ \Delta $ is a form of degree five in two variables $ z $ and $ t $. We know that $ l, l_i, l_i' $ could be of two types.
\begin{enumerate}
\item $ l : \br{t = 0} $, $ l_1 : \br{x = 0} $, $ l_1' = \br{y = 0} $.
\item $ l : \br{t = 0} $, $ l_1 : \br{x = 0} $, $ l_1' = \br{x = t} $.
\end{enumerate}
Assume we are in case $ 1 $. Suppose $ z = 0 $ is a solution, then we have to prove that $ z^2 $ is not a solution. Then the equation of $ f $ is
$$ f = txy + gz. $$
So $ B = t + az $, where $ a \in K $, then $ \Delta \equiv -t^2F \mod z^2 $. If $ F \ne 0 $ then $ \Delta $ is non-zero, but $ F $ is non-zero because $ F $ is nonsingular, thus there are no multiple roots.
\end{proof}

\begin{corollary}
$ S $ has at least two distinct lines.
\end{corollary}

\begin{proof}
Just take $ l_1 $ and $ l_2 $.
\end{proof}

\begin{lemma}
If $ l_1, \dots, l_4 \in \P^3 $ are disjoint lines then
\begin{itemize}
\item either all four lines lie on a smooth quadric and they have an infinite number of transversals,
\item or the four lines do not lie in any quadric and they have either one or two common transversals.
\end{itemize}
\end{lemma}

\begin{proof}
Any three lines lie in a smooth quadric $ Q $.
\end{proof}

\begin{lemma}
\hfill
\begin{itemize}
\item Any line not the $ 17 $ lines intersect exactly three of the lines $ l_1, \dots, l_5 $.
\item Conversely, given $ ijk \subset \cbr{1, 2, 3, 4, 5} $ there is a line passing through $ l_i, l_j, l_k $.
\end{itemize}
\end{lemma}

\pagebreak

\section{Grassmannians}

\lecture{22}{Friday}{01/03/19}

\begin{definition}
Let $ V $ be a vector space of dimension $ n $, then
$$ G\br{k; n} = \cbr{S \subseteq V \mid S \ \text{subspace of dimension} \ k}. $$
\end{definition}

\begin{remark}
A point in $ G\br{k; n} $ can be expressed as a basis $ \sbr{v_1, \dots, v_k} $ for a $ k $-dimensional space.
\end{remark}

\begin{theorem}
The map
$$ \function[p]{G\br{k; n}}{\P\br{\bigwedge^k\br{V}} \cong \P^{^nC_k - 1}}{\sbr{v_1, \dots, v_k}}{\sbr{v_1 \wedge \dots \wedge v_k}} $$
is an embedding. That is, image of $ p $ is closed.
\end{theorem}

\begin{example}
Claim that a line $ L \subseteq \P^3 $ gives a point in $ G\br{2; 4} \hookrightarrow \P^5 $. $ G\br{2; 4} $ is a quadric in $ \P^5 $ given by $ Z\br{xs - yt + zw} $.
\end{example}

\begin{proof}
Now we will see the coordinates of the map $ p $. Given a vector space $ V $ of dimension $ n $ and a vector subspace $ S \subseteq V $ of dimension $ k $, then let $ v_1, \dots, v_n $ be a basis for $ V $, and $ s_1, \dots, s_k $ be a basis for $ S $, then the basis for $ S $ can be seen as a $ k \times n $ matrix
$$ M_S = \threebythree{s_{11}}{\dots}{s_{1n}}{\vdots}{\ddots}{\vdots}{s_{k1}}{\dots}{s_{kn}}. $$
If we change the basis for $ S $ then the matrix above gets multiplied by an invertible $ k \times k $ matrix. Then this $ k \times k $ matrix acts on the $ k \times k $ minors of $ M_S $. Suppose the first minor $ K_1 $ is non-zero then choose the inverse of that minor as a base change so that $ M_S $ will have the form
$$
\begin{pmatrix}
1 & & & b_{11} & \dots & b_{1n - k} \\
& \ddots & & \vdots & \ddots & \vdots \\
& & 1 & b_{k1} & \dots & b_{kn - k}
\end{pmatrix}.
$$
This gives a correspondence between matrices $ M_S $ with first non-zero minor and $ \A^{k\br{n - k}} $. Therefore, the image of $ p $ has dimension $ k\br{n - k} $.
\end{proof}

Similarly, we can define flag varieties. Given a vector space $ V $ and \textbf{flag}
$$ 0 \subseteq V_1 \subseteq \dots \subseteq V_n \subseteq V $$
of vector subspaces of dimension $ V_i $, the \textbf{flag variety} denoted by $ F\br{V} $ is the set of flags on $ V $.

\pagebreak

\section{Divisors on curves}

\begin{definition}
A \textbf{Weil divisor} is a formal finite sum
$$ D = \sum_i a_iY_i, \qquad a_i \in \Z, $$
of algebraic subvarieties of codimension one.
\end{definition}

\begin{definition}
More generally, an \textbf{algebraic cycle} is a formal sum of codimension $ p $ subvarieties
$$ C = \sum_i a_iY_i \subseteq X, \qquad a_i \in \Z. $$
\end{definition}

By integrating algebraic cycles, we get a map from the space of $ p $-cycles into the cohomology of the variety. \textbf{Hodge conjecture} states that this defines a bijection.

\begin{example}
In case $ \dim\br{X} = 1 $, then a divisor is just a sum of points with multiplicity. If $ K = \C $ and
$$ f = \dfrac{\br{z - 1}\br{z - 2}}{\br{z - 3}\br{z - 4}}, $$
then $ \br{f} = \overline{1} + \overline{2} - \overline{3} - \overline{4} $.
\end{example}

\begin{definition}
Let $ K = \C $ and $ \dim\br{X} = 1 $. Let $ D, V \subseteq X $. Then \textbf{linear equivalence} is $ D \sim V $ if and only if $ D - V = \br{f} $ for $ f \in K\br{x} $.
\end{definition}

\begin{definition}
The \textbf{class group} is divisors modulo $ \sim $.
\end{definition}

\begin{definition}
Let $ X \subseteq Y $ be a subvariety, then
$$ \OO_{X, Y} = \cbr{\br{U, f} \mid f \ \text{regular at} \ U, \ U \cap Y \ne \emptyset}. $$
Let $ f $ be a rational function, then
$$ \br{f} = \sum_Y v_Y\br{f}Y, $$
where $ v_Y $ is the valuation associated to $ \OO_{X, Y} $.
\end{definition}

\lecture{23}{Monday}{04/03/19}

\begin{definition}
Let $ X $ be a smooth projective curve.
\begin{itemize}
\item A \textbf{divisor} $ D $ is a formal sum $ K_1p_1 + \dots + K_np_n $ of points, where $ K_i \in \Z $,
\item We say a divisor $ D $ is \textbf{effective} if $ K_i \ge 0 $.
\item Given two divisors $ D, E $, $ D \ge E $ if and only if $ D - E \ge 0 $.
\item The \textbf{degree} of $ D $, denoted $ \deg\br{D} $, is the sum $ \sum_{i = 1}^n K_i $.
\end{itemize}
\end{definition}

\begin{remark}
Degree gives a map $ \deg : Div \to \Z $. The set of all divisors on $ X $ has a natural group structure given by addition, we denote this group by $ Div\br{X} $.
\end{remark}

\begin{notation}
The subgroup of \textbf{degree zero divisors} is denoted by $ Div^0\br{X} $.
\end{notation}

\pagebreak

\begin{definition}
\hfill
\begin{itemize}
\item For a non-zero homogeneous polynomial $ f \in S\br{X} $ the \textbf{divisor} of $ f $ is
$$ \br{f} = div\br{f} = \sum_{a \in V_X\br{f}} mult_a\br{f} \cdot a \in Div\br{f}. $$
By B\'ezout's theorem, $ \deg\br{div\br{f}} = \deg\br{X}\deg\br{f} $.
\item If $ Y \subseteq \P^2 $ not containing $ X $, then the \textbf{intersection} of $ X $ and $ Y $ is
$$ X \cdot Y = \sum_{a \in X \cap Y} mult_a\br{X, Y} \cdot a. $$
\end{itemize}
\end{definition}

\begin{example}
Let $ X = Z\br{xz - y^2} $ and $ Y = Z\br{z} $ then $ X \cap Y = \cbr{\sbr{1 : 0 : 0}} $, so
$$ X \cdot Y = 2 \cdot \sbr{1 : 0 : 0}. $$
\end{example}

\begin{lemma}
$ mult_a\br{fg} = mult_a\br{f} + mult_a\br{g} $ gives $ div\br{fg} = div\br{f} + div\br{g} $.
\end{lemma}

\begin{proof}
Recall that $ mult_a\br{f} = len\br{\OO_a / \abr{f}} = \dim_K\br{\OO_a / \abr{f}} $. Thus there is a short exact sequence
$$ 0 \to \dfrac{\OO_a}{\abr{f}} \xrightarrow{g} \dfrac{\OO_a}{\abr{fg}} \to \dfrac{\OO_a}{\abr{g}} \to 0. $$
\end{proof}

\begin{definition}
Let $ f \in K^*\br{X} $, then if $ f = g / h $ we define $ mult_a\br{f} = mult_a\br{g} - mult_a\br{h} $.
\end{definition}

If we take a different representation of $ f $, say $ f = g' / h' $, then $ g / h = g' / h' $ gives $ gh' = hg' $, so
$$
\begin{cases}
mult_a\br{gh'} = mult_a\br{g} + mult_a\br{h'} \\
mult_a\br{hg'} = mult_a\br{h} + mult_a\br{g'}
\end{cases}
\qquad \implies \qquad mult_a\br{g} - mult_a\br{h} = mult_a\br{g'} - mult_a\br{h'}. $$
Analogously, we have
$$ div\br{f} = \sum_{a \in Z\br{g} \cup Z\br{h}} mult_a\br{f}a = div\br{g} \cdot div\br{h}. $$

\begin{example}
Let $ f = xy / \br{x - y}^2 $ on $ \P^1 $. Then
$$ div\br{f} = \sbr{1 : 0} + \sbr{0 : 1} - 2\sbr{1 : 1}. $$
\end{example}

\begin{remark}
Note that $ \deg\br{div\br{f}} $ is always zero because
$$ \deg\br{div\br{f}} = \deg\br{div\br{g}} - \deg\br{div\br{h}} = \br{\deg\br{X}}\br{\deg\br{g}} - \br{\deg\br{X}}\br{\deg\br{h}} = 0. $$
\end{remark}

\begin{definition}
A divisor on $ X $ is called principal if it is of the form $ div\br{f} $ for some $ f \in K^*\br{X} $. We denote the subgroup of all principal divisors by $ Prin\br{X} $.
\end{definition}

\begin{definition}
The quotient
$$ Pic\br{X} = \dfrac{Div\br{X}}{Prin\br{X}} $$
is called the \textbf{Picard group} of $ X $. Restricting to degree zero divisors, we get
$$ Pic^0\br{X} = \dfrac{Div^0\br{X}}{Prin\br{X}}, $$
where $ Div^0\br{X} $ are the divisors of degree zero.
\end{definition}

By the degree map $ \deg : Div\br{X} \to \Z $ we have
$$ \dfrac{Pic\br{X}}{Pic^0\br{X}} \cong \dfrac{Div\br{X}}{Div^0\br{X}} \cong \Z. $$

\pagebreak

\begin{example}
Every degree zero divisor is principal. Suppose
$$ D = K_1\sbr{a_{1, 0} : a_{1, 1}} + \dots + K_n\sbr{a_{n, 0} : a_{n, 1}}, \qquad \sum_{i = 1}^n K_i = 0, $$
then set
$$ f\sbr{x_0 : x_1} = \prod_{i = 1}^n \br{a_{i, 1}x_0 - a_{i, 0}x_1}^{K_i}. $$
So $ Pic^0\br{\P^1} = \cbr{0} $ so $ Pic\br{\P^1} = \Z $.
\end{example}

\lecture{24}{Tuesday}{05/03/19}

\begin{lemma}[Nakayama lemma]
If $ R $ is local with maximal ideal $ \mm $ and $ M $ is finitely generated then $ M = \mm M $ gives $ M = 0 $.
\end{lemma}

\begin{corollary}
If $ R $ is local with maximal ideal $ \mm $ then $ \abr{t_1, \dots, t_n} = \mm $ if and only if $ \abr{\overline{t_1}, \dots, \overline{t_n}} = \mm / \mm^2 $.
\end{corollary}

\begin{proof}
Let $ N = \abr{t_1, \dots, t_n} \subseteq \mm $. Suppose $ \abr{\overline{t_1}, \dots, \overline{t_m}} = \mm / \mm^2 $. Then
$$ N + \mm^2 = \mm + \mm^2 \qquad \implies \qquad \dfrac{N + \mm^2}{N} = \dfrac{\mm + \mm^2}{N} \qquad \implies \qquad \mm\br{\dfrac{\mm}{N}} = \dfrac{\mm}{N} \qquad \implies \qquad \dfrac{\mm}{N} = 0, $$
so $ \abr{t_1, \dots, t_n} = \mm $.
\end{proof}

\begin{lemma}
Let $ X \subseteq \P^2 $ be a smooth curve, and $ I_a \subseteq \OO_a $ be the maximal ideal of the local ring $ \OO_a $.
\begin{enumerate}
\item $ I_a $ is principal, so $ I_a = \abr{\phi_a} $ with $ mult_a\br{\phi_a} = 1 $.
\item Any non-zero $ \phi \in \OO_a $ can be written as $ c\phi_a^m $, where $ m = mult_a\br{\phi} $.
\end{enumerate}
\end{lemma}

\begin{proof}
\hfill
\begin{enumerate}
\item Since $ \OO_a $ is regular, $ \dim\br{\mm / \mm^2} = 1 $, in particular $ \mm \ne \mm^2 $ and we can find $ \phi_a \in \mm \setminus \mm^2 $, so $ \abr{\phi_a} = \mm $. Thus any ideal has to be of the form $ \abr{\phi_a^k} $ since $ \abr{\phi_a} $ is maximal.
\item Take $ \phi \in \OO_a $ non-zero then $ \abr{\phi} = \abr{\phi_a^m} $ gives $ \phi = c\phi_a^m $, so
$$ mult\br{\phi} = mult\br{c\phi_a^m} = mult\br{c} + mult\br{\phi_a^m} = m \cdot mult\br{\phi_a} = m. $$
\end{enumerate}
\end{proof}

\begin{lemma}
Let $ X \subseteq \P^2 $ be a smooth curve, and $ a \in X $.
\begin{enumerate}
\item If $ f, g \in S $, of same degree with $ mult\br{X, f} \ge m $ and $ mult\br{X, g} \ge m $ then
\begin{itemize}
\item $ mult_a\br{X, \lambda f + \mu g} \ge m $, and
\item there exist $ \lambda, \mu $ such that $ mult_a\br{X, \lambda f + \mu g} \ge m + 1 $.
\end{itemize}
\item Let $ Y \subseteq \P^2 $ be another curve and $ m = mult_a\br{X, Y} $. If $ f \in S $ with $ mult_a\br{X, f} \ge m $ then $ mult_a\br{Y, f} \ge m $.
\end{enumerate}
\end{lemma}

\begin{proof}
\hfill
\begin{enumerate}
\item Write $ f = u\phi_a^m $ and $ g = v\phi_a^m $, so for any $ \lambda, \mu $ we have $ \lambda f + \mu g = \br{\lambda u + \mu v}\phi_a^m $ so $ mult_a\br{\lambda f + \mu g} \ge m $, and we can find $ \lambda', \mu' $ such that $ \lambda'u + \mu'v = 0 $ at $ a $, so that $ mult_a\br{\lambda f + \mu g} \ge m + 1 $.
\item Let $ I\br{X} = \abr{g} $, $ I\br{Y} = \abr{h} $, and $ k = mult_a\br{X, f} \ge m = mult_a\br{X, h} $. $ f = u\phi_a^k $ and $ h = v\phi_a^m $, so $ \abr{f} \subset \abr{h} $. $ \abr{f, g} \subset \abr{g, h} $, and we also have $ \abr{f, h} \subset \abr{g, h} $, so $ mult_a\br{f, h} \ge mult_a\br{g, h} $ gives $ mult_a\br{Y, f} \ge mult_a\br{X, Y} = m $.
\end{enumerate}
\end{proof}

\pagebreak

\begin{lemma}
Let $ X \subset \P^2 $ be smooth and $ g, h \in S\br{X} $.
\begin{enumerate}
\item If $ div\br{g} = div\br{h} $ then $ g, h $ are linearly dependent on $ S\br{X} $.
\item If $ h $ is linear and $ div\br{g} \ge div\br{h} $ then $ h \mid g $ in $ S\br{X} $.
\end{enumerate}
\end{lemma}

\begin{proof}
\hfill
\begin{enumerate}
\item By B\'ezout's theorem, $ \deg\br{X} \cdot \deg\br{g} = \deg\br{X} \cdot \deg\br{h} $, so $ \deg\br{g} = \deg\br{h} $. We know by the previous lemma that $ mult_a\br{\lambda g + \mu h} \ge m_a $, and we can find $ b \in X $ such that $ mult_b\br{\lambda g + \mu h} \ge m_b + 1 $. Summing up, we have
$$ \sum_{a \in X} mult_a\br{\lambda g + \mu h} \ge d\deg\br{X} + 1, $$
but $ \lambda g + \mu h $ has degree $ d $, so this is a contradiction unless $ \lambda g + \mu h = 0 $, that is $ g, h $ are linearly dependent.
\item Exercise.
\end{enumerate}
\end{proof}

\begin{proposition}
Let $ X \subseteq \P^2 $ be a smooth cubic. Then for all distinct $ a, b \in X $ we have $ a - b \ne 0 $, so there is no rational function $ \phi $ such that $ div\br{\phi} = a - b $.
\end{proposition}

\begin{proof}
Assume that the result is false. Then there are $ f, g \in S\br{X} $ of degree $ d $ such that
\begin{itemize}
\item there are points $ a_1, \dots, a_{3d - 1} $ and $ a \ne b $ on $ X $ such that
$$ div\br{g} = a_1 + \dots + a_{3d - 1} + a, \qquad div\br{f} = a_1 + \dots + a_{3d - 1} + b, $$
\item among $ a_1, \dots, a_{3d - 1} $, there are at least $ 2d $ distinct points, since we can multiply $ f $ and $ g $ by a linear polynomial with distinct roots so the degree increases by one but the number of distinct points increases by three.
\end{itemize}
Pick a minimal $ d $. If $ d = 1 $ then
$$ div\br{g} = a_1 + a_2 + a, \qquad div\br{f} = a_1 + a_2 + b, $$
so $ a = b = \Psi\br{a_1, a_2} $, a contradiction. So $ d > 1 $. Consider $ \br{\lambda f + \mu g} $, so
$$ div\br{\lambda f + \mu g} \ge a_1 + \dots + a_{3d - 1}. $$
We can choose $ \lambda, \mu $ such that
$$ div\br{\lambda f + \mu g} \ge a_1 + \dots + a_{3d - 1} + c, $$
for any given $ c $. By B\'ezout's theorem,
$$ div\br{\lambda f + \mu g} = a_1 + \dots + a_{3d - 1} + c. $$
So we can choose $ a $ and $ b $, so that $ a = \Psi\br{a_1, a_2} $ and $ b = \Psi\br{a_1, a_3} $.
$$ div\br{f} = \br{a_1 + a_2 + \Psi\br{a_1, a_2}} + a_3 + \dots + a_{3d - 1}, \qquad div\br{g} = \br{a_1 + a_3 + \Psi\br{a_1, a_3}} + a_2 + \dots + a_{3d - 1}. $$
Set $ l, l' $ linear polynomials such that
$$ div\br{l} = a_1 + a_2 + \Psi\br{a_1, a_2}, \qquad div\br{l} = a_1 + a_3 + \Psi\br{a_1, a_3}. $$
The quotient by $ div\br{l} $ and $ div\br{l'} $ gives a polynomial whose divisor is
$$ a_4 + \dots + a_{3d - 1} + a_3, \qquad a_4 + \dots + a_{3d - 1} + a_2, $$
but since we chose $ d $ minimum this gives a contradiction.
\end{proof}

\pagebreak

\section{Elliptic curves}

\lecture{25}{Friday}{08/03/19}

\begin{definition}
An \textbf{abelian variety} $ A $ is a smooth connected projective variety which has a group structure such that addition and taking inverse are regular functions.
\end{definition}

\begin{definition}
An \textbf{elliptic curve} is a one-dimensional abelian variety.
\end{definition}

\begin{proposition}
Let $ X $ be an elliptic curve in $ \P^2 $, and fix $ a_0 \in X $, then there is a bijection
$$ \function[\Phi]{X}{Pic^0\br{X} = \dfrac{Div^0\br{X}}{Prin\br{X}}}{a}{a - a_0}. $$
\end{proposition}

\begin{proof}
\hfill
\begin{itemize}
\item $ \Phi $ is injective by the last proposition.
\item $ \Phi $ is surjective. Suppose
$$ D = a_1 + \dots + a_m - b_1 - \dots - b_m. $$
Consider the function $ l $ with $ div\br{l} = a_1 + a_2 + \Psi\br{a_1, a_2} $, so
$$ D = D + div\br{l} = div\br{l} - \Psi\br{a_1, a_2} + a_3 + \dots. $$
So we can assume $ D = a_1 - b_1 = \Phi\br{\cdot} $.
$$ a_0 + a_1 + \Psi\br{a_0, a_1} - b_1 - \Psi\br{a_0, a_1} - \Psi\br{b_1, \Psi\br{a_0, a_1}} = 0, $$
so
$$ D = a_1 - b_1 = \Psi\br{b_1, \Psi\br{a_0, a_1}} - a_0 = \Phi\br{\Psi\br{b_1, \Psi\br{a_0, a_1}}}. $$
Thus $ \Phi $ is surjective.
\end{itemize}
\end{proof}

We know that $ X \cong Pic^0\br{X} $, which is a group. But what is the expression for $ g_1 + g_2 $ for $ g_1, g_2 \in X $? $ \Phi\br{g_1 + g_2} = \Phi\br{g_1} + \Phi\br{g_2} $, so
\begin{align*}
g_1 + g_2
& = \Phi^{-1}\br{\Phi\br{g_1} + \Phi\br{g_2}} = \Phi^{-1}\br{g_1 - g_0 + g_2 - g_0} = \Phi^{-1}\br{\Psi\br{g_0, \Psi\br{g_1, g_2}} - g_0} \\
& = \Phi^{-1}\br{\Phi\br{\Psi\br{g_0, \Psi\br{g_1, g_2}}}} = \br{\Phi^{-1} \circ \Phi}\br{\Psi\br{g_0, \Psi\br{g_1, g_2}}} = \Psi\br{g_0, \Psi\br{g_1, g_2}}.
\end{align*}

Let $ X $ be an elliptic curve and $ a_0 \in X $, can consider the group law based on $ a_0 $. Then for $ n \in \Z $, we can define $ n \times a = a + \dots + a $.

\begin{note*}
Given $ a, b \in X $, the problem of finding whether or not there exists $ n $ such that $ a = n \times b $ is extremely hard.
\end{note*}

\lecture{26}{Monday}{11/03/19}

Lecture 26 is a problem class.

\lecture{27}{Tuesday}{12/03/19}

Lecture 27 is a problem class.

\lecture{28}{Friday}{15/03/19}

Lecture 28 is a problem class.

\lecture{29}{Monday}{18/03/19}

Lecture 29 is a problem class.

\lecture{30}{Tuesday}{19/03/19}

What's next in Lecture 30?

\end{document}