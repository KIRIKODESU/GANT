\def\module{M4P57 Complex Manifolds}
\def\lecturer{Prof Paolo Cascini}
\def\term{Spring 2020}
\def\cover{}
\def\syllabus{}
\def\thm{section}

\documentclass{article}

% Packages

\usepackage{amssymb}
\usepackage{amsthm}
\usepackage[UKenglish]{babel}
\usepackage{commath}
\usepackage{enumitem}
\usepackage{etoolbox}
\usepackage{fancyhdr}
\usepackage[margin=1in]{geometry}
\usepackage{graphicx}
\usepackage[hidelinks]{hyperref}
\usepackage[utf8]{inputenc}
\usepackage{listings}
\usepackage{mathtools}
\usepackage{stmaryrd}
\usepackage{tikz-cd}
\usepackage{csquotes}

% Formatting

\addto\captionsUKenglish{\renewcommand{\abstractname}{Syllabus}}
\delimitershortfall5pt
\ifx\thm\undefined\newtheorem{n}{}\else\newtheorem{n}{}[\thm]\fi
\newcommand\newoperator[1]{\ifcsdef{#1}{\cslet{#1}{\relax}}{}\csdef{#1}{\operatorname{#1}}}
\setlength{\parindent}{0cm}

% Environments

\theoremstyle{plain}
\newtheorem{algorithm}[n]{Algorithm}
\newtheorem*{algorithm*}{Algorithm}
\newtheorem{algorithm**}{Algorithm}
\newtheorem{conjecture}[n]{Conjecture}
\newtheorem*{conjecture*}{Conjecture}
\newtheorem{conjecture**}{Conjecture}
\newtheorem{corollary}[n]{Corollary}
\newtheorem*{corollary*}{Corollary}
\newtheorem{corollary**}{Corollary}
\newtheorem{lemma}[n]{Lemma}
\newtheorem*{lemma*}{Lemma}
\newtheorem{lemma**}{Lemma}
\newtheorem{proposition}[n]{Proposition}
\newtheorem*{proposition*}{Proposition}
\newtheorem{proposition**}{Proposition}
\newtheorem{theorem}[n]{Theorem}
\newtheorem*{theorem*}{Theorem}
\newtheorem{theorem**}{Theorem}

\theoremstyle{definition}
\newtheorem{aim}[n]{Aim}
\newtheorem*{aim*}{Aim}
\newtheorem{aim**}{Aim}
\newtheorem{axiom}[n]{Axiom}
\newtheorem*{axiom*}{Axiom}
\newtheorem{axiom**}{Axiom}
\newtheorem{condition}[n]{Condition}
\newtheorem*{condition*}{Condition}
\newtheorem{condition**}{Condition}
\newtheorem{definition}[n]{Definition}
\newtheorem*{definition*}{Definition}
\newtheorem{definition**}{Definition}
\newtheorem{example}[n]{Example}
\newtheorem*{example*}{Example}
\newtheorem{example**}{Example}
\newtheorem{exercise}[n]{Exercise}
\newtheorem*{exercise*}{Exercise}
\newtheorem{exercise**}{Exercise}
\newtheorem{fact}[n]{Fact}
\newtheorem*{fact*}{Fact}
\newtheorem{fact**}{Fact}
\newtheorem{goal}[n]{Goal}
\newtheorem*{goal*}{Goal}
\newtheorem{goal**}{Goal}
\newtheorem{law}[n]{Law}
\newtheorem*{law*}{Law}
\newtheorem{law**}{Law}
\newtheorem{plan}[n]{Plan}
\newtheorem*{plan*}{Plan}
\newtheorem{plan**}{Plan}
\newtheorem{problem}[n]{Problem}
\newtheorem*{problem*}{Problem}
\newtheorem{problem**}{Problem}
\newtheorem{question}[n]{Question}
\newtheorem*{question*}{Question}
\newtheorem{question**}{Question}
\newtheorem{warning}[n]{Warning}
\newtheorem*{warning*}{Warning}
\newtheorem{warning**}{Warning}
\newtheorem{acknowledgements}[n]{Acknowledgements}
\newtheorem*{acknowledgements*}{Acknowledgements}
\newtheorem{acknowledgements**}{Acknowledgements}
\newtheorem{annotations}[n]{Annotations}
\newtheorem*{annotations*}{Annotations}
\newtheorem{annotations**}{Annotations}
\newtheorem{assumption}[n]{Assumption}
\newtheorem*{assumption*}{Assumption}
\newtheorem{assumption**}{Assumption}
\newtheorem{conclusion}[n]{Conclusion}
\newtheorem*{conclusion*}{Conclusion}
\newtheorem{conclusion**}{Conclusion}
\newtheorem{claim}[n]{Claim}
\newtheorem*{claim*}{Claim}
\newtheorem{claim**}{Claim}
\newtheorem{notation}[n]{Notation}
\newtheorem*{notation*}{Notation}
\newtheorem{notation**}{Notation}
\newtheorem{note}[n]{Note}
\newtheorem*{note*}{Note}
\newtheorem{note**}{Note}
\newtheorem{remark}[n]{Remark}
\newtheorem*{remark*}{Remark}
\newtheorem{remark**}{Remark}

% Lectures

\newcommand{\lecture}[3]{ % Lecture
  \marginpar{
    Lecture #1 \\
    #2 \\
    #3
  }
}

% Blackboard

\renewcommand{\AA}{\mathbb{A}} % Blackboard A
\newcommand{\BB}{\mathbb{B}}   % Blackboard B
\newcommand{\CC}{\mathbb{C}}   % Blackboard C
\newcommand{\DD}{\mathbb{D}}   % Blackboard D
\newcommand{\EE}{\mathbb{E}}   % Blackboard E
\newcommand{\FF}{\mathbb{F}}   % Blackboard F
\newcommand{\GG}{\mathbb{G}}   % Blackboard G
\newcommand{\HH}{\mathbb{H}}   % Blackboard H
\newcommand{\II}{\mathbb{I}}   % Blackboard I
\newcommand{\JJ}{\mathbb{J}}   % Blackboard J
\newcommand{\KK}{\mathbb{K}}   % Blackboard K
\newcommand{\LL}{\mathbb{L}}   % Blackboard L
\newcommand{\MM}{\mathbb{M}}   % Blackboard M
\newcommand{\NN}{\mathbb{N}}   % Blackboard N
\newcommand{\OO}{\mathbb{O}}   % Blackboard O
\newcommand{\PP}{\mathbb{P}}   % Blackboard P
\newcommand{\QQ}{\mathbb{Q}}   % Blackboard Q
\newcommand{\RR}{\mathbb{R}}   % Blackboard R
\renewcommand{\SS}{\mathbb{S}} % Blackboard S
\newcommand{\TT}{\mathbb{T}}   % Blackboard T
\newcommand{\UU}{\mathbb{U}}   % Blackboard U
\newcommand{\VV}{\mathbb{V}}   % Blackboard V
\newcommand{\WW}{\mathbb{W}}   % Blackboard W
\newcommand{\XX}{\mathbb{X}}   % Blackboard X
\newcommand{\YY}{\mathbb{Y}}   % Blackboard Y
\newcommand{\ZZ}{\mathbb{Z}}   % Blackboard Z

% Brackets

\renewcommand{\eval}[1]{\left. #1 \right|}          % Evaluation
\newcommand{\br}{\del}                              % Brackets
\newcommand{\abr}[1]{\left\langle #1 \right\rangle} % Angle brackets
\newcommand{\fbr}[1]{\left\lfloor #1 \right\rfloor} % Floor brackets
\newcommand{\lbr}[1]{\left\lfloor #1 \right\rfloor} % Ceiling brackets
\newcommand{\st}{\ \middle| \ }                     % Such that

% Calligraphic

\newcommand{\AAA}{\mathcal{A}} % Calligraphic A
\newcommand{\BBB}{\mathcal{B}} % Calligraphic B
\newcommand{\CCC}{\mathcal{C}} % Calligraphic C
\newcommand{\DDD}{\mathcal{D}} % Calligraphic D
\newcommand{\EEE}{\mathcal{E}} % Calligraphic E
\newcommand{\FFF}{\mathcal{F}} % Calligraphic F
\newcommand{\GGG}{\mathcal{G}} % Calligraphic G
\newcommand{\HHH}{\mathcal{H}} % Calligraphic H
\newcommand{\III}{\mathcal{I}} % Calligraphic I
\newcommand{\JJJ}{\mathcal{J}} % Calligraphic J
\newcommand{\KKK}{\mathcal{K}} % Calligraphic K
\newcommand{\LLL}{\mathcal{L}} % Calligraphic L
\newcommand{\MMM}{\mathcal{M}} % Calligraphic M
\newcommand{\NNN}{\mathcal{N}} % Calligraphic N
\newcommand{\OOO}{\mathcal{O}} % Calligraphic O
\newcommand{\PPP}{\mathcal{P}} % Calligraphic P
\newcommand{\QQQ}{\mathcal{Q}} % Calligraphic Q
\newcommand{\RRR}{\mathcal{R}} % Calligraphic R
\newcommand{\SSS}{\mathcal{S}} % Calligraphic S
\newcommand{\TTT}{\mathcal{T}} % Calligraphic T
\newcommand{\UUU}{\mathcal{U}} % Calligraphic U
\newcommand{\VVV}{\mathcal{V}} % Calligraphic V
\newcommand{\WWW}{\mathcal{W}} % Calligraphic W
\newcommand{\XXX}{\mathcal{X}} % Calligraphic X
\newcommand{\YYY}{\mathcal{Y}} % Calligraphic Y
\newcommand{\ZZZ}{\mathcal{Z}} % Calligraphic Z

% Fraktur

\newcommand{\aaa}{\mathfrak{a}}   % Fraktur a
\newcommand{\bbb}{\mathfrak{b}}   % Fraktur b
\newcommand{\ccc}{\mathfrak{c}}   % Fraktur c
\newcommand{\ddd}{\mathfrak{d}}   % Fraktur d
\newcommand{\eee}{\mathfrak{e}}   % Fraktur e
\newcommand{\fff}{\mathfrak{f}}   % Fraktur f
\renewcommand{\ggg}{\mathfrak{g}} % Fraktur g
\newcommand{\hhh}{\mathfrak{h}}   % Fraktur h
\newcommand{\iii}{\mathfrak{i}}   % Fraktur i
\newcommand{\jjj}{\mathfrak{j}}   % Fraktur j
\newcommand{\kkk}{\mathfrak{k}}   % Fraktur k
\renewcommand{\lll}{\mathfrak{l}} % Fraktur l
\newcommand{\mmm}{\mathfrak{m}}   % Fraktur m
\newcommand{\nnn}{\mathfrak{n}}   % Fraktur n
\newcommand{\ooo}{\mathfrak{o}}   % Fraktur o
\newcommand{\ppp}{\mathfrak{p}}   % Fraktur p
\newcommand{\qqq}{\mathfrak{q}}   % Fraktur q
\newcommand{\rrr}{\mathfrak{r}}   % Fraktur r
\newcommand{\sss}{\mathfrak{s}}   % Fraktur s
\newcommand{\ttt}{\mathfrak{t}}   % Fraktur t
\newcommand{\uuu}{\mathfrak{u}}   % Fraktur u
\newcommand{\vvv}{\mathfrak{v}}   % Fraktur v
\newcommand{\www}{\mathfrak{w}}   % Fraktur w
\newcommand{\xxx}{\mathfrak{x}}   % Fraktur x
\newcommand{\yyy}{\mathfrak{y}}   % Fraktur y
\newcommand{\zzz}{\mathfrak{z}}   % Fraktur z

% Geometry

\newcommand{\CP}{\mathbb{CP}}                                              % Complex projective space
\newcommand{\iintd}[4]{\iint_{#1} \, #2 \, \dif #3 \, \dif #4}             % Double integral
\newcommand{\RP}{\mathbb{RP}}                                              % Real projective space
\newcommand{\intd}[4]{\int_{#1}^{#2} \, #3 \, \dif #4}                     % Single integral
\newcommand{\iiintd}[5]{\iint_{#1} \, #2 \, \dif #3 \, \dif #4 \, \dif #5} % Triple integral

% Logic

\newcommand{\iffb}[2]{\br{#1 \leftrightarrow #2}} % Biconditional
\newcommand{\andb}[2]{\br{#1 \land #2}}           % Conjunction
\newcommand{\orb}[2]{\br{#1 \lor #2}}             % Disjunction
\newcommand{\nib}[2]{\br{#1 \notin #2}}           % Element of
\newcommand{\eqb}[2]{\br{#1 = #2}}                % Equal to
\newcommand{\teb}[1]{\br{\exists #1}}             % Existential quantifier
\newcommand{\impb}[2]{\br{#1 \rightarrow #2}}     % Implication
\newcommand{\ltb}[2]{\br{#1 < #2}}                % Less than
\newcommand{\leb}[2]{\br{#1 \le #2}}              % Less than or equal to
\newcommand{\notb}[1]{\br{\neg #1}}               % Negation
\newcommand{\inb}[2]{\br{#1 \in #2}}              % Not element of
\newcommand{\neb}[2]{\br{#1 \ne #2}}              % Not equal to
\newcommand{\subb}[2]{\br{#1 \subseteq #2}}       % Subset
\newcommand{\fab}[1]{\br{\forall #1}}             % Universal quantifier

% Maps

\newcommand{\bijection}[7][]{    % Bijection
  \ifx &#1&
    \begin{array}{rcl}
      #2 & \longleftrightarrow & #3 \\
      #4 & \longmapsto         & #5 \\
      #6 & \longmapsfrom       & #7
    \end{array}
  \else
    \begin{array}{ccrcl}
      #1 & : & #2 & \longrightarrow & #3 \\
         &   & #4 & \longmapsto     & #5 \\
         &   & #6 & \longmapsfrom   & #7
    \end{array}
  \fi
}
\newcommand{\birational}[7][]{   % Birational map
  \ifx &#1&
    \begin{array}{rcl}
      #2 & \dashrightarrow & #3 \\
      #4 & \longmapsto     & #5 \\
      #6 & \longmapsfrom   & #7
    \end{array}
  \else
    \begin{array}{ccrcl}
      #1 & : & #2 & \dashrightarrow & #3 \\
         &   & #4 & \longmapsto     & #5 \\
         &   & #6 & \longmapsfrom   & #7
    \end{array}
  \fi
}
\newcommand{\correspondence}[2]{ % Correspondence
  \cbr{
    \begin{array}{c}
      #1
    \end{array}
  }
  \qquad
  \leftrightsquigarrow
  \qquad
  \cbr{
    \begin{array}{c}
      #2
    \end{array}
  }
}
\newcommand{\function}[5][]{     % Function
  \ifx &#1&
    \begin{array}{rcl}
      #2 & \longrightarrow & #3 \\
      #4 & \longmapsto     & #5
    \end{array}
  \else
    \begin{array}{ccrcl}
      #1 & : & #2 & \longrightarrow & #3 \\
         &   & #4 & \longmapsto     & #5
    \end{array}
  \fi
}
\newcommand{\functions}[7][]{    % Functions
  \ifx &#1&
    \begin{array}{rcl}
      #2 & \longrightarrow & #3 \\
      #4 & \longmapsto     & #5 \\
      #6 & \longmapsto     & #7
    \end{array}
  \else
    \begin{array}{ccrcl}
      #1 & : & #2 & \longrightarrow & #3 \\
         &   & #4 & \longmapsto     & #5 \\
         &   & #6 & \longmapsto     & #7
    \end{array}
  \fi
}
\newcommand{\rational}[5][]{     % Rational map
  \ifx &#1&
    \begin{array}{rcl}
      #2 & \dashrightarrow & #3 \\
      #4 & \longmapsto     & #5
    \end{array}
  \else
    \begin{array}{ccrcl}
      #1 & : & #2 & \dashrightarrow & #3 \\
         &   & #4 & \longmapsto     & #5
    \end{array}
  \fi
}

% Matrices

\newcommand{\onebytwo}[2]{      % One by two matrix
  \begin{pmatrix}
    #1 & #2
  \end{pmatrix}
}
\newcommand{\onebythree}[3]{    % One by three matrix
  \begin{pmatrix}
    #1 & #2 & #3
  \end{pmatrix}
}
\newcommand{\twobyone}[2]{      % Two by one matrix
  \begin{pmatrix}
    #1 \\
    #2
  \end{pmatrix}
}
\newcommand{\twobytwo}[4]{      % Two by two matrix
  \begin{pmatrix}
    #1 & #2 \\
    #3 & #4
  \end{pmatrix}
}
\newcommand{\threebyone}[3]{    % Three by one matrix
  \begin{pmatrix}
    #1 \\
    #2 \\
    #3
  \end{pmatrix}
}
\newcommand{\threebythree}[9]{  % Three by three matrix
  \begin{pmatrix}
    #1 & #2 & #3 \\
    #4 & #5 & #6 \\
    #7 & #8 & #9
  \end{pmatrix}
}
\newcommand{\twobytwosmall}[4]{ % Two by two small matrix
  \begin{psmallmatrix}
    #1 & #2 \\
    #3 & #4
  \end{psmallmatrix}
}

% Number theory

\renewcommand{\symbol}[2]{\br{\tfrac{#1}{#2}}} % Power residue symbol
\newcommand{\unit}[1]{\br{\ZZ / #1\ZZ}^\times} % Unit group

% Operators

\newoperator{ab}    % Abelian
\newoperator{AG}    % Affine geometry
\newoperator{alg}   % Algebraic
\newoperator{Ann}   % Annihilator
\newoperator{area}  % Area
\newoperator{Aut}   % Automorphism
\newoperator{card}  % Cardinality
\newoperator{ch}    % Characteristic
\newoperator{Cl}    % Class
\newoperator{Coker} % Cokernel
\newoperator{col}   % Column
\newoperator{Corr}  % Correspondence
\newoperator{diam}  % Diameter
\newoperator{Disc}  % Discriminant
\newoperator{dom}   % Domain
\newoperator{Eig}   % Eigenvalue
\newoperator{Em}    % Embedding
\newoperator{End}   % Endomorphism
\newoperator{fin}   % Finite
\newoperator{Fix}   % Fixed
\newoperator{Frac}  % Fraction
\newoperator{Frob}  % Frobenius
\newoperator{Fun}   % Function
\newoperator{Gal}   % Galois
\newoperator{GL}    % General linear
\newoperator{Ham}   % Hamming
\newoperator{Homeo} % Homeomorphism
\newoperator{Hom}   % Homomorphism
\newoperator{id}    % Identity
\newoperator{Im}    % Image
\newoperator{Ind}   % Index
\newoperator{Ker}   % Kernel
\newoperator{lcm}   % Least common multiple
\newoperator{Mat}   % Matrix
\newoperator{mult}  % Multiplicity
\newoperator{new}   % New
\newoperator{Nm}    % Norm
\newoperator{old}   % Old
\newoperator{op}    % Opposite
\newoperator{ord}   % Order
\newoperator{Pay}   % Payley
\newoperator{PG}    % Projective geometry
\newoperator{PGL}   % Projective general linear
\newoperator{PSL}   % Projective special linear
\newoperator{rad}   % Radical
\newoperator{ran}   % Range
\newoperator{Res}   % Residue
\newoperator{rk}    % Rank
\newoperator{Re}    % Real
\newoperator{row}   % Row
\newoperator{sgn}   % Sign
\newoperator{Sing}  % Singular
\newoperator{SK}    % Skeleton
\newoperator{sp}    % Span
\newoperator{SL}    % Special linear
\newoperator{SO}    % Special orthogonal
\newoperator{Spec}  % Spectrum
\newoperator{Stab}  % Stabiliser
\newoperator{star}  % Star
\newoperator{srg}   % Strongly regular graph
\newoperator{supp}  % Support
\newoperator{Sym}   % Symmetric
\newoperator{tors}  % Torsion
\newoperator{Tr}    % Trace
\newoperator{vol}   % Volume
\newoperator{wt}    % Weight

% Roman

\newcommand{\A}{\mathrm{A}}   % Roman A
\newcommand{\B}{\mathrm{B}}   % Roman B
\newcommand{\C}{\mathrm{C}}   % Roman C
\newcommand{\D}{\mathrm{D}}   % Roman D
\newcommand{\E}{\mathrm{E}}   % Roman E
\newcommand{\F}{\mathrm{F}}   % Roman F
\newcommand{\G}{\mathrm{G}}   % Roman G
\renewcommand{\H}{\mathrm{H}} % Roman H
\newcommand{\I}{\mathrm{I}}   % Roman I
\newcommand{\J}{\mathrm{J}}   % Roman J
\newcommand{\K}{\mathrm{K}}   % Roman K
\renewcommand{\L}{\mathrm{L}} % Roman L
\newcommand{\M}{\mathrm{M}}   % Roman M
\newcommand{\N}{\mathrm{N}}   % Roman N
\renewcommand{\O}{\mathrm{O}} % Roman O
\renewcommand{\P}{\mathrm{P}} % Roman P
\newcommand{\Q}{\mathrm{Q}}   % Roman Q
\newcommand{\R}{\mathrm{R}}   % Roman R
\renewcommand{\S}{\mathrm{S}} % Roman S
\newcommand{\T}{\mathrm{T}}   % Roman T
\newcommand{\U}{\mathrm{U}}   % Roman U
\newcommand{\V}{\mathrm{V}}   % Roman V
\newcommand{\W}{\mathrm{W}}   % Roman W
\newcommand{\X}{\mathrm{X}}   % Roman X
\newcommand{\Y}{\mathrm{Y}}   % Roman Y
\newcommand{\Z}{\mathrm{Z}}   % Roman Z

\renewcommand{\a}{\mathrm{a}} % Roman a
\renewcommand{\b}{\mathrm{b}} % Roman b
\renewcommand{\c}{\mathrm{c}} % Roman c
\renewcommand{\d}{\mathrm{d}} % Roman d
\newcommand{\e}{\mathrm{e}}   % Roman e
\newcommand{\f}{\mathrm{f}}   % Roman f
\newcommand{\g}{\mathrm{g}}   % Roman g
\newcommand{\h}{\mathrm{h}}   % Roman h
\renewcommand{\i}{\mathrm{i}} % Roman i
\renewcommand{\j}{\mathrm{j}} % Roman j
\renewcommand{\k}{\mathrm{k}} % Roman k
\renewcommand{\l}{\mathrm{l}} % Roman l
\newcommand{\m}{\mathrm{m}}   % Roman m
\renewcommand{\n}{\mathrm{n}} % Roman n
\renewcommand{\o}{\mathrm{o}} % Roman o
\newcommand{\p}{\mathrm{p}}   % Roman p
\newcommand{\q}{\mathrm{q}}   % Roman q
\renewcommand{\r}{\mathrm{r}} % Roman r
\newcommand{\s}{\mathrm{s}}   % Roman s
\renewcommand{\t}{\mathrm{t}} % Roman t
\renewcommand{\u}{\mathrm{u}} % Roman u
\renewcommand{\v}{\mathrm{v}} % Roman v
\newcommand{\w}{\mathrm{w}}   % Roman w
\newcommand{\x}{\mathrm{x}}   % Roman x
\newcommand{\y}{\mathrm{y}}   % Roman y
\newcommand{\z}{\mathrm{z}}   % Roman z

% Tikz

\tikzset{
  arrow symbol/.style={"#1" description, allow upside down, auto=false, draw=none, sloped},
  subset/.style={arrow symbol={\subset}},
  cong/.style={arrow symbol={\cong}}
}

% Fancy header

\pagestyle{fancy}
\lhead{\module}
\rhead{\nouppercase{\leftmark}}

% Make title

\title{\module}
\author{Lectured by \lecturer \\ Typed by David Kurniadi Angdinata}
\date{\term}

\begin{document}

% Title page
\maketitle
\cover
\vfill
\begin{abstract}
\noindent\syllabus
\end{abstract}

\pagebreak

% Contents page
\tableofcontents

\pagebreak

% Document page
\setcounter{section}{-1}

\setcounter{section}{0}

\section{Introduction}

\lecture{1}{Thursday}{09/01/20}

The following are references.
\begin{itemize}
\item O Biquard and A H\"oring, K\"ahler geometry and Hodge theory, 2008.
\item J P Demailly, Complex analytic and differential geometry, 2012.
\item C Voisin, Hodge theory and complex algebraic geometry, 2002.
\item R O Wells, Differential analysis on complex manifolds, 1973.
\item A Gathmann, Algebraic geometry, 2002
\item P Griffiths and J Harris, Principles of algebraic geometry, 1978.
\end{itemize}

Complex manifolds are manifolds over $ \CC^n $.

\begin{example}
$ \CC^1 $ is a complex manifold. Any open $ U \subset \CC^n $ is a complex manifold.
\end{example}

\begin{example}
The sphere $ \S^2 \subset \RR^3 $ is a complex manifold by
$$ \S^2 \cong \CC \cup \cbr{\infty} = \PP_\CC^1 = \CP^1. $$
More in general $ \PP_\CC^n $ is a complex manifold for all $ n $.
\end{example}

\begin{example}
The torus
$$ \S^1 \times \S^1 = \RR^2 / \ZZ^2 = \CC / \ZZ^2 $$
is a complex manifold. More in general a $ 2n $-dimensional torus $ \CC^n / \Lambda $ for a lattice $ \Lambda \cong \ZZ^{2n} $ is a complex manifold.
\end{example}

\begin{example}
Compact Riemannian surfaces of genus $ g > 1 $, called \textbf{hyperbolics}, are all complex manifolds.
\end{example}

\begin{example}
\label{eg:1.5}
Let $ f : \CC \to \CC $ be holomorphic. The graph of $ f $,
$$ \Gamma_f = \cbr{\br{x, f\br{x}} \st x \in \CC} \subset \CC \times \CC, $$
is a complex manifold. From $ \Gamma_f $ we can recover $ f $, by
$$ f\br{x} = q\br{p^{-1}\br{x} \cap \Gamma_f}, $$
where $ p, q : \CC^2 \to \CC $ are the projections to the first and second factors. This allows us to define $ f^{-1} $. Assume $ f $ is bijective. Define
$$ \function[\tau]{\CC^2}{\CC^2}{\br{x, y}}{\br{y, x}}. $$
Define
$$ \Gamma_{f^{-1}} = \tau\br{\Gamma_f}. $$
Then $ f^{-1} $ is the function induced by $ \Gamma_{f^{-1}} $. This makes sense even if $ f $ is not bijective. Then we get a multivalued function, such as $ \log z $ as the inverse of $ \exp z $.
\end{example}

\begin{example}
Generalising Example \ref{eg:1.5}, we can consider two complex manifolds $ M $ and $ N $ and we can consider holomorphisms $ f : M \to N $. Given $ M $,
$$ \Aut M = \cbr{f : M \to M \ \text{holomorphic bijective and} \ f^{-1} \ \text{holomorphic}}. $$
If $ M = \CC $, there are lots of $ \C^\infty $-functions $ \CC \to \CC $ but the automorphisms of $ \CC $ are just affine linear maps. If $ M = \CC / \ZZ^2 $, then $ \Aut M $ is interesting.
\end{example}

\pagebreak

\begin{example}
Algebraic geometry is the zeroes of polynomials. That is, fix $ m $, and take polynomials $ f_1, \dots, f_k $ in $ m $ variables. Define
$$ M = \cbr{\br{x_1, \dots, x_m} \in \CC^m \st f_1\br{x_1, \dots, x_m} = \dots = f_k\br{x_1, \dots, x_m} = 0}. $$
Then $ M $ is called an \textbf{algebraic variety}. If $ M $ is smooth then $ M $ is a complex manifold. Fix $ m $, take homogeneous polynomials $ F_1, \dots, F_k $ in $ m + 1 $ variables, where $ F $ is \textbf{homogeneous} if it is the sum of monomials of the same degree. Consider
$$ N = \cbr{\br{x_0, \dots, x_m} \in \PP_\CC^m \st F_1\br{x_0, \dots, x_m} = \dots = F_k\br{x_0, \dots, x_m} = 0}. $$
Then $ N $ is called a \textbf{projective variety}. If $ N $ is smooth then $ N $ is a complex manifold.
\end{example}

The idea is if $ M $ is a differentiable manifold, then $ M $ contains lots of submanifolds $ N $. This is not true for complex manifolds. There exist complex manifolds without any proper complex submanifolds, which is not true for projective varieties. The following are questions.
\begin{itemize}
\item What can we say about the topology of complex manifolds? For example, what is $ \pi_1\br{M} $? What is the cohomology of $ M $?
\item Assume that $ M $ and $ N $ are complex manifolds which are diffeomorphic. Are they also isomorphic, so there exists a biholomorphism $ M \to N $?
\end{itemize}
What is next?
\begin{itemize}
\item Hodge decomposition theorem. Understand the cohomology of $ M $ by using the complex structure.
\item Kodaira embedding theorem. Understand when a compact complex manifold is projective.
\end{itemize}

\begin{note*}
If $ M \subset \PP_\CC^m $ is a compact complex manifold then $ M $ is projective.
\end{note*}

\begin{example*}
Let $ M = \Gamma_{\exp} $ for $ \exp : \CC \to \CC $. Then $ M \subset \CC^2 $ but it is not algebraic.
\end{example*}

\pagebreak

\section{Local theory}

\subsection{Holomorphic functions in several variables}

\lecture{2}{Thursday}{09/01/20}

\begin{notation}
Given $ z_0 \in \CC $ and $ r > 0 $, the \textbf{disc} is
$$ \D\br{z_0, r} = \cbr{z \in \CC \st \abs{z - z_0} < r}, $$
and $ \partial\D\br{z_0, r} $ is the boundary of $ \D\br{z_0, r} $.
\end{notation}

\begin{definition}
Let $ U \subset \CC $, and let $ f : U \to \CC $ be a function. Then $ f $ is \textbf{holomorphic at $ z_0 \in U $} if
$$ \lim_{z \to z_0} \dfrac{f\br{z} - f\br{z_0}}{z - z_0} $$
exists.
\end{definition}

\begin{theorem}[Cauchy]
\label{thm:2.3}
Let $ U \subset \CC $ be open, let $ f $ be holomorphic on $ U $, and let $ z_0 \in U $. Assume that if $ D = \D\br{z_0, r} \subset U $ then $ \overline{D} \subset U $. Then
$$ f\br{z_0} = \dfrac{1}{2\pi i}\intd{\partial D}{}{\dfrac{f\br{z}}{z - z_0}}{z}. $$
\end{theorem}

\begin{notation}
Fix $ z_0 = \br{z_{01}, \dots, z_{0n}} \in \CC^n $ and $ R = \br{r_1, \dots, r_n} \in \RR_{> 0}^n $. Then the \textbf{polydisc} is
$$ \D\br{z_0, R} = \cbr{z = \br{z_1, \dots, z_n} \in \CC^n \st \abs{z_i - z_{0i}} < r_i \ \text{for each} \ i}, $$
where $ R $ is the \textbf{polyradius}.
\end{notation}

\begin{definition}
Let $ U \subset \CC^n $ be open, let $ f : U \to \CC $ be a continuous function, and let $ z = \br{z_1, \dots, z_n} \in \CC^n $. Then $ f $ is \textbf{holomorphic} at $ z $, if assuming that $ D = \D\br{z, R} \subset U $ for some $ R = \br{r_1, \dots, r_n} $ then
$$ f\br{z_1, \dots, z_{i - 1}, \cdot, z_{i + 1}, \dots, z_n} : \D\br{z_i, r_i} \to \CC $$
is holomorphic for all $ i $.
\end{definition}

\begin{example}
Any convergent power series in $ n $-variables is holomorphic.
\end{example}

The opposite is also true.

\begin{theorem}[Cauchy]
\label{thm:2.7}
Let $ U \subset \CC^n $ be an open set, let $ f : U \to \CC $ be holomorphic, and let $ z = \br{z_1, \dots, z_n} \in U $. Assume that if $ D = \D\br{z_0, R} $ for some $ R = \br{r_1, \dots, r_n} $ then $ \overline{D} \subset U $. If $ z' = \br{z_1', \dots, z_n'} \in D $ then
$$ f\br{z'} = \dfrac{1}{\br{2\pi i}^n}\intd{\partial\D\br{z_1, r_1}}{}{\dots \intd{\partial\D\br{z_n, r_n}}{}{\dfrac{f\br{z}}{\br{z - z_1'} \dots \br{z - z_n'}}}{z_n \dots}}{z_1}. $$
\end{theorem}

\begin{proof}
Use induction on $ n $ and Cauchy theorem at each step.
\end{proof}

\begin{corollary}
Let $ U \subset \CC^n $ be open, let $ f : U \to \CC $ be holomorphic, and let $ z = \br{z_1, \dots, z_n} \in U $. Then there exists $ D = \D\br{z, R} \subset U $ for some $ R = \br{r_1, \dots, r_n} $ and there exists
$$ p\br{w} = \sum_{m_1, \dots, m_n \ge 0} a_{m_1, \dots, m_n} \br{w_1 - z_1}^{m_1} \dots \br{w_n - z_n}^{m_n}, $$
such that $ p $ is convergent on $ D $ and $ f\br{w} = p\br{w} $ inside $ D $.
\end{corollary}

\begin{proof}
The idea is to use Theorem \ref{thm:2.7} and $ 1 / \br{1 - w} = \sum_{k \ge 0} w^k $.
\end{proof}

\begin{definition}
Let $ U \subset \CC^n $ be open. Then $ f : U \to \CC^m $ is \textbf{holomorphic} if $ f_i = \p_i \circ f $ is holomorphic for any $ i = 1, \dots, m $ where $ \p_i : \CC^m \to \CC $ is the $ i $-th projection, so $ f = \br{f_1, \dots, f_m} $.
\end{definition}

\begin{fact*}
If $ f : U \to \CC^m $ is holomorphic and $ g : V \to \CC^p $ is holomorphic where $ V \supset f\br{U} $ then $ g \circ f $ is holomorphic.
\end{fact*}

\begin{definition}
Let $ U \subset \CC^n $ be open. A holomorphic function $ f : U \to \CC^m $ is \textbf{biholomorphic at $ z_0 \in U $} if there exists an open neighbourhood $ V \subset U $ of $ z_0 $ such that $ f : V \to f\br{V} $ is bijective and $ f^{-1} : f\br{V} \to V $ is holomorphic. Then $ f $ is \textbf{biholomorphic} if $ f $ is bijective and $ f $ is biholomorphic at any point.
\end{definition}

\begin{note*}
$ f\br{V} $ is automatically open in $ \CC^m $ if $ m = n $.
\end{note*}

\pagebreak

\begin{example}
Let $ \Phi : \CC^n \to \CC^n $ be linear such that $ \det \Phi \ne 0 $. Then $ \Phi $ is a biholomorphism.
\end{example}

\begin{example}
Let $ U = \CC \setminus \cbr{0} $ and
$$ \function[f]{U}{U}{z}{z^2}. $$
Check that $ f $ is biholomorphic at any point of $ U $ but $ f $ is not biholomorphic.
\end{example}

\begin{remark*}
$ \CC^n \cong \RR^{2n} $ and $ \CC^m \cong \RR^{2m} $. Then a holomorphic $ f : U \subset \CC^n \to \CC^m $ is also a diffeomorphism $ U \subset \RR^{2n} \to \RR^{2m} $.
\end{remark*}

\begin{theorem}[Hartogs]
Let $ n \ge 2 $, let $ \epsilon = \br{\epsilon_1, \dots, \epsilon_n} $ and $ \delta = \br{\delta_1, \dots, \delta_n} $ such that $ \epsilon_i > \delta_i > 0 $, let $ U = \D\br{0, \epsilon} \setminus \overline{\D\br{0, \delta}} $, and let $ f : U \to \CC^m $ be holomorphic. Then there exists a holomorphic $ \overline{f} : \D\br{0, \epsilon} \to \CC^m $ such that $ \eval{\overline{f}}_U = f $.
\end{theorem}

\begin{example*}
Hartogs theorem is false for $ n = 1 $. If $ f\br{z} = 1 / z $, for all $ \epsilon > \delta > 0 $, then $ f $ cannot be extended.
\end{example*}

\subsection{Cauchy formula in one variable}

\lecture{3}{Tuesday}{14/01/20}

Let $ \omega = x + iy \in \CC $ for $ x, y \in \RR $, and let $ f : U \to \CC $ be $ \C^\infty $ for some $ U \subset \CC $. Recall that
$$ \dpd{f}{\omega} = \dfrac{1}{2}\br{\dpd{}{x} - i\dpd{}{y}}f, \qquad \dpd{f}{\overline{\omega}} = \dfrac{1}{2}\br{\dpd{}{x} + i\dpd{}{y}}f. $$
Recall that $ f $ is holomorphic if and only if $ \tpd{f}{\overline{\omega}} = 0 $ on $ U $. More in general, let $ U \subset \CC^n $ be open, let $ z_i = x_i + iy_i $, and let $ f : U \to \CC $ be a $ \C^\infty $-function. Then $ f $ is holomorphic if and only if $ \tpd{f}{\overline{z_i}} = 0 $ for all $ i = 1, \dots, n $. Let $ \omega \in \CC $. Since $ \d x \wedge \d y = -\d y \wedge \d x $, let
$$ \d A = \dfrac{i}{2} \d\omega \wedge \d\overline{\omega} = \dfrac{i}{2} \br{\d x + i \d y} \wedge \br{\d x - i \d y} = \d x \wedge \d y, $$
which is the Lebesgue measure on $ \RR^2 \cong \CC $.

\begin{proposition}
Let $ f : U \to \CC $ for $ U \subset \CC $ be a $ \C^\infty $-function, and let $ D = \D\br{z, r} $ such that $ \overline{D} \subset U $. Then
$$ f\br{z} = \dfrac{1}{2\pi i}\intd{\partial D}{}{\dfrac{f}{\omega - z}}{\omega} - \dfrac{1}{\pi}\intd{D}{}{\dfrac{1}{\omega - z}\dpd{f}{\overline{\omega}}}{A}. $$
\end{proposition}

\begin{proof}
Assume $ z = 0 $. Recall that $ f\br{\omega} = 1 / \omega $ is locally integrable around zero, so
$$ \intd{D}{}{\dfrac{1}{\omega}\dpd{f}{\overline{\omega}}}{A} = \lim_{\epsilon \to 0} \intd{D \setminus \D\br{0, \epsilon}}{}{\dfrac{1}{\omega}\dpd{f}{\overline{\omega}}}{A}. $$
Away from zero
\begin{align*}
\d\br{\dfrac{f}{\omega} \d\omega}
& = \dfrac{1}{\omega} \d f \wedge \d\omega + f \d\br{\dfrac{1}{\omega}} \wedge \d\omega
= \dfrac{1}{\omega} \br{\dpd{f}{\omega} \d\omega + \dpd{f}{\overline{\omega}} \d\overline{\omega}} \wedge \d\omega + f\dpd{}{\omega}\br{\dfrac{1}{\omega}} \d\omega \wedge \d\omega \\
& = \dfrac{1}{\omega}\dpd{f}{\overline{\omega}} \d\overline{\omega} \wedge \d\omega
= \dfrac{2i}{\omega}\dpd{f}{\overline{\omega}} \d A.
\end{align*}
Then
\begin{align*}
\dfrac{1}{\pi}\intd{D}{}{\dfrac{1}{\omega}\dpd{f}{\overline{\omega}}}{A}
& = \dfrac{1}{\pi}\lim_{\epsilon \to 0} \intd{D \setminus \D\br{0, \epsilon}}{}{\dfrac{1}{\omega}\dpd{f}{\overline{\omega}}}{A} \\
& = \dfrac{1}{2\pi i}\lim_{\epsilon \to 0} \intd{D \setminus \D\br{0, \epsilon}}{}{}{\br{\dfrac{f}{\omega} \d\omega}} & \dfrac{1}{\omega}\dpd{f}{\overline{\omega}} \d A = \dfrac{1}{2i} \d\br{\dfrac{f}{\omega} \d\omega} \\
& = \dfrac{1}{2\pi i}\lim_{\epsilon \to 0} \br{\intd{\partial D}{}{\dfrac{f}{\omega}}{\omega} - \intd{\partial\D\br{0, \epsilon}}{}{\dfrac{f}{\omega}}{\omega}} & \text{Stokes' theorem} \\
& = \dfrac{1}{2\pi i}\br{\intd{\partial D}{}{\dfrac{f}{\omega}}{\omega} - 2\pi if\br{0}} & \lim_{\epsilon \to 0} \intd{\partial\D\br{0, \epsilon}}{}{\dfrac{1}{\omega}}{\omega} = 2\pi i.
\end{align*}
\end{proof}

If $ f $ is holomorphic, then $ \tpd{f}{\overline{\omega}} = 0 $, which implies Theorem \ref{thm:2.3}.

\pagebreak

\subsection{Rank theorem}

Let $ U \subset \CC^n $ be open, and let $ f : U \to \CC^m $ be holomorphic. Then the \textbf{Jacobian} is
$$ \J_f = \br{\dpd{f_j}{z_i}\br{z}}, $$
where $ f_j = \p_j \circ f $ and $ \p_j : \CC^m \to \CC $ is the $ j $-th projection.

\begin{exercise*}
Show that the real Jacobian, which is $ 2n \times 2n $, has non-negative determinants.
\end{exercise*}

\begin{theorem}[Rank theorem]
Let $ z \in U $ such that $ r = \rk \J_f\br{z'} $ is constant around $ z $. Then there exist open $ z \in V \subset U \subset \CC^n $ and $ f\br{z} \in W \subset f\br{U} \subset \CC^m $ such that $ \phi : \D\br{0, 1}^n \to V $ and $ \psi : \D\br{0, 1}^m \to W $ are biholomorphisms such that
$$ \function[\eta = \psi^{-1} \circ f \circ \phi]{\D\br{0, 1}^n}{\D\br{0, 1}^m}{\br{z_1, \dots, z_n}}{\br{z_1, \dots, z_r, 0, \dots, 0}}, $$
so
$$
\begin{tikzcd}
\CC^n \supset U & V \arrow[subset]{l} \ni z \arrow{r}{f} & f\br{z} \in W \arrow[subset]{r} & f\br{U} \subset \CC^m \\
& \D\br{0, 1}^n \arrow{u}{\phi} \arrow[swap]{r}{\eta} & \D\br{0, 1}^m \arrow[swap]{u}{\psi} &
\end{tikzcd}.
$$
\end{theorem}

\begin{corollary}[Inverse function theorem]
Let $ f : U \to \CC^n $ be holomorphic for $ U \subset \CC^n $, and let $ z \in U $ such that $ \det \J_f\br{z} \ne 0 $. Then $ f $ is a biholomorphism at $ z $.
\end{corollary}

\begin{proof}
$ \det \J_f\br{z} \ne 0 $ if and only if $ \rk \J_f\br{z} = n $, so $ \rk \J_f\br{z'} = n $ around $ z $, since $ \det \J_f\br{z} $ is a continuous function. Let $ \phi $ and $ \psi $ as in the theorem. Then $ \eta = \psi^{-1} \circ f \circ \phi = \id $, so on $ V $, $ f = \psi \circ \phi^{-1} $ is a composition of biholomorphisms, which is a biholomorphism.
\end{proof}

\begin{remark}
Let $ f : U \to \CC^n $ for $ U \subset \CC^n $. Then $ \det \J_f\br{z} $ is a holomorphism, so
$$ Z = \cbr{z \in U \st \det \J_f\br{z} = 0} $$
is closed.
\end{remark}

\subsection{Holomorphic differential forms}

Let $ U \subset \CC^n $ be open.

\begin{definition}
A \textbf{holomorphic vector field} on $ U $ is the expression
$$ X = \sum_i a_i\dpd{}{z_i}, $$
where $ a_i $ are holomorphic functions on $ U $.
\end{definition}

For all $ x \in U $, the \textbf{tangent space} is
$$ \T_xU = \abr{\dpd{}{x_1}, \dots, \dpd{}{x_n}} \cong \CC^n. $$
If $ x \in U $, then $ X\br{x} \in \T_xU $.

\begin{notation}
$$ \H^0\br{U, \OOO_U} = \cbr{\text{holomorphic functions} \ f : U \to \CC}, \qquad \H^0\br{U, \T_U} = \cbr{\text{holomorphic vector fields on} \ U}. $$
\end{notation}

\begin{remark*}
$ R = \H^0\br{U, \OOO_U} $ is a ring and $ M = \H^0\br{U, \T_U} $ is a module over $ R $. That is, if $ X \in \H^0\br{U, \T_U} $ and $ f \in \H^0\br{U, \OOO_U} $, then $ fX \in \H^0\br{U, \T_U} $.
\end{remark*}

\pagebreak

\begin{definition}
Let $ R $ be a ring and $ M $ be an $ R $-module for $ p \ge 1 $. The \textbf{$ p $-th exterior power} $ \Lambda^pM $ of $ M $ is the $ R $-module $ M^{\otimes p} $ with the relations
$$ m_1 \otimes \dots \otimes m_p - \epsilon\br{\sigma}m_{\sigma\br{1}} \otimes \dots \otimes m_{\sigma\br{p}}, \qquad m_1, \dots, m_p \in M, \qquad \sigma \in \SSS_p, $$
where $ \epsilon\br{\sigma} = \br{-1}^m $ is the signature of $ \sigma $ and $ m $ is the number of transpositions defining $ \sigma $. Then $ M^* = \Hom_R\br{M, R} $ is the \textbf{dual} of $ M $ as an $ R $-module.
\end{definition}

\lecture{4}{Thursday}{16/01/20}

Let $ R = \H^0\br{U, \OOO_U} $ and $ M = \H^0\br{U, \T_U} $.

\begin{definition}
Let $ p > 0 $. We define a \textbf{holomorphic $ p $-form}, as an element of
$$ \H^0\br{U, \Omega_U^p} = \Lambda^pM^*. $$
If $ p = 0 $, by convention a \textbf{holomorphic $ 0 $-form} is just an element in $ R $.
\end{definition}

Let $ \br{z_1, \dots, z_n} $ be coordinates for $ U $. Recall $ \eta \in M $ is given by $ \eta = \sum_i a_i\tpd{}{z_i} $ for holomorphic functions $ a_i \in R $. Then $ \omega \in M^* $ is given by the expression
$$ \sum_i b_i \d z_i, \qquad b_i \in R, \qquad \d z_i\br{\dpd{}{z_j}} = \delta_{ij}. $$
More in general $ \omega \in \H^0\br{U, \Omega_U^p} $ is given by
$$ \omega = \sum_{\abs{I} = p} f_I \d z_{i_1} \wedge \dots \wedge \d z_{i_p}, \qquad f_I \in R, \qquad I = \br{i_1, \dots, i_p}, \qquad i_1 < \dots < i_p, $$
where $ \d z_{i_1}, \dots, \d z_{i_p} $ is an $ R $-basis of $ \H^0\br{U, \Omega_U^p} $.

\begin{example*}
$$ \H^0\br{U, \Omega_U^p} \cong \Lambda^p\H^0\br{U, \Omega_U^1} $$
is an isomorphism as $ R $-modules. This is not true for complex manifolds in general.
\end{example*}

The \textbf{exterior product} is
$$ \function{\H^0\br{U, \Omega_U^p} \otimes \H^0\br{U, \Omega_U^q}}{\H^0\br{U, \Omega_U^{p + q}}}{\omega_1 \otimes \omega_2}{\omega_1 \wedge \omega_2}, $$
where we just need to define
$$ \omega_1 \wedge \omega_2 = f \d z_{i_1} \wedge \d z_{i_p} \otimes g \d z_{j_1} \wedge \d z_{j_q} = fg \d z_{i_1} \wedge \dots \wedge \d z_{i_p} \wedge \d z_{j_1} \wedge \dots \wedge \d z_{j_q}, $$
by linearity. Then $ \omega_1 \wedge \omega_2 = 0 $ if $ \cbr{i_1, \dots, i_p} \cap \cbr{j_1, \dots, j_q} \ne \emptyset $, since $ \d z_i \wedge \d z_i = 0 $.

\begin{exercise*}
Check that this definition coincides with the definition in M4P54.
\end{exercise*}

The \textbf{exterior derivative} is
$$ \function[\d]{\H^0\br{U, \Omega_U^p}}{\H^0\br{U, \Omega_U^{p + 1}}}{f \d z_{i_1} \wedge \dots \wedge \d z_{i_p}}{\sum_{j = 1}^n \dpd{f}{z_j} \d z_j \wedge \d z_{i_1} \wedge \dots \wedge \d z_{i_p}}. $$
By definition $ \d $ is $ \CC $-linear, but not $ R $-linear. That is,
$$ \d\br{a\omega_1 + b\omega_2} = a \d\omega_1 + b \d\omega_2, \qquad \omega_1, \omega_2 \in \H^0\br{U, \Omega_U^p}, \qquad a, b \in \CC. $$

\begin{proposition}
Let $ U \subset \CC^n $ be open. Then
\begin{itemize}
\item the Leibnitz rule
$$ \d\br{\omega_1 \wedge \omega_2} = \d\omega_1 \wedge \omega_2 + \br{-1}^p \omega_1 \wedge \d\omega_2, \qquad \omega_1 \in \H^0\br{U, \Omega_U^p}, \qquad \omega_2 \in \H^0\br{U, \Omega_U^q}, $$
\item $ \d^2 = 0 $, that is
$$ \d\br{\d\omega} = 0, \qquad \omega \in \H^0\br{U, \Omega_U^p}. $$
\end{itemize}
\end{proposition}

\pagebreak

\begin{definition}
Let $ f : U \subset \CC^n \to \CC^m $ be holomorphic, let $ f_i = \p_i \circ f : V \to \CC $ where $ \p_i : \CC^m \to \CC $ is the $ i $-th projection, and let $ f\br{U} \subset V \subset \CC^m $ be open. Then if
$$ \omega = h \d z_{i_1} \wedge \dots \wedge \d z_{i_p} \in \H^0\br{V, \Omega_V^p}, \qquad h \in \H^0\br{U, \OOO_U}, $$
then we can define the \textbf{pull-back} of $ \omega $,
$$ f^*\omega = h \circ f \d f_{i_1} \wedge \dots \wedge \d f_{i_p} \in \H^0\br{U, \Omega_U^p}, $$
since $ f_i \in \H^0\br{V, \OOO_V} = \H^0\br{V, \Omega_V^0} $ implies that $ \d f_i \in \H^0\br{V, \Omega_V^1} $, so
$$
\begin{tikzcd}
U \arrow{r}{f} \arrow[swap]{dr}{h \circ f \in \H^0\br{U, \OOO_U}} & f\br{U} \subset V \arrow{d}{h} \\
& \CC
\end{tikzcd}.
$$
\end{definition}

This is linear over $ \CC $ and over $ \H^0\br{U, \OOO_U} $.

\begin{proposition}
Let $ U \subset \CC^n $, $ V \subset \CC^m $, and $ W \subset \CC^{m'} $ be open, let $ f : U \to \CC^m $ and $ g : V \to \CC^{m'} $ be holomorphic such that $ V \supset f\br{U} $ and $ W \supset g\br{V} $, and let $ \omega \in \H^0\br{V, \Omega_V^p} $ and $ \eta \in \H^0\br{V, \Omega_V^q} $. Then
\begin{itemize}
\item $ f^*\br{\omega + \eta} = f^*\omega + f^*\eta $ if $ p = q $,
\item $ f^*\br{\omega \wedge \eta} = f^*\omega \wedge f^*\eta $,
\item $ \d f^*\omega = f^*\d\omega $, and
\item $ f^*g^*\omega = \br{g \circ f}^*\omega $.
\end{itemize}
\end{proposition}

Let $ U \subset \CC^n \cong \RR^{2n} $, and let $ z_i = x_i + iy_i $ for $ i = 1, \dots, n $ and $ x_i, y_i \in \RR $. Then
$$ \d z_i = \d x_i + i \d y_i, $$
so any holomorphic form is a differentiable form on $ \RR^{2n} $. A \textbf{$ \br{p, q} $-form} is a differentiable $ \br{p + q} $-form of the expression
$$ \omega = \sum_{\abs{I} = p, \ \abs{J} = q} f_{I, J} \d z_{i_1} \wedge \dots \wedge \d z_{i_p} \wedge \d\overline{z_{j_1}} \wedge \dots \wedge \d\overline{z_{j_q}}, \qquad f_{I, J} : U \to \CC \cong \RR^2 \in \C^\infty, $$
where $ \d\overline{z_j} = \d x_j - i \d y_j $. We denote
$$ \C^\infty\br{U, \Omega_U^{p, q}} = \cbr{\text{differentiable $ \br{p + q} $-forms on} \ U}. $$
If $ \omega $ is a $ \br{p, q} $-form, then the \textbf{conjugate} $ \overline{\omega} $ of $ \omega $ is the $ \br{q, p} $-form defined by
$$ \overline{\omega} = \sum_{\abs{I} = p, \ \abs{J} = q} \overline{f_{I, J}} \d\overline{z_{i_1}} \wedge \dots \wedge \d\overline{z_{i_p}} \wedge \d z_{j_1} \wedge \dots \wedge \d z_{j_q}. $$

\pagebreak

\section{Complex manifolds}

\subsection{Complex manifolds}

\begin{definition}
A \textbf{complex manifold} of dimension $ n $ is a connected Hausdorff topological space $ X $, with a countable open cover $ \cbr{U_\alpha} $ of $ X $ such that for all $ \alpha $, there exists $ \phi_\alpha : U_\alpha \to \CC^n $ such that $ \phi_\alpha : U_\alpha \to \phi_\alpha\br{U_\alpha} $ is a homeomorphism and
$$ \phi_\alpha \circ \phi_\beta^{-1} : \phi_\beta\br{U_\alpha \cap U_\beta} \to \phi_\alpha\br{U_\alpha \cap U_\beta} $$
is a biholomorphism for each $ \alpha $ and $ \beta $, so
$$
\begin{tikzcd}
& U_\alpha \cap U_\beta \arrow[swap]{dl}{\phi_\alpha} \arrow{dr}{\phi_\beta} & \\
\CC^n \supset \phi_\alpha\br{U_\alpha \cap U_\beta} \arrow[swap]{rr}{\phi_\alpha \circ \phi_\beta^{-1}} & & \phi_\beta\br{U_\alpha \cap U_\beta} \subset \CC^n
\end{tikzcd}.
$$
The pair $ \br{U_\alpha, \phi_\alpha} $ is called a \textbf{holomorphic chart}. The set $ \cbr{\br{U_\alpha, \phi_\alpha}} $ is called a \textbf{holomorphic atlas} or a \textbf{complex structure}.
\end{definition}

Recall $ X $ is Hausdorff if for all $ x, y \in X $ there exist $ U $ and $ V $ open in $ X $ such that $ U \cap V = \emptyset $ and $ x \in U $ and $ y \in V $.

\lecture{5}{Thursday}{16/01/20}

\begin{example}
\hfill
\begin{itemize}
\item If $ U \subset \CC^n $ is an open set then $ U $ is a complex manifold. More in general if $ X $ is a complex manifold and $ U \subset X $ is open then $ U $ is a complex manifold. Let $ \cbr{\br{U_\alpha, \phi_\alpha}} $ be a complex structure on $ X $. Then
$$ \cbr{\br{\overline{U_\alpha}, \overline{\phi_\alpha}}} = \cbr{\br{U_\alpha \cap U, \eval{\phi_\alpha}_{\overline{U_\alpha}}}} $$
is a complex structure of $ X $.
\item If $ X $ and $ Y $ are complex manifolds, then $ X \times Y $ is a complex manifold.
\end{itemize}
\end{example}

\begin{example}
The projective space $ \PP_\CC^n $ or $ \CP^n $. Let $ V^* = \CC^{n + 1} \setminus \cbr{0} $, with coordinates $ \br{z_0, \dots, z_n} $. Define an equivalence on $ V^* $ as
$$ v_1 \sim v_2 \qquad \iff \qquad \exists \lambda \in \CC, \ v_1 = \lambda v_2. $$
Check that $ \sim $ is an equivalence. Consider the Euclidean topology on $ V^* $. Then there exists an induced topology on $ X = V^* / \sim = \cbr{\sbr{v} \st v \in V^*} $, with quotient map
$$ \function[q]{V^*}{X}{v}{\sbr{v}}. $$
Given $ v = \br{z_0, \dots, z_n} \in V^* $ we denote $ \sbr{v} = \sbr{z_0, \dots, z_n} $ such that $ z_i \ne 0 $ for some $ i $. Two elements $ \sbr{x_0, \dots, x_n} $ and $ \sbr{y_0, \dots, y_n} $ of $ X $ define the same point if and only if there exists $ \lambda $ such that $ x_i = \lambda y_i $ for all $ i $. Let
$$ V_i = \cbr{\br{z_0, \dots, z_n} \in V^* \st z_i \ne 0}, $$
which is open in $ V^* $, and let $ U_i = q\br{V_i} $, which is open in $ X $, such that $ \cbr{U_i} $ is a cover of $ X $, that is $ \bigcup_i U_i = X $. Let
$$ H_i = \cbr{\br{z_0, \dots, z_n} \in V^* \st z_i = 1}. $$
Then there exists a homeomorphism
$$ \function[r_i]{H_i}{\CC^n}{\br{z_0, \dots, z_n}}{\sbr{z_0, \dots, z_{i - 1}, z_{i + 1}, \dots, z_n}}, $$
and let
$$ \function[q_i = \eval{q}_{H_i}]{H_i \subset V^*}{U_i \subset X}{\br{z_0, \dots, z_n}}{\sbr{z_0, \dots, z_n}} $$
be also a homeomorphism.

\pagebreak

\begin{itemize}
\item $ q_i $ is surjective. Take $ \sbr{x_0, \dots, x_n} \in U_i $. Then $ x_i \ne 0 $, so choose $ \lambda = 1 / x_i $. Then
$$ \sbr{x_0, \dots, x_n} = \sbr{\dfrac{x_0}{x_i}, \dots, \dfrac{x_n}{x_i}} = q\br{z_0, \dots, z_n}, \qquad z_j = \dfrac{x_j}{x_i}, $$
and in particular $ z_i = 1 $, so there exists $ \br{z_0, \dots, z_n} \in H_i $ such that $ q_i\br{z_0, \dots, z_n} = \sbr{x_0, \dots, x_n} $.
\item $ q_i $ is injective. \footnote{Exercise}
\end{itemize}
For all $ i $, $ q_i^{-1} : U_i \to H_i $ and $ r_i : H_i \to \CC^n $ are homeomorphisms, so $ \phi_i = r_i \circ q_i^{-1} : U_i \to \CC^n $ is also a homeomorphism. We want to show that $ \br{U_i, \phi_i} $ define a holomorphic atlas, so
$$ \phi_i \circ \phi_j^{-1} : \phi_j\br{U_i \cap U_j} \to \phi_i\br{U_i \cap U_j} $$
is a biholomorphism. Consider the case $ j = 0 $ and $ i = 1 $. Then $ \phi_0\br{U_0 \cap U_1} = \cbr{\br{x_1, \dots, x_n} \st x_1 \ne 0} $, so
$$ \function[\phi_1 \circ \phi_0^{-1}]{\phi_0\br{U_0 \cap U_1}}{\phi_1\br{U_0 \cap U_1}}{\br{x_1, \dots, x_n}}{\br{1, \dfrac{x_2}{x_1}, \dots, \dfrac{x_n}{x_1}}} $$
is a biholomorphism. Thus $ X $ is a compact complex manifold. If $ n = 1 $, then $ \PP_\CC^1 \cong \S^2 $.
\end{example}

\begin{example}
The complex torus. Let
$$ \function{\Lambda = \ZZ^{2n}}{\CC^n}{\br{a_1, \dots, a_n, b_1, \dots, b_n}}{\br{a_1 + ib_1, \dots, a_n + ib_n}}. $$
Define an equivalence on $ \CC^n $ by
$$ v_1 \sim v_2 \qquad \iff \qquad v_1 - v_2 \in \Lambda. $$
Then $ X = \CC^n / \sim $ with quotient map $ q : \CC^n \to X $ is Hausdorff and compact. Topologically $ X \cong \sbr{0, 1}^{2n} / \sim $. For each $ x \in \CC^n $, we can find an open set $ x \in U \subset \CC^n $ such that $ \eval{q}_U : U \to X $ is a homeomorphism. The idea is if $ x \in \br{0, 1}^{2n} $ then we can take $ U = \br{0, 1}^{2n} $. If not, there exists a translation of $ \CC^n \to \CC^n $ such that the property holds. We define
$$ \phi_V = \eval{q}_U^{-1} : V \subset \CC^n / \Lambda \to U \subset \CC^n, \qquad V = q\br{U}. $$
Show that $ \br{V, \phi_V} $ define a complex structure on $ X $. \footnote{Exercise} This is also a compact complex manifold. More in general $ \CC^n / \Lambda $ where $ \Lambda \cong \ZZ^{2n} $ is a lattice is a compact complex manifold.
\end{example}

\subsection{Holomorphic functions on complex manifolds}

\lecture{6}{Tuesday}{21/01/20}

\begin{definition}
Let $ f : X \to Y $ be a continuous morphism between complex manifolds. Then $ f $ is \textbf{holomorphic} if there exists a complex structure $ \cbr{\br{U_\alpha, \phi_\alpha}} $ on $ Y $ and for all $ y \in Y $ there exists a holomorphic chart $ \br{V_\alpha, \psi_\alpha} $ such that $ x \in V_\alpha $ and $ f\br{V_\alpha} \subset U_\alpha $ around any point $ x $ of $ f^{-1}\br{y} $ and $ \phi_\alpha \circ f \circ \psi_\alpha^{-1} $ is holomorphic, so
$$
\begin{tikzcd}
X \supset V_\alpha \arrow{r}{f} \arrow[swap]{d}{\psi_\alpha} & U_\alpha \subset Y \arrow{d}{\phi_\alpha} \\
\psi_\alpha\br{V_\alpha} \arrow[swap]{r}{\widetilde{f}} & \phi_\alpha\br{U_\alpha}
\end{tikzcd}.
$$
Then $ \J_f = \J_{\widetilde{f}} $, and a \textbf{holomorphic function on $ X $} is a holomorphic function $ f : X \to \CC $.
\end{definition}

\begin{exercise}
If $ X $ is a compact complex manifold then any holomorphic function $ f : X \to \CC $ is constant.
\end{exercise}

\pagebreak

\begin{definition}
Let $ f : X \to Y $ be a holomorphic function between complex manifolds. Then $ f $ is
\begin{itemize}
\item a \textbf{submersion} if $ \dim Y \ge \dim Y = r $ and $ \rk \J_f = r $ at any point,
\item an \textbf{immersion} if $ r = \dim X \le \dim Y $ and $ \rk \J_f = r $ at any point, and
\item an \textbf{embedding} if it is an immersion and $ f : X \to f\br{X} $ is a homeomorphism.
\end{itemize}
\end{definition}

\begin{example}
Let $ f_2, \dots, f_n : \CC \to \CC $ be holomorphic, and let
$$ \function[f]{\CC}{\CC^n}{z}{\br{z, f_2\br{z}, \dots, f_n\br{z}}}. $$
Then $ f $ is an embedding.
\end{example}

\begin{example}
Let $ X = \CC^2 / \Lambda $ for $ \Lambda = \ZZ^4 \subset \CC^2 $, and let $ q : \CC^2 \to X $. Fix $ \lambda \in \CC $. Let
$$ \function[f]{\CC}{\CC^2}{z}{\br{z, \lambda z}}. $$
Then $ \widetilde{f} = q \circ f : \CC \to X $ is an immersion.
\begin{itemize}
\item If $ \lambda = 0 $ or $ \lambda = \tfrac{1}{2} $, then $ \widetilde{f}\br{\CC} $ is a closed submanifold.
\item If $ \lambda $ is general then $ \widetilde{f}\br{\CC} $ is dense inside $ X $, so it is not closed. Thus it is not a complex submanifold of $ X $.
\end{itemize}
\end{example}

\subsection{Complex submanifolds}

\begin{definition}
Let $ i : X \to Y $ be an embedding of complex manifolds. If $ i\br{X} \subset Y $ is closed then $ i\br{X} $ is called a \textbf{complex submanifold} of $ Y $. The \textbf{codimension} of $ X $ in $ Y $ is $ \dim Y - \dim X $.
\end{definition}

\begin{theorem}
\label{thm:3.11}
\hfill
\begin{enumerate}
\item Let $ i : X \to Y $ be a submanifold of codimension $ k $, and let $ n = \dim X $. Then for all $ x \in X $, there exists an open neighbourhood $ x \in U \subset Y $ and a submersion $ f : U \to \D\br{0, 1}^k \subset \CC^k $ such that $ X \cap U = f^{-1}\br{0} $.
\item If $ X \subset Y $ is a closed subset such that for all $ x \in X $ there exists $ U \ni x $ open in $ Y $ and a submersion $ f : U \to \D\br{0, 1}^k $ such that $ X \cap U = f^{-1}\br{0} $, then $ X $ is a complex submanifold.
\end{enumerate}
\end{theorem}

\begin{proof}
\hfill
\begin{enumerate}
\item We can assume that if there exists a holomorphic chart $ \br{U, \psi} $ on $ Y $ such that $ x \in U $ and if $ V = i^{-1}\br{U} $ then there exists $ \phi : V \to \CC^n $ such that $ \br{V, \phi} $ is a holomorphic chart on $ X $. After possibly shrinking $ U $ smaller, by the rank theorem, there exist biholomorphic $ a : \psi\br{U} \to \D\br{0, 1}^{n + k} $ and $ b : \phi\br{U} \to \D\br{0, 1}^n $ such that the induced morphism is given by
$$ \function{\D\br{0, 1}^n}{\D\br{0, 1}^{n + k}}{\br{z_1, \dots, z_n}}{\br{z_1, \dots, z_n, 0, \dots, 0}}. $$
Let
$$ \function[c]{\D\br{0, 1}^{n + k}}{\D\br{0, 1}^k}{\br{z_1, \dots, z_{n + k}}}{\br{z_{n + 1}, \dots, z_{n + k}}}, $$
so
$$
\begin{tikzcd}
Y & U \arrow[subset]{l} \arrow{r}{\phi} & \phi\br{U} \arrow{r}{b} & \D\br{0, 1}^n \subset \CC^n \arrow{d} \\
X \arrow[hookrightarrow]{u}{i} & V \arrow[hookrightarrow]{u}{i} \arrow[subset]{l} \arrow[swap]{r}{\psi} & \psi\br{U} \arrow[swap]{r}{a} & \D\br{0, 1}^{n + k} \subset \CC^{n + k} \arrow[bend right=90, dashed, swap]{u}{c}
\end{tikzcd}.
$$
Then $ f $ is the composition $ c \circ a \circ \psi : U \to \D\br{0, 1}^n $.

\pagebreak

\item Let $ \cbr{\br{U_\alpha, \phi_\alpha}} $ be a complex structure on $ Y $, and let $ V_\alpha = X \cap U_\alpha $ and $ \psi_\alpha = \eval{\phi_\alpha}_{V_\alpha} $. The goal is to show that $ \cbr{\br{V_\alpha, \psi_\alpha}} $ defines a complex structure on $ X $. By assumption,
$$ \phi_\alpha \circ \phi_\beta^{-1} : \phi_\beta\br{U_\alpha \cap U_\beta} \subset \CC^{n + k} \to \phi_\alpha\br{U_\alpha \cap U_\beta} \subset \CC^{n + k} $$
is biholomorphic. Let $ U' = \phi_\beta\br{U} $, let $ X' = \phi_\beta\br{X \cap U} $, and let $ f' = f \circ \phi_\beta^{-1} $, so
$$
\begin{tikzcd}
& & & \phi_\alpha\br{U} \arrow[subset]{r} & \phi_\alpha\br{U_\alpha \cap U_\beta} \subset \CC^{n + k} \\
Y & U_\alpha \cap U_\beta \arrow[subset]{l} & U \arrow{ur}{\phi_\alpha} \arrow[subset]{l} \arrow{r}{\phi_\beta} \arrow{drr}{f} & U' \arrow[subset]{r} \arrow{dr}{f'} & \phi_\beta\br{U_\alpha \cap U_\beta} \arrow[swap]{u}{\phi_\alpha \circ \phi_\beta^{-1}} \subset \CC^{n + k} \\
X \arrow[hookrightarrow]{u}{i} & X \cap U_\alpha \cap U_\beta \arrow[subset]{u} \arrow[subset]{l} & X \cap U \arrow[subset]{u} \arrow[subset]{l} & X' \arrow[near start, subset]{u} & \D\br{0, 1}^k \subset \CC^k
\end{tikzcd}.
$$
Then $ f'^{-1}\br{0} = \phi_\beta\br{X \cap U_\alpha \cap U_\beta} $ and $ f' $ is also a submersion. By the rank theorem, we may assume that $ U' = \D\br{0, 1}^{n + k} $ and $ f'\br{z_1, \dots, z_{n + k}} = \br{z_1, \dots, z_k} $, so $ \phi_\beta\br{X' \cap U_\alpha \cap U_\beta} = f'^{-1}\br{0} $. Thus
$$ \br{\psi_\alpha \circ \psi_\beta^{-1}}\br{z_1, \dots, z_n} = \br{\phi_\alpha \circ \phi_\beta^{-1}}\br{z_1, \dots, z_n, 0, \dots, 0} $$
is also a biholomorphism.
\end{enumerate}
\end{proof}

\subsection{Examples of complex manifolds}

\lecture{7}{Thursday}{23/01/20}

\begin{example}
Let $ U \subset \CC^n $ be open, let $ k \le n $, let $ f_1, \dots, f_k : U \to \CC $, and let
$$ V = \cbr{x \in \CC^n \st f_1\br{x} = \dots = f_k\br{x} = 0}. $$
Assume that $ \br{\tpd{f_i}{z_j}} $ has maximal rank $ k $ at any point of $ U $. Then $ V $ is a complex submanifold of $ U $. The idea is if $ f = \br{f_1, \dots, f_k} : U \to \CC^k $, then $ f $ is a submersion around any point of $ V $, and use the previous Theorem \ref{thm:3.11}.
\end{example}

\begin{example}
Let $ f : X \to Y $ be a holomorphism between complex manifolds, and let $ W \subset X $ be a submanifold. Then $ \eval{f}_W : W \to Y $ is holomorphic.
\end{example}

\begin{exercise}
Let $ X = \CC^n $. Show that all the compact submanifolds of $ X $ are zero-dimensional, that is points.
\end{exercise}

\begin{exercise}
Let $ X $ and $ Y $ be compact manifolds. Recall that $ X \times Y $ is also a complex manifold. Assume $ f : X \to Y $, so
$$ \Gamma_f = \cbr{\br{x, f\br{x}} \st x \in X} \subset X \times Y. $$
Show that $ \Gamma_f $ is a complex submanifold.
\end{exercise}

\begin{example}
Let $ n, m > 0 $, and let
$$ \Mat_{n, m} \CC = \cbr{\text{$ \br{n \times m} $-matrices}} \cong \CC^{n \cdot m}. $$
Then $ \Mat_{n, m} \CC $ is a complex manifold. Let
$$ \GL_n \CC = \cbr{\text{$ \br{n \times n} $-matrices} \ A \st A \ \text{invertible}}. $$
Then $ \GL_n \CC $ is a complex manifold, open in $ \Mat_{n, n} \CC $.
\end{example}

\pagebreak

\begin{example}
Projective manifolds. Let $ R = \CC\sbr{x_0, \dots, x_n} $ be the ring of polynomials, and let $ X = \PP_\CC^n $ be the complex projective space. Then $ f \in R $ is homogeneous of degree $ d $ if $ f\br{\lambda x} = \lambda^df\br{x} $. Let $ q : \CC^{n + 1} \setminus \cbr{0} \to \PP_\CC^n $, let $ F_1, \dots, F_k $ be homogeneous polynomials in $ R $, and let
$$ V = \cbr{F_1 = \dots = F_k = 0} \subset \CC^{n + 1} \setminus \cbr{0}, \qquad W = q\br{V} \subset \PP_\CC^n, $$
so $ q^{-1}\br{W} = V $, because $ F_i $ are homogeneous. Since $ V $ is closed in $ \CC^{n + 1} \setminus \cbr{0} $, $ W $ is closed in $ \PP_\CC^n $. Claim that if $ V $ is a submanifold of $ \CC^{n + 1} \setminus \cbr{0} $ then $ W $ is a compact submanifold of $ \PP_\CC^n $. If $ \cbr{U_i} $ is the open covering given by
$$ U_i = \cbr{\sbr{x_0, \dots, x_n} \st x_i \ne 0}, $$
then it is enough to show that $ W \cap U_i $ is a complex submanifold of $ U_i $ for all $ i $. Assume $ i = n $. Let $ \CC^* = \CC \setminus \cbr{0} $. Then $ q\br{x} = \CC^* $ for all $ x \in X $ but $ \PP_\CC^n \times \CC^* \ne \CC^{n + 1} \setminus \cbr{0} $. We want to show there exists a biholomorphism
$$ \function[\phi_n]{U_n \times \CC^*}{q^{-1}\br{U_n} = \cbr{\br{x_0, \dots, x_n} \in \CC^{n + 1} \st x_n \ne 0}}{\br{\sbr{x_0, \dots, x_n}, t}}{\br{\dfrac{tx_0}{x_n}, \dots, \dfrac{tx_{n - 1}}{x_n}, t}}, $$
such that
$$ \function[\phi_n^{-1}]{q^{-1}\br{U_n}}{U_n \times \CC^*}{\br{y_0, \dots, y_n}}{\br{q\br{y_0, \dots, y_n}, y_n} = \br{\sbr{y_0, \dots, y_n}, y_n}}. $$
From this, it follows that $ V \cap q^{-1}\br{U_n} \cong \br{W \cap U_n} \times \CC^* $, so the claim follows.
\end{example}

\begin{example}
Plane curves. Let $ X = \PP_\CC^2 $, let $ F \in R\sbr{x_0, x_1, x_2} $ be homogeneous of degree $ d $, and let $ W = \cbr{F = 0} \subset \PP_\CC^2 $. Then $ W $ is a compact complex submanifold if and only if for all $ x \in \PP_\CC^2 $, $ \partial_{x_i} F\br{x} \ne 0 $ for some $ i $.
\begin{itemize}[leftmargin=0.5in]
\item[$ d = 1 $.] $ W $ is the projective line, so $ F = ax_0 + bx_1 + cx_2 $ for $ a, b, c $ not all zero. Then $ W $ is a complex submanifold. There exists a biholomorphism $ \PP_\CC^1 \to W $.
\item[$ d = 2 $.] $ W $ is a conic, so $ F $ is a degree two polynomial. Then $ F = x_0x_1 $ does not define a manifold. If $ F = x_0x_1 - x_2^2 $, then $ W $ is a complex submanifold of $ X $. There exists
$$ \function{\PP_\CC^1}{W \subset X}{\sbr{t_0, t_1}}{\sbr{t_0^2, t_1^2, t_0t_1}}. $$
Check that it is a biholomorphism. \footnote{Exercise} This is true for any $ f $ of degree two such that $ W $ is a complex submanifold.
\item[$ d \ge 3 $.] If $ W $ is a complex submanifold then we will show that $ W $ is not biholomorphic to $ \PP_\CC^1 $.
\end{itemize}
\end{example}

\subsection{Tangent spaces of complex manifolds}

\begin{definition}
Let $ X $ be a complex manifold of dimension $ n $, and let $ x \in X $. Then there exists a chart $ \br{U, \phi} $ around $ x $ such that $ \phi\br{U} \subset \CC^n $. The \textbf{holomorphic tangent space} $ \T_xX $ of $ X $ at $ x $, is the vector space over $ \CC $ generated by
$$ \br{\dpd{}{z_1}, \dots, \dpd{}{z_n}}. $$
Let $ X $ be a real manifold. The \textbf{real tangent space} $ \T_x^\RR X $ is the vector space over $ \RR $ defined by
$$ \br{\dpd{}{x_1}, \dots, \dpd{}{x_n}, \dpd{}{y_1}, \dots, \dpd{}{y_n}}, $$
where $ \br{x_1, \dots, x_n, y_1, \dots, y_n} $ are coordinates of $ \RR^{2n} $. The \textbf{complex tangent space} $ \T_x^\CC X $ is the vector space over $ \CC $ generated by
$$ \br{\dpd{}{z_1}, \dots, \dpd{}{z_n}, \dpd{}{\overline{z_1}}, \dots, \dpd{}{\overline{z_n}}}, $$
a $ 2n $-dimensional vector space over $ \CC $. Then $ \T_x^\CC X = \T_x^\RR X \otimes_\RR \CC $.
\end{definition}

\pagebreak

\subsection{Holomorphic differential forms on complex manifolds}

\begin{definition}
Let $ X $ be a complex manifold of dimension $ n $. Let $ \cbr{\br{U_\alpha, \phi_\alpha}} $ be a complex structure on $ X $. A \textbf{holomorphic $ p $-form} on $ X $ is the data $ \omega_\alpha $, the $ p $-forms on $ \phi_\alpha\br{U_\alpha} \subset \CC^n $ such that if
$$ h_{\alpha\beta} = \phi_\alpha \circ \phi_\beta^{-1} : \phi_\beta\br{U_\alpha \cap U_\beta} \to \phi_\alpha\br{U_\alpha \cap U_\beta}, $$
then $ h_{\alpha\beta}^*\omega_\beta = \omega_\alpha $ for all $ \alpha $ and $ \beta $.
\end{definition}

\lecture{8}{Thursday}{23/01/20}

\begin{notation}
$$ \Omega_x^p\br{X} = \H^0\br{X, \Omega_x^p} = \cbr{\text{holomorphic $ p $-forms on} \ X}, $$
$$ \OOO_x\br{X} = \H^0\br{X, \OOO_x} = \cbr{\text{holomorphic functions on} \ X}. $$
\end{notation}

$ R = \OOO_x\br{X} $ is a ring and $ M = \Omega_x^p\br{X} $ is an $ R $-module.

\begin{lemma}
Let $ f : X \to Y $ be holomorphic. Then $ f^* : \Omega^p\br{Y} \to \Omega^p\br{X} $.
\end{lemma}

\begin{proof}
Let $ \cbr{\br{U_\alpha, \phi_\alpha}} $ be a complex structure on $ Y $. We can write $ f^{-1}\br{U_\alpha} = \bigcup_{\alpha, \beta} V_{\alpha, \beta} $ where $ \cbr{\br{V_{\alpha, \beta}, \psi_{\alpha, \beta}}} $ is a complex structure on $ X $, so
$$ \CC^n \xleftarrow{\psi_{\alpha, \beta}} V_{\alpha, \beta} \xrightarrow{\eval{f}_{V_{\alpha, \beta}}} U_\alpha \xrightarrow{\phi_\alpha} \CC^n. $$
Assume $ \omega $ is defined by $ \omega_\alpha $ on $ \phi_\alpha\br{U_\alpha} $. Let
$$ \omega_{\alpha, \beta} = \br{\br{\psi_{\alpha, \beta}^{-1}}^* \circ f^* \circ \phi_\alpha^*}\br{\omega_\alpha} $$
be a $ p $-form on $ \psi_{\alpha, \beta}\br{V_{\alpha, \beta}} $. Check that $ \omega_{\alpha, \beta} $ are compatible with respect to the atlas on $ X $. \footnote{Exercise}
\end{proof}

As in the local case, we can define
$$ \function{\Omega_x^p\br{X} \otimes \Omega_x^q\br{X}}{\Omega_x^{p + q}\br{X}}{\omega_1 \otimes \omega_2}{\omega_1 \wedge \omega_2}. $$
Similarly there exists $ \d : \Omega_x^p\br{X} \to \Omega_x^{p + 1}\br{X} $.

\pagebreak

\section{Vector bundles}

\subsection{Holomorphic vector bundles}

\begin{definition}
Let $ X $ be a complex manifold. A \textbf{holomorphic vector bundle} $ E $ of rank $ r $ on $ X $ is a complex manifold $ E $, a holomorphism $ \pi : E \to X $, and an open covering $ U_\alpha $ of $ X $ such that there exists a biholomorphism
$$ \psi_\alpha : \pi^{-1}\br{U_\alpha} \to U_\alpha \times \CC^r, $$
such that if $ \p_\alpha : U_\alpha \times \CC^r \to U_\alpha $ is the projection then $ \eval{\pi}_{\pi^{-1}\br{U_\alpha}} = \p_\alpha \circ \psi_\alpha $, so
$$
\begin{tikzcd}
E \arrow[swap]{d}{\pi} & \pi^{-1}\br{U_\alpha} \arrow[subset]{l} \arrow{r}{\psi_\alpha} \arrow[swap]{d}{\pi} & U_\alpha \times \CC^r \arrow{dl}{\p_\alpha} \\
X & U_\alpha \arrow[subset]{l} &
\end{tikzcd}.
$$
A vector bundle of rank one is called a \textbf{line bundle}.
\end{definition}

For any $ x \in X $, there exists $ \alpha $ such that $ x \in U_\alpha $, so
$$
\begin{tikzcd}
\pi^{-1}\br{x} \arrow{r}{\psi_\alpha} \arrow[swap]{d}{\pi} & \cbr{x} \times \CC^r \arrow{dl}{\p_\alpha} \\
x &
\end{tikzcd}.
$$
Then $ E\br{x} = \pi^{-1}\br{x} $ is a vector space of rank $ r $ over $ \CC $. Let $ U_\alpha \ni x \in U_\beta $. There exists a biholomorphism
$$ \CC^r \cong \p_\alpha^{-1}\br{x} \to \p_\beta^{-1}\br{x} \cong \CC^r, $$
because they are both biholomorphic to $ \pi^{-1}\br{x} $, so $ g_{\alpha\beta}\br{x} \in \GL_r \CC $ because all the biholomorphisms from $ \CC^r \to \CC^r $ are linear. The holomorphisms
$$ g_{\alpha\beta} : U_\alpha \cap U_\beta \to \GL_r \CC $$
are called \textbf{transition functions}. Then
$$
\begin{tikzcd}
\p_\alpha^{-1}\br{x} \arrow{rr}{\id} \arrow{dr} & & \p_\alpha^{-1}\br{x} \\
& \p_\beta^{-1}\br{x} \arrow{ur} &
\end{tikzcd},
$$
so
$$ \br{g_{\alpha\beta} \circ g_{\beta\alpha}}\br{x} = x, \qquad x \in U_\alpha \cap U_\beta, $$
and
$$
\begin{tikzcd}
\p_\alpha^{-1}\br{x} \arrow{rr}{g_{\alpha\gamma}} \arrow{dr} & & \p_\gamma^{-1}\br{x} \\
& \p_\beta^{-1}\br{x} \arrow{ur} &
\end{tikzcd},
$$
so
$$ \br{g_{\alpha\beta} \circ g_{\beta\gamma}}\br{x} = g_{\alpha\gamma}\br{x}, \qquad x \in U_\alpha \cap U_\beta \cap U_\gamma. $$

\begin{definition}
Let $ X $ be a complex manifold, and let $ E $ and $ F $ be vector bundles on $ X $ of rank $ r $ and $ s $ respectively, with $ \pi : E \to X $ and $ \pi' : F \to X $. A \textbf{holomorphic map} $ f : E \to F $ is a holomorphic function $ E \to F $ such that $ \pi = \pi' \circ f $ and such that the rank of the induced linear map $ E\br{x} \to F\br{x} $ is independent of $ x \in X $, so
$$
\begin{tikzcd}
E \arrow{rr}{f} \arrow[swap]{dr}{\pi} & & F \arrow{dl}{\pi'} \\
& X &
\end{tikzcd},
\qquad
\begin{tikzcd}
E\br{x} = \pi^{-1}\br{x} \arrow{rr}{f} \arrow[swap]{dr}{\pi} & & \pi'^{-1}\br{x} = F\br{x} \arrow{dl}{\pi'} \\
& x &
\end{tikzcd}.
$$
\end{definition}

\pagebreak

\subsection{Examples of holomorphic vector bundles}

\begin{example}
$ \pi : E = X \times \CC^r \to X $ is a vector bundle of rank $ r $, called \textbf{trivial}.
\end{example}

\begin{example}
Algebra of vector bundles. Let $ \pi : E \to X $ and $ \pi'^{-1} : F \to X $ be vector bundles on $ X $ of rank $ r $ and $ s $ respectively.
\begin{itemize}
\item The \textbf{direct sum} $ E \oplus F $ is the $ \br{r + s} $-vector bundle such that
$$ \br{E \oplus F}\br{x} = E\br{x} \oplus F\br{x}, \qquad x \in X. $$
The idea is to take an open cover which trivialises both $ E $ and $ F $. Find the transition function of $ E \oplus F $. \footnote{Exercise}
\item The \textbf{tensor product} $ E \otimes F $ is the $ \br{r \cdot s} $-vector bundle such that
$$ \br{E \otimes F}\br{x} = E\br{x} \otimes F\br{x}, \qquad x \in X. $$
\item The \textbf{$ p $-th exterior power} of $ E $ is the vector bundle $ \Lambda^pE $ such that
$$ \br{\Lambda^pE}\br{x} = \Lambda^p\br{E\br{x}}, \qquad x \in X. $$
If $ p = r = \rk E $ then $ \det E = \Lambda^rE $ is a line bundle on $ X $.
\item The \textbf{dual} of $ E $ is the rank $ r $ vector bundle $ E^* $ such that
$$ E^*\br{x} = \br{E\br{x}}^*, \qquad x \in X, $$
the dual $ \Hom\br{E\br{x}, \CC} $ of $ E\br{x} $.
\item Let $ f : E \to F $ be a holomorphic map. Then the \textbf{kernel} $ \Ker f $ is a vector bundle such that
$$ \br{\Ker f}\br{x} = \Ker f\br{x} \subset E\br{x}, \qquad x \in X. $$
The \textbf{cokernel} $ \Coker f $ is a vector bundle such that
$$ \br{\Coker f}\br{x} = \Coker f\br{x} \subset F\br{x}, \qquad x \in X. $$
\end{itemize}
\end{example}

\lecture{9}{Tuesday}{28/01/20}

\begin{example}
Let $ X = \PP_\CC^1 $, and let
$$ \OOO\br{-1} = \cbr{\br{x, v} \st x = \sbr{x_0, \dots, x_n} \in \PP_\CC^n, \ v = \mu\br{x_0, \dots, x_n}, \ \mu \in \CC} \subset \PP_\CC^n \times \CC^{n + 1}. $$
Then $ \pi = \p_1 : \OOO\br{-1} \to \PP_\CC^n $, so
$$ \pi^{-1}\br{\sbr{x_0, \dots, x_n}} = \cbr{v = \mu\br{x_0, \dots, x_n} \st \mu \in \CC} \cong \CC^1. $$
Let $ \cbr{U_i} $ be an open covering of $ X $ given by $ U_i = \cbr{\sbr{x_0, \dots, x_n} \st x_i \ne 0} $. We define
$$ \function[\psi_i]{\pi^{-1}\br{U_i}}{U_i \times \CC}{\br{\sbr{x_0, \dots, x_n}, \br{v_0, \dots, v_n}}}{\br{\sbr{x_0, \dots, x_n}, v_i}}, $$
which is a biholomorphism. Thus $ \OOO\br{-1} $ is a complex manifold and $ \OOO\br{-1} $ is a line bundle. The \textbf{tautological line bundle} $ \OOO\br{1} $ is the dual of $ \OOO\br{-1} $. Let
$$ \OOO\br{k} =
\begin{cases}
X \times \CC & k = 0 \\
\OOO\br{1}^{\otimes k} & k > 0 \\
\OOO\br{-1}^{\otimes k} & k < 0
\end{cases}.
$$
Then $ \OOO\br{k} = \OOO\br{-k}^* $. \footnote{Exercise} On $ \PP_\CC^n $ these are the only line bundles. That is, if $ \LLL $ is a line bundle on $ \PP_\CC^1 $, there exists $ k \in \ZZ $ such that $ \LLL \cong \OOO\br{k} $. Let $ X = \PP_\CC^1 $, and let $ E $ be a line bundle of rank $ r $ on $ X $. Then
$$ E \cong \bigoplus_{i = 1}^r \OOO\br{a_i}, \qquad a_1, \dots, a_r \in \ZZ. $$
This is false for $ X = \PP_\CC^n $, with $ n \ge 2 $.
\end{example}

\pagebreak

\begin{definition}
Let $ f : Y \to X $ be a holomorphism between complex manifolds, and let $ E $ be a vector bundle of rank $ r $ on $ X $. Then there exists a vector bundle $ f^*E $ of rank $ r $ on $ Y $ defined by
$$ f^*E = \cbr{\br{y, v} \in Y \times E \st f\br{y} = \pi\br{v}}, $$
the \textbf{fibre product} of $ E $ and $ Y $ over $ X $, such that
$$
\begin{tikzcd}
f^*E \arrow{r}{f'} \arrow[swap]{d}{\pi'} & E \arrow{d}{\pi} \\
Y \arrow[swap]{r}{f} & X
\end{tikzcd}.
$$
\end{definition}

Let $ U = \cbr{U_i} $ be an open cover of $ X $ which trivialises $ E $, so
$$
\begin{tikzcd}
\pi^{-1}\br{U_i} \arrow{rr}{\psi_i} \arrow[swap]{dr}{\pi} & & U_i \times \CC^r \arrow{dl}{\p_1} \\
& U_i &
\end{tikzcd}.
$$
Then $ U' = \cbr{f^{-1}\br{U_i}} $ is an open covering of $ Y $, so
$$
\begin{tikzcd}
\pi'^{-1}\br{f^{-1}\br{U_i}} \arrow{r}{f'} \arrow[swap]{d}{\pi'} & \pi^{-1}\br{U_i} \arrow{r}{\psi_i} \arrow{d}{\pi} & U_i \times \CC^r \arrow{r}{\p_2} & \CC^r \\
f^{-1}\br{U_i} \arrow[swap]{r}{f} & U_i & &
\end{tikzcd},
$$
and
$$ \function{\pi'^{-1}\br{f^{-1}\br{U_i}} = \cbr{\br{y, v} \in f^{-1}\br{U_i} \times \pi^{-1}\br{U_i} \st f\br{y} = \pi\br{v}}}{f^{-1}\br{U_i} \times \CC^r}{\br{y, v}}{\br{y, \p_2\br{\psi_i\br{v}}}} $$
is a biholomorphism. Thus $ f^*E $ is a vector bundle, where
$$ f^*E\br{y} = \pi'^{-1}\br{y} = E\br{f\br{y}}, \qquad y \in Y. $$

\begin{notation}
Let $ f : Y \to X $ be a morphism, and let $ E $ be a vector bundle on $ X $. Then $ f^*E = \eval{E}_Y $, mostly used if $ f : Y \hookrightarrow X $.
\end{notation}

\begin{definition}
Let $ E $ be a holomorphic vector bundle on a complex manifold $ X $, and let $ \pi : E \to X $. A \textbf{section} of $ E $ is a holomorphic function $ s : X \to E $ such that $ \pi \circ s = \id_X $.
\end{definition}

\begin{example}
Let $ E = X \times \CC^r $ be the trivial vector bundle of rank $ r $. Fix $ v \in \CC^r $. Then
$$ \function[s_v]{X}{E}{x}{\br{x, v}} $$
is a section of $ E $. If $ v_1, \dots, v_r $ is a basis of $ \CC^r $ then $ s_{v_1}, \dots, s_{v_r} $ have the property that $ s_{v_1}\br{x}, \dots, s_{v_r}\br{x} $ forms a basis of $ E\br{x} $. Vice versa, assume $ E $ is a vector bundle on $ X $ of rank $ r $ such that there exist sections $ s_1, \dots, s_r $ of $ E $ such that for all $ x \in X $, $ s_1\br{x}, \dots, s_r\br{x} $ is a basis of $ E\br{x} $. Then $ E \cong X \times \CC^r $, since
$$ \function{X \times \CC^r}{E}{\br{x, \br{v_1, \dots, v_r}}}{\sum_i v_is_i\br{x}} $$
is a biholomorphism. Then $ s_1, \dots, s_r $ is called a \textbf{holomorphic frame} for $ E $. Recall that for all $ E \to X $ and for all $ x \in X $ there exists open $ U \ni x $ such that $ \eval{E}_U $ is trivial, so there exists a frame on $ U $ for $ \eval{E}_U $. This is called a \textbf{local frame} around $ x $.
\end{example}

\pagebreak

\begin{example}
Let $ X $ be a complex manifold of dimension $ n $, and let $ \br{z_1, \dots, z_n} $ be coordinates on $ \CC^n $. There exists an atlas $ \cbr{\br{U_\alpha, \phi_\alpha}} $ for $ \phi_\alpha : U_\alpha \to V_\alpha \subset \CC^n $. For all $ x \in U_\alpha $, $ \T_xU_\alpha \to \T_{\phi_\alpha\br{x}}V_\alpha $, and $ \T_{\phi_\alpha\br{x}}V_\alpha = \abr{\tpd{}{z_1}, \dots, \tpd{}{z_n}} $ is a frame of $ \T_{V_\alpha} $. Let
$$ \T_X = \bigcup_{x \in X} \T_xX, $$
and let $ \pi^{-1} : \T_X \to X $ such that $ \pi^{-1}\br{x} = \T_xX $. Then $ \T_X $ is a holomorphic vector bundle of rank $ n $ called the \textbf{tangent bundle}, where $ U = \cbr{U_\alpha} $ and
$$ \psi_\alpha : \pi^{-1}\br{U_\alpha} = \eval{\T_X}_{U_\alpha} \to \eval{\T_{\CC^n}}_{V_\alpha} \cong V_\alpha \times \CC^r \to U_\alpha \times \CC^r $$
defines the trivialisation. The \textbf{cotangent bundle} of $ X $ is
$$ \Omega_X^1 = \T_X^*, $$
and let
$$ \Omega_X^p = \Lambda^p\Omega_X^1, \qquad p \ge 1. $$
A holomorphic $ p $-form on $ X $ is a section of $ \Omega_X^p $. \footnote{Exercise}
\end{example}

\subsection{Complexification of tangent bundles}

\lecture{10}{Thursday}{30/01/20}

Let $ X $ be a complex manifold. How to view $ X $ as a differentiable manifold? Let $ V $ be a vector space of dimension $ m $ over $ \RR $. An \textbf{almost complex structure} on $ V $ is a linear map $ J : V \to V $ such that $ J^2 = -\id_V $. If $ V $ admits an almost complex structure, then $ V $ can be seen as a vector space over $ \CC $. Let $ \lambda = a + ib $ for $ a, b \in \RR $, and let $ v \in V $. Define
$$ \lambda v = av + bJ\br{v}. $$
If $ \lambda_1, \lambda_2 \in \CC $, then $ \lambda_1\br{\lambda_2 v} = \br{\lambda_1\lambda_2}v $. \footnote{Exercise} Let $ v_1, \dots, v_n \in V $ be a basis over $ \CC $. Then
$$ v_1, \dots, v_n, J\br{v_1}, \dots, J\br{v_n} $$
is a basis of $ V $ over $ \RR $. The idea is to assume that $ a_i, b_i \in \RR $ such that $ \sum_i a_iv_i + \sum_i b_iJ\br{v_i} = 0 $, then
$$ 0 = \sum_i a_iv_i + \sum_i b_iJ\br{v_i} = \sum_i \br{a_iv_i + b_iJ\br{v_i}} = \sum_i \br{a_i + ib_i}v_i, $$
so $ a_i + ib_i = 0 $ for all $ i $. Thus $ a_i = b_i = 0 $, so $ m = 2n $. On a vector space an almost complex structure is a complex structure. Let $ V $ be a vector space of dimension $ 2n $ over $ \RR $. Then the \textbf{complexification} $ V_\CC = V \otimes_\RR \CC $ of $ V $ is a $ \CC $-vector space of dimension $ 2n $ over $ \CC $, where
$$ \function[\lambda]{V_\CC}{V_\CC}{v \otimes \mu}{v \otimes \mu\lambda}, \qquad \lambda \in \CC. $$
Let $ J $ be an almost complex structure on $ V $. Then we can extend $ J $ to a linear map
$$ \function[J]{V_\CC}{V_\CC}{v \otimes \mu}{J\br{v} \otimes \mu}, $$
such that $ J^2 = -\id_{V_\CC} $, \footnote{Exercise} so $ J^2 + \id_{V_\CC} = 0 $. Thus the eigenvalues of $ J $ on $ V_\CC $ are $ \pm i $. Let $ V^{1, 0} $ be the eigenspace for $ i $ and $ V^{0, 1} $ be the eigenspace for $ -i $, so
$$ V_\CC = V^{1, 0} \oplus V^{0, 1}. $$
The \textbf{conjugation}
$$ \function[\overline{\cdot}]{V_\CC}{V_\CC}{v \otimes \mu}{v \otimes \overline{\mu}} $$
on $ V_\CC $ in linear over $ \RR $, such that $ \overline{V^{1, 0}} = V^{0, 1} $ and $ \overline{V^{0, 1}} = V^{1, 0} $, \footnote{Exercise} so $ V^{1, 0} $ and $ V^{0, 1} $ are $ \CC $-vector spaces of dimension $ n $.

\pagebreak

\begin{example}
Let $ W = \CC^n $ with coordinates $ \br{z_1, \dots, z_n} $, and let $ z_j = x_j + iy_j $ with coordinates $ \br{x_1, y_1, \dots, x_n, y_n} $ for $ \RR^{2n} $. Define
$$ \function[J]{\RR^{2n}}{\RR^{2n}}{\br{x_1, y_1, \dots, x_n, y_n}}{\br{-y_1, x_1, \dots, -y_n, x_n}}. $$
Then $ J^2 = \id_{\RR^{2n}} $, and $ J $ is the \textbf{standard almost complex structure} on $ \RR^{2n} $. Let $ V = \RR^{2n} $, so $ V_\CC \cong \CC^{2n} $ with complex coordinates $ \br{x_1, y_1, \dots, x_n, y_n} $. Then $ V^{0, 1} $ is spanned by $ x_j - iy_j $ and $ V^{1, 0} $ is spanned by $ x_j + iy_j $, where $ \overline{x_j + iy_j} = x_j - iy_j $ for $ j = 1, \dots, n $.
\end{example}

\begin{definition}
Let $ X $ be a differentiable manifold. A \textbf{real, or complex, vector bundle} of rank $ r $ is a differentiable manifold $ E $ with a smooth morphism $ \pi : E \to X $ such that if $ K = \RR $, or $ K = \CC $, then there exists an open covering $ U = \cbr{U_i} $ of $ X $ such that
\begin{itemize}
\item for all $ x \in X $, the fibre of $ \pi $, $ E\br{x} = \pi^{-1}\br{x} $, is a vector space of rank $ r $ over $ K $,
\item for all $ i $ there exists a diffeomorphism $ h_i $ such that
$$
\begin{tikzcd}
\pi^{-1}\br{U_i} \arrow{rr}{h_i} \arrow[swap]{dr}{\pi} & & U_i \times K^r \arrow{r}{\p_2} \arrow{dl}{\p_1} & K^r \\
& U_i & &
\end{tikzcd},
$$
and for all $ x $, $ \p_2 \circ h_i : E\br{x} \to K^r $ is an isomorphism of vector spaces.
\end{itemize}
\end{definition}

Pull-backs, sections, exterior powers, tensors, direct sums, frames, etc are the same as holomorphic vector bundles, where holomorphic becomes smooth and biholomorphic becomes diffeomorphic, and for all $ X $ there exists a tangent bundle $ \T_X $. Assume $ X $ is a complex manifold of dimension $ n $. Let $ \T_X $ be the holomorphic tangent bundle of $ X $. Then $ X $ is also a differentiable manifold of dimension $ 2n $, so let $ \T_X^\RR $ be the \textbf{real tangent bundle} of $ X $, which is a rank $ 2n $ vector bundle, and let $ \T_X^\CC = \T_X^\RR \otimes_\RR \CC $ be the \textbf{complex tangent bundle} of $ X $, which is a non-holomorphic complex vector bundle of rank $ 2n $. Smooth morphisms of real or complex vector bundles are defined similarly as holomorphisms between holomorphic vector bundles such that the rank of the image is constant, so
$$
\begin{tikzcd}
E \arrow{rr}{f} \arrow[swap]{dr}{\pi} & & F \arrow{dl}{\pi'} \\
& X &
\end{tikzcd}.
$$

Let $ X $ be a differentiable manifold of dimension $ m = 2n $. Then an \textbf{almost complex structure} on $ X $ is a smooth morphism between the real tangent bundle $ J : \T_X^\RR \to \T_X^\RR $ such that $ J^2 = -\id $. In particular, $ J\br{x} : \T_x^\RR X \to \T_x^\RR X $ is an almost complex structure for all $ x \in X $.

\lecture{11}{Thursday}{30/01/20}

\begin{proposition}
Let $ X $ be a complex manifold. Then the underlying differentiable manifold admits an almost complex structure $ J : \T_X^\RR \to \T_X^\RR $ such that $ J^2 = -\id $.
\end{proposition}

\begin{proof}
Let $ x \in X $, and let $ \br{U, \phi} $ be a complex chart around $ x $ such that
$$ \function[\phi]{U}{V}{x}{0}. $$
Fix holomorphic coordinates $ \br{z_1, \dots, z_n} $ on $ U $. The tangent bundle of $ X $ on $ U $ is trivial, with a local frame $ \tpd{}{z_1}, \dots, \tpd{}{z_n} $, so
$$ \eval{\T_X}_U \xrightarrow{\sim} \T_V = V \times \CC^n. $$
Define $ x_i = \Re z_i $ and $ y_i = \Im e_i $. Then $ \br{x_1, y_1, \dots, x_n, y_n} $ are smooth coordinates $ U \to \RR $ around $ x $, and $ \tpd{}{x_1}, \tpd{}{y_1}, \dots, \tpd{}{x_n}, \tpd{}{y_n} $ define a local smooth frame of $ \T_X^\RR $ on $ U $, so
$$ \eval{\T_X^\RR}_U \xrightarrow{\sim} \T_V = V \times \RR^{2n}. $$
In particular, there exists an almost complex structure $ J_U $ for $ \T_V \cong \eval{\T_X^\RR}_U $, so
$$ J_U : \eval{\T_X^\RR}_U \to \eval{\T_X^\RR}_U, \qquad J_U^2 = -\id. $$

\pagebreak

Let $ f : V \to V $ be a biholomorphism, so
$$
\begin{tikzcd}
& U \cap U' \arrow[swap]{dl}{\phi} \arrow{dr}{\phi} & \\
V \arrow[swap]{rr}{f} & & V
\end{tikzcd},
$$
and let $ z_1', \dots, z_n' $ be local holomorphic coordinates given by
$$ z_i' = f_i\br{z_1, \dots, z_n}, \qquad f_i = \p_i \circ f, $$
where $ \p_i : \CC^n \to \CC $ is the $ i $-th projection. Define
$$ x_i' = \Re z_i' = \Re f_i\br{z_1, \dots, z_n} = u_i\br{z_1, \dots, z_n}, \qquad y_i' = \Im z_i' = \Im f_i\br{z_1, \dots, z_n} = v_i\br{z_1, \dots, z_n}, $$
so $ f_j = u_j + iv_j $. The real Jacobian $ \J_f $ of $ f $ is given by the derivatives of $ u_j $ and $ v_j $ with respect to $ x_1, y_1, \dots, x_n, y_n $, a $ \br{2n \times 2n} $-matrix of $ n \times n $ blocks of $ 2 \times 2 $ blocks of
$$ \twobytwo{\tpd{u_j}{x_k}}{\tpd{u_j}{y_k}}{\tpd{v_j}{x_k}}{\tpd{v_j}{y_k}}. $$
These define the transition function of $ \T_X^\RR $. To show that $ J $ extends to $ X $, it is enough to show that $ J $ commutes with $ \J_f $ at each point, so
$$
\begin{tikzcd}
\eval{\T_X^\RR}_{U \cap U'} \arrow{r}{\J_f} \arrow[swap]{d}{J} & \eval{\T_X^\RR}_{U \cap U'} \arrow{d}{J} \\
\eval{\T_X^\RR}_{U \cap U'} \arrow[swap]{r}{\J_f} & \eval{\T_X^\RR}_{U \cap U'}
\end{tikzcd}.
$$
Since $ f_j $ is holomorphic $ \tpd{f_j}{\overline{z_k}} = 0 $ for all $ j $ and $ k $, so the Cauchy-Riemann equations
$$ \dpd{u_j}{x_k} - \dpd{v_j}{y_k} = 0, \qquad \dpd{v_j}{x_k} + \dpd{u_j}{y_k} = 0, $$
or
$$ \twobytwo{\tpd{u_j}{x_k}}{\tpd{u_j}{y_k}}{\tpd{v_j}{x_k}}{\tpd{v_j}{y_k}} = \twobytwo{\tpd{v_j}{y_k}}{\tpd{u_j}{y_k}}{-\tpd{u_j}{y_k}}{\tpd{v_j}{y_k}}, $$
hold. Since $ J $ is the standard almost complex structure on $ \RR^{2n} $, where $ x_j \mapsto y_j $ and $ y_j \mapsto -x_j $,
$$ J =
\begin{pmatrix}
0 & 1 & & & \\
-1 & 0 & & 0 & \\
& & \ddots & & \\
& 0 & & 0 & 1 \\
& & & -1 & 0
\end{pmatrix}.
$$
Check that $ \J_f $ commutes with $ J $. \footnote{Exercise}
\end{proof}

\begin{corollary}
Every complex manifold is orientable.
\end{corollary}

\begin{proof}
We prove that if $ \T_X^\RR $ admits an almost complex structure then $ X $ is an orientable manifold. For all $ x \in X $ choose the orientation on $ \T_x^\RR X $, a vector space of dimension $ 2n $ over $ \RR $, given by any ordered basis of the form
$$ v_1, \dots, v_n, J\br{v_1}, \dots, J\br{v_n}. $$
Assume that $ v_1, \dots, v_n, J\br{v_1}, \dots, J\br{v_n} $ and $ w_1, \dots, w_n, J\br{w_1}, \dots, J\br{w_n} $ are ordered bases. Show that the determinant of the matrix given by the change of basis is positive. \footnote{Exercise}
\end{proof}

\pagebreak

\subsection{Differential forms on complex tangent bundles}

Let $ X $ be a complex manifold. Then there exists an almost complex structure $ J : \T_X^\RR \to \T_X^\RR $ on $ X $. Then $ J $ extends to
$$ \function[J]{\T_X^\CC}{\T_X^\CC}{v \otimes \mu}{J\br{v} \otimes \mu}. $$
For all $ x $, $ J\br{x} $ has two eigenvalues $ \pm i $, so
$$ \T_X^\CC = \T_X^{1, 0} \oplus \T_X^{0, 1}, $$
which are complex vector bundles and \textbf{eigenbundles}. Locally $ \T_X^{1, 0} $ and $ \T_X^{0, 1} $ are spanned by the frames $ \tpd{}{z_1}, \dots, \tpd{}{z_n} $ and $ \tpd{}{\overline{z_1}}, \dots, \tpd{}{\overline{z_n}} $ respectively. Moreover there exists a conjugation
$$ \function{\T_X^\CC}{\T_X^\CC}{v \otimes \mu}{v \otimes \overline{\mu}} $$
over $ \RR $, such that $ \overline{\T_X^{1, 0}} = \T_X^{0, 1} $ and $ \overline{\T_X^{0, 1}} = \T_X^{1, 0} $. Let
$$ \Omega_{X, \CC}^1 = \br{\T_X^\CC}^* $$
be the dual of the complex vector bundle $ \T_X^\CC $. Then
$$ \Omega_{X, \CC}^1 = \Omega_{X, \RR}^1 \otimes_\RR \CC = \Omega_X^{1, 0} \oplus \Omega_X^{0, 1} = \br{\T_X^{1, 0}}^* \oplus \br{\T_X^{0, 1}}^*. $$

\begin{exercise*}
Let $ V $ and $ W $ be vector spaces. Show that
$$ \Lambda^k\br{V \oplus W} = \bigoplus_{p + q = k} \Lambda^pV \otimes \Lambda^qW $$
is a canonical isomorphism.
\end{exercise*}

\lecture{12}{Tuesday}{04/02/20}

Thus,
$$ \Omega_{X, \CC}^k = \Lambda^k\Omega_{X, \CC}^1 = \bigoplus_{p + q = k} \Omega_X^{p, q}, \qquad \Omega_X^{p, q} = \Lambda^p\Omega_X^{1, 0} \otimes \Lambda^q\Omega_X^{0, 1}, \qquad k \ge 0, $$
where $ \Omega_X^{p, q} $ is a complex vector bundle for any $ p $ and $ q $.

\begin{definition}
The sections of $ \Omega_X^{p, q} $ are called \textbf{$ \br{p, q} $-forms} on $ X $, or \textbf{forms of type $ \br{p, q} $}.
\end{definition}

Locally, let $ x \in X $, and let $ \br{U \ni x, \phi} $ be a holomorphic chart for $ \phi : U \xrightarrow{\sim} V \subset \CC^n $. A $ \br{p, q} $-form on $ U $ can be locally written as
$$ \omega = \sum_{I, J} \alpha_{I, J} \d z_I \wedge \d\overline{z_J} = \sum_{\abs{I} = p, \ \abs{J} = q} \alpha_{I, J} \d z_{i_1} \wedge \dots \wedge \d z_{i_p} \wedge \d\overline{z_{j_1}} \wedge \dots \wedge \d\overline{z_{j_q}}, $$
where $ \alpha_{I, J} $ are smooth functions on $ U $. Let $ X $ be a manifold. If $ E $ is a complex vector bundle then
$$ \C^\infty\br{X, E} = \cbr{\text{smooth sections of} \ E}. $$
The \textbf{differential}
$$ \d : \C^\infty\br{X, \Omega_{X, \CC}^k} \to \C^\infty\br{X, \Omega_{X, \CC}^{k + 1}} $$
satisfies the Leibnitz rule and $ \d^2 = 0 $, so $ \d\br{\d\omega} = 0 $. If $ \omega \in \C^\infty\br{X, \Omega_X^{p, q}} $, then $ \d\omega \in \C^\infty\br{X, \Omega_{X, \CC}^{p + q + 1}} $. Assume that locally $ \omega = \sum_{I, J} \alpha_{I, J} \d z_I \wedge \d\overline{z_J} $. Then
$$ \d\omega = \sum_{I, J} \d\alpha_{I, J} \d z_I \wedge \d\overline{z_J}, \qquad \d\alpha_{I, J} = \sum_{i = 1}^n \dpd{}{z_i}\alpha_{I, J} \d z_i + \sum_{i = 1}^n \dpd{}{\overline{z_i}}\alpha_{I, J} \d\overline{z_i}. $$

\pagebreak

Let
$$ \partial\alpha_{I, J} = \sum_{i = 1}^n \dpd{}{z_i}\alpha_{I, J} \d z_i \in \C^\infty\br{X, \Omega_X^{1, 0}}, \qquad \overline{\partial}\alpha_{I, J} = \sum_{i = 1}^n \dpd{}{z_i}\alpha_{I, J} \d\overline{z_i} \in \C^\infty\br{X, \Omega_X^{0, 1}}. $$
Then $ \d = \partial + \overline{\partial} $ for smooth functions. Back to $ \d\omega $. Then
$$ \d\omega = \sum_{I, J} \d\alpha_{I, J} \d z_I \wedge \d\overline{z_J} = \sum_{I, J} \partial\alpha_{I, J} \d z_I \wedge \d\overline{z_J} + \sum_{I, J} \overline{\partial}\alpha_{I, J} \d z_I \wedge \d\overline{z_J}. $$
Let
$$ \partial\omega = \sum_{I, J} \partial\alpha_{I, J} \d z_I \wedge \d\overline{z_J}, \qquad \overline{\partial}\omega = \sum_{I, J} \overline{\partial}\alpha_{I, J} \d z_I \wedge \d\overline{z_J}. $$
Then $ \d = \partial + \overline{\partial} $ for $ \omega $.

\begin{lemma}
The linear maps
$$ \partial : \C^\infty\br{X, \Omega_X^{p, q}} \to \C^\infty\br{X, \Omega_X^{p + 1, q}}, \qquad \overline{\partial} : \C^\infty\br{X, \Omega_X^{p, q}} \to \C^\infty\br{X, \Omega_X^{p, q + 1}} $$
satisfy the Leibnitz rule. That is, if $ \omega \in \C^\infty\br{X, \Omega_X^{p, q}} $ and $ \eta \in \C^\infty\br{X, \Omega_X^{p', q'}} $, then
$$ \partial\br{\omega \wedge \eta} = \partial\omega \wedge \eta + \br{-1}^{p + q}\omega + \partial\eta, \qquad \overline{\partial}\br{\omega \wedge \eta} = \overline{\partial}\omega \wedge \eta + \br{-1}^{p + q}\omega + \overline{\partial}\eta. $$
\end{lemma}

\begin{proof}
$ \d $ satisfies the Leibnitz rule
$$ \d\br{\omega \wedge \eta} = \d\omega \wedge \eta + \br{-1}^{p + q}\omega \wedge \d\eta, $$
since $ \omega \in \C^\infty\br{X, \Omega_{X, \CC}^{p + q}} $, so
\begin{align*}
\partial\br{\omega \wedge \eta} + \overline{\partial}\br{\omega \wedge \eta}
& = \br{\partial\omega + \overline{\partial}\omega} \wedge \eta + \br{-1}^{p + q}\omega \wedge \br{\partial\eta + \overline{\partial}\eta} \\
& = \br{\partial\omega \wedge \eta + \br{-1}^{p + q}\omega \wedge \partial\eta} + \br{\overline{\partial}\omega \wedge \eta + \br{-1}^{p + q}\omega \wedge \overline{\partial}\eta}.
\end{align*}
Then $ \partial\br{\omega \wedge \eta} $ and $ \partial\omega \wedge \eta + \br{-1}^{p + q}\omega \wedge \partial\eta $ are $ \br{p + 1, q} $-forms, and $ \overline{\partial}\br{\omega \wedge \eta} $ and $ \overline{\partial}\omega \wedge \eta + \br{-1}^{p + q}\omega \wedge \overline{\partial}\eta $ are $ \br{p, q + 1} $-forms. Forms of the same type in the decomposition of $ \d\br{\omega \wedge \eta} $ must coincide.
\end{proof}

\begin{lemma}
$ \partial^2 = \overline{\partial}^2 = \overline{\partial}\partial + \partial\overline{\partial} = 0 $.
\end{lemma}

\begin{proof}
Let $ \omega \in \C^\infty\br{X, \Omega_X^{p, q}} $. Because $ \d^2 = 0 $,
$$ 0 = \d^2\omega = \br{\partial + \overline{\partial}}\br{\br{\partial + \overline{\partial}}\omega} = \partial^2\omega + \partial\overline{\partial}\omega + \overline{\partial}\partial\omega + \overline{\partial}^2\omega. $$
Then $ \d^2\omega $ is a $ \br{p + q + 2} $-form, $ \partial^2\omega $ is a $ \br{p + 2, q} $-form, $ \partial\overline{\partial}\omega + \overline{\partial}\partial\omega $ is a $ \br{p + 1, q + 1} $-form, and $ \overline{\partial}^2\omega $ is a $ \br{p, q + 2} $-form. Forms of the same type in the decomposition of $ \d^2\omega $ must coincide.
\end{proof}

Let $ X $ be a complex manifold. Fix $ p, q \ge 0 $. Let
\begin{align*}
\ZZZ^{p, q}\br{X}
& = \Ker \br{\overline{\partial} : \C^\infty\br{X, \Omega_X^{p, q}} \to \C^\infty\br{X, \Omega_X^{p, q + 1}}} \\
& = \cbr{\omega \in \C^\infty\br{X, \Omega_X^{p, q}} \st \overline{\partial}\omega = 0}
\end{align*}
and let
\begin{align*}
\BBB^{p, q}\br{X}
& = \Im \br{\overline{\partial} : \C^\infty\br{X, \Omega_X^{p, q - 1}} \to \C^\infty\br{X, \Omega_X^{p, q}}} \\
& = \cbr{\omega \in \C^\infty\br{X, \Omega_X^{p, q}} \st \exists \eta \in \C^\infty\br{X, \Omega_X^{p, q - 1}}, \ \omega = \overline{\partial}\eta}.
\end{align*}
Since $ \overline{\partial}^2 = 0 $ we have $ \BBB^{p, q}\br{X} \subset \ZZZ^{p, q}\br{X} $ for all $ p $ and $ q $. The \textbf{Dolbeault cohomology} is
$$ \H^{p, q}\br{X} = \ZZZ^{p, q}\br{X} / \BBB^{p, q}\br{X}. $$

\pagebreak

\begin{exercise*}
Assume $ X $ and $ Y $ are biholomorphic complex manifolds. Then
$$ \H^{p, q}\br{X} = \H^{p, q}\br{Y}. $$
\end{exercise*}

If $ \H^{p, q}\br{X} $ is finite dimensional then we define the \textbf{Hodge numbers} of $ X $ as
$$ \h^{p, q} = \dim_\CC \H^{p, q}\br{X}. $$

\lecture{13}{Thursday}{06/02/20}

Our goal is if $ X $ is K\"ahler and compact
$$ \bigoplus_{p + q = k} \H^{p, q}\br{X} = \H^{p + q}\br{X}, $$
as the de Rham cohomology. In particular this is true if $ X $ is projective. How to compute $ \H^{p, q}\br{X} $? We need to use analysis.

\begin{proposition}
Let $ X $ be a complex manifold. Then there exists an isomorphism
$$ \H^{p, 0}\br{X} \cong \H^0\br{X, \Omega_X^p} = \cbr{\text{holomorphic sections of} \ \Omega_X^p} = \cbr{\text{holomorphic $ p $-forms on} \ X}, \qquad p \ge 0. $$
\end{proposition}

\begin{remark*}
If $ X $ is compact then $ \H^{0, 0}\br{X} = \CC $ because $ \H^{0, 0}\br{X} = \H^0\br{X, \OOO_X} $ are constants.
\end{remark*}

\begin{proof}
$$ \H^{p, 0}\br{X} = \ZZZ^{p, 0}\br{X} / \BBB^{p, 0}\br{X} = \ZZZ^{p, 0}\br{X} = \cbr{\omega \in \C^\infty\br{X, \Omega_X^{p, 0}} \st \overline{\partial}\omega = 0}. $$
Locally $ \omega = \sum_{\abs{I} = p} \alpha_I\d z_I $. Then
$$ \overline{\partial}\omega = \sum_{\abs{I} = p} \overline{\partial}\alpha_I \d z_I = \sum_{\abs{I} = p} \sum_{i = 1}^n \dpd{}{\overline{z_j}}\alpha_I \d\overline{z_j} \wedge \d z_I, $$
where $ \d\overline{z_j} \wedge \d z_I $ are linearly independent. For all $ I $ and for all $ j $, the Cauchy-Riemann equations $ \tpd{}{\overline{z_j}}\alpha_I = 0 $ hold, so for all $ I $, $ \alpha_I $ is holomorphic. Then $ \omega = \sum_{\abs{I} = p} \alpha_I \d z_I $ is a holomorphic $ p $-form, so $ \omega \in \H^0\br{X, \Omega_X^p} $.
\end{proof}

\pagebreak

\section{Connection, curvature, and metric}

\subsection{Connections}

Let $ X $ be a differentiable manifold, and let $ E $ be a complex vector bundle. Then
$$ \C^\infty\br{X, E} = \cbr{\text{$ \C^\infty $-sections of} \ E}. $$
Is there a way to compute the derivatives of these sections?

\begin{definition}
Let $ X $ and $ E $ be as above. A \textbf{connection} of $ E $ is a $ \CC $-linear map
$$ \nabla : \C^\infty\br{X, E} \to \C^\infty\br{X, \Omega_{X, \CC}^1 \otimes E} $$
such that the Leibnitz rule holds, so
$$ \nabla\br{f\omega} = f \cdot \nabla\omega + \d f \otimes \omega, \qquad f \in \C^\infty\br{X}, \qquad \omega \in \C^\infty\br{X, E}. $$
\end{definition}

The following is the idea. Let $ \omega \in \C^\infty\br{X, E} $. Then
$$ \nabla\omega = \sum_i \eta_i \otimes \omega_i, $$
where $ \eta_i $ are $ 1 $-forms on $ X $ and $ \omega_i $ are sections of $ E $. Let $ x \in X $, and let $ v \in \T_xX $. Then
$$ \nabla_v\omega_x = \sum_i \eta_i\br{v}\omega_i $$
is a section of $ E $ at $ x $. The goal is to study connections locally. Let $ x \in X $, and let $ \br{U, \phi} $ be a chart around $ x $ that trivialises $ E $, so $ \pi^{-1}\br{U} = U \times \CC^r $ for $ \pi : E \to X $ and $ r = \rk E $. Then there exists a frame $ s_1, \dots, s_r \in \C^\infty\br{U, E} $ of $ E $ on $ U $. Let $ \sigma \in \C^\infty\br{X, E} $ be any section. Locally on $ U $ we write
$$ \sigma \overset{U}{=} f = \br{f_1, \dots, f_r}, \qquad \sigma = \sum_{i = 1}^r f_is_i, \qquad f_1, \dots, f_r \in \C^\infty\br{U}. $$
By the Leibnitz rule, on $ U $,
$$ \nabla\sigma = \sum_{i = 1}^r \nabla\br{f_is_i} = \sum_{i = 1}^r \br{f_i \cdot \nabla s_i + \d f_i \otimes s_i} \in \C^\infty\br{U, \Omega_{X, \CC}^1 \otimes E}. $$

\begin{notation*}
$ \d f = \br{\d f_1, \dots, \d f_r} $.
\end{notation*}

Then
$$ \nabla s_j = \sum_{i = 1}^r a_{ij} \otimes s_i, \qquad a_{ij} \in \C^\infty\br{U, \Omega_{X, \CC}^1}. $$

\begin{notation*}
$ A = \br{a_{ij}} $ is an $ \br{r \times r} $-matrix with coefficients $ 1 $-forms.
\end{notation*}

With this notation, this becomes
$$ \nabla\sigma \overset{U}{=} A \cdot f + \d f. $$
\begin{itemize}
\item $ A $ depends very much on the choice of the frame.
\item Locally on $ U $, $ \nabla $ is determined by $ A $.
\end{itemize}
Consider another chart $ \br{U', \phi'} $ which also gives a trivialisation of $ E $. So we can choose a corresponding frame $ s_1', \dots, s_r' $. Assume $ \sigma \in \C^\infty\br{U \cap U', E} $. Then
$$ \sigma \overset{U'}{=} f' = \br{f_1', \dots, f_r'}, \qquad \sigma = \sum_{j = 1}^r f_j's_j', \qquad f_1', \dots, f_r' \in \C^\infty\br{U}. $$
Let $ A' $ be the matrix with respect to this frame. Then
$$ \nabla\sigma \overset{U'}{=} A' \cdot f' + \d f'. $$

\pagebreak

\lecture{14}{Thursday}{06/02/20}

Let
$$ g : \br{U \cap U'} \times \CC^r \to \br{U \cap U'} \times \CC^r $$
be the transition function from the trivialisation of $ U' $ to the trivialisation of $ U $. Then $ g\br{x} \in \GL_r \CC $ for all $ x \in U \cap U' $, and $ f = g \cdot f' $. Denote by $ \D g $ the differential of $ g $. Then
$$ \d f = \d\br{g \cdot f'} = \D g \cdot f' + g \cdot \d f' = g \cdot \br{g^{-1} \cdot \D g \cdot f' + \d f'}, $$
by the Leibnitz rule. Thus,
\begin{align*}
A' \cdot f' + \d f'
& \overset{U'}{=} A \cdot f + \d f
\overset{U}{=} A \cdot g \cdot f' + g \cdot \br{g^{-1} \cdot \D g \cdot f' + \d f'}
\overset{U}{=} g \cdot \br{\br{g^{-1} \cdot \D g + g^{-1} \cdot A \cdot g}f' + \d f'} \\
& \overset{U'}{=} \br{g^{-1} \cdot \D g + g^{-1} \cdot A \cdot g} \cdot f' + \d f',
\end{align*}
so
$$ A' = g^{-1} \cdot \D g + g^{-1} \cdot A \cdot g. $$

\subsection{Curvature operators}

What is $ \nabla^2 $? The idea is
$$ \C^\infty\br{X, E} \xrightarrow{\nabla} \C^\infty\br{X, \Omega_{X, \CC}^1 \otimes E} \xrightarrow{\nabla} \C^\infty\br{X, \Omega_{X, \CC}^1 \otimes \Omega_{X, \CC}^1 \otimes E} \xrightarrow{\wedge} \C^\infty\br{X, \Omega_{X, \CC}^2 \otimes E}. $$
The \textbf{curvature tensor} is
$$ \nabla^2 : \C^\infty\br{X, E} \to \C^\infty\br{X, \Omega_{X, \CC}^2 \otimes E}. $$

\begin{remark*}
If $ X $ has dimension one, then $ \Omega_{X, \CC}^2 = 0 $, so $ \nabla^2 = 0 $.
\end{remark*}

Again for all $ x \in X $, take $ U $ as above. Let $ s_1, \dots, s_r $ be a frame, let $ A = \br{a_{ij}} $ be the $ \br{r \times r} $-matrix of $ 1 $-forms, and let $ \D A $ be the differential of $ A $.

\begin{notation*}
$ A \wedge A = \br{\sum_{k = 1}^r \br{a_{ik} \wedge a_{kj}}} $ is an $ \br{r \times r} $-matrix of $ 2 $-forms.
\end{notation*}

Let $ \sigma \overset{U}{=} \br{f_1, \dots, f_r} = \sum_i f_is_i $ on $ U $. Then
\begin{align*}
\nabla^2\sigma
& = \nabla\br{A \cdot f + \d f}
= A \wedge \br{A \cdot f + \d f} + \d\br{A \cdot f + \d f} \\
& = A \wedge A \cdot f + A \wedge \d f + \D A \cdot f - A \wedge \d f + \d^2f
= \br{A \wedge A + \D A} \cdot f
\end{align*}
is $ \C^\infty $-linear, so $ \nabla^2\br{h\sigma} = h\nabla^2\sigma $. The \textbf{curvature operator} is
$$ \Theta_\nabla \overset{U}{=} A \wedge A + DA, $$
so $ \Theta_\nabla\br{\sigma} = \nabla^2\sigma $.

\subsection{Hermitian metrics}

\begin{definition}
Let $ V $ be a vector space over $ \CC $. A \textbf{Hermitian inner product} on $ V $ is a map
$$ \function{V \times V}{\CC}{\br{v, w}}{\abr{v, w}}, $$
such that
\begin{itemize}
\item $ \abr{v, w} = \overline{\abr{w, v}} $,
\item it is linear on the first factor, and
\item $ \abr{v, v} \ge 0 $ and $ \abr{v, v} = 0 $ if and only if $ v = 0 $.
\end{itemize}
\end{definition}

\begin{example*}
$ V = \CC $ and $ \abr{z_1, z_2} = z_1 \cdot \overline{z_2} $.
\end{example*}

\begin{definition}
Let $ X $ be a manifold, and let $ E $ be a complex vector bundle on $ X $. A \textbf{Hermitian metric} $ h $, or $ \abr{\cdot, \cdot} $, on $ E $ is a choice of a Hermitian inner product
$$ h_x = \abr{\cdot, \cdot}_x : E\br{x} \times E\br{x} \to \CC, \qquad x \in X, $$
such that for any open set $ U \subset X $ and for $ s, t \in \C^\infty\br{U, E} $, $ \abr{s\br{x}, t\br{x}}_x $ is a $ \C^\infty $-function with respect to $ x $ on $ U $. The pair $ \br{E, \abr{\cdot, \cdot}} = \br{E, h} $ is called a \textbf{Hermitian vector bundle}.
\end{definition}

\pagebreak

Let $ \br{E, h} $ be a Hermitian vector bundle, and let $ x \in X $. Locally, let $ s_1, \dots, s_r $ be a frame on $ U \ni x $. For any $ x \in U $, $ \abr{s_i\br{x}, s_j\br{x}}_x = h_{ij}\br{x} $ is a smooth function for all $ i $ and $ j $, so
$$ H = \br{h_{ij}}_{i, j = 1}^r $$
is an $ \br{r \times r} $-matrix of smooth functions. Let $ \sigma, \sigma' \in \C^\infty\br{U, E} $, and let $ \sigma \overset{U}{=} f = \br{f_1, \dots, f_r} $ and $ \sigma' \overset{U}{=} f' = \br{f_1', \dots, f_r'} $. Then
$$ \abr{\sigma\br{x}, \sigma'\br{x}}_x = f^\intercal \cdot H \cdot \overline{f'}. $$
Now assume that $ U' $ is a different open set with frame $ \br{s_1', \dots, s_r'} $. Assume
$$ g : \br{U \cap U'} \times \CC^r \to \br{U \cap U'} \times \CC^r $$
is the transition function from the trivialisation on $ U' $ to the trivialisation on $ U $. Let $ H' $ be the matrix of $ h $ with respect to $ s_1', \dots, s_r' $. Then
$$ H' = g^\intercal \cdot H \cdot \overline{g}. $$

\begin{proposition}
Let $ \pi : E \to X $ be a complex vector bundle on $ X $. Then $ E $ always admits a Hermitian metric.
\end{proposition}

Before proving the proposition, we recall the definition of a partition of the unity.

\begin{definition}
Let $ M $ be a manifold and let $ U = \cbr{U_\alpha} $ be an open covering. A \textbf{partition of unity} with respect to $ U $ is a collection of smooth functions $ f_\alpha : M \to \sbr{0, 1} $ such that
\begin{itemize}
\item $ \supp f_\alpha \subset U_\alpha $ for all $ \alpha $, in particular, $ f_\alpha = 0 $ outside $ U_\alpha $,
\item $ \sum_\alpha f_\alpha\br{x} = 1 $ for all $ x \in M $, and
\item for all $ x \in M $, there exists an open neighbourhood $ V $ of $ x $ such that $ \supp f_\alpha \cap V \ne 0 $ for only finitely many $ \alpha $.
\end{itemize}
\end{definition}

It can be shown that if $ M $ is a manifold and $ U = \cbr{U_\alpha} $ is an open cover of $ M $, then there exists a partition of the unity $ \cbr{f_\alpha} $ with respect to such a cover.

\begin{proof}
Let $ U = \cbr{U_i} $ be an open cover of open sets of $ X $, trivialising $ E $, so $ \phi_i : \pi^{-1}\br{U_i} \xrightarrow{\sim} U_i \times \CC^r $, and let $ f_i : X \to \sbr{0, 1} $ be a partition of unity with respect to $ U $. For each $ i $, consider a Hermitian metric on $ \CC^r $. Then there is a Hermitian metric $ \widetilde{h_i} $ on $ U_i \times \CC^r $. Let $ h_i $ be the Hermitian metric on $ \eval{E}_{U_i} $ induced by $ \phi_i $. Take $ h = \sum_i f_ih_i $. Check that $ h $ defines a Hermitian metric on $ X $. \footnote{Exercise}
\end{proof}

\lecture{15}{Tuesday}{11/02/20}

Let $ E \to X $ be a complex Hermitian vector bundle of rank $ r $. Fix $ p, q \ge 0 $. There exists a bilinear
$$ \function{\C^\infty\br{X, \Omega_{X, \CC}^p \otimes E} \times \C^\infty\br{X, \Omega_{X, \CC}^q \otimes E}}{\C^\infty\br{X, \Omega_{X, \CC}^{p + q}}}{\br{\sigma, \tau}}{\cbr{\sigma, \tau}}, $$
where $ \cbr{\sigma, \tau} $ is defined as follows. Let $ x \in X $, let $ s_1, \dots, s_r $ be a frame of $ E $ around $ x $, let $ H $ be the matrix associated to the Hermitian metric with respect to the frame, and let
$$ \sigma = \sum_i \sigma_i \otimes s_i, \qquad \tau = \sum_i \tau_i \otimes s_i, \qquad \sigma_i \in \C^\infty\br{X, \Omega_{X, \CC}^p}, \qquad \tau_i \in \C^\infty\br{X, \Omega_{X, \CC}^q}. $$
Then we define, around $ x $,
$$ \cbr{\sigma, \tau} = \sigma^\intercal \cdot H \cdot \overline{\tau} = \sum_{i, j = 1}^r h_{ij}\sigma_i \wedge \overline{\tau_j}. $$

This is uniquely defined, and does not depend on the frame, so it extends to $ X $. In particular $ \cbr{\sigma, \tau} $ is a smooth $ \br{p + q} $-form.

\pagebreak

\begin{definition}
Let $ E $ be a complex Hermitian vector bundle on $ X $, and let $ \nabla $ be a connection on $ E $. We say that $ \nabla $ is \textbf{Hermitian}, or \textbf{compatible with the metric}, if the Leibnitz rule holds, so we have
$$ \d\cbr{\sigma, \tau} = \cbr{\nabla\sigma, \tau} + \br{-1}^p\cbr{\sigma, \nabla\tau}, \qquad \sigma \in \C^\infty\br{X, E \otimes \Omega_{X, \CC}^p}, \qquad \tau \in \C^\infty\br{X, E \otimes \Omega_{X, \CC}^q}. $$
\end{definition}

Let $ x \in X $, and let $ s_1, \dots, s_r $ be a local frame of $ E $. Assume $ s_1, \dots, s_r $ is an orthonormal frame around $ x \in X $. Let $ \nabla $ be a connection compatible with the metric, and let $ A $ be the associated matrix with respect to $ s_1, \dots, s_r $. Gram-Schmidt is an algorithm that gives an orthonormal basis of $ E\br{x} $ for all $ x $, which is $ \C^\infty $, say $ s_1', \dots, s_r' $. Then with respect to this frame $ H = \id_r $ because $ \abr{s_i', s_j'}_x = \delta_{ij} $.

\begin{proposition}
$ A $ is anti-autodual, that is
$$ \overline{A}^\intercal = -A. $$
\end{proposition}

\begin{proof}
Let $ \sigma $ and $ \tau $ be as before, and let $ \sigma_1, \dots, \sigma_r $ and $ \tau_1, \dots, \tau_r $ be the components of $ \sigma $ and $ \tau $ with respect to the frame $ s_1, \dots, s_r $. Then $ \cbr{\sigma, \tau} = \sigma^\intercal \wedge \overline{\tau} $. Since $ \nabla $ is Hermitian, the Leibnitz rule holds, so
$$ \d\cbr{\sigma, \tau} = \d\br{\sigma^\intercal \wedge \overline{\tau}} = \d\sigma^\intercal \wedge \overline{\tau} + \br{-1}^p\sigma^\intercal \wedge \d\overline{\tau}, $$
by the usual Leibnitz rule for $ \d $. Then
$$ \cbr{\nabla\sigma, \tau} = \cbr{A \wedge \sigma + \d\sigma, \tau} = \cbr{A \wedge \sigma, \tau} + \cbr{\d\sigma, \tau} = \br{A \wedge \sigma}^\intercal \wedge \overline{\tau} + \d\sigma^\intercal \wedge \overline{\tau} = \br{-1}^p\sigma^\intercal \wedge A^\intercal \wedge \overline{\tau} + \d\sigma^\intercal \wedge \overline{\tau}, $$
and
$$ \cbr{\sigma, \nabla\tau} = \sigma^\intercal \wedge \overline{\nabla\tau} = \sigma^\intercal \wedge \br{\overline{A \wedge \tau + \d\tau}} = \sigma^\intercal \wedge \overline{A} \wedge \overline{\tau} + \sigma^\intercal \wedge \d\overline{\tau}. $$
By the Leibnitz rule,
$$ \sigma^\intercal \wedge \br{A^\intercal + \overline{A}} \wedge \overline{\tau} = 0. $$
This is true for all $ \sigma $ and $ \tau $, so $ A^\intercal + \overline{A} = 0 $.
\end{proof}

\begin{exercise*}
Let $ s_1, \dots, s_r $ be any frame, let $ H $ be the matrix given by the metric with respect to $ s_1, \dots, s_r $, and let $ A $ be the matrix given by the connection with respect to $ s_1, \dots, s_r $ where the connection is Hermitian. Then
$$ \D H = A^\intercal \cdot H + H \cdot \overline{A}, $$
where if $ H = \br{h_{ij}} $ then $ \D H = \br{\d h_{ij}} $. A hint is to do the same calculation.
\end{exercise*}

\begin{theorem}
If $ E \to X $ is a complex Hermitian vector bundle, then there exists a connection $ \nabla $ compatible with $ h $.
\end{theorem}

\subsection{Holomorphic vector bundles}

\lecture{16}{Thursday}{13/02/20}

\begin{proposition}
\label{prop:5.9}
Let $ X $ be a complex manifold, and let $ \pi : E \to X $ be a holomorphic vector bundle of rank $ r $. Then for all $ q \ge 0 $ there exists a $ \CC $-linear map
$$ \overline{\partial_E} : \C^\infty\br{X, \Omega_X^{0, q} \otimes E} \to \C^\infty\br{X, \Omega_X^{0, q + 1} \otimes E}, $$
which satisfies the Leibnitz rule and $ \overline{\partial_E} = 0 $. Moreover if $ \sigma $ is a holomorphic section of $ \Omega_X^{0, q} \otimes E $ then $ \overline{\partial_E}\sigma = 0 $.
\end{proposition}

The idea is to do it locally in a canonical way, so does not depend on the choice of the trivialisation.

\begin{proof}
Let $ x \in X $. There exists a holomorphic frame $ s_1, \dots, s_r $ of $ E $ locally around $ x $ in $ U $. Let $ \sigma \in \C^\infty\br{X, \Omega_X^{0, q} \otimes E} $. Then locally, $ \sigma \overset{U}{=} \sum_{i = 1}^r f_i \otimes s_i $ where $ f_i \in \C^\infty\br{U} $ are $ \br{0, q} $-forms locally around $ x $. We define
$$ \overline{\partial_E}x \overset{U}{=} \sum_{i = 1}^r \overline{\partial}f_i \otimes s_i \in \C^\infty\br{U, \Omega_X^{0, q + 1} \otimes E}. $$

\pagebreak

We want to show that it can be extended to $ X $. Let $ U' \subset X $ be open, let $ s_1', \dots, s_r' $ be a holomorphic frame on $ U' $ of $ E $, and let
$$ g : \br{U \cap U'} \times \CC^r \to \br{U \cap U'} \times \CC^r $$
be the transition map from the trivialisation of $ U' $ to the trivialisation of $ U $. Then $ \sigma \overset{U}{=} \sum_{i = 1}^r f_i' \otimes s_i' $, and
$$ \overline{\partial_E}x \overset{U'}{=} \sum_{i = 1}^r \overline{\partial}f_i' \otimes s_i'. $$
Since $ g $ is holomorphic, that is $ \overline{\partial}g = 0 $, this implies that $ \overline{\partial_E} $ on $ U $ coincides with $ \overline{\partial_E} $ on $ U' $. Recall for $ \nabla $ the change of frame was
$$ A' = g^{-1} \cdot \D g + g^{-1} \cdot A \cdot g, $$
so $ \overline{\partial_E} $ extends to $ X $. Let $ \sigma $ be a holomorphic section of $ \Omega_X^{0, q} \otimes E $. Then, on $ U $ if $ s_i $ and $ f_i $ are as before, then $ f_i $ are holomorphic $ \br{0, q} $-forms. Thus $ \overline{\partial}f_i = 0 $, so $ \overline{\partial_E}\sigma = 0 $.
\end{proof}

Vice versa if $ \nabla : \C^\infty\br{X, E} \to \C^\infty\br{X, \Omega_{X, \CC}^1 \otimes E} $ is a connection and $ X $ is a complex manifold, then
$$ \Omega_{X, \CC}^1 \xrightarrow{\sim} \Omega_X^{1, 0} \oplus \Omega_X^{0, 1}, \qquad \Omega_{X, \CC}^1 \otimes E = \br{\Omega_X^{1, 0} \otimes E} \oplus \br{\Omega_X^{0, 1} \otimes E}. $$
Then for all $ \sigma $,
$$ \nabla\sigma = \nabla^{1, 0}\sigma + \nabla^{0, 1}\sigma, $$
where
$$ \nabla^{1, 0} : \C^\infty\br{X, E} \to \C^\infty\br{X, \Omega_X^{1, 0} \otimes E}, \qquad \nabla^{0, 1} : \C^\infty\br{X, E} \to \C^\infty\br{X, \Omega_X^{0, 1} \otimes E}. $$

\begin{theorem}
Assume $ X $ is a complex manifold and $ E $ is a holomorphic Hermitian vector bundle of rank $ r $. Then there exists a unique connection
$$ \nabla_E : \C^\infty\br{X, E} \to \C^\infty\br{X, \Omega_{X, \CC}^1 \otimes E}, $$
such that $ \nabla_E^{0, 1} = \overline{\partial_E} $, defined in Proposition \ref{prop:5.9}, and $ \nabla_E $ is compatible with $ h $.
\end{theorem}

$ \nabla_E $ is called the \textbf{Chern connection} and $ \nabla_E^2 $ is called the \textbf{Chern curvature}.

\begin{proof}
Fix $ x \in X $, on $ U \ni x $. There exists a local holomorphic frame $ s_1, \dots, s_r $. Let $ H $ be the matrix defining the metric $ h $ on $ U $, so $ H = \br{h_{ij}} $ is an $ \br{r \times r} $-matrix for $ h_{ij} \in \C^\infty\br{U} $. Define the $ \br{r \times r} $-matrix $ \partial H = \br{\partial h_{ij}} $ for $ \partial h_{ij} \in \C^\infty\br{U, \Omega_X^{1, 0}} $. We define
$$ A = \overline{H}^{-1} \cdot \partial\overline{H}, $$
an $ \br{r \times r} $-matrix of $ 1 $-forms on $ U $. This $ A $ will be the matrix defining $ \nabla_E $. Let $ \sigma \overset{U}{=} \sum_i f_is_i \in \C^\infty\br{U, E} $ where $ f_i \in \C^\infty\br{U} $. Then
$$ \nabla_E\sigma \overset{U}{=} A \cdot f + \d f. $$
Let $ A = \br{a_{ij}} $ where by definition of $ A $, $ a_{ij} $ are $ \br{1, 0} $-forms. Thus
$$ \nabla_E^{0, 1}\sigma = A^{0, 1} \cdot f + \overline{\partial}f \overset{U}{=} \overline{\partial_E}\sigma. $$
Recall that $ \nabla $ associated to $ A $ is compatible with $ h $ if and only if $ \D H = A^\intercal \cdot H + H \cdot \overline{A} $. Since $ H $ is Hermitian, it follows that $ H^\intercal = \overline{H} $, so
$$ A^\intercal H = \br{\overline{H}^{-1} \cdot \partial\overline{H}}^\intercal \cdot H = \br{\partial\overline{H}}^\intercal \cdot \br{\overline{H}^{-1}}^\intercal \cdot H = \partial H \cdot H^{-1} \cdot H = \partial H, $$
and
$$ H \cdot \overline{A} = H \cdot \overline{\overline{H}^{-1} \cdot \partial\overline{H}} = H \cdot H^{-1} \cdot \overline{\partial} H = \overline{\partial} H. $$
Thus
$$ \D H = \br{\d h_{ij}} = \br{\partial h_{ij} + \overline{\partial}h_{ij}} = \partial H + \overline{\partial H} = A^\intercal \cdot H + H \cdot \overline{A}, $$
so on $ U $, $ \nabla_E $ is compatible with $ h $.
\end{proof}

\end{document}