\def\module{M4P57 Complex Manifolds}
\def\lecturer{Prof Paolo Cascini}
\def\term{Spring 2020}
\def\cover{
$$
\begin{tikzcd}[ampersand replacement=\&, column sep=tiny]
\displaystyle\bigoplus_{p + q = k} \H_{\BC}^{p, q}\br{X} \arrow{d}{\sim}[swap]{\ref{thm:6.39}} \& \coloneqq \& \displaystyle\bigoplus_{p + q = k} \dfrac{\ker \br{\CO{X}{p, q} \xrightarrow{\d} \CO{X}{p + 1, q + 1}}}{\im \br{\CO{X}{p - 1, q - 1} \xrightarrow{\da\dab} \CO{X}{p, q}}} \\
\displaystyle\bigoplus_{p + q = k} \H_\D^{p, q}\br{X} \arrow{d}{\sim}[swap]{\ref{thm:6.22}} \& \coloneqq \& \displaystyle\bigoplus_{p + q = k} \dfrac{\ker \br{\CO{X}{p, q} \xrightarrow{\dab} \CO{X}{p, q + 1}}}{\im \br{\CO{X}{p, q - 1} \xrightarrow{\dab} \CO{X}{p, q}}} \\
\displaystyle\bigoplus_{p + q = k} \HHH_{\Delta_{\dab}}^{p, q}\br{X} \arrow{d}{\sim}[swap]{\ref{thm:6.35}} \& \coloneqq \& \displaystyle\bigoplus_{p + q = k} \ker \br{\CO{X}{p, q} \xrightarrow{\Delta_{\dab}} \CO{X}{p, q}} \\
\HHH_\Delta^k\br{X} \arrow{d}{\sim}[swap]{\ref{thm:6.18}} \& \coloneqq \& \ker \br{\COC{X}{k} \xrightarrow{\Delta} \COC{X}{k}} \\
\H_{\d\R}^k\br{X, \CC} \arrow{d}{\sim}[swap]{\ref{thm:6.48}} \& \coloneqq \& \dfrac{\ker \br{\COC{X}{k} \xrightarrow{\d} \COC{X}{k + 1}}}{\im \br{\COC{X}{k - 1} \xrightarrow{\d} \COC{X}{k}}} \\
\displaystyle\bigoplus_{r \ge 0} \L^r\H_{\d\R}^{k - 2r}\br{X, \CC}_{\prim} \& \coloneqq \& \displaystyle\bigoplus_{r \ge 0} \L^r\ker \br{\H_{\d\R}^{k - 2r}\br{X, \CC} \xrightarrow{\L^{n - \br{k - 2r} + 1}} \H_{\d\R}^{2n - \br{k - 2r} + 2}\br{X, \CC}}
\end{tikzcd}
$$
}
\def\syllabus{Holomorphic functions and differential forms. Complex manifolds and submanifolds. Holomorphic vector bundles and the tautological line bundle. Real and complex tangent bundles. Dolbeault cohomology. Connections and curvature operators. Hermitian metrics and vector bundles. Chern connections and curvature. The first Chern class. De Rham cohomology. K\"ahler forms and manifolds. Hodge $ \star $ and Hodge-de Rham harmonic operators. Lefschetz operators and K\"ahler identities. Hodge decomposition. Bott-Chern cohomology. Lefschetz decomposition.}
\def\thm{section}

% Macros
\newcommand{\Ci}[1]{\C^\infty\br{#1}}
\newcommand{\Cic}[1]{\C_\c^\infty\br{#1}}
\newcommand{\CO}[2]{\Ci{#1, \Omega_{#1}^{#2}}}
\newcommand{\COc}[2]{\Cic{#1, \Omega_{#1}^{#2}}}
\newcommand{\COC}[2]{\Ci{#1, \Omega_{#1, \CC}^{#2}}}
\newcommand{\COCc}[2]{\Cic{#1, \Omega_{#1, \CC}^{#2}}}
\newcommand{\COE}[3]{\Ci{#1, \Omega_{#1}^{#2} \otimes #3}}
\newcommand{\COCE}[3]{\Ci{#1, \Omega_{#1, \CC}^{#2} \otimes #3}}
\newcommand{\da}{\partial}
\newcommand{\dab}{\overline{\da}}
\newcommand{\ext}{\textstyle\bigwedge}

\documentclass{article}

% Packages

\usepackage{amssymb}
\usepackage{amsthm}
\usepackage[UKenglish]{babel}
\usepackage{commath}
\usepackage{enumitem}
\usepackage{etoolbox}
\usepackage{fancyhdr}
\usepackage[margin=1in]{geometry}
\usepackage{graphicx}
\usepackage[hidelinks]{hyperref}
\usepackage[utf8]{inputenc}
\usepackage{listings}
\usepackage{mathtools}
\usepackage{stmaryrd}
\usepackage{tikz-cd}
\usepackage{csquotes}

% Formatting

\addto\captionsUKenglish{\renewcommand{\abstractname}{Syllabus}}
\delimitershortfall5pt
\ifx\thm\undefined\newtheorem{n}{}\else\newtheorem{n}{}[\thm]\fi
\newcommand\newoperator[1]{\ifcsdef{#1}{\cslet{#1}{\relax}}{}\csdef{#1}{\operatorname{#1}}}
\setlength{\parindent}{0cm}

% Environments

\theoremstyle{plain}
\newtheorem{algorithm}[n]{Algorithm}
\newtheorem*{algorithm*}{Algorithm}
\newtheorem{algorithm**}{Algorithm}
\newtheorem{conjecture}[n]{Conjecture}
\newtheorem*{conjecture*}{Conjecture}
\newtheorem{conjecture**}{Conjecture}
\newtheorem{corollary}[n]{Corollary}
\newtheorem*{corollary*}{Corollary}
\newtheorem{corollary**}{Corollary}
\newtheorem{lemma}[n]{Lemma}
\newtheorem*{lemma*}{Lemma}
\newtheorem{lemma**}{Lemma}
\newtheorem{proposition}[n]{Proposition}
\newtheorem*{proposition*}{Proposition}
\newtheorem{proposition**}{Proposition}
\newtheorem{theorem}[n]{Theorem}
\newtheorem*{theorem*}{Theorem}
\newtheorem{theorem**}{Theorem}

\theoremstyle{definition}
\newtheorem{aim}[n]{Aim}
\newtheorem*{aim*}{Aim}
\newtheorem{aim**}{Aim}
\newtheorem{axiom}[n]{Axiom}
\newtheorem*{axiom*}{Axiom}
\newtheorem{axiom**}{Axiom}
\newtheorem{condition}[n]{Condition}
\newtheorem*{condition*}{Condition}
\newtheorem{condition**}{Condition}
\newtheorem{definition}[n]{Definition}
\newtheorem*{definition*}{Definition}
\newtheorem{definition**}{Definition}
\newtheorem{example}[n]{Example}
\newtheorem*{example*}{Example}
\newtheorem{example**}{Example}
\newtheorem{exercise}[n]{Exercise}
\newtheorem*{exercise*}{Exercise}
\newtheorem{exercise**}{Exercise}
\newtheorem{fact}[n]{Fact}
\newtheorem*{fact*}{Fact}
\newtheorem{fact**}{Fact}
\newtheorem{goal}[n]{Goal}
\newtheorem*{goal*}{Goal}
\newtheorem{goal**}{Goal}
\newtheorem{law}[n]{Law}
\newtheorem*{law*}{Law}
\newtheorem{law**}{Law}
\newtheorem{plan}[n]{Plan}
\newtheorem*{plan*}{Plan}
\newtheorem{plan**}{Plan}
\newtheorem{problem}[n]{Problem}
\newtheorem*{problem*}{Problem}
\newtheorem{problem**}{Problem}
\newtheorem{question}[n]{Question}
\newtheorem*{question*}{Question}
\newtheorem{question**}{Question}
\newtheorem{warning}[n]{Warning}
\newtheorem*{warning*}{Warning}
\newtheorem{warning**}{Warning}
\newtheorem{acknowledgements}[n]{Acknowledgements}
\newtheorem*{acknowledgements*}{Acknowledgements}
\newtheorem{acknowledgements**}{Acknowledgements}
\newtheorem{annotations}[n]{Annotations}
\newtheorem*{annotations*}{Annotations}
\newtheorem{annotations**}{Annotations}
\newtheorem{assumption}[n]{Assumption}
\newtheorem*{assumption*}{Assumption}
\newtheorem{assumption**}{Assumption}
\newtheorem{conclusion}[n]{Conclusion}
\newtheorem*{conclusion*}{Conclusion}
\newtheorem{conclusion**}{Conclusion}
\newtheorem{claim}[n]{Claim}
\newtheorem*{claim*}{Claim}
\newtheorem{claim**}{Claim}
\newtheorem{notation}[n]{Notation}
\newtheorem*{notation*}{Notation}
\newtheorem{notation**}{Notation}
\newtheorem{note}[n]{Note}
\newtheorem*{note*}{Note}
\newtheorem{note**}{Note}
\newtheorem{remark}[n]{Remark}
\newtheorem*{remark*}{Remark}
\newtheorem{remark**}{Remark}

% Lectures

\newcommand{\lecture}[3]{ % Lecture
  \marginpar{
    Lecture #1 \\
    #2 \\
    #3
  }
}

% Blackboard

\renewcommand{\AA}{\mathbb{A}} % Blackboard A
\newcommand{\BB}{\mathbb{B}}   % Blackboard B
\newcommand{\CC}{\mathbb{C}}   % Blackboard C
\newcommand{\DD}{\mathbb{D}}   % Blackboard D
\newcommand{\EE}{\mathbb{E}}   % Blackboard E
\newcommand{\FF}{\mathbb{F}}   % Blackboard F
\newcommand{\GG}{\mathbb{G}}   % Blackboard G
\newcommand{\HH}{\mathbb{H}}   % Blackboard H
\newcommand{\II}{\mathbb{I}}   % Blackboard I
\newcommand{\JJ}{\mathbb{J}}   % Blackboard J
\newcommand{\KK}{\mathbb{K}}   % Blackboard K
\newcommand{\LL}{\mathbb{L}}   % Blackboard L
\newcommand{\MM}{\mathbb{M}}   % Blackboard M
\newcommand{\NN}{\mathbb{N}}   % Blackboard N
\newcommand{\OO}{\mathbb{O}}   % Blackboard O
\newcommand{\PP}{\mathbb{P}}   % Blackboard P
\newcommand{\QQ}{\mathbb{Q}}   % Blackboard Q
\newcommand{\RR}{\mathbb{R}}   % Blackboard R
\renewcommand{\SS}{\mathbb{S}} % Blackboard S
\newcommand{\TT}{\mathbb{T}}   % Blackboard T
\newcommand{\UU}{\mathbb{U}}   % Blackboard U
\newcommand{\VV}{\mathbb{V}}   % Blackboard V
\newcommand{\WW}{\mathbb{W}}   % Blackboard W
\newcommand{\XX}{\mathbb{X}}   % Blackboard X
\newcommand{\YY}{\mathbb{Y}}   % Blackboard Y
\newcommand{\ZZ}{\mathbb{Z}}   % Blackboard Z

% Brackets

\renewcommand{\eval}[1]{\left. #1 \right|}          % Evaluation
\newcommand{\br}{\del}                              % Brackets
\newcommand{\abr}[1]{\left\langle #1 \right\rangle} % Angle brackets
\newcommand{\fbr}[1]{\left\lfloor #1 \right\rfloor} % Floor brackets
\newcommand{\lbr}[1]{\left\lfloor #1 \right\rfloor} % Ceiling brackets
\newcommand{\st}{\ \middle| \ }                     % Such that

% Calligraphic

\newcommand{\AAA}{\mathcal{A}} % Calligraphic A
\newcommand{\BBB}{\mathcal{B}} % Calligraphic B
\newcommand{\CCC}{\mathcal{C}} % Calligraphic C
\newcommand{\DDD}{\mathcal{D}} % Calligraphic D
\newcommand{\EEE}{\mathcal{E}} % Calligraphic E
\newcommand{\FFF}{\mathcal{F}} % Calligraphic F
\newcommand{\GGG}{\mathcal{G}} % Calligraphic G
\newcommand{\HHH}{\mathcal{H}} % Calligraphic H
\newcommand{\III}{\mathcal{I}} % Calligraphic I
\newcommand{\JJJ}{\mathcal{J}} % Calligraphic J
\newcommand{\KKK}{\mathcal{K}} % Calligraphic K
\newcommand{\LLL}{\mathcal{L}} % Calligraphic L
\newcommand{\MMM}{\mathcal{M}} % Calligraphic M
\newcommand{\NNN}{\mathcal{N}} % Calligraphic N
\newcommand{\OOO}{\mathcal{O}} % Calligraphic O
\newcommand{\PPP}{\mathcal{P}} % Calligraphic P
\newcommand{\QQQ}{\mathcal{Q}} % Calligraphic Q
\newcommand{\RRR}{\mathcal{R}} % Calligraphic R
\newcommand{\SSS}{\mathcal{S}} % Calligraphic S
\newcommand{\TTT}{\mathcal{T}} % Calligraphic T
\newcommand{\UUU}{\mathcal{U}} % Calligraphic U
\newcommand{\VVV}{\mathcal{V}} % Calligraphic V
\newcommand{\WWW}{\mathcal{W}} % Calligraphic W
\newcommand{\XXX}{\mathcal{X}} % Calligraphic X
\newcommand{\YYY}{\mathcal{Y}} % Calligraphic Y
\newcommand{\ZZZ}{\mathcal{Z}} % Calligraphic Z

% Fraktur

\newcommand{\aaa}{\mathfrak{a}}   % Fraktur a
\newcommand{\bbb}{\mathfrak{b}}   % Fraktur b
\newcommand{\ccc}{\mathfrak{c}}   % Fraktur c
\newcommand{\ddd}{\mathfrak{d}}   % Fraktur d
\newcommand{\eee}{\mathfrak{e}}   % Fraktur e
\newcommand{\fff}{\mathfrak{f}}   % Fraktur f
\renewcommand{\ggg}{\mathfrak{g}} % Fraktur g
\newcommand{\hhh}{\mathfrak{h}}   % Fraktur h
\newcommand{\iii}{\mathfrak{i}}   % Fraktur i
\newcommand{\jjj}{\mathfrak{j}}   % Fraktur j
\newcommand{\kkk}{\mathfrak{k}}   % Fraktur k
\renewcommand{\lll}{\mathfrak{l}} % Fraktur l
\newcommand{\mmm}{\mathfrak{m}}   % Fraktur m
\newcommand{\nnn}{\mathfrak{n}}   % Fraktur n
\newcommand{\ooo}{\mathfrak{o}}   % Fraktur o
\newcommand{\ppp}{\mathfrak{p}}   % Fraktur p
\newcommand{\qqq}{\mathfrak{q}}   % Fraktur q
\newcommand{\rrr}{\mathfrak{r}}   % Fraktur r
\newcommand{\sss}{\mathfrak{s}}   % Fraktur s
\newcommand{\ttt}{\mathfrak{t}}   % Fraktur t
\newcommand{\uuu}{\mathfrak{u}}   % Fraktur u
\newcommand{\vvv}{\mathfrak{v}}   % Fraktur v
\newcommand{\www}{\mathfrak{w}}   % Fraktur w
\newcommand{\xxx}{\mathfrak{x}}   % Fraktur x
\newcommand{\yyy}{\mathfrak{y}}   % Fraktur y
\newcommand{\zzz}{\mathfrak{z}}   % Fraktur z

% Geometry

\newcommand{\CP}{\mathbb{CP}}                                              % Complex projective space
\newcommand{\iintd}[4]{\iint_{#1} \, #2 \, \dif #3 \, \dif #4}             % Double integral
\newcommand{\RP}{\mathbb{RP}}                                              % Real projective space
\newcommand{\intd}[4]{\int_{#1}^{#2} \, #3 \, \dif #4}                     % Single integral
\newcommand{\iiintd}[5]{\iint_{#1} \, #2 \, \dif #3 \, \dif #4 \, \dif #5} % Triple integral

% Logic

\newcommand{\iffb}[2]{\br{#1 \leftrightarrow #2}} % Biconditional
\newcommand{\andb}[2]{\br{#1 \land #2}}           % Conjunction
\newcommand{\orb}[2]{\br{#1 \lor #2}}             % Disjunction
\newcommand{\nib}[2]{\br{#1 \notin #2}}           % Element of
\newcommand{\eqb}[2]{\br{#1 = #2}}                % Equal to
\newcommand{\teb}[1]{\br{\exists #1}}             % Existential quantifier
\newcommand{\impb}[2]{\br{#1 \rightarrow #2}}     % Implication
\newcommand{\ltb}[2]{\br{#1 < #2}}                % Less than
\newcommand{\leb}[2]{\br{#1 \le #2}}              % Less than or equal to
\newcommand{\notb}[1]{\br{\neg #1}}               % Negation
\newcommand{\inb}[2]{\br{#1 \in #2}}              % Not element of
\newcommand{\neb}[2]{\br{#1 \ne #2}}              % Not equal to
\newcommand{\subb}[2]{\br{#1 \subseteq #2}}       % Subset
\newcommand{\fab}[1]{\br{\forall #1}}             % Universal quantifier

% Maps

\newcommand{\bijection}[7][]{    % Bijection
  \ifx &#1&
    \begin{array}{rcl}
      #2 & \longleftrightarrow & #3 \\
      #4 & \longmapsto         & #5 \\
      #6 & \longmapsfrom       & #7
    \end{array}
  \else
    \begin{array}{ccrcl}
      #1 & : & #2 & \longrightarrow & #3 \\
         &   & #4 & \longmapsto     & #5 \\
         &   & #6 & \longmapsfrom   & #7
    \end{array}
  \fi
}
\newcommand{\birational}[7][]{   % Birational map
  \ifx &#1&
    \begin{array}{rcl}
      #2 & \dashrightarrow & #3 \\
      #4 & \longmapsto     & #5 \\
      #6 & \longmapsfrom   & #7
    \end{array}
  \else
    \begin{array}{ccrcl}
      #1 & : & #2 & \dashrightarrow & #3 \\
         &   & #4 & \longmapsto     & #5 \\
         &   & #6 & \longmapsfrom   & #7
    \end{array}
  \fi
}
\newcommand{\correspondence}[2]{ % Correspondence
  \cbr{
    \begin{array}{c}
      #1
    \end{array}
  }
  \qquad
  \leftrightsquigarrow
  \qquad
  \cbr{
    \begin{array}{c}
      #2
    \end{array}
  }
}
\newcommand{\function}[5][]{     % Function
  \ifx &#1&
    \begin{array}{rcl}
      #2 & \longrightarrow & #3 \\
      #4 & \longmapsto     & #5
    \end{array}
  \else
    \begin{array}{ccrcl}
      #1 & : & #2 & \longrightarrow & #3 \\
         &   & #4 & \longmapsto     & #5
    \end{array}
  \fi
}
\newcommand{\functions}[7][]{    % Functions
  \ifx &#1&
    \begin{array}{rcl}
      #2 & \longrightarrow & #3 \\
      #4 & \longmapsto     & #5 \\
      #6 & \longmapsto     & #7
    \end{array}
  \else
    \begin{array}{ccrcl}
      #1 & : & #2 & \longrightarrow & #3 \\
         &   & #4 & \longmapsto     & #5 \\
         &   & #6 & \longmapsto     & #7
    \end{array}
  \fi
}
\newcommand{\rational}[5][]{     % Rational map
  \ifx &#1&
    \begin{array}{rcl}
      #2 & \dashrightarrow & #3 \\
      #4 & \longmapsto     & #5
    \end{array}
  \else
    \begin{array}{ccrcl}
      #1 & : & #2 & \dashrightarrow & #3 \\
         &   & #4 & \longmapsto     & #5
    \end{array}
  \fi
}

% Matrices

\newcommand{\onebytwo}[2]{      % One by two matrix
  \begin{pmatrix}
    #1 & #2
  \end{pmatrix}
}
\newcommand{\onebythree}[3]{    % One by three matrix
  \begin{pmatrix}
    #1 & #2 & #3
  \end{pmatrix}
}
\newcommand{\twobyone}[2]{      % Two by one matrix
  \begin{pmatrix}
    #1 \\
    #2
  \end{pmatrix}
}
\newcommand{\twobytwo}[4]{      % Two by two matrix
  \begin{pmatrix}
    #1 & #2 \\
    #3 & #4
  \end{pmatrix}
}
\newcommand{\threebyone}[3]{    % Three by one matrix
  \begin{pmatrix}
    #1 \\
    #2 \\
    #3
  \end{pmatrix}
}
\newcommand{\threebythree}[9]{  % Three by three matrix
  \begin{pmatrix}
    #1 & #2 & #3 \\
    #4 & #5 & #6 \\
    #7 & #8 & #9
  \end{pmatrix}
}
\newcommand{\twobytwosmall}[4]{ % Two by two small matrix
  \begin{psmallmatrix}
    #1 & #2 \\
    #3 & #4
  \end{psmallmatrix}
}

% Number theory

\renewcommand{\symbol}[2]{\br{\tfrac{#1}{#2}}} % Power residue symbol
\newcommand{\unit}[1]{\br{\ZZ / #1\ZZ}^\times} % Unit group

% Operators

\newoperator{ab}    % Abelian
\newoperator{AG}    % Affine geometry
\newoperator{alg}   % Algebraic
\newoperator{Ann}   % Annihilator
\newoperator{area}  % Area
\newoperator{Aut}   % Automorphism
\newoperator{card}  % Cardinality
\newoperator{ch}    % Characteristic
\newoperator{Cl}    % Class
\newoperator{Coker} % Cokernel
\newoperator{col}   % Column
\newoperator{Corr}  % Correspondence
\newoperator{diam}  % Diameter
\newoperator{Disc}  % Discriminant
\newoperator{dom}   % Domain
\newoperator{Eig}   % Eigenvalue
\newoperator{Em}    % Embedding
\newoperator{End}   % Endomorphism
\newoperator{fin}   % Finite
\newoperator{Fix}   % Fixed
\newoperator{Frac}  % Fraction
\newoperator{Frob}  % Frobenius
\newoperator{Fun}   % Function
\newoperator{Gal}   % Galois
\newoperator{GL}    % General linear
\newoperator{Ham}   % Hamming
\newoperator{Homeo} % Homeomorphism
\newoperator{Hom}   % Homomorphism
\newoperator{id}    % Identity
\newoperator{Im}    % Image
\newoperator{Ind}   % Index
\newoperator{Ker}   % Kernel
\newoperator{lcm}   % Least common multiple
\newoperator{Mat}   % Matrix
\newoperator{mult}  % Multiplicity
\newoperator{new}   % New
\newoperator{Nm}    % Norm
\newoperator{old}   % Old
\newoperator{op}    % Opposite
\newoperator{ord}   % Order
\newoperator{Pay}   % Payley
\newoperator{PG}    % Projective geometry
\newoperator{PGL}   % Projective general linear
\newoperator{PSL}   % Projective special linear
\newoperator{rad}   % Radical
\newoperator{ran}   % Range
\newoperator{Res}   % Residue
\newoperator{rk}    % Rank
\newoperator{Re}    % Real
\newoperator{row}   % Row
\newoperator{sgn}   % Sign
\newoperator{Sing}  % Singular
\newoperator{SK}    % Skeleton
\newoperator{sp}    % Span
\newoperator{SL}    % Special linear
\newoperator{SO}    % Special orthogonal
\newoperator{Spec}  % Spectrum
\newoperator{Stab}  % Stabiliser
\newoperator{star}  % Star
\newoperator{srg}   % Strongly regular graph
\newoperator{supp}  % Support
\newoperator{Sym}   % Symmetric
\newoperator{tors}  % Torsion
\newoperator{Tr}    % Trace
\newoperator{vol}   % Volume
\newoperator{wt}    % Weight

% Roman

\newcommand{\A}{\mathrm{A}}   % Roman A
\newcommand{\B}{\mathrm{B}}   % Roman B
\newcommand{\C}{\mathrm{C}}   % Roman C
\newcommand{\D}{\mathrm{D}}   % Roman D
\newcommand{\E}{\mathrm{E}}   % Roman E
\newcommand{\F}{\mathrm{F}}   % Roman F
\newcommand{\G}{\mathrm{G}}   % Roman G
\renewcommand{\H}{\mathrm{H}} % Roman H
\newcommand{\I}{\mathrm{I}}   % Roman I
\newcommand{\J}{\mathrm{J}}   % Roman J
\newcommand{\K}{\mathrm{K}}   % Roman K
\renewcommand{\L}{\mathrm{L}} % Roman L
\newcommand{\M}{\mathrm{M}}   % Roman M
\newcommand{\N}{\mathrm{N}}   % Roman N
\renewcommand{\O}{\mathrm{O}} % Roman O
\renewcommand{\P}{\mathrm{P}} % Roman P
\newcommand{\Q}{\mathrm{Q}}   % Roman Q
\newcommand{\R}{\mathrm{R}}   % Roman R
\renewcommand{\S}{\mathrm{S}} % Roman S
\newcommand{\T}{\mathrm{T}}   % Roman T
\newcommand{\U}{\mathrm{U}}   % Roman U
\newcommand{\V}{\mathrm{V}}   % Roman V
\newcommand{\W}{\mathrm{W}}   % Roman W
\newcommand{\X}{\mathrm{X}}   % Roman X
\newcommand{\Y}{\mathrm{Y}}   % Roman Y
\newcommand{\Z}{\mathrm{Z}}   % Roman Z

\renewcommand{\a}{\mathrm{a}} % Roman a
\renewcommand{\b}{\mathrm{b}} % Roman b
\renewcommand{\c}{\mathrm{c}} % Roman c
\renewcommand{\d}{\mathrm{d}} % Roman d
\newcommand{\e}{\mathrm{e}}   % Roman e
\newcommand{\f}{\mathrm{f}}   % Roman f
\newcommand{\g}{\mathrm{g}}   % Roman g
\newcommand{\h}{\mathrm{h}}   % Roman h
\renewcommand{\i}{\mathrm{i}} % Roman i
\renewcommand{\j}{\mathrm{j}} % Roman j
\renewcommand{\k}{\mathrm{k}} % Roman k
\renewcommand{\l}{\mathrm{l}} % Roman l
\newcommand{\m}{\mathrm{m}}   % Roman m
\renewcommand{\n}{\mathrm{n}} % Roman n
\renewcommand{\o}{\mathrm{o}} % Roman o
\newcommand{\p}{\mathrm{p}}   % Roman p
\newcommand{\q}{\mathrm{q}}   % Roman q
\renewcommand{\r}{\mathrm{r}} % Roman r
\newcommand{\s}{\mathrm{s}}   % Roman s
\renewcommand{\t}{\mathrm{t}} % Roman t
\renewcommand{\u}{\mathrm{u}} % Roman u
\renewcommand{\v}{\mathrm{v}} % Roman v
\newcommand{\w}{\mathrm{w}}   % Roman w
\newcommand{\x}{\mathrm{x}}   % Roman x
\newcommand{\y}{\mathrm{y}}   % Roman y
\newcommand{\z}{\mathrm{z}}   % Roman z

% Tikz

\tikzset{
  arrow symbol/.style={"#1" description, allow upside down, auto=false, draw=none, sloped},
  subset/.style={arrow symbol={\subset}},
  cong/.style={arrow symbol={\cong}}
}

% Fancy header

\pagestyle{fancy}
\lhead{\module}
\rhead{\nouppercase{\leftmark}}

% Make title

\title{\module}
\author{Lectured by \lecturer \\ Typed by David Kurniadi Angdinata}
\date{\term}

\begin{document}

% Title page
\maketitle
\cover
\vfill
\begin{abstract}
\noindent\syllabus
\end{abstract}

\pagebreak

% Contents page
\tableofcontents

\pagebreak

% Document page
\setcounter{section}{-1}

\setcounter{section}{0}

\section{Introduction}

\lecture{1}{Thursday}{09/01/20}

The following are references.
\begin{itemize}
\item O Biquard and A H\"oring, K\"ahler geometry and Hodge theory, 2008.
\item J P Demailly, Complex analytic and differential geometry, 2012.
\item C Voisin, Hodge theory and complex algebraic geometry, 2002.
\item R O Wells, Differential analysis on complex manifolds, 1973.
\item A Gathmann, Algebraic geometry, 2002
\item P Griffiths and J Harris, Principles of algebraic geometry, 1978.
\end{itemize}

Complex manifolds are manifolds over $ \CC^n $.

\begin{example}
$ \CC^1 $ is a complex manifold. Any open $ U \subset \CC^n $ is a complex manifold.
\end{example}

\begin{example}
The \textbf{sphere} $ \S^2 \subset \RR^3 $ is a complex manifold by $ \S^2 \cong \CC \cup \cbr{\infty} = \PP_\CC^1 = \CC\PP^1 $. More in general $ \PP_\CC^n $ is a complex manifold for all $ n $.
\end{example}

\begin{example}
The \textbf{torus} $ \S^1 \times \S^1 = \RR^2 / \ZZ^2 = \CC / \ZZ^2 $ is a complex manifold. More in general a $ 2n $-dimensional torus $ \CC^n / \Lambda $ for a lattice $ \Lambda \cong \ZZ^{2n} $ is a complex manifold.
\end{example}

\begin{example}
Compact Riemannian surfaces of genus $ g > 1 $, called \textbf{hyperbolics}, are all complex manifolds.
\end{example}

\begin{example}
\label{eg:1.5}
Let $ f : \CC \to \CC $ be holomorphic. The graph of $ f $,
$$ \Gamma_f = \cbr{\br{x, f\br{x}} \st x \in \CC} \subset \CC \times \CC, $$
is a complex manifold. From $ \Gamma_f $ we can recover $ f $, by
$$ f\br{x} = \q\br{\p^{-1}\br{x} \cap \Gamma_f}, $$
where $ \p, \q : \CC^2 \to \CC $ are the projections to the first and second factors. This allows us to define $ f^{-1} $. Assume $ f $ is bijective. Define
$$ \function[\tau]{\CC^2}{\CC^2}{\br{x, y}}{\br{y, x}}. $$
Define
$$ \Gamma_{f^{-1}} = \tau\br{\Gamma_f}. $$
Then $ f^{-1} $ is the function induced by $ \Gamma_{f^{-1}} $. This makes sense even if $ f $ is not bijective. Then we get a multivalued function, such as $ \log z $ as the inverse of $ \exp z $.
\end{example}

\begin{example}
Generalising Example \ref{eg:1.5}, we can consider two complex manifolds $ M $ and $ N $ and we can consider holomorphisms $ f : M \to N $. Given $ M $,
$$ \Aut M = \cbr{f : M \to M \ \text{holomorphic bijective and} \ f^{-1} \ \text{holomorphic}}. $$
If $ M = \CC $, there are lots of $ \C^\infty $-functions $ \CC \to \CC $ but the automorphisms of $ \CC $ are just affine linear maps. If $ M = \CC / \ZZ^2 $, then $ \Aut M $ is interesting.
\end{example}

\begin{example}
Algebraic geometry is the zeroes of polynomials. That is, fix $ m $, and take polynomials $ f_1, \dots, f_k $ in $ m $ variables. Define
$$ M = \cbr{\br{x_1, \dots, x_m} \in \CC^m \st f_1\br{x_1, \dots, x_m} = \dots = f_k\br{x_1, \dots, x_m} = 0}. $$
Then $ M $ is called an \textbf{algebraic variety}. If $ M $ is smooth then $ M $ is a complex manifold. Fix $ m $, take homogeneous polynomials $ F_1, \dots, F_k $ in $ m + 1 $ variables, where $ F $ is \textbf{homogeneous} if it is the sum of monomials of the same degree. Consider
$$ N = \cbr{\br{x_0, \dots, x_m} \in \PP_\CC^m \st F_1\br{x_0, \dots, x_m} = \dots = F_k\br{x_0, \dots, x_m} = 0}. $$
Then $ N $ is called a \textbf{projective variety}. If $ N $ is smooth then $ N $ is a complex manifold.
\end{example}

\pagebreak

The idea is if $ M $ is a differentiable manifold, then $ M $ contains lots of submanifolds $ N $. This is not true for complex manifolds. There exist complex manifolds without any proper complex submanifolds, which is not true for projective varieties. The following are questions.
\begin{itemize}
\item What can we say about the topology of complex manifolds? For example, what is $ \pi_1\br{M} $? What is the cohomology of $ M $?
\item Assume that $ M $ and $ N $ are complex manifolds which are diffeomorphic. Are they also isomorphic, so there exists a biholomorphism $ M \to N $?
\end{itemize}
What is next?
\begin{itemize}
\item Hodge decomposition theorem. Understand the cohomology of $ M $ by using the complex structure.
\item Kodaira embedding theorem. Understand when a compact complex manifold is projective.
\end{itemize}

\begin{note*}
If $ M \subset \PP_\CC^m $ is a compact complex manifold then $ M $ is projective.
\end{note*}

\begin{example*}
Let $ M = \Gamma_{\exp} $ for $ \exp : \CC \to \CC $. Then $ M \subset \CC^2 $ but it is not algebraic.
\end{example*}

\pagebreak

\section{Local theory}

\subsection{Holomorphic functions in several variables}

\lecture{2}{Thursday}{09/01/20}

\begin{notation}
Given $ z_0 \in \CC $ and $ r > 0 $, the \textbf{disc} is
$$ \D\br{z_0, r} = \cbr{z \in \CC \st \abs{z - z_0} < r}, $$
and $ \da\D\br{z_0, r} $ is the boundary of $ \D\br{z_0, r} $.
\end{notation}

\begin{definition}
Let $ U \subset \CC $, and let $ f : U \to \CC $ be a function. Then $ f $ is \textbf{holomorphic at $ z_0 \in U $} if
$$ \lim_{z \to z_0} \dfrac{f\br{z} - f\br{z_0}}{z - z_0} $$
exists.
\end{definition}

\begin{theorem}[Cauchy]
\label{thm:2.3}
Let $ U \subset \CC $ be open, let $ f $ be holomorphic on $ U $, and let $ z_0 \in U $. Assume that if $ D = \D\br{z_0, r} \subset U $ then $ \overline{D} \subset U $. Then
$$ f\br{z_0} = \dfrac{1}{2\pi i}\int_{\da D} \dfrac{f\br{z}}{z - z_0}\d z. $$
\end{theorem}

\begin{notation}
Fix $ z_0 = \br{z_{01}, \dots, z_{0n}} \in \CC^n $ and $ R = \br{r_1, \dots, r_n} \in \RR_{> 0}^n $. Then the \textbf{polydisc of polyradius $ R $} is
$$ \D\br{z_0, R} = \cbr{z = \br{z_1, \dots, z_n} \in \CC^n \st \forall i, \ \abs{z_i - z_{0i}} < r_i}. $$
\end{notation}

\begin{definition}
Let $ U \subset \CC^n $ be open, let $ f : U \to \CC $ be a continuous function, and let $ z = \br{z_1, \dots, z_n} \in \CC^n $. Then $ f $ is \textbf{holomorphic at $ z $}, if assuming that $ D = \D\br{z, R} \subset U $ for some $ R = \br{r_1, \dots, r_n} $ then
$$ f\br{z_1, \dots, z_{i - 1}, \cdot, z_{i + 1}, \dots, z_n} : \D\br{z_i, r_i} \to \CC $$
is holomorphic for all $ i $.
\end{definition}

\begin{example}
Any convergent power series in $ n $ variables is holomorphic.
\end{example}

The opposite is also true.

\begin{theorem}[Cauchy]
\label{thm:2.7}
Let $ U \subset \CC^n $ be an open set, let $ f : U \to \CC $ be holomorphic, and let $ z = \br{z_1, \dots, z_n} \in U $. Assume that if $ D = \D\br{z_0, R} $ for some $ R = \br{r_1, \dots, r_n} $ then $ \overline{D} \subset U $. If $ z' = \br{z_1', \dots, z_n'} \in D $ then
$$ f\br{z'} = \dfrac{1}{\br{2\pi i}^n}\int_{\da\D\br{z_1, r_1}} \dots \int_{\da\D\br{z_n, r_n}} \dfrac{f\br{z}}{\br{z - z_1'} \dots \br{z - z_n'}}\d z_n \dots \d z_1. $$
\end{theorem}

\begin{proof}
Use induction on $ n $ and Cauchy theorem at each step.
\end{proof}

\begin{corollary}
Let $ U \subset \CC^n $ be open, let $ f : U \to \CC $ be holomorphic, and let $ z = \br{z_1, \dots, z_n} \in U $. Then there exists $ D = \D\br{z, R} \subset U $ for some $ R = \br{r_1, \dots, r_n} $ and there exists
$$ p\br{w} = \sum_{m_1, \dots, m_n \ge 0} a_{m_1, \dots, m_n} \br{w_1 - z_1}^{m_1} \dots \br{w_n - z_n}^{m_n}, $$
such that $ p $ is convergent on $ D $ and $ f\br{w} = p\br{w} $ inside $ D $.
\end{corollary}

\begin{proof}
The idea is to use Theorem \ref{thm:2.7} and $ 1 / \br{1 - w} = \sum_{k \ge 0} w^k $.
\end{proof}

\begin{definition}
Let $ U \subset \CC^n $ be open. Then $ f : U \to \CC^m $ is \textbf{holomorphic} if $ f_i = \p_i \circ f $ is holomorphic for any $ i = 1, \dots, m $ where $ \p_i : \CC^m \to \CC $ is the $ i $-th projection, so $ f = \br{f_1, \dots, f_m} $.
\end{definition}

\begin{fact*}
If $ f : U \to \CC^m $ is holomorphic and $ g : V \to \CC^p $ is holomorphic where $ V \supset f\br{U} $ then $ g \circ f $ is holomorphic.
\end{fact*}

\begin{definition}
Let $ U \subset \CC^n $ be open. A holomorphic function $ f : U \to \CC^m $ is \textbf{biholomorphic at $ z_0 \in U $} if there exists an open neighbourhood $ V \subset U $ of $ z_0 $ such that $ f : V \to f\br{V} $ is bijective and $ f^{-1} : f\br{V} \to V $ is holomorphic. Then $ f $ is \textbf{biholomorphic} if $ f $ is bijective and $ f $ is biholomorphic at any point.
\end{definition}

\begin{note*}
$ f\br{V} $ is automatically open in $ \CC^m $ if $ m = n $.
\end{note*}

\pagebreak

\begin{example}
Let $ \Phi : \CC^n \to \CC^n $ be linear such that $ \det \Phi \ne 0 $. Then $ \Phi $ is a biholomorphism.
\end{example}

\begin{example}
Let $ U = \CC \setminus \cbr{0} $ and
$$ \function[f]{U}{U}{z}{z^2}. $$
Check that $ f $ is biholomorphic at any point of $ U $ but $ f $ is not biholomorphic.
\end{example}

\begin{remark*}
$ \CC^n \cong \RR^{2n} $ and $ \CC^m \cong \RR^{2m} $. Then a holomorphic $ f : U \subset \CC^n \to \CC^m $ is also a diffeomorphism $ U \subset \RR^{2n} \to \RR^{2m} $.
\end{remark*}

\begin{theorem}[Hartogs]
Let $ n \ge 2 $, let $ \epsilon = \br{\epsilon_1, \dots, \epsilon_n} $ and $ \delta = \br{\delta_1, \dots, \delta_n} $ such that $ \epsilon_i > \delta_i > 0 $, let $ U = \D\br{0, \epsilon} \setminus \overline{\D\br{0, \delta}} $, and let $ f : U \to \CC^m $ be holomorphic. Then there exists a holomorphic $ \overline{f} : \D\br{0, \epsilon} \to \CC^m $ such that $ \eval{\overline{f}}_U = f $.
\end{theorem}

\begin{example*}
Hartogs' theorem is false for $ n = 1 $. If $ f\br{z} = 1 / z $, for all $ \epsilon > \delta > 0 $, then $ f $ cannot be extended.
\end{example*}

\subsection{Cauchy formula in one variable}

\lecture{3}{Tuesday}{14/01/20}

Let $ \omega = x + iy \in \CC $ for $ x, y \in \RR $, and let $ f : U \to \CC $ be $ \C^\infty $ for some $ U \subset \CC $. Recall that
$$ \dpd{}{\omega}f = \dfrac{1}{2}\br{\dpd{}{x} - i\dpd{}{y}}f, \qquad \dpd{}{\overline{\omega}}f = \dfrac{1}{2}\br{\dpd{}{x} + i\dpd{}{y}}f. $$
Recall that $ f $ is holomorphic if and only if $ \tpd{}{\overline{\omega}}f = 0 $ on $ U $. More in general, let $ U \subset \CC^n $ be open, let $ z_i = x_i + iy_i $, and let $ f : U \to \CC $ be a $ \C^\infty $-function. Then $ f $ is holomorphic if and only if $ \tpd{}{\overline{z_i}}f = 0 $ for all $ i = 1, \dots, n $. Let $ \omega \in \CC $. Since $ \d x \wedge \d y = -\d y \wedge \d x $, let
$$ \d A = \dfrac{i}{2}\d\omega \wedge \d\overline{\omega} = \dfrac{i}{2} \br{\d x + i\d y} \wedge \br{\d x - i\d y} = \d x \wedge \d y, $$
which is the Lebesgue measure on $ \RR^2 \cong \CC $.

\begin{proposition}
Let $ f : U \to \CC $ for $ U \subset \CC $ be a $ \C^\infty $-function, and let $ D = \D\br{z, r} $ such that $ \overline{D} \subset U $. Then
$$ f\br{z} = \dfrac{1}{2\pi i}\int_{\da D} \dfrac{f}{\omega - z}\d\omega - \dfrac{1}{\pi}\int_D \dfrac{1}{\omega - z}\dpd{}{\overline{\omega}}f\d A. $$
\end{proposition}

\begin{proof}
Assume $ z = 0 $. Recall that $ f\br{\omega} = 1 / \omega $ is locally integrable around zero, so
$$ \int_D \dfrac{1}{\omega}\dpd{}{\overline{\omega}}f\d A = \lim_{\epsilon \to 0} \int_{D \setminus \D\br{0, \epsilon}} \dfrac{1}{\omega}\dpd{}{\overline{\omega}}f\d A. $$
Away from zero
\begin{align*}
\d\br{\dfrac{f}{\omega}\d\omega}
& = \dfrac{1}{\omega}\d f \wedge \d\omega + f\d\br{\dfrac{1}{\omega}} \wedge \d\omega
= \dfrac{1}{\omega} \br{\dpd{}{\omega}f\d\omega + \dpd{}{\overline{\omega}}f\d\overline{\omega}} \wedge \d\omega + f\dpd{}{\omega}\br{\dfrac{1}{\omega}}\d\omega \wedge \d\omega \\
& = \dfrac{1}{\omega}\dpd{}{\overline{\omega}}f\d\overline{\omega} \wedge \d\omega
= \dfrac{2i}{\omega}\dpd{}{\overline{\omega}}f\d A.
\end{align*}
Then
\begin{align*}
\dfrac{1}{\pi}\int_D \dfrac{1}{\omega}\dpd{}{\overline{\omega}}f\d A
& = \dfrac{1}{\pi}\lim_{\epsilon \to 0} \int_{D \setminus \D\br{0, \epsilon}} \dfrac{1}{\omega}\dpd{}{\overline{\omega}}f\d A \\
& = \dfrac{1}{2\pi i}\lim_{\epsilon \to 0} \int_{D \setminus \D\br{0, \epsilon}}\d\br{\dfrac{f}{\omega}\d\omega} & \dfrac{1}{\omega}\dpd{}{\overline{\omega}}f\d A = \dfrac{1}{2i}\d\br{\dfrac{f}{\omega}\d\omega} \\
& = \dfrac{1}{2\pi i}\lim_{\epsilon \to 0} \br{\int_{\da D} \dfrac{f}{\omega}\d\omega - \int_{\da\D\br{0, \epsilon}} \dfrac{f}{\omega}\d\omega} & \text{Stokes' theorem} \\
& = \dfrac{1}{2\pi i}\br{\int_{\da D} \dfrac{f}{\omega}\d\omega - 2\pi if\br{0}} & \lim_{\epsilon \to 0} \int_{\da\D\br{0, \epsilon}} \dfrac{1}{\omega}\d\omega = 2\pi i.
\end{align*}
\end{proof}

If $ f $ is holomorphic, then $ \tpd{}{\overline{\omega}}f = 0 $, which implies Theorem \ref{thm:2.3}.

\pagebreak

\subsection{Rank theorem}

Let $ U \subset \CC^n $ be open, and let $ f : U \to \CC^m $ be holomorphic. Then the \textbf{Jacobian} is
$$ \J_f = \br{\dpd{}{z_i}f_j\br{z}}, $$
where $ f_j = \p_j \circ f $ and $ \p_j : \CC^m \to \CC $ is the $ j $-th projection.

\begin{exercise*}
Show that the real Jacobian, which is $ 2n \times 2n $, has non-negative determinants.
\end{exercise*}

\begin{theorem}[Rank theorem]
\label{thm:2.15}
Let $ z \in U $ such that $ r = \rk \J_f\br{z'} $ is constant around $ z $. Then there exist open $ z \in V \subset U \subset \CC^n $ and $ f\br{z} \in W \subset f\br{U} \subset \CC^m $ such that $ \phi : \D\br{0, 1}^n \to V $ and $ \psi : \D\br{0, 1}^m \to W $ are biholomorphisms such that
$$ \function[\eta = \psi^{-1} \circ f \circ \phi]{\D\br{0, 1}^n}{\D\br{0, 1}^m}{\br{z_1, \dots, z_n}}{\br{z_1, \dots, z_r, 0, \dots, 0}}, $$
so
$$
\begin{tikzcd}
\CC^n \supset U & V \arrow[subset]{l} \ni z \arrow{r}{f} & f\br{z} \in W \arrow[subset]{r} & f\br{U} \subset \CC^m \\
& \D\br{0, 1}^n \arrow{u}{\phi} \arrow{r}[swap]{\eta} & \D\br{0, 1}^m \arrow{u}[swap]{\psi} &
\end{tikzcd}.
$$
\end{theorem}

\begin{corollary}[Inverse function theorem]
Let $ f : U \to \CC^n $ be holomorphic for $ U \subset \CC^n $, and let $ z \in U $ such that $ \det \J_f\br{z} \ne 0 $. Then $ f $ is a biholomorphism at $ z $.
\end{corollary}

\begin{proof}
$ \det \J_f\br{z} \ne 0 $ if and only if $ \rk \J_f\br{z} = n $, so $ \rk \J_f\br{z'} = n $ around $ z $, since $ \det \J_f\br{z} $ is a continuous function. Let $ \phi $ and $ \psi $ as in Theorem \ref{thm:2.15}. Then $ \eta = \psi^{-1} \circ f \circ \phi = \id $, so on $ V $, $ f = \psi \circ \phi^{-1} $ is a composition of biholomorphisms, which is a biholomorphism.
\end{proof}

\begin{remark}
Let $ f : U \to \CC^n $ for $ U \subset \CC^n $. Then $ \det \J_f\br{z} $ is a holomorphism, so
$$ Z = \cbr{z \in U \st \det \J_f\br{z} = 0} $$
is closed.
\end{remark}

\subsection{Holomorphic differential forms}

Let $ U \subset \CC^n $ be open.

\begin{definition}
A \textbf{holomorphic vector field} on $ U $ is an expression
$$ X = \sum_i a_i\dpd{}{z_i}, $$
where $ a_i $ are holomorphic functions on $ U $.
\end{definition}

For all $ x \in U $, the \textbf{tangent space} is
$$ \T_xU = \abr{\dpd{}{x_1}, \dots, \dpd{}{x_n}} \cong \CC^n. $$
If $ x \in U $, then $ X\br{x} \in \T_xU $.

\begin{notation}
Let
$$ \H^0\br{U, \OOO_U} = \cbr{\text{holomorphic functions} \ f : U \to \CC}, \qquad \H^0\br{U, \T_U} = \cbr{\text{holomorphic vector fields on} \ U}. $$
\end{notation}

\begin{remark*}
$ R = \H^0\br{U, \OOO_U} $ is a ring and $ M = \H^0\br{U, \T_U} $ is a module over $ R $. That is, if $ X \in \H^0\br{U, \T_U} $ and $ f \in \H^0\br{U, \OOO_U} $, then $ fX \in \H^0\br{U, \T_U} $.
\end{remark*}

\pagebreak

\begin{definition}
Let $ R $ be a ring and $ M $ be an $ R $-module for $ p \ge 1 $. The \textbf{$ p $-th exterior power} $ \ext^pM $ of $ M $ is the $ R $-module $ M^{\otimes p} $ with the relations
$$ m_1 \otimes \dots \otimes m_p - \epsilon\br{\sigma}m_{\sigma\br{1}} \otimes \dots \otimes m_{\sigma\br{p}}, \qquad m_1, \dots, m_p \in M, \qquad \sigma \in \SSS_p, $$
where $ \epsilon\br{\sigma} = \br{-1}^m $ is the signature of $ \sigma $ and $ m $ is the number of transpositions defining $ \sigma $. Then $ M^* = \Hom_R\br{M, R} $ is the \textbf{dual} of $ M $ as an $ R $-module.
\end{definition}

\lecture{4}{Thursday}{16/01/20}

Let $ R = \H^0\br{U, \OOO_U} $ and $ M = \H^0\br{U, \T_U} $.

\begin{definition}
Let $ p > 0 $. We define a \textbf{holomorphic $ p $-form}, as an element of
$$ \H^0\br{U, \Omega_U^p} = \ext^pM^*. $$
If $ p = 0 $, by convention a \textbf{holomorphic $ 0 $-form} is just an element in $ R $.
\end{definition}

Let $ \br{z_1, \dots, z_n} $ be coordinates for $ U $. Recall $ \eta \in M $ is given by $ \eta = \sum_i a_i\tpd{}{z_i} $ for holomorphic functions $ a_i \in R $. Then $ \omega \in M^* $ is given by the expression
$$ \sum_i b_i\d z_i, \qquad b_i \in R, \qquad \d z_i\br{\dpd{}{z_j}} = \delta_{ij}. $$
More in general $ \omega \in \H^0\br{U, \Omega_U^p} $ is given by
$$ \omega = \sum_{\abs{I} = p} f_I\d z_{i_1} \wedge \dots \wedge \d z_{i_p}, \qquad f_I \in R, \qquad I = \br{i_1, \dots, i_p}, \qquad i_1 < \dots < i_p, $$
where $ \d z_{i_1}, \dots, \d z_{i_p} $ is an $ R $-basis of $ \H^0\br{U, \Omega_U^p} $.

\begin{example*}
$$ \H^0\br{U, \Omega_U^p} \cong \ext^p\H^0\br{U, \Omega_U^1} $$
is an isomorphism as $ R $-modules. This is not true for complex manifolds in general.
\end{example*}

The \textbf{exterior product} is
$$ \function{\H^0\br{U, \Omega_U^p} \otimes \H^0\br{U, \Omega_U^q}}{\H^0\br{U, \Omega_U^{p + q}}}{\omega_1 \otimes \omega_2}{\omega_1 \wedge \omega_2}, $$
where we just need to define
$$ \omega_1 \wedge \omega_2 = f\d z_{i_1} \wedge \d z_{i_p} \otimes g\d z_{j_1} \wedge \d z_{j_q} = fg\d z_{i_1} \wedge \dots \wedge \d z_{i_p} \wedge \d z_{j_1} \wedge \dots \wedge \d z_{j_q}, $$
by linearity. Then $ \omega_1 \wedge \omega_2 = 0 $ if $ \cbr{i_1, \dots, i_p} \cap \cbr{j_1, \dots, j_q} \ne \emptyset $, since $ \d z_i \wedge \d z_i = 0 $.

\begin{exercise*}
Check that this definition coincides with the definition in M4P54.
\end{exercise*}

The \textbf{exterior derivative} is
$$ \function[\d]{\H^0\br{U, \Omega_U^p}}{\H^0\br{U, \Omega_U^{p + 1}}}{f\d z_{i_1} \wedge \dots \wedge \d z_{i_p}}{\sum_{j = 1}^n \dpd{}{z_j}f\d z_j \wedge \d z_{i_1} \wedge \dots \wedge \d z_{i_p}}. $$
By definition $ \d $ is $ \CC $-linear, but not $ R $-linear. That is,
$$ \d\br{a\omega_1 + b\omega_2} = a\d\omega_1 + b\d\omega_2, \qquad \omega_1, \omega_2 \in \H^0\br{U, \Omega_U^p}, \qquad a, b \in \CC. $$

\begin{proposition}
Let $ U \subset \CC^n $ be open. Then
\begin{itemize}
\item the Leibnitz rule
$$ \d\br{\omega_1 \wedge \omega_2} = \d\omega_1 \wedge \omega_2 + \br{-1}^p \omega_1 \wedge \d\omega_2, \qquad \omega_1 \in \H^0\br{U, \Omega_U^p}, \qquad \omega_2 \in \H^0\br{U, \Omega_U^q}, $$
\item $ \d^2 = 0 $, that is
$$ \d\d\omega = 0, \qquad \omega \in \H^0\br{U, \Omega_U^p}. $$
\end{itemize}
\end{proposition}

\pagebreak

\begin{definition}
Let $ f : U \subset \CC^n \to \CC^m $ be holomorphic, let $ f_i = \p_i \circ f : V \to \CC $ where $ \p_i : \CC^m \to \CC $ is the $ i $-th projection, and let $ f\br{U} \subset V \subset \CC^m $ be open. Then if
$$ \omega = h\d z_{i_1} \wedge \dots \wedge \d z_{i_p} \in \H^0\br{V, \Omega_V^p}, \qquad h \in \H^0\br{U, \OOO_U}, $$
then we can define the \textbf{pull-back} of $ \omega $,
$$ f^*\omega = h \circ f\d f_{i_1} \wedge \dots \wedge \d f_{i_p} \in \H^0\br{U, \Omega_U^p}, $$
since $ f_i \in \H^0\br{V, \OOO_V} = \H^0\br{V, \Omega_V^0} $ implies that $ \d f_i \in \H^0\br{V, \Omega_V^1} $, so
$$
\begin{tikzcd}
U \arrow{r}{f} \arrow{dr}[swap]{h \circ f \in \H^0\br{U, \OOO_U}} & f\br{U} \subset V \arrow{d}{h} \\
& \CC
\end{tikzcd}.
$$
\end{definition}

This is linear over $ \CC $ and over $ \H^0\br{U, \OOO_U} $.

\begin{proposition}
Let $ U \subset \CC^n $, $ V \subset \CC^m $, and $ W \subset \CC^{m'} $ be open, let $ f : U \to \CC^m $ and $ g : V \to \CC^{m'} $ be holomorphic such that $ V \supset f\br{U} $ and $ W \supset g\br{V} $, and let $ \omega \in \H^0\br{V, \Omega_V^p} $ and $ \eta \in \H^0\br{V, \Omega_V^q} $. Then
\begin{itemize}
\item $ f^*\br{\omega + \eta} = f^*\omega + f^*\eta $ if $ p = q $,
\item $ f^*\br{\omega \wedge \eta} = f^*\omega \wedge f^*\eta $,
\item $ \d\br{f^*\omega} = f^*\d\omega $, and
\item $ f^*g^*\omega = \br{g \circ f}^*\omega $.
\end{itemize}
\end{proposition}

Let $ U \subset \CC^n \cong \RR^{2n} $, and let $ z_i = x_i + iy_i $ for $ i = 1, \dots, n $ and $ x_i, y_i \in \RR $. Then
$$ \d z_i = \d x_i + i\d y_i, $$
so any holomorphic form is a differentiable form on $ \RR^{2n} $. A \textbf{$ \br{p, q} $-form} is a differentiable $ \br{p + q} $-form of the expression
$$ \omega = \sum_{\abs{I} = p, \ \abs{J} = q} f_{I, J}\d z_{i_1} \wedge \dots \wedge \d z_{i_p} \wedge \d\overline{z_{j_1}} \wedge \dots \wedge \d\overline{z_{j_q}}, \qquad f_{I, J} : U \to \CC \cong \RR^2 \in \C^\infty, $$
where $ \d\overline{z_j} = \d x_j - i\d y_j $. We denote
$$ \CO{U}{p, q} = \cbr{\text{differentiable $ \br{p + q} $-forms on} \ U}. $$
If $ \omega $ is a $ \br{p, q} $-form, then the \textbf{conjugate} $ \overline{\omega} $ of $ \omega $ is the $ \br{q, p} $-form defined by
$$ \overline{\omega} = \sum_{\abs{I} = p, \ \abs{J} = q} \overline{f_{I, J}}\d\overline{z_{i_1}} \wedge \dots \wedge \d\overline{z_{i_p}} \wedge \d z_{j_1} \wedge \dots \wedge \d z_{j_q}. $$

\pagebreak

\section{Complex manifolds}

\subsection{Complex manifolds}

\begin{definition}
A \textbf{complex manifold} of dimension $ n $ is a connected Hausdorff topological space $ X $, with a countable open cover $ \cbr{U_\alpha} $ of $ X $ such that for all $ \alpha $, there exists $ \phi_\alpha : U_\alpha \to \CC^n $ such that $ \phi_\alpha : U_\alpha \to \phi_\alpha\br{U_\alpha} $ is a homeomorphism and
$$ \phi_\alpha \circ \phi_\beta^{-1} : \phi_\beta\br{U_\alpha \cap U_\beta} \to \phi_\alpha\br{U_\alpha \cap U_\beta} $$
is a biholomorphism for each $ \alpha $ and $ \beta $, so
$$
\begin{tikzcd}
& U_\alpha \cap U_\beta \arrow{dl}[swap]{\phi_\alpha} \arrow{dr}{\phi_\beta} & \\
\CC^n \supset \phi_\alpha\br{U_\alpha \cap U_\beta} \arrow{rr}[swap]{\phi_\alpha \circ \phi_\beta^{-1}} & & \phi_\beta\br{U_\alpha \cap U_\beta} \subset \CC^n
\end{tikzcd}.
$$
The pair $ \br{U_\alpha, \phi_\alpha} $ is called a \textbf{holomorphic chart}. The set $ \cbr{\br{U_\alpha, \phi_\alpha}} $ is called a \textbf{holomorphic atlas} or a \textbf{complex structure}.
\end{definition}

Recall $ X $ is Hausdorff if for all $ x, y \in X $ there exist $ U $ and $ V $ open in $ X $ such that $ U \cap V = \emptyset $ and $ x \in U $ and $ y \in V $.

\lecture{5}{Thursday}{16/01/20}

\begin{example}
\hfill
\begin{itemize}
\item If $ U \subset \CC^n $ is an open set then $ U $ is a complex manifold. More in general if $ X $ is a complex manifold and $ U \subset X $ is open then $ U $ is a complex manifold. Let $ \cbr{\br{U_\alpha, \phi_\alpha}} $ be a complex structure on $ X $. Then
$$ \cbr{\br{\overline{U_\alpha}, \overline{\phi_\alpha}}} = \cbr{\br{U_\alpha \cap U, \eval{\phi_\alpha}_{\overline{U_\alpha}}}} $$
is a complex structure of $ X $.
\item If $ X $ and $ Y $ are complex manifolds, then $ X \times Y $ is a complex manifold.
\end{itemize}
\end{example}

\begin{example}
The projective space $ \PP_\CC^n $ or $ \CC\PP^n $. Let $ V^* = \CC^{n + 1} \setminus \cbr{0} $, with coordinates $ \br{z_0, \dots, z_n} $. Define an equivalence on $ V^* $ as
$$ v_1 \sim v_2 \qquad \iff \qquad \exists \lambda \in \CC, \ v_1 = \lambda v_2. $$
Check that $ \sim $ is an equivalence. Consider the Euclidean topology on $ V^* $. Then there exists an induced topology on
$$ X = V^* / \sim = \cbr{\sbr{v} \st v \in V^*}, $$
with quotient map
$$ \function[q]{V^*}{X}{v}{\sbr{v}}. $$
Given $ v = \br{z_0, \dots, z_n} \in V^* $ we denote $ \sbr{v} = \sbr{z_0, \dots, z_n} $ such that $ z_i \ne 0 $ for some $ i $. Two elements $ \sbr{x_0, \dots, x_n} $ and $ \sbr{y_0, \dots, y_n} $ of $ X $ define the same point if and only if there exists $ \lambda $ such that $ x_i = \lambda y_i $ for all $ i $. Let
$$ V_i = \cbr{\br{z_0, \dots, z_n} \in V^* \st z_i \ne 0}, $$
which is open in $ V^* $, and let $ U_i = q\br{V_i} $, which is open in $ X $, such that $ \cbr{U_i} $ is a cover of $ X $, that is $ \bigcup_i U_i = X $. Let
$$ H_i = \cbr{\br{z_0, \dots, z_n} \in V^* \st z_i = 1}. $$
Then there exists a homeomorphism
$$ \function[r_i]{H_i}{\CC^n}{\br{z_0, \dots, z_n}}{\sbr{z_0, \dots, z_{i - 1}, z_{i + 1}, \dots, z_n}}, $$
and let
$$ \function[q_i = \eval{q}_{H_i}]{H_i \subset V^*}{U_i \subset X}{\br{z_0, \dots, z_n}}{\sbr{z_0, \dots, z_n}} $$
be also a homeomorphism.

\pagebreak

\begin{itemize}
\item $ q_i $ is surjective. Take $ \sbr{x_0, \dots, x_n} \in U_i $. Then $ x_i \ne 0 $, so choose $ \lambda = 1 / x_i $. Then
$$ \sbr{x_0, \dots, x_n} = \sbr{\dfrac{x_0}{x_i}, \dots, \dfrac{x_n}{x_i}} = q\br{z_0, \dots, z_n}, \qquad z_j = \dfrac{x_j}{x_i}, $$
and in particular $ z_i = 1 $, so there exists $ \br{z_0, \dots, z_n} \in H_i $ such that $ q_i\br{z_0, \dots, z_n} = \sbr{x_0, \dots, x_n} $.
\item $ q_i $ is injective. \footnote{Exercise}
\end{itemize}
For all $ i $, $ q_i^{-1} : U_i \to H_i $ and $ r_i : H_i \to \CC^n $ are homeomorphisms, so $ \phi_i = r_i \circ q_i^{-1} : U_i \to \CC^n $ is also a homeomorphism. We want to show that $ \br{U_i, \phi_i} $ define a holomorphic atlas, so
$$ \phi_i \circ \phi_j^{-1} : \phi_j\br{U_i \cap U_j} \to \phi_i\br{U_i \cap U_j} $$
is a biholomorphism. Consider the case $ j = 0 $ and $ i = 1 $. Then $ \phi_0\br{U_0 \cap U_1} = \cbr{\br{x_1, \dots, x_n} \st x_1 \ne 0} $, so
$$ \function[\phi_1 \circ \phi_0^{-1}]{\phi_0\br{U_0 \cap U_1}}{\phi_1\br{U_0 \cap U_1}}{\br{x_1, \dots, x_n}}{\br{1, \dfrac{x_2}{x_1}, \dots, \dfrac{x_n}{x_1}}} $$
is a biholomorphism. Thus $ X $ is a compact complex manifold. If $ n = 1 $, then $ \PP_\CC^1 \cong \S^2 $.
\end{example}

\begin{example}
The complex torus. Let
$$ \function{\Lambda = \ZZ^{2n}}{\CC^n}{\br{a_1, \dots, a_n, b_1, \dots, b_n}}{\br{a_1 + ib_1, \dots, a_n + ib_n}}. $$
Define an equivalence on $ \CC^n $ by
$$ v_1 \sim v_2 \qquad \iff \qquad v_1 - v_2 \in \Lambda. $$
Then
$$ X = \CC^n / \sim, $$
with quotient map $ q : \CC^n \to X $ is Hausdorff and compact. Topologically $ X \cong \sbr{0, 1}^{2n} / \sim $. For each $ x \in \CC^n $, we can find an open set $ x \in U \subset \CC^n $ such that $ \eval{q}_U : U \to X $ is a homeomorphism. The idea is if $ x \in \br{0, 1}^{2n} $ then we can take $ U = \br{0, 1}^{2n} $. If not, there exists a translation of $ \CC^n \to \CC^n $ such that the property holds. We define
$$ \phi_V = \eval{q}_U^{-1} : V \subset \CC^n / \Lambda \to U \subset \CC^n, \qquad V = q\br{U}. $$
Show that $ \br{V, \phi_V} $ define a complex structure on $ X $. \footnote{Exercise} This is also a compact complex manifold. More in general $ \CC^n / \Lambda $ for a lattice $ \Lambda \cong \ZZ^{2n} $ is a compact complex manifold.
\end{example}

\subsection{Holomorphic functions on complex manifolds}

\lecture{6}{Tuesday}{21/01/20}

\begin{definition}
Let $ f : X \to Y $ be a continuous morphism between complex manifolds. Then $ f $ is \textbf{holomorphic} if there exists a complex structure $ \cbr{\br{U_\alpha, \phi_\alpha}} $ on $ Y $ and for all $ y \in Y $ there exists a holomorphic chart $ \br{V_\alpha, \psi_\alpha} $ such that $ x \in V_\alpha $ and $ f\br{V_\alpha} \subset U_\alpha $ around any point $ x $ of $ f^{-1}\br{y} $ and $ \phi_\alpha \circ f \circ \psi_\alpha^{-1} $ is holomorphic, so
$$
\begin{tikzcd}
X \supset V_\alpha \arrow{r}{f} \arrow{d}[swap]{\psi_\alpha} & U_\alpha \subset Y \arrow{d}{\phi_\alpha} \\
\psi_\alpha\br{V_\alpha} \arrow{r}[swap]{\widetilde{f}} & \phi_\alpha\br{U_\alpha}
\end{tikzcd}.
$$
Let $ \J_f = \J_{\widetilde{f}} $, and let a \textbf{holomorphic function on $ X $} be a holomorphic function $ f : X \to \CC $.
\end{definition}

\begin{exercise}
If $ X $ is a compact complex manifold then any holomorphic function $ f : X \to \CC $ is constant.
\end{exercise}

\pagebreak

\begin{definition}
Let $ f : X \to Y $ be a holomorphic function between complex manifolds. Then $ f $ is
\begin{itemize}
\item a \textbf{submersion} if $ \dim Y \ge \dim Y = r $ and $ \rk \J_f = r $ at any point,
\item an \textbf{immersion} if $ r = \dim X \le \dim Y $ and $ \rk \J_f = r $ at any point, and
\item an \textbf{embedding} if it is an immersion and $ f : X \to f\br{X} $ is a homeomorphism.
\end{itemize}
\end{definition}

\begin{example}
Let $ f_2, \dots, f_n : \CC \to \CC $ be holomorphic, and let
$$ \function[f]{\CC}{\CC^n}{z}{\br{z, f_2\br{z}, \dots, f_n\br{z}}}. $$
Then $ f $ is an embedding.
\end{example}

\begin{example}
Let $ X = \CC^2 / \Lambda $ for $ \Lambda = \ZZ^4 \subset \CC^2 $, and let $ q : \CC^2 \to X $. Fix $ \lambda \in \CC $. Let
$$ \function[f]{\CC}{\CC^2}{z}{\br{z, \lambda z}}. $$
Then $ \widetilde{f} = q \circ f : \CC \to X $ is an immersion.
\begin{itemize}
\item If $ \lambda = 0 $ or $ \lambda = \tfrac{1}{2} $, then $ \widetilde{f}\br{\CC} $ is a closed submanifold.
\item If $ \lambda $ is general then $ \widetilde{f}\br{\CC} $ is dense inside $ X $, so it is not closed. Thus it is not a complex submanifold of $ X $.
\end{itemize}
\end{example}

\subsection{Complex submanifolds}

\begin{definition}
Let $ i : X \to Y $ be an embedding of complex manifolds. If $ i\br{X} \subset Y $ is closed then $ i\br{X} $ is called a \textbf{complex submanifold} of $ Y $. The \textbf{codimension} of $ X $ in $ Y $ is $ \dim Y - \dim X $.
\end{definition}

\begin{theorem}
\label{thm:3.11}
\hfill
\begin{enumerate}
\item Let $ i : X \to Y $ be a submanifold of codimension $ k $, and let $ n = \dim X $. Then for all $ x \in X $, there exists an open neighbourhood $ x \in U \subset Y $ and a submersion $ f : U \to \D\br{0, 1}^k \subset \CC^k $ such that $ X \cap U = f^{-1}\br{0} $.
\item If $ X \subset Y $ is a closed subset such that for all $ x \in X $ there exists $ U \ni x $ open in $ Y $ and a submersion $ f : U \to \D\br{0, 1}^k $ such that $ X \cap U = f^{-1}\br{0} $, then $ X $ is a complex submanifold.
\end{enumerate}
\end{theorem}

\begin{proof}
\hfill
\begin{enumerate}
\item We can assume that if there exists a holomorphic chart $ \br{U, \psi} $ on $ Y $ such that $ x \in U $ and if $ V = i^{-1}\br{U} $ then there exists $ \phi : V \to \CC^n $ such that $ \br{V, \phi} $ is a holomorphic chart on $ X $. After possibly shrinking $ U $ smaller, by the rank theorem, there exist biholomorphic $ a : \psi\br{U} \to \D\br{0, 1}^{n + k} $ and $ b : \phi\br{U} \to \D\br{0, 1}^n $ such that the induced morphism is given by
$$ \function{\D\br{0, 1}^n}{\D\br{0, 1}^{n + k}}{\br{z_1, \dots, z_n}}{\br{z_1, \dots, z_n, 0, \dots, 0}}. $$
Let
$$ \function[c]{\D\br{0, 1}^{n + k}}{\D\br{0, 1}^k}{\br{z_1, \dots, z_{n + k}}}{\br{z_{n + 1}, \dots, z_{n + k}}}, $$
so
$$
\begin{tikzcd}
Y & U \arrow[subset]{l} \arrow{r}{\phi} & \phi\br{U} \arrow{r}{b} & \D\br{0, 1}^n \subset \CC^n \arrow{d} \\
X \arrow[hookrightarrow]{u}{i} & V \arrow[hookrightarrow]{u}{i} \arrow[subset]{l} \arrow{r}[swap]{\psi} & \psi\br{U} \arrow{r}[swap]{a} & \D\br{0, 1}^{n + k} \subset \CC^{n + k} \arrow[bend right=90, dashed]{u}[swap]{c}
\end{tikzcd}.
$$
Then $ f $ is the composition $ c \circ a \circ \psi : U \to \D\br{0, 1}^n $.

\pagebreak

\item Let $ \cbr{\br{U_\alpha, \phi_\alpha}} $ be a complex structure on $ Y $, and let $ V_\alpha = X \cap U_\alpha $ and $ \psi_\alpha = \eval{\phi_\alpha}_{V_\alpha} $. The goal is to show that $ \cbr{\br{V_\alpha, \psi_\alpha}} $ defines a complex structure on $ X $. By assumption,
$$ \phi_\alpha \circ \phi_\beta^{-1} : \phi_\beta\br{U_\alpha \cap U_\beta} \subset \CC^{n + k} \to \phi_\alpha\br{U_\alpha \cap U_\beta} \subset \CC^{n + k} $$
is biholomorphic. Let $ U' = \phi_\beta\br{U} $, let $ X' = \phi_\beta\br{X \cap U} $, and let $ f' = f \circ \phi_\beta^{-1} $, so
$$
\begin{tikzcd}
& & & \phi_\alpha\br{U} \arrow[subset]{r} & \phi_\alpha\br{U_\alpha \cap U_\beta} \subset \CC^{n + k} \\
Y & U_\alpha \cap U_\beta \arrow[subset]{l} & U \arrow{ur}{\phi_\alpha} \arrow[subset]{l} \arrow{r}{\phi_\beta} \arrow{drr}{f} & U' \arrow[subset]{r} \arrow{dr}{f'} & \phi_\beta\br{U_\alpha \cap U_\beta} \arrow{u}[swap]{\phi_\alpha \circ \phi_\beta^{-1}} \subset \CC^{n + k} \\
X \arrow[hookrightarrow]{u}{i} & X \cap U_\alpha \cap U_\beta \arrow[subset]{u} \arrow[subset]{l} & X \cap U \arrow[subset]{u} \arrow[subset]{l} & X' \arrow[near start, subset]{u} & \D\br{0, 1}^k \subset \CC^k
\end{tikzcd}.
$$
Then $ f'^{-1}\br{0} = \phi_\beta\br{X \cap U_\alpha \cap U_\beta} $ and $ f' $ is also a submersion. By the rank theorem, we may assume that $ U' = \D\br{0, 1}^{n + k} $ and $ f'\br{z_1, \dots, z_{n + k}} = \br{z_1, \dots, z_k} $, so $ \phi_\beta\br{X' \cap U_\alpha \cap U_\beta} = f'^{-1}\br{0} $. Thus
$$ \br{\psi_\alpha \circ \psi_\beta^{-1}}\br{z_1, \dots, z_n} = \br{\phi_\alpha \circ \phi_\beta^{-1}}\br{z_1, \dots, z_n, 0, \dots, 0} $$
is also a biholomorphism.
\end{enumerate}
\end{proof}

\subsection{Examples of complex manifolds}

\lecture{7}{Thursday}{23/01/20}

\begin{example}
Let $ U \subset \CC^n $ be open, let $ k \le n $, let $ f_1, \dots, f_k : U \to \CC $, and let
$$ V = \cbr{x \in \CC^n \st f_1\br{x} = \dots = f_k\br{x} = 0}. $$
Assume that $ \br{\tpd{}{z_j}f_i} $ has maximal rank $ k $ at any point of $ U $. Then $ V $ is a complex submanifold of $ U $. The idea is if $ f = \br{f_1, \dots, f_k} : U \to \CC^k $, then $ f $ is a submersion around any point of $ V $, and use Theorem \ref{thm:3.11}.
\end{example}

\begin{example}
Let $ f : X \to Y $ be a holomorphism between complex manifolds, and let $ W \subset X $ be a submanifold. Then $ \eval{f}_W : W \to Y $ is holomorphic.
\end{example}

\begin{exercise}
Let $ X = \CC^n $. Show that all the compact submanifolds of $ X $ are zero-dimensional, that is points.
\end{exercise}

\begin{exercise}
Let $ X $ and $ Y $ be compact manifolds. Recall that $ X \times Y $ is also a complex manifold. Assume $ f : X \to Y $, so
$$ \Gamma_f = \cbr{\br{x, f\br{x}} \st x \in X} \subset X \times Y. $$
Show that $ \Gamma_f $ is a complex submanifold.
\end{exercise}

\begin{example}
Let $ n, m > 0 $, and let
$$ \Mat_{n, m} \CC = \cbr{\text{$ \br{n \times m} $-matrices}} \cong \CC^{n \cdot m}. $$
Then $ \Mat_{n, m} \CC $ is a complex manifold. Let
$$ \GL_n \CC = \cbr{\text{$ \br{n \times n} $-matrices} \ A \st A \ \text{invertible}}. $$
Then $ \GL_n \CC $ is a complex manifold, open in $ \Mat_{n, n} \CC $.
\end{example}

\pagebreak

\begin{example}
Projective manifolds. Let $ R = \CC\sbr{x_0, \dots, x_n} $ be the ring of polynomials, and let $ X = \PP_\CC^n $ be the complex projective space. Then $ f \in R $ is \textbf{homogeneous of degree $ d $} if $ f\br{\lambda x} = \lambda^df\br{x} $. Let $ q : \CC^{n + 1} \setminus \cbr{0} \to \PP_\CC^n $, let $ F_1, \dots, F_k $ be homogeneous polynomials in $ R $, and let
$$ V = \cbr{F_1 = \dots = F_k = 0} \subset \CC^{n + 1} \setminus \cbr{0}, \qquad W = q\br{V} \subset \PP_\CC^n, $$
so $ q^{-1}\br{W} = V $, because $ F_i $ are homogeneous. Since $ V $ is closed in $ \CC^{n + 1} \setminus \cbr{0} $, $ W $ is closed in $ \PP_\CC^n $. Claim that if $ V $ is a submanifold of $ \CC^{n + 1} \setminus \cbr{0} $ then $ W $ is a compact submanifold of $ \PP_\CC^n $. If $ \cbr{U_i} $ is the open covering given by
$$ U_i = \cbr{\sbr{x_0, \dots, x_n} \st x_i \ne 0}, $$
then it is enough to show that $ W \cap U_i $ is a complex submanifold of $ U_i $ for all $ i $. Assume $ i = n $. Let $ \CC^* = \CC \setminus \cbr{0} $. Then $ q\br{x} = \CC^* $ for all $ x \in X $ but $ \PP_\CC^n \times \CC^* \ne \CC^{n + 1} \setminus \cbr{0} $. We want to show there exists a biholomorphism
$$ \function[\phi_n]{U_n \times \CC^*}{q^{-1}\br{U_n} = \cbr{\br{x_0, \dots, x_n} \in \CC^{n + 1} \st x_n \ne 0}}{\br{\sbr{x_0, \dots, x_n}, t}}{\br{\dfrac{tx_0}{x_n}, \dots, \dfrac{tx_{n - 1}}{x_n}, t}}, $$
such that
$$ \function[\phi_n^{-1}]{q^{-1}\br{U_n}}{U_n \times \CC^*}{\br{y_0, \dots, y_n}}{\br{q\br{y_0, \dots, y_n}, y_n} = \br{\sbr{y_0, \dots, y_n}, y_n}}. $$
From this, it follows that $ V \cap q^{-1}\br{U_n} \cong \br{W \cap U_n} \times \CC^* $, so the claim follows.
\end{example}

\begin{example}
Plane curves. Let $ X = \PP_\CC^2 $, let $ F \in R\sbr{x_0, x_1, x_2} $ be homogeneous of degree $ d $, and let $ W = \cbr{F = 0} \subset \PP_\CC^2 $. Then $ W $ is a compact complex submanifold if and only if for all $ x \in \PP_\CC^2 $, $ \da_{x_i} F\br{x} \ne 0 $ for some $ i $.
\begin{itemize}[leftmargin=0.5in]
\item[$ d = 1 $.] $ W $ is the projective line, so $ F = ax_0 + bx_1 + cx_2 $ for $ a, b, c $ not all zero. Then $ W $ is a complex submanifold. There exists a biholomorphism $ \PP_\CC^1 \to W $.
\item[$ d = 2 $.] $ W $ is a conic, so $ F $ is a degree two polynomial. Then $ F = x_0x_1 $ does not define a manifold. If $ F = x_0x_1 - x_2^2 $, then $ W $ is a complex submanifold of $ X $. There exists
$$ \function{\PP_\CC^1}{W \subset X}{\sbr{t_0, t_1}}{\sbr{t_0^2, t_1^2, t_0t_1}}. $$
Check that it is a biholomorphism. \footnote{Exercise} This is true for any $ f $ of degree two such that $ W $ is a complex submanifold.
\item[$ d \ge 3 $.] If $ W $ is a complex submanifold then we will show that $ W $ is not biholomorphic to $ \PP_\CC^1 $.
\end{itemize}
\end{example}

\subsection{Tangent spaces of complex manifolds}

\begin{definition}
Let $ X $ be a complex manifold of dimension $ n $, and let $ x \in X $. Then there exists a chart $ \br{U, \phi} $ around $ x $ such that $ \phi\br{U} \subset \CC^n $. The \textbf{holomorphic tangent space} $ \T_xX $ of $ X $ at $ x $, is the vector space over $ \CC $ generated by
$$ \abr{\dpd{}{z_1}, \dots, \dpd{}{z_n}}. $$
Let $ X $ be a real manifold. The \textbf{real tangent space} $ \T_x^\RR X $ is the vector space over $ \RR $ defined by
$$ \abr{\dpd{}{x_1}, \dots, \dpd{}{x_n}, \dpd{}{y_1}, \dots, \dpd{}{y_n}}, $$
where $ \br{x_1, \dots, x_n, y_1, \dots, y_n} $ are coordinates of $ \RR^{2n} $. The \textbf{complex tangent space} $ \T_x^\CC X $ is the vector space over $ \CC $ generated by
$$ \abr{\dpd{}{z_1}, \dots, \dpd{}{z_n}, \dpd{}{\overline{z_1}}, \dots, \dpd{}{\overline{z_n}}}, $$
a $ 2n $-dimensional vector space over $ \CC $. Then $ \T_x^\CC X = \T_x^\RR X \otimes_\RR \CC $.
\end{definition}

\pagebreak

\subsection{Holomorphic differential forms on complex manifolds}

\begin{definition}
Let $ X $ be a complex manifold of dimension $ n $, and let $ \cbr{\br{U_\alpha, \phi_\alpha}} $ be a complex structure on $ X $. A \textbf{holomorphic $ p $-form} on $ X $ is the data $ \omega_\alpha $, the $ p $-forms on $ \phi_\alpha\br{U_\alpha} \subset \CC^n $ such that if
$$ h_{\alpha\beta} = \phi_\alpha \circ \phi_\beta^{-1} : \phi_\beta\br{U_\alpha \cap U_\beta} \to \phi_\alpha\br{U_\alpha \cap U_\beta}, $$
then $ h_{\alpha\beta}^*\omega_\beta = \omega_\alpha $ for all $ \alpha $ and $ \beta $.
\end{definition}

\lecture{8}{Thursday}{23/01/20}

\begin{notation}
Let
$$ \Omega_X^p\br{X} = \H^0\br{X, \Omega_X^p} = \cbr{\text{holomorphic $ p $-forms on} \ X}, $$
$$ \OOO_X\br{X} = \H^0\br{X, \OOO_X} = \cbr{\text{holomorphic functions on} \ X}. $$
\end{notation}

Then $ R = \OOO_X\br{X} $ is a ring and $ M = \Omega_X^p\br{X} $ is an $ R $-module.

\begin{lemma}
Let $ f : X \to Y $ be holomorphic. Then $ f^* : \Omega_Y^p\br{Y} \to \Omega_X^p\br{X} $.
\end{lemma}

\begin{proof}
Let $ \cbr{\br{U_\alpha, \phi_\alpha}} $ be a complex structure on $ Y $. We can write $ f^{-1}\br{U_\alpha} = \bigcup_{\alpha, \beta} V_{\alpha, \beta} $ where $ \cbr{\br{V_{\alpha, \beta}, \psi_{\alpha, \beta}}} $ is a complex structure on $ X $, so
$$ \CC^n \xleftarrow{\psi_{\alpha, \beta}} V_{\alpha, \beta} \xrightarrow{\eval{f}_{V_{\alpha, \beta}}} U_\alpha \xrightarrow{\phi_\alpha} \CC^n. $$
Assume $ \omega $ is defined by $ \omega_\alpha $ on $ \phi_\alpha\br{U_\alpha} $. Let
$$ \omega_{\alpha, \beta} = \br{\br{\psi_{\alpha, \beta}^{-1}}^* \circ f^* \circ \phi_\alpha^*}\br{\omega_\alpha} $$
be a $ p $-form on $ \psi_{\alpha, \beta}\br{V_{\alpha, \beta}} $. Check that $ \omega_{\alpha, \beta} $ are compatible with respect to the atlas on $ X $. \footnote{Exercise}
\end{proof}

As in the local case, we can define
$$ \function{\Omega_X^p\br{X} \otimes \Omega_X^q\br{X}}{\Omega_X^{p + q}\br{X}}{\omega_1 \otimes \omega_2}{\omega_1 \wedge \omega_2}. $$
Similarly there exists $ \d : \Omega_X^p\br{X} \to \Omega_X^{p + 1}\br{X} $.

\pagebreak

\section{Vector bundles}

\subsection{Holomorphic vector bundles}

\begin{definition}
Let $ X $ be a complex manifold. A \textbf{holomorphic vector bundle $ E $ of rank $ r $} on $ X $ is a complex manifold $ E $, a holomorphism $ \pi : E \to X $, and an open covering $ U_\alpha $ of $ X $ such that there exists a biholomorphism
$$ \psi_\alpha : \pi^{-1}\br{U_\alpha} \to U_\alpha \times \CC^r, $$
such that if $ \p_\alpha : U_\alpha \times \CC^r \to U_\alpha $ is the projection then $ \eval{\pi}_{\pi^{-1}\br{U_\alpha}} = \p_\alpha \circ \psi_\alpha $, so
$$
\begin{tikzcd}
E \arrow{d}[swap]{\pi} & \pi^{-1}\br{U_\alpha} \arrow[subset]{l} \arrow{r}{\psi_\alpha} \arrow{d}[swap]{\pi} & U_\alpha \times \CC^r \arrow{dl}{\p_\alpha} \\
X & U_\alpha \arrow[subset]{l} &
\end{tikzcd}.
$$
A vector bundle of rank one is called a \textbf{line bundle}.
\end{definition}

For any $ x \in X $, there exists $ \alpha $ such that $ x \in U_\alpha $, so
$$
\begin{tikzcd}
\pi^{-1}\br{x} \arrow{r}{\psi_\alpha} \arrow{d}[swap]{\pi} & \cbr{x} \times \CC^r \arrow{dl}{\p_\alpha} \\
x &
\end{tikzcd}.
$$
Then $ E\br{x} = \pi^{-1}\br{x} $ is a vector space of rank $ r $ over $ \CC $. Let $ U_\alpha \ni x \in U_\beta $. There exists a biholomorphism
$$ \CC^r \cong \p_\alpha^{-1}\br{x} \to \p_\beta^{-1}\br{x} \cong \CC^r, $$
because they are both biholomorphic to $ \pi^{-1}\br{x} $, so $ g_{\alpha\beta}\br{x} \in \GL_r \CC $ because all the biholomorphisms from $ \CC^r \to \CC^r $ are linear. The holomorphisms
$$ g_{\alpha\beta} : U_\alpha \cap U_\beta \to \GL_r \CC $$
are called \textbf{transition functions}. Then
$$
\begin{tikzcd}
\p_\alpha^{-1}\br{x} \arrow{rr}{\id} \arrow{dr} & & \p_\alpha^{-1}\br{x} \\
& \p_\beta^{-1}\br{x} \arrow{ur} &
\end{tikzcd},
$$
so
$$ \br{g_{\alpha\beta} \circ g_{\beta\alpha}}\br{x} = x, \qquad x \in U_\alpha \cap U_\beta, $$
and
$$
\begin{tikzcd}
\p_\alpha^{-1}\br{x} \arrow{rr}{g_{\alpha\gamma}} \arrow{dr} & & \p_\gamma^{-1}\br{x} \\
& \p_\beta^{-1}\br{x} \arrow{ur} &
\end{tikzcd},
$$
so
$$ \br{g_{\alpha\beta} \circ g_{\beta\gamma}}\br{x} = g_{\alpha\gamma}\br{x}, \qquad x \in U_\alpha \cap U_\beta \cap U_\gamma. $$

\begin{definition}
Let $ X $ be a complex manifold, and let $ E $ and $ F $ be vector bundles on $ X $ of rank $ r $ and $ s $ respectively, with $ \pi : E \to X $ and $ \pi' : F \to X $. A \textbf{holomorphic map} $ f : E \to F $ is a holomorphic function $ E \to F $ such that $ \pi = \pi' \circ f $ and such that the rank of the induced linear map $ E\br{x} \to F\br{x} $ is independent of $ x \in X $, so
$$
\begin{tikzcd}
E \arrow{rr}{f} \arrow{dr}[swap]{\pi} & & F \arrow{dl}{\pi'} \\
& X &
\end{tikzcd},
\qquad
\begin{tikzcd}
E\br{x} = \pi^{-1}\br{x} \arrow{rr}{f} \arrow{dr}[swap]{\pi} & & \pi'^{-1}\br{x} = F\br{x} \arrow{dl}{\pi'} \\
& x &
\end{tikzcd}.
$$
\end{definition}

\pagebreak

\subsection{Examples of holomorphic vector bundles}

\begin{example}
$ \pi : E = X \times \CC^r \to X $ is a vector bundle of rank $ r $, called \textbf{trivial}.
\end{example}

\begin{example}
Algebra of vector bundles. Let $ \pi : E \to X $ and $ \pi' : F \to X $ be vector bundles on $ X $ of rank $ r $ and $ s $ respectively.
\begin{itemize}
\item The \textbf{direct sum} $ E \oplus F $ is the $ \br{r + s} $-vector bundle such that
$$ \br{E \oplus F}\br{x} = E\br{x} \oplus F\br{x}, \qquad x \in X. $$
The idea is to take an open cover which trivialises both $ E $ and $ F $. Find the transition function of $ E \oplus F $. \footnote{Exercise}
\item The \textbf{tensor product} $ E \otimes F $ is the $ \br{r \cdot s} $-vector bundle such that
$$ \br{E \otimes F}\br{x} = E\br{x} \otimes F\br{x}, \qquad x \in X. $$
\item The \textbf{$ p $-th exterior power} of $ E $ is the vector bundle $ \ext^pE $ such that
$$ \br{\ext^pE}\br{x} = \ext^pE\br{x}, \qquad x \in X. $$
If $ p = r = \rk E $ then $ \det E = \ext^rE $ is a line bundle on $ X $.
\item The \textbf{dual} of $ E $ is the rank $ r $ vector bundle $ E^* $ such that
$$ E^*\br{x} = \br{E\br{x}}^*, \qquad x \in X, $$
the dual $ \Hom\br{E\br{x}, \CC} $ of $ E\br{x} $.
\item Let $ f : E \to F $ be a holomorphic map. Then the \textbf{kernel} $ \ker f $ is a vector bundle such that
$$ \br{\ker f}\br{x} = \ker f\br{x} \subset E\br{x}, \qquad x \in X. $$
The \textbf{cokernel} $ \coker f $ is a vector bundle such that
$$ \br{\coker f}\br{x} = \coker f\br{x} \subset F\br{x}, \qquad x \in X. $$
\end{itemize}
\end{example}

\lecture{9}{Tuesday}{28/01/20}

\begin{example}
Let $ X = \PP_\CC^n $, and let
$$ \OOO\br{-1} = \cbr{\br{x, v} \st x = \sbr{x_0, \dots, x_n} \in \PP_\CC^n, \ v = \mu\br{x_0, \dots, x_n}, \ \mu \in \CC} \subset \PP_\CC^n \times \CC^{n + 1}. $$
Then $ \pi = \p_1 : \OOO\br{-1} \to \PP_\CC^n $, so
$$ \pi^{-1}\br{\sbr{x_0, \dots, x_n}} = \cbr{v = \mu\br{x_0, \dots, x_n} \st \mu \in \CC} \cong \CC^1. $$
Let $ \cbr{U_i} $ be an open covering of $ X $ given by $ U_i = \cbr{\sbr{x_0, \dots, x_n} \st x_i \ne 0} $. We define
$$ \function[\psi_i]{\pi^{-1}\br{U_i}}{U_i \times \CC}{\br{\sbr{x_0, \dots, x_n}, \br{v_0, \dots, v_n}}}{\br{\sbr{x_0, \dots, x_n}, v_i}}, $$
which is a biholomorphism. Thus $ \OOO\br{-1} $ is a complex manifold and $ \OOO\br{-1} $ is a line bundle. The \textbf{tautological line bundle} $ \OOO\br{1} $ is the dual of $ \OOO\br{-1} $. Let
$$ \OOO\br{k} =
\begin{cases}
X \times \CC & k = 0 \\
\OOO\br{1}^{\otimes k} & k > 0 \\
\OOO\br{-1}^{\otimes k} & k < 0
\end{cases}.
$$
Then $ \OOO\br{k} = \OOO\br{-k}^* $. \footnote{Exercise} On $ \PP_\CC^n $ these are the only line bundles. That is, if $ \LLL $ is a line bundle on $ \PP_\CC^1 $, there exists $ k \in \ZZ $ such that $ \LLL \cong \OOO\br{k} $. Let $ X = \PP_\CC^1 $, and let $ E $ be a line bundle of rank $ r $ on $ X $. Then
$$ E \cong \bigoplus_{i = 1}^r \OOO\br{a_i}, \qquad a_1, \dots, a_r \in \ZZ. $$
This is false for $ X = \PP_\CC^n $, with $ n \ge 2 $.
\end{example}

\pagebreak

\begin{definition}
Let $ f : Y \to X $ be a holomorphism between complex manifolds, and let $ E $ be a vector bundle of rank $ r $ on $ X $. Then there exists a vector bundle $ f^*E $ of rank $ r $ on $ Y $ defined by
$$ f^*E = \cbr{\br{y, v} \in Y \times E \st f\br{y} = \pi\br{v}}, $$
the \textbf{fibre product} of $ E $ and $ Y $ over $ X $, such that
$$
\begin{tikzcd}
f^*E \arrow{r}{f'} \arrow{d}[swap]{\pi'} & E \arrow{d}{\pi} \\
Y \arrow{r}[swap]{f} & X
\end{tikzcd}.
$$
\end{definition}

Let $ \UUU = \cbr{U_i} $ be an open cover of $ X $ which trivialises $ E $, so
$$
\begin{tikzcd}
\pi^{-1}\br{U_i} \arrow{rr}{\psi_i} \arrow{dr}[swap]{\pi} & & U_i \times \CC^r \arrow{dl}{\p_1} \\
& U_i &
\end{tikzcd}.
$$
Then $ \UUU' = \cbr{f^{-1}\br{U_i}} $ is an open covering of $ Y $, so
$$
\begin{tikzcd}
\pi'^{-1}\br{f^{-1}\br{U_i}} \arrow{r}{f'} \arrow{d}[swap]{\pi'} & \pi^{-1}\br{U_i} \arrow{r}{\psi_i} \arrow{d}{\pi} & U_i \times \CC^r \arrow{r}{\p_2} & \CC^r \\
f^{-1}\br{U_i} \arrow{r}[swap]{f} & U_i & &
\end{tikzcd},
$$
and
$$ \function{\pi'^{-1}\br{f^{-1}\br{U_i}} = \cbr{\br{y, v} \in f^{-1}\br{U_i} \times \pi^{-1}\br{U_i} \st f\br{y} = \pi\br{v}}}{f^{-1}\br{U_i} \times \CC^r}{\br{y, v}}{\br{y, \p_2\br{\psi_i\br{v}}}} $$
is a biholomorphism. Thus $ f^*E $ is a vector bundle, where
$$ f^*E\br{y} = \pi'^{-1}\br{y} = E\br{f\br{y}}, \qquad y \in Y. $$

\begin{notation}
Let $ f : Y \to X $ be a morphism, and let $ E $ be a vector bundle on $ X $. Then $ f^*E = \eval{E}_Y $, mostly used if $ f : Y \hookrightarrow X $.
\end{notation}

\begin{definition}
Let $ E $ be a holomorphic vector bundle on a complex manifold $ X $, and let $ \pi : E \to X $. A \textbf{section} of $ E $ is a holomorphic function $ s : X \to E $ such that $ \pi \circ s = \id_X $.
\end{definition}

\begin{example}
Let $ E = X \times \CC^r $ be the trivial vector bundle of rank $ r $. Fix $ v \in \CC^r $. Then
$$ \function[s_v]{X}{E}{x}{\br{x, v}} $$
is a section of $ E $. If $ v_1, \dots, v_r $ is a basis of $ \CC^r $ then $ s_{v_1}, \dots, s_{v_r} $ have the property that $ s_{v_1}\br{x}, \dots, s_{v_r}\br{x} $ forms a basis of $ E\br{x} $. Vice versa, assume $ E $ is a vector bundle on $ X $ of rank $ r $ such that there exist sections $ s_1, \dots, s_r $ of $ E $ such that for all $ x \in X $, $ s_1\br{x}, \dots, s_r\br{x} $ is a basis of $ E\br{x} $. Then $ E \cong X \times \CC^r $, since
$$ \function{X \times \CC^r}{E}{\br{x, \br{v_1, \dots, v_r}}}{\sum_i v_is_i\br{x}} $$
is a biholomorphism. Then $ s_1, \dots, s_r $ is called a \textbf{holomorphic frame} for $ E $. Recall that for all $ E \to X $ and for all $ x \in X $ there exists an open $ U \ni x $ such that $ \eval{E}_U $ is trivial, so there exists a frame on $ U $ for $ \eval{E}_U $. This is called a \textbf{local frame} around $ x $.
\end{example}

\pagebreak

\begin{example}
Let $ X $ be a complex manifold of dimension $ n $, and let $ \br{z_1, \dots, z_n} $ be coordinates on $ \CC^n $. There exists an atlas $ \cbr{\br{U_\alpha, \phi_\alpha}} $ for $ \phi_\alpha : U_\alpha \to V_\alpha \subset \CC^n $. For all $ x \in U_\alpha $, $ \T_xU_\alpha \to \T_{\phi_\alpha\br{x}}V_\alpha $, and $ \T_{\phi_\alpha\br{x}}V_\alpha = \abr{\tpd{}{z_1}, \dots, \tpd{}{z_n}} $ is a frame of $ \T_{V_\alpha} $. Let
$$ \T_X = \bigcup_{x \in X} \T_xX, $$
and let $ \pi^{-1} : \T_X \to X $ such that $ \pi^{-1}\br{x} = \T_xX $. Then $ \T_X $ is a holomorphic vector bundle of rank $ n $ called the \textbf{tangent bundle}, where $ \UUU = \cbr{U_\alpha} $ and
$$ \psi_\alpha : \pi^{-1}\br{U_\alpha} = \eval{\T_X}_{U_\alpha} \to \eval{\T_{\CC^n}}_{V_\alpha} \cong V_\alpha \times \CC^r \to U_\alpha \times \CC^r $$
defines the trivialisation. The \textbf{cotangent bundle} of $ X $ is
$$ \Omega_X^1 = \T_X^*, $$
and let
$$ \Omega_X^p = \ext^p\Omega_X^1, \qquad p \ge 1. $$
A holomorphic $ p $-form on $ X $ is a section of $ \Omega_X^p $. \footnote{Exercise}
\end{example}

\subsection{Complexification of tangent bundles}

\lecture{10}{Thursday}{30/01/20}

Let $ X $ be a complex manifold. How to view $ X $ as a differentiable manifold? Let $ V $ be a vector space of dimension $ m $ over $ \RR $. An \textbf{almost complex structure} on $ V $ is a linear map $ J : V \to V $ such that $ J^2 = -\id_V $. If $ V $ admits an almost complex structure, then $ V $ can be seen as a vector space over $ \CC $. Let $ \lambda = a + ib $ for $ a, b \in \RR $, and let $ v \in V $. Define
$$ \lambda v = av + bJ\br{v}. $$
If $ \lambda_1, \lambda_2 \in \CC $, then $ \lambda_1\br{\lambda_2 v} = \br{\lambda_1\lambda_2}v $. \footnote{Exercise} Let $ v_1, \dots, v_n \in V $ be a basis over $ \CC $. Then
$$ v_1, \dots, v_n, J\br{v_1}, \dots, J\br{v_n} $$
is a basis of $ V $ over $ \RR $. The idea is to assume that $ a_i, b_i \in \RR $ such that $ \sum_i a_iv_i + \sum_i b_iJ\br{v_i} = 0 $, then
$$ 0 = \sum_i a_iv_i + \sum_i b_iJ\br{v_i} = \sum_i \br{a_iv_i + b_iJ\br{v_i}} = \sum_i \br{a_i + ib_i}v_i, $$
so $ a_i + ib_i = 0 $ for all $ i $. Thus $ a_i = b_i = 0 $, so $ m = 2n $. On a vector space an almost complex structure is a complex structure. Let $ V $ be a vector space of dimension $ 2n $ over $ \RR $. Then the \textbf{complexification} $ V_\CC = V \otimes_\RR \CC $ of $ V $ is a $ \CC $-vector space of dimension $ 2n $ over $ \CC $, where
$$ \function[\lambda]{V_\CC}{V_\CC}{v \otimes \mu}{v \otimes \mu\lambda}, \qquad \lambda \in \CC. $$
Let $ J $ be an almost complex structure on $ V $. Then we can extend $ J $ to a linear map
$$ \function[J]{V_\CC}{V_\CC}{v \otimes \mu}{J\br{v} \otimes \mu}, $$
such that $ J^2 = -\id_{V_\CC} $, \footnote{Exercise} so $ J^2 + \id_{V_\CC} = 0 $. Thus the eigenvalues of $ J $ on $ V_\CC $ are $ \pm i $. Let $ V^{1, 0} $ be the eigenspace for $ i $ and $ V^{0, 1} $ be the eigenspace for $ -i $, so
$$ V_\CC = V^{1, 0} \oplus V^{0, 1}. $$
The \textbf{conjugation}
$$ \function[\overline{\cdot}]{V_\CC}{V_\CC}{v \otimes \mu}{v \otimes \overline{\mu}} $$
on $ V_\CC $ in linear over $ \RR $, such that $ \overline{V^{1, 0}} = V^{0, 1} $ and $ \overline{V^{0, 1}} = V^{1, 0} $, \footnote{Exercise} so $ V^{1, 0} $ and $ V^{0, 1} $ are $ \CC $-vector spaces of dimension $ n $.

\pagebreak

\begin{example}
Let $ W = \CC^n $ with coordinates $ \br{z_1, \dots, z_n} $, and let $ z_j = x_j + iy_j $ with coordinates $ \br{x_1, y_1, \dots, x_n, y_n} $ for $ \RR^{2n} $. Define
$$ \function[J]{\RR^{2n}}{\RR^{2n}}{\br{x_1, y_1, \dots, x_n, y_n}}{\br{-y_1, x_1, \dots, -y_n, x_n}}. $$
Then $ J^2 = \id_{\RR^{2n}} $, and $ J $ is the \textbf{standard almost complex structure} on $ \RR^{2n} $. Let $ V = \RR^{2n} $, so $ V_\CC \cong \CC^{2n} $ with complex coordinates $ \br{x_1, y_1, \dots, x_n, y_n} $. Then $ V^{0, 1} $ is spanned by $ x_j - iy_j $ and $ V^{1, 0} $ is spanned by $ x_j + iy_j $, where $ \overline{x_j + iy_j} = x_j - iy_j $ for $ j = 1, \dots, n $.
\end{example}

\begin{definition}
Let $ X $ be a differentiable manifold. A \textbf{real, or complex, vector bundle of rank $ r $} is a differentiable manifold $ E $ with a smooth morphism $ \pi : E \to X $ such that if $ K = \RR $, or $ K = \CC $, then there exists an open covering $ \UUU = \cbr{U_i} $ of $ X $ such that
\begin{itemize}
\item for all $ x \in X $, the fibre of $ \pi $, $ E\br{x} = \pi^{-1}\br{x} $, is a vector space of rank $ r $ over $ K $,
\item for all $ i $ there exists a diffeomorphism $ h_i $ such that
$$
\begin{tikzcd}
\pi^{-1}\br{U_i} \arrow{rr}{h_i} \arrow{dr}[swap]{\pi} & & U_i \times K^r \arrow{r}{\p_2} \arrow{dl}{\p_1} & K^r \\
& U_i & &
\end{tikzcd},
$$
and for all $ x $, $ \p_2 \circ h_i : E\br{x} \to K^r $ is an isomorphism of vector spaces.
\end{itemize}
\end{definition}

Pull-backs, sections, exterior powers, tensors, direct sums, frames, etc are the same as holomorphic vector bundles, where holomorphic becomes smooth and biholomorphic becomes diffeomorphic, and for all $ X $ there exists a tangent bundle $ \T_X $. Assume $ X $ is a complex manifold of dimension $ n $. Let $ \T_X $ be the holomorphic tangent bundle of $ X $. Then $ X $ is also a differentiable manifold of dimension $ 2n $, so let $ \T_{X, \RR} $ be the \textbf{real tangent bundle} of $ X $, which is a rank $ 2n $ vector bundle, and let
$$ \T_{X, \CC} = \T_{X, \RR} \otimes_\RR \CC $$
be the \textbf{complex tangent bundle} of $ X $, which is a non-holomorphic complex vector bundle of rank $ 2n $. Smooth morphisms of real or complex vector bundles are defined similarly as holomorphisms between holomorphic vector bundles such that the rank of the image is constant. Let $ X $ be a differentiable manifold of dimension $ m = 2n $. Then an \textbf{almost complex structure} on $ X $ is a smooth morphism between the real tangent bundle $ J : \T_{X, \RR} \to \T_{X, \RR} $ such that $ J^2 = -\id $. In particular, $ J\br{x} : \T_x^\RR X \to \T_x^\RR X $ is an almost complex structure for all $ x \in X $.

\lecture{11}{Thursday}{30/01/20}

\begin{proposition}
Let $ X $ be a complex manifold. Then the underlying differentiable manifold admits an almost complex structure $ J : \T_{X, \RR} \to \T_{X, \RR} $ such that $ J^2 = -\id $.
\end{proposition}

\begin{proof}
Let $ x \in X $, and let $ \br{U, \phi} $ be a complex chart around $ x $ such that
$$ \function[\phi]{U}{V}{x}{0}. $$
Fix holomorphic coordinates $ \br{z_1, \dots, z_n} $ on $ U $. The tangent bundle of $ X $ on $ U $ is trivial, with a local frame $ \tpd{}{z_1}, \dots, \tpd{}{z_n} $, so
$$ \eval{\T_X}_U \xrightarrow{\sim} \T_V = V \times \CC^n. $$
Define $ x_i = \Re z_i $ and $ y_i = \Im e_i $. Then $ \br{x_1, y_1, \dots, x_n, y_n} $ are smooth coordinates $ U \to \RR $ around $ x $, and $ \tpd{}{x_1}, \tpd{}{y_1}, \dots, \tpd{}{x_n}, \tpd{}{y_n} $ define a local smooth frame of $ \T_{X, \RR} $ on $ U $, so
$$ \eval{\T_{X, \RR}}_U \xrightarrow{\sim} \T_V = V \times \RR^{2n}. $$
In particular, there exists an almost complex structure $ J_U $ for $ \T_V \cong \eval{\T_{X, \RR}}_U $, so
$$ J_U : \eval{\T_{X, \RR}}_U \to \eval{\T_{X, \RR}}_U, \qquad J_U^2 = -\id. $$

\pagebreak

Let $ f : V \to V $ be a biholomorphism, so
$$
\begin{tikzcd}
& U \cap U' \arrow{dl}[swap]{\phi} \arrow{dr}{\phi} & \\
V \arrow{rr}[swap]{f} & & V
\end{tikzcd},
$$
and let $ z_1', \dots, z_n' $ be local holomorphic coordinates given by
$$ z_i' = f_i\br{z_1, \dots, z_n}, \qquad f_i = \p_i \circ f, $$
where $ \p_i : \CC^n \to \CC $ is the $ i $-th projection. Define
$$ x_i' = \Re z_i' = \Re f_i\br{z_1, \dots, z_n} = u_i\br{z_1, \dots, z_n}, \qquad y_i' = \Im z_i' = \Im f_i\br{z_1, \dots, z_n} = v_i\br{z_1, \dots, z_n}, $$
so $ f_j = u_j + iv_j $. The real Jacobian $ \J_f $ of $ f $ is given by the derivatives of $ u_j $ and $ v_j $ with respect to $ x_1, y_1, \dots, x_n, y_n $, a $ \br{2n \times 2n} $-matrix of $ n \times n $ blocks of $ 2 \times 2 $ blocks of
$$ \twobytwo{\tpd{u_j}{x_k}}{\tpd{u_j}{y_k}}{\tpd{v_j}{x_k}}{\tpd{v_j}{y_k}}. $$
These define the transition function of $ \T_{X, \RR} $. To show that $ J $ extends to $ X $, it is enough to show that $ J $ commutes with $ \J_f $ at each point, so
$$
\begin{tikzcd}
\eval{\T_{X, \RR}}_{U \cap U'} \arrow{r}{\J_f} \arrow{d}[swap]{J} & \eval{\T_{X, \RR}}_{U \cap U'} \arrow{d}{J} \\
\eval{\T_{X, \RR}}_{U \cap U'} \arrow{r}[swap]{\J_f} & \eval{\T_{X, \RR}}_{U \cap U'}
\end{tikzcd}.
$$
Since $ f_j $ is holomorphic $ \tpd{}{\overline{z_k}}f_j = 0 $ for all $ j $ and $ k $, so the Cauchy-Riemann equations
$$ \dpd{u_j}{x_k} - \dpd{v_j}{y_k} = 0, \qquad \dpd{v_j}{x_k} + \dpd{u_j}{y_k} = 0, $$
or
$$ \twobytwo{\tpd{u_j}{x_k}}{\tpd{u_j}{y_k}}{\tpd{v_j}{x_k}}{\tpd{v_j}{y_k}} = \twobytwo{\tpd{v_j}{y_k}}{\tpd{u_j}{y_k}}{-\tpd{u_j}{y_k}}{\tpd{v_j}{y_k}}, $$
hold. Since $ J $ is the standard almost complex structure on $ \RR^{2n} $, where $ x_j \mapsto y_j $ and $ y_j \mapsto -x_j $,
$$ J =
\begin{pmatrix}
0 & 1 & & & \\
-1 & 0 & & 0 & \\
& & \ddots & & \\
& 0 & & 0 & 1 \\
& & & -1 & 0
\end{pmatrix}.
$$
Check that $ \J_f $ commutes with $ J $. \footnote{Exercise}
\end{proof}

\begin{corollary}
Every complex manifold is orientable.
\end{corollary}

\begin{proof}
We prove that if $ \T_{X, \RR} $ admits an almost complex structure then $ X $ is an orientable manifold. For all $ x \in X $ choose the orientation on $ \T_x^\RR X $, a vector space of dimension $ 2n $ over $ \RR $, given by any ordered basis of the form
$$ v_1, \dots, v_n, J\br{v_1}, \dots, J\br{v_n}. $$
Assume that $ v_1, \dots, v_n, J\br{v_1}, \dots, J\br{v_n} $ and $ w_1, \dots, w_n, J\br{w_1}, \dots, J\br{w_n} $ are ordered bases. Show that the determinant of the matrix given by the change of basis is positive. \footnote{Exercise}
\end{proof}

\pagebreak

\subsection{Differential forms on complex tangent bundles}

Let $ X $ be a complex manifold. Then there exists an almost complex structure $ J : \T_{X, \RR} \to \T_{X, \RR} $ on $ X $. Then $ J $ extends to
$$ \function[J]{\T_{X, \CC}}{\T_{X, \CC}}{v \otimes \mu}{J\br{v} \otimes \mu}. $$
For all $ x $, $ J\br{x} $ has two eigenvalues $ \pm i $, so
$$ \T_{X, \CC} = \T_X^{1, 0} \oplus \T_X^{0, 1}, $$
which are complex vector bundles and \textbf{eigenbundles}. Locally $ \T_X^{1, 0} $ and $ \T_X^{0, 1} $ are spanned by the frames $ \tpd{}{z_1}, \dots, \tpd{}{z_n} $ and $ \tpd{}{\overline{z_1}}, \dots, \tpd{}{\overline{z_n}} $ respectively. Moreover there exists a conjugation
$$ \function{\T_{X, \CC}}{\T_{X, \CC}}{v \otimes \mu}{v \otimes \overline{\mu}} $$
over $ \RR $, such that $ \overline{\T_X^{1, 0}} = \T_X^{0, 1} $ and $ \overline{\T_X^{0, 1}} = \T_X^{1, 0} $. Let
$$ \Omega_{X, \CC}^1 = \T_{X, \CC}^* $$
be the dual of the complex vector bundle $ \T_{X, \CC} $. Then
$$ \Omega_{X, \CC}^1 = \Omega_{X, \RR}^1 \otimes_\RR \CC = \Omega_X^{1, 0} \oplus \Omega_X^{0, 1} = \br{\T_X^{1, 0}}^* \oplus \br{\T_X^{0, 1}}^*. $$

\begin{exercise*}
Let $ V $ and $ W $ be vector spaces. Show that
$$ \ext^k\br{V \oplus W} = \bigoplus_{p + q = k} \ext^pV \otimes \ext^qW $$
is a canonical isomorphism.
\end{exercise*}

\lecture{12}{Tuesday}{04/02/20}

Thus,
$$ \Omega_{X, \CC}^k = \ext^k\Omega_{X, \CC}^1 = \bigoplus_{p + q = k} \Omega_X^{p, q}, \qquad \Omega_X^{p, q} = \ext^p\Omega_X^{1, 0} \otimes \ext^q\Omega_X^{0, 1}, \qquad k \ge 0, $$
where $ \Omega_X^{p, q} $ is a complex vector bundle for any $ p $ and $ q $.

\begin{definition}
The sections of $ \Omega_X^{p, q} $ are called \textbf{$ \br{p, q} $-forms} on $ X $, or \textbf{forms of type $ \br{p, q} $}.
\end{definition}

Locally, let $ x \in X $, and let $ \br{U \ni x, \phi} $ be a holomorphic chart for $ \phi : U \xrightarrow{\sim} V \subset \CC^n $. A $ \br{p, q} $-form on $ U $ can be locally written as
$$ \omega = \sum_{I, J} \alpha_{I, J}\d z_I \wedge \d\overline{z_J} = \sum_{\abs{I} = p, \ \abs{J} = q} \alpha_{I, J}\d z_{i_1} \wedge \dots \wedge \d z_{i_p} \wedge \d\overline{z_{j_1}} \wedge \dots \wedge \d\overline{z_{j_q}}, $$
where $ \alpha_{I, J} $ are smooth functions on $ U $. Let $ X $ be a manifold. If $ E $ is a complex vector bundle then
$$ \Ci{X, E} = \cbr{\text{smooth sections of} \ E}. $$
The \textbf{differential}
$$ \d : \COC{X}{k} \to \COC{X}{k + 1} $$
satisfies the Leibnitz rule and $ \d^2 = 0 $, so $ \d\d\omega = 0 $. If $ \omega \in \CO{X}{p, q} $, then $ \d\omega \in \COC{X}{p + q + 1} $. Assume that locally $ \omega = \sum_{I, J} \alpha_{I, J}\d z_I \wedge \d\overline{z_J} $. Then
$$ \d\omega = \sum_{I, J} \d\alpha_{I, J}\d z_I \wedge \d\overline{z_J}, \qquad \d\alpha_{I, J} = \sum_{i = 1}^n \dpd{}{z_i}\alpha_{I, J}\d z_i + \sum_{i = 1}^n \dpd{}{\overline{z_i}}\alpha_{I, J}\d\overline{z_i}. $$

\pagebreak

Let
$$ \da\alpha_{I, J} = \sum_{i = 1}^n \dpd{}{z_i}\alpha_{I, J}\d z_i \in \CO{X}{1, 0}, \qquad \dab\alpha_{I, J} = \sum_{i = 1}^n \dpd{}{z_i}\alpha_{I, J}\d\overline{z_i} \in \CO{X}{0, 1}. $$
Then $ \d = \da + \dab $ for smooth functions. Back to $ \d\omega $. Then
$$ \d\omega = \sum_{I, J} \d\alpha_{I, J}\d z_I \wedge \d\overline{z_J} = \sum_{I, J} \da\alpha_{I, J}\d z_I \wedge \d\overline{z_J} + \sum_{I, J} \dab\alpha_{I, J}\d z_I \wedge \d\overline{z_J}. $$
Let
$$ \da\omega = \sum_{I, J} \da\alpha_{I, J}\d z_I \wedge \d\overline{z_J}, \qquad \dab\omega = \sum_{I, J} \dab\alpha_{I, J}\d z_I \wedge \d\overline{z_J}. $$
Then $ \d = \da + \dab $ for $ \omega $.

\begin{lemma}
The linear maps
$$ \da : \CO{X}{p, q} \to \CO{X}{p + 1, q}, \qquad \dab : \CO{X}{p, q} \to \CO{X}{p, q + 1} $$
satisfy the Leibnitz rule. That is, if $ \omega \in \CO{X}{p, q} $ and $ \eta \in \CO{X}{p', q'} $, then
$$ \da\br{\omega \wedge \eta} = \da\omega \wedge \eta + \br{-1}^{p + q}\omega + \da\eta, \qquad \dab\br{\omega \wedge \eta} = \dab\omega \wedge \eta + \br{-1}^{p + q}\omega + \dab\eta. $$
\end{lemma}

\begin{proof}
$ \d $ satisfies the Leibnitz rule
$$ \d\br{\omega \wedge \eta} = \d\omega \wedge \eta + \br{-1}^{p + q}\omega \wedge \d\eta, $$
since $ \omega \in \COC{X}{p + q} $, so
\begin{align*}
\da\br{\omega \wedge \eta} + \dab\br{\omega \wedge \eta}
& = \br{\da\omega + \dab\omega} \wedge \eta + \br{-1}^{p + q}\omega \wedge \br{\da\eta + \dab\eta} \\
& = \br{\da\omega \wedge \eta + \br{-1}^{p + q}\omega \wedge \da\eta} + \br{\dab\omega \wedge \eta + \br{-1}^{p + q}\omega \wedge \dab\eta}.
\end{align*}
Then $ \da\br{\omega \wedge \eta} $ and $ \da\omega \wedge \eta + \br{-1}^{p + q}\omega \wedge \da\eta $ are $ \br{p + 1, q} $-forms, and $ \dab\br{\omega \wedge \eta} $ and $ \dab\omega \wedge \eta + \br{-1}^{p + q}\omega \wedge \dab\eta $ are $ \br{p, q + 1} $-forms. Forms of the same type in the decomposition of $ \d\br{\omega \wedge \eta} $ must coincide.
\end{proof}

\subsection{Dolbeault cohomology}

\begin{lemma}
$ \da^2 = \dab^2 = \dab\da + \da\dab = 0 $.
\end{lemma}

\begin{proof}
Let $ \omega \in \CO{X}{p, q} $. Because $ \d^2 = 0 $,
$$ 0 = \d^2\omega = \br{\da + \dab}\br{\br{\da + \dab}\omega} = \da^2\omega + \da\dab\omega + \dab\da\omega + \dab^2\omega. $$
Then $ \d^2\omega $ is a $ \br{p + q + 2} $-form, $ \da^2\omega $ is a $ \br{p + 2, q} $-form, $ \da\dab\omega + \dab\da\omega $ is a $ \br{p + 1, q + 1} $-form, and $ \dab^2\omega $ is a $ \br{p, q + 2} $-form. Forms of the same type in the decomposition of $ \d^2\omega $ must coincide.
\end{proof}

Let $ X $ be a complex manifold. Fix $ p, q \ge 0 $. Let
\begin{align*}
\ZZZ^{p, q}\br{X}
& = \ker \br{\dab : \CO{X}{p, q} \to \CO{X}{p, q + 1}} \\
& = \cbr{\omega \in \CO{X}{p, q} \st \dab\omega = 0}
\end{align*}
and let
\begin{align*}
\BBB^{p, q}\br{X}
& = \im \br{\dab : \CO{X}{p, q - 1} \to \CO{X}{p, q}} \\
& = \cbr{\omega \in \CO{X}{p, q} \st \exists \eta \in \CO{X}{p, q - 1}, \ \omega = \dab\eta}.
\end{align*}
Since $ \dab^2 = 0 $ we have $ \BBB^{p, q}\br{X} \subset \ZZZ^{p, q}\br{X} $ for all $ p $ and $ q $. The \textbf{Dolbeault cohomology group} of $ X $ is
$$ \H^{p, q}\br{X} = \ZZZ^{p, q}\br{X} / \BBB^{p, q}\br{X}. $$

\pagebreak

\begin{exercise*}
Assume $ X $ and $ Y $ are biholomorphic complex manifolds. Then
$$ \H^{p, q}\br{X} = \H^{p, q}\br{Y}. $$
\end{exercise*}

If $ \H^{p, q}\br{X} $ is finite dimensional then we define the \textbf{Hodge numbers} of $ X $ as
$$ \h^{p, q}\br{X} = \dim_\CC \H^{p, q}\br{X}. $$

\lecture{13}{Thursday}{06/02/20}

Our goal is if $ X $ is K\"ahler and compact
$$ \bigoplus_{p + q = k} \H^{p, q}\br{X} = \H^{p + q}\br{X}, $$
as the de Rham cohomology. In particular this is true if $ X $ is projective. How to compute $ \H^{p, q}\br{X} $? We need to use analysis.

\begin{proposition}
Let $ X $ be a complex manifold. Then there exists an isomorphism
$$ \H^{p, 0}\br{X} \cong \H^0\br{X, \Omega_X^p} = \cbr{\text{holomorphic sections of} \ \Omega_X^p} = \cbr{\text{holomorphic $ p $-forms on} \ X}, \qquad p \ge 0. $$
\end{proposition}

\begin{remark*}
If $ X $ is compact then
$$ \H^{0, 0}\br{X} = \CC, $$
because $ \H^{0, 0}\br{X} = \H^0\br{X, \OOO_X} $ are constants.
\end{remark*}

\begin{proof}
$$ \H^{p, 0}\br{X} = \ZZZ^{p, 0}\br{X} / \BBB^{p, 0}\br{X} = \ZZZ^{p, 0}\br{X} = \cbr{\omega \in \CO{X}{p, 0} \st \dab\omega = 0}. $$
Locally $ \omega = \sum_{\abs{I} = p} \alpha_I\d z_I $. Then
$$ \dab\omega = \sum_{\abs{I} = p} \dab\alpha_I\d z_I = \sum_{\abs{I} = p} \sum_{i = 1}^n \dpd{}{\overline{z_j}}\alpha_I\d\overline{z_j} \wedge \d z_I, $$
where $ \d\overline{z_j} \wedge \d z_I $ are linearly independent. For all $ I $ and for all $ j $, the Cauchy-Riemann equations $ \tpd{}{\overline{z_j}}\alpha_I = 0 $ hold, so for all $ I $, $ \alpha_I $ is holomorphic. Then $ \omega = \sum_{\abs{I} = p} \alpha_I\d z_I $ is a holomorphic $ p $-form, so $ \omega \in \H^0\br{X, \Omega_X^p} $.
\end{proof}

\pagebreak

\section{Connection, curvature, and metric}

\subsection{Connections}

Let $ X $ be a differentiable manifold, and let $ E $ be a complex vector bundle. Then
$$ \Ci{X, E} = \cbr{\text{$ \C^\infty $-sections of} \ E}. $$
Is there a way to compute the derivatives of these sections?

\begin{definition}
Let $ X $ and $ E $ be as above. A \textbf{connection} of $ E $ is a $ \CC $-linear map
$$ \nabla : \Ci{X, E} \to \COCE{X}{1}{E} $$
such that the Leibnitz rule holds, so
$$ \nabla\br{f\omega} = f \cdot \nabla\omega + \d f \otimes \omega, \qquad f \in \Ci{X}, \qquad \omega \in \Ci{X, E}. $$
\end{definition}

The following is the idea. Let $ \omega \in \Ci{X, E} $. Then
$$ \nabla\omega = \sum_i \eta_i \otimes \omega_i, $$
where $ \eta_i $ are $ 1 $-forms on $ X $ and $ \omega_i $ are sections of $ E $. Let $ x \in X $, and let $ v \in \T_xX $. Then
$$ \nabla_v\omega_x = \sum_i \eta_i\br{v}\omega_i $$
is a section of $ E $ at $ x $. The goal is to study connections locally. Let $ x \in X $, and let $ \br{U, \phi} $ be a chart around $ x $ that trivialises $ E $, so $ \pi^{-1}\br{U} = U \times \CC^r $ for $ \pi : E \to X $ and $ r = \rk E $. Then there exists a frame $ s_1, \dots, s_r \in \Ci{U, E} $ of $ E $ on $ U $. Let $ \sigma \in \Ci{X, E} $ be any section. Locally on $ U $ we write
$$ \sigma \overset{U}{=} f = \br{f_1, \dots, f_r}, \qquad \sigma = \sum_{i = 1}^r f_is_i, \qquad f_1, \dots, f_r \in \Ci{U}. $$
By the Leibnitz rule, on $ U $,
$$ \nabla\sigma = \sum_{i = 1}^r \nabla\br{f_is_i} = \sum_{i = 1}^r \br{f_i \cdot \nabla s_i + \d f_i \otimes s_i} \in \COCE{U}{1}{E}. $$

\begin{notation*}
$ \d f = \br{\d f_1, \dots, \d f_r} $.
\end{notation*}

Then
$$ \nabla s_j = \sum_{i = 1}^r a_{ij} \otimes s_i, \qquad a_{ij} \in \CO{U}{1}. $$

\begin{notation*}
$ A = \br{a_{ij}} $ is an $ \br{r \times r} $-matrix with coefficients $ 1 $-forms.
\end{notation*}

With this notation, this becomes
$$ \nabla\sigma \overset{U}{=} A \cdot f + \d f. $$
\begin{itemize}
\item $ A $ depends very much on the choice of the frame.
\item Locally on $ U $, $ \nabla $ is determined by $ A $.
\end{itemize}
Consider another chart $ \br{U', \phi'} $ which also gives a trivialisation of $ E $. So we can choose a corresponding frame $ s_1', \dots, s_r' $. Assume $ \sigma \in \Ci{U \cap U', E} $. Then
$$ \sigma \overset{U'}{=} f' = \br{f_1', \dots, f_r'}, \qquad \sigma = \sum_{j = 1}^r f_j's_j', \qquad f_1', \dots, f_r' \in \Ci{U}. $$
Let $ A' $ be the matrix with respect to this frame. Then
$$ \nabla\sigma \overset{U'}{=} A' \cdot f' + \d f'. $$

\pagebreak

\lecture{14}{Thursday}{06/02/20}

Let
$$ g : \br{U \cap U'} \times \CC^r \to \br{U \cap U'} \times \CC^r $$
be the transition function from the trivialisation of $ U' $ to the trivialisation of $ U $. Then $ g\br{x} \in \GL_r \CC $ for all $ x \in U \cap U' $, and $ f = g \cdot f' $. Denote by $ \D g $ the differential of $ g $. Then
$$ \d f = \d\br{g \cdot f'} = \D g \cdot f' + g \cdot \d f' = g \cdot \br{g^{-1} \cdot \D g \cdot f' + \d f'}, $$
by the Leibnitz rule. Thus,
\begin{align*}
A' \cdot f' + \d f'
& \overset{U'}{=} A \cdot f + \d f
\overset{U}{=} A \cdot g \cdot f' + g \cdot \br{g^{-1} \cdot \D g \cdot f' + \d f'}
\overset{U}{=} g \cdot \br{\br{g^{-1} \cdot \D g + g^{-1} \cdot A \cdot g}f' + \d f'} \\
& \overset{U'}{=} \br{g^{-1} \cdot \D g + g^{-1} \cdot A \cdot g} \cdot f' + \d f',
\end{align*}
so
$$ A' = g^{-1} \cdot \D g + g^{-1} \cdot A \cdot g. $$

\subsection{Curvature operators}

What is $ \nabla^2 $? The idea is that
$$ \Ci{X, E} \xrightarrow{\nabla} \COCE{X}{1}{E} \xrightarrow{\nabla} \COCE{X}{1}{\Omega_{X, \CC}^1 \otimes E} \xrightarrow{\wedge} \COCE{X}{2}{E}. $$
The \textbf{curvature tensor} is
$$ \nabla^2 : \Ci{X, E} \to \COCE{X}{2}{E}. $$

\begin{remark*}
If $ X $ has dimension one, then $ \Omega_{X, \CC}^2 = 0 $, so $ \nabla^2 = 0 $.
\end{remark*}

Again for all $ x \in X $, take $ U $ as above. Let $ s_1, \dots, s_r $ be a frame, let $ A = \br{a_{ij}} $ be the $ \br{r \times r} $-matrix of $ 1 $-forms, and let $ \D A $ be the differential of $ A $.

\begin{notation*}
$ A \wedge A = \br{\sum_{k = 1}^r a_{ik} \wedge a_{kj}} $ is an $ \br{r \times r} $-matrix of $ 2 $-forms.
\end{notation*}

Let $ \sigma \overset{U}{=} \br{f_1, \dots, f_r} = \sum_i f_is_i $ on $ U $. Then
\begin{align*}
\nabla^2\sigma
& = \nabla\br{A \cdot f + \d f}
= A \wedge \br{A \cdot f + \d f} + \d\br{A \cdot f + \d f} \\
& = A \wedge A \cdot f + A \wedge \d f + \D A \cdot f - A \wedge \d f + \d^2f
= \br{A \wedge A + \D A} \cdot f
\end{align*}
is $ \C^\infty $-linear, so $ \nabla^2\br{h\sigma} = h\nabla^2\sigma $. The \textbf{curvature operator} is
$$ \Theta_\nabla \overset{U}{=} A \wedge A + DA, $$
so $ \Theta_\nabla\br{\sigma} = \nabla^2\sigma $.

\subsection{Hermitian metrics}

\begin{definition}
Let $ V $ be a vector space over $ \CC $. A \textbf{Hermitian inner product} on $ V $ is a map
$$ \function{V \times V}{\CC}{\br{v, w}}{\abr{v, w}}, $$
such that
\begin{itemize}
\item $ \abr{v, w} = \overline{\abr{w, v}} $,
\item it is linear on the first factor, and
\item $ \abr{v, v} \ge 0 $ and $ \abr{v, v} = 0 $ if and only if $ v = 0 $.
\end{itemize}
\end{definition}

\begin{example*}
$ V = \CC $ and $ \abr{z_1, z_2} = z_1 \cdot \overline{z_2} $.
\end{example*}

\pagebreak

\begin{definition}
Let $ X $ be a manifold, and let $ E $ be a complex vector bundle on $ X $. A \textbf{Hermitian metric} $ h $, or $ \abr{\cdot, \cdot} $, on $ E $ is a choice of a Hermitian inner product
$$ h_x = \abr{\cdot, \cdot}_x : E\br{x} \times E\br{x} \to \CC, \qquad x \in X, $$
such that for any open set $ U \subset X $ and for $ s, t \in \Ci{U, E} $, $ \abr{s\br{x}, t\br{x}}_x $ is a $ \C^\infty $-function with respect to $ x $ on $ U $. The pair $ \br{E, \abr{\cdot, \cdot}} = \br{E, h} $ is called a \textbf{Hermitian vector bundle}.
\end{definition}

Let $ \br{E, h} $ be a Hermitian vector bundle, and let $ x \in X $. Locally, let $ s_1, \dots, s_r $ be a frame on $ U \ni x $. For any $ x \in U $, $ \abr{s_i\br{x}, s_j\br{x}}_x = h_{ij}\br{x} $ is a smooth function for all $ i $ and $ j $, so
$$ H = \br{h_{ij}}_{i, j = 1}^r $$
is an $ \br{r \times r} $-matrix of smooth functions. Let $ \sigma, \sigma' \in \Ci{U, E} $, and let $ \sigma \overset{U}{=} f = \br{f_1, \dots, f_r} $ and $ \sigma' \overset{U}{=} f' = \br{f_1', \dots, f_r'} $. Then
$$ \abr{\sigma\br{x}, \sigma'\br{x}}_x = f^\intercal \cdot H \cdot \overline{f'}. $$
Now assume that $ U' $ is a different open set with frame $ \br{s_1', \dots, s_r'} $. Assume
$$ g : \br{U \cap U'} \times \CC^r \to \br{U \cap U'} \times \CC^r $$
is the transition function from the trivialisation on $ U' $ to the trivialisation on $ U $. Let $ H' $ be the matrix of $ h $ with respect to $ s_1', \dots, s_r' $. Then
$$ H' = g^\intercal \cdot H \cdot \overline{g}. $$

\begin{proposition}
\label{prop:5.4}
Let $ \pi : E \to X $ be a complex vector bundle on $ X $. Then $ E $ always admits a Hermitian metric.
\end{proposition}

Before proving Proposition \ref{prop:5.4}, we recall the definition of a partition of the unity.

\begin{definition}
Let $ M $ be a manifold and let $ \UUU = \cbr{U_\alpha} $ be an open covering. A \textbf{partition of unity} with respect to $ \UUU $ is a collection of smooth functions $ f_\alpha : M \to \sbr{0, 1} $ such that
\begin{itemize}
\item $ \supp f_\alpha \subset U_\alpha $ for all $ \alpha $, in particular, $ f_\alpha = 0 $ outside $ U_\alpha $,
\item $ \sum_\alpha f_\alpha\br{x} = 1 $ for all $ x \in M $, and
\item for all $ x \in M $, there exists an open neighbourhood $ V $ of $ x $ such that $ \supp f_\alpha \cap V \ne 0 $ for only finitely many $ \alpha $.
\end{itemize}
\end{definition}

It can be shown that if $ M $ is a manifold and $ \UUU = \cbr{U_\alpha} $ is an open cover of $ M $, then there exists a partition of the unity $ \cbr{f_\alpha} $ with respect to such a cover.

\begin{proof}
Let $ \UUU = \cbr{U_i} $ be an open cover of open sets of $ X $, trivialising $ E $, so $ \phi_i : \pi^{-1}\br{U_i} \xrightarrow{\sim} U_i \times \CC^r $, and let $ f_i : X \to \sbr{0, 1} $ be a partition of unity with respect to $ \UUU $. For each $ i $, consider a Hermitian metric on $ \CC^r $. Then there is a Hermitian metric $ \widetilde{h_i} $ on $ U_i \times \CC^r $. Let $ h_i $ be the Hermitian metric on $ \eval{E}_{U_i} $ induced by $ \phi_i $. Take $ h = \sum_i f_ih_i $. Check that $ h $ defines a Hermitian metric on $ X $. \footnote{Exercise}
\end{proof}

\lecture{15}{Tuesday}{11/02/20}

Let $ E \to X $ be a complex Hermitian vector bundle of rank $ r $. Fix $ p, q \ge 0 $. There exists a bilinear \textbf{cup product}
$$ \function[\cbr{\cdot, \cdot}]{\COCE{X}{p}{E} \times \COCE{X}{q}{E}}{\COC{X}{p + q}}{\br{\sigma, \tau}}{\cbr{\sigma, \tau}}, $$
where $ \cbr{\sigma, \tau} $ is defined as follows. Let $ x \in X $, let $ s_1, \dots, s_r $ be a frame of $ E $ around $ x $, let $ H $ be the matrix associated to the Hermitian metric with respect to the frame, and let
$$ \sigma = \sum_i \sigma_i \otimes s_i, \qquad \tau = \sum_i \tau_i \otimes s_i, \qquad \sigma_i \in \COC{X}{p}, \qquad \tau_i \in \COC{X}{q}. $$

\pagebreak

Then we define, around $ x $,
$$ \cbr{\sigma, \tau} = \sigma^\intercal \cdot H \cdot \overline{\tau} = \sum_{i, j = 1}^r h_{ij}\sigma_i \wedge \overline{\tau_j}. $$
This is uniquely defined, and does not depend on the frame, so it extends to $ X $. In particular $ \cbr{\sigma, \tau} $ is a smooth $ \br{p + q} $-form.

\begin{definition}
Let $ E $ be a complex Hermitian vector bundle on $ X $, and let $ \nabla $ be a connection on $ E $. We say that $ \nabla $ is \textbf{Hermitian}, or \textbf{compatible with the metric}, if the Leibnitz rule holds, so we have
$$ \d\cbr{\sigma, \tau} = \cbr{\nabla\sigma, \tau} + \br{-1}^p\cbr{\sigma, \nabla\tau}, \qquad \sigma \in \COCE{X}{p}{E}, \qquad \tau \in \COCE{X}{q}{E}. $$
\end{definition}

Let $ x \in X $, and let $ s_1, \dots, s_r $ be a local frame of $ E $. Assume $ s_1, \dots, s_r $ is an orthonormal frame around $ x \in X $. Let $ \nabla $ be a connection compatible with the metric, and let $ A $ be the associated matrix with respect to $ s_1, \dots, s_r $. Gram-Schmidt is an algorithm that gives an orthonormal basis of $ E\br{x} $ for all $ x $, which is $ \C^\infty $, say $ s_1', \dots, s_r' $. Then with respect to this frame $ H = \id_r $ because $ \abr{s_i', s_j'}_x = \delta_{ij} $.

\begin{proposition}
$ A $ is anti-autodual, that is
$$ \overline{A}^\intercal = -A. $$
\end{proposition}

\begin{proof}
Let $ \sigma $ and $ \tau $ be as before, and let $ \sigma_1, \dots, \sigma_r $ and $ \tau_1, \dots, \tau_r $ be the components of $ \sigma $ and $ \tau $ with respect to the frame $ s_1, \dots, s_r $. Then $ \cbr{\sigma, \tau} = \sigma^\intercal \wedge \overline{\tau} $. Since $ \nabla $ is Hermitian, the Leibnitz rule holds, so
$$ \d\cbr{\sigma, \tau} = \d\br{\sigma^\intercal \wedge \overline{\tau}} = \d\sigma^\intercal \wedge \overline{\tau} + \br{-1}^p\sigma^\intercal \wedge \d\overline{\tau}, $$
by the usual Leibnitz rule for $ \d $. Then
$$ \cbr{\nabla\sigma, \tau} = \cbr{A \wedge \sigma + \d\sigma, \tau} = \cbr{A \wedge \sigma, \tau} + \cbr{\d\sigma, \tau} = \br{A \wedge \sigma}^\intercal \wedge \overline{\tau} + \d\sigma^\intercal \wedge \overline{\tau} = \br{-1}^p\sigma^\intercal \wedge A^\intercal \wedge \overline{\tau} + \d\sigma^\intercal \wedge \overline{\tau}, $$
and
$$ \cbr{\sigma, \nabla\tau} = \sigma^\intercal \wedge \overline{\nabla\tau} = \sigma^\intercal \wedge \br{\overline{A \wedge \tau + \d\tau}} = \sigma^\intercal \wedge \overline{A} \wedge \overline{\tau} + \sigma^\intercal \wedge \d\overline{\tau}. $$
By the Leibnitz rule,
$$ \sigma^\intercal \wedge \br{A^\intercal + \overline{A}} \wedge \overline{\tau} = 0. $$
This is true for all $ \sigma $ and $ \tau $, so $ A^\intercal + \overline{A} = 0 $.
\end{proof}

\begin{exercise*}
Let $ s_1, \dots, s_r $ be any frame, let $ H $ be the matrix given by the metric with respect to $ s_1, \dots, s_r $, and let $ A $ be the matrix given by the connection with respect to $ s_1, \dots, s_r $ where the connection is Hermitian. Then
$$ \D H = A^\intercal \cdot H + H \cdot \overline{A}, $$
where if $ H = \br{h_{ij}} $ then $ \D H = \br{\d h_{ij}} $. A hint is to do the same calculation.
\end{exercise*}

\begin{theorem}
If $ E \to X $ is a complex Hermitian vector bundle, then there exists a connection $ \nabla $ compatible with $ h $.
\end{theorem}

\subsection{Holomorphic vector bundles}

\lecture{16}{Thursday}{13/02/20}

\begin{proposition}
\label{prop:5.9}
Let $ X $ be a complex manifold, and let $ \pi : E \to X $ be a holomorphic vector bundle of rank $ r $. Then for all $ q \ge 0 $ there exists a $ \CC $-linear map
$$ \dab_E : \COE{X}{0, q}{E} \to \COE{X}{0, q + 1}{E}, $$
which satisfies the Leibnitz rule and $ \dab_E^2 = 0 $. Moreover if $ \sigma $ is a holomorphic section of $ \Omega_X^{0, q} \otimes E $ then $ \dab_E\sigma = 0 $.
\end{proposition}

\pagebreak

The idea is to do it locally in a canonical way, so does not depend on the choice of the trivialisation.

\begin{proof}
Let $ x \in X $. There exists a holomorphic frame $ s_1, \dots, s_r $ of $ E $ locally around $ x $ in $ U $. Let $ \sigma \in \COE{X}{0, q}{E} $. Then locally, $ \sigma \overset{U}{=} \sum_{i = 1}^r f_i \otimes s_i $ where $ f_i \in \Ci{U} $ are $ \br{0, q} $-forms locally around $ x $. We define
$$ \dab_Ex \overset{U}{=} \sum_{i = 1}^r \dab f_i \otimes s_i \in \COE{U}{0, q + 1}{E}. $$
We want to show that it can be extended to $ X $. Let $ U' \subset X $ be open, let $ s_1', \dots, s_r' $ be a holomorphic frame on $ U' $ of $ E $, and let
$$ g : \br{U \cap U'} \times \CC^r \to \br{U \cap U'} \times \CC^r $$
be the transition map from the trivialisation of $ U' $ to the trivialisation of $ U $. Then $ \sigma \overset{U}{=} \sum_{i = 1}^r f_i' \otimes s_i' $, and
$$ \dab_Ex \overset{U'}{=} \sum_{i = 1}^r \dab f_i' \otimes s_i'. $$
Since $ g $ is holomorphic, that is $ \dab g = 0 $, this implies that $ \dab_E $ on $ U $ coincides with $ \dab_E $ on $ U' $. Recall for $ \nabla $ the change of frame was
$$ A' = g^{-1} \cdot \D g + g^{-1} \cdot A \cdot g, $$
so $ \dab_E $ extends to $ X $. Let $ \sigma $ be a holomorphic section of $ \Omega_X^{0, q} \otimes E $. Then, on $ U $ if $ s_i $ and $ f_i $ are as before, then $ f_i $ are holomorphic $ \br{0, q} $-forms. Thus $ \dab f_i = 0 $, so $ \dab_E\sigma = 0 $.
\end{proof}

Vice versa if $ \nabla : \Ci{X, E} \to \COCE{X}{1}{E} $ is a connection and $ X $ is a complex manifold, then
$$ \Omega_{X, \CC}^1 \xrightarrow{\sim} \Omega_X^{1, 0} \oplus \Omega_X^{0, 1}, \qquad \Omega_{X, \CC}^1 \otimes E = \br{\Omega_X^{1, 0} \otimes E} \oplus \br{\Omega_X^{0, 1} \otimes E}. $$
Then for all $ \sigma $,
$$ \nabla\sigma = \nabla^{1, 0}\sigma + \nabla^{0, 1}\sigma, $$
where
$$ \nabla^{1, 0} : \Ci{X, E} \to \COE{X}{1, 0}{E}, \qquad \nabla^{0, 1} : \Ci{X, E} \to \COE{X}{0, 1}{E}. $$

\begin{theorem}
Assume $ X $ is a complex manifold and $ E $ is a holomorphic Hermitian vector bundle of rank $ r $. Then there exists a unique connection
$$ \nabla_E : \Ci{X, E} \to \COCE{X}{1}{E}, $$
such that $ \nabla_E^{0, 1} = \dab_E $, which defined in Proposition \ref{prop:5.9}, and $ \nabla_E $ is compatible with $ h $.
\end{theorem}

Then $ \nabla_E $ is called the \textbf{Chern connection} and $ \nabla_E^2 $ is called the \textbf{Chern curvature}.

\begin{proof}
Fix $ x \in X $, on $ U \ni x $. There exists a local holomorphic frame $ s_1, \dots, s_r $. Let $ H $ be the matrix defining the metric $ h $ on $ U $, so $ H = \br{h_{ij}} $ is an $ \br{r \times r} $-matrix for $ h_{ij} \in \Ci{U} $. Define the $ \br{r \times r} $-matrix $ \da H = \br{\da h_{ij}} $ for $ \da h_{ij} \in \CO{U}{1, 0} $. We define
$$ A = \overline{H}^{-1} \cdot \da\overline{H}, $$
an $ \br{r \times r} $-matrix of $ 1 $-forms on $ U $. This $ A $ will be the matrix defining $ \nabla_E $.
\begin{itemize}
\item Let $ \sigma \overset{U}{=} \sum_i f_is_i \in \Ci{U, E} $ where $ f_i \in \Ci{U} $. Then
$$ \nabla_E\sigma \overset{U}{=} A \cdot f + \d f. $$
Let $ A = \br{a_{ij}} $ where by definition of $ A $, $ a_{ij} $ are $ \br{1, 0} $-forms. Thus
$$ \nabla_E^{0, 1}\sigma = A^{0, 1} \cdot f + \dab f \overset{U}{=} \dab_E\sigma. $$

\pagebreak

\item Recall that $ \nabla $ associated to $ A $ is compatible with $ h $ if and only if $ \D H = A^\intercal \cdot H + H \cdot \overline{A} $. Since $ H $ is Hermitian, it follows that $ H^\intercal = \overline{H} $, so
$$ A^\intercal \cdot H = \br{\overline{H}^{-1} \cdot \da\overline{H}}^\intercal \cdot H = \br{\da\overline{H}}^\intercal \cdot \br{\overline{H}^{-1}}^\intercal \cdot H = \da H \cdot H^{-1} \cdot H = \da H, $$
and
$$ H \cdot \overline{A} = H \cdot \overline{\overline{H}^{-1} \cdot \da\overline{H}} = H \cdot H^{-1} \cdot \dab H = \dab H. $$
Thus
$$ \D H = \br{\d h_{ij}} = \br{\da h_{ij} + \dab h_{ij}} = \da H + \overline{\da H} = A^\intercal \cdot H + H \cdot \overline{A}, $$
so on $ U $, $ \nabla_E $ is compatible with $ h $.

\lecture{17}{Thursday}{13/02/20}

\item Let $ \nabla $ be another connection satisfying $ \nabla^{0, 1} = \dab_E $ and $ \nabla $ is compatible with $ h $. As before $ s_1, \dots, s_r $ is the local holomorphic frame on $ U $. Let $ B = \br{b_{ij}} $ be the $ \br{r \times r} $-matrix of $ 1 $-forms associated to $ \nabla $, and let $ B = B^{1, 0} + B^{0, 1} $, so $ b_{ij} = b_{ij}^{1, 0} + b_{ij}^{0, 1} $. For all $ f = \br{f_1, \dots, f_r} $ if $ \sigma = \sum_i f_is_i $ then
$$ \nabla\sigma \overset{U}{=} B \cdot f + \d f, $$
so
$$ B^{0, 1} \cdot f + \dab f \overset{U}{=} \nabla^{0, 1}\sigma = \dab_E\sigma \overset{U}{=} \dab f. $$
Then for all $ f $, $ B^{0, 1} \cdot f = 0 $, so $ B^{0, 1} = 0 $ and $ B = B^{1, 0} $. Since $ \nabla $ is compatible with $ h $, $ \D H = B^\intercal \cdot H + H \cdot \overline{B} $, so $ \overline{\D H} = \overline{B^\intercal} \cdot \overline{H} + \overline{H} \cdot B $. Then
$$ B = B^{1, 0} = \br{\overline{H}^{-1} \cdot \br{\overline{\D H} - \overline{B^\intercal} \cdot \overline{H}}}^{1, 0} = \overline{H}^{-1} \cdot \da\overline{H} + \overline{H}^{-1} \cdot 0 \cdot \overline{H} = \overline{H}^{-1} \cdot \da\overline{H} = A, $$
since $ \overline{\D H}^{1, 0} = \overline{\br{\d h_{ij}}}^{1, 0} = \br{\d \overline{h_{ij}}}^{1, 0} = \da\overline{h_{ij}} $ and $ \br{\overline{B}}^{1, 0} = \overline{B^{1, 0}} = 0 $, so $ \nabla = \nabla_E $. We define in the same way $ \nabla_E^U $ on any open $ U $ of $ X $. On $ U \cap U' $, by unicity $ \nabla_E^U = \nabla_E^{U'} $. Thus $ \nabla_E $ can be extended to $ X $.
\end{itemize}
\end{proof}

\begin{corollary}
\label{cor:5.11}
Let $ X $ be a complex manifold, let $ \br{E, h} $ be a Hermitian vector bundle on $ X $, let $ \nabla_E $ be the Chern connection, and let $ \Theta_E = \nabla_E^2 $ be the Chern curvature. Locally at $ x \in U $, let $ s_1, \dots, s_r $ be a holomorphic frame, and let $ A $ be the matrix associated to $ \nabla_E $. Then
\begin{enumerate}
\item $ A $ is of type $ \br{1, 0} $ and $ \da A = -A \wedge A $,
\item $ \Theta_E = \dab A $ is of type $ \br{1, 1} $, and
\item $ \dab\Theta_E = 0 $.
\end{enumerate}
\end{corollary}

\begin{proof}
\hfill
\begin{enumerate}
\item Let $ H $ be as above. Recall $ A = \overline{H}^{-1} \cdot \da\overline{H} $ is a $ \br{1, 0} $-form matrix. Then
$$ 0 = \da I = \da\br{\overline{H} \cdot \overline{H}^{-1}} = \overline{H} \cdot \da\overline{H}^{-1} + \da\overline{H} \cdot \overline{H}^{-1}, $$
so
\begin{align*}
\da A
& = \da\br{\overline{H}^{-1} \cdot \da\overline{H}}
= \da\overline{H}^{-1} \wedge \da\overline{H} + \overline{H}^{-1} \cdot \da^2\overline{H} \\
& = -\br{\overline{H}^{-1} \cdot \da\overline{H} \cdot \overline{H}^{-1}} \wedge \da\overline{H}
= -\br{\overline{H}^{-1} \cdot \da\overline{H}} \wedge \br{\overline{H}^{-1} \cdot \da\overline{H}}
= -A \wedge A.
\end{align*}
\item By $ 1 $,
$$ \Theta_E = A \wedge A + \D A = A \wedge A + \da A + \dab A = \dab A. $$
\item By $ 2 $,
$$ \dab\Theta_E = \dab\dab A = 0. $$
\end{enumerate}
\end{proof}

\pagebreak

\begin{lemma}
Let $ X $ be a complex manifold of dimension $ n $, let $ \br{E, h} $ be a Hermitian vector bundle on $ X $ of rank $ r $, let $ \nabla_E $ be the Chern connection compatible with $ h $ such that $ \nabla_E^{0, 1} = \dab_E $, and let $ \Theta_E = \nabla_E^2 $ be the Chern curvature. Locally around $ x \in X $, there exists an open neighbourhood $ U \ni x $ with local coordinates $ z_1, \dots, z_n $ such that $ x = \br{0, \dots, 0} $ and there exists a holomorphic frame $ s_1, \dots, s_r $ for $ E $ on $ U $ such that if $ H $ is the matrix associated to the metric with respect to $ s_1, \dots, s_r $ then
\begin{enumerate}
\item $ H\br{z} = \id + \OOO\br{\abs{z}^2} $, and
\item $ \Theta_E\br{0} \overset{U}{=} -\da\dab\overline{H}\br{0} $.
\end{enumerate}
\end{lemma}

$ 1 $ means $ h_{ij} = \delta_{ij} + \OOO\br{\abs{z}^2} $, where $ \br{h_{ij} - \delta_{ij}} / \abs{z}^2 < C $ for some $ C $.

\begin{proof}
\hfill
\begin{enumerate}
\item Let $ U \ni x $ be an open set, let $ t_1, \dots, t_r $ be a holomorphic frame for $ E $ on $ U $, and let $ H_1 $ be the matrix associated to $ h $ with respect to $ t_1, \dots, t_r $, so $ H_1\br{0} $ is a Hermitian matrix which gives a metric on $ E\br{x} $. There exists an orthonormal basis of $ E\br{x} $, that is there exists an $ \br{r \times r} $-matrix $ B \in \GL_r \CC $ such that
$$ B^\intercal \cdot H_1\br{0} \cdot \overline{B} = \id. $$
Let $ t_i' = B \cdot t_i $, so $ t_1', \dots, t_r' $ is a holomorphic frame. If $ H_2 $ is the matrix of $ h $ associated to the frame $ t_1', \dots, t_r' $ then
$$ H_2\br{0} = \id, \qquad H_2\br{z} = \id + \OOO\br{\abs{z}}. $$
The goal is to find a new local frame. We want to apply a change of basis given by the matrix $ C\br{z} = \id + C_0\br{z} $ where $ C_0\br{z} $ has coefficients linear in $ z $. Recall that with respect to the new frame $ s_1, \dots, s_r $,
$$ H\br{z} = \br{\id + C_0^\intercal} \cdot H_2\br{z} \cdot \br{\id + \overline{C_0}}. $$
In order to prove $ 1 $, we want $ \D H\br{0} = 0 $. Recall $ H_2\br{0} = \id $. Then
$$ \D H = \D H_2 + \D\br{\id + C_0^\intercal} \cdot H_2 + H_2 \cdot \D\br{\id + \overline{C_0}} + \OOO\br{\abs{z}}, $$
so
$$ \D H\br{0} = \D H_2\br{0} + \D C_0^\intercal\br{0} + \D\overline{C_0}\br{0} = \br{\da H_2\br{0} + \D C_0^\intercal\br{0}} + \br{\dab H_2\br{0} + \D\overline{C_0}\br{0}}. $$
Write
$$ C_0 = \br{c_{ij}}, \qquad c_{ij} = -\sum_{l = 1}^n \dpd{}{z_l}\br{H_2}_{ji}\br{0}z_l. $$
Then
$$ \d c_{ij} = -\sum_{l = 1}^n \sum_{k = 1}^n \dpd{}{z_l}\br{H_2}_{ji}\br{0}\dpd{}{z_k}z_l\d z_k = -\sum_{l = 1}^n \dpd{}{z_l}\br{H_2}_{ji}\br{0}\d z_l, $$
so
$$ \D C_0^\intercal\br{0} = \da C_0^\intercal\br{0} = -\da H_2\br{0}. $$
Similarly
$$ \D\overline{C_0}\br{0} = \dab\overline{C_0}\br{0} = -\dab H_2\br{0}. $$
With this choice, we get $ \D H\br{0} = 0 $, so $ H\br{z} = \id + \OOO\br{\abs{z}^2} $.

\lecture{18}{Tuesday}{18/02/20}

\item When we constructed $ \nabla_E $, we set $ A = \overline{H}^{-1} \cdot \da\overline{H} $ and we proved $ \Theta_E\br{z} = \dab A\br{z} $ in Corollary \ref{cor:5.11}. Since $ H\br{0} = \id $, $ \D H\br{0} = 0 $, so $ \da H\br{0} = 0 $ and $ \dab H\br{0} = 0 $. Then
$$ \Theta_E\br{0} = \dab A\br{0} = \dab\br{\overline{H}^{-1} \cdot \da\overline{H}}\br{0} = \dab\overline{H}^{-1}\br{0} \cdot \da\overline{H}\br{0} + \overline{H}^{-1}\br{0} \cdot \dab\da\overline{H}\br{0} = \dab\da\overline{H}\br{0} = -\da\dab\overline{H}\br{0}, $$
since $ \da\dab + \dab\da = 0 $.
\end{enumerate}
\end{proof}

\pagebreak

\subsection{De Rham cohomology}

Given a complex manifold $ X $, we define
\begin{align*}
\ZZZ^k\br{X}
& = \ker \br{\d : \COC{X}{k} \to \COC{X}{k + 1}}, \qquad k \ge 0 \\
& = \cbr{\omega \in \COC{X}{k} \st \d\omega = 0},
\end{align*}
and we define
\begin{align*}
\BBB^k\br{X}
& = \im \br{\d : \COC{X}{k - 1} \to \COC{X}{k}}, \qquad k \ge 1 \\
& = \cbr{\omega \in \COC{X}{k} \st \exists \eta \in \COC{X}{k - 1}, \ \omega = \d\eta}.
\end{align*}
For convenience, we define $ \BBB^0 = 0 $. Since $ \d \circ \d = 0 $, it follows that $ \BBB^k\br{X} \subset \ZZZ^k\br{X} $ for each $ k \ge 0 $. Thus, we may define
$$ \H^k\br{X, \CC} = \ZZZ^k\br{X} / \BBB^k\br{X}. $$
The group $ \H^k\br{X, \CC} $ is called the \textbf{de Rham cohomology group} of $ X $. If it is finite dimensional, then their dimension
$$ \b_k\br{X} = \dim \H^k\br{X, \CC} $$
is called the \textbf{Betti number} of $ X $.

\begin{remark}
If $ X $ and $ X' $ are diffeomorphic complex manifolds then
$$ \H^k\br{X, \CC} \cong \H^k\br{X', \CC}, \qquad k \ge 0. $$
The same result is not true for the Dolbeault cohomology groups.
\end{remark}

\subsection{Holomorphic line bundles}

Let $ X $ be a complex manifold, let $ L $ be a complex line bundle, and let $ \nabla $ be a connection on $ L $. Then $ \Theta_\nabla $ is a $ \C^\infty $-linear operator. The idea is that $ L^* \otimes L = \Hom\br{L, L} $, so $ \Theta_\nabla \in \COC{X}{2} $.

\begin{proposition}
\hfill
\begin{enumerate}
\item The curvature of $ \nabla $ defines a global $ 2 $-form $ \Theta_\nabla \in \COC{X}{2} $ such that $ \d\Theta_\nabla = 0 $.
\item If $ \nabla' $ is also a connection then there exists a $ 1 $-form $ \eta $ such that $ \Theta_{\nabla'} - \Theta_\nabla = \d\eta $.
\end{enumerate}
\end{proposition}

\begin{proof}
\hfill
\begin{enumerate}
\item Let $ x \in X $, and let $ x \in U $ be open with a non-zero local section $ s \in \Ci{U, L} $. There exists $ A = \br{a} $ with the $ 1 $-form $ a $ on $ U $ representing $ \nabla $. That is, if $ \sigma = fs \in \Ci{U, L} $ where $ f \in \Ci{U} $, then $ \nabla\sigma \overset{U}{=} f \cdot a + \d f $ and $ \nabla^2\sigma = f \cdot \Theta_\nabla $. Recall that $ \Theta_\nabla = a \wedge a + \d a = \d a $ is a $ 2 $-form on $ U $, since $ a $ is a $ 1 $-form. Note that $ \d\Theta_\nabla = \d^2a = 0 $, so $ 1 $ holds. Let $ U' \subset X $ be another open set trivialising $ L $, and let
$$ g : \br{U \cap U'} \times \CC \to \br{U \cap U'} \times \CC $$
be the transition. Recall that if $ \br{a'} = A' $ is the matrix representing $ \nabla $ with respect to the trivialisation on $ U $, then $ a' = g^{-1} \cdot \d g + a $, so
$$ \d a' = \d\br{g^{-1} \cdot \d g} + \d a = g^{-2} \cdot \d g \wedge \d g + g^{-1} \cdot \d^2g + \d a = \d a, $$
since $ \d g^{-1} = g^{-2} \cdot \d g $. \footnote{Exercise} Thus $ \Theta_{\nabla'} = \d a' = \d a $ does not depend on $ U $, so $ \Theta_{\nabla'} $ is a global $ 2 $-form on $ X $.

\lecture{19}{Thursday}{20/02/20}

\item Let $ \nabla' $ be also a connection on $ L $. On $ U $, let $ b $ be the $ 1 $-form representing $ \nabla' $ so that $ \Theta_{\nabla'} \overset{U}{=} \d b $, so $ \Theta_{\nabla'} - \Theta_\nabla \overset{U}{=} \d\br{b - a} $. Let $ U' $, $ g $, and $ a' $ be as above, and let $ b' $ be the $ 1 $-form representing $ \nabla' $ on $ U' $. Then
$$ b' - a' = \br{g^{-1} \cdot \d g + b} - \br{g^{-1} \cdot \d g + a} = b - a. $$
Thus $ \eta = b - a $ is a global $ 1 $-form.
\end{enumerate}
\end{proof}

\pagebreak

\begin{remark}
Thus, if $ L $ is a line bundle on a complex manifold $ X $, there exists a $ 2 $-form $ \Theta_\nabla $ on $ X $ such that $ \sbr{\Theta_\nabla} $ does not depend on $ \nabla $, and depends only on $ L $, as an element in $ \H^2\br{X, \CC} $, the de Rham cohomology over $ \CC $. We can define
$$ \c_1\br{L} = \sbr{\dfrac{i}{2\pi}\Theta_\nabla} \in \H^2\br{X, \CC}, $$
the \textbf{first Chern class} of $ L $. For vector bundles $ E $ of rank $ r $ on $ X $, then we can define
$$ \c_1\br{E} = \c_1\br{\ext^rE}, $$
where $ \ext^rE $ is a line bundle.
\end{remark}

Let $ X $ be a complex manifold, and let $ \br{L, h} $ be a Hermitian holomorphic line bundle. Then there exists a unique Chern connection $ \nabla_L $ compatible with $ h $ and such that $ \nabla_L^{0, 1} = \dab_L $. Fix a non-vanishing section $ s \in \Ci{U, L} $. Then $ h\br{x} = \abr{s, s}_x : U \to \RR $, because $ \abr{v, w} = \overline{\abr{w, v}} $ and $ h $ is positive definite, so
$$ \phi = -\log h\br{x}, $$
the \textbf{weight} of $ \br{L, h} $ on $ U $ with respect to $ s $, is well-defined, and
$$ h = e^{-\phi}. $$
Let $ a $ be the $ 1 $-form defining $ \nabla_L $. Recall that
$$ a = h^{-1} \cdot \da h = e^\phi \cdot \da e^{-\phi} = e^\phi \cdot \br{-e^{\phi}} \cdot \da\phi = -\da\phi, $$
so
$$ \Theta_L = \Theta_{\nabla_L} = \d a = \br{\da + \dab}\br{-\da\phi} = -\dab\da\phi = \da\dab\phi. $$
In particular $ \Theta_L $ is a $ \br{1, 1} $-form on $ X $.

\begin{remark*}
Linear algebra. Let $ V $ be a vector space over $ \CC $ of dimension $ n $. Then $ V_\RR $ is a vector space over $ \RR $ of dimension $ 2n $, so $ V_\CC = V_\RR \otimes_\RR \CC $ is a vector space over $ \CC $ of dimension $ 2n $. There exists a conjugation
$$ \function{V_\CC}{V_\CC}{v}{\overline{v}}. $$
Let $ W_\CC = V_\CC^* = W^{1, 0} \oplus W^{0, 1} $. Then
$$ \ext^kW_\CC = \bigoplus_{p + q = k} W^{p, q}, \qquad W^{p, q} = \ext^pW^{1, 0} \otimes \ext^qW^{0, 1}. $$
There exists a conjugation on $ \ext^kW_\CC $. Then the eigenspace with respect to the eigenvalue one via the conjugation is the real forms on $ V $.
\end{remark*}

\begin{example*}
Let $ V = \CC^n $. Then $ \d z_j + \d\overline{z_j} $ is real and $ i\br{\d z_j - \d\overline{z_j}} $ is real.
\end{example*}

Back to $ L $. Then
$$ \overline{i\Theta_L} = -i\dab\da\phi = i\da\dab\phi = i\Theta_L, $$
so $ \tfrac{i}{2\pi}\Theta_L $ is a real $ \br{1, 1} $-form. Thus the first Chern class of a holomorphic line bundle is defined by a real $ \br{1, 1} $-form.

\begin{remark}
Assume $ \br{L, h} $ is a holomorphic line bundle with $ h = e^{-\phi} $ locally at $ x \in X $. Then if $ \br{L^{-1}, h'} $ is with respect to the induced frame, we can write $ h' = e^\phi $.
\end{remark}

\begin{definition}
Let $ \br{L, h} $ be a Hermitian holomorphic line bundle on $ X $. Then $ L $ is \textbf{positive} if for all $ x \in X $, $ h = e^{-\phi} $ locally at $ x $, such that
$$ \br{\dmd{}{2}{z_i}{}{\overline{z_j}}{}\phi} $$
is positive definite, where $ z_1, \dots, z_n $ are local coordinates around $ x $.
\end{definition}

\begin{example}
Let $ L = X \times \CC $, let $ s $ be the constant section, and let $ \phi = 1 $ on $ X $. Then $ \Theta_L = 0 $, so $ \c_1\br{L} = 0 $.
\end{example}

\pagebreak

\lecture{20}{Thursday}{20/02/20}

\begin{example}
The Fubini-Study metric. Let $ X = \PP_\CC^n $, let
$$ \OOO\br{-1} = \cbr{\br{\sbr{x}, v} \st \sbr{x} \in \PP_\CC^n, \ v = \lambda x, \ \lambda \in \CC}, $$
and let $ U_i = \cbr{\sbr{x} \in \PP_\CC^n \st x_i \ne 0} \subset \PP_\CC^n $ be a trivialising open set. Then $ \pi^{-1}\br{U_i} \cong U_i \times \CC $. Define
$$ \phi_i\br{\sbr{x_0, \dots, x_n}} = -\log \dfrac{\sum_j \abs{x_j}^2}{\abs{x_i}^2} \in \br{0, \infty}. $$
Then $ \phi_i $ is well-defined. Claim that it defines $ h $ on $ \OOO\br{-1} $. Let
$$ g_{ij} : \br{U_j \cap U_i} \times \CC \to \br{U_j \cap U_i} \times \CC $$
be the transition $ g_{ij} = x_i / x_j $ from $ U_j $ to $ U_i $, and let $ h_i = e^{-\phi_i} $ on $ U_i $. Then
$$ h_j = g_{ij} \cdot h_i \cdot \overline{g_{ij}}, \footnote{Exercise} $$
which extends globally to $ X $, is a metric on $ \OOO\br{-1} $. Let $ \OOO\br{1} $ be the dual of $ \OOO\br{-1} $. Define
$$ \psi_i = -\phi_i = \log \dfrac{\sum_j \abs{x_j}^2}{\abs{x_i}^2}. $$
Then $ \psi_i $ defines a metric on $ \OOO\br{1} $. Claim that $ \OOO\br{1} $ is positive, so
$$ \br{\dmd{}{2}{z_i}{}{\overline{z_j}}{}\phi} $$
is positive definite. Let us take $ i = 0 $, so $ z_j = x_j / x_0 $ are coordinates on $ U_0 \cong \CC^n $, and
$$ \psi_0 = \log \br{1 + \sum_{j = 1}^n \abs{z_j}^2}. $$
Then
$$ \dpd{}{z_k}\br{\dpd{}{\overline{z_l}}\psi_0} = \dpd{}{z_k}\br{\dfrac{z_l}{1 + \sum_j \abs{z_j}^2}} = \dfrac{\delta_{kl}\br{1 + \sum_j \abs{z_j}^2} - z_l\overline{z_k}}{\br{1 + \sum_j \abs{z_j}^2}^2}. $$
Fix $ z \in \CC^n $, and let
$$ T = \br{\dmd{}{2}{z_k}{}{\overline{z_l}}{}\psi_0\br{z}}. $$
We want to show that $ T $ is positive definite. If $ n = 1 $, then $ T = \br{1 + \abs{z}^2}^{-2} > 0 $, so ok. If $ n > 1 $, let $ \abr{\cdot, \cdot} $ be the standard inner product on $ \CC^n $ and let $ \norm{\cdot} $ be the norm induced by it. For each $ w \in \CC^n $, we have
$$ \abr{Tw, w} = \dfrac{\br{1 + \norm{z}^2}\norm{w}^2 - \abs{\abr{z, w}}^2}{\br{1 + \norm{z}^2}^2} $$
The Cauchy-Schwarz inequality implies $ \abs{\abr{z, w}}^2 \le \norm{z}^2\norm{w}^2 $. Thus,
$$ \abr{Tw, w} \ge \dfrac{\norm{w}^2}{\br{1 + \norm{z}^2}} \ge 0. $$
and the equality holds if and only if $ w = 0 $. Thus $ T $ is positive definite and $ \OOO\br{1} $ is a positive line bundle.
\end{example}

\pagebreak

\section{K\"ahler manifolds}

The idea is if $ \br{X, \omega} $ is compact K\"ahler and $ k \ge 0 $, then
$$ \H^k\br{X, \CC} \cong \bigoplus_{p + q = k} \H^{p, q}\br{X}. $$

\subsection{K\"ahler manifolds}

\begin{definition}
Let $ V \subset \CC^n $ be open. A \textbf{positive} real $ \br{1, 1} $-form on $ V $ is a real $ \br{1, 1} $-form
\begin{equation}
\label{eq:1}
\omega = \dfrac{i}{2}\sum_{j, k = 1}^n h_{jk}\d z_j \wedge \d\overline{z_k},
\end{equation}
such that $ \br{h_{jk}} $ is positive definite.
\end{definition}

If $ z_1 = x_1 + iy_1 $, then $ \tfrac{i}{2}\d z_1 \wedge \d\overline{z_1} = \d x_1 \wedge \d y_1 $. Then $ \omega $ defines a Hermitian metric on $ \T_{V, \CC} = \abr{\tpd{}{z_i}, \tpd{}{\overline{z_j}}} $. Let $ \omega $ be a positive real $ \br{1, 1} $-form as in $ \br{\ref{eq:1}} $. Then $ \omega^n $ is a real $ \br{n, n} $-form such that if $ z_j = x_j + iy_j $ then
$$ \omega^n = \det h_{ij}\d x_1 \wedge \d y_1 \wedge \dots \wedge \d x_n \wedge \d y_n $$
is a volume form.

\begin{definition}
Globally, let $ X $ be a complex manifold, and let $ \omega \in \CO{X}{1, 1} $ be a real $ \br{1, 1} $-form. Then $ \omega $ is said to be \textbf{positive} if for all $ x \in X $, there exists an open $ U \ni x $ and there exists a biholomorphism $ \phi : U \to V \subset \CC^n $ such that $ \br{\phi^{-1}}^*\omega $ is a positive $ \br{1, 1} $-form on $ V $.
\end{definition}

In particular $ \omega^n $ is a volume form on $ X $, so $ X $ is oriented.

\begin{definition}
A complex manifold $ X $ is called \textbf{K\"ahler} if there exists a positive real $ \br{1, 1} $-form $ \omega $ on $ X $ such that $ \d\omega = 0 $. Such $ \omega $ is called a \textbf{K\"ahler form} on $ X $.
\end{definition}

\begin{notation*}
$ \br{X, \omega} $, where $ X $ is a K\"ahler manifold and $ \omega $ is a K\"ahler form.
\end{notation*}

\begin{example}
\label{eg:6.4}
Let $ X = \CC^n $, and let $ \omega = \tfrac{i}{2}\sum_j \d z_j \wedge \d\overline{z_j} $. Then $ \omega $ is K\"ahler, so $ \CC^n $ is K\"ahler.
\end{example}

\begin{example}
Let $ X = \CC^n / \Lambda $ be the complex torus for a lattice $ \Lambda \subset \CC^n $. Claim that $ X $ is K\"ahler. Let $ \omega $ be as in Example \ref{eg:6.4}. Consider
$$ \function[\psi]{\CC^n}{\CC^n}{z}{z + \lambda}, $$
for some fixed $ \lambda \in \Lambda $. Then $ \psi^*\omega = \omega $, so $ \omega $ descends to a positive closed real $ \br{1, 1} $-form on $ X $, that is there exists $ \omega' $ on $ X $ such that $ q^*\omega' = \omega $ for $ q : \CC^n \to X $. Thus $ X $ is K\"ahler.
\end{example}

\begin{example}
\label{eg:6.6}
Let $ X = \PP_\CC^n $. Recall that if $ h $ is the Fubini-Study metric on $ \OOO\br{1} $, then $ i\Theta_h $ is a real positive $ \br{1, 1} $-form and $ \d\Theta_h = 0 $, so $ X $ is K\"ahler.
\end{example}

\begin{lemma}
\label{lem:6.7}
Let $ X $ be a complex manifold, let $ \omega $ be a K\"ahler form on $ X $, and let $ i : Y \hookrightarrow X $ be an immersion for a complex submanifold $ Y $. Then $ i^*\omega $ is a K\"ahler form on $ Y $. In particular $ Y $ is K\"ahler.
\end{lemma}

\begin{proof}
Exercise. \footnote{Exercise}
\end{proof}

\begin{corollary}
Let $ X $ be a projective manifold. Then $ X $ is K\"ahler.
\end{corollary}

\begin{proof}
$ X $, by definition, is a complex submanifold of $ \PP_\CC^n $. By Example \ref{eg:6.6} $ \PP_\CC^n $ is K\"ahler, so $ X $ is K\"ahler by Lemma \ref{lem:6.7}.
\end{proof}

\begin{fact*}
Every compact complex submanifold of $ \PP_\CC^n $ is a projective manifold.
\end{fact*}

\begin{example*}
Let $ X $ be a complex manifold of dimension one. Then $ X $ is K\"ahler. \footnote{Exercise}
\end{example*}

\lecture{21}{Tuesday}{25/02/20}

Lecture 21 is a problems class.

\pagebreak

\lecture{22}{Thursday}{27/02/20}

Let $ \br{X, \omega} $ be compact K\"ahler. For all $ x \ge 1 $, $ \omega^k = \omega \wedge \dots \wedge \omega $ is closed by the Leibnitz rule. Claim that $ \sbr{\omega^k} \ne 0 $ in $ \H^k\br{X, \CC} $. Assume $ \sbr{\omega^k} = 0 $, so there exists a $ \br{2k - 1} $-form $ \eta $ such that $ \omega^k = \d\eta $. Since $ \omega $ is closed and $ \omega^n $ is a volume form,
$$ 0 < \int_X \omega^n = \int_X \omega^{n - k} \wedge \d\eta = \int_X \d\br{\omega^{n - k} \wedge \eta} = \int_{\da X} \omega^{n - k} \wedge \eta = 0, $$
by the Leibnitz rule. Thus
$$ \H^k\br{X, \CC} \ne 0, \qquad k \in 2\ZZ. $$

\begin{example}
Pick $ \lambda \in \CC $ such that $ 0 < \abs{\lambda} < 1 $. Then $ \ZZ $ acts on $ \CC^n \setminus \cbr{0} $ by
$$ \function{\ZZ \times \br{\CC^n \setminus \cbr{0}}}{\CC^n \setminus \cbr{0}}{\br{n, z}}{\lambda^nz}. $$
Then $ X = \CC^n \setminus \cbr{0} / \ZZ $ is a \textbf{Hopf manifold}. Similarly to the case of complex tori, $ X $ can be shown to have a complex structure. Then
$$ \S^{2n - 1} \subset \RR^{2n} \setminus \cbr{0} = \CC^n \setminus \cbr{0} \cong \S^{2n - 1} \times \RR_{> 0}, $$
so $ X \sim \S^{2n - 1} \times \S^1 $. Thus if $ n \ge 2 $, then $ \H^k\br{X, \CC} = 0 $, so $ X $ is not K\"ahler.
\end{example}

\subsection{Hodge \texorpdfstring{$ \star $}{*} operator}

Let $ V $ be a vector space over $ \RR $ of dimension $ n $ with an inner product $ \abr{\cdot, \cdot} : V \times V \to \RR $. There is a canonical inner product on $ \ext^pV $ for all $ p \ge 1 $,
$$ \abr{v_1 \wedge \dots \wedge v_p, w_1 \wedge \dots \wedge w_p} = \det \br{\abr{v_i, w_j}}. $$
In particular there exists a unique up to orientation $ \omega \in \ext^nV $ such that $ \norm{\omega} = 1 $. The \textbf{Hodge $ \star $ operator} is
$$ \star : \ext^pV \to \ext^{n - p}V, \qquad p \ge 0, $$
such that
$$ \alpha \wedge \star\beta = \abr{\alpha, \beta}\omega \in \ext^nV, \qquad \alpha, \beta \in \ext^pV. $$
Let $ e_1, \dots, e_n $ be an orthonormal basis of $ V $. Then
\begin{itemize}
\item $ \star 1 = \omega $,
\item $ \star\omega = 1 $,
\item $ \star e_1 = e_2 \wedge \dots \wedge e_n $,
\item $ \star e_i = \br{-1}^{i - 1}e_1 \wedge \dots \wedge \widehat{e_i} \wedge \dots \wedge e_n $, and
\item more in general if $ I = \cbr{i_1, \dots, i_k} \subset \cbr{1, \dots, n} $ is ordered such that $ e_I = e_{i_1} \wedge \dots \wedge e_{i_k} $, and $ \CCC I = \cbr{1, \dots, n} \setminus I $ such that $ \sigma : \cbr{1, \dots, n} \to \cbr{I, \CCC I} $ is a permutation, then
$$ \star e_I = \epsilon\br{\sigma}e_{\CCC I}, $$
where $ \epsilon $ is the signature of $ \sigma $, so
$$ \star\star\eta = \br{-1}^{k\br{n - k}}\eta, \qquad \eta \in \ext^kV. $$
\end{itemize}
Assume now that $ V $ is a complex vector space of dimension $ n $ with a Hermitian metric $ \abr{\cdot, \cdot} $. Then, for each $ k \ge 0 $, we can extend the Hodge $ \star $ operator to $ V_\CC $ to a $ \CC $-linear map
$$ \star : \ext^kV_\CC \to \ext^{2n - k}V_\CC, $$
so that
$$ \alpha \wedge \overline{\star\beta} = \abr{\alpha, \beta}\omega, \qquad \alpha, \beta \in \ext^kV_\CC. $$

\begin{note*}
In particular, $ \overline{\star\beta} = \star\overline{\beta} $.
\end{note*}

\pagebreak

Let $ X $ be a complex manifold, and let $ E $ be a Hermitian holomorphic vector bundle. Recall that we defined
$$ \cbr{\cdot, \cdot} : \COCE{X}{p}{E} \times \COCE{X}{q}{E} \to \COC{X}{p + q}. $$
Take $ p = q = 0 $ and $ E = \Omega_{X, \CC}^k $. If $ \omega $ is a positive real $ \br{1, 1} $-form then $ \omega $ induces a Hermitian metric on $ \T_{X, \CC} $. Locally, let $ e_1, \dots, e_n $ be an orthonormal frame of $ \T_{X, \CC} $. Then $ e_1^*, \dots, e_n^* $ define a metric on $ \Omega_{X, \CC}^1 $ locally, where $ e_i^*\br{e_j} = \delta_{ij} $. It is easy to check that such a choice is canonical, so the metric on $ \Omega_{X, \CC}^1 $ extends to $ X $. This induces a metric on $ \Omega_{X, \CC}^k $ for all $ k \ge 0 $, so there exists a cup product
$$ \cbr{\cdot, \cdot} : \COC{X}{k} \times \COC{X}{k} \to \Ci{X}. $$

\begin{lemma}
Let $ \br{X, \omega} $ be K\"ahler of dimension $ n $. Then there exists a $ \CC $-linear
$$ \star : \COC{X}{k} \to \COC{X}{2n - k}, \qquad k \ge 0, $$
such that
$$ \alpha \wedge \overline{\star\beta} = \cbr{\alpha, \beta}\omega^n, \qquad \alpha, \beta \in \COC{X}{k}. $$
\end{lemma}

Check that it is defined globally on $ X $. Let $ E $ be a vector bundle. Then let
$$ \Cic{X, E} = \cbr{s \in \Ci{X, E} \st s \ \text{has compact support}}. $$
Let $ E $ be Hermitian, and let $ \omega $ be a positive real $ \br{1, 1} $-form. Then let
$$ \br{\alpha, \beta}_E = \int_X \cbr{\alpha, \beta}\omega^n, \qquad \alpha, \beta \in \Cic{X, E}. $$
Let $ \br{X, \omega} $ be K\"ahler, let $ E $ and $ F $ be Hermitian vector bundles on $ X $, and let $ P : \Cic{X, E} \to \Cic{X, F} $ be $ \CC $-linear. Then the \textbf{adjoint} of $ P $ is a $ \CC $-linear map $ P^* : \Cic{X, F} \to \Cic{X, E} $ such that
$$ \br{P\alpha, \beta}_F = \br{\alpha, P^*\beta}_E, \qquad \alpha \in \Cic{X, E}, \qquad \beta \in \Cic{X, F}. $$
In particular if $ \d : \COCc{X}{k} \to \COCc{X}{k + 1} $ then
$$ \d^* : \COCc{X}{k + 1} \to \COCc{X}{k}. $$

\begin{lemma}
\label{lem:6.11}
Let $ \br{X, \omega} $ be K\"ahler of dimension $ n $. Then
$$ \d^*\beta = \br{-1}\star\d\star\beta, \qquad \beta \in \COCc{X}{k + 1}. $$
\end{lemma}

\lecture{23}{Thursday}{27/02/20}

\begin{proof}
Let $ \alpha \in \COCc{X}{k} $. Then
\begin{align*}
\br{\d\alpha, \beta}
& = \int_X \d\alpha \wedge \overline{\star\beta} & \cbr{\eta_1, \eta_2}\omega^n = \eta_1 \wedge \overline{\star\eta_2} \\
& = \int_X \d\br{\alpha \wedge \overline{\star\beta}} - \br{-1}^k\int_X \alpha \wedge \d\overline{\star\beta} & \text{the Leibnitz rule} \\
& = \br{-1}^{k + 1}\int_X \alpha \wedge \d\overline{\star\beta} & \text{Stokes} \\
& = \br{-1}^{\br{2n - k}k + k + 1}\int_X \alpha \wedge \star\star\d\overline{\star\beta} & \star\star\eta = \br{-1}^{k\br{2n - k}}\eta \\
& = -\int_X \alpha \wedge \star\star\d\overline{\star\beta} & k^2 - k \ \text{is even} \\
& = -\int_X \cbr{\alpha, \star\d\overline{\star\beta}}\omega^n & \cbr{\eta_1, \eta_2}\omega^n = \eta_1 \wedge \overline{\star\eta_2} \\
& = -\int_X \cbr{\alpha, \star\d\overline{\star\beta}}\omega^n & \overline{\star\eta} = \star\overline{\eta} \\
& = -\br{\alpha, \star\d\star\beta}.
\end{align*}
\end{proof}

\pagebreak

\subsection{Harmonic forms}

\begin{definition}
Let $ \br{X, \omega} $ be a K\"ahler manifold. The \textbf{Hodge-de Rham operator} is
$$ \Delta = \d\d^* + \d^*\d : \COCc{X}{k} \to \COCc{X}{k}. $$
Then $ \alpha \in \COCc{X}{k} $ is said to be \textbf{harmonic} if $ \Delta\alpha = 0 $. Let
$$ \HHH^k\br{X} = \cbr{\text{harmonic $ k $-forms} \ \alpha \ \text{on} \ X} = \ker \Delta. $$
\end{definition}

\begin{lemma}
\label{lem:6.13}
Let $ \br{X, \omega} $ be K\"ahler, and let $ \alpha \in \COCc{X}{k} $. Then $ \alpha $ is harmonic if and only if $ \d\alpha = \d^*\alpha = 0 $.
\end{lemma}

\begin{proof}
If $ \d\alpha = \d^*\alpha = 0 $, then $ \Delta\alpha = 0 $. Assume $ \Delta\alpha = 0 $. Then
$$ 0 = \br{\Delta\alpha, \alpha} = \br{\d\d^*\alpha + \d^*\d\alpha, \alpha} = \br{\d^*\alpha, \d^*\alpha} + \br{\d\alpha, \d\alpha} = \norm{\d^*\alpha}^2 + \norm{\d\alpha}^2, $$
so $ \d\alpha = \d^*\alpha = 0 $.
\end{proof}

\begin{example}
Let $ X = \CC^n $, and let $ \omega = \tfrac{i}{2}\sum_i \d z_i \wedge \d\overline{z_i} $. Then $ \br{X, \omega} $ is K\"ahler. Let $ k = 0 $, let $ z_j = x_j + iy_j $, and let $ f \in \Ci{X} $ such that $ f = f\br{x_1, \dots, x_n, y_1, \dots, y_n} $. Then
$$ \Delta f = \sum_{i = 1}^n \br{\dpd[2]{}{x_i}f + \dpd[2]{}{y_i}f} \in \Ci{X}, \footnote{Exercise} $$
which is the Laplacian.
\end{example}

\begin{lemma}
Let $ \br{X, \omega} $ be K\"ahler. Then $ \Delta $ and $ \star $ commute, that is
$$ \Delta\star\alpha = \star\Delta\alpha, \qquad \alpha \in \COC{X}{k}. $$
\end{lemma}

\begin{proof}
Use $ \d^* = \br{-1}\star\d\star $ and $ \star\star = \br{-1}^{k\br{2n - k}} $.
\end{proof}

\begin{lemma}
Let $ \br{X, \omega} $ be K\"ahler. Then $ \Delta $ is auto-adjoint, that is
$$ \cbr{\Delta\alpha, \beta}\omega^n = \cbr{\alpha, \Delta\beta}\omega^n, \qquad \alpha, \beta \in \COC{X}{k}. $$
\end{lemma}

\begin{proof}
$$ \cbr{\Delta\alpha, \beta}\omega^n = \cbr{\d\d^*\alpha + \d^*\d\alpha, \beta}\omega^n = \cbr{\alpha, \d^*\d\beta + \d\d^*\beta}\omega^n = \cbr{\alpha, \Delta\beta}\omega^n. $$
\end{proof}

\begin{theorem}
\label{thm:6.17}
Let $ \br{X, \omega} $ be a compact K\"ahler manifold. Then, for all $ k \ge 0 $,
\begin{enumerate}
\item $ \HHH^k\br{X} $ is a finite dimensional vector space, and
\item there exist orthogonal decompositions
$$ \COC{X}{k} = \HHH^k\br{X} \oplus \Delta\COC{X}{k} = \HHH^k\br{X} \oplus \d\COC{X}{k - 1} \oplus \d^*\COC{X}{k + 1}, $$
such that
$$ \ker \d = \HHH^k\br{X} \oplus \d\COC{X}{k - 1}, \qquad \ker \d^* = \HHH^k\br{X} \oplus \d^*\COC{X}{k + 1}. $$
\end{enumerate}
\end{theorem}

\begin{proof}
\hfill
\begin{enumerate}
\item Hard, omit the proof, see H\"oring.
\item If $ \alpha \in \HHH^k\br{X} $ and $ \beta \in \COC{X}{k} $, then $ \br{\alpha, \Delta\beta} = \br{\Delta\alpha, \beta} = 0 $. If $ \alpha \in \COC{X}{k - 1} $ and $ \beta \in \COC{X}{k + 1} $, then $ \br{\d\alpha, \d^*\beta} = \br{\d\d\alpha, \beta} = 0 $.
\end{enumerate}
\end{proof}

\pagebreak

\begin{theorem}
\label{thm:6.18}
Let $ \br{X, \omega} $ be compact K\"ahler, and let $ k \ge 0 $. Then
$$ \HHH^k\br{X} = \H^k\br{X, \CC}. $$
\end{theorem}

\begin{proof}
By Theorem \ref{thm:6.17} $ \ker \d = \HHH^k\br{X} \oplus \d\COC{X}{k - 1} $, so
$$ \HHH^k\br{X} = \ker \d / \d\COC{X}{k - 1} = \ZZZ^k\br{X} / \BBB^k\br{X} = \H^k\br{X, \CC}. $$
\end{proof}

\begin{theorem}[Poincar\'e duality]
Let $ \br{X, \omega} $ be a compact K\"ahler manifold. Then there exists an isomorphism
$$ \H^k\br{X, \CC} \to \H^{2n - k}\br{X, \CC}, \qquad k \ge 0. $$
\end{theorem}

\begin{proof}
Want to check
$$ \star : \HHH^k\br{X} \xrightarrow{\sim} \HHH^{n - k}\br{X}. $$
Given a harmonic $ k $-form $ \alpha $ then $ \star\alpha $ is a harmonic $ k $-form, since $ \Delta\star\alpha = \star\Delta\alpha = \star 0 = 0 $, by Theorem \ref{lem:6.13}.
\end{proof}

\subsection{Harmonic \texorpdfstring{$ \br{p, q} $}{(p, q)}-forms}

\lecture{24}{Tuesday}{03/03/20}

The goal is to study a similar decomposition for $ \br{p, q} $-forms. Let $ \omega $ be a positive real $ \br{1, 1} $-form. Locally at $ x \in X $, choose local coordinates $ z_1, \dots, z_n $ around $ x $. Then $ \Omega_{X, \CC}^1 $ is spanned by $ \d z_1, \dots, \d z_n, \d\overline{z_1}, \dots, \d\overline{z_n} $. After a linear change of basis, we may assume that the frame is orthonormal at $ x $, and not locally around $ x $, so
$$ \threebyone{z_1'}{\vdots}{z_n'} = A \cdot \threebyone{z_1}{\vdots}{z_n}, $$
where $ A $ is fixed. Then $ h_{ij} = \id + \OOO\br{\abs{z}} $, so $ \cbr{\d z_i, \d z_j} = \delta_{ij} $. This implies that if $ \alpha $ is a $ \br{p, q} $-form around $ x $, so
$$ \alpha = \sum_{\abs{I} = p, \ \abs{J} = q} \alpha_{I, J}\d z_I \wedge \d\overline{z_J}, $$
then $ \star\alpha $ is an $ \br{n - p, n - q} $-form at the point $ x $. Since $ \star $ does not depend on choice of coordinates, $ \star\alpha $ is a $ \br{n - p, n - q} $-form, so
$$ \star : \CO{X}{p, q} \to \CO{X}{n - p, n - q}. $$
Thus, if $ \alpha $ is a $ \br{p, q} $-form and $ \beta $ is a $ \br{p', q'} $-form such that $ p + q = p' + q' $ then $ \cbr{\alpha, \beta} = 0 $ unless $ p = p' $ and $ q = q' $, so
$$ \COC{X}{k} = \bigoplus_{p + q = k} \CO{X}{p, q} $$
is an orthogonal decomposition. Recall that $ \da : \CO{X}{p, q} \to \CO{X}{p + 1, q} $. Then there exist
$$ \da^* : \CO{X}{p + 1, q} \to \CO{X}{p, q}, \qquad \dab^* : \CO{X}{p, q + 1} \to \CO{X}{p, q}. $$
Like Lemma \ref{lem:6.11}, like in the case of $ \d $,
$$ \da^* = \br{-1}\star\da\star, \qquad \dab^* = \br{-1}\star\dab\star. $$
Moreover we define
$$ \Delta_\da = \da\da^* + \da^*\da, \qquad \Delta_{\dab} = \dab\dab^* + \dab^*\dab. $$
We say that a form $ \alpha $ is \textbf{$ \Delta_\da $-harmonic} if $ \Delta_\da\alpha = 0 $ and \textbf{$ \Delta_{\dab} $-harmonic} if $ \Delta_{\dab}\alpha = 0 $. As for $ \Delta $ we have the following.

\pagebreak

\begin{lemma}
\label{lem:6.20}
\hfill
\begin{itemize}
\item $ \Delta_\da\alpha = 0 $ if and only if $ \da\alpha = \da^*\alpha = 0 $.
\item $ \Delta_{\dab}\alpha = 0 $ if and only if $ \dab\alpha = \dab^*\alpha = 0 $.
\end{itemize}
\end{lemma}

Let
$$ \HHH^{p, q}\br{X} = \cbr{\text{harmonic $ \br{p, q} $-forms} \ \alpha \st \Delta_{\dab}\alpha = 0}. $$

\begin{theorem}
\label{thm:6.21}
Let $ \br{X, \omega} $ be a compact K\"ahler manifold. Then $ \HHH^{p, q}\br{X} $ are finite dimensional vector spaces,
$$ \CO{X}{p, q} = \HHH^{p, q}\br{X} \oplus \Delta\CO{X}{p, q} = \HHH^{p, q}\br{X} \oplus \dab\CO{X}{p, q - 1} \oplus \dab^*\CO{X}{p, q + 1}, $$
and
$$ \ker \dab = \HHH^{p, q}\br{X} \oplus \dab\CO{X}{p, q - 1}, \qquad \ker \dab^* = \HHH^{p, q}\br{X} \oplus \dab^*\CO{X}{p, q + 1}. $$
\end{theorem}

\begin{theorem}
\label{thm:6.22}
The setup is as before. Then
$$ \HHH^{p, q}\br{X} = \H^{p, q}\br{X} = \ker \dab / \im \dab. $$
\end{theorem}

Recall that $ \HHH^k\br{X} = \H^k\br{X, \CC} $. The goal is if $ \br{X, \omega} $ is K\"ahler and compact then
$$ \HHH^k\br{X} = \bigoplus_{p + q = k} \HHH^{p, q}\br{X}. $$

\subsection{Lefschetz operator}

\begin{definition}
Let $ \br{X, \omega} $ be K\"ahler. The \textbf{Lefschetz operator} is
$$ \function[\L]{\COCc{X}{k}}{\COCc{X}{k + 2}}{\alpha}{\alpha \wedge \omega}. $$
There exists an adjoint operator
$$ \Lambda = \L^* : \COCc{X}{k + 2} \to \COCc{X}{k}. $$
\end{definition}

\begin{lemma}
For all $ k $, if $ \beta \in \COCc{X}{k + 2} $ then
$$ \Lambda\beta = \br{-1}^k\star\L\star\beta. $$
\end{lemma}

\begin{proof}
Let $ \alpha \in \COCc{X}{k} $. Since $ \star\star = \br{-1}^k $,
\begin{align*}
\br{\L\alpha, \beta}
& = \int_X \cbr{\L\alpha, \beta}\omega^n
= \int_X \L\alpha \wedge \star\beta
= \int_X \alpha \wedge \omega \wedge \star\beta
= \int_X \omega \wedge \alpha \wedge \star\beta \\
& = \int_X \omega \wedge \alpha \wedge \star\br{-1}^k\star\star\beta
= \int_X \cbr{\alpha, \br{-1}^k\star\L\star\beta}\omega^n
= \br{\alpha, \br{-1}^k\star\L\star\beta},
\end{align*}
and $ \br{-1}^k\star\L\star $ is the adjoint of $ \L $.
\end{proof}

\begin{definition}
Let $ p : E \to X $ and $ q : F \to X $ be holomorphic vector bundles on $ X $. Then a $ \CC $-linear
$$ P : \Ci{X, E} \to \Ci{X, F} $$
is called a \textbf{differential operator of order $ d $} if for all $ x \in X $, there exists an open set $ U \ni x $, local coordinates $ z_1, \dots, z_n $, a frame $ s_1, \dots, s_r $ for $ E $, and a frame $ t_1, \dots, t_l $ for $ F $, such that
$$ P\br{\sum_{i = 1}^r f_is_i} = \sum_{i = 1, \dots, l, \ j = 1, \dots, r, \ I = \br{i_1, \dots, i_n}} P_{I, i, j}\dpd{f_j}{x_I}t_i, \qquad f_j \in \Ci{U}, $$
where $ P_{I, i, j} = 0 $ if $ \abs{I} > d $ and $ P_{I, i, j} \ne 0 $ for some $ \abs{I} = d $.
\end{definition}

\begin{fact*}
The definition and the order of $ \P $ do not depend on the coordinates and on the frames.
\end{fact*}

\pagebreak

\begin{notation}
Let $ A $ be an operator of order $ a $, and let $ B $ be an operator of order $ b $. Then the \textbf{Lie bracket} $ \sbr{A, B} $ is an operator of order $ a + b $ given by
$$ \sbr{A, B} = AB - \br{-1}^{a \cdot b}BA. $$
\end{notation}

\begin{definition}
Let $ X $ be a complex manifold, let $ v $ be a vector field, and let $ \omega $ be a $ k $-form on $ X $. Then $ v \ \lrcorner \ \omega $, the \textbf{contraction} of $ \omega $ with respect to $ v $, is a $ \br{k - 1} $-form defined by
$$ \br{v \ \lrcorner \ \omega}\br{v_1, \dots, v_{k - 1}} = \omega\br{v, v_1, \dots, v_{k - 1}}, $$
on $ \Ci{X} $.
\end{definition}

\lecture{25}{Thursday}{05/03/20}

\begin{example}
Let $ U \subset \CC^n $ be open. Then $ \T_U $ is spanned by the frame $ \tpd{}{z_1}, \dots, \tpd{}{z_n} $. Let $ I = \br{i_1, \dots, i_k} $. It is easy to check
$$ \dpd{}{z_m} \ \lrcorner \ \d z_I =
\begin{cases}
0 & m \notin \cbr{i_1, \dots, i_k} \\
\br{-1}^{l - 1}\d z_{i_1} \wedge \dots \wedge \widehat{\d z_{i_l}} \wedge \dots \wedge \d z_{i_k} & m = i_l
\end{cases}.
$$
\end{example}

\begin{exercise*}
Let $ v \in \Ci{U, \T_U} $, and let $ \alpha \in \COC{U}{p} $ and $ \beta \in \COC{U}{q} $. Then
$$ v \ \lrcorner \ \br{\alpha \wedge \beta} = \br{v \ \lrcorner \ \alpha} \wedge \beta + \br{-1}^p\alpha \wedge \br{v \ \lrcorner \ \beta}. $$
\end{exercise*}

\subsection{K\"ahler identities}

The goal is the Hodge decomposition. Want to show if $ \br{X, \omega} $ is a K\"ahler manifold then
$$ \HHH^k\br{X} = \bigoplus_{p + q = k} \HHH^{p, q}\br{X}, $$
where $ \HHH^k\br{X} $ are $ \Delta $-harmonic forms and $ \HHH^{p, q}\br{X} $ are $ \Delta_{\dab} $-harmonic forms. We want to compose $ \Delta $ with $ \Delta_{\dab} $, by
\begin{itemize}
\item K\"ahler identities on $ \CC^n $ with the standard metric, and
\item showing any K\"ahler manifold is locally like $ \CC^n $ with the standard metric.
\end{itemize}

\begin{example}
Let $ U \subset \CC^n $, and let $ \omega = \tfrac{i}{2}\sum_i \d z_i \wedge \d\overline{z_i} $ be K\"ahler. With respect to such a metric the frame $ \tpd{}{z_1}, \dots, \tpd{}{z_n} $ is orthonormal. Let
$$ \alpha = \sum_{\abs{I} = p, \ \abs{J} = q} \alpha_{I, J}\d z_I \wedge \d\overline{z_J} \in \COc{U}{p, q}, \qquad \alpha_{I, J} \in \Cic{U}. $$
Recall $ \da^* = \br{-1}\star\da\star $. From the definition of $ \star $,
$$ \da^*\alpha = -\sum_{k = 1}^n \sum_{\abs{I} = p, \ \abs{J} = q} \dpd{}{z_k}\alpha_{I, J}\dpd{}{\overline{z_k}} \ \lrcorner \ \br{\d z_I \wedge \d\overline{z_J}}. \footnote{Exercise} $$
The notation is
$$ \dpd{}{z_k}\alpha = \sum_{\abs{I} = p, \ \abs{J} = q} \dpd{}{z_k}\alpha_{I, J}\d z_I \wedge \d\overline{z_J}. $$
Thus
$$ \da^*\alpha = -\sum_{k = 1}^n \dpd{}{\overline{z_k}} \ \lrcorner \ \dpd{}{z_k}\alpha. $$
\end{example}

\pagebreak

\begin{lemma}[K\"ahler identity on $ \CC^n $]
Let $ U \subset \CC^n $ be open, and let $ \omega = \tfrac{i}{2}\sum_i \d z_i \wedge \d\overline{z_i} $. Then
$$ \sbr{\dab^*, \L} = i\da. $$
\end{lemma}

\begin{proof}
By $ \CC $-linearity, we may assume $ \alpha = \alpha_{I, J}\d z_I \wedge \d\overline{z_J} $ for some $ \abs{I} = p $ and $ \abs{J} = q $. Then $ \da^*\alpha = -\sum_{k = 1}^n \tpd{}{\overline{z_k}} \ \lrcorner \ \tpd{}{z_k}\alpha $, so
$$ \sbr{\dab^*, \L}\alpha = \dab^*\L\alpha - \L\dab^*\alpha = -\sum_{k = 1}^n \dpd{}{z_k} \ \lrcorner \ \dpd{}{\overline{z_k}}\br{\omega \wedge \alpha} + \omega \wedge \br{\sum_{k = 1}^n \dpd{}{z_k} \ \lrcorner \ \dpd{}{\overline{z_k}}\alpha}. $$
Since $ \tpd{}{\overline{z_k}}\br{\omega \wedge \alpha} = \omega \wedge \tpd{}{\overline{z_k}}\alpha $,
$$ \sbr{\dab^*, \L}\alpha = -\sum_{k = 1}^n \dpd{}{z_k} \ \lrcorner \ \br{\omega \wedge \dpd{}{\overline{z_k}}\alpha} + \omega \wedge \br{\sum_{k = 1}^n \dpd{}{z_k} \ \lrcorner \ \dpd{}{\overline{z_k}}\alpha}. $$
Recall that $ v \ \lrcorner \ \br{\alpha \wedge \beta} = \br{v \ \lrcorner \ \alpha} \wedge \beta + \br{-1}^p\alpha \wedge \br{v \ \lrcorner \ \beta} $, so
$$ \dpd{}{z_k} \ \lrcorner \ \br{\omega \wedge \dpd{}{\overline{z_k}}\alpha} = \br{\dpd{}{z_k} \ \lrcorner \ \omega} \wedge \dpd{}{\overline{z_k}}\alpha + \omega \wedge \br{\dpd{}{z_k} \ \lrcorner \ \dpd{}{\overline{z_k}}\alpha}, $$
since $ p = 2 $. Since $ \tpd{}{z_k} \ \lrcorner \ \omega = i\d\overline{z_k} $,
$$ \dpd{}{z_k} \ \lrcorner \ \br{\omega \wedge \dpd{}{\overline{z_k}}\alpha} = i\d\overline{z_k} \wedge \dpd{}{\overline{z_k}}\alpha + \omega \wedge \br{\dpd{}{z_k} \ \lrcorner \ \dpd{}{\overline{z_k}}\alpha}. $$
Thus
$$ \sbr{\dab^*, \L}\alpha = \sum_{k = 1}^n i\d\overline{z_k} \wedge \dpd{}{\overline{z_k}}\alpha = i\da\alpha. $$
\end{proof}

\begin{theorem}
Let $ \br{X, \omega} $ be a K\"ahler manifold, and let $ x \in X $. There exist local holomorphic coordinates $ z_1, \dots, z_n $ around $ x $ such that if
$$ \omega = \dfrac{i}{2}\sum_{j, k = 1}^n h_{jk}\d z_k \wedge \d\overline{z_k}, $$
then
$$ h_{jk} = \delta_{jk} + \OOO\br{\abs{z}^2}. $$
\end{theorem}

\begin{remark*}
Assume that $ X $ is a complex manifold and $ \omega $ is a positive real $ \br{1, 1} $-form which satisfies $ \br{\ref{eq:1}} $. Then $ \omega $ is K\"ahler, so $ \d\omega = 0 $. \footnote{Exercise}
\end{remark*}

\begin{proof}
Recall there exists a linear change of coordinates such that at $ x $, $ h_{jk}\br{x} = \delta_{jk} $, that is $ h_{jk}\br{z} = \delta_{jk} + \OOO\br{\abs{z}} $. Let
$$ \omega = \dfrac{i}{2}\sum_{j, k = 1}^n h_{jk}\d z_k \wedge \d\overline{z_k}. $$
Then
$$ h_{jk} = \delta_{jk} + \sum_{l = 1}^n \br{a_{jkl}z_l + a_{jkl}'\overline{z_l}} + \OOO\br{\abs{z}^2}, \qquad a_{jkl}, a_{jkl}' \in \CC. $$
Since $ h_{jk} $ is Hermitian $ \overline{a_{kjl}} = a_{jkl}' $. Since $ \omega $ is K\"ahler, $ \d\omega = 0 $, so $ a_{jkl} = a_{lkj} $. Write
$$ \xi_k = z_k + \dfrac{1}{2}\sum_{j, l = 1}^n a_{jkl}z_jz_l, \qquad k = 1, \dots, n. $$

\pagebreak

Then $ \xi_k $ is holomorphic. Let $ \phi\br{z_1, \dots, z_n} = \br{\xi_1, \dots, \xi_n} $. Then $ \d\phi_x = \id $, so around $ x $, $ \det \d\phi \ne 0 $. By the implicit function theorem, $ \phi $ is locally an isomorphism, so $ \xi_1, \dots, \xi_n $ are homogeneous holomorphic coordinates at $ x $. Then
$$ \d\xi_k = \d z_k + \dfrac{1}{2}\sum_{j, l = 1}^n a_{jkl}\br{z_j\d z_j + z_l\d z_l} = \d z_k + \dfrac{1}{2}\sum_{j, l = 1}^n \br{a_{jkl} + a_{lkj}}z_l\d z_j = \d z_k + \sum_{j, l = 1}^n a_{jkl}z_l\d z_j, $$
so
\begin{align*}
i\sum_{k = 1}^n \br{\d\xi_k \wedge \d\overline{\xi_k}}
& = i\sum_{k = 1}^n \d z_k \wedge \d\overline{z_k} + i\sum_{j, k, l = 1}^n \br{\overline{a_{jkl}z_l}\d z_k \wedge \d\overline{z_j} + a_{jkl}z_l\d z_j \wedge \d\overline{z_k}} + \OOO\br{\abs{z}^2} \\
& = i\sum_{j, k = 1}^n \delta_{jk}\d z_j \wedge \d\overline{z_k} + i\sum_{j, k, l = 1}^n \br{a_{jkl}'\overline{z_l} + a_{jkl}z_l}\d z_j \wedge \d\overline{z_k} + \OOO\br{\abs{z}^2} \\
& = i\sum_{j, k = 1}^n \br{\delta_{jk} + \sum_{l = 1}^n \br{a_{jkl}'\overline{z_l} + a_{jkl}z_l}}\d z_j \wedge \d\overline{z_k} + \OOO\br{\abs{z}^2} \\
& = i\sum_{j, k = 1}^n h_{jk}\d z_j \wedge \d\overline{z_k}
= 2\omega.
\end{align*}
\end{proof}

\lecture{26}{Thursday}{05/03/20}

Then $ \xi_1, \dots, \xi_n $ are \textbf{normal coordinates} for the K\"ahler metric.

\begin{theorem}[K\"ahler identities]
Let $ \br{X, \omega} $ be K\"ahler. Then
\begin{enumerate}
\item $ \sbr{\dab^*, \L} = i\da $,
\item $ \sbr{\da^*, \L} = -i\dab $,
\item $ \sbr{\Lambda, \dab} = -i\da^* $, and
\item $ \sbr{\Lambda, \da} = i\dab^* $.
\end{enumerate}
\end{theorem}

\begin{proof}
\hfill
\begin{enumerate}
\item $ \sbr{\dab^*, \L} = \dab^*\L - \L\dab^* = \br{-1}\star\dab\star\L + \L\star\dab\star $. We want to check that $ 1 $ holds at $ x \in X $. But around $ x $, we may assume that
$$ \omega = \dfrac{i}{2}\sum_{j, k = 1}^n h_{jk}\d z_k \wedge \d\overline{z_k}, \qquad h_{jk} = \delta_{jk} + \OOO\br{\abs{z}^2}. $$
In the calculation only the first order of $ \omega $ appears at $ x $, so we can pretend that $ X = \CC^n $ and $ \omega $ is the standard metric. We checked that $ 1 $ holds on this setting, so $ 1 $ holds.
\item Since $ \omega $ is real $ \L = \overline{\L} $, so $ 2 $ holds.
\item Recall $ \Lambda $ is the adjoint of $ \L $. Let $ \alpha $ and $ \beta $ be $ \br{p, q} $-forms. Then
\begin{align*}
\br{\sbr{\Lambda, \dab}\alpha, \beta}
& = \br{\Lambda\dab\alpha - \dab\Lambda\alpha, \beta}
= \br{\dab\alpha, \L\beta} - \br{\Lambda\alpha, \dab^*\beta}
= \br{\alpha, \dab^*\L\beta - \L\dab^*\beta} \\
& = \br{\alpha, \sbr{\dab^*, \L}\beta}
= \br{\alpha, i\da\beta}
= \br{-i\da^*\alpha, \beta}.
\end{align*}
\item $ 3 $ implies $ 4 $, because $ \overline{\Lambda} = \Lambda $.
\end{enumerate}
\end{proof}

\pagebreak

\begin{theorem}
\label{thm:6.33}
Let $ \br{X, \omega} $ be a K\"ahler manifold. Then
$$ \Delta = 2\Delta_\da = 2\Delta_{\dab}. $$
\end{theorem}

\begin{proof}
Since $ \da^2 = 0 $,
\begin{align*}
\Delta
& = \d\d^* + \d^*\d \\
& = \br{\da + \dab}\br{\da^* + \dab^*} + \br{\da^* + \dab^*}\br{\da + \dab} & \d = \da + \dab \\
& = \Delta_\da + \da\dab^* + \dab\da^* + \dab\dab^* + \da^*\dab + \dab^*\da + \dab^*\dab & \Delta_\da = \da\da^* + \da^*\da \\
& = \Delta_\da + \da\dab^* + \dab\dab^* + \dab^*\da + \dab^*\dab & \dab\da^* = i\dab\sbr{\Lambda, \dab} = i\dab\Lambda\dab = -i\sbr{\Lambda, \dab}\dab = \da^*\dab \ \text{by} \ 3 \\
& = \Delta_\da + \dab\dab^* + \dab^*\dab & \da\dab^* = -i\da\sbr{\Lambda, \da} = -i\da\Lambda\da = i\sbr{\Lambda, \da}\da = \dab^*\da \ \text{by} \ 4 \\
& = \Delta_\da - i\dab\Lambda\da + i\dab\da\Lambda - i\Lambda\da\dab + i\da\Lambda\dab & \dab^* = -i\sbr{\Lambda, \da} = -i\Lambda\da + i\da\Lambda \ \text{by} \ 4 \\
& = \Delta_\da + i\sbr{\Lambda, \dab}\da + i\da\sbr{\Lambda, \dab} & \dab\da + \da\dab = 0 \\
& = \Delta_\da + \da^*\da + \da\da^* & \da^* = i\sbr{\Lambda, \dab} \ \text{by} \ 3 \\
& = 2\Delta_\da & \Delta_\da = \da\da^* + \da^*\da.
\end{align*}
Similarly $ \Delta = 2\Delta_{\dab} $.
\end{proof}

\subsection{Hodge decomposition}

\begin{lemma}
If $ \alpha $ is a $ \br{p, q} $-form then $ \Delta\alpha $ is a $ \br{p, q} $-form.
\end{lemma}

\begin{proof}
Easy.
\end{proof}

\begin{theorem}
\label{thm:6.35}
Let $ \br{X, \omega} $ be K\"ahler. Then
\begin{enumerate}
\item for all $ k \ge 0 $,
$$ \HHH^k\br{X} = \bigoplus_{p + q = k} \HHH^{p, q}\br{X}, $$
\item for all $ p $ and $ q $,
$$ \HHH^{p, q}\br{X} = \overline{\HHH^{q, p}\br{X}}. $$
\end{enumerate}
\end{theorem}

\begin{proof}
\hfill
\begin{enumerate}
\item Let $ \alpha \in \COCc{X}{k} $. Then there is a unique decomposition
$$ \alpha = \sum_{p + q = k} \alpha_{p, q}, \qquad \alpha_{p, q} \in \COc{X}{p, q}, $$
so $ \Delta\alpha = \sum_{p + q = k} \Delta\alpha_{p, q} $. Thus
\begin{align*}
\alpha \in \HHH^k\br{X}
& \iff \Delta\alpha = 0 \\
& \iff \forall p, q, \ p + q = k, \ \Delta\alpha_{p, q} = 0 \\
& \iff \forall p, q, \ p + q = k, \ \Delta_{\dab}\alpha_{p, q} = 0 & \text{Theorem \ref{thm:6.33}} \\
& \iff \forall p, q, \ p + q = k, \ \alpha_{p, q} \in \HHH^{p, q}\br{X}.
\end{align*}
\item Let $ \alpha \in \COc{X}{p, q} $. Then $ \overline{\alpha} \in \COc{X}{q, p} $. Thus by Theorem \ref{thm:6.33},
$$ \Delta\alpha = 0 \qquad \iff \qquad \Delta_\da\alpha = 0 \qquad \iff \qquad \Delta_{\dab}\overline{\alpha} = 0 \qquad \iff \qquad \Delta\overline{\alpha} = 0. $$
\end{enumerate}
\end{proof}

\pagebreak

\begin{theorem}[Hodge decomposition]
\label{thm:6.36}
Let $ \br{X, \omega} $ be a compact K\"ahler manifold. Then
\begin{enumerate}
\item for all $ k \ge 0 $,
$$ \H^k\br{X, \CC} = \bigoplus_{p + q = k} \H^{p, q}\br{X}, $$
\item for all $ p $ and $ q $,
$$ \H^{p, q}\br{X} = \overline{\H^{q, p}\br{X}}. $$
\end{enumerate}
\end{theorem}

\begin{exercise*}
If $ X = \CC $ Theorem \ref{thm:6.36} is false.
\end{exercise*}

\begin{proof}
$ \H^k\br{X, \CC} = \HHH^k\br{X} $ by Theorem \ref{thm:6.18} and $ \H^{p, q}\br{X} = \HHH^{p, q}\br{X} $ by Theorem \ref{thm:6.22}.
\end{proof}

\begin{example}
If $ X = \PP_\CC^n $, by Mayer-Vietoris,
$$ \H^k\br{X, \CC} =
\begin{cases}
\CC & k = 0, \dots, 2n \\
0 & \text{otherwise}
\end{cases}.
$$
Since $ \H^{k, k}\br{X} \ne 0 $ for all $ k = 0, \dots, n $,
$$ \H^{p, q}\br{X} =
\begin{cases}
\CC & p = q \\
0 & \text{otherwise}
\end{cases}.
$$
\end{example}

\subsection{Bott-Chern cohomology}

\lecture{27}{Tuesday}{10/03/20}

Let $ \br{X, \omega} $ be K\"ahler. Then for any $ k \ge 0 $,
$$ \H^k\br{X, \CC} = \bigoplus_{p + q = k} \H^{p, q}\br{X}, $$
where the de Rham cohomology $ \H^k\br{X, \CC} $ depends on the topology, and the Dolbeault cohomology $ \H^{p, q}\br{X} $ depends on the complex structure. The proof depended on
$$ \HHH^k\br{X} = \bigoplus_{p + q = k} \HHH^{p, q}\br{X}, $$
where $ \HHH^k\br{X} $ are $ \Delta $-harmonic forms and $ \HHH^{p, q}\br{X} $ are $ \Delta_{\dab} $-harmonic forms. Recall that $ \Delta = \d\d^* + \d\d^* $ depends on $ \d^* = \br{-1}^k\star\d\star $, which depends on $ \star $, and $ \star $ depends on the metric on $ \omega $.

\begin{example*}
If $ X $ is projective there exist many ways for $ X \subset \PP^N $, so there exist many metrics on $ X $.
\end{example*}

The goal is to show that the Hodge decomposition does not depend on $ \omega $. Let $ \alpha \in \CO{X}{p - 1, q - 1} $. Then $ \da\dab\alpha \in \CO{X}{p, q} $ is closed, since $ \d\da\dab\alpha = \da\da\dab\alpha + \dab\br{-\dab\da\alpha} = 0 $. Thus we can define the \textbf{Bott-Chern cohomology group}
$$ \H_{\BC}^{p, q}\br{X} = \cbr{\alpha \in \CO{X}{p, q} \st \d\alpha = 0} / \da\dab\CO{X}{p - 1, q - 1}, $$
which is independent of $ \omega $. There exists a $ \CC $-linear map $ \Phi : \cbr{\alpha \in \CO{X}{p, q} \st \d\alpha = 0} \to \H^{p + q}\br{X, \CC} $. Let $ \da\dab\alpha \in \da\dab\CO{X}{p - 1, q - 1} $. Then $ \da\dab\alpha = \d\dab\alpha = 0 $ in $ \H^{p + q}\br{X, \CC} $, so
$$ \Phi : \H_{\BC}^{p, q}\br{X} \to \H^{p + q}\br{X, \CC}. $$
We want to show that there exists a $ \CC $-linear map
$$ \Phi : \H_{\BC}^{p, q}\br{X} \to \H^{p, q}\br{X}. $$
Indeed if $ \alpha \in \CO{X}{p, q} $ such that $ \d\alpha = 0 $, then $ \da\alpha = \dab\alpha = 0 $, so there exists a map $ \Phi : \cbr{\d\alpha = 0} \to \H^{p, q}\br{X} $. Let $ \da\dab\alpha \in \da\dab\CO{X}{p - 1, q - 1} $. Then $ \da\dab\alpha = \dab\br{-\da\alpha} = 0 $ in $ \H^{p, q}\br{X} $.

\pagebreak

\begin{lemma}[$ \da\dab $-lemma]
Let $ \br{X, \omega} $ be a compact K\"ahler manifold, and let $ \alpha \in \CO{X}{p, q} $ such that
\begin{enumerate}
\item $ \d\alpha = 0 $, and
\item $ \alpha $ is $ \da $-exact, or $ \dab $-exact.
\end{enumerate}
Then there exists $ \beta \in \CO{X}{p - 1, q - 1} $ such that $ \alpha = \da\dab\beta $.
\end{lemma}

\begin{proof}
By $ 1 $, $ \da\alpha = \dab\alpha = 0 $. Assume that $ \alpha $ is $ \da $-exact. Then there exists $ \beta \in \CO{X}{p - 1, q} $ such that $ \alpha = \da\beta $. Recall, by Theorem \ref{thm:6.21}, that
$$ \CO{X}{p - 1, q} = \HHH^{p - 1, q}\br{X} \oplus \dab\CO{X}{p - 1, q - 1} \oplus \dab^*\CO{X}{p - 1, q + 1}. $$
Then
$$ \beta = \beta_1 + \dab\beta_2 + \dab^*\beta_3, \qquad \beta_1 \in \HHH^{p - 1, q}\br{X}, \qquad \beta_2 \in \CO{X}{p - 1, q - 1}, \qquad \beta_3 \in \CO{X}{p - 1, q + 1}, $$
so $ \alpha = \da\beta = \da\beta_1 + \da\dab\beta_2 + \da\dab^*\beta_3 $. The goal is $ \da\beta_1 = \da\dab^*\beta_3 = 0 $. Since $ \beta_1 \in \HHH^{p - 1, q}\br{X} $, $ \Delta_{\dab}\beta_1 = 0 $. Since $ X $ is K\"ahler, $ \Delta_\da = \Delta_{\dab} $, so $ \Delta_\da\beta_1 = 0 $, if and only if $ \da\beta_1 = \dab^*\beta_1 = 0 $ by Lemma \ref{lem:6.20}. Since $ X $ is K\"ahler, by the K\"ahler identity $ \sbr{\Lambda, \da} = i\dab^* $,
$$ \dab^*\da = -i\sbr{\Lambda, \da}\da = -i\br{\Lambda\da - \da\Lambda}\da = i\da\Lambda\da = i\da\sbr{\Lambda, \da} = i\da i\dab^* = -\da\dab^*. $$
Then $ 0 = \dab\alpha = \dab\br{\da\dab\beta_2 + \da\dab^*\beta_3} = \dab\da\dab^*\beta_3 = -\dab\dab^*\da\beta_3 $, so $ \dab\dab^*\da\beta_3 = 0 $. Recall, by Theorem \ref{thm:6.21}, that
$$ \ker \dab^* = \HHH^{p, q}\br{X} \oplus \dab^*\CO{X}{p, q + 1}. $$
Then $ \da\dab^*\beta_3 = -\dab^*\da\beta_3 \in \dab^*\CO{X}{p, q + 1} $, but $ \dab^*\da\dab^*\beta_3 = 0 $ and $ \dab\da\dab^*\beta_3 = 0 $, so $ \da\dab^*\beta_3 \in \HHH^{p, q}\br{X} $ is $ \Delta_{\dab} $-harmonic. Thus $ \da\dab^*\beta_3 = 0 $.
\end{proof}

\begin{theorem}
\label{thm:6.39}
Let $ X $ be a compact K\"ahler manifold. There exist isomorphisms
$$ \Phi : \H_{\BC}^{p, q}\br{X} \to \H^{p, q}\br{X}, \qquad \Psi : \bigoplus_{p + q = k} \H_{\BC}^{p, q}\br{X} \to \H^k\br{X, \CC}. $$
\end{theorem}

\begin{proof}
$ \Phi $ implies $ \Psi $, so the goal is to prove that $ \Phi $ is an isomorphism. Let $ \sbr{\alpha'} \in \H^{p, q}\br{X} \cong \HHH^{p, q}\br{X} $. There exists $ \alpha \in \HHH^{p, q}\br{X} $ such that $ \sbr{\alpha} = \sbr{\alpha'} $ and $ \alpha $ is $ \Delta_{\dab} $-harmonic. Since $ X $ is K\"ahler and $ \Delta = 2\Delta_{\dab} $, $ \alpha $ is $ \Delta $-harmonic, so $ \d\alpha = 0 $. Since $ \H_{\BC}^{p, q}\br{X} = \cbr{\d\alpha = 0} / \da\dab\CO{X}{p - 1, q - 1} $, $ \sbr{\alpha} \in \H_{\BC}^{p, q}\br{X} $, so $ \Phi $ is surjective. Assume $ \Phi\br{\sbr{\beta}} = 0 $, so $ \d\beta = 0 $, since $ \beta \in \H_{\BC}^{p, q}\br{X} $. Then $ \sbr{\beta} = 0 $ inside $ \H^{p, q}\br{X} = \ker \dab / \im \dab $, so $ \beta = \dab\gamma $, that is $ \beta $ is $ \dab $-exact. By the $ \da\dab $-lemma, $ \beta = \da\dab\eta $, so $ \sbr{\beta} = 0 $ in $ \H_{\BC}^{p, q}\br{X} $. Thus $ \Phi $ is injective.
\end{proof}

\subsection{Lefschetz decomposition}

\lecture{28}{Thursday}{12/03/20}

Let $ X $ be a compact K\"ahler manifold of dimension $ n $. The Hodge numbers are $ \h^{p, q}\br{X} = \dim \H^{p, q}\br{X} $, and the \textbf{Hodge diamond} is
$$
\begin{tikzcd}[column sep=0, row sep=0]
& & & \h^{0, 0}\br{X} & & & \\
& & \h^{1, 0}\br{X} & & \h^{0, 1}\br{X} & & \\
& {\displaystyle\cdot}^{{\displaystyle\cdot}^{\displaystyle\cdot}} & & \vdots & & \ddots & \\
\h^{n, 0}\br{X} & & \dots & & \dots & & \h^{0, n}\br{X} \\
& \ddots & & \vdots & & {\displaystyle\cdot}^{{\displaystyle\cdot}^{\displaystyle\cdot}} & \\
& & \h^{n, n - 1}\br{X} & & \h^{n - 1, n}\br{X} & & \\
& & & \h^{n, n}\br{X} & & &
\end{tikzcd}.
$$

\pagebreak

At each line, the sum corresponds to the Betti numbers $ \b_0\br{X}, \dots, \b_{2n}\br{X} $, because it is a real manifold of dimension $ 2n $. What else do we know about these numbers?
\begin{itemize}
\item Since we are working with a compact K\"ahler manifold, $ \h^{p, p}\br{X} \ne 0 $ for all $ p = 0, \dots, n $.
\item Since $ X $ is connected, $ \h^{n, n}\br{X} = 1 $ and $ \h^{0, 0}\br{X} = 1 $.
\item Since $ \H^{p, q}\br{X} = \overline{\H^{q, p}\br{X}} $, and the dimension is the same if we take conjugation, $ \h^{p, q}\br{X} = \h^{q, p}\br{X} $.
\item Since $ \H^{p, q}\br{X} = \H^{n - p, n - q}\br{X} $, and the Hodge $ \star $ operator preserves $ \dab $-closed, $ \h^{p, q}\br{X} = \h^{n - p, n - q}\br{X} $.
\end{itemize}
The goal is $ \h^{p, q}\br{X} $ are increasing. More specifically, we have $ \h^{p, q}\br{X} \le \h^{p + 1, q + 1}\br{X} $ if $ p + 1 \le n $ and $ q + 1 \le n $ where $ n = \dim X $. If $ \alpha $ is a $ \br{p, q} $-form, then $ \alpha \wedge \omega $ is a $ \br{p + 1, q + 1} $-form, and this process is injective. Today we study
$$ \function[\L]{\CO{X}{p, q}}{\CO{X}{p + 1, q + 1}}{\alpha}{\alpha \wedge \omega}. $$
Since $ \omega $ is real,
$$ \L : \H^k\br{X, \RR} \to \H^{k + 2}\br{X, \RR}. $$
We need to work with
$$ \HHH^k\br{X, \RR} = \cbr{\alpha \in \Ci{X, \Omega_{X, \RR}^k} \st \Delta\alpha = 0}. $$

\begin{corollary}
In the same context as before, we have an isomorphism
$$ \function{\HHH^k\br{X, \RR}}{\H^k\br{X, \RR}}{\alpha}{\sbr{\alpha}}, \qquad k \ge 0. $$
\end{corollary}

\begin{lemma}
\label{lem:6.41}
Let $ X $ be a compact K\"ahler manifold of dimension $ n $.
\begin{enumerate}
\item $ \sbr{\Delta, \L} = 0 $.
\item $ \sbr{\L, \Lambda}\alpha = \br{k - n}\alpha $ for any $ \alpha \in \COC{X}{k} $.
\end{enumerate}
\end{lemma}

\begin{proof}
\hfill
\begin{enumerate}
\item Recall that $ \Delta = 2\Delta_\da $. Only have to show that $ \L $ commutes with $ \Delta_\da $. Let $ \alpha $ be a $ k $-form on $ X $. Since $ \omega $ is a closed $ \br{1, 1} $-form on $ X $, $ \da\omega = 0 $. Then $ \da\L\alpha = \da\br{\omega \wedge \alpha} = \omega \wedge \da\alpha = \L\da\alpha $, so $ \da\L = \L\da $. By the K\"ahler identity, $ \da^*\L - \L\da^* = \sbr{\da^*, \L} = -i\dab $. Then
$$ \sbr{\Delta_\da, \L} = \br{\da\da^* - \da^*\da}\L - \L\br{\da\da^* - \da^*\da} = \da\da^*\L - \da^*\L\da - \da\L\da^* + \L\da^*\da = -i\da\dab - \br{-i\dab\da} = 0. $$
\item $ \sbr{\L, \Lambda} $ can be computed pointwise. That is, if we fix $ x \in X $ we may assume that there exist coordinates $ z_1, \dots, z_n $ such that $ \d z_i $ is an orthonormal basis with respect to $ \omega $, so we may consider a Hermitian vector space $ V $ and a real $ \br{1, 1} $-form $ \omega $. Proceed by induction on $ \dim V = n $. We are working on $ V_\CC = V_\RR \otimes_\RR \CC $.
\begin{itemize}[leftmargin=0.5in]
\item[$ n = 1 $.] $ \sbr{\L, \Lambda}\br{1} = -1 $ for a $ 0 $-form $ 1 $, $ \sbr{\L, \Lambda}\eta = 0 $ for a $ 1 $-form $ \eta $, and $ \sbr{\L, \Lambda}\omega = \omega $ for a $ 2 $-form $ \omega $. \footnote{Exercise}
\item[$ n > 1 $.] Let $ V = W_1 \oplus W_2 $ be an orthogonal decomposition such that $ \dim W_1 = n - 1 $ and $ \dim W_2 = 1 $. Then $ \omega = \omega_1 + \omega_2 $ where $ \omega_i $ is a metric on $ W_i $ and $ \Lambda = \Lambda_1 + \Lambda_2 $ where $ \Lambda_i $ is the adjoint of $ \L_i = \omega_i \wedge \cdot $. By linear algebra,
$$ \ext^kV^* = \br{\ext^kW_1^* \otimes \ext^0W_2^*} \oplus \br{\ext^{k - 1}W_1^* \otimes \ext^1W_2^*}. $$
Let $ \eta = \eta_1 + \eta_2 $ for $ \eta_1 \in \ext^kW_1^* \otimes \ext^0W_2^* $ and $ \eta_2 \in \ext^{k - 1}W_1^* \otimes \ext^1W_2^* $. We need to check that $ 2 $ holds for $ \eta_i $ for $ i = 1, 2 $. Let $ \eta_1 = \eta_1 \cdot 1 $ for $ \eta_1 \in \ext^kW_1^* $ and $ 1 \in \ext^0W_2^* $. By induction,
$$ \sbr{\L, \Lambda}\eta_1 = \sbr{\L_1, \Lambda_1}\eta_1 + \eta_1\sbr{\L_2, \Lambda_2}\br{1} = \br{k - \br{n - 1}}\eta_1 + \eta_1\br{-1} = \br{k - n}\eta_1. $$
Similarly, $ \sbr{\L, \Lambda}\eta_2 = \br{k - n}\eta_2 $. \footnote{Exercise}
\end{itemize}
\end{enumerate}
\end{proof}

\pagebreak

\begin{proposition}
\label{prop:6.42}
For all $ k $,
$$ \L^{n - k} : \Omega_{X, \RR}^k \xrightarrow{\sim} \Omega_{X, \RR}^{2n - k} $$
is an isomorphism, where a $ k $-form is mapped to a $ \br{2n - k} $-form.
\end{proposition}

\lecture{29}{Thursday}{12/03/20}

\begin{proof}
It is enough to show that it is injective, since
$$ \rk \Omega_{X, \RR}^k = \binom{2n}{k} = \binom{2n}{2n - k} = \rk \Omega_{X, \RR}^{2n - k}. $$
Recall that if $ \alpha $ is a $ k $-form $ \sbr{\L, \Lambda}\alpha = \br{k - n}\alpha $. Then
$$ \sbr{\L^r, \Lambda}\alpha = \L\br{\L^{r - 1}\Lambda - \Lambda\L^{r - 1}}\alpha + \br{\L\Lambda - \Lambda\L}\L^{r - 1}\alpha = \L\sbr{\L^{r - 1}, \Lambda}\alpha + \br{2\br{r - 1} + k - n}\L^{r - 1}\alpha, \qquad r > 0. $$
By induction,
\begin{equation}
\label{eq:2}
\sbr{\L^r, \Lambda}\alpha = \br{r\br{k - n} + r\br{r - 1}}\L^{r - 1}\alpha.
\end{equation}
The goal is for all $ k \in \cbr{0, \dots, n} $ and for all $ r \in \cbr{0, \dots, n - k} $, $ \L^r $ is injective. Double induction.
\begin{itemize}[leftmargin=0.5in]
\item[$ r = 0 $.] $ \L^0 = \id $, so ok.
\item[$ r > 0 $.] Assume $ k \ge 0 $. Assume that $ \L^r\alpha = 0 $ for some $ \alpha \in \Ci{X, \Omega_{X, \RR}^k} $. Since $ r \le n - k $,
$$ p = r\br{k - n} + r\br{r - 1} \ne 0. $$
By $ \br{\ref{eq:2}} $, $ \L^r\Lambda\alpha = p\L^{r - 1}\alpha $, so $ \L^{r - 1}\br{\L\Lambda\alpha - p\alpha} = 0 $. Then $ \L^{r - 1} $ is injective by induction, so $ \L\Lambda\alpha = p\alpha $.
\begin{itemize}[leftmargin=0.5in]
\item[$ k \le 1 $.] Since $ \Lambda\alpha $ is a $ \br{k - 2} $-form, $ \Lambda\alpha = 0 $, so $ \alpha = 0 $.
\item[$ k \ge 2 $.] Let $ \beta = \Lambda\alpha $ be a $ \br{k - 2} $-form. Then $ \L^{r + 1}\beta = p\L^r\alpha = 0 $, by $ \br{\ref{eq:2}} $. Since $ \L^{r + 1} $ is injective for all $ r + 2 \le n - \br{k - 2} $, $ \beta = 0 $, so $ \alpha = 0 $, since $ \L\beta = p\alpha $.
\end{itemize}
\end{itemize}
\end{proof}

\begin{theorem}[Hard Lefschetz theorem]
Let $ \br{X, \omega} $ be a compact K\"ahler manifold of dimension $ n $. Then
$$ \L^{n - k} : \H^k\br{X, \RR} \to \H^{2n - k}\br{X, \RR}, \qquad \L^{n - p - q} : \H^{p, q}\br{X} \to \H^{n - p, n - q}\br{X} $$
are isomorphisms.
\end{theorem}

\begin{proof}
$ \H^k\br{X, \RR} $ and $ \H^{2n - k}\br{X, \RR} $ have the same dimension because of Poincar\'e duality. It is enough to show that $ \L^r $ is injective for all $ r \le n - k $. It is enough to show that
$$ \L^r : \HHH^k\br{X, \RR} \to \HHH^{2r + k}\br{X, \RR} $$
is injective. By Proposition \ref{prop:6.42}, it is enough to show that $ \L^r\br{\HHH^k\br{X, \RR}} \subset \HHH^{2r + k}\br{X, \RR} $. By $ 1 $ of Lemma \ref{lem:6.41}, $ \sbr{\Delta, \L} = 0 $.
\end{proof}

The goal is the Lefschetz decomposition.

\begin{definition}
Let $ k \in \cbr{0, \dots, n} $. A $ k $-form $ \alpha $ on $ X $ is called \textbf{primitive} if $ \L^{n - k + 1}\alpha = 0 $.
\end{definition}

\begin{lemma}
\label{lem:6.45}
$ \alpha $ is primitive if and only if $ \Lambda\alpha = 0 $.
\end{lemma}

\begin{proof}
Recall
$$ \sbr{\L^r, \Lambda}\alpha = \br{r\br{k - n} + r\br{r - 1}}\L^{r - 1}\alpha. $$
If $ r = n - k + 1 $ then $ \L^{n - k + 1}\Lambda = \Lambda\L^{n - k + 1} $. Assume $ \alpha $ is primitive, so $ \L^{n - k + 1}\alpha = 0 $. Then $ \L^{n - k + 1}\Lambda\alpha = 0 $. Since $ \Lambda\alpha $ is a $ \br{k - 2} $-form, $ \L^{n - k + 1} $ is injective for all $ n - k + 1 \le n - \br{k - 2} $, so $ \Lambda\alpha = 0 $. Assume $ \Lambda\alpha = 0 $. Let $ r > 0 $ be minimal such that $ \L^r\alpha = 0 $. Then $ 0 = \sbr{\L^r, \Lambda}\alpha = r\br{k - n + r - 1}\L^{r - 1}\alpha $. Since $ \L^{r - 1}\alpha \ne 0 $, $ k - n + r - 1 = 0 $, so $ r = n - k + 1 $.
\end{proof}

\pagebreak

\begin{proposition}
For all $ k \in \cbr{0, \dots, n} $ and $ r $-forms $ \alpha $, there exists a unique decomposition
$$ \alpha = \sum_{r \ge 0} \L^r\alpha_r, $$
where $ \alpha_r $ are primitive $ \br{k - 2r} $-forms.
\end{proposition}

\begin{proof}
\hfill
\begin{itemize}
\item Existence. By induction on $ k $.
\begin{itemize}[leftmargin=0.5in]
\item[$ k \le 1 $.] Since $ \Lambda\alpha $ is a $ \br{k - 2} $-form, $ \Lambda\alpha = 0 $. By Lemma \ref{lem:6.45}, $ \alpha $ is primitive, so $ \alpha = \alpha_0 $.
\item[$ k \ge 2 $.] By Proposition \ref{prop:6.42} there exists a $ \br{k - 2} $-form $ \beta $ such that $ \L^{n - k + 2}\beta = \L^{n - k + 1}\alpha $, which is a $ \br{k + 2\br{n - k + 1}} $-form, so $ \L^{n - k + 1}\br{\alpha - \L\beta} = 0 $, where $ \alpha - \L\beta $ is a $ k $-form. By definition, $ \alpha_0 = \alpha - \L\beta $ is primitive, so $ \alpha = \alpha_0 + \L\beta $. By induction on $ k $, $ \beta = \sum_{r \ge 0} \L^r\beta_r $, where $ \beta_r $ are primitive. Then $ \alpha = \alpha_0 + \sum_{r \ge 0} \L^{r + 1}\beta_r $, so existence is ok.
\end{itemize}
\item Unicity. By induction on $ k $.
\begin{itemize}[leftmargin=0.5in]
\item[$ k = 0 $.] $ \alpha = \alpha_0 $ is ok.
\item[$ k > 0 $.] There exists a primitive $ \beta_r $ such that $ \sum_{r \ge 0} \L^r\beta_r = 0 $. Since $ \beta_0 $ is primitive $ k $-form, $ \L^{n - k + 1}\beta_0 = 0 $, so
$$ 0 = \L^{n - k + 1}\br{\sum_{r \ge 0} \L^r\beta_r} = \sum_{r \ge 1} \L^{n - k + 1 + r}\beta_r = \L^{r - k + 2}\br{\sum_{r \ge 1} \L^{r - 1}\beta_r}, $$
where $ \sum_{r \ge 1} \L^{r - 1}\beta_r $ are $ \br{k - 2} $-forms. By Proposition \ref{prop:6.42}, $ \sum_{r \ge 1} \L^{r - 1}\beta_r = 0 $. By induction $ \beta_r = 0 $ for all $ r \ge 1 $, so $ \beta_0 = 0 $.
\end{itemize}
\end{itemize}
\end{proof}

\begin{definition}
Let
$$ \H^{k - 2r}\br{X, \RR}_{\prim} = \cbr{\text{primitive $ \d $-closed $ \br{k - 2r} $-forms}}. $$
\end{definition}

\begin{theorem}[Lefschetz decomposition]
\label{thm:6.48}
Let $ \br{X, \omega} $ be a compact K\"ahler manifold of dimension $ n $. Then
$$ \H^k\br{X, \RR} = \bigoplus_r \L^r\H^{k - 2r}\br{X, \RR}_{\prim}. $$
The same holds over $ \CC $.
\end{theorem}

\begin{remark}
Let $ \br{X, \omega} $ be compact K\"ahler. Then
$$ \b_{k - 2}\br{X} \le \b_k\br{X}, \qquad k = 2, \dots, n, $$
and
$$ \h^{p - 1, q - 1}\br{X} \le \h^{p, q}\br{X}, \qquad p, q = 1, \dots, n. $$
\end{remark}

\lecture{30}{Tuesday}{17/03/20}

Lecture 30 is a problems class.

\end{document}