\def\module{M3P21 Geometry II: Algebraic Topology}
\def\lecturer{Dr Christian Urech}
\def\term{Spring 2019}

\def\thm{section}

\documentclass{article}

% Packages

\usepackage{amssymb}
\usepackage{amsthm}
\usepackage[UKenglish]{babel}
\usepackage{commath}
\usepackage{enumitem}
\usepackage{etoolbox}
\usepackage{fancyhdr}
\usepackage[margin=1in]{geometry}
\usepackage{graphicx}
\usepackage[hidelinks]{hyperref}
\usepackage[utf8]{inputenc}
\usepackage{listings}
\usepackage{mathtools}
\usepackage{stmaryrd}
\usepackage{tikz-cd}
\usepackage{csquotes}

% Formatting

\addto\captionsUKenglish{\renewcommand{\abstractname}{Syllabus}}
\delimitershortfall5pt
\ifx\thm\undefined\newtheorem{n}{}\else\newtheorem{n}{}[\thm]\fi
\newcommand\newoperator[1]{\ifcsdef{#1}{\cslet{#1}{\relax}}{}\csdef{#1}{\operatorname{#1}}}
\setlength{\parindent}{0cm}

% Environments

\theoremstyle{plain}
\newtheorem{algorithm}[n]{Algorithm}
\newtheorem*{algorithm*}{Algorithm}
\newtheorem{algorithm**}{Algorithm}
\newtheorem{conjecture}[n]{Conjecture}
\newtheorem*{conjecture*}{Conjecture}
\newtheorem{conjecture**}{Conjecture}
\newtheorem{corollary}[n]{Corollary}
\newtheorem*{corollary*}{Corollary}
\newtheorem{corollary**}{Corollary}
\newtheorem{lemma}[n]{Lemma}
\newtheorem*{lemma*}{Lemma}
\newtheorem{lemma**}{Lemma}
\newtheorem{proposition}[n]{Proposition}
\newtheorem*{proposition*}{Proposition}
\newtheorem{proposition**}{Proposition}
\newtheorem{theorem}[n]{Theorem}
\newtheorem*{theorem*}{Theorem}
\newtheorem{theorem**}{Theorem}

\theoremstyle{definition}
\newtheorem{aim}[n]{Aim}
\newtheorem*{aim*}{Aim}
\newtheorem{aim**}{Aim}
\newtheorem{axiom}[n]{Axiom}
\newtheorem*{axiom*}{Axiom}
\newtheorem{axiom**}{Axiom}
\newtheorem{condition}[n]{Condition}
\newtheorem*{condition*}{Condition}
\newtheorem{condition**}{Condition}
\newtheorem{definition}[n]{Definition}
\newtheorem*{definition*}{Definition}
\newtheorem{definition**}{Definition}
\newtheorem{example}[n]{Example}
\newtheorem*{example*}{Example}
\newtheorem{example**}{Example}
\newtheorem{exercise}[n]{Exercise}
\newtheorem*{exercise*}{Exercise}
\newtheorem{exercise**}{Exercise}
\newtheorem{fact}[n]{Fact}
\newtheorem*{fact*}{Fact}
\newtheorem{fact**}{Fact}
\newtheorem{goal}[n]{Goal}
\newtheorem*{goal*}{Goal}
\newtheorem{goal**}{Goal}
\newtheorem{law}[n]{Law}
\newtheorem*{law*}{Law}
\newtheorem{law**}{Law}
\newtheorem{plan}[n]{Plan}
\newtheorem*{plan*}{Plan}
\newtheorem{plan**}{Plan}
\newtheorem{problem}[n]{Problem}
\newtheorem*{problem*}{Problem}
\newtheorem{problem**}{Problem}
\newtheorem{question}[n]{Question}
\newtheorem*{question*}{Question}
\newtheorem{question**}{Question}
\newtheorem{warning}[n]{Warning}
\newtheorem*{warning*}{Warning}
\newtheorem{warning**}{Warning}
\newtheorem{acknowledgements}[n]{Acknowledgements}
\newtheorem*{acknowledgements*}{Acknowledgements}
\newtheorem{acknowledgements**}{Acknowledgements}
\newtheorem{annotations}[n]{Annotations}
\newtheorem*{annotations*}{Annotations}
\newtheorem{annotations**}{Annotations}
\newtheorem{assumption}[n]{Assumption}
\newtheorem*{assumption*}{Assumption}
\newtheorem{assumption**}{Assumption}
\newtheorem{conclusion}[n]{Conclusion}
\newtheorem*{conclusion*}{Conclusion}
\newtheorem{conclusion**}{Conclusion}
\newtheorem{claim}[n]{Claim}
\newtheorem*{claim*}{Claim}
\newtheorem{claim**}{Claim}
\newtheorem{notation}[n]{Notation}
\newtheorem*{notation*}{Notation}
\newtheorem{notation**}{Notation}
\newtheorem{note}[n]{Note}
\newtheorem*{note*}{Note}
\newtheorem{note**}{Note}
\newtheorem{remark}[n]{Remark}
\newtheorem*{remark*}{Remark}
\newtheorem{remark**}{Remark}

% Lectures

\newcommand{\lecture}[3]{ % Lecture
  \marginpar{
    Lecture #1 \\
    #2 \\
    #3
  }
}

% Blackboard

\renewcommand{\AA}{\mathbb{A}} % Blackboard A
\newcommand{\BB}{\mathbb{B}}   % Blackboard B
\newcommand{\CC}{\mathbb{C}}   % Blackboard C
\newcommand{\DD}{\mathbb{D}}   % Blackboard D
\newcommand{\EE}{\mathbb{E}}   % Blackboard E
\newcommand{\FF}{\mathbb{F}}   % Blackboard F
\newcommand{\GG}{\mathbb{G}}   % Blackboard G
\newcommand{\HH}{\mathbb{H}}   % Blackboard H
\newcommand{\II}{\mathbb{I}}   % Blackboard I
\newcommand{\JJ}{\mathbb{J}}   % Blackboard J
\newcommand{\KK}{\mathbb{K}}   % Blackboard K
\newcommand{\LL}{\mathbb{L}}   % Blackboard L
\newcommand{\MM}{\mathbb{M}}   % Blackboard M
\newcommand{\NN}{\mathbb{N}}   % Blackboard N
\newcommand{\OO}{\mathbb{O}}   % Blackboard O
\newcommand{\PP}{\mathbb{P}}   % Blackboard P
\newcommand{\QQ}{\mathbb{Q}}   % Blackboard Q
\newcommand{\RR}{\mathbb{R}}   % Blackboard R
\renewcommand{\SS}{\mathbb{S}} % Blackboard S
\newcommand{\TT}{\mathbb{T}}   % Blackboard T
\newcommand{\UU}{\mathbb{U}}   % Blackboard U
\newcommand{\VV}{\mathbb{V}}   % Blackboard V
\newcommand{\WW}{\mathbb{W}}   % Blackboard W
\newcommand{\XX}{\mathbb{X}}   % Blackboard X
\newcommand{\YY}{\mathbb{Y}}   % Blackboard Y
\newcommand{\ZZ}{\mathbb{Z}}   % Blackboard Z

% Brackets

\renewcommand{\eval}[1]{\left. #1 \right|}          % Evaluation
\newcommand{\br}{\del}                              % Brackets
\newcommand{\abr}[1]{\left\langle #1 \right\rangle} % Angle brackets
\newcommand{\fbr}[1]{\left\lfloor #1 \right\rfloor} % Floor brackets
\newcommand{\lbr}[1]{\left\lfloor #1 \right\rfloor} % Ceiling brackets
\newcommand{\st}{\ \middle| \ }                     % Such that

% Calligraphic

\newcommand{\AAA}{\mathcal{A}} % Calligraphic A
\newcommand{\BBB}{\mathcal{B}} % Calligraphic B
\newcommand{\CCC}{\mathcal{C}} % Calligraphic C
\newcommand{\DDD}{\mathcal{D}} % Calligraphic D
\newcommand{\EEE}{\mathcal{E}} % Calligraphic E
\newcommand{\FFF}{\mathcal{F}} % Calligraphic F
\newcommand{\GGG}{\mathcal{G}} % Calligraphic G
\newcommand{\HHH}{\mathcal{H}} % Calligraphic H
\newcommand{\III}{\mathcal{I}} % Calligraphic I
\newcommand{\JJJ}{\mathcal{J}} % Calligraphic J
\newcommand{\KKK}{\mathcal{K}} % Calligraphic K
\newcommand{\LLL}{\mathcal{L}} % Calligraphic L
\newcommand{\MMM}{\mathcal{M}} % Calligraphic M
\newcommand{\NNN}{\mathcal{N}} % Calligraphic N
\newcommand{\OOO}{\mathcal{O}} % Calligraphic O
\newcommand{\PPP}{\mathcal{P}} % Calligraphic P
\newcommand{\QQQ}{\mathcal{Q}} % Calligraphic Q
\newcommand{\RRR}{\mathcal{R}} % Calligraphic R
\newcommand{\SSS}{\mathcal{S}} % Calligraphic S
\newcommand{\TTT}{\mathcal{T}} % Calligraphic T
\newcommand{\UUU}{\mathcal{U}} % Calligraphic U
\newcommand{\VVV}{\mathcal{V}} % Calligraphic V
\newcommand{\WWW}{\mathcal{W}} % Calligraphic W
\newcommand{\XXX}{\mathcal{X}} % Calligraphic X
\newcommand{\YYY}{\mathcal{Y}} % Calligraphic Y
\newcommand{\ZZZ}{\mathcal{Z}} % Calligraphic Z

% Fraktur

\newcommand{\aaa}{\mathfrak{a}}   % Fraktur a
\newcommand{\bbb}{\mathfrak{b}}   % Fraktur b
\newcommand{\ccc}{\mathfrak{c}}   % Fraktur c
\newcommand{\ddd}{\mathfrak{d}}   % Fraktur d
\newcommand{\eee}{\mathfrak{e}}   % Fraktur e
\newcommand{\fff}{\mathfrak{f}}   % Fraktur f
\renewcommand{\ggg}{\mathfrak{g}} % Fraktur g
\newcommand{\hhh}{\mathfrak{h}}   % Fraktur h
\newcommand{\iii}{\mathfrak{i}}   % Fraktur i
\newcommand{\jjj}{\mathfrak{j}}   % Fraktur j
\newcommand{\kkk}{\mathfrak{k}}   % Fraktur k
\renewcommand{\lll}{\mathfrak{l}} % Fraktur l
\newcommand{\mmm}{\mathfrak{m}}   % Fraktur m
\newcommand{\nnn}{\mathfrak{n}}   % Fraktur n
\newcommand{\ooo}{\mathfrak{o}}   % Fraktur o
\newcommand{\ppp}{\mathfrak{p}}   % Fraktur p
\newcommand{\qqq}{\mathfrak{q}}   % Fraktur q
\newcommand{\rrr}{\mathfrak{r}}   % Fraktur r
\newcommand{\sss}{\mathfrak{s}}   % Fraktur s
\newcommand{\ttt}{\mathfrak{t}}   % Fraktur t
\newcommand{\uuu}{\mathfrak{u}}   % Fraktur u
\newcommand{\vvv}{\mathfrak{v}}   % Fraktur v
\newcommand{\www}{\mathfrak{w}}   % Fraktur w
\newcommand{\xxx}{\mathfrak{x}}   % Fraktur x
\newcommand{\yyy}{\mathfrak{y}}   % Fraktur y
\newcommand{\zzz}{\mathfrak{z}}   % Fraktur z

% Geometry

\newcommand{\CP}{\mathbb{CP}}                                              % Complex projective space
\newcommand{\iintd}[4]{\iint_{#1} \, #2 \, \dif #3 \, \dif #4}             % Double integral
\newcommand{\RP}{\mathbb{RP}}                                              % Real projective space
\newcommand{\intd}[4]{\int_{#1}^{#2} \, #3 \, \dif #4}                     % Single integral
\newcommand{\iiintd}[5]{\iint_{#1} \, #2 \, \dif #3 \, \dif #4 \, \dif #5} % Triple integral

% Logic

\newcommand{\iffb}[2]{\br{#1 \leftrightarrow #2}} % Biconditional
\newcommand{\andb}[2]{\br{#1 \land #2}}           % Conjunction
\newcommand{\orb}[2]{\br{#1 \lor #2}}             % Disjunction
\newcommand{\nib}[2]{\br{#1 \notin #2}}           % Element of
\newcommand{\eqb}[2]{\br{#1 = #2}}                % Equal to
\newcommand{\teb}[1]{\br{\exists #1}}             % Existential quantifier
\newcommand{\impb}[2]{\br{#1 \rightarrow #2}}     % Implication
\newcommand{\ltb}[2]{\br{#1 < #2}}                % Less than
\newcommand{\leb}[2]{\br{#1 \le #2}}              % Less than or equal to
\newcommand{\notb}[1]{\br{\neg #1}}               % Negation
\newcommand{\inb}[2]{\br{#1 \in #2}}              % Not element of
\newcommand{\neb}[2]{\br{#1 \ne #2}}              % Not equal to
\newcommand{\subb}[2]{\br{#1 \subseteq #2}}       % Subset
\newcommand{\fab}[1]{\br{\forall #1}}             % Universal quantifier

% Maps

\newcommand{\bijection}[7][]{    % Bijection
  \ifx &#1&
    \begin{array}{rcl}
      #2 & \longleftrightarrow & #3 \\
      #4 & \longmapsto         & #5 \\
      #6 & \longmapsfrom       & #7
    \end{array}
  \else
    \begin{array}{ccrcl}
      #1 & : & #2 & \longrightarrow & #3 \\
         &   & #4 & \longmapsto     & #5 \\
         &   & #6 & \longmapsfrom   & #7
    \end{array}
  \fi
}
\newcommand{\birational}[7][]{   % Birational map
  \ifx &#1&
    \begin{array}{rcl}
      #2 & \dashrightarrow & #3 \\
      #4 & \longmapsto     & #5 \\
      #6 & \longmapsfrom   & #7
    \end{array}
  \else
    \begin{array}{ccrcl}
      #1 & : & #2 & \dashrightarrow & #3 \\
         &   & #4 & \longmapsto     & #5 \\
         &   & #6 & \longmapsfrom   & #7
    \end{array}
  \fi
}
\newcommand{\correspondence}[2]{ % Correspondence
  \cbr{
    \begin{array}{c}
      #1
    \end{array}
  }
  \qquad
  \leftrightsquigarrow
  \qquad
  \cbr{
    \begin{array}{c}
      #2
    \end{array}
  }
}
\newcommand{\function}[5][]{     % Function
  \ifx &#1&
    \begin{array}{rcl}
      #2 & \longrightarrow & #3 \\
      #4 & \longmapsto     & #5
    \end{array}
  \else
    \begin{array}{ccrcl}
      #1 & : & #2 & \longrightarrow & #3 \\
         &   & #4 & \longmapsto     & #5
    \end{array}
  \fi
}
\newcommand{\functions}[7][]{    % Functions
  \ifx &#1&
    \begin{array}{rcl}
      #2 & \longrightarrow & #3 \\
      #4 & \longmapsto     & #5 \\
      #6 & \longmapsto     & #7
    \end{array}
  \else
    \begin{array}{ccrcl}
      #1 & : & #2 & \longrightarrow & #3 \\
         &   & #4 & \longmapsto     & #5 \\
         &   & #6 & \longmapsto     & #7
    \end{array}
  \fi
}
\newcommand{\rational}[5][]{     % Rational map
  \ifx &#1&
    \begin{array}{rcl}
      #2 & \dashrightarrow & #3 \\
      #4 & \longmapsto     & #5
    \end{array}
  \else
    \begin{array}{ccrcl}
      #1 & : & #2 & \dashrightarrow & #3 \\
         &   & #4 & \longmapsto     & #5
    \end{array}
  \fi
}

% Matrices

\newcommand{\onebytwo}[2]{      % One by two matrix
  \begin{pmatrix}
    #1 & #2
  \end{pmatrix}
}
\newcommand{\onebythree}[3]{    % One by three matrix
  \begin{pmatrix}
    #1 & #2 & #3
  \end{pmatrix}
}
\newcommand{\twobyone}[2]{      % Two by one matrix
  \begin{pmatrix}
    #1 \\
    #2
  \end{pmatrix}
}
\newcommand{\twobytwo}[4]{      % Two by two matrix
  \begin{pmatrix}
    #1 & #2 \\
    #3 & #4
  \end{pmatrix}
}
\newcommand{\threebyone}[3]{    % Three by one matrix
  \begin{pmatrix}
    #1 \\
    #2 \\
    #3
  \end{pmatrix}
}
\newcommand{\threebythree}[9]{  % Three by three matrix
  \begin{pmatrix}
    #1 & #2 & #3 \\
    #4 & #5 & #6 \\
    #7 & #8 & #9
  \end{pmatrix}
}
\newcommand{\twobytwosmall}[4]{ % Two by two small matrix
  \begin{psmallmatrix}
    #1 & #2 \\
    #3 & #4
  \end{psmallmatrix}
}

% Number theory

\renewcommand{\symbol}[2]{\br{\tfrac{#1}{#2}}} % Power residue symbol
\newcommand{\unit}[1]{\br{\ZZ / #1\ZZ}^\times} % Unit group

% Operators

\newoperator{ab}    % Abelian
\newoperator{AG}    % Affine geometry
\newoperator{alg}   % Algebraic
\newoperator{Ann}   % Annihilator
\newoperator{area}  % Area
\newoperator{Aut}   % Automorphism
\newoperator{card}  % Cardinality
\newoperator{ch}    % Characteristic
\newoperator{Cl}    % Class
\newoperator{Coker} % Cokernel
\newoperator{col}   % Column
\newoperator{Corr}  % Correspondence
\newoperator{diam}  % Diameter
\newoperator{Disc}  % Discriminant
\newoperator{dom}   % Domain
\newoperator{Eig}   % Eigenvalue
\newoperator{Em}    % Embedding
\newoperator{End}   % Endomorphism
\newoperator{fin}   % Finite
\newoperator{Fix}   % Fixed
\newoperator{Frac}  % Fraction
\newoperator{Frob}  % Frobenius
\newoperator{Fun}   % Function
\newoperator{Gal}   % Galois
\newoperator{GL}    % General linear
\newoperator{Ham}   % Hamming
\newoperator{Homeo} % Homeomorphism
\newoperator{Hom}   % Homomorphism
\newoperator{id}    % Identity
\newoperator{Im}    % Image
\newoperator{Ind}   % Index
\newoperator{Ker}   % Kernel
\newoperator{lcm}   % Least common multiple
\newoperator{Mat}   % Matrix
\newoperator{mult}  % Multiplicity
\newoperator{new}   % New
\newoperator{Nm}    % Norm
\newoperator{old}   % Old
\newoperator{op}    % Opposite
\newoperator{ord}   % Order
\newoperator{Pay}   % Payley
\newoperator{PG}    % Projective geometry
\newoperator{PGL}   % Projective general linear
\newoperator{PSL}   % Projective special linear
\newoperator{rad}   % Radical
\newoperator{ran}   % Range
\newoperator{Res}   % Residue
\newoperator{rk}    % Rank
\newoperator{Re}    % Real
\newoperator{row}   % Row
\newoperator{sgn}   % Sign
\newoperator{Sing}  % Singular
\newoperator{SK}    % Skeleton
\newoperator{sp}    % Span
\newoperator{SL}    % Special linear
\newoperator{SO}    % Special orthogonal
\newoperator{Spec}  % Spectrum
\newoperator{Stab}  % Stabiliser
\newoperator{star}  % Star
\newoperator{srg}   % Strongly regular graph
\newoperator{supp}  % Support
\newoperator{Sym}   % Symmetric
\newoperator{tors}  % Torsion
\newoperator{Tr}    % Trace
\newoperator{vol}   % Volume
\newoperator{wt}    % Weight

% Roman

\newcommand{\A}{\mathrm{A}}   % Roman A
\newcommand{\B}{\mathrm{B}}   % Roman B
\newcommand{\C}{\mathrm{C}}   % Roman C
\newcommand{\D}{\mathrm{D}}   % Roman D
\newcommand{\E}{\mathrm{E}}   % Roman E
\newcommand{\F}{\mathrm{F}}   % Roman F
\newcommand{\G}{\mathrm{G}}   % Roman G
\renewcommand{\H}{\mathrm{H}} % Roman H
\newcommand{\I}{\mathrm{I}}   % Roman I
\newcommand{\J}{\mathrm{J}}   % Roman J
\newcommand{\K}{\mathrm{K}}   % Roman K
\renewcommand{\L}{\mathrm{L}} % Roman L
\newcommand{\M}{\mathrm{M}}   % Roman M
\newcommand{\N}{\mathrm{N}}   % Roman N
\renewcommand{\O}{\mathrm{O}} % Roman O
\renewcommand{\P}{\mathrm{P}} % Roman P
\newcommand{\Q}{\mathrm{Q}}   % Roman Q
\newcommand{\R}{\mathrm{R}}   % Roman R
\renewcommand{\S}{\mathrm{S}} % Roman S
\newcommand{\T}{\mathrm{T}}   % Roman T
\newcommand{\U}{\mathrm{U}}   % Roman U
\newcommand{\V}{\mathrm{V}}   % Roman V
\newcommand{\W}{\mathrm{W}}   % Roman W
\newcommand{\X}{\mathrm{X}}   % Roman X
\newcommand{\Y}{\mathrm{Y}}   % Roman Y
\newcommand{\Z}{\mathrm{Z}}   % Roman Z

\renewcommand{\a}{\mathrm{a}} % Roman a
\renewcommand{\b}{\mathrm{b}} % Roman b
\renewcommand{\c}{\mathrm{c}} % Roman c
\renewcommand{\d}{\mathrm{d}} % Roman d
\newcommand{\e}{\mathrm{e}}   % Roman e
\newcommand{\f}{\mathrm{f}}   % Roman f
\newcommand{\g}{\mathrm{g}}   % Roman g
\newcommand{\h}{\mathrm{h}}   % Roman h
\renewcommand{\i}{\mathrm{i}} % Roman i
\renewcommand{\j}{\mathrm{j}} % Roman j
\renewcommand{\k}{\mathrm{k}} % Roman k
\renewcommand{\l}{\mathrm{l}} % Roman l
\newcommand{\m}{\mathrm{m}}   % Roman m
\renewcommand{\n}{\mathrm{n}} % Roman n
\renewcommand{\o}{\mathrm{o}} % Roman o
\newcommand{\p}{\mathrm{p}}   % Roman p
\newcommand{\q}{\mathrm{q}}   % Roman q
\renewcommand{\r}{\mathrm{r}} % Roman r
\newcommand{\s}{\mathrm{s}}   % Roman s
\renewcommand{\t}{\mathrm{t}} % Roman t
\renewcommand{\u}{\mathrm{u}} % Roman u
\renewcommand{\v}{\mathrm{v}} % Roman v
\newcommand{\w}{\mathrm{w}}   % Roman w
\newcommand{\x}{\mathrm{x}}   % Roman x
\newcommand{\y}{\mathrm{y}}   % Roman y
\newcommand{\z}{\mathrm{z}}   % Roman z

% Tikz

\tikzset{
  arrow symbol/.style={"#1" description, allow upside down, auto=false, draw=none, sloped},
  subset/.style={arrow symbol={\subset}},
  cong/.style={arrow symbol={\cong}}
}

% Fancy header

\pagestyle{fancy}
\lhead{\module}
\rhead{\nouppercase{\leftmark}}

% Make title

\title{\module}
\author{Lectured by \lecturer \\ Typed by David Kurniadi Angdinata}
\date{\term}

\begin{document}

% Title page
\maketitle
\cover
\vfill
\begin{abstract}
\noindent\syllabus
\end{abstract}

\pagebreak

% Contents page
\tableofcontents

\pagebreak

% Document page
\setcounter{section}{-1}

\section{Some underlying geometric notions}

\subsection{Introduction}

\lecture{1}{Friday}{11/01/19}

Combines topological spaces with algebraic objects, groups.
\begin{itemize}
\item How to show that a torus is not homeomorphic to a sphere?
\item How to show that $ \R^n \ncong \R^m $ if $ n \ne m $?
\end{itemize}

Content is fundamental groups and homology. We will follow chapter one and two from
\begin{itemize}
\item A Hatcher, Algebraic topology, 2002
\end{itemize}

Prerequisites are the following.
\begin{itemize}
\item Point set topology. Topological spaces, continuous maps, product and quotient topologies, Hausdorff spaces, etc.
\item Basic group theory. Normal subgroups and quotients, isomorphism theorems, free groups, presentation of groups, etc.
\end{itemize}

\subsection{Homotopy}

Let $ X, Y $ be topological spaces and $ I = \sb{0, 1} $.

\begin{definition}
A \textbf{homotopy} is a continuous map $ F : X \times I \to Y $. For every $ t \in I $ we obtain a continuous map
$$ \function[f_t]{X}{Y}{x}{f_t\rb{x} = F\rb{x, t}}. $$
\end{definition}

\begin{definition}
Two continuous maps $ f_0, f_1 : X \to Y $ are \textbf{homotopic} if there exists a homotopy $ F : X \times I \to Y $ such that
$$ f_0\rb{x} = F\rb{x, 0}, \qquad f_1\rb{x} = F\rb{x, 1}, $$
for all $ x \in X $. We write $ f_0 \cong f_1 $.
\end{definition}

(Exercise: this is an equivalence relation)

\begin{definition}
Let $ A \subseteq X $ be a subspace. A \textbf{retraction} of $ X $ onto $ A $ is a continuous map $ r : X \to A $ such that
\begin{itemize}
\item $ r\rb{X} = A $, and
\item $ r\mid_A = id_A $.
\end{itemize}
\end{definition}

\begin{example1}
If $ X \ne \emptyset $, $ p \in X $, then $ X $ retracts to $ p $ by the constant map $ X \to \cb{p} $.
\end{example1}

\begin{definition}
A \textbf{deformation retraction} of $ X $ onto $ A \subseteq X $ is a retraction that is homotopic to the identity. That is, there is a continuous map
$$ \function[F]{X \times I}{A}{\rb{x, t}}{f_t\rb{x}}, $$
such that $ f_0 = id_X $ and $ f_1 : X \to A $ is the deformation retraction.
\end{definition}

\begin{example1}
The closed $ n $-dimensional \textbf{$ n $-disc}
$$ D^n = \cb{x \in \R^n \mid \abs{x} \le 1} $$
deformation retracts to $ \rb{0, \dots, 0} \in \R^n $. Let $ f_t\rb{x} = t \cdot x $. $ t = 1 $ gives $ f_1 = id_{D^n} $ and $ t = 0 $ gives $ f_0 : D^n \to \rb{0, \dots, 0} $.
\end{example1}

\begin{example1}
Let $ S^n $ be the \textbf{$ n $-sphere},
$$ \partial D^{n + 1} = S^n = \cb{x \in \R^n \mid \abs{x} = 1}. $$
The cylinder $ S^n \times I $ deformation retracts to $ S^n \times \cb{0} $, by defining $ f_t\rb{x, r} = \rb{x, t \cdot r} $.
\end{example1}

An observation is if $ X $ is a topological space, and $ f : X \to \cb{p} $ for $ p \in X $ is a deformation retraction of $ X $ to $ p $, then $ X $ is path connected. Indeed, if $ F : X \times I \to X $ is a homotopy from $ id_X $ to $ f $ and $ x \in X $ is a point, then this gives a path
$$ \function{I}{X}{t}{F\rb{x, t}} $$
that connects $ x $ to $ p $. This implies that not all retractions are deformation retractions.

\begin{example1}
A retraction that is not a deformation retraction. Take a space that is not path connected and retract it to a point. Let $ X = \cb{0, 1} $ with discrete topology. $ x \mapsto 0 $ is a retraction, but not a deformation retraction because $ X $ is not path connected.
\end{example1}

\begin{definition}
A continuous map $ f : X \to Y $ is a \textbf{homotopy equivalence} if there is a continuous map $ g : Y \to X $ such that $ fg \cong id_Y $ and $ gf \cong id_X $. If there exists a homotopy equivalence between $ X $ and $ Y $, $ X $ and $ Y $ are \textbf{homotopy equivalent} or they have the same \textbf{homotopy type}.
\end{definition}

\begin{lemma}
A deformation retraction $ f : X \to A $ is a homotopy equivalence.
\end{lemma}

\begin{proof}
Let $ i : A \hookrightarrow X $ be the inclusion map. Then $ fi = id_A $ and $ if = f \cong id_X $ by definition.
\end{proof}

\begin{example1}
The disc with two holes is equivalent to $ \infty $.
\end{example1}

\begin{example1}
$ \R^n $ deformation retracts to a point, by $ f_t\rb{x} = t \cdot x $.
\end{example1}

\begin{definition}
\hfill
\begin{itemize}
\item $ X $ is \textbf{contractible} if it is homotopy equivalent to a point.
\item A continuous map is \textbf{nullhomotopic} if it is homotopy equivalent to a constant map.
\end{itemize}
\end{definition}

\subsection{Cell complexes}

\begin{example1}
The torus $ S^1 \times S^1 $ is the union of a point, two open intervals, and the open disc $ Int\rb{D^2} $.
\end{example1}

These are called \textbf{cells}. Can think of discs $ D^n $ glued together.

\lecture{2}{Tuesday}{15/01/19}

\begin{definition}
A \textbf{CW-complex}, or \textbf{cell complex}, is a topological space $ X $ such that there exists a decomposition
$$ X = \bigcup_{n \in \N} X^n, $$
where the $ X^n $ are constructed inductively in the following way.
\begin{itemize}
\item $ X^n $ is a discrete set.
\item For each $ n \ge 0 $ there is an collection of closed $ n $-discs $ \cb{D_\alpha^n} $ together with continuous maps
$$ \phi_\alpha : \partial D_\alpha^n \to X^{n - 1}, $$
such that
$$ X^n = \dfrac{X^{n - 1} \sqcup \bigsqcup_\alpha D_\alpha^n}{\sim}, $$
where $ x \sim \phi_\alpha\rb{x} $ for all $ x \in \partial D_\alpha^n $ for all $ \alpha $.
\item A subset $ U \subseteq X $ is open if and only if $ U \cap X^n $ is open for all $ n $.
\end{itemize}
\end{definition}

\begin{remark1}
\hfill
\begin{itemize}
\item As a set,
$$ X^n = X^{n - 1} \sqcup \bigsqcup_\alpha e_\alpha^n, $$
where each $ e_\alpha^n $ is homeomorphic to an open $ n $-disc. These $ e_\alpha^n $ are called the \textbf{$ n $-cells} of $ X $.
\item If $ X = X^m $ for some $ m $, then $ X $ is called \textbf{finite dimensional}. The minimal $ m $ such that $ X = X^m $ is the \textbf{dimension} of $ X $.
\end{itemize}
\end{remark1}

\begin{example1}
\hfill
\begin{itemize}
\item $ \sb{0, 1} $ is a CW-complex.
\item $ \R $ is a CW-complex.
\item $ S^1 $ is a CW-complex.
\item A graph is a CW-complex.
\item $ S^n = D^n / \partial D^n $ is a CW-complex. See worksheet $ 1 $.
\end{itemize}
Can also decompose CW-complexes.
\begin{itemize}
\item The sphere $ S^2 $ is one $ 0 $-cell, one $ 1 $-cell, and two $ 2 $-cells.
\item The torus $ S^1 \times S^1 $ is one $ 0 $-cell, two $ 1 $-cells, and one $ 2 $-cell.
\item The M\"obius strip is two $ 0 $-cells, three $ 1 $-cells, and one $ 2 $-cell.
\item The Klein bottle is one $ 0 $-cell, two $ 1 $-cells, and one $ 2 $-cell.
\end{itemize}
\end{example1}

\begin{definition}
If $ X $ is a CW-complex with finitely many cells the \textbf{Euler characteristic} $ \chi\rb{X} $ of $ X $ is the number of even cells minus the number of odd cells.
\end{definition}

A fact is that $ \chi\rb{X} $ does not depend of the choice of cells decomposition.

\begin{example1}
\hfill
\begin{itemize}
\item $ \chi\rb{S^n} = 0 $ if $ n $ is odd and $ \chi\rb{S^n} = 2 $ if $ n $ is even.
\item $ \chi\rb{S^1 \times S^1} = 0 $.
\end{itemize}
\end{example1}

This is the generalisation of the following observation by Leonhard Euler. Let $ P $ be a convex polyhedron, where
\begin{itemize}
\item $ V $ is the number of vertices of $ P $,
\item $ E $ is the number of edges of $ P $, and
\item $ F $ is the number of faces of $ P $.
\end{itemize}
Then $ V - E + F = 2 $.

\begin{example1}
A topological space that is not a CW-complex. $ X = \cb{0, 1} $ with trivial topology does not contain any closed points. A fact is that CW-complexes are always Hausdorff.
\end{example1}

\pagebreak

\section{The fundamental group}

\subsection{Paths and homotopy}

Let $ X $ be a topological space. A \textbf{path} is a continuous map $ f : I \to X $, where $ I = \sb{0, 1} $.

\begin{definition}
Two paths $ f_0, f_1 $ are \textbf{homotopic} if there exists a homotopy between $ f_0 $ and $ f_1 $ preserving the endpoints, that is a continuous map
$$ \function[F]{I \times I}{X}{\rb{s, t}}{f_t\rb{s}}, $$
such that
$$ f_t\rb{0} = f_0\rb{0}, \qquad f_t\rb{1} = f_0\rb{1}, $$
for all $ t \in I $, and
$$ F\rb{s, 0} = f_0\rb{s}, \qquad F\rb{s, 1} = f_1\rb{s}, $$
for all $ s \in I $.
\end{definition}

\begin{example1}
Let $ X \subseteq \R^n $ be a convex set. Then all the paths in $ X $ are homotopic if they have the same endpoints.
\end{example1}

\begin{proof}
Let $ f_0, f_1 : I \to X $ be two paths such that $ f_0\rb{0} = f_1\rb{0} $ and $ f_0\rb{1} = f_1\rb{1} $. Define $ f_t\rb{s} = \rb{1 - t}f_0\rb{s} + tf_1\rb{s} $.
\end{proof}

\begin{lemma}
Being homotopic is an equivalence relation on the set of paths with fixed endpoints. We will write $ f_0 \cong f_1 $ for two homotopic paths $ f_0 $ and $ f_1 $.
\end{lemma}

\begin{proof}
\hfill
\begin{itemize}
\item $ f $ is homotopic to $ f $.
\item If $ f_0 $ is homotopic to $ f_1 $ by a homotopy $ f_t $, then $ f_1 $ is homotopic to $ f_0 $ by the homotopy $ f_{1 - t} $.
\item If $ f_0 $ is homotopic to $ f_1 $ by a homotopy $ f_t $ and $ f_1 = g_0 $ is homotopic to $ g_1 $ by a homotopy $ g_t $, then $ f_0 $ is homotopic to $ g_1 $ by the homotopy
$$ h_t =
\begin{cases}
f_{2t} & 0 \le t \le \tfrac{1}{2} \\
g_{2t - 1} & \tfrac{1}{2} \le t \le 1
\end{cases}.
$$
Then
$$ \function[H]{I \times I}{X}{\rb{s, t}}{h_t\rb{s}} $$
is continuous because its restriction to the closed subsets $ I \times \sb{0, 1 / 2} $ and $ I \times \sb{1 / 2, 1} $ is continuous, since if the restriction to two closed subsets is continuous then the restriction to the union of these subsets is continuous.
\end{itemize}
\end{proof}

\lecture{3}{Wednesday}{16/01/19}

Let $ X $ be a topological space and $ I = \sb{0, 1} $. If $ f : I \to X $ is a path, $ \sb{f} $ is the class of all paths on $ X $ homotopic to $ f $.

\begin{definition}
Let $ f, g : I \to X $ be two paths such that $ f\rb{1} = g\rb{0} $. The \textbf{product path} $ f \cdot g $ is the path
$$ \rb{f \cdot g}\rb{s} =
\begin{cases}
f\rb{2s} & 0 \le s \le \tfrac{1}{2} \\
g\rb{2s - 1} & \tfrac{1}{2} \le s \le 1
\end{cases}.
$$
\end{definition}

A convention is that whenever we write $ f \cdot g $ we implicitly assume $ f\rb{1} = g\rb{0} $.

\begin{lemma}
\label{lem:1.2}
Let $ f_0, f_1, g_0, g_1 $ be paths on $ X $ such that $ f_1 \cong f_0 $ and $ g_0 \cong g_1 $. Then $ f_0 \cdot g_0 \cong f_1 \cdot g_1 $.
\end{lemma}

\begin{proof}
$$ \function{I \times I}{X}{\rb{s, t}}{\rb{f_t \cdot g_t}\rb{s}} $$
is a homotopy between $ f_0 \cdot g_0 $ and $ f_1 \cdot g_1 $.
\end{proof}

\begin{remark1}
\textbf{Reparametrisation}. Let $ \phi : \sb{0, 1} \to \sb{0, 1} $ be continuous such that $ \phi\rb{0} = 0 $ and $ \phi\rb{1} = 1 $. If $ f : I \to X $ is a path, then $ f \circ \phi \cong f $.
\end{remark1}

\begin{proof}
Define $ \phi_t\rb{s} = \rb{1 - t}\phi\rb{s} + ts $, then $ f \circ \phi_t $ is a homotopy between $ f \circ \phi $ and $ f $.
\end{proof}

For $ x \in X $, let the \textbf{constant path} at $ x $ be
$$ \function[c_x]{I}{X}{s}{x}. $$
For a path $ f : I \to X $, define
$$ \function[f^{-1}]{I}{X}{s}{f\rb{1 - s}}. $$

\begin{lemma}
\label{lem:1.3}
Let $ f, g, h : I \to X $ be paths. Then
\begin{enumerate}
\item $ \rb{f \cdot g} \cdot h \cong f \cdot \rb{g \cdot h} $,
\item $ f \cdot c_{f\rb{1}} \cong f $ and $ c_{f\rb{0}} \cdot f \cong f $, and
\item $ f \cdot f^{-1} \cong c_{f\rb{0}} $ and $ f^{-1} \cdot f \cong c_{f\rb{1}} $.
\end{enumerate}
\end{lemma}

\begin{proof}
\hfill
\begin{enumerate}
\item $ \rb{\rb{f \cdot g} \cdot h} \circ \phi = f \cdot \rb{g \cdot h} $, where
$$ \phi\rb{s} =
\begin{cases}
\tfrac{s}{2} & s \in \sb{0, \tfrac{1}{2}} \\
s - \tfrac{1}{4} & s \in \sb{\tfrac{1}{2}, \tfrac{3}{4}} \\
2s - 1 & s \in \sb{\tfrac{3}{4}, 1}
\end{cases},
$$
so $ \rb{f \cdot g} \cdot h \cong f \cdot \rb{g \cdot h} $ by reparametrisation.
\item Again reparametrisation, by
$$ \psi\rb{s} =
\begin{cases}
2s & s \in \sb{0, \tfrac{1}{2}} \\
1 & s \in \sb{\tfrac{1}{2}, 1}
\end{cases},
\qquad \chi\rb{s} =
\begin{cases}
0 & s \in \sb{0, \tfrac{1}{2}} \\
2s - 1 & s \in \sb{\tfrac{1}{2}, 1}
\end{cases}.
$$
\item Define
$$ H\rb{s, t} =
\begin{cases}
f\rb{\max\cb{1 - 2s, t}} & s \in \sb{0, \tfrac{1}{2}} \\
f\rb{\max\cb{2s - 1, t}} & s \in \sb{\tfrac{1}{2}, 1}
\end{cases}.
$$
$ H $ is continuous, and
$$ H\rb{s, 0} = f^{-1} \cdot f, \qquad H\rb{s, 1} = c_{f\rb{1}}. $$
\end{enumerate}
\end{proof}

\begin{definition}
A \textbf{loop} with \textbf{basepoint} $ x_0 \in X $ is a path $ f : I \to X $ such that $ f\rb{0} = f\rb{1} = x_0 $.
\end{definition}

\begin{definition}
Denote by $ \pi_1\rb{X, x_0} $ the set of homotopy classes $ \sb{f} $ of loops $ f : I \to X $ with basepoint $ x_0 $.
\end{definition}

\begin{proposition}
$ \pi_1\rb{X, x_0} $ is a group with product $ \sb{f}\sb{g} = \sb{f \cdot g} $ and neutral element $ c_{x_0} : I \to X $, the constant path at $ x_0 $.
\end{proposition}

\begin{proof}
Follows directly from Lemma \ref{lem:1.2} and Lemma \ref{lem:1.3}.
\end{proof}

\begin{definition}
$ \pi_1\rb{X, x_0} $ is the \textbf{fundamental group} of $ X $ at $ x_0 $.
\end{definition}

\begin{example1}
Let $ X \subseteq \R^n $ be a convex set and $ x_0 \in X $. Then $ \pi_1\rb{X, x_0} = 0 $.
\end{example1}

\begin{proof}
$ X $ is convex gives that all loops are homotopic to each other.
\end{proof}

\begin{example1}
\hfill
\begin{itemize}
\item The fundamental group of a space $ X $ with the trivial topology is trivial, since $ X $ is simply connected, because all maps $ f : I \to X $ are continuous, so path connected and all paths are homotopic.
\item The fundamental group of a space $ X $ with the discrete topology is trivial, since $ f : I \to X $ continuous gives $ f $ constant.
\end{itemize}
\end{example1}

Assume $ x_0, x_1 \in X $ such that $ x_0 $ and $ x_1 $ are in the same path component of $ X $. Let $ h : I \to X $ be a path such that $ h\rb{0} = x_0 $ and $ h\rb{1} = x_1 $. Define
$$ \function[\beta_h]{\pi_1\rb{X, x_1}}{\pi_1\rb{X, x_0}}{\sb{f}}{\sb{h \cdot f \cdot h^{-1}}}. $$
This is well-defined by Lemma \ref{lem:1.2}.

\begin{proposition}
$ \beta_h : \pi_1\rb{X, x_1} \to \pi_1\rb{X, x_0} $ is an isomorphism.
\end{proposition}

\begin{proof}
It is a homomorphism.
$$ \beta_h\sb{f \cdot g} = \sb{h \cdot f \cdot g \cdot h^{-1}} = \sb{h \cdot f \cdot h^{-1}}\sb{h \cdot g \cdot h^{-1}} = \beta_h\sb{f} \cdot \beta_h\sb{g}, $$
and $ \beta_h\sb{c_{x_1}} = \sb{c_{x_1}} $. It is bijective with $ \rb{\beta_h}^{-1} = \beta_{h^{-1}} $.
\end{proof}

If $ X $ is path connected, we often write $ \pi_1\rb{X} $ instead of $ \pi_1\rb{X, x_0} $.

\begin{definition}
$ X $ is \textbf{simply connected} if it is path connected and $ \pi_1\rb{X} = 0 $.
\end{definition}

\begin{proposition}
$ X $ is simply connected if and only if there exists a unique homotopy class of paths between any two points of $ X $.
\end{proposition}

\begin{proof}
\hfill
\begin{itemize}
\item[$ \implies $] There exists a path between any two points. Let $ f, g $ be two paths from $ x_0 $ to $ x_1 $ for $ x_0, x_1 \in X $. $ f \cdot g^{-1} \cong g \cdot g^{-1} $ gives $ f \cong f \cdot g^{-1} \cdot g \cong g \cdot g^{-1} \cdot g \cong g $.
\item[$ \impliedby $] $ X $ is path connected. $ x_1 = x_0 $ gives that all loops at $ x_0 $ are homotopic to each other, so $ \pi_1\rb{X} = 0 $.
\end{itemize}
\end{proof}

\subsection{The fundamental group of the circle}

Goal is to show that $ \pi_1\rb{S^1} \cong \Z $.

\lecture{4}{Friday}{18/01/19}

\begin{definition}
A \textbf{covering space} of a space $ X $ is a space $ \widetilde{X} $ and a continuous map $ p : \widetilde{X} \to X $ such that for each $ x \in X $ there is an open $ x \in U \subseteq X $ such that
\begin{itemize}
\item $ p^{-1}\rb{U} = \bigcup_{j \in J} \widetilde{U_j} $, where $ \widetilde{U_j} \subseteq \widetilde{X} $ is open,
\item $ \widetilde{U_i} \cap \widetilde{U_j} = \emptyset $ if $ i \ne j $, and
\item $ p \mid_{\widetilde{U_j}} : \widetilde{U_j} \to U_i $ is a homeomorphism for all $ j \in J $.
\end{itemize}
Such a $ U $ is called \textbf{evenly covered}. The $ \widetilde{U_j} $ are called \textbf{sheets}.
\end{definition}

\begin{example1}
$$ \function[p]{\R}{S^1}{s}{\rb{\cos\rb{2\pi s, \sin\rb{2\pi s}}}} $$
\end{example1}

\begin{definition}
Let $ p : \widetilde{X} \to X $ be a covering space. A \textbf{lift} of a continuous map $ f : Y \to X $ is a continuous map $ \widetilde{f} : Y \to \widetilde{X} $ such that $ p \circ \widetilde{f} = f $, so
$$
\begin{tikzcd}
& \widetilde{X} \arrow{d}{p} \\
Y \arrow{ur}{\widetilde{f}} \arrow[swap]{r}{f} & X
\end{tikzcd}.
$$
\end{definition}

\begin{proposition}[Unique lifting property]
\label{prop:1.34}
Let $ p : \widetilde{X} \to X $ be a covering space and $ f : Y \to X $ be a continuous map. If there are two lifts $ \widetilde{f_1}, \widetilde{f_2} : Y \to \widetilde{X} $ of $ f $ such that $ \widetilde{f_1}\rb{y} = \widetilde{f_2}\rb{y} $ for some $ y \in Y $ and if $ Y $ is connected, then $ \widetilde{f_1} = \widetilde{f_2} $.
\end{proposition}

\begin{proof}
Let $ y \in Y $ and $ U \subseteq X $ be an evenly covered neighbourhood of $ f\rb{y} $. Then
$$ p^{-1}\rb{U} = \bigcup_j \widetilde{U_j}. $$
Let $ \widetilde{U_1} $ be the sheet such that $ \widetilde{f_1}\rb{y} \in \widetilde{U_1} $, and let $ \widetilde{U_2} $ be the sheet such that $ \widetilde{f_2}\rb{y} \in \widetilde{U_2} $. Let $ N \subseteq Y $ be open and $ y \in N $ such that $ \widetilde{f_1}\rb{N} \subseteq \widetilde{U_1} $ and $ \widetilde{f_2}\rb{N} \subseteq \widetilde{U_2} $. We have $ p \circ \widetilde{f_1} = p \circ \widetilde{f_2} $.
$$ \widetilde{f_1}\rb{y} = \widetilde{f_2}\rb{y} \qquad \iff \qquad \widetilde{U_1} = \widetilde{U_2} \qquad \iff \qquad \widetilde{f_1} \mid_N = \widetilde{f_2} \mid_N. $$
Let
$$ A = \cb{y \in Y \mid \widetilde{f_1}\rb{y} = \widetilde{f_2}\rb{y}}, $$
so $ A $ is open and $ Y \setminus A $ is open. Thus $ A \ne \emptyset $ gives $ A = Y $.
\end{proof}

\begin{proposition}[Homotopy lifting property]
\label{prop:1.30}
Let $ p : \widetilde{X} \to X $ be a covering space and $ F : Y \times I \to X $ be a continuous map such that there exists a lift $ \widetilde{f_0} : Y \times \cb{0} \to \widetilde{X} $ of $ F \mid_{Y \times \cb{0}} $. Then there is a unique lift $ \widetilde{F} : Y \times I \to \widetilde{X} $ of $ F $ such that $ \widetilde{F} \mid_{Y \times \cb{0}} = \widetilde{f_0} $.
\end{proposition}

\begin{proof}
Let $ y_0 \in Y $ and $ t \in I $. There are open $ y_0 \in N_t \subseteq Y $ and $ t \in \rb{a_t, b_t} \subseteq I $ such that $ F\rb{N_t \times \rb{a_t, b_t}} \subseteq U \subseteq X $, where $ U \subseteq X $ is open and evenly covered. Compactness of $ I $ gives that there exist
$$ 0 = t_0 < \dots < t_m = 1, $$
and there exists $ y_0 \in N \subseteq Y $ open such that $ F\rb{N \times \sb{t_i, t_{i + 1}}} \subseteq U_i \subseteq X $, where $ U_i \subseteq X $ is open and evenly covered. We inductively construct a lift $ \widetilde{F} \mid_{N \times I} $ of $ F \mid_{N \times I} $.
\begin{itemize}
\item $ \widetilde{F} \mid_{N \times \sb{0, 0}} = \widetilde{f_0} \mid_{N \times \sb{0, 0}} $ exists.
\item Assume the lift has been constructed on $ N \times \sb{0, t_i} $. Let $ \widetilde{U_i} \subseteq \widetilde{X} $ be such that $ p \mid_{\widetilde{U_i}} : \widetilde{U_i} \to U_i $ such that $ \widetilde{F}\rb{y_0, t_i} \subseteq \widetilde{U_i} $. After shrinking $ N $, may assume $ \widetilde{F}\rb{N \times \cb{t_i}} \subseteq \widetilde{U_i} $. Define $ \widetilde{F} $ on $ N \times \sb{t_i, t_{i + 1}} $ to be composition of $ F $ with the homeomorphism $ p^{-1} : U_i \to \widetilde{U_i} $.
\end{itemize}
After finitely many steps we obtain a lift $ \widetilde{F} : N \times I \to \widetilde{X} $, where $ y_0 \in N \subseteq Y $ is open, so for each $ y \in Y $ there is a neighbourhood $ N_y \subseteq Y $ such that $ F \mid_{N_y \times I} : N_y \times I \to X $ lifts. For all $ y \in Y $, $ \cb{y} \times I $ is connected and can be lifted, so Proposition \ref{prop:1.34} gives that the lift of $ N \times I $ is unique. Thus there is a unique lift $ \widetilde{F} : Y \times I \to \widetilde{X} $.
\end{proof}

\begin{example1}
Let $ X $ be a topological space and $ A $ be discrete. Then $ p : X \times A \to X $ is a covering space. This is the \textbf{trivial covering}. (Exercise: show the unique lifting property and the homotopy lifting property for the trivial covering)
\end{example1}

\begin{corollary}
Let $ f : I \to X $ be a path, $ f\rb{0} = x_0 $, and $ p : \widetilde{X} \to X $ be a covering space. For each $ \widetilde{x_0} \in p^{-1}\rb{x_0} $, there is a unique lift $ \widetilde{f} : I \to \widetilde{X} $ such that $ \widetilde{f}\rb{0} = \widetilde{x_0} $.
\end{corollary}

\begin{proof}
Proposition \ref{prop:1.30} for $ Y $ a point.
\end{proof}

\begin{theorem}
\label{thm:1.7}
Let $ x_0 = \rb{1, 0} \in S^1 $. $ \pi_1\rb{S^1, x_0} $ is the infinite cyclic group generated by the homotopy class of the loop
$$ \function[\omega]{I}{S^1}{s}{\rb{\cos\rb{2\pi s}, \sin\rb{2\pi s}}}. $$
\end{theorem}

\begin{remark1}
\hfill
\begin{itemize}
\item $ \sb{\omega}^n = \sb{\omega_n} $, where
$$ \omega_n\rb{s} = \rb{\cos\rb{2\pi ns}, \sin\rb{2\pi ns}}. $$
\item
$$ \function[p]{\R}{S^1}{s}{\rb{\cos\rb{2\pi s}, \sin\rb{2\pi s}}} $$
is a covering space.
\item $ \omega_n $ lifts to
$$ \function[\widetilde{\omega_n}]{I}{\R}{s}{ns}, $$
such that $ \widetilde{\omega_n}\rb{0} = 0 $ and $ \widetilde{\omega_n}\rb{1} = n $.
\end{itemize}
\end{remark1}

\begin{proof}[Proof of Theorem \ref{thm:1.7}]
\hfill
\begin{itemize}
\item If $ f : I \to S^1 $ be a loop at $ x_0 $, then the homotopy lifting property gives that there exists a lift $ \widetilde{f} : I \to \R $ such that $ \widetilde{f}\rb{0} = 0 $. Since $ p\rb{\widetilde{f}\rb{1}} = f\rb{1} = x_0 $, then $ \widetilde{f}\rb{1} = n $ for some $ n \in \Z $. $ \widetilde{\omega_n} : I \to \R $ is another path such that $ \widetilde{\omega_n}\rb{0} = 0 $ and $ \widetilde{\omega_n}\rb{1} = n $, so $ \widetilde{f} \cong \widetilde{\omega_n} $. Let $ F : I \times I \to \R $ be a homotopy equivalence between $ \widetilde{f} $ and $ \widetilde{\omega_n} $. Then $ p \circ F : I \times I \to S^1 $ gives a homotopy between $ p \circ \widetilde{f} = f $ and $ p \circ \widetilde{\omega_n} = \omega_n $.
\item Let $ m, n \in \Z $ and assume $ \omega_m \cong \omega_n $. Let $ F : I \times I \to S^1 $ be a homotopy.
$$ F\rb{0, t} = \omega_m\rb{t}, \qquad F\rb{1, t} = \omega_n\rb{t}, \qquad F\rb{s, 0} = F\rb{s, 1} = x_0, $$
for all $ s, t \in I $. The unique lifting property gives that $ \widetilde{\omega_n}, \widetilde{\omega_m} : I \to \R $ are unique lifts such that $ \widetilde{\omega_n}\rb{0} = 0 = \widetilde{\omega_m}\rb{0} $. The homotopy lifting property gives that $ F $ lifts uniquely to a homotopy $ \widetilde{F} : I \times I \to \R $ between $ \widetilde{\omega_n} $ and $ \widetilde{\omega_m} $, and $ \widetilde{F}\rb{s, 1} \in \Z $ for all $ s \in I $. Thus $ \widetilde{F}\rb{s, 1} = n = m $, so $ \omega_m \cong \omega_n $ if and only if $ n = m $.
\end{itemize}
\end{proof}

\lecture{5}{Tuesday}{22/01/19}

Lecture 5 is a problem class.

\pagebreak

\appendix

\section{Quotient topology}

Recall that if $ X $ is a set with equivalence relation $ \sim $, there is a quotient set $ X / \sim $. The quotient map
$$ \function[\pi]{X}{\dfrac{X}{\sim}}{x}{\sb{x}} $$
is characterised by the following universal property. For every map $ g : X \to Y $ such that
$$ a \sim b \qquad \implies \qquad g\rb{a} = g\rb{b}, $$
there exists a unique $ f : X / \sim \to Y $ such that $ g = f \cdot \pi $, so
$$
\begin{tikzcd}
X \arrow{dr}{g} \arrow[swap]{d}{\pi} & \\
\dfrac{X}{\sim} \arrow[swap]{r}{\exists !f} & Y
\end{tikzcd}.
$$
Let $ X $ be a topological space and $ \sim $ be an equivalence relation on $ X $. We define a topology on $ X / \sim $ by
$$ U \subseteq \dfrac{X}{\sim} \ \text{open} \qquad \iff \qquad \pi^{-1}\rb{U} \ \text{open}. $$

\begin{remark1}
\hfill
\begin{itemize}
\item This is the largest topology on $ X / \sim $ such that $ \pi $ is continuous. Exercise $ 1 $ states that if $ Z $ is a topological space and $ f : X / \sim \to Z $ is a map, then $ f $ is continuous if and only if $ f \circ \pi : X \to Z $ is continuous. This implies that the topological quotient $ \pi : X \to X / \sim $ is characterised by the following universal property. For any topological space $ Z $ and a continuous $ g : X \to Z $ such that
$$ a \sim b \qquad \implies \qquad g\rb{a} = g\rb{b}, $$
there exists a unique continuous map $ f : X / \sim \to Z $ such that $ g f \cdot \pi $, so
$$
\begin{tikzcd}
X \arrow[swap]{d}{\pi} \arrow{dr}{g} & \\
\dfrac{X}{\sim} \arrow[swap]{r}{\exists !f} & Z
\end{tikzcd}.
$$
\item The quotient map is in general not open. For example, if $ \pi : \sb{0, 1} \to S^1 $, then $ \sb{0, 1} \subset [0, 1) $ is open but $ \pi\rb{[0, 1)} \subseteq S^1 $ is not open.
\item If $ X $ is Hausdorff, in general $ X / \sim $ is not Hausdorff.
\item If $ \sim $ is the trivial relation, then $ \pi : X \to X / \sim $ is a homeomorphism. Exercise $ 3 $ states that if $ X, Y $ are topological spaces, $ X $ is compact, $ Y $ is Hausdorff, and $ \pi : X \to Y $ is surjective and continuous, then $ \pi $ is a quotient, that is there exists $ \sim $ on $ X $ and $ \pi : X \to Y \cong X / \sim $ is a quotient map.
\item In particular, if $ \pi : X \to Y $ is bijective, then $ \pi $ is a homeomorphism. Exercise $ 4, 5, 6 $ states that if $ f $ is continuous and surjective, $ f\rb{\partial D^n} $ is a point, and $ f $ is a bijection on $ D^n \setminus \partial D^n $, then
$$
\begin{tikzcd}
D^n \arrow{dr}{f} \arrow[swap]{d}{\pi} & \\
\dfrac{D^n}{\partial D^n} \arrow[swap]{r}{\sim} & S^1
\end{tikzcd}.
$$
\end{itemize}
\end{remark1}

\end{document}