\def\module{M3P21 Geometry II: Algebraic Topology}
\def\lecturer{Dr Christian Urech}
\def\term{Spring 2019}

\def\thm{section}

\documentclass{article}

% Packages

\usepackage{amssymb}
\usepackage{amsthm}
\usepackage[UKenglish]{babel}
\usepackage{commath}
\usepackage{enumitem}
\usepackage{etoolbox}
\usepackage{fancyhdr}
\usepackage[margin=1in]{geometry}
\usepackage{graphicx}
\usepackage[hidelinks]{hyperref}
\usepackage[utf8]{inputenc}
\usepackage{listings}
\usepackage{mathtools}
\usepackage{stmaryrd}
\usepackage{tikz-cd}
\usepackage{csquotes}

% Formatting

\addto\captionsUKenglish{\renewcommand{\abstractname}{Syllabus}}
\delimitershortfall5pt
\ifx\thm\undefined\newtheorem{n}{}\else\newtheorem{n}{}[\thm]\fi
\newcommand\newoperator[1]{\ifcsdef{#1}{\cslet{#1}{\relax}}{}\csdef{#1}{\operatorname{#1}}}
\setlength{\parindent}{0cm}

% Environments

\theoremstyle{plain}
\newtheorem{algorithm}[n]{Algorithm}
\newtheorem*{algorithm*}{Algorithm}
\newtheorem{algorithm**}{Algorithm}
\newtheorem{conjecture}[n]{Conjecture}
\newtheorem*{conjecture*}{Conjecture}
\newtheorem{conjecture**}{Conjecture}
\newtheorem{corollary}[n]{Corollary}
\newtheorem*{corollary*}{Corollary}
\newtheorem{corollary**}{Corollary}
\newtheorem{lemma}[n]{Lemma}
\newtheorem*{lemma*}{Lemma}
\newtheorem{lemma**}{Lemma}
\newtheorem{proposition}[n]{Proposition}
\newtheorem*{proposition*}{Proposition}
\newtheorem{proposition**}{Proposition}
\newtheorem{theorem}[n]{Theorem}
\newtheorem*{theorem*}{Theorem}
\newtheorem{theorem**}{Theorem}

\theoremstyle{definition}
\newtheorem{aim}[n]{Aim}
\newtheorem*{aim*}{Aim}
\newtheorem{aim**}{Aim}
\newtheorem{axiom}[n]{Axiom}
\newtheorem*{axiom*}{Axiom}
\newtheorem{axiom**}{Axiom}
\newtheorem{condition}[n]{Condition}
\newtheorem*{condition*}{Condition}
\newtheorem{condition**}{Condition}
\newtheorem{definition}[n]{Definition}
\newtheorem*{definition*}{Definition}
\newtheorem{definition**}{Definition}
\newtheorem{example}[n]{Example}
\newtheorem*{example*}{Example}
\newtheorem{example**}{Example}
\newtheorem{exercise}[n]{Exercise}
\newtheorem*{exercise*}{Exercise}
\newtheorem{exercise**}{Exercise}
\newtheorem{fact}[n]{Fact}
\newtheorem*{fact*}{Fact}
\newtheorem{fact**}{Fact}
\newtheorem{goal}[n]{Goal}
\newtheorem*{goal*}{Goal}
\newtheorem{goal**}{Goal}
\newtheorem{law}[n]{Law}
\newtheorem*{law*}{Law}
\newtheorem{law**}{Law}
\newtheorem{plan}[n]{Plan}
\newtheorem*{plan*}{Plan}
\newtheorem{plan**}{Plan}
\newtheorem{problem}[n]{Problem}
\newtheorem*{problem*}{Problem}
\newtheorem{problem**}{Problem}
\newtheorem{question}[n]{Question}
\newtheorem*{question*}{Question}
\newtheorem{question**}{Question}
\newtheorem{warning}[n]{Warning}
\newtheorem*{warning*}{Warning}
\newtheorem{warning**}{Warning}
\newtheorem{acknowledgements}[n]{Acknowledgements}
\newtheorem*{acknowledgements*}{Acknowledgements}
\newtheorem{acknowledgements**}{Acknowledgements}
\newtheorem{annotations}[n]{Annotations}
\newtheorem*{annotations*}{Annotations}
\newtheorem{annotations**}{Annotations}
\newtheorem{assumption}[n]{Assumption}
\newtheorem*{assumption*}{Assumption}
\newtheorem{assumption**}{Assumption}
\newtheorem{conclusion}[n]{Conclusion}
\newtheorem*{conclusion*}{Conclusion}
\newtheorem{conclusion**}{Conclusion}
\newtheorem{claim}[n]{Claim}
\newtheorem*{claim*}{Claim}
\newtheorem{claim**}{Claim}
\newtheorem{notation}[n]{Notation}
\newtheorem*{notation*}{Notation}
\newtheorem{notation**}{Notation}
\newtheorem{note}[n]{Note}
\newtheorem*{note*}{Note}
\newtheorem{note**}{Note}
\newtheorem{remark}[n]{Remark}
\newtheorem*{remark*}{Remark}
\newtheorem{remark**}{Remark}

% Lectures

\newcommand{\lecture}[3]{ % Lecture
  \marginpar{
    Lecture #1 \\
    #2 \\
    #3
  }
}

% Blackboard

\renewcommand{\AA}{\mathbb{A}} % Blackboard A
\newcommand{\BB}{\mathbb{B}}   % Blackboard B
\newcommand{\CC}{\mathbb{C}}   % Blackboard C
\newcommand{\DD}{\mathbb{D}}   % Blackboard D
\newcommand{\EE}{\mathbb{E}}   % Blackboard E
\newcommand{\FF}{\mathbb{F}}   % Blackboard F
\newcommand{\GG}{\mathbb{G}}   % Blackboard G
\newcommand{\HH}{\mathbb{H}}   % Blackboard H
\newcommand{\II}{\mathbb{I}}   % Blackboard I
\newcommand{\JJ}{\mathbb{J}}   % Blackboard J
\newcommand{\KK}{\mathbb{K}}   % Blackboard K
\newcommand{\LL}{\mathbb{L}}   % Blackboard L
\newcommand{\MM}{\mathbb{M}}   % Blackboard M
\newcommand{\NN}{\mathbb{N}}   % Blackboard N
\newcommand{\OO}{\mathbb{O}}   % Blackboard O
\newcommand{\PP}{\mathbb{P}}   % Blackboard P
\newcommand{\QQ}{\mathbb{Q}}   % Blackboard Q
\newcommand{\RR}{\mathbb{R}}   % Blackboard R
\renewcommand{\SS}{\mathbb{S}} % Blackboard S
\newcommand{\TT}{\mathbb{T}}   % Blackboard T
\newcommand{\UU}{\mathbb{U}}   % Blackboard U
\newcommand{\VV}{\mathbb{V}}   % Blackboard V
\newcommand{\WW}{\mathbb{W}}   % Blackboard W
\newcommand{\XX}{\mathbb{X}}   % Blackboard X
\newcommand{\YY}{\mathbb{Y}}   % Blackboard Y
\newcommand{\ZZ}{\mathbb{Z}}   % Blackboard Z

% Brackets

\renewcommand{\eval}[1]{\left. #1 \right|}          % Evaluation
\newcommand{\br}{\del}                              % Brackets
\newcommand{\abr}[1]{\left\langle #1 \right\rangle} % Angle brackets
\newcommand{\fbr}[1]{\left\lfloor #1 \right\rfloor} % Floor brackets
\newcommand{\lbr}[1]{\left\lfloor #1 \right\rfloor} % Ceiling brackets
\newcommand{\st}{\ \middle| \ }                     % Such that

% Calligraphic

\newcommand{\AAA}{\mathcal{A}} % Calligraphic A
\newcommand{\BBB}{\mathcal{B}} % Calligraphic B
\newcommand{\CCC}{\mathcal{C}} % Calligraphic C
\newcommand{\DDD}{\mathcal{D}} % Calligraphic D
\newcommand{\EEE}{\mathcal{E}} % Calligraphic E
\newcommand{\FFF}{\mathcal{F}} % Calligraphic F
\newcommand{\GGG}{\mathcal{G}} % Calligraphic G
\newcommand{\HHH}{\mathcal{H}} % Calligraphic H
\newcommand{\III}{\mathcal{I}} % Calligraphic I
\newcommand{\JJJ}{\mathcal{J}} % Calligraphic J
\newcommand{\KKK}{\mathcal{K}} % Calligraphic K
\newcommand{\LLL}{\mathcal{L}} % Calligraphic L
\newcommand{\MMM}{\mathcal{M}} % Calligraphic M
\newcommand{\NNN}{\mathcal{N}} % Calligraphic N
\newcommand{\OOO}{\mathcal{O}} % Calligraphic O
\newcommand{\PPP}{\mathcal{P}} % Calligraphic P
\newcommand{\QQQ}{\mathcal{Q}} % Calligraphic Q
\newcommand{\RRR}{\mathcal{R}} % Calligraphic R
\newcommand{\SSS}{\mathcal{S}} % Calligraphic S
\newcommand{\TTT}{\mathcal{T}} % Calligraphic T
\newcommand{\UUU}{\mathcal{U}} % Calligraphic U
\newcommand{\VVV}{\mathcal{V}} % Calligraphic V
\newcommand{\WWW}{\mathcal{W}} % Calligraphic W
\newcommand{\XXX}{\mathcal{X}} % Calligraphic X
\newcommand{\YYY}{\mathcal{Y}} % Calligraphic Y
\newcommand{\ZZZ}{\mathcal{Z}} % Calligraphic Z

% Fraktur

\newcommand{\aaa}{\mathfrak{a}}   % Fraktur a
\newcommand{\bbb}{\mathfrak{b}}   % Fraktur b
\newcommand{\ccc}{\mathfrak{c}}   % Fraktur c
\newcommand{\ddd}{\mathfrak{d}}   % Fraktur d
\newcommand{\eee}{\mathfrak{e}}   % Fraktur e
\newcommand{\fff}{\mathfrak{f}}   % Fraktur f
\renewcommand{\ggg}{\mathfrak{g}} % Fraktur g
\newcommand{\hhh}{\mathfrak{h}}   % Fraktur h
\newcommand{\iii}{\mathfrak{i}}   % Fraktur i
\newcommand{\jjj}{\mathfrak{j}}   % Fraktur j
\newcommand{\kkk}{\mathfrak{k}}   % Fraktur k
\renewcommand{\lll}{\mathfrak{l}} % Fraktur l
\newcommand{\mmm}{\mathfrak{m}}   % Fraktur m
\newcommand{\nnn}{\mathfrak{n}}   % Fraktur n
\newcommand{\ooo}{\mathfrak{o}}   % Fraktur o
\newcommand{\ppp}{\mathfrak{p}}   % Fraktur p
\newcommand{\qqq}{\mathfrak{q}}   % Fraktur q
\newcommand{\rrr}{\mathfrak{r}}   % Fraktur r
\newcommand{\sss}{\mathfrak{s}}   % Fraktur s
\newcommand{\ttt}{\mathfrak{t}}   % Fraktur t
\newcommand{\uuu}{\mathfrak{u}}   % Fraktur u
\newcommand{\vvv}{\mathfrak{v}}   % Fraktur v
\newcommand{\www}{\mathfrak{w}}   % Fraktur w
\newcommand{\xxx}{\mathfrak{x}}   % Fraktur x
\newcommand{\yyy}{\mathfrak{y}}   % Fraktur y
\newcommand{\zzz}{\mathfrak{z}}   % Fraktur z

% Geometry

\newcommand{\CP}{\mathbb{CP}}                                              % Complex projective space
\newcommand{\iintd}[4]{\iint_{#1} \, #2 \, \dif #3 \, \dif #4}             % Double integral
\newcommand{\RP}{\mathbb{RP}}                                              % Real projective space
\newcommand{\intd}[4]{\int_{#1}^{#2} \, #3 \, \dif #4}                     % Single integral
\newcommand{\iiintd}[5]{\iint_{#1} \, #2 \, \dif #3 \, \dif #4 \, \dif #5} % Triple integral

% Logic

\newcommand{\iffb}[2]{\br{#1 \leftrightarrow #2}} % Biconditional
\newcommand{\andb}[2]{\br{#1 \land #2}}           % Conjunction
\newcommand{\orb}[2]{\br{#1 \lor #2}}             % Disjunction
\newcommand{\nib}[2]{\br{#1 \notin #2}}           % Element of
\newcommand{\eqb}[2]{\br{#1 = #2}}                % Equal to
\newcommand{\teb}[1]{\br{\exists #1}}             % Existential quantifier
\newcommand{\impb}[2]{\br{#1 \rightarrow #2}}     % Implication
\newcommand{\ltb}[2]{\br{#1 < #2}}                % Less than
\newcommand{\leb}[2]{\br{#1 \le #2}}              % Less than or equal to
\newcommand{\notb}[1]{\br{\neg #1}}               % Negation
\newcommand{\inb}[2]{\br{#1 \in #2}}              % Not element of
\newcommand{\neb}[2]{\br{#1 \ne #2}}              % Not equal to
\newcommand{\subb}[2]{\br{#1 \subseteq #2}}       % Subset
\newcommand{\fab}[1]{\br{\forall #1}}             % Universal quantifier

% Maps

\newcommand{\bijection}[7][]{    % Bijection
  \ifx &#1&
    \begin{array}{rcl}
      #2 & \longleftrightarrow & #3 \\
      #4 & \longmapsto         & #5 \\
      #6 & \longmapsfrom       & #7
    \end{array}
  \else
    \begin{array}{ccrcl}
      #1 & : & #2 & \longrightarrow & #3 \\
         &   & #4 & \longmapsto     & #5 \\
         &   & #6 & \longmapsfrom   & #7
    \end{array}
  \fi
}
\newcommand{\birational}[7][]{   % Birational map
  \ifx &#1&
    \begin{array}{rcl}
      #2 & \dashrightarrow & #3 \\
      #4 & \longmapsto     & #5 \\
      #6 & \longmapsfrom   & #7
    \end{array}
  \else
    \begin{array}{ccrcl}
      #1 & : & #2 & \dashrightarrow & #3 \\
         &   & #4 & \longmapsto     & #5 \\
         &   & #6 & \longmapsfrom   & #7
    \end{array}
  \fi
}
\newcommand{\correspondence}[2]{ % Correspondence
  \cbr{
    \begin{array}{c}
      #1
    \end{array}
  }
  \qquad
  \leftrightsquigarrow
  \qquad
  \cbr{
    \begin{array}{c}
      #2
    \end{array}
  }
}
\newcommand{\function}[5][]{     % Function
  \ifx &#1&
    \begin{array}{rcl}
      #2 & \longrightarrow & #3 \\
      #4 & \longmapsto     & #5
    \end{array}
  \else
    \begin{array}{ccrcl}
      #1 & : & #2 & \longrightarrow & #3 \\
         &   & #4 & \longmapsto     & #5
    \end{array}
  \fi
}
\newcommand{\functions}[7][]{    % Functions
  \ifx &#1&
    \begin{array}{rcl}
      #2 & \longrightarrow & #3 \\
      #4 & \longmapsto     & #5 \\
      #6 & \longmapsto     & #7
    \end{array}
  \else
    \begin{array}{ccrcl}
      #1 & : & #2 & \longrightarrow & #3 \\
         &   & #4 & \longmapsto     & #5 \\
         &   & #6 & \longmapsto     & #7
    \end{array}
  \fi
}
\newcommand{\rational}[5][]{     % Rational map
  \ifx &#1&
    \begin{array}{rcl}
      #2 & \dashrightarrow & #3 \\
      #4 & \longmapsto     & #5
    \end{array}
  \else
    \begin{array}{ccrcl}
      #1 & : & #2 & \dashrightarrow & #3 \\
         &   & #4 & \longmapsto     & #5
    \end{array}
  \fi
}

% Matrices

\newcommand{\onebytwo}[2]{      % One by two matrix
  \begin{pmatrix}
    #1 & #2
  \end{pmatrix}
}
\newcommand{\onebythree}[3]{    % One by three matrix
  \begin{pmatrix}
    #1 & #2 & #3
  \end{pmatrix}
}
\newcommand{\twobyone}[2]{      % Two by one matrix
  \begin{pmatrix}
    #1 \\
    #2
  \end{pmatrix}
}
\newcommand{\twobytwo}[4]{      % Two by two matrix
  \begin{pmatrix}
    #1 & #2 \\
    #3 & #4
  \end{pmatrix}
}
\newcommand{\threebyone}[3]{    % Three by one matrix
  \begin{pmatrix}
    #1 \\
    #2 \\
    #3
  \end{pmatrix}
}
\newcommand{\threebythree}[9]{  % Three by three matrix
  \begin{pmatrix}
    #1 & #2 & #3 \\
    #4 & #5 & #6 \\
    #7 & #8 & #9
  \end{pmatrix}
}
\newcommand{\twobytwosmall}[4]{ % Two by two small matrix
  \begin{psmallmatrix}
    #1 & #2 \\
    #3 & #4
  \end{psmallmatrix}
}

% Number theory

\renewcommand{\symbol}[2]{\br{\tfrac{#1}{#2}}} % Power residue symbol
\newcommand{\unit}[1]{\br{\ZZ / #1\ZZ}^\times} % Unit group

% Operators

\newoperator{ab}    % Abelian
\newoperator{AG}    % Affine geometry
\newoperator{alg}   % Algebraic
\newoperator{Ann}   % Annihilator
\newoperator{area}  % Area
\newoperator{Aut}   % Automorphism
\newoperator{card}  % Cardinality
\newoperator{ch}    % Characteristic
\newoperator{Cl}    % Class
\newoperator{Coker} % Cokernel
\newoperator{col}   % Column
\newoperator{Corr}  % Correspondence
\newoperator{diam}  % Diameter
\newoperator{Disc}  % Discriminant
\newoperator{dom}   % Domain
\newoperator{Eig}   % Eigenvalue
\newoperator{Em}    % Embedding
\newoperator{End}   % Endomorphism
\newoperator{fin}   % Finite
\newoperator{Fix}   % Fixed
\newoperator{Frac}  % Fraction
\newoperator{Frob}  % Frobenius
\newoperator{Fun}   % Function
\newoperator{Gal}   % Galois
\newoperator{GL}    % General linear
\newoperator{Ham}   % Hamming
\newoperator{Homeo} % Homeomorphism
\newoperator{Hom}   % Homomorphism
\newoperator{id}    % Identity
\newoperator{Im}    % Image
\newoperator{Ind}   % Index
\newoperator{Ker}   % Kernel
\newoperator{lcm}   % Least common multiple
\newoperator{Mat}   % Matrix
\newoperator{mult}  % Multiplicity
\newoperator{new}   % New
\newoperator{Nm}    % Norm
\newoperator{old}   % Old
\newoperator{op}    % Opposite
\newoperator{ord}   % Order
\newoperator{Pay}   % Payley
\newoperator{PG}    % Projective geometry
\newoperator{PGL}   % Projective general linear
\newoperator{PSL}   % Projective special linear
\newoperator{rad}   % Radical
\newoperator{ran}   % Range
\newoperator{Res}   % Residue
\newoperator{rk}    % Rank
\newoperator{Re}    % Real
\newoperator{row}   % Row
\newoperator{sgn}   % Sign
\newoperator{Sing}  % Singular
\newoperator{SK}    % Skeleton
\newoperator{sp}    % Span
\newoperator{SL}    % Special linear
\newoperator{SO}    % Special orthogonal
\newoperator{Spec}  % Spectrum
\newoperator{Stab}  % Stabiliser
\newoperator{star}  % Star
\newoperator{srg}   % Strongly regular graph
\newoperator{supp}  % Support
\newoperator{Sym}   % Symmetric
\newoperator{tors}  % Torsion
\newoperator{Tr}    % Trace
\newoperator{vol}   % Volume
\newoperator{wt}    % Weight

% Roman

\newcommand{\A}{\mathrm{A}}   % Roman A
\newcommand{\B}{\mathrm{B}}   % Roman B
\newcommand{\C}{\mathrm{C}}   % Roman C
\newcommand{\D}{\mathrm{D}}   % Roman D
\newcommand{\E}{\mathrm{E}}   % Roman E
\newcommand{\F}{\mathrm{F}}   % Roman F
\newcommand{\G}{\mathrm{G}}   % Roman G
\renewcommand{\H}{\mathrm{H}} % Roman H
\newcommand{\I}{\mathrm{I}}   % Roman I
\newcommand{\J}{\mathrm{J}}   % Roman J
\newcommand{\K}{\mathrm{K}}   % Roman K
\renewcommand{\L}{\mathrm{L}} % Roman L
\newcommand{\M}{\mathrm{M}}   % Roman M
\newcommand{\N}{\mathrm{N}}   % Roman N
\renewcommand{\O}{\mathrm{O}} % Roman O
\renewcommand{\P}{\mathrm{P}} % Roman P
\newcommand{\Q}{\mathrm{Q}}   % Roman Q
\newcommand{\R}{\mathrm{R}}   % Roman R
\renewcommand{\S}{\mathrm{S}} % Roman S
\newcommand{\T}{\mathrm{T}}   % Roman T
\newcommand{\U}{\mathrm{U}}   % Roman U
\newcommand{\V}{\mathrm{V}}   % Roman V
\newcommand{\W}{\mathrm{W}}   % Roman W
\newcommand{\X}{\mathrm{X}}   % Roman X
\newcommand{\Y}{\mathrm{Y}}   % Roman Y
\newcommand{\Z}{\mathrm{Z}}   % Roman Z

\renewcommand{\a}{\mathrm{a}} % Roman a
\renewcommand{\b}{\mathrm{b}} % Roman b
\renewcommand{\c}{\mathrm{c}} % Roman c
\renewcommand{\d}{\mathrm{d}} % Roman d
\newcommand{\e}{\mathrm{e}}   % Roman e
\newcommand{\f}{\mathrm{f}}   % Roman f
\newcommand{\g}{\mathrm{g}}   % Roman g
\newcommand{\h}{\mathrm{h}}   % Roman h
\renewcommand{\i}{\mathrm{i}} % Roman i
\renewcommand{\j}{\mathrm{j}} % Roman j
\renewcommand{\k}{\mathrm{k}} % Roman k
\renewcommand{\l}{\mathrm{l}} % Roman l
\newcommand{\m}{\mathrm{m}}   % Roman m
\renewcommand{\n}{\mathrm{n}} % Roman n
\renewcommand{\o}{\mathrm{o}} % Roman o
\newcommand{\p}{\mathrm{p}}   % Roman p
\newcommand{\q}{\mathrm{q}}   % Roman q
\renewcommand{\r}{\mathrm{r}} % Roman r
\newcommand{\s}{\mathrm{s}}   % Roman s
\renewcommand{\t}{\mathrm{t}} % Roman t
\renewcommand{\u}{\mathrm{u}} % Roman u
\renewcommand{\v}{\mathrm{v}} % Roman v
\newcommand{\w}{\mathrm{w}}   % Roman w
\newcommand{\x}{\mathrm{x}}   % Roman x
\newcommand{\y}{\mathrm{y}}   % Roman y
\newcommand{\z}{\mathrm{z}}   % Roman z

% Tikz

\tikzset{
  arrow symbol/.style={"#1" description, allow upside down, auto=false, draw=none, sloped},
  subset/.style={arrow symbol={\subset}},
  cong/.style={arrow symbol={\cong}}
}

% Fancy header

\pagestyle{fancy}
\lhead{\module}
\rhead{\nouppercase{\leftmark}}

% Make title

\title{\module}
\author{Lectured by \lecturer \\ Typed by David Kurniadi Angdinata}
\date{\term}

\begin{document}

% Title page
\maketitle
\cover
\vfill
\begin{abstract}
\noindent\syllabus
\end{abstract}

\pagebreak

% Contents page
\tableofcontents

\pagebreak

% Document page
\setcounter{section}{-1}

\section{Introduction}

\lecture{1}{Friday}{11/01/19}

\subsection{Introduction}

Combines topological spaces with algebraic objects, which are groups.
\begin{itemize}
\item How to show that a torus is not homeomorphic to a sphere?
\item How to show that $ \R^n \ncong \R^m $ if $ n \ne m $?
\end{itemize}

Content is fundamental groups and homology. We will follow chapter one and two from
\begin{itemize}
\item A Hatcher, Algebraic topology, 2002
\end{itemize}

The following are prerequisites.
\begin{itemize}
\item Point set topology. Topological spaces, continuous maps, product and quotient topologies, Hausdorff spaces, etc.
\item Basic group theory. Normal subgroups and quotients, isomorphism theorems, free groups, presentation of groups, etc.
\end{itemize}

\subsection{Some underlying geometric notions}

\subsubsection{Homotopy}

Let $ X, Y $ be topological spaces and $ I = \sb{0, 1} $.

\begin{definition*}
A \textbf{homotopy} is a continuous map $ F : X \times I \to Y $. For every $ t \in I $ we obtain a continuous map
$$ \function[f_t]{X}{Y}{x}{f_t\rb{x} = F\rb{x, t}}. $$
\end{definition*}

\begin{definition*}
Two continuous maps $ f_0, f_1 : X \to Y $ are \textbf{homotopic} if there exists a homotopy $ F : X \times I \to Y $ such that
$$ f_0\rb{x} = F\rb{x, 0}, \qquad f_1\rb{x} = F\rb{x, 1}, $$
for all $ x \in X $. We write $ f_0 \cong f_1 $. (Exercise: this is an equivalence relation)
\end{definition*}

\begin{definition*}
Let $ A \subseteq X $ be a subspace. A \textbf{retraction} of $ X $ onto $ A $ is a continuous map $ r : X \to A $ such that
\begin{itemize}
\item $ r\rb{X} = A $, and
\item $ r\mid_A = id_A $.
\end{itemize}
\end{definition*}

\begin{example*}
If $ X \ne \emptyset $, $ p \in X $, then $ X $ retracts to $ p $ by the constant map $ X \to \cb{p} $.
\end{example*}

\begin{definition*}
A \textbf{deformation retraction} of $ X $ onto $ A \subseteq X $ is a retraction that is homotopic to the identity. That is, there is a continuous map
$$ \function[F]{X \times I}{A}{\rb{x, t}}{f_t\rb{x}}, $$
such that $ f_0 = id_X $ and $ f_1 : X \to A $ is the deformation retraction.
\end{definition*}

\begin{example*}
The closed $ n $-dimensional \textbf{$ n $-disc}
$$ D^n = \cb{x \in \R^n \mid \abs{x} \le 1} $$
deformation retracts to $ \rb{0, \dots, 0} \in \R^n $. Let $ f_t\rb{x} = t \cdot x $. $ t = 1 $ gives $ f_1 = id_{D^n} $ and $ t = 0 $ gives $ f_0 : D^n \to \rb{0, \dots, 0} $.
\end{example*}

\begin{example*}
Let $ S^n $ be the \textbf{$ n $-sphere},
$$ \partial D^{n + 1} = S^n = \cb{x \in \R^n \mid \abs{x} = 1}. $$
The cylinder $ S^n \times I $ deformation retracts to $ S^n \times \cb{0} $, by defining $ f_t\rb{x, r} = \rb{x, t \cdot r} $.
\end{example*}

An observation is if $ X $ is a topological space, and $ f : X \to \cb{p} $ for $ p \in X $ is a deformation retraction of $ X $ to $ p $, then $ X $ is path-connected. Indeed, if $ F : X \times I \to X $ is a homotopy from $ id_X $ to $ f $ and $ x \in X $ is a point, then this gives a path
$$ \function{I}{X}{t}{F\rb{x, t}} $$
that connects $ x $ to $ p $. This implies that not all retractions are deformation retractions.

\begin{example*}
A retraction that is not a deformation retraction. Take a space that is not path-connected and retract it to a point. Let $ X = \cb{0, 1} $ with discrete topology. $ x \mapsto 0 $ is a retraction, but not a deformation retraction because $ X $ is not path-connected.
\end{example*}

\begin{definition*}
A continuous map $ f : X \to Y $ is a \textbf{homotopy equivalence} if there is a continuous map $ g : Y \to X $ such that $ fg \cong id_Y $ and $ gf \cong id_X $. If there exists a homotopy equivalence between $ X $ and $ Y $, $ X $ and $ Y $ are \textbf{homotopy equivalent} or they have the same \textbf{homotopy type}.
\end{definition*}

\begin{lemma}
A deformation retraction $ f : X \to A $ is a homotopy equivalence.
\end{lemma}

\begin{proof}
Let $ i : A \hookrightarrow X $ be the inclusion map. Then $ fi = id_A $ and $ if = f \cong id_X $ by definition.
\end{proof}

\begin{example*}
The disc with two holes is equivalent to $ \text{O} \cdot \text{O} $.
\end{example*}

\begin{example*}
$ \R^n $ deformation retracts to a point, by $ f_t\rb{x} = t \cdot x $.
\end{example*}

\begin{definition*}
\hfill
\begin{itemize}
\item $ X $ is \textbf{contractible} if it is homotopy equivalent to a point.
\item A continuous map is \textbf{nullhomotopic} if it is homotopy equivalent to a constant map.
\end{itemize}
\end{definition*}

\subsubsection{Cell complexes}

\begin{example*}
The torus $ S^1 \times S^1 $ is the union of a point, two open intervals, and the open disc $ Int\rb{D^2} $.
\end{example*}

These are called \textbf{cells}. Can think of discs $ D^n $ glued together.

\lecture{2}{Tuesday}{15/01/19}

\begin{definition*}
A \textbf{CW-complex}, or \textbf{cell complex}, is a topological space $ X $ such that there exists a decomposition
$$ X = \bigcup_{n \in \N} X^n, $$
where the $ X^n $ are constructed inductively in the following way.
\begin{itemize}
\item $ X^n $ is a discrete set.
\item For each $ n \ge 0 $ there is an collection of closed $ n $-discs $ \cb{D_\alpha^n} $ together with continuous maps $ \phi_\alpha : \partial D_\alpha^n \to X^{n - 1} $, such that
$$ X^n = \dfrac{X^{n - 1} \sqcup \bigsqcup_\alpha D_\alpha^n}{\sim}, $$
where $ x \sim \phi_\alpha\rb{x} $ for all $ x \in \partial D_\alpha^n $ for all $ \alpha $.
\item A subset $ U \subseteq X $ is open if and only if $ U \cap X^n $ is open for all $ n $.
\end{itemize}
\end{definition*}

\begin{remark*}
\hfill
\begin{itemize}
\item As a set,
$$ X^n = X^{n - 1} \sqcup \bigsqcup_\alpha e_\alpha^n, $$
where each $ e_\alpha^n $ is homeomorphic to an open $ n $-disc. These $ e_\alpha^n $ are called the \textbf{$ n $-cells} of $ X $.
\item If $ X = X^m $ for some $ m $, then $ X $ is called \textbf{finite dimensional}. The minimal $ m $ such that $ X = X^m $ is the \textbf{dimension} of $ X $.
\end{itemize}
\end{remark*}

\begin{example*}
\hfill
\begin{itemize}
\item $ \sb{0, 1} $ is a CW-complex.
\item $ \R $ is a CW-complex.
\item $ S^1 $ is a CW-complex.
\item A graph is a CW-complex.
\item $ S^n = D^n / \partial D^n $ is a CW-complex. See worksheet $ 1 $.
\end{itemize}
Can also decompose CW-complexes.
\begin{itemize}
\item The sphere $ S^2 $ is one $ 0 $-cell, one $ 1 $-cell, and two $ 2 $-cells.
\item The torus $ S^1 \times S^1 $ is one $ 0 $-cell, two $ 1 $-cells, and one $ 2 $-cell.
\item The M\"obius strip is two $ 0 $-cells, three $ 1 $-cells, and one $ 2 $-cell.
\item The Klein bottle is one $ 0 $-cell, two $ 1 $-cells, and one $ 2 $-cell.
\end{itemize}
\end{example*}

\begin{definition*}
If $ X $ is a CW-complex with finitely many cells the \textbf{Euler characteristic} $ \chi\rb{X} $ of $ X $ is the number of even cells minus the number of odd cells.
\end{definition*}

\begin{fact*}
$ \chi\rb{X} $ does not depend of the choice of cells decomposition.
\end{fact*}

\begin{example*}
\hfill
\begin{itemize}
\item $ \chi\rb{S^n} = 0 $ if $ n $ is odd and $ \chi\rb{S^n} = 2 $ if $ n $ is even.
\item $ \chi\rb{S^1 \times S^1} = 0 $.
\end{itemize}
\end{example*}

This is the generalisation of the following observation by Leonhard Euler. Let $ P $ be a convex polyhedron, where
\begin{itemize}
\item $ V $ is the number of vertices of $ P $,
\item $ E $ is the number of edges of $ P $, and
\item $ F $ is the number of faces of $ P $.
\end{itemize}
Then $ V - E + F = 2 $.

\begin{example*}
A topological space that is not a CW-complex. $ X = \cb{0, 1} $ with trivial topology does not contain any closed points.
\end{example*}

\begin{fact*}
CW-complexes are always Hausdorff.
\end{fact*}

\pagebreak

\section{The fundamental group}

\subsection{Basic constructions}

\subsubsection{Paths and homotopy}

Let $ X $ be a topological space. A \textbf{path} is a continuous map $ f : I \to X $, where $ I = \sb{0, 1} $.

\begin{definition*}
Two paths $ f_0, f_1 $ are \textbf{homotopic} if there exists a homotopy between $ f_0 $ and $ f_1 $ preserving the endpoints, that is a continuous map
$$ \function[F]{I \times I}{X}{\rb{s, t}}{f_t\rb{s}}, $$
such that
$$ f_t\rb{0} = f_0\rb{0}, \qquad f_t\rb{1} = f_0\rb{1}, $$
for all $ t \in I $, and
$$ F\rb{s, 0} = f_0\rb{s}, \qquad F\rb{s, 1} = f_1\rb{s}, $$
for all $ s \in I $.
\end{definition*}

\begin{example*}
Let $ X \subseteq \R^n $ be a convex set. Then all the paths in $ X $ are homotopic if they have the same endpoints.
\end{example*}

\begin{proof}
Let $ f_0, f_1 : I \to X $ be two paths such that $ f_0\rb{0} = f_1\rb{0} $ and $ f_0\rb{1} = f_1\rb{1} $. Define
$$ f_t\rb{s} = \rb{1 - t}f_0\rb{s} + tf_1\rb{s}. $$
\end{proof}

\begin{lemma}
Being homotopic is an equivalence relation on the set of paths with fixed endpoints. We will write $ f_0 \cong f_1 $ for two homotopic paths $ f_0 $ and $ f_1 $.
\end{lemma}

\begin{proof}
\hfill
\begin{itemize}
\item $ f $ is homotopic to $ f $.
\item If $ f_0 $ is homotopic to $ f_1 $ by a homotopy $ f_t $, then $ f_1 $ is homotopic to $ f_0 $ by the homotopy $ f_{1 - t} $.
\item If $ f_0 $ is homotopic to $ f_1 $ by a homotopy $ f_t $ and $ f_1 = g_0 $ is homotopic to $ g_1 $ by a homotopy $ g_t $, then $ f_0 $ is homotopic to $ g_1 $ by the homotopy
$$ h_t =
\begin{cases}
f_{2t} & 0 \le t \le \tfrac{1}{2} \\
g_{2t - 1} & \tfrac{1}{2} \le t \le 1
\end{cases}.
$$
Then
$$ \function[H]{I \times I}{X}{\rb{s, t}}{h_t\rb{s}} $$
is continuous because its restriction to the closed subsets $ I \times \sb{0, 1 / 2} $ and $ I \times \sb{1 / 2, 1} $ is continuous, since if the restriction to two closed subsets is continuous then the restriction to the union of these subsets is continuous.
\end{itemize}
\end{proof}

\lecture{3}{Wednesday}{16/01/19}

Let $ X $ be a topological space and $ I = \sb{0, 1} $. If $ f : I \to X $ is a path, $ \sb{f} $ is the class of all paths on $ X $ homotopic to $ f $.

\begin{definition*}
Let $ f, g : I \to X $ be two paths such that $ f\rb{1} = g\rb{0} $. The \textbf{product path} $ f \cdot g $ is the path
$$ \rb{f \cdot g}\rb{s} =
\begin{cases}
f\rb{2s} & 0 \le s \le \tfrac{1}{2} \\
g\rb{2s - 1} & \tfrac{1}{2} \le s \le 1
\end{cases}.
$$
\end{definition*}

A convention is that whenever we write $ f \cdot g $ we implicitly assume $ f\rb{1} = g\rb{0} $.

\begin{lemma}
\label{lem:1.2}
Let $ f_0, f_1, g_0, g_1 $ be paths on $ X $ such that $ f_1 \cong f_0 $ and $ g_0 \cong g_1 $. Then $ f_0 \cdot g_0 \cong f_1 \cdot g_1 $.
\end{lemma}

\begin{proof}
$$ \function{I \times I}{X}{\rb{s, t}}{\rb{f_t \cdot g_t}\rb{s}} $$
is a homotopy between $ f_0 \cdot g_0 $ and $ f_1 \cdot g_1 $.
\end{proof}

\begin{remark*}
Let $ \phi : \sb{0, 1} \to \sb{0, 1} $ be continuous such that $ \phi\rb{0} = 0 $ and $ \phi\rb{1} = 1 $. If $ f : I \to X $ is a path, then $ f\phi \cong f $. This is a \textbf{reparametrisation}.
\end{remark*}

\begin{proof}
Define
$$ \phi_t\rb{s} = \rb{1 - t}\phi\rb{s} + ts, $$
then $ f\phi_t $ is a homotopy between $ f\phi $ and $ f $.
\end{proof}

For $ x \in X $, let the \textbf{constant path} at $ x $ be
$$ \function[c_x]{I}{X}{s}{x}. $$
For a path $ f : I \to X $, define
$$ \function[f^{-1}]{I}{X}{s}{f\rb{1 - s}}. $$

\begin{lemma}
\label{lem:1.3}
Let $ f, g, h : I \to X $ be paths. Then
\begin{enumerate}
\item $ \rb{f \cdot g} \cdot h \cong f \cdot \rb{g \cdot h} $,
\item $ f \cdot c_{f\rb{1}} \cong f $ and $ c_{f\rb{0}} \cdot f \cong f $, and
\item $ f \cdot f^{-1} \cong c_{f\rb{0}} $ and $ f^{-1} \cdot f \cong c_{f\rb{1}} $.
\end{enumerate}
\end{lemma}

\begin{proof}
\hfill
\begin{enumerate}
\item $ \rb{\rb{f \cdot g} \cdot h}\phi = f \cdot \rb{g \cdot h} $, where
$$ \phi\rb{s} =
\begin{cases}
\tfrac{s}{2} & s \in \sb{0, \tfrac{1}{2}} \\
s - \tfrac{1}{4} & s \in \sb{\tfrac{1}{2}, \tfrac{3}{4}} \\
2s - 1 & s \in \sb{\tfrac{3}{4}, 1}
\end{cases},
$$
so $ \rb{f \cdot g} \cdot h \cong f \cdot \rb{g \cdot h} $ by reparametrisation.
\item Again reparametrisation, by
$$ \psi\rb{s} =
\begin{cases}
2s & s \in \sb{0, \tfrac{1}{2}} \\
1 & s \in \sb{\tfrac{1}{2}, 1}
\end{cases},
\qquad \chi\rb{s} =
\begin{cases}
0 & s \in \sb{0, \tfrac{1}{2}} \\
2s - 1 & s \in \sb{\tfrac{1}{2}, 1}
\end{cases}.
$$
\item Define
$$ H\rb{s, t} =
\begin{cases}
f\rb{\max\cb{1 - 2s, t}} & s \in \sb{0, \tfrac{1}{2}} \\
f\rb{\max\cb{2s - 1, t}} & s \in \sb{\tfrac{1}{2}, 1}
\end{cases}.
$$
$ H $ is continuous, and
$$ H\rb{s, 0} = f^{-1} \cdot f, \qquad H\rb{s, 1} = c_{f\rb{1}}. $$
The inverse is similar.
\end{enumerate}
\end{proof}

\begin{definition*}
A \textbf{loop} with \textbf{basepoint} $ x_0 \in X $ is a path $ f : I \to X $ such that $ f\rb{0} = f\rb{1} = x_0 $.
\end{definition*}

\begin{definition*}
Denote by $ \pi_1\rb{X, x_0} $ the set of homotopy classes $ \sb{f} $ of loops $ f : I \to X $ with basepoint $ x_0 $.
\end{definition*}

\begin{proposition}
$ \pi_1\rb{X, x_0} $ is a group with product $ \sb{f}\sb{g} = \sb{f \cdot g} $ and neutral element $ c_{x_0} : I \to X $, the constant path at $ x_0 $.
\end{proposition}

\begin{proof}
Follows directly from Lemma \ref{lem:1.2} and Lemma \ref{lem:1.3}.
\end{proof}

\begin{definition*}
$ \pi_1\rb{X, x_0} $ is the \textbf{fundamental group} of $ X $ at $ x_0 $.
\end{definition*}

\begin{example*}
Let $ X \subseteq \R^n $ be a convex set and $ x_0 \in X $. Then $ \pi_1\rb{X, x_0} = 0 $.
\end{example*}

\begin{proof}
$ X $ is convex gives that all loops are homotopic to each other.
\end{proof}

\begin{example*}
\hfill
\begin{itemize}
\item The fundamental group of a space $ X $ with the trivial topology is trivial, since $ X $ is simply-connected, because all maps $ f : I \to X $ are continuous, so path-connected and all paths are homotopic.
\item The fundamental group of a space $ X $ with the discrete topology is trivial, since $ f : I \to X $ continuous gives $ f $ constant.
\end{itemize}
\end{example*}

Assume $ x_0, x_1 \in X $ such that $ x_0 $ and $ x_1 $ are in the same path component of $ X $. Let $ h : I \to X $ be a path such that $ h\rb{0} = x_0 $ and $ h\rb{1} = x_1 $. Define
$$ \function[\beta_h]{\pi_1\rb{X, x_1}}{\pi_1\rb{X, x_0}}{\sb{f}}{\sb{h \cdot f \cdot h^{-1}}}. $$
This is well-defined by Lemma \ref{lem:1.2}.

\begin{proposition}
$ \beta_h : \pi_1\rb{X, x_1} \to \pi_1\rb{X, x_0} $ is an isomorphism.
\end{proposition}

\begin{proof}
It is a homomorphism.
$$ \beta_h\sb{f \cdot g} = \sb{h \cdot f \cdot g \cdot h^{-1}} = \sb{h \cdot f \cdot h^{-1}}\sb{h \cdot g \cdot h^{-1}} = \beta_h\sb{f} \cdot \beta_h\sb{g}, $$
and $ \beta_h\sb{c_{x_1}} = \sb{c_{x_1}} $. It is bijective with $ \rb{\beta_h}^{-1} = \beta_{h^{-1}} $.
\end{proof}

If $ X $ is path-connected, we often write $ \pi_1\rb{X} $ instead of $ \pi_1\rb{X, x_0} $.

\begin{definition*}
$ X $ is \textbf{simply-connected} if it is path-connected and $ \pi_1\rb{X} = 0 $.
\end{definition*}

\begin{proposition}
\label{prop:1.6}
$ X $ is simply-connected if and only if there exists a unique homotopy class of paths between any two points of $ X $.
\end{proposition}

\begin{proof}
\hfill
\begin{itemize}
\item[$ \implies $] There exists a path between any two points. Let $ f, g $ be two paths from $ x_0 $ to $ x_1 $ for $ x_0, x_1 \in X $. $ f \cdot g^{-1} \cong g \cdot g^{-1} $ gives $ f \cong f \cdot g^{-1} \cdot g \cong g \cdot g^{-1} \cdot g \cong g $.
\item[$ \impliedby $] $ X $ is path-connected. $ x_1 = x_0 $ gives that all loops at $ x_0 $ are homotopic to each other, so $ \pi_1\rb{X} = 0 $.
\end{itemize}
\end{proof}

\subsubsection{The fundamental group of the circle}

Goal is to show that $ \pi_1\rb{S^1} \cong \Z $.

\lecture{4}{Friday}{18/01/19}

\begin{definition*}
A \textbf{covering space} of a space $ X $ is a space $ \widetilde{X} $ and a continuous map $ p : \widetilde{X} \to X $ such that for each $ x \in X $ there is an open $ x \in U \subseteq X $ such that
\begin{itemize}
\item $ p^{-1}\rb{U} = \bigcup_{j \in J} \widetilde{U_j} $, where $ \widetilde{U_j} \subseteq \widetilde{X} $ is open,
\item $ \widetilde{U_i} \cap \widetilde{U_j} = \emptyset $ if $ i \ne j $, and
\item $ p \mid_{\widetilde{U_j}} : \widetilde{U_j} \to U $ is a homeomorphism for all $ j \in J $.
\end{itemize}
Such a $ U $ is called \textbf{evenly covered}. The $ \widetilde{U_j} $ are called \textbf{sheets}.
\end{definition*}

\begin{example*}
$$ \function[p]{\R}{S^1}{s}{\rb{\cos\rb{2\pi s}, \sin\rb{2\pi s}}}. $$
\end{example*}

\begin{definition*}
Let $ p : \widetilde{X} \to X $ be a covering space. A \textbf{lift} of a continuous map $ f : Y \to X $ is a continuous map $ \widetilde{f} : Y \to \widetilde{X} $ such that $ p\widetilde{f} = f $, so
$$
\begin{tikzcd}
& \widetilde{X} \arrow{d}{p} \\
Y \arrow{ur}{\widetilde{f}} \arrow[swap]{r}{f} & X
\end{tikzcd}.
$$
\end{definition*}

\begin{proposition}[Unique lifting property]
\label{prop:1.34}
Let $ p : \widetilde{X} \to X $ be a covering space and $ f : Y \to X $ be a continuous map. If there are two lifts $ \widetilde{f_1}, \widetilde{f_2} : Y \to \widetilde{X} $ of $ f $ such that $ \widetilde{f_1}\rb{y} = \widetilde{f_2}\rb{y} $ for some $ y \in Y $ and if $ Y $ is connected, then $ \widetilde{f_1} = \widetilde{f_2} $.
\end{proposition}

\begin{proof}
Let $ y \in Y $ and $ U \subseteq X $ be an evenly covered neighbourhood of $ f\rb{y} $. Then
$$ p^{-1}\rb{U} = \bigcup_j \widetilde{U_j}. $$
Let $ \widetilde{U_1} $ be the sheet such that $ \widetilde{f_1}\rb{y} \in \widetilde{U_1} $, and let $ \widetilde{U_2} $ be the sheet such that $ \widetilde{f_2}\rb{y} \in \widetilde{U_2} $. Let $ N \subseteq Y $ be open and $ y \in N $ such that $ \widetilde{f_1}\rb{N} \subseteq \widetilde{U_1} $ and $ \widetilde{f_2}\rb{N} \subseteq \widetilde{U_2} $. We have $ p\widetilde{f_1} = p\widetilde{f_2} $.
$$ \widetilde{f_1}\rb{y} = \widetilde{f_2}\rb{y} \qquad \iff \qquad \widetilde{U_1} = \widetilde{U_2} \qquad \iff \qquad \widetilde{f_1} \mid_N = \widetilde{f_2} \mid_N. $$
Let
$$ A = \cb{y \in Y \mid \widetilde{f_1}\rb{y} = \widetilde{f_2}\rb{y}}, $$
so $ A $ is open and $ Y \setminus A $ is open. Thus $ A \ne \emptyset $ gives $ A = Y $.
\end{proof}

\begin{proposition}[Homotopy lifting property]
\label{prop:1.30}
Let $ p : \widetilde{X} \to X $ be a covering space and $ F : Y \times I \to X $ be a continuous map such that there exists a lift $ \widetilde{f_0} : Y \times \cb{0} \to \widetilde{X} $ of $ F \mid_{Y \times \cb{0}} $. Then there is a unique lift $ \widetilde{F} : Y \times I \to \widetilde{X} $ of $ F $ such that $ \widetilde{F} \mid_{Y \times \cb{0}} = \widetilde{f_0} $.
\end{proposition}

\begin{proof}
Let $ y_0 \in Y $ and $ t \in I $. There are open $ y_0 \in N_t \subseteq Y $ and $ t \in \rb{a_t, b_t} \subseteq I $ such that $ F\rb{N_t \times \rb{a_t, b_t}} \subseteq U \subseteq X $, where $ U \subseteq X $ is open and evenly covered. Compactness of $ I $ gives that there exist
$$ 0 = t_0 < \dots < t_m = 1, $$
and there exists $ y_0 \in N \subseteq Y $ open such that $ F\rb{N \times \sb{t_i, t_{i + 1}}} \subseteq U_i \subseteq X $, where $ U_i \subseteq X $ is open and evenly covered. We inductively construct a lift $ \widetilde{F} \mid_{N \times I} $ of $ F \mid_{N \times I} $.
\begin{itemize}
\item $ \widetilde{F} \mid_{N \times \sb{0, 0}} = \widetilde{f_0} \mid_{N \times \sb{0, 0}} $ exists.
\item Assume the lift has been constructed on $ N \times \sb{0, t_i} $. Let $ \widetilde{U_i} \subseteq \widetilde{X} $ be such that $ p \mid_{\widetilde{U_i}} : \widetilde{U_i} \to U_i $ such that $ \widetilde{F}\rb{y_0, t_i} \subseteq \widetilde{U_i} $. After shrinking $ N $, may assume $ \widetilde{F}\rb{N \times \cb{t_i}} \subseteq \widetilde{U_i} $. Define $ \widetilde{F} $ on $ N \times \sb{t_i, t_{i + 1}} $ to be composition of $ F $ with the homeomorphism $ p^{-1} : U_i \to \widetilde{U_i} $.
\end{itemize}
After finitely many steps we obtain a lift $ \widetilde{F} : N \times I \to \widetilde{X} $, where $ y_0 \in N \subseteq Y $ is open, so for each $ y \in Y $ there is a neighbourhood $ N_y \subseteq Y $ such that $ F \mid_{N_y \times I} : N_y \times I \to X $ lifts. For all $ y \in Y $, $ \cb{y} \times I $ is connected and can be lifted, so Proposition \ref{prop:1.34} gives that the lift of $ N \times I $ is unique. Thus there is a unique lift $ \widetilde{F} : Y \times I \to \widetilde{X} $.
\end{proof}

\begin{example*}
Let $ X $ be a topological space and $ A $ be discrete. Then $ p : X \times A \to X $ is a covering space. This is the \textbf{trivial covering}. (Exercise: show the unique lifting property and the homotopy lifting property for the trivial covering)
\end{example*}

\begin{corollary}
Let $ f : I \to X $ be a path, $ f\rb{0} = x_0 $, and $ p : \widetilde{X} \to X $ be a covering space. For each $ \widetilde{x_0} \in p^{-1}\rb{x_0} $, there is a unique lift $ \widetilde{f} : I \to \widetilde{X} $ such that $ \widetilde{f}\rb{0} = \widetilde{x_0} $.
\end{corollary}

\begin{proof}
Proposition \ref{prop:1.30} for $ Y $ a point.
\end{proof}

\begin{theorem}
\label{thm:1.7}
Let $ x_0 = \rb{1, 0} \in S^1 $. $ \pi_1\rb{S^1, x_0} $ is the infinite cyclic group generated by the homotopy class of the loop
$$ \function[\omega]{I}{S^1}{s}{\rb{\cos\rb{2\pi s}, \sin\rb{2\pi s}}}. $$
\end{theorem}

\begin{remark*}
\hfill
\begin{itemize}
\item $ \sb{\omega}^n = \sb{\omega_n} $, where
$$ \omega_n\rb{s} = \rb{\cos\rb{2\pi ns}, \sin\rb{2\pi ns}}. $$
\item
$$ \function[p]{\R}{S^1}{s}{\rb{\cos\rb{2\pi s}, \sin\rb{2\pi s}}} $$
is a covering space.
\item $ \omega_n $ lifts to
$$ \function[\widetilde{\omega_n}]{I}{\R}{s}{ns}, $$
such that $ \widetilde{\omega_n}\rb{0} = 0 $ and $ \widetilde{\omega_n}\rb{1} = n $.
\end{itemize}
\end{remark*}

\begin{proof}[Proof of Theorem \ref{thm:1.7}]
\hfill
\begin{itemize}
\item If $ f : I \to S^1 $ be a loop at $ x_0 $, then the homotopy lifting property gives that there exists a lift $ \widetilde{f} : I \to \R $ such that $ \widetilde{f}\rb{0} = 0 $. Since $ p\rb{\widetilde{f}\rb{1}} = f\rb{1} = x_0 $, then $ \widetilde{f}\rb{1} = n $ for some $ n \in \Z $. $ \widetilde{\omega_n} : I \to \R $ is another path such that $ \widetilde{\omega_n}\rb{0} = 0 $ and $ \widetilde{\omega_n}\rb{1} = n $, so $ \widetilde{f} \cong \widetilde{\omega_n} $. Let $ F : I \times I \to \R $ be a homotopy equivalence between $ \widetilde{f} $ and $ \widetilde{\omega_n} $. Then $ pF : I \times I \to S^1 $ gives a homotopy between $ p\widetilde{f} = f $ and $ p\widetilde{\omega_n} = \omega_n $.
\item Let $ m, n \in \Z $ and assume $ \omega_m \cong \omega_n $. Let $ F : I \times I \to S^1 $ be a homotopy.
$$ F\rb{0, t} = \omega_m\rb{t}, \qquad F\rb{1, t} = \omega_n\rb{t}, \qquad F\rb{s, 0} = F\rb{s, 1} = x_0, $$
for all $ s, t \in I $. The unique lifting property gives that $ \widetilde{\omega_n}, \widetilde{\omega_m} : I \to \R $ are unique lifts such that $ \widetilde{\omega_n}\rb{0} = 0 = \widetilde{\omega_m}\rb{0} $. The homotopy lifting property gives that $ F $ lifts uniquely to a homotopy $ \widetilde{F} : I \times I \to \R $ between $ \widetilde{\omega_n} $ and $ \widetilde{\omega_m} $, and $ \widetilde{F}\rb{s, 1} \in \Z $ for all $ s \in I $. Thus $ \widetilde{F}\rb{s, 1} = n = m $, so $ \omega_m \cong \omega_n $ if and only if $ n = m $.
\end{itemize}
\end{proof}

\lecture{5}{Tuesday}{22/01/19}

Lecture 5 is a problem class.

\lecture{6}{Wednesday}{23/01/19}

\begin{theorem}
Every non-constant polynomial $ p \in \C\sb{z} $ has a root in $ \C $.
\end{theorem}

\begin{proof}
May assume
$$ p\rb{z} = z^n + a_1z^{n - 1} + \dots + a_n. $$
Assume $ p $ has no roots in $ \C $. For each $ r \in \R_{\ge 0} $ we obtain a loop
$$ \function[f_r]{I}{\C}{s}{\dfrac{p\rb{re^{2\pi is}} / p\rb{r}}{\abs{p\rb{re^{2\pi is}} / p\rb{r}}}}, $$
so $ \abs{f_r\rb{s}} = 1 $. $ f_r\rb{0} = 1 $ and $ f_r\rb{1} = 1 $, so $ f_r $ is a loop based at $ 1 $. $ f_0 $ is the constant loop at $ 1 $. $ f_r\rb{s} $ depends continuously on $ r $, so $ f_r \cong f_0 $ for all $ r \in \R_{\ge 0} $ and $ \sb{f_r} = \sb{f_0} = 0 \in \pi_1\rb{S^1} $. Fix $ r \in \R_{\ge 0} $ such that $ r > 1 $ and $ r > \abs{a_1} + \dots + \abs{a_n} $. For $ \abs{z} = r $ we have
$$ \abs{z^n} > \rb{\abs{a_1} + \dots + \abs{a_n}}\abs{z^{n - 1}} \ge \abs{a_1z^{n - 1}} + \dots + \abs{a_n} \ge \abs{a_1z^{n - 1} + \dots + a_n}. $$
Hence, for $ 0 \le t \le 1 $ the polynomial $ p_t\rb{z} = z^n + t\rb{a_1z^{n - 1} + \dots + a_n} $ has no root $ z $ with $ \abs{z} = r $. Define
$$ F_r\rb{t, s} = \dfrac{p_t\rb{re^{2\pi is}} / p_t\rb{r}}{\abs{p_t\rb{re^{2\pi is}} / p_t\rb{r}}}. $$
$ F_r\rb{0, s} = \omega_n\rb{s} $ and $ F_r\rb{1, s} = f_r\rb{s} $, so $ \sb{\omega_n} = \sb{f_r} = 0 \in \pi_1\rb{S^1} $. Theorem \ref{thm:1.7} gives that $ n = 0 $, so $ p $ is constant.
\end{proof}

See Hatcher Theorem 1.9 and Theorem 1.10 for more applications.

\begin{proposition}
Let $ X, Y $ be topological spaces, $ x_0 \in X $, and $ y_0 \in Y $. Then
$$ \pi_1\rb{X \times Y, \rb{x_0, y_0}} \cong \pi_1\rb{X, x_0} \times \pi_1\rb{Y, y_0}. $$
\end{proposition}

\begin{proof}
A map
$$ \function[f]{Z}{X \times Y}{z}{\rb{g\rb{z}, h\rb{z}}} $$
is continuous if and only if $ g : Z \to X $ and $ h : Z \to Y $ are continuous. For $ Z = I $,
$$ \correspondence{\text{loops in} \ X \times Y \ \text{based at} \ \rb{x_0, y_0}}{\text{loops in} \ X \ \text{based at} \ x_0 \ \} \times \{ \ \text{loops in} \ Y \ \text{based at} \ y_0}. $$
Two loops
$$ \function[f_1]{I}{X \times Y}{s}{\rb{g_1\rb{s}, h_1\rb{s}}}, \qquad \function[f_2]{I}{X \times Y}{s}{\rb{g_2\rb{s}, h_2\rb{s}}} $$
are homotopic if and only if $ g_1 \cong g_2 $ and $ h_1 \cong h_2 $, so there is a bijection
$$ \pi_1\rb{X \times Y, \rb{x_0, y_0}} \cong \pi_1\rb{X, x_0} \times \pi_1\rb{Y, y_0}. $$
$ f_1 \cdot f_2 = \rb{g_1 \cdot g_2, h_1 \cdot h_2} $ and the constant loop is mapped to the constant loop, so this is also a group isomorphism.
\end{proof}

\begin{example*}
The torus $ S^1 \times S^1 $ has
$$ \pi_1\rb{S^1 \times S^1} \cong \pi_1\rb{S^1} \times \pi_1\rb{S^1} \cong \Z^2. $$
\end{example*}

\subsubsection{Induced homomorphisms}

Let $ X, Y $ be topological spaces, $ x_0 \in X $, and $ \phi : X \to Y $. An observation is that $ \phi $ induces a homomorphism
$$ \function[\phi_*]{\pi_1\rb{X, x_0}}{\pi_1\rb{Y, \phi\rb{x_0}}}{\sb{f}}{\sb{\phi f}}. $$
$ \phi_* $ is well-defined, since if $ f_t $ is a homotopy between the loops $ f_0 $ and $ f_1 $ based at $ x_0 $, then $ \phi f_t $ is a homotopy of loops between $ \phi f_0 $ and $ \phi f_1 $. Moreover,
$$ \phi\rb{f \cdot g} = \rb{\phi f} \cdot \rb{\phi g}, $$
and $ \phi $ maps the constant path at $ x_0 $ to the constant path at $ \phi\rb{x_0} $, so $ \phi $ is a homomorphism.

\begin{proposition}
\hfill
\begin{enumerate}
\item Let $ \psi : X \to Y $ and $ \phi : Y \to Z $ be continuous maps between topological spaces, $ x_0 \in X $, and
$$ \psi_* : \pi_1\rb{X, x_0} \to \pi_1\rb{Y, \psi\rb{x_0}}, \qquad \phi_* : \pi_1\rb{Y, \psi\rb{x_0}} \to \pi_1\rb{Z, \phi\psi\rb{x_0}}, $$
$$ \rb{\phi\psi}_* : \pi_1\rb{X, x_0} \to \pi_1\rb{Z, \phi\psi\rb{x_0}}. $$
Then $ \rb{\phi\psi}_* = \phi_*\psi_* $.
\item Let $ id_X : X \to X $ be the identity then
$$ \rb{id_X}_* : \pi_1\rb{X, x_0} \to \pi_1\rb{X, x_0} $$
is the identity.
\end{enumerate}
\end{proposition}

\begin{proof}
\hfill
\begin{enumerate}
\item Let $ f : I \to X $ be a loop at $ x_0 $, then
$$ \rb{\phi\psi}_*\rb{\sb{f}} = \sb{\rb{\phi\psi}f} = \sb{\phi\rb{\psi f}} = \phi_*\rb{\sb{\psi f}} = \phi_*\psi_*\rb{\sb{f}}. $$
\item $ \rb{id_X}_*\rb{\sb{f}} = \sb{id_Xf} = \sb{f} $.
\end{enumerate}
\end{proof}

These two observations yield in particular that if $ \phi : X \to Y $ is a homeomorphism with inverse $ \psi : Y \to X $, then
$$ \phi_* : \pi_1\rb{X, x_0} \to \pi_1\rb{Y, \phi\rb{x_0}} $$
is an isomorphism with inverse $ \psi_* $.

\lecture{7}{Friday}{25/01/19}

\begin{proposition}
\label{prop:1.18}
Let $ \phi : X \to Y $ be a homotopy equivalence. Then
$$ \phi_* : \pi_1\rb{X, x_0} \to \pi_1\rb{Y, \phi\rb{x_0}} $$
is an isomorphism for all $ x_0 \in X $.
\end{proposition}

Recall that if $ x_0, x_1 \in X $ and $ h : I \to X $ is a path such that $ h\rb{0} = x_0 $ and $ h\rb{1} = x_1 $, then we obtain an isomorphism
$$ \function[\beta_h]{\pi_1\rb{X, x_1}}{\pi_1\rb{X, x_0}}{\sb{f}}{\sb{h \cdot f \cdot h^{-1}}}. $$

\begin{lemma}
\label{lem:1.19}
Let $ \phi_t : X \to Y $ be a homotopy and $ x_0 \in X $. Define the path
$$ \function[h]{I}{Y}{s}{\phi_s\rb{x_0}}, $$
where $ h\rb{0} = \phi_0\rb{x_0} $ and $ h\rb{1} = \phi_1\rb{x_0} $. Then $ \rb{\phi_0}_* = \beta_h\rb{\phi_1}_* $, that is the following diagram commutes.
$$
\begin{tikzcd}
& \pi_1\rb{Y, \phi_1\rb{x_0}} \arrow{dd}{\beta_h}[swap]{\sim} \\
\pi_1\rb{X, x_0} \arrow{ur}{\rb{\phi_1}_*} \arrow[swap]{dr}{\rb{\phi_0}_*} & \\
& \pi_1\rb{Y, \phi_0\rb{x_0}}
\end{tikzcd}.
$$
\end{lemma}

\begin{proof}
For $ t \in I $, define the path
$$ \function[h_t]{I}{X}{s}{h\rb{ts}}, $$
where $ h_t\rb{0} = \phi_0\rb{x_0} $ and $ h_t\rb{1} = h\rb{t} = \phi_t\rb{x_0} $. Let $ f $ be a loop at $ x_0 $. Define
$$ F_t = h_t \cdot \rb{\phi_tf} \cdot h_t^{-1}. $$
Then $ F_t $ is a loop at $ \phi_0\rb{x_0} $, which is continuous in $ t $. So $ F_t $ is a homotopy of loops between
$$ F_0 = h_0 \cdot \rb{\phi_0f} \cdot h_0^{-1} \cong \phi_0f, \qquad F_1 = h_1 \cdot \rb{\phi_1f} \cdot h_1^{-1} = h \cdot \rb{\phi_1f} \cdot h^{-1}. $$
Hence
$$ \rb{\phi_0}_*\rb{\sb{f}} = \sb{\phi_0f} = \sb{h \cdot \rb{\phi_1f} \cdot h^{-1}} = \beta_h\rb{\sb{\phi_1f}} = \beta_h\rb{\phi_1}_*\rb{\sb{f}}. $$
\end{proof}

Lemma \ref{lem:1.19} implies in particular the following.

\begin{corollary}
If $ \psi : X \to X $ is continuous and $ \psi \cong id_X $, then
$$ \psi_* : \pi_1\rb{X, x_0} \to \pi_1\rb{X, \psi\rb{x_0}} $$
is an isomorphism for all $ x_0 \in X $.
\end{corollary}

\begin{proof}
By Lemma \ref{lem:1.19} there is a path $ h $ from $ \psi\rb{x_0} $ to $ x_0 $ such that
$$
\begin{tikzcd}
& \pi_1\rb{X, x_0} \arrow{dd}{\beta_h}[swap]{\sim} \\
\pi_1\rb{X, x_0} \arrow{ur}{\rb{id_X}_*} \arrow[swap]{dr}{\psi_*} & \\
& \pi_1\rb{X, \psi\rb{x_0}}
\end{tikzcd},
$$
so $ \psi_* = \beta_h $ hence an isomorphism.
\end{proof}

\begin{proof}[Proof of Proposition \ref{prop:1.18}]
Let $ \phi : X \to Y $ be a homotopy equivalence. Let $ \psi : Y \to X $ be a homotopy inverse of $ \phi $, that is $ \phi\psi \cong id_Y $ and $ \psi\phi \cong id_X $.
$$ \pi_1\rb{X, x_0} \xrightarrow{\phi_*} \pi_1\rb{Y, \phi\rb{x_0}} \xrightarrow{\psi_*} \pi_1\rb{X, \psi\phi\rb{x_0}} \xrightarrow{\phi_*} \pi_1\rb{Y, \psi\phi\psi\rb{x_0}}. $$
Have to show that $ \phi_* $ is bijective. The observation above gives that $ \rb{\psi\phi}_* = \psi_*\phi_* $ is an isomorphism, so $ \phi_* $ is injective and $ \psi_* $ is surjective. Similarly $ \rb{\phi\psi}_* = \phi_*\psi_* $ is an isomorphism, so $ \psi_* $ is injective and $ \phi_* $ is surjective.
\end{proof}

\begin{lemma}
\label{lem:1.15}
Let $ X $ be a topological space and $ x_0 \in X $. Assume
$$ X = \bigcup_{\alpha \in \Lambda} A_\alpha, $$
such that
\begin{itemize}
\item the $ A_\alpha $ are all open and path-connected,
\item $ x_0 \in A_\alpha $ for all $ \alpha \in \Lambda $, and
\item all the intersections $ A_\alpha \cap A_\beta $ are path-connected for all $ \alpha, \beta \in \Lambda $.
\end{itemize}
If $ f $ is a loop in $ X $ at $ x_0 $, then we can write $ \sb{f} = \sb{h_1} \dots \sb{h_m} $, such that the $ h_i $ are loops at $ x_0 $, and each contained in a single $ A_{\alpha_i} $.
\end{lemma}

\begin{proof}
$ f $ is continuous, so for all $ s \in I $ there is an open neighbourhood $ V_s $ such that $ f\rb{V_s} $ such that $ f\rb{V_s} \subseteq A_\alpha $ for some $ \alpha $. We can choose $ V_s $ to be an interval $ \rb{a_s, b_s} $ such that $ f\rb{\sb{a_s, b_s}} \subseteq A_\alpha $. $ I $ is compact gives that a finite number of such intervals cover $ I $, so there is a partition
$$ 0 = s_0 < \dots < s_m = 1, $$
such that $ f\rb{\sb{s_{i - 1}, s_i}} \subseteq A_{\alpha_i} $ for some $ \alpha_i $. Let $ f_i $ be the path obtained by restricting $ f $ to $ \sb{s_{i - 1}, s_i} $, and rescaling. $ f \cong f_1 \cdot \dots \cdot f_m $ for $ f_i \subseteq A_{\alpha_i} $ and $ A_{\alpha_i} \cap A_{\alpha_j} $ is path-connected. Let $ g_i $ be a path from $ x_0 $ to $ f\rb{s_i} $ in $ A_{\alpha_i} \cap A_{\alpha_{i + 1}} $. Let $ g_0, g_m $ be the constant loops at $ x_0 $. $ h_i = g_{i - 1} \cdot f_i \cdot g_i^{-1} $ is a loop based at $ x_0 $ and $ h_i \subseteq A_{\alpha_i} $. Thus
$$ f \cong \rb{g_0 \cdot f_1 \cdot g_1^{-1}} \cdot \dots \cdot \rb{g_{m - 1} \cdot f_m \cdot g_m^{-1}}, $$
so $ \sb{f} = \sb{h_1} \dots \sb{h_m} $.
\end{proof}

\lecture{8}{Tuesday}{29/01/19}

\begin{example*}
M\"obius strip $ M $ deformation retracts to $ S^1 $. Thus $ \phi : M \to S^1 $ is a homotopy equivalence, so $ \pi_1\rb{M} \cong \pi_1\rb{S^1} \cong \Z $.
\end{example*}

\begin{example*}
There is no deformation retraction of $ S^1 $ to a point $ p \in S^1 $ because $ \pi_1\rb{S^1} \ncong \pi_1\rb{p} $.
\end{example*}

\begin{example*}
There is no retraction of the disc $ D^2 $ to its boundary $ S^1 \subseteq D^2 $.
\end{example*}

\begin{proof}
Assume there is a retraction $ r : D^2 \to S^1 $, consider the embedding $ i : S^1 \hookrightarrow D^2 $. Then $ ri = id_{S^1} $. Thus
$$
\begin{tikzcd}[row sep=tiny]
\pi_1\rb{S^1} \arrow{r}{i_*} \arrow[cong]{d} & \pi_1\rb{D^2} \arrow{r}{r_*} \arrow[cong]{d} & \pi_1\rb{S^1} \arrow[cong]{d} \\
\Z & 0 & \Z
\end{tikzcd},
$$
so $ r_*i_*\rb{\pi_1\rb{S^1}} = 0 $ but $ r_*i_* = \rb{ri}_* = id_{\pi_1\rb{S^1}} $, a contradiction.
\end{proof}

\begin{theorem}[Brouwer fixed point theorem]
Let $ h : D^2 \to D^2 $ be a continuous map. Then $ h $ has a fixed point, that is there exists $ x \in D^2 $ such that $ h\rb{x} = x $.
\end{theorem}

\begin{proof}
Assume $ h\rb{x} \ne x $ for all $ x \in D^2 $. Define $ r : D^2 \to S^1 $ by defining $ r\rb{x} $ to be the intersection of the ray starting at $ h\rb{x} $ towards $ x $ with $ S^1 $. $ r $ is continuous, and if $ x \in S^1 $, then $ r\rb{x} = x $, so $ r $ is a retraction, a contradiction.
\end{proof}

Lemma \ref{lem:1.15} gives that if $ U_1, U_2 \subseteq X $ are open and path-connected such that $ U_1 \cup U_2 = X $ and $ U_1 \cap U_2 $ is path-connected and $ x_0 \in U_1 \cap U_2 $, then every $ \sb{f} \in \pi_1\rb{X, x_0} $ can be factorised as $ \sb{f} = \sb{g_1}\sb{h_1} \dots \sb{g_n}\sb{h_n} $ such that the $ g_i $ are loops at $ x_0 $ contained in $ U_1 $ and the $ h_i $ are loops at $ x_0 $ contained in $ U_2 $. In other words, $ i_1 : U_1 \hookrightarrow X $ and $ i_2 : U_2 \hookrightarrow X $, so
$$ \rb{i_1}_* : \pi_1\rb{U_1, x_0} \to \pi_1\rb{X, x_0}, \qquad \rb{i_2}_* : \pi_1\rb{U_2, x_0} \to \pi_1\rb{X, x_0}. $$
Lemma \ref{lem:1.15} gives that $ \rb{i_1}_*\rb{\pi_1\rb{U_1, x_0}} \cup \rb{i_2}_*\rb{\pi_1\rb{U_2, x_0}} $ generate $ \pi_1\rb{X, x_0} $.

\begin{proposition}
$ \pi_1\rb{S^n} = 0 $ if $ n \ge 2 $.
\end{proposition}

\begin{proof}
Let $ U_1 = S^n \setminus \cb{\rb{1, 0, \dots, 0}} $ and $ U_2 = S^n \setminus \cb{\rb{-1, 0, \dots, 0}} $. Then $ U_1 \cong \R^n $ and $ U_2 \cong \R^n $, by stereographic projection. $ U_1 \cup U_2 = S^n $ and $ U_1 \cap U_2 $ is path-connected. Let $ x_0 \in U_1 \cap U_2 $. $ \pi_1\rb{U_1, x_0} = 0 $ and $ \pi_1\rb{U_2, x_0} = 0 $, so Lemma \ref{lem:1.15} gives that $ \pi_1\rb{S^n, x_0} $.
\end{proof}

\pagebreak

\subsection{Seifert-van Kampen theorem}

\subsubsection{Free products with amalgamation}

\begin{definition*}
If $ S $ is a set, then $ F_S $ is the \textbf{free group} on $ S $. We can write any group $ G $ as a quotient of some free group $ F_S $,
$$ G = \dfrac{F}{\ab{\ab{R}}}, $$
where $ \ab{\ab{R}} $ is the \textbf{normal closure} of $ R \subseteq F_S $, the smallest normal subgroup of $ F_S $ containing $ R $. We write $ G = \ab{S \mid R} $. This is called a \textbf{presentation} of $ G $.
\end{definition*}

Let $ G_0, G_1, G_2 $ be groups, and $ f_1 : G_0 \to G_1 $ and $ f_2 : G_0 \to G_2 $ be homomorphisms.

\begin{definition*}
A group $ H $ together with homomorphisms $ h_1 : G_1 \to H $ and $ h_2 : G_2 \to H $ such that $ h_1f_1 = h_2f_2 $ is an \textbf{amalgamated product} of $ G_1 $ and $ G_2 $ over $ G_0 $ if it satisfies the following universal property. For every group $ G $ and all homomorphisms $ h_1' : G_1 \to G $ and $ h_2' : G_2 \to G $ such that $ h_1'f_1 = h_2'f_2 $, there exists a unique homomorphism $ \alpha : H \to G $ such that $ h_1' = \alpha h_1 $ and $ h_2' = \alpha h_2 $.
$$
\begin{tikzcd}
G_0 \arrow{r}{f_1} \arrow[swap]{d}{f_2} & G_1 \arrow{d}{h_1} \arrow[bend left=30]{ddr}{h_1'} & \\
G_2 \arrow[swap]{r}{h_2} \arrow[bend right=30, swap]{drr}{h_2'} & H \arrow{dr}{\exists !\alpha} & \\
& & G.
\end{tikzcd}
$$
\end{definition*}

\begin{theorem}
Given $ f_1 : G_0 \to G_1 $ and $ f_2 : G_0 \to G_2 $. Then there exists an amalgamated product, unique up to isomorphism. We denote it by
$ G_1 \underset{G_0}{*} G_2 $.
\end{theorem}

\begin{proof}
Worksheet $ 2 $.
\end{proof}

\lecture{9}{Wednesday}{30/01/19}

$ G_0 = \cb{id} $ is the \textbf{free product}. We write $ G_1 * G_2 $ instead of $ G_1 \underset{\cb{id}}{*} G_2 $. Let $ G_1 = \ab{S_1 \mid R_1} $ and $ G_2 = \ab{S_2 \mid R_2} $. Then $ G_1 * G_2 = \ab{S_1 \sqcup S_2 \mid R_1 \cup R_2} $, with injections $ G_i \hookrightarrow G_1 * G_2 $ for $ i = 1, 2 $. More generally,
$$ G_1 * G_2 \cong \dfrac{G_1 \underset{G_0}{*} G_2}{N}. $$
where $ N $ is the normal closure of the set
$$ \cb{f_1\rb{g}f_2\rb{g}^{-1} \mid g \in G_0} \subseteq G_1 * G_2. $$

\subsubsection{The Seifert van-Kampen theorem}

\begin{theorem}[Seifert-van Kampen]
Let $ X $ be a topological space and $ U_1, U_2 \subseteq X $ be open and path-connected such that $ X = U_1 \cup U_2 $ and $ U_1 \cap U_2 $ is path-connected and let $ x_0 \in U_1 \cap U_2 $. Then
$$ \pi_1\rb{X, x_0} \cong \pi_1\rb{U_1, x_0} \underset{\pi_1\rb{U_1 \cap U_2, x_0}}{*} \pi_2\rb{U_2, x_0} \cong \dfrac{\pi_1\rb{U_1, x_0} * \pi_1\rb{U_2, x_0}}{N}, $$
where $ N $ is the normal closure of the set
$$ \cb{\rb{j_1}_*\rb{\omega}\rb{j_2}_*\rb{\omega}^{-1} \mid \omega \in \pi_1\rb{U_1 \cap U_2, x_0}}, $$
and $ j_i : U_1 \cap U_2 \hookrightarrow U_i $.
$$
\begin{tikzcd}
U_1 \cap U_2 \arrow[hookrightarrow]{r}{i_1} \arrow[hookrightarrow, swap]{d}{i_2} & U_1 \arrow[hookrightarrow]{d}{j_1} \\
U_2 \arrow[hookrightarrow, swap]{r}{j_2} & X
\end{tikzcd}
\qquad \implies \qquad
\begin{tikzcd}
\pi_1\rb{U_1 \cap U_2, x_0} \arrow{r}{\rb{i_1}_*} \arrow[swap]{d}{\rb{i_2}_*} & \pi_1\rb{U_1, x_0} \arrow{d}{\rb{j_1}_*} \\
\pi_1\rb{U_2, x_0} \arrow[swap]{r}{\rb{j_2}_*} & \pi_1\rb{U_1, x_0} \underset{\pi_1\rb{U_1 \cap U_2, x_0}}{*} \pi_2\rb{U_2, x_0}
\end{tikzcd}.
$$
\end{theorem}

\begin{proof}
Consider the natural homomorphism
$$ \Phi : \pi_1\rb{U_1, x_0} * \pi_1\rb{U_2, x_0} \to \pi_1\rb{X, x_0}. $$
$ \Phi $ is surjective by Lemma \ref{lem:1.15}. $ N \subseteq Ker\rb{\Phi} $. Want to show that $ N = Ker\rb{\Phi} $. A \textbf{factorisation} of an element $ \sb{f} \in \pi_1\rb{X, x_0} $ is a formal product $ \sb{f_1} \dots \sb{f_k} $ such that
\begin{itemize}
\item each $ f_i $ is a loop at $ x_0 $ in one of the $ U_i $ and $ \sb{f_i} \in \pi_1\rb{U_i, x_0} $ is its homotopy class, and
\item the loop $ f_1 \cdot \dots \cdot f_k $ is homotopic to $ f $ in $ X $.
\end{itemize}
A factorisation of $ \sb{f} $ is a word in $ \pi_1\rb{U_1, x_0} * \pi_1\rb{U_2, x_0} $ that is mapped to $ \sb{f} $ by $ \Phi $. Two factorisations of $ \sb{f} $ are \textbf{equivalent} if they are related by finitely many of the following two moves.
\begin{itemize}
\item If $ \sb{f_i} $ and $ \sb{f_{i + 1}} $ lie in the same group $ \pi_1\rb{U_i, x_0} $, exchange $ \sb{f_i}\sb{f_{i + 1}} $ with $ \sb{f_i \cdot f_{i + 1}} $. These are the relations in $ \pi_1\rb{U_i, x_0} * \pi_1\rb{U_i, x_0} $.
\item If $ f_i $ is a loop in $ U_1 \cap U_2 $, consider $ \sb{f_i} $ as an element in $ \pi_1\rb{U_1, x_0} $ instead of $ \pi_1\rb{U_2, x_0} $, and vice versa. These are the relations in $ \pi_1\rb{U_1, x_0} * \pi_1\rb{U_2, x_0} / N $.
\end{itemize}
Given $ \sb{f} \in \pi_1\rb{X, x_0} $, we want to show that any two factorisations of $ \sb{f} $ are equivalent. Let $ \sb{f_1} \dots \sb{f_k} $ and $ \sb{f_1'} \dots \sb{f_l'} $ be two factorisations of $ \sb{f} $, so the two loops $ f_1 \cdot \dots \cdot f_k $ and $ f_1' \cdot \dots \cdot f_k' $ are homotopic. Let $ F : I \times I \to X $ be a homotopy. By compactness, there exist
$$ 0 = s_0 < \dots < s_m = 1, \qquad 0 = t_0 < \dots < t_n = 1, $$
such that $ R_{i, j} = \sb{s_{i - 1}, s_i} \times \sb{t_{j - 1}, t_j} $ and $ F\rb{R_{ij}} \subseteq U_1 $ or $ F\rb{R_{ij}} \subseteq U_2 $. May assume $ 0 = s_0 < \dots < s_m = 1 $ subdivides the products $ f_1 \cdot \dots \cdot f_k $ and $ f_1' \cdot \dots \cdot f_l' $. Relabel the $ R_{ij} $ to $ R_1, \dots, R_{mn} $.
$$
\begin{array}{|c|c|c|}
\hline
mn - m + 1 & \dots & mn \\
\hline
\vdots & \ddots & \vdots \\
\hline
1 & \dots & m \\
\hline
\end{array}
$$
A path $ \gamma $ in $ I \times I $ from left to right gives a loop $ F \mid_\gamma $ in $ X $ at $ x_0 $. Let $ \gamma_r $ be the path separating the first $ r $ rectangles from the others, so
$$ F \mid_{\gamma_0} \cong f_1 \cdot \dots \cdot f_k, \qquad F \mid_{\gamma_{mn}} = f_1' \cdot \dots \cdot f_l'. $$
Let $ v $ be a grid point. Choose a path $ g_v $ in $ X $ from $ x_0 $ to $ F\rb{v} $, such that $ g_v $ is contained in $ U_1 \cap U_2 $ if $ F\rb{v} \in U_1 \cap U_2 $ and in a single $ U_i $ otherwise. This gives us a factorisation of $ \sb{F \mid_{\gamma_r}} $ into loops only contained in $ U_1 $ or $ U_2 $. The factorisations associated to $ \gamma_r $ and $ \gamma_{r + 1} $ are equivalent, because the homotopy between $ F \mid_{\gamma_r} $ and $ F \mid_{\gamma_{r + 1}} $ by pushing $ \gamma_r $ through $ R_r $ takes place within a single $ U_i $.
\end{proof}

\lecture{10}{Friday}{01/02/19}

\begin{theorem}[Seifert-van Kampen, strong version]
Let $ X $ be a path-connected topological space such that
\begin{itemize}
\item $ X = \bigcup_\alpha A_\alpha $,
\item $ A_\alpha $, $ A_\alpha \cap A_\beta $, and $ A_\alpha \cap A_\beta \cap A_\gamma $ are open and path-connected for all $ \alpha, \beta, \gamma $, and
\item $ x_0 \in \cap_\alpha A_\alpha $.
\end{itemize}
Then
$$ \pi_1\rb{X, x_0} \cong \dfrac{\underset{\alpha}{*} \pi_1\rb{A_\alpha, x_0}}{N}, $$
where $ N \subseteq \underset{\alpha}{*} \pi_1\rb{A_\alpha, x_0} $ is the normal closure of the set
$$ \cb{\rb{i_{\alpha\beta}}_*\rb{\omega}\rb{i_{\beta\alpha}}_*\rb{\omega}^{-1} \mid \omega \in \pi_1\rb{A_\alpha \cap A_\beta}}, $$
and $ i_{\alpha\beta} : A_\alpha \cap A_\beta \hookrightarrow A_\alpha $ is the inclusion.
\end{theorem}

\begin{example*}
Let $ S^1 \vee S^1 $ be the wedge product. Fix $ x \in S^1 $ and $ y \in S^1 $. Then
$$ S^1 \vee S^1 = \dfrac{S^1 \sqcup S^1}{x \sim y} = \overset{b}{\text{O}}\cdot\overset{a}{\text{O}}. $$
Let
$$ A = \text{O}\cdot(, \qquad B = \ )\cdot\text{O}, \qquad A \cap B = \ )\cdot(. $$
$ \pi_1\rb{A} \cong \ab{b} \cong \Z $, $ \pi_1\rb{B} \cong \ab{a} \cong \Z $, and $ \pi_1\rb{A \cap B} = \cb{id} $. $ A $, $ B $, and $ A \cap B $ are open and path-connected. Van Kampen gives
$$ \pi_1\rb{S^1 \vee S^1} \cong \pi_1\rb{A} * \pi_1\rb{B} \cong \Z * \Z \cong F_{\cb{a, b}}. $$
More generally, let $ X = S_{a_1}^1 \vee \dots \vee S_{a_n}^1 $. By induction,
$$ \pi_1\rb{X} = \Z * \dots * \Z \cong F_{\cb{a_1, \dots, a_n}}. $$
Similarly, let $ X = \bigvee_{\alpha \in \Lambda} S_\alpha^1 $. Strong version of van Kampen gives
$$ \pi_1\rb{X} = \underset{\alpha \in \Lambda}{*} \Z = F_\Lambda. $$
\end{example*}

\begin{example*}
Let $ T $ be a torus and $ x_0 \in T $. Let
$$ A = T \setminus \cb{\text{small closed disc} \ D}, \qquad B = \cb{\text{open set that contains} \ D \ \text{and} \ x_0}. $$
\begin{itemize}
\item $ A $ is homotopy equivalent to $ S^1 \vee S^1 $, so $ \pi_1\rb{A} \cong F_{\cb{a, b}} $.
\item $ B $ is homeomorphic to $ D^2 $, so $ \pi_1\rb{B} = \cb{id} $.
\item $ A \cap B $ is homotopy equivalent to $ S^1 $, so $ \pi_1\rb{A \cap B} \cong \Z $.
\end{itemize}
$ A $, $ B $, and $ A \cap B $ are open and path-connected. Van Kampen gives
$$ \pi_1\rb{T} \cong \dfrac{\pi_1\rb{A}}{\ab{\ab{i_*\rb{\pi_1\rb{A \cap B}}}}}, $$
where $ i : A \cap B \hookrightarrow A $. Then
$$ \function[i_*]{\pi_1\rb{A \cap B} = \ab{\omega}}{\pi_1\rb{A}}{\omega}{aba^{-1}b^{-1}}, $$
so
$$ \pi_1\rb{T} \cong \dfrac{F_{\cb{a, b}}}{\ab{\ab{aba^{-1}b^{-1}}}} = \ab{a, b \mid aba^{-1}b^{-1}} \cong \Z^2. $$
\end{example*}

\subsubsection{Applications to CW-complexes}

Let $ X $ be a path-connected topological space. Let $ Y $ be the space obtained by attaching $ 2 $-cells $ \cb{e_\alpha^2} $ to $ X $ along maps $ \phi_\alpha : \partial D^2 = S^1 \to X $. Consider the loops
$$ \function[\phi_\alpha']{I}{X}{s}{\phi_\alpha\rb{\cos\rb{2\pi s}, \sin\rb{2\pi s}}}, $$
based at $ \phi_\alpha'\rb{0} $. Let $ \gamma_\alpha $ be a path from $ x_0 $ to $ \phi_\alpha'\rb{0} $ for each $ \alpha $. Then $ \gamma_\alpha \cdot \phi_\alpha \cdot \gamma_\alpha^{-1} $ is a loop at $ x_0 $. After attaching $ e_\alpha^2 $, the loop $ \gamma_\alpha \cdot \phi_\alpha \cdot \gamma_\alpha^{-1} $ is homotopic to the constant loop at $ x_0 $. Let $ N \subseteq \pi_1\rb{X, x_0} $ be the normal closure of all the elements of the form $ \sb{\gamma_\alpha \cdot \phi_\alpha \cdot \gamma_\alpha^{-1}} $. The inclusion $ i : X \hookrightarrow Y $ yields
$$ i_* : \pi_1\rb{X, x_0} \to \pi_1\rb{Y, x_0}, $$
and $ N \subseteq Ker\rb{i_*} $.

\begin{proposition}
\label{prop:1.26}
This inclusion $ i : X \hookrightarrow Y $ induces a surjection
$$ i_* : \pi_1\rb{X, x_0} \to \pi_1\rb{Y, x_0}, $$
and $ Ker\rb{i_*} = N $, so
$$ \pi_1\rb{Y, x_0} \cong \dfrac{\pi_1\rb{X, x_0}}{N}. $$
\end{proposition}

\begin{proof}
Construct a space $ Z $ from $ Y $ by attaching a strip $ I \times I $ to $ Y $ by identifying the lower edge $ I \times \cb{0} $ with the path $ \gamma_\alpha $ and the right edge $ \cb{1} \times I $ with an arch on $ e_\alpha^2 $. Attach all the left edges of the strips with each other. $ Z $ deformation retracts to $ Y $. Choose a point $ y_\alpha \in e_\alpha^2 $ for each $ \alpha $, such that $ y_\alpha $ is not contained in $ X $ or in the attached strip. Let
$$ A = Z \setminus \bigcup_\alpha \cb{y_\alpha}, \qquad B = Z \setminus X. $$
\begin{itemize}
\item $ A $ deformation retracts to $ X $.
\item $ B $ is homotopy equivalent to a point.
\item $ A \cap B $ is homotopy equivalent to
$$ \cb{\text{paths} \ \gamma_\alpha \ \text{from} \ x_0 \ \text{to loops} \ \phi_\alpha'} = \overset{\phi_\alpha'}{\text{O}}\overset{\gamma_\alpha}{-}\overset{x_0}{\cdot}\overset{\gamma_\alpha}{-}\overset{\phi_\alpha'}{\text{O}}. $$
\end{itemize}
$ A $, $ B $, and $ A \cap B $ are open and path-connected. Van Kampen gives
$$ \pi_1\rb{Y} \cong \pi_1\rb{Z} = \dfrac{\pi_1\rb{A}}{\ab{\ab{j_*\rb{\pi_1\rb{A \cap B}}}}}, $$
where $ j : A \cap B \hookrightarrow A $ is the inclusion. So $ \ab{\ab{j_*\rb{\pi_1\rb{A \cap B}}}} $ is exactly $ N $. Thus $ \pi_1\rb{A} = \pi_1\rb{X} $.
\end{proof}

\lecture{11}{Tuesday}{05/02/19}

\begin{corollary}
For every group $ G $ there exists a two-dimensional CW-complex $ X_G $ such that $ \pi_1\rb{X_G} = G $.
\end{corollary}

\begin{proof}
Let $ G = \ab{\cb{g_\alpha} \mid \cb{r_\beta}} $ be a presentation of $ G $, that is
$$ G = \dfrac{F_{\cb{g_\alpha}}}{\ab{\ab{\cb{r_\beta}}}}. $$
Seen last time that $ \pi_1\rb{\bigvee_{g_\alpha} S_{g_\alpha}^1} = F_{\cb{g_\alpha}} $. Each word $ r_\beta $ defines a loop in $ \bigvee_{g_\alpha} S_{g_\alpha}^1 $. Attach $ 2 $-cells to $ \bigvee_{g_\alpha} S_{g_\alpha}^1 $ along the loops defined by the relations $ \cb{r_\beta} $. Call this new CW-complex $ Y $. Proposition \ref{prop:1.26} gives that
$$ \pi_1\rb{Y, x_0} \cong \dfrac{\pi_1\rb{X, x_0}}{\ab{\ab{\cb{r_\beta}}}} \cong \dfrac{F_{\cb{g_\alpha}}}{\ab{\ab{\cb{r_\beta}}}} \cong G. $$
\end{proof}

\begin{remark*}
Let $ X = \bigcup_n X^n $ be a CW-complex, path-connected. Proposition \ref{prop:1.26} can be used to show the following two facts.
\begin{itemize}
\item The inclusion $ X^1 \hookrightarrow X $ induces a surjective homomorphism $ \pi_1\rb{X^1} \to \pi_1\rb{X} $.
\item The inclusion $ X^2 \hookrightarrow X $ induces an isomorphism $ \pi_1\rb{X^2} \to \pi_1\rb{X} $.
\end{itemize}
\end{remark*}

\pagebreak

\subsection{Covering spaces}

\subsubsection{Lifting properties}

Let $ X $ be a topological space. Recall that a \textbf{covering space} is $ p : \widetilde{X} \to X $ such that each $ x \in X $ has an open neighbourhood $ U $ such that
$$ p^{-1}\rb{U} = \bigcup_\alpha \widetilde{U_\alpha}, $$
where $ U_\alpha $ are pairwise disjoint and $ p \mid_{\widetilde{U_\alpha}} : \widetilde{U_\alpha} \to U $ is a homeomorphism for all $ \alpha $.

\begin{example*}
$$ \function{\R}{S^1}{s}{\rb{\cos\rb{2\pi s}, \sin\rb{2\pi s}}}, \qquad \function{S^1}{S^1}{z}{z^n}, \qquad \text{O} \cdot \text{O} \cdot \text{O} \to S^1 \vee S^1 = \text{O} \cdot \text{O}. $$
\end{example*}

Let $ f : Y \to X $ be a continuous map. A \textbf{lift} of $ f $ is a continuous map $ \widetilde{f} : Y \to \widetilde{X} $ such that $ p\widetilde{f} = f $, where $ p : \widetilde{X} \to X $ is a covering space. Let $ Y $ be connected.
\begin{itemize}
\item \textbf{Unique lifting property} states that if two lifts $ \widetilde{f_1} $ and $ \widetilde{f_2} $ of $ f $ coincide at one point, then they coincide on all of $ Y $.
\item \textbf{Homotopy lifting property} states that if $ f_t : Y \to X $ is a homotopy and $ \widetilde{f_0} : Y \to \widetilde{X} $ is a lift of $ f_0 $ then there exists a unique homotopy $ \widetilde{f_t} : Y \to \widetilde{X} $ of $ \widetilde{f_0} $ that lifts $ f_t $.
\end{itemize}

\begin{remark*}
\hfill
\begin{itemize}
\item If $ Y $ is a point, this is called the \textbf{path lifting property}. Let $ f : I \to X $ be a path with $ f\rb{0} = x_0 $. If $ \widetilde{x_0} \in p^{-1}\rb{x_0} $, then there is a unique path $ \widetilde{f} : I \to \widetilde{X} $ lifting $ f $ and starting at $ \widetilde{x_0} $.
\item In particular, the lift of a constant path is constant.
\item This implies in particular that the lift of a homotopy of paths is again a homotopy of paths. The endpoints $ f_t\rb{0} $ and $ f_t\rb{1} $ define constant paths as $ t $ varies.
\end{itemize}
\end{remark*}

Fix $ x_0 \in X $ and $ \widetilde{x_0} \in \widetilde{X} $ such that $ p\rb{\widetilde{x_0}} = x_0 $, so
$$ p_* : \pi_1\rb{\widetilde{X}, \widetilde{x_0}} \to \pi_1\rb{X, x_0}. $$
To every element in $ \pi_1\rb{X, x_0} $ we can associate a homotopy class of paths in $ \widetilde{X} $ starting at $ \widetilde{x_0} $.

\begin{proposition}
\label{prop:1.31}
\hfill
\begin{enumerate}
\item $ p_* : \pi_1\rb{\widetilde{X}, \widetilde{x_0}} \to \pi_1\rb{X, x_0} $ is injective.
\item $ p_*\rb{\pi_1\rb{\widetilde{X}, \widetilde{x_0}}} \subseteq \pi_1\rb{X, x_0} $ consists of the homotopy classes of loops at $ x_0 $ whose lifts to $ \widetilde{X} $ starting at $ \widetilde{x_0} $ are loops.
\end{enumerate}
\end{proposition}

\begin{proof}
\hfill
\begin{enumerate}
\item Let $ \widetilde{f_0} : I \to \widetilde{X} $ be a loop at $ \widetilde{x_0} $ such that $ \sb{\widetilde{f_0}} \in Ker\rb{p_*} $, so $ p\widetilde{f_0} = f_0 $ is homotopic to the constant loop at $ x_0 $. Let $ f_t : I \to X $ be a homotopy between $ f_0 $ and the constant loop. Homotopy lifting property and remark gives that $ f_t $ lifts to a homotopy $ \widetilde{f_t} $ of paths between $ \widetilde{f_0} $ and the constant loop, so $ \sb{\widetilde{f_0}} = id \in \pi_1\rb{\widetilde{X}, \widetilde{x_0}} $ and $ p_* $ is injective.
\item Let $ f : I \to X $ be a loop at $ x_0 $ that lifts to a loop $ \widetilde{f} $ at $ \widetilde{x_0} $. Then $ p\widetilde{f} = f $, so $ p_*\rb{\sb{\widetilde{f}}} = \sb{f} $. On the other hand, if $ f : I \to X $ is a loop at $ x_0 $ such that there exists a loop $ \widetilde{f} : I \to \widetilde{X} $ at $ \widetilde{x_0} $ with $ p_*\rb{\sb{\widetilde{f}}} = \sb{f} $, then $ f $ is homotopic to $ p\widetilde{f} $. Homotopy lifting property gives that there exists a loop $ \widetilde{f'} : I \to \widetilde{X} $ at $ x_0 $ such that $ p\widetilde{f'} = f $.
\end{enumerate}
\end{proof}

\lecture{12}{Wednesday}{06/02/19}

Let $ p : \widetilde{X} \to X $ be a covering space. Let $ U \subseteq X $ be an evenly covered neighbourhood of $ x \in X $. Let
$$ p^{-1}\rb{U} = \bigsqcup_{\alpha \in \Lambda} \widetilde{U_\alpha}. $$
Then the cardinality $ \abs{p^{-1}\rb{x}} $ of $ p^{-1}\rb{x} $ is exactly the cardinality of $ \abs{\Lambda} $. The set of sheets is in bijection with $ p^{-1}\rb{x} $. So the cardinality of $ p^{-1}\rb{x} $ is locally constant. If $ X $ is connected, the cardinality of $ p^{-1}\rb{x} $ is constant.

\begin{notation*}
Let $ X, Y $ be topological spaces, $ x \in X $, and $ y \in Y $. A continuous map
$$ f : \rb{X, x} \to \rb{Y, y} $$
is a continuous map $ f : X \to Y $ such that $ f\rb{x} = y $.
\end{notation*}

\begin{proposition}
Let $ X, \widetilde{X} $ be path-connected and
$$ p : \rb{\widetilde{X}, \widetilde{x_0}} \to \rb{X, x_0} $$
be a covering space. Then the number of sheets of $ p $ equals the index of $ p_*\rb{\pi_1\rb{\widetilde{X}, \widetilde{x_0}}} $ in $ \pi_1\rb{X, x_0} $.
\end{proposition}

\begin{proof}
Let $ g $ be a loop in $ X $ at $ x_0 $ and $ \widetilde{g} $ be its lift to $ \widetilde{X} $ starting at $ \widetilde{x_0} $. Let $ H = p_*\rb{\pi_1\rb{\widetilde{X}, \widetilde{x_0}}} $ and let $ \sb{h} \in H $. Then $ h \cdot g $ lifts to a path $ \widetilde{h} \cdot \widetilde{g} $ in $ \widetilde{X} $ starting at $ \widetilde{x_0} $ with the same endpoint as $ \widetilde{g} $, because $ \widetilde{h} $ is a loop, by Proposition \ref{prop:1.31}. Define
$$ \function[\Phi]{\cb{\text{cosets of} \ H \ \text{in} \ \pi_1\rb{X, x_0}}}{p^{-1}\rb{x_0}}{H\sb{g}}{\widetilde{g}\rb{1}}, $$
so $ \Phi $ is well-defined. Want to show that $ \Phi $ is bijective.
\begin{itemize}
\item $ \Phi $ is surjective because $ \widetilde{X} $ is path-connected. Let $ \widetilde{g} $ be a path in $ \widetilde{X} $ from $ \widetilde{x_0} $ to any point $ \widetilde{x_0'} \in p^{-1}\rb{x_0} $, then $ g = p \cdot \widetilde{g} $ and $ \Phi\rb{H\sb{g}} = \widetilde{x_0'} $.
\item $ \Phi $ is injective, since if $ \Phi\rb{H\sb{g_1}} = \Phi\rb{H\sb{g_2}} $ then the lift $ \widetilde{g_1} \cdot \widetilde{g_2}^{-1} $ of $ g_1 \cdot g_2^{-1} $ defines a loop in $ \widetilde{X} $ at $ \widetilde{x_0} $. Proposition \ref{prop:1.31} gives $ \sb{g_1}\sb{g_2}^{-1} \in H $, so $ H\sb{g_1} = H\sb{g_2} $.
\end{itemize}
\end{proof}

We say that a topological space $ X $ has a certain property $ \rb{P} $ \textbf{locally} if for each point $ x \in X $ and each neighbourhood $ U $ of $ x $ there is an open neighbourhood $ V \subseteq U $ having this property $ \rb{P} $.

\begin{example*}
$ X $ is locally path-connected or $ X $ is locally simply-connected.
\end{example*}

\begin{proposition}
Let
$$ p : \rb{\widetilde{X}, \widetilde{x_0}} \to \rb{X, x_0} $$
be a covering space and
$$ f : \rb{Y, y_0} \to \rb{X, x_0} $$
a continuous map, where $ Y $ is path-connected and locally path-connected. Then there is a lift
$$ \widetilde{f} : \rb{Y, y_0} \to \rb{\widetilde{X}, \widetilde{x_0}} $$
if and only if $ f_*\rb{\pi_1\rb{Y, y_0}} \subseteq p_*\rb{\pi_1\rb{\widetilde{X}, \widetilde{x_0}}} $.
$$
\begin{tikzcd}
& \rb{\widetilde{X}, \widetilde{x_0}} \arrow{d}{p} \\
\rb{Y, y_0} \arrow{ur}{\widetilde{f}} \arrow[swap]{r}{f} & \rb{X, x_0}
\end{tikzcd}.
$$
\end{proposition}

\begin{proof}
\hfill
\begin{itemize}
\item[$ \implies $] Clear, because $ f = p\widetilde{f} $ implies $ f_* = p_*\widetilde{f_*} $.
\item[$ \impliedby $] Assume $ f_*\rb{\pi_1\rb{Y, y_0}} \subseteq p_*\rb{\pi_1\rb{\widetilde{X}, \widetilde{x_0}}} $. For each $ y \in Y $ choose a path $ \gamma $ from $ y_0 $ to $ y $, so $ f\gamma $ is a path in $ X $ from $ x_0 $ to $ f\rb{y} $. By path lifting, we can lift $ f\gamma $ to a path $ \widetilde{f\gamma} $ in $ \widetilde{X} $ starting at $ \widetilde{x_0} $. Define the map
$$ \function[\widetilde{f}]{\rb{Y, y_0}}{\rb{\widetilde{X}, \widetilde{x_0}}}{y}{\widetilde{f\gamma}\rb{1}}. $$
$$
\begin{tikzcd}
& \widetilde{x_0} \underset{\widetilde{f\gamma'}}{\overset{\widetilde{f\gamma}}{=}} \widetilde{f}\rb{y} \arrow{d}{p} \\
y_0 \underset{\gamma'}{\overset{\gamma}{=}} y \arrow{ur}{\widetilde{f}} \arrow[swap]{r}{f} & x_0 \underset{f\gamma'}{\overset{f\gamma}{=}} f\rb{y}
\end{tikzcd}.
$$
\begin{itemize}
\item This map is well-defined, that is does not depend on the choice of $ \gamma $. Let $ \gamma' $ be another path from $ y_0 $ to $ y $. Then $ h_0 = \rb{f\gamma'} \cdot \rb{f\gamma}^{-1} $ is a loop at $ x_0 $ and $ \sb{h_0} \in f_*\rb{\pi_1\rb{Y, y_0}} \subseteq p_*\rb{\pi_1\rb{\widetilde{X}, \widetilde{x_0}}} $. Proposition \ref{prop:1.31} gives that can lift $ h_0 $ to a loop $ \widetilde{h_0} $ at $ \widetilde{x_0} $. The first half of $ \widetilde{h_0} $ is $ \widetilde{f\gamma'} $ and the second half is $ \widetilde{f\gamma}^{-1} $, so $ \widetilde{f\gamma}\rb{1} = \widetilde{f\gamma'}\rb{1} $. Thus $ \widetilde{f} $ is well-defined.
\item We have $ p\widetilde{f} = f $, so $ \widetilde{f} $ lifts $ f $.
\item It remains to show that $ \widetilde{f} $ is continuous. Let $ y \in Y $ and let $ U $ be an evenly covered neighbourhood of $ f\rb{y} $. Let $ \widetilde{U} $ be the sheet above $ U $ such that $ \widetilde{f}\rb{y} \in \widetilde{U} $, so $ p \mid_{\widetilde{U}} : \widetilde{U} \to U $ is a homeomorphism. Let $ V \subseteq Y $ be a path-connected neighbourhood of $ y $ such that $ f\rb{V} \subseteq U $. Fix a path $ \gamma $ from $ y_0 $ to $ y $. Let $ y' \in V $ be arbitrary and $ \eta $ be a path from $ y $ to $ y' $, so $ \gamma \cdot \eta $ is a path from $ y_0 $ to $ y' $. Then $ \rb{f\gamma} \cdot \rb{f\eta} $ is a path in $ U $ from $ x_0 $ to $ f\rb{y'} $. $ \widetilde{f\eta} = \rb{p \mid_{\widetilde{U}}}^{-1}f\eta $, so $ \widetilde{f} \mid_V = \rb{p \mid_{\widetilde{U}}}^{-1}f $. Thus $ \widetilde{f} \mid_V : V \to \widetilde{U} $ is continuous, so $ \widetilde{f} $ is continuous.
\end{itemize}
\end{itemize}
\end{proof}

\lecture{13}{Friday}{08/02/19}

\subsubsection{The classification of covering spaces}

\begin{definition*}
A covering space $ p : \widetilde{X} \to X $ is a \textbf{universal cover} if $ \widetilde{X} $ is simply-connected.
\end{definition*}

\begin{definition*}
A topological space $ X $ is \textbf{semilocally simply-connected} if each $ x \in X $ has a neighbourhood $ U $ such that
$$ i_* : \pi_1\rb{U, x} \to \pi_1\rb{X, x} $$
is trivial, where $ i : U \hookrightarrow X $ is the inclusion.
\end{definition*}

\begin{example*}
Let $ X = \bigcup_n C_n \subseteq \R^2 $ be the Hawaiian earrings, where $ C_n \subseteq \R^2 $ is the circle of radius $ 1 / n $ and centre $ \rb{1 / n, 0} $. Then $ X $ is not semilocally simply-connected.
\end{example*}

\begin{proposition}
If $ p : \widetilde{X} \to X $ is a universal cover, then $ X $ is semilocally simply-connected.
\end{proposition}

\begin{proof}
Let $ U \subseteq X $ be an evenly covered neighbourhood of $ x_0 \in X $, $ \widetilde{U} \subseteq \widetilde{X} $ be a sheet over $ U $, and $ \gamma \subseteq U $ be a loop at $ x_0 $, so $ \gamma $ lifts to a loop $ \widetilde{\gamma} \subseteq \widetilde{U} $ at $ \widetilde{x_0} $. $ \widetilde{\gamma} $ is homotopic to the constant loop at $ \widetilde{x_0} $. Compose this homotopy with $ p $ gives that $ \gamma $ is homotopic to the constant loop at $ x_0 $ in $ X $, so
$$ \pi_1\rb{U, x_0} \to \pi_1\rb{X, x_0} $$
is trivial.
\end{proof}

\begin{theorem}
Let $ X $ be path-connected, locally path-connected, and semilocally simply-connected. Then there exists a universal cover $ p : \widetilde{X} \to X $.
\end{theorem}

\begin{remark*}
If
$$ p : \rb{\widetilde{X}, \widetilde{x_0}} \to \rb{X, x_0} $$
is a universal cover, each point $ \widetilde{x} \in \widetilde{X} $ can be joined to $ \widetilde{x_0} $ by a unique homotopy class of paths, by Proposition \ref{prop:1.6}.
\begin{align*}
\cb{\text{points in} \ \widetilde{X}} \qquad
& \leftrightsquigarrow \qquad \cb{\text{homotopy classes of paths in} \ \widetilde{X} \ \text{starting at} \ \widetilde{x_0}} \\
& \leftrightsquigarrow \qquad \cb{\text{homotopy classes of paths in} \ X \ \text{starting at} \ x_0},
\end{align*}
by the homotopy lifting property.
\end{remark*}

\begin{proof}
Let $ x_0 \in X $,
$$ \widetilde{X} = \cb{\sb{\gamma} \mid \gamma \ \text{is a path in} \ X \ \text{starting at} \ x_0}, \qquad \function[p]{\widetilde{X}}{X}{\sb{\gamma}}{\gamma\rb{1}}. $$
Have to
\begin{enumerate}
\item give $ \widetilde{X} $ a topology,
\item show that $ p : \widetilde{X} \to X $ is a covering, and
\item show that $ \widetilde{X} $ is simply-connected.
\end{enumerate}
Recall that a \textbf{basis} for a topology on a set $ Y $ is a collection $ \BB $ of subsets such that
\begin{itemize}
\item $ Y = \bigcup_{U \in \BB} U $, and
\item if $ U_1, U_2 \in \BB $ and $ y \in U_1 \cap U_2 $ then there exists $ V \in \BB $ such that $ y \in V $ and $ V \subseteq U_1 \cap U_2 $.
\end{itemize}
A basis defines a topology on $ Y $, by $ A \subseteq Y $ is open if and only if $ A $ is the union of elements of $ \BB $. A map $ f : Z \to Y $ is continuous if and only if $ f^{-1}\rb{U} $ is open for all $ U \in \BB $.
\begin{enumerate}
\item Let $ \UU $ be the collection of all path-connected open sets $ U \subseteq X $ such that $ \pi_1\rb{U} \to \pi_1\rb{X} $ is trivial. Then $ X = \bigcup_{U \in \UU} U $ because $ X $ is semilocally simply-connected. Let $ U_1, U_2 \in \UU $ and $ y \in U_1 \cap U_2 $, and let $ y \in V \subseteq U_1 \cap U_2 $ be path-connected and open.
$$
\begin{tikzcd}[row sep=tiny]
V \arrow[hookrightarrow]{r} & U_1 \arrow[hookrightarrow]{r} & X \\
\pi_1\rb{V} \arrow{r} \arrow[bend right=15, swap]{rr}{\text{trivial}} & \pi_1\rb{U_1} \arrow{r}{\text{trivial}} & \pi_1\rb{X}
\end{tikzcd},
$$
so $ V \in \UU $ gives that $ \UU $ is a basis for the topology on $ X $. For $ U \in \UU $ and $ \gamma $ a path in $ X $ from $ x_0 $ to a point in $ U $, we define
$$ U_{\sb{\gamma}} = \cb{\sb{\gamma \cdot \eta} \mid \eta \ \text{a path in} \ U \ \text{such that} \ \eta\rb{0} = \eta\rb{1}} \subseteq \widetilde{X}. $$
$ U_{\sb{\gamma}} $ only depends on the class $ \sb{\gamma} $, so $ p \mid_{U_{\sb{\gamma}}} : U_{\sb{\gamma}} \to U $ is bijective. Surjective because $ U $ is path-connected and injective because all paths $ \eta $ in $ U $ with the same endpoint are homotopic. Claim that $ \cb{U_{\sb{\gamma}}} $ forms a basis on $ \widetilde{X} $.
\begin{itemize}
\item $ \bigcup_{U \in \UU, \ \gamma} U_{\sb{\gamma}} = \widetilde{X} $, because $ \bigcup_{U \in \UU} U = X $.
\item Observe that if $ \sb{\gamma'} \in U_{\sb{\gamma}} $ then $ U_{\sb{\gamma}} = U_{\sb{\gamma'}} $. If $ \gamma' = \gamma \cdot \eta $ for $ \eta $ a path in $ U $, then elements in $ U_{\sb{\gamma'}} $ have the form $ \sb{\gamma \cdot \eta \cdot \mu} $, so $ U_{\sb{\gamma'}} \subseteq U_{\sb{\gamma}} $. Elements in $ U_{\sb{\gamma}} $ have the form $ \sb{\gamma \cdot \mu} = \sb{\gamma \cdot \eta \cdot \eta^{-1} \cdot \mu} = \sb{\gamma' \cdot \eta^{-1} \cdot \mu} $, so $ U_{\sb{\gamma}} \subseteq U_{\sb{\gamma'}} $. Consider $ U_{\sb{\gamma}} $ and $ V_{\sb{\gamma'}} $ and let $ \sb{\gamma''} \in U_{\sb{\gamma}} \cap V_{\sb{\gamma'}} $, so $ U_{\sb{\gamma}} = U_{\sb{\gamma''}} $ and $ V_{\sb{\gamma'}} = V_{\sb{\gamma''}} $. Let $ W \in \UU $ such that $ W \subseteq U \cap V $ and such that $ \gamma''\rb{1} \in W $, so $ W_{\sb{\gamma''}} \subseteq U_{\sb{\gamma''}} \cap V_{\sb{\gamma''}} $ and $ \sb{\gamma''} \in W_{\sb{\gamma''}} $. This proves the claim.
\end{itemize}
\item $ p \mid_{U_{\sb{\gamma}}} : U_{\sb{\gamma}} \to U $ is a homeomorphism. It is bijective, let $ V_{\sb{\gamma'}} \subseteq U_{\sb{\gamma}} $ be an element of the basis, so $ p\rb{V_{\sb{\gamma'}}} = V \in \UU $. $ p^{-1}\rb{V} \cap U_{\sb{\gamma}} = V_{\sb{\gamma'}} $. Thus $ p : \widetilde{X} \to X $ is continuous. If $ U \in \UU $, then $ p^{-1}\rb{U} = \bigsqcup_{\sb{\gamma}} U_{\sb{\gamma}} $, so $ p : \widetilde{X} \to X $ is a covering space.
\end{enumerate}
\end{proof}

\end{document}