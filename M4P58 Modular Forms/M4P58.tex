\def\module{M4P58 Modular Forms}
\def\lecturer{Dr David Helm}
\def\term{Autumn 2019}
\def\cover{}
\def\syllabus{}
\def\thm{subsection}

\documentclass{article}

% Packages

\usepackage{amssymb}
\usepackage{amsthm}
\usepackage[UKenglish]{babel}
\usepackage{commath}
\usepackage{enumitem}
\usepackage{etoolbox}
\usepackage{fancyhdr}
\usepackage[margin=1in]{geometry}
\usepackage{graphicx}
\usepackage[hidelinks]{hyperref}
\usepackage[utf8]{inputenc}
\usepackage{listings}
\usepackage{mathtools}
\usepackage{stmaryrd}
\usepackage{tikz-cd}
\usepackage{csquotes}

% Formatting

\addto\captionsUKenglish{\renewcommand{\abstractname}{Syllabus}}
\delimitershortfall5pt
\ifx\thm\undefined\newtheorem{n}{}\else\newtheorem{n}{}[\thm]\fi
\newcommand\newoperator[1]{\ifcsdef{#1}{\cslet{#1}{\relax}}{}\csdef{#1}{\operatorname{#1}}}
\setlength{\parindent}{0cm}

% Environments

\theoremstyle{plain}
\newtheorem{algorithm}[n]{Algorithm}
\newtheorem*{algorithm*}{Algorithm}
\newtheorem{algorithm**}{Algorithm}
\newtheorem{conjecture}[n]{Conjecture}
\newtheorem*{conjecture*}{Conjecture}
\newtheorem{conjecture**}{Conjecture}
\newtheorem{corollary}[n]{Corollary}
\newtheorem*{corollary*}{Corollary}
\newtheorem{corollary**}{Corollary}
\newtheorem{lemma}[n]{Lemma}
\newtheorem*{lemma*}{Lemma}
\newtheorem{lemma**}{Lemma}
\newtheorem{proposition}[n]{Proposition}
\newtheorem*{proposition*}{Proposition}
\newtheorem{proposition**}{Proposition}
\newtheorem{theorem}[n]{Theorem}
\newtheorem*{theorem*}{Theorem}
\newtheorem{theorem**}{Theorem}

\theoremstyle{definition}
\newtheorem{aim}[n]{Aim}
\newtheorem*{aim*}{Aim}
\newtheorem{aim**}{Aim}
\newtheorem{axiom}[n]{Axiom}
\newtheorem*{axiom*}{Axiom}
\newtheorem{axiom**}{Axiom}
\newtheorem{condition}[n]{Condition}
\newtheorem*{condition*}{Condition}
\newtheorem{condition**}{Condition}
\newtheorem{definition}[n]{Definition}
\newtheorem*{definition*}{Definition}
\newtheorem{definition**}{Definition}
\newtheorem{example}[n]{Example}
\newtheorem*{example*}{Example}
\newtheorem{example**}{Example}
\newtheorem{exercise}[n]{Exercise}
\newtheorem*{exercise*}{Exercise}
\newtheorem{exercise**}{Exercise}
\newtheorem{fact}[n]{Fact}
\newtheorem*{fact*}{Fact}
\newtheorem{fact**}{Fact}
\newtheorem{goal}[n]{Goal}
\newtheorem*{goal*}{Goal}
\newtheorem{goal**}{Goal}
\newtheorem{law}[n]{Law}
\newtheorem*{law*}{Law}
\newtheorem{law**}{Law}
\newtheorem{plan}[n]{Plan}
\newtheorem*{plan*}{Plan}
\newtheorem{plan**}{Plan}
\newtheorem{problem}[n]{Problem}
\newtheorem*{problem*}{Problem}
\newtheorem{problem**}{Problem}
\newtheorem{question}[n]{Question}
\newtheorem*{question*}{Question}
\newtheorem{question**}{Question}
\newtheorem{warning}[n]{Warning}
\newtheorem*{warning*}{Warning}
\newtheorem{warning**}{Warning}
\newtheorem{acknowledgements}[n]{Acknowledgements}
\newtheorem*{acknowledgements*}{Acknowledgements}
\newtheorem{acknowledgements**}{Acknowledgements}
\newtheorem{annotations}[n]{Annotations}
\newtheorem*{annotations*}{Annotations}
\newtheorem{annotations**}{Annotations}
\newtheorem{assumption}[n]{Assumption}
\newtheorem*{assumption*}{Assumption}
\newtheorem{assumption**}{Assumption}
\newtheorem{conclusion}[n]{Conclusion}
\newtheorem*{conclusion*}{Conclusion}
\newtheorem{conclusion**}{Conclusion}
\newtheorem{claim}[n]{Claim}
\newtheorem*{claim*}{Claim}
\newtheorem{claim**}{Claim}
\newtheorem{notation}[n]{Notation}
\newtheorem*{notation*}{Notation}
\newtheorem{notation**}{Notation}
\newtheorem{note}[n]{Note}
\newtheorem*{note*}{Note}
\newtheorem{note**}{Note}
\newtheorem{remark}[n]{Remark}
\newtheorem*{remark*}{Remark}
\newtheorem{remark**}{Remark}

% Lectures

\newcommand{\lecture}[3]{ % Lecture
  \marginpar{
    Lecture #1 \\
    #2 \\
    #3
  }
}

% Blackboard

\renewcommand{\AA}{\mathbb{A}} % Blackboard A
\newcommand{\BB}{\mathbb{B}}   % Blackboard B
\newcommand{\CC}{\mathbb{C}}   % Blackboard C
\newcommand{\DD}{\mathbb{D}}   % Blackboard D
\newcommand{\EE}{\mathbb{E}}   % Blackboard E
\newcommand{\FF}{\mathbb{F}}   % Blackboard F
\newcommand{\GG}{\mathbb{G}}   % Blackboard G
\newcommand{\HH}{\mathbb{H}}   % Blackboard H
\newcommand{\II}{\mathbb{I}}   % Blackboard I
\newcommand{\JJ}{\mathbb{J}}   % Blackboard J
\newcommand{\KK}{\mathbb{K}}   % Blackboard K
\newcommand{\LL}{\mathbb{L}}   % Blackboard L
\newcommand{\MM}{\mathbb{M}}   % Blackboard M
\newcommand{\NN}{\mathbb{N}}   % Blackboard N
\newcommand{\OO}{\mathbb{O}}   % Blackboard O
\newcommand{\PP}{\mathbb{P}}   % Blackboard P
\newcommand{\QQ}{\mathbb{Q}}   % Blackboard Q
\newcommand{\RR}{\mathbb{R}}   % Blackboard R
\renewcommand{\SS}{\mathbb{S}} % Blackboard S
\newcommand{\TT}{\mathbb{T}}   % Blackboard T
\newcommand{\UU}{\mathbb{U}}   % Blackboard U
\newcommand{\VV}{\mathbb{V}}   % Blackboard V
\newcommand{\WW}{\mathbb{W}}   % Blackboard W
\newcommand{\XX}{\mathbb{X}}   % Blackboard X
\newcommand{\YY}{\mathbb{Y}}   % Blackboard Y
\newcommand{\ZZ}{\mathbb{Z}}   % Blackboard Z

% Brackets

\renewcommand{\eval}[1]{\left. #1 \right|}          % Evaluation
\newcommand{\br}{\del}                              % Brackets
\newcommand{\abr}[1]{\left\langle #1 \right\rangle} % Angle brackets
\newcommand{\fbr}[1]{\left\lfloor #1 \right\rfloor} % Floor brackets
\newcommand{\lbr}[1]{\left\lfloor #1 \right\rfloor} % Ceiling brackets
\newcommand{\st}{\ \middle| \ }                     % Such that

% Calligraphic

\newcommand{\AAA}{\mathcal{A}} % Calligraphic A
\newcommand{\BBB}{\mathcal{B}} % Calligraphic B
\newcommand{\CCC}{\mathcal{C}} % Calligraphic C
\newcommand{\DDD}{\mathcal{D}} % Calligraphic D
\newcommand{\EEE}{\mathcal{E}} % Calligraphic E
\newcommand{\FFF}{\mathcal{F}} % Calligraphic F
\newcommand{\GGG}{\mathcal{G}} % Calligraphic G
\newcommand{\HHH}{\mathcal{H}} % Calligraphic H
\newcommand{\III}{\mathcal{I}} % Calligraphic I
\newcommand{\JJJ}{\mathcal{J}} % Calligraphic J
\newcommand{\KKK}{\mathcal{K}} % Calligraphic K
\newcommand{\LLL}{\mathcal{L}} % Calligraphic L
\newcommand{\MMM}{\mathcal{M}} % Calligraphic M
\newcommand{\NNN}{\mathcal{N}} % Calligraphic N
\newcommand{\OOO}{\mathcal{O}} % Calligraphic O
\newcommand{\PPP}{\mathcal{P}} % Calligraphic P
\newcommand{\QQQ}{\mathcal{Q}} % Calligraphic Q
\newcommand{\RRR}{\mathcal{R}} % Calligraphic R
\newcommand{\SSS}{\mathcal{S}} % Calligraphic S
\newcommand{\TTT}{\mathcal{T}} % Calligraphic T
\newcommand{\UUU}{\mathcal{U}} % Calligraphic U
\newcommand{\VVV}{\mathcal{V}} % Calligraphic V
\newcommand{\WWW}{\mathcal{W}} % Calligraphic W
\newcommand{\XXX}{\mathcal{X}} % Calligraphic X
\newcommand{\YYY}{\mathcal{Y}} % Calligraphic Y
\newcommand{\ZZZ}{\mathcal{Z}} % Calligraphic Z

% Fraktur

\newcommand{\aaa}{\mathfrak{a}}   % Fraktur a
\newcommand{\bbb}{\mathfrak{b}}   % Fraktur b
\newcommand{\ccc}{\mathfrak{c}}   % Fraktur c
\newcommand{\ddd}{\mathfrak{d}}   % Fraktur d
\newcommand{\eee}{\mathfrak{e}}   % Fraktur e
\newcommand{\fff}{\mathfrak{f}}   % Fraktur f
\renewcommand{\ggg}{\mathfrak{g}} % Fraktur g
\newcommand{\hhh}{\mathfrak{h}}   % Fraktur h
\newcommand{\iii}{\mathfrak{i}}   % Fraktur i
\newcommand{\jjj}{\mathfrak{j}}   % Fraktur j
\newcommand{\kkk}{\mathfrak{k}}   % Fraktur k
\renewcommand{\lll}{\mathfrak{l}} % Fraktur l
\newcommand{\mmm}{\mathfrak{m}}   % Fraktur m
\newcommand{\nnn}{\mathfrak{n}}   % Fraktur n
\newcommand{\ooo}{\mathfrak{o}}   % Fraktur o
\newcommand{\ppp}{\mathfrak{p}}   % Fraktur p
\newcommand{\qqq}{\mathfrak{q}}   % Fraktur q
\newcommand{\rrr}{\mathfrak{r}}   % Fraktur r
\newcommand{\sss}{\mathfrak{s}}   % Fraktur s
\newcommand{\ttt}{\mathfrak{t}}   % Fraktur t
\newcommand{\uuu}{\mathfrak{u}}   % Fraktur u
\newcommand{\vvv}{\mathfrak{v}}   % Fraktur v
\newcommand{\www}{\mathfrak{w}}   % Fraktur w
\newcommand{\xxx}{\mathfrak{x}}   % Fraktur x
\newcommand{\yyy}{\mathfrak{y}}   % Fraktur y
\newcommand{\zzz}{\mathfrak{z}}   % Fraktur z

% Geometry

\newcommand{\CP}{\mathbb{CP}}                                              % Complex projective space
\newcommand{\iintd}[4]{\iint_{#1} \, #2 \, \dif #3 \, \dif #4}             % Double integral
\newcommand{\RP}{\mathbb{RP}}                                              % Real projective space
\newcommand{\intd}[4]{\int_{#1}^{#2} \, #3 \, \dif #4}                     % Single integral
\newcommand{\iiintd}[5]{\iint_{#1} \, #2 \, \dif #3 \, \dif #4 \, \dif #5} % Triple integral

% Logic

\newcommand{\iffb}[2]{\br{#1 \leftrightarrow #2}} % Biconditional
\newcommand{\andb}[2]{\br{#1 \land #2}}           % Conjunction
\newcommand{\orb}[2]{\br{#1 \lor #2}}             % Disjunction
\newcommand{\nib}[2]{\br{#1 \notin #2}}           % Element of
\newcommand{\eqb}[2]{\br{#1 = #2}}                % Equal to
\newcommand{\teb}[1]{\br{\exists #1}}             % Existential quantifier
\newcommand{\impb}[2]{\br{#1 \rightarrow #2}}     % Implication
\newcommand{\ltb}[2]{\br{#1 < #2}}                % Less than
\newcommand{\leb}[2]{\br{#1 \le #2}}              % Less than or equal to
\newcommand{\notb}[1]{\br{\neg #1}}               % Negation
\newcommand{\inb}[2]{\br{#1 \in #2}}              % Not element of
\newcommand{\neb}[2]{\br{#1 \ne #2}}              % Not equal to
\newcommand{\subb}[2]{\br{#1 \subseteq #2}}       % Subset
\newcommand{\fab}[1]{\br{\forall #1}}             % Universal quantifier

% Maps

\newcommand{\bijection}[7][]{    % Bijection
  \ifx &#1&
    \begin{array}{rcl}
      #2 & \longleftrightarrow & #3 \\
      #4 & \longmapsto         & #5 \\
      #6 & \longmapsfrom       & #7
    \end{array}
  \else
    \begin{array}{ccrcl}
      #1 & : & #2 & \longrightarrow & #3 \\
         &   & #4 & \longmapsto     & #5 \\
         &   & #6 & \longmapsfrom   & #7
    \end{array}
  \fi
}
\newcommand{\birational}[7][]{   % Birational map
  \ifx &#1&
    \begin{array}{rcl}
      #2 & \dashrightarrow & #3 \\
      #4 & \longmapsto     & #5 \\
      #6 & \longmapsfrom   & #7
    \end{array}
  \else
    \begin{array}{ccrcl}
      #1 & : & #2 & \dashrightarrow & #3 \\
         &   & #4 & \longmapsto     & #5 \\
         &   & #6 & \longmapsfrom   & #7
    \end{array}
  \fi
}
\newcommand{\correspondence}[2]{ % Correspondence
  \cbr{
    \begin{array}{c}
      #1
    \end{array}
  }
  \qquad
  \leftrightsquigarrow
  \qquad
  \cbr{
    \begin{array}{c}
      #2
    \end{array}
  }
}
\newcommand{\function}[5][]{     % Function
  \ifx &#1&
    \begin{array}{rcl}
      #2 & \longrightarrow & #3 \\
      #4 & \longmapsto     & #5
    \end{array}
  \else
    \begin{array}{ccrcl}
      #1 & : & #2 & \longrightarrow & #3 \\
         &   & #4 & \longmapsto     & #5
    \end{array}
  \fi
}
\newcommand{\functions}[7][]{    % Functions
  \ifx &#1&
    \begin{array}{rcl}
      #2 & \longrightarrow & #3 \\
      #4 & \longmapsto     & #5 \\
      #6 & \longmapsto     & #7
    \end{array}
  \else
    \begin{array}{ccrcl}
      #1 & : & #2 & \longrightarrow & #3 \\
         &   & #4 & \longmapsto     & #5 \\
         &   & #6 & \longmapsto     & #7
    \end{array}
  \fi
}
\newcommand{\rational}[5][]{     % Rational map
  \ifx &#1&
    \begin{array}{rcl}
      #2 & \dashrightarrow & #3 \\
      #4 & \longmapsto     & #5
    \end{array}
  \else
    \begin{array}{ccrcl}
      #1 & : & #2 & \dashrightarrow & #3 \\
         &   & #4 & \longmapsto     & #5
    \end{array}
  \fi
}

% Matrices

\newcommand{\onebytwo}[2]{      % One by two matrix
  \begin{pmatrix}
    #1 & #2
  \end{pmatrix}
}
\newcommand{\onebythree}[3]{    % One by three matrix
  \begin{pmatrix}
    #1 & #2 & #3
  \end{pmatrix}
}
\newcommand{\twobyone}[2]{      % Two by one matrix
  \begin{pmatrix}
    #1 \\
    #2
  \end{pmatrix}
}
\newcommand{\twobytwo}[4]{      % Two by two matrix
  \begin{pmatrix}
    #1 & #2 \\
    #3 & #4
  \end{pmatrix}
}
\newcommand{\threebyone}[3]{    % Three by one matrix
  \begin{pmatrix}
    #1 \\
    #2 \\
    #3
  \end{pmatrix}
}
\newcommand{\threebythree}[9]{  % Three by three matrix
  \begin{pmatrix}
    #1 & #2 & #3 \\
    #4 & #5 & #6 \\
    #7 & #8 & #9
  \end{pmatrix}
}
\newcommand{\twobytwosmall}[4]{ % Two by two small matrix
  \begin{psmallmatrix}
    #1 & #2 \\
    #3 & #4
  \end{psmallmatrix}
}

% Number theory

\renewcommand{\symbol}[2]{\br{\tfrac{#1}{#2}}} % Power residue symbol
\newcommand{\unit}[1]{\br{\ZZ / #1\ZZ}^\times} % Unit group

% Operators

\newoperator{ab}    % Abelian
\newoperator{AG}    % Affine geometry
\newoperator{alg}   % Algebraic
\newoperator{Ann}   % Annihilator
\newoperator{area}  % Area
\newoperator{Aut}   % Automorphism
\newoperator{card}  % Cardinality
\newoperator{ch}    % Characteristic
\newoperator{Cl}    % Class
\newoperator{Coker} % Cokernel
\newoperator{col}   % Column
\newoperator{Corr}  % Correspondence
\newoperator{diam}  % Diameter
\newoperator{Disc}  % Discriminant
\newoperator{dom}   % Domain
\newoperator{Eig}   % Eigenvalue
\newoperator{Em}    % Embedding
\newoperator{End}   % Endomorphism
\newoperator{fin}   % Finite
\newoperator{Fix}   % Fixed
\newoperator{Frac}  % Fraction
\newoperator{Frob}  % Frobenius
\newoperator{Fun}   % Function
\newoperator{Gal}   % Galois
\newoperator{GL}    % General linear
\newoperator{Ham}   % Hamming
\newoperator{Homeo} % Homeomorphism
\newoperator{Hom}   % Homomorphism
\newoperator{id}    % Identity
\newoperator{Im}    % Image
\newoperator{Ind}   % Index
\newoperator{Ker}   % Kernel
\newoperator{lcm}   % Least common multiple
\newoperator{Mat}   % Matrix
\newoperator{mult}  % Multiplicity
\newoperator{new}   % New
\newoperator{Nm}    % Norm
\newoperator{old}   % Old
\newoperator{op}    % Opposite
\newoperator{ord}   % Order
\newoperator{Pay}   % Payley
\newoperator{PG}    % Projective geometry
\newoperator{PGL}   % Projective general linear
\newoperator{PSL}   % Projective special linear
\newoperator{rad}   % Radical
\newoperator{ran}   % Range
\newoperator{Res}   % Residue
\newoperator{rk}    % Rank
\newoperator{Re}    % Real
\newoperator{row}   % Row
\newoperator{sgn}   % Sign
\newoperator{Sing}  % Singular
\newoperator{SK}    % Skeleton
\newoperator{sp}    % Span
\newoperator{SL}    % Special linear
\newoperator{SO}    % Special orthogonal
\newoperator{Spec}  % Spectrum
\newoperator{Stab}  % Stabiliser
\newoperator{star}  % Star
\newoperator{srg}   % Strongly regular graph
\newoperator{supp}  % Support
\newoperator{Sym}   % Symmetric
\newoperator{tors}  % Torsion
\newoperator{Tr}    % Trace
\newoperator{vol}   % Volume
\newoperator{wt}    % Weight

% Roman

\newcommand{\A}{\mathrm{A}}   % Roman A
\newcommand{\B}{\mathrm{B}}   % Roman B
\newcommand{\C}{\mathrm{C}}   % Roman C
\newcommand{\D}{\mathrm{D}}   % Roman D
\newcommand{\E}{\mathrm{E}}   % Roman E
\newcommand{\F}{\mathrm{F}}   % Roman F
\newcommand{\G}{\mathrm{G}}   % Roman G
\renewcommand{\H}{\mathrm{H}} % Roman H
\newcommand{\I}{\mathrm{I}}   % Roman I
\newcommand{\J}{\mathrm{J}}   % Roman J
\newcommand{\K}{\mathrm{K}}   % Roman K
\renewcommand{\L}{\mathrm{L}} % Roman L
\newcommand{\M}{\mathrm{M}}   % Roman M
\newcommand{\N}{\mathrm{N}}   % Roman N
\renewcommand{\O}{\mathrm{O}} % Roman O
\renewcommand{\P}{\mathrm{P}} % Roman P
\newcommand{\Q}{\mathrm{Q}}   % Roman Q
\newcommand{\R}{\mathrm{R}}   % Roman R
\renewcommand{\S}{\mathrm{S}} % Roman S
\newcommand{\T}{\mathrm{T}}   % Roman T
\newcommand{\U}{\mathrm{U}}   % Roman U
\newcommand{\V}{\mathrm{V}}   % Roman V
\newcommand{\W}{\mathrm{W}}   % Roman W
\newcommand{\X}{\mathrm{X}}   % Roman X
\newcommand{\Y}{\mathrm{Y}}   % Roman Y
\newcommand{\Z}{\mathrm{Z}}   % Roman Z

\renewcommand{\a}{\mathrm{a}} % Roman a
\renewcommand{\b}{\mathrm{b}} % Roman b
\renewcommand{\c}{\mathrm{c}} % Roman c
\renewcommand{\d}{\mathrm{d}} % Roman d
\newcommand{\e}{\mathrm{e}}   % Roman e
\newcommand{\f}{\mathrm{f}}   % Roman f
\newcommand{\g}{\mathrm{g}}   % Roman g
\newcommand{\h}{\mathrm{h}}   % Roman h
\renewcommand{\i}{\mathrm{i}} % Roman i
\renewcommand{\j}{\mathrm{j}} % Roman j
\renewcommand{\k}{\mathrm{k}} % Roman k
\renewcommand{\l}{\mathrm{l}} % Roman l
\newcommand{\m}{\mathrm{m}}   % Roman m
\renewcommand{\n}{\mathrm{n}} % Roman n
\renewcommand{\o}{\mathrm{o}} % Roman o
\newcommand{\p}{\mathrm{p}}   % Roman p
\newcommand{\q}{\mathrm{q}}   % Roman q
\renewcommand{\r}{\mathrm{r}} % Roman r
\newcommand{\s}{\mathrm{s}}   % Roman s
\renewcommand{\t}{\mathrm{t}} % Roman t
\renewcommand{\u}{\mathrm{u}} % Roman u
\renewcommand{\v}{\mathrm{v}} % Roman v
\newcommand{\w}{\mathrm{w}}   % Roman w
\newcommand{\x}{\mathrm{x}}   % Roman x
\newcommand{\y}{\mathrm{y}}   % Roman y
\newcommand{\z}{\mathrm{z}}   % Roman z

% Tikz

\tikzset{
  arrow symbol/.style={"#1" description, allow upside down, auto=false, draw=none, sloped},
  subset/.style={arrow symbol={\subset}},
  cong/.style={arrow symbol={\cong}}
}

% Fancy header

\pagestyle{fancy}
\lhead{\module}
\rhead{\nouppercase{\leftmark}}

% Make title

\title{\module}
\author{Lectured by \lecturer \\ Typed by David Kurniadi Angdinata}
\date{\term}

\begin{document}

% Title page
\maketitle
\cover
\vfill
\begin{abstract}
\noindent\syllabus
\end{abstract}

\pagebreak

% Contents page
\tableofcontents

\pagebreak

% Document page
\setcounter{section}{-1}

\section{Introduction}

\lecture{1}{Friday}{04/10/19}

The following are textbooks.
\begin{itemize}
\item Serre, A course in arithmetic, 1973
\item J Shurman and F Diamond, A first course in modular forms, 2005
\end{itemize}

Let
$$ f = q\prod_{n = 1}^\infty \br{1 - q^n}^2\br{1 - q^{11n}}^2 = \sum_{n = 1} b_nq^n = q - 2q^2 - q^3 + 2q^4 + q^5 + 2q^6 - 2q^7 + \dots, $$
and let $ a_n $ be the number of solutions modulo $ n $ to the elliptic curve
$$ E = \cbr{\br{x, y} \in \ZZ \st y^2 + y = x^3 - x^2 - 10x - 20}. $$
\begin{itemize}
\item Modulo $ 2 $, there are $ a_2 = 4 $ solutions $ \br{0, 0} $, $ \br{0, 1} $, $ \br{1, 0} $, $ \br{1, 1} $.
\item Modulo $ 3 $, there are $ a_3 = 4 $ solutions $ \br{1, 0} $, $ \br{1, -1} $, $ \br{-1, 0} $, $ \br{-1, -1} $.
\item Modulo $ 5 $, there are $ a_5 = 4 $ solutions $ \br{0, 0} $, $ \br{0, -1} $, $ \br{1, 0} $, $ \br{-1, -1} $.
\item Modulo $ 7 $, there are $ a_7 = 9 $ solutions $ \br{1, 3} $, $ \br{2, 2} $, $ \br{2, -3} $, $ \br{-1, 1} $, $ \br{-1, -2} $, $ \br{-2, 1} $, $ \br{-2, -2} $, $ \br{-3, 1} $, $ \br{-3, -2} $.
\end{itemize}
If $ p \ne 11 $, then
$$ a_p - p = -b_p. $$
The following are some questions.
\begin{itemize}
\item What is the relationship between $ E $ and $ f $?
\item Can we find similar relationships for other $ E $?
\item How does one prove something like this?
\end{itemize}
Let
$$ \HH = \cbr{x + iy \st x, y \in \RR, \ y > 0} \subseteq \CC. $$
Then $ \HH $ has an action of
$$ \SL_2\br{\RR} = \cbr{\twobytwo{a}{b}{c}{d} \st a, b, c, d \in \RR, \ ad - bc = 1}. $$
Modular forms are complex functions on $ \HH $ with a high degree of symmetry. These functions are symmetric under the action of large discrete subgroups of $ \SL_2\br{\RR} $, in particular
$$ \SL_2\br{\ZZ} = \cbr{\twobytwo{a}{b}{c}{d} \st a, b, c, d \in \ZZ, \ ad - bc = 1} \subseteq \SL_2\br{\RR}. $$
Why are these interesting to number theorists? Power series expansions often involve expressions of interest to number theorists. For example,
\begin{itemize}
\item Bernoulli numbers,
\item divisor functions $ \sigma_k\br{n} = \sum_{d \mid n} d^k $,
\item number of points on elliptic curves, and
\item traces of Galois representations.
\end{itemize}

\pagebreak

\section{Modular forms of level one}

\subsection{Modular functions and forms}

\subsubsection{Modular actions}

Let
$$ \gamma = \twobytwo{a}{b}{c}{d}, \qquad a, b, c, d \in \RR. $$
Then $ \SL_2\br{\RR} $ acts on $ \CC \cup \cbr{\infty} $ by
$$ \gamma \cdot z =
\begin{cases}
\dfrac{az + b}{cz + d} & z \ne -\dfrac{d}{c} \\
\infty & z = -\dfrac{d}{c}
\end{cases}
\qquad \gamma \cdot \infty = \dfrac{a}{c}.
$$
One checks that this gives a bijection from $ \CC \cup \cbr{\infty} $ to $ \CC \cup \cbr{\infty} $, where inverse is given by the inverse matrix
$$ \gamma^{-1} = \twobytwo{d}{-b}{-c}{a}, $$
and $ \gamma \cdot \br{\gamma' \cdot z} = \gamma\gamma' \cdot z $. One obtains a left action of $ \SL_2\br{\RR} $ on $ \CC \cup \cbr{\infty} $. An observation is
$$ \Im \gamma z = \Im \dfrac{az + b}{cz + d} = \Im \dfrac{\br{az + b}\br{c\overline{z} + d}}{\abs{cz + d}^2} = \dfrac{\Im \br{az + b}\br{c\overline{z} + d}}{\abs{cz + d}^2} = \dfrac{\br{ad - bc}\Im z}{\abs{cz + d}^2}. $$

\lecture{2}{Friday}{04/10/19}

In particular, if $ \gamma \in \SL_2\br{\RR} $, then
$$ \Im \gamma z = \dfrac{\Im z}{\abs{cz + d}^2}. $$
So $ \SL_2\br{\RR} $ preserves $ \HH \cup \cbr{\infty} $. More generally, if $ \gamma \in \GL_2\br{\RR} $, then
$$ \Im \gamma z = \dfrac{\det \gamma\Im z}{\abs{cz + d}^2}. $$
So
$$ \GL_2\br{\RR}_+ = \cbr{\gamma \in \GL_2\br{\RR} \st \det \gamma > 0} $$
preserves $ \HH \cup \cbr{\infty} $. Define
$$ \function[\eval{f}_{k, \gamma}]{\HH}{\CC}{z}{\det \gamma^{k - 1}f\br{\gamma z}\br{cz + d}^{-k}}, \qquad f : \HH \to \CC, \qquad \gamma \in \GL_2\br{\RR}_+, \qquad k \in \ZZ, $$
where $ \det \gamma^{k - 1} $ is the fudge factor, which is one for $ \gamma \in \SL_2\br{\RR} $, and $ \br{cz + d}^{-k} $ is the twisted action on functions. Check that
$$ \eval{f}_{k, \id} = f, \qquad \eval{\br{\eval{f}_{k, \gamma}}}_{k, \gamma'} = \eval{f}_{k, \gamma'\gamma}. $$
This gives, for each $ k $, a left action of $ \GL_2\br{\RR}_+ $ on functions $ \HH \to \CC $, a \textbf{modular action of weight $ k $}. A modular form of weight $ k $ will be a sufficiently nice function $ f : \HH \to \CC $ such that $ \eval{f}_{k, \gamma} = f $ for all $ \gamma \in \SL_2\br{\ZZ} $. That is, for all $ \gamma \in \SL_2\br{\ZZ} $ and all $ z \in \HH $,
$$ f\br{\gamma z}\br{cz + d}^{-k} = f\br{z}, \qquad \implies \qquad f\br{\gamma z} = f\br{z}\br{cz + d}^k, $$
the \textbf{modular transformation law of weight $ k $}. The following are some observations.
\begin{itemize}
\item Let $ k = 0 $. Then constant functions satisfy $ f\br{\gamma z} = f\br{z} $. It will turn out that all functions of weight zero are constant.
\item Let $ k $ be odd, and $ \gamma = -\id $. Then $ \gamma z = z $ for all $ z $ and $ cz + d = -1 $, so $ f\br{\gamma z} = f\br{z}\br{cz + d}^k $ gives $ f\br{z} = f\br{z}\br{-1}^k $, so $ f\br{z} = -f\br{z} $, so $ f\br{z} = 0 $ for all $ z $. So no functions $ f : \HH \to \CC $ satisfy the modular transformation law of weight $ k $, for all $ \gamma \in \SL_2\br{\ZZ} $, when $ k $ is odd.
\end{itemize}

\pagebreak

\subsubsection{Review of complex analysis}

Let $ f : U \to \CC $, for $ U \subseteq \CC $ open, and let $ p \in U $.

\begin{definition}
$ f $ is \textbf{holomorphic} at $ p $ if
$$ f'\br{p'} = \lim_{\epsilon \to 0, \ \epsilon \in \CC} \dfrac{f\br{p' + \epsilon} - f\br{p'}}{\epsilon} $$
exists for all $ p' $ in a neighbourhood of $ p $.
\end{definition}

\begin{proposition}
$ f $ is holomorphic at $ p $ implies that $ f $ is continuous.
\end{proposition}

\begin{proposition}
$ f $ is holomorphic at $ p $ implies that $ f $ is infinitely differentiable at $ p $, that is $ f^{\br{n}}\br{p} $ exists for all $ n \ge 0 $. Moreover, we have
$$ f\br{z} = \sum_{n = 0}^\infty \dfrac{f^{\br{n}}\br{p}}{n!}\br{z - p}^n = f\br{p} + f'\br{p}\br{z - p} + \dfrac{f'\br{p}}{2}\br{z - p}^2 + \dots, $$
for all $ z $ in a neighbourhood of $ p $.
\end{proposition}

\begin{corollary}
If $ f $ is holomorphic and not identically zero on an open set $ U $, then the zeroes of $ f $ are isolated on $ U $.
\end{corollary}

More generally is the following.

\begin{definition}
$ f $ is \textbf{meromorphic} at $ p $ if there exists a neighbourhood $ U $ of $ p $ and $ g, h : U \to \CC $ holomorphic on $ U $ such that $ f = g / h $ on $ U \setminus \cbr{p} $. Such an $ f $ has a \textbf{Laurent series expansion} at $ p $,
$$ f\br{z} = \sum_{i = -N}^\infty c_i\br{z - p}^i. $$
The smallest $ i $ such that $ c_i \ne 0 $ is denoted by $ \ord_p f $, the \textbf{order of vanishing} of $ f $ at $ p $.
\begin{itemize}
\item If $ \ord_p f = -n $ for $ n > 0 $, we say $ f $ has a \textbf{pole of order $ n $}.
\item If $ \ord_p f = n $ for $ n > 0 $, we say $ f $ has a \textbf{zero of order $ n $}.
\end{itemize}
\end{definition}

\begin{proposition}
\hfill
\begin{itemize}
\item $ \ord_p fg = \ord_p f + \ord_p g $.
\item $ \ord_p \br{f + g} \ge \min\cbr{\ord_p f, \ord_p g} $, with equality if $ \ord_p f \ne \ord_p g $.
\end{itemize}
\end{proposition}

If $ f $ is holomorphic on $ U \setminus \cbr{p} $ for $ U $ a neighbourhood of $ p $, then $ f $ may or may not be meromorphic at $ p $.

\begin{example*}
$ f\br{z} = e^{-1 / z^2} $ is holomorphic on $ \CC \setminus \cbr{0} $, but not meromorphic at zero.
\end{example*}

\begin{theorem}
Let $ f $ be holomorphic on $ U \setminus \cbr{p} $, and there exists $ n > 0 $ such that
$$ \lim_{x \to p} \br{x - p}^nf\br{x} $$
exists. Then $ f $ is meromorphic on $ U $, and $ \ord_p f \ge -n $.
\end{theorem}

\pagebreak

\subsubsection{Modular functions}

\begin{definition}
$ f : \HH \to \CC $ is a \textbf{weakly modular function of weight $ k $} if
\begin{itemize}
\item $ f $ is meromorphic on $ \HH $, and
\item $ f $ satisfies the modular transformation law of weight $ k $.
\end{itemize}
\end{definition}

Consider
$$ \gamma = \twobytwo{1}{1}{0}{1}, $$
so $ \gamma z = z + 1 $ and $ cz + d = 1 $. The modular transformation law gives $ f\br{z + 1} = f\br{z} $. Let
$$ \D = \cbr{q \st \abs{q} < 1}. $$
Can define a function
$$ \function[g]{\D \setminus \cbr{0}}{\CC}{q}{f\br{\dfrac{\log q}{2\pi i}}}, $$
that is $ f\br{z} = g\br{e^{2\pi iz}} $ for $ z \in \HH $, where $ g $ is holomorphic or meromorphic on $ \cbr{z \st 0 < \abs{z} < 1} $ if and only if $ f $ is holomorphic or meromorphic on $ \HH $.

\begin{definition}
$ f : \HH \to \CC $ is a \textbf{modular form of weight $ k $} if
\begin{enumerate}
\item $ f $ satisfies the modular transformation law of weight $ k $,
\item $ f $ is holomorphic on $ \HH $, and
\item $ f $ is holomorphic at $ \infty $, so the function $ g : \D \setminus \cbr{0} \to \CC $, which is holomorphic on $ \D \setminus \cbr{0} $ by $ 2 $, extends to a holomorphic function on $ \D $.
\end{enumerate}
\end{definition}

Then $ q \to 0 $ in $ \D $ if and only if $ \Im z \to +\infty $. Then $ 3 $ means $ g\br{q} $ is bounded as $ q \to 0 $ so $ f\br{z} $ is bounded as $ \Im z \to +\infty $. For $ f $ satisfying $ 3 $, $ g : \D \setminus \cbr{0} \to \CC $ has a series expansion
$$ g\br{q} = \sum_n a_nq^n = a_0 + a_1q + \dots $$
in $ q = e^{2\pi iz} $. We call this the \textbf{$ q $-expansion} for $ f $.

\lecture{3}{Monday}{07/10/19}

\begin{definition}
$ f : \HH \to \CC $ is a \textbf{meromorphic modular form of weight $ k $} if the same conditions $ 1 $ to $ 3 $ hold, but with holomorphic weakened to meromorphic.
\end{definition}

\begin{note*}
If $ f $ is only meromorphic at $ \infty $ then a finite number of negative powers of $ q $ can appear.
\end{note*}

\begin{example*}
\hfill
\begin{itemize}
\item The \textbf{modular discriminant}
$$ \Delta\br{z} = q\prod_{n = 1}^\infty \br{1 - q^n}^{24} = q - 24q^2 + 252q^3 - 1472q^4 + \dots $$
is a modular form of weight $ 12 $.
\item The \textbf{$ \j $-invariant}
$$ \j\br{z} = \dfrac{1}{q} + 744 + 196844q + 21493760q^2 + \dots $$
is a meromorphic modular form of weight $ 0 $.
\end{itemize}
\end{example*}

\pagebreak

\subsubsection{Lattice functions}

How can we construct modular forms?

\begin{definition}
A \textbf{lattice} in $ \CC $ is an abelian subgroup of $ \CC $ of the form $ \ZZ w_1 + \ZZ w_2 $, where $ w_1, w_2 \in \CC $ are $ \RR $-linearly independent. More generally if $ V $ is an $ \RR $-vector space, a \textbf{lattice} $ L $ in $ V $ is a discrete abelian subgroup of $ V $ that spans $ V $ over $ \RR $. For $ L \subseteq \CC $ a lattice and $ \lambda \in \CC^\times $, let
$$ \lambda L = \cbr{\lambda x \st x \in L} \subseteq \CC. $$
We say that $ L $ and $ \lambda L $ are \textbf{homothetic}. For $ z \in \HH $, let
$$ \L_{z, 1} = \ZZ + \ZZ z = \cbr{az + b \st a, b \in \ZZ} \subseteq \CC. $$
\end{definition}

A question is when is $ \L_{z, 1} $ homothetic to $ \L_{z', 1} $, and what is a homothety factor?
\begin{itemize}
\item Suppose $ \L_{z, 1} = \lambda \L_{z', 1} $. Then there exist $ a, b, c, d $ such that $ \lambda z' = az + b $ and $ \lambda = cz + d $, so
\begin{equation}
\label{eq:1}
\twobyone{\lambda z'}{\lambda} = \gamma\twobyone{z}{1}.
\end{equation}
On the other hand there exist $ a', b', c', d' $ such that $ z = a'\lambda z' + b'\lambda $ and $ 1 = c'\lambda z' + d'\lambda $, so
\begin{equation}
\label{eq:2}
\gamma'\twobyone{\lambda z'}{\lambda} = \twobyone{z}{1}.
\end{equation}
$ \br{\ref{eq:1}} $ and $ \br{\ref{eq:2}} $ imply that
$$ \gamma'\gamma\twobyone{z}{1} = \twobyone{z}{1}, $$
so $ \gamma \in \SL_2\br{\ZZ} $. Moreover $ \br{\ref{eq:1}} $ implies that $ z' = \br{az + b} / \br{cz + d} $.
\item Conversely, if $ \gamma \in \SL_2\br{\ZZ} $, then $ \gamma z = \br{az + b} / \br{cz + d} $, so
$$ \L_{\gamma z, 1} = \br{cz + d}^{-1}\L_{az + b, cz + d}. $$
But certainly $ \L_{az + b, cz + d} \subseteq \L_{z, 1} $. On the other hand if $ \gamma' $ is inverse to $ \gamma $,
$$ \twobyone{z}{1} = \gamma'\gamma\twobyone{z}{1} = \gamma\twobyone{az + b}{cz + d} = \twobyone{a'\br{az + b} + b'\br{cz + d}}{c'\br{az + b} + d'\br{cz + d}}, $$
so $ z \in \L_{az + b, cz + d} $ and $ 1 \in \L_{az + b, cz + d} $. So $ \L_{az + b, cz + d} = \L_{z, 1} $, so $ \L_{\gamma z, 1} = \br{cz + d}^{-1}\L_{z, 1} $.
\end{itemize}

\begin{definition}
A \textbf{lattice function of weight $ k $} is a function $ F : \cbr{\text{lattices in} \ \CC} \to \CC $ such that
$$ F\br{\lambda L} = \lambda^{-k}F\br{L}, $$
for all lattices $ L $. Given such an $ F $, can define
$$ \function[f]{\HH}{\CC}{z}{F\br{\L_{z, 1}}}. $$
\end{definition}

If $ F $ has weight $ k $, then
$$ f\br{\gamma z} = F\br{\L_{\gamma z, 1}} = F\br{\br{cz + d}^{-1}\L_{z, 1}} = \br{cz + d}^kF\br{\L_{z, 1}} = \br{cz + d}^kf\br{z}. $$

\pagebreak

\subsection{Eisenstein series}

\lecture{4}{Friday}{11/10/19}

\begin{definition}
For $ L \in \CC $, define the \textbf{Eisenstein series}
$$ \G_k\br{L} = \sum_{w \in L, \ w \ne 0} \dfrac{1}{w^k}, \qquad \g_k\br{z} = \G_k\br{\L_{z, 1}} = \underset{\br{m, n} \ne \br{0, 0}}{\sum_{m = -\infty}^\infty \sum_{n = -\infty}^\infty} \dfrac{1}{\br{mz + n}^k}. $$
\end{definition}

Then
$$ \G_k\br{\lambda L} = \sum_{w' \in \lambda L, \ w' \ne 0} \dfrac{1}{w'^k} = \sum_{w \in L, \ w \ne 0} \dfrac{1}{\br{\lambda w}^k} = \lambda^{-k}\G_k\br{L}. $$

\begin{corollary}
$ \g_k $ satisfies the modular transformation law of weight $ k $.
\end{corollary}

The following are some questions.
\begin{itemize}
\item Does $ \G_k $, or $ \g_k $, converge?
\item Is $ \g_k $ holomorphic or meromorphic on $ \HH $?
\item Is $ \g_k $ holomorphic at $ \infty $?
\item What is the $ q $-expansion of $ \g_k $?
\end{itemize}

\subsubsection{Convergence and holomorphy on \texorpdfstring{$ \HH $}{H}}

\begin{definition}
Let $ U \subseteq \CC $ be open. A sequence of functions $ f_n : U \to \CC $ \textbf{converges uniformly on compact sets} to $ f $ if for all $ C \subseteq U $ compact and all $ \epsilon > 0 $, there exists $ N \in \ZZ $ such that for all $ n > N $,
$$ \abs{f\br{z} - f_n\br{z}} < \epsilon, \qquad z \in C. $$
\end{definition}

\begin{theorem}
A uniform limit of holomorphic functions is holomorphic. If $ f_n $ converges to $ f $ uniformly on compact sets and $ f_n $ is holomorphic on $ U $, then $ f $ is holomorphic on $ U $.
\end{theorem}

\begin{theorem}
Let $ k \ge 4 $. The series $ \g_k\br{z} $ converges absolutely and uniformly on compact subsets of $ \HH $.
\end{theorem}

\begin{proof}
Let
$$ P_{z, r} = \cbr{az + b \st a, b \in \RR, \ \max\br{\abs{a}, \abs{b}} = r} \subseteq \CC, $$
so $ P_{z, r} = rP_{z, 1} $, and there are $ 8r $ points on $ P_{z, r} \cap \L_{z, 1} $. Then
$$ \g_k\br{z} = \sum_{r = 1}^\infty \sum_{w \in \L_{z, 1} \cap P_{z, r}} \dfrac{1}{w^k}. $$
The function $ z \mapsto \abs{z} $ attains a non-zero minimum $ \delta\br{z} $ on $ P_{z, 1} $, so on $ P_{z, 1} $, have $ \abs{z} > \delta\br{z} $, so $ 1 / \abs{z}^k < 1 / \delta\br{z}^k $. On $ P_{z, r} $, have $ \abs{z} > r\delta\br{z} $, so $ 1 / \abs{z}^k < 1 / r^k\delta\br{z}^k $. Let $ C \subseteq \HH $ be compact. Then $ z \mapsto \delta\br{z} $ is a continuous function on $ C $ and attains a minimum $ \delta_C $. For all $ z \in C $ and all $ w \in P_{z, r} $, get $ \abs{w} > r\delta_C $, so
$$ \dfrac{1}{\abs{w}^k} < \dfrac{1}{r^k\delta_C^k}. $$
Thus for $ z \in C $, $ \g_k\br{z} $ is dominated by
$$ \sum_{r = 1}^\infty \dfrac{8r}{r^k\delta_C^k} = \dfrac{8}{\delta_C^k}\sum_{r = 1}^\infty \dfrac{1}{r^{k - 1}}, $$
which converges absolutely for $ k \ge 4 $.
\end{proof}

\begin{corollary}
$ \g_k\br{z} $ is holomorphic on $ \HH $.
\end{corollary}

\subsubsection{\texorpdfstring{$ q $}{q}-expansion and holomorphy at \texorpdfstring{$ \infty $}{infinity}}

The idea is to understand series of the form
$$ \sum_{n = -\infty}^\infty \dfrac{1}{\br{z + n}^k}. $$

\begin{theorem}
A bounded holomorphic function on all of $ \CC $ is constant.
\end{theorem}

\begin{lemma}
\hfill
\begin{enumerate}
\item
$$ \dfrac{\pi^2}{\sin^2 \pi z} =  \sum_{n = -\infty}^\infty \dfrac{1}{\br{z - n}^2} = \sum_{n = -\infty}^\infty \dfrac{1}{\br{z - n}^2}. $$
\item
$$ \pi\cot \pi z = \dfrac{1}{z} + \sum_{n = 1}^\infty \br{\dfrac{1}{z - n} + \dfrac{1}{z + n}} = \dfrac{1}{z} + \sum_{n = 1}^\infty \dfrac{2z}{z^2 - n^2}. $$
\end{enumerate}
\end{lemma}

\begin{proof}
\hfill
\begin{enumerate}
\item The right hand side converges absolutely and uniformly on compact subsets of $ \CC \setminus \ZZ $, so the right hand side is holomorphic on $ \CC \setminus \ZZ $. Locally around $ z = n $, the series looks like
$$ \sum_{n = -\infty}^\infty \dfrac{1}{\br{z - n}^2} = \dots + \dfrac{1}{\br{z - n + 1}^2} + \dfrac{1}{\br{z - n}^2} + \dfrac{1}{\br{z - n - 1}^2} + \dots = \dfrac{1}{\br{z - n}^2} + h_1\br{z}, $$
where $ h_1\br{z} $ is holomorphic in a neighbourhood of $ z = n $. Similarly, the left hand side is meromorphic on $ \CC $, and the Laurent series near $ z = n $ is
$$ \dfrac{\pi^2}{\sin^2 \pi z} = \pi\br{\dfrac{1}{\pi^2\br{z - n}^2} + \dfrac{1}{3} + \dfrac{1}{15}\pi^2\br{z - n}^2 + \dots} = \dfrac{1}{\br{z - n}^2} + h_2\br{z}, $$
where $ h_2\br{z} $ is a holomorphic function. So the difference
$$ g\br{z} = \sum_{n = -\infty}^\infty \dfrac{1}{\br{z - n}^2} - \dfrac{\pi^2}{\sin^2 \pi z} $$
is meromorphic on $ \CC $ and holomorphic on $ \CC \setminus \ZZ $, and the Laurent expression around $ z = n $ is
$$ g\br{z} = \dfrac{1}{\br{z - n}^2} + h_1\br{z} - \br{\dfrac{1}{\br{z - n}^2} + h_2\br{z}} = h_1\br{z} - h_2\br{z}, $$
so $ g\br{z} $ is holomorphic at $ z = n $ for all $ n $. Consider $ t \to \pm\infty $ for $ z = a + it $. The right hand side is
$$ R = \sum_{n = -\infty}^\infty \dfrac{1}{\br{z - n}^2} = \sum_{n = a - N}^{a + N} \dfrac{1}{\br{z - n}^2} + \sum_{n = -\infty}^{a - N - 1} \dfrac{1}{\br{z - n}^2} + \sum_{n = a + N + 1}^\infty \dfrac{1}{\br{z - n}^2} = R_0 + R_- + R_+, $$
where $ R_0 $ has finitely many terms that converge to less than $ \epsilon / 2 $ as $ t \to \pm\infty $ and $ R_- + R_+ < \epsilon / 2 $ for $ N \gg 0 $ independent of $ t $, so $ R < \epsilon $ converges to zero. Similarly, the left hand side is
$$ \abs{\dfrac{\pi^2}{\sin^2 \pi z}} = \abs{\dfrac{2\pi^2}{e^{\pi iz} - e^{-\pi iz}}} \to 0, $$
so $ \lim_{t \to \infty} g\br{a + it} = 0 $. Moreover, $ g\br{z + 1} = g\br{z} $ for all $ z $. Then
$$ S = \cbr{z \in \CC \st n - 1 \le \Re z \le n, \ -N \le \Im z \le N}, \qquad n \in \ZZ $$
is compact, so $ \abs{g\br{z}} $ attains a maximum in $ S $, so $ g\br{z} $ is bounded in $ S $. Since $ g\br{z} $ is also bounded in $ R_- + R_+ $, $ g\br{z} $ is bounded in $ \CC $, so $ g $ is constant. Since $ \lim_{t \to \infty} g\br{a + it} = 0 $, $ g = 0 $.

\pagebreak

\lecture{5}{Friday}{11/10/19}

\item Check that the right hand side converges absolutely and uniformly on compact subsets of $ \CC \setminus \ZZ $, so the right hand side is meromorphic on $ \CC \setminus \ZZ $. Similarly, the left hand side is also meromorphic on $ \CC \setminus \ZZ $. Comparing derivatives,
$$ -\dfrac{\pi^2}{\sin^2 \pi z} = -\dfrac{1}{z^2} - \sum_{n = 1}^\infty \br{\dfrac{1}{\br{z - n}^2} + \dfrac{1}{\br{z + n}^2}}, $$
so the difference is constant. Let $ z = \tfrac{1}{2} $. The left hand side is $ \pi\cot \pi / 2 = 0 $ and the right hand side is
$$ \dfrac{2}{1} + \br{-\dfrac{2}{1} + \dfrac{2}{3}} + \br{-\dfrac{2}{3} + \dfrac{2}{5}} + \dots \to 0, \qquad n \to \infty, $$
so the difference is zero.
\end{enumerate}
\end{proof}

Thus
$$ \dfrac{1}{z} + \sum_{n = 1}^\infty \br{\dfrac{1}{z - n} + \dfrac{1}{z + n}} = \pi\cot \pi z = \pi i\dfrac{e^{\pi iz} + e^{-\pi iz}}{e^{\pi iz} - e^{-\pi iz}} = \pi i\dfrac{q + 1}{q - 1} = \pi i - \dfrac{2\pi i}{1 - q} = \pi i - 2\pi i\sum_{n = 0}^\infty q^n. $$
Take $ \tod[k - 1]{}{z} $. For $ k \ge 2 $ even, get
$$ -\br{k - 1}!\sum_{n = -\infty}^\infty \dfrac{1}{\br{z + n}^k} = -\br{2\pi i}^k\sum_{n = 1}^\infty n^{k - 1}q^n, $$
so
$$ \sum_{n = -\infty}^\infty \dfrac{1}{\br{z + n}^k} = \dfrac{\br{2\pi i}^k}{\br{k - 1}!}\sum_{n = 1}^\infty n^{k - 1}q^n. $$
Collecting powers of $ q $,
\begin{align*}
\g_k\br{z}
& = \underset{\br{m, n} \ne \br{0, 0}}{\sum_{m = -\infty}^\infty \sum_{n = -\infty}^\infty} \dfrac{1}{\br{mz + n}^k} \\
& = 2\sum_{n = 1}^\infty \dfrac{1}{n^k} + 2\sum_{m = 1}^\infty \sum_{n = -\infty}^\infty \dfrac{1}{\br{mz + n}^k} \\
& = 2\zeta\br{k} + \dfrac{2\br{2\pi i}^k}{\br{k - 1}!}\sum_{m = 1}^\infty \sum_{n = 1}^\infty n^{k - 1}q^{nm} & \zeta\br{s} = \sum_{n = 1}^\infty n^{-s} \\
& = 2\zeta\br{k} + \dfrac{2\br{2\pi i}^k}{\br{k - 1}!}\sum_{n = 1}^\infty \sigma_{k - 1}\br{n}q^n & \sigma_{k - 1}\br{n} = \sum_{d \mid n, \ d > 0} d^{k - 1}.
\end{align*}

\begin{corollary}
$ \g_k\br{z} $ is holomorphic at $ \infty $. In particular, $ \g_k $ is a modular form of weight $ k $.
\end{corollary}

\subsubsection{Bernoulli numbers}

\begin{definition}
The \textbf{Bernoulli numbers} $ b_k $ are defined by
$$ \sum_{k = 0}^\infty b_k\dfrac{x^k}{k!} = \dfrac{x}{e^x - 1}, $$
a formal power series with rational coefficients.
\end{definition}

Then
$$ b_0 = 1, \qquad b_1 = -\tfrac{1}{2}, \qquad b_2 = \tfrac{1}{6}, \qquad b_3 = 0, \qquad b_4 = -\tfrac{1}{20}, \qquad \dots, \qquad b_{2k} \in \QQ, \qquad b_{2k + 1} = 0, \qquad \dots. $$

\pagebreak

\begin{proposition}
For all even $ k $,
$$ \zeta\br{k} = -b_k\dfrac{\br{2\pi i}^k}{2k!}. $$
\end{proposition}

\begin{proof}
On one hand,
$$ \pi z\cot \pi z = \pi iz + \dfrac{2\pi iz}{e^{2\pi iz} - 1} = \pi iz + \sum_{k = 0}^\infty b_k\dfrac{\br{2\pi iz}^k}{k!}. $$
On the other hand,
\begin{align*}
\pi\cot \pi z
& = \dfrac{1}{z} + \sum_{n = 1}^\infty \dfrac{2z}{z^2 - n^2}
= \dfrac{1}{z} - \dfrac{2z}{n^2}\sum_{n = 1}^\infty \dfrac{1}{1 - z^2 / n^2} \\
& = \dfrac{1}{z} - \sum_{n = 1}^\infty \dfrac{2}{z}\sum_{k = 1}^\infty \br{\dfrac{z^2}{n^2}}^k
= \dfrac{1}{z} - \dfrac{2}{z}\sum_{k = 1}^\infty z^{2k}\sum_{n = 1}^\infty \dfrac{1}{n^{2k}}
= \dfrac{1}{z} - \dfrac{2}{z}\sum_{k = 1}^\infty \zeta\br{2k}z^{2k},
\end{align*}
so
$$ \pi iz + \sum_{k = 0}^\infty b_k\dfrac{\br{2\pi iz}^k}{k!} = \pi z\cot \pi z = 1 - 2\sum_{k = 1}^\infty \zeta\br{2k}z^{2k}. $$
Comparing,
$$ b_{2k}\dfrac{\br{2\pi i}^{2k}}{\br{2k}!} = -2\zeta\br{2k}, $$
get the desired formula.
\end{proof}

So
$$ \g_k\br{z} = \dfrac{-b_k\br{2\pi i}^k}{k!} + \dfrac{2\br{2\pi i}^k}{\br{k - 1}!}\sum_{n = 1}^\infty \sigma_{k - 1}\br{n}q^n. $$
Set the \textbf{normalised Eisenstein series}
$$ \E_k = \dfrac{\g_k}{2\zeta\br{k}} = 1 - \dfrac{2k}{b_k}\sum_{n = 1}^\infty \sigma_{k - 1}\br{n}q^n. $$

\begin{example*}
$$ \E_4 = 1 + 240\sum_{n = 1}^\infty \sigma_3\br{n}q^n, \qquad \E_6 = 1 - 504\sum_{n = 1}^\infty \sigma_5\br{n}q^n, $$
$$ \E_8 = 1 + 480\sum_{n = 1}^\infty \sigma_7\br{n}q^n, \qquad \E_{12} = 1 + \dfrac{65520}{691}\sum_{n = 1}^\infty \sigma_{11}\br{n}q^n. $$
\end{example*}

\lecture{6}{Monday}{14/10/19}

An observation is if $ f $ is modular of weight $ k $ and $ g $ is modular of weight $ k' $, then $ fg $ is modular of weight $ k + k' $, and if $ k = k' $, then $ f + g $ is modular of weight $ k $.

\begin{example*}
Important examples.
\begin{itemize}
\item The \textbf{modular discriminant}
$$ \Delta\br{z} = \dfrac{\E_4 - \E_6^2}{1728} = q - 24q^2 + 252q^3 + \dots $$
is a modular form of weight $ 12 $.
\item The \textbf{$ \j $-invariant}
$$ \j\br{z} = \dfrac{\E_4^3}{\Delta} = \dfrac{1}{q} + 744 + 196844q + \dots $$
is a meromorphic modular form of weight $ 0 $.
\end{itemize}
\end{example*}

\pagebreak

\subsection{Controlling modular forms}

\subsubsection{The fundamental domain}

The idea is to control the action of $ \SL_2\br{\ZZ} $ on $ \HH $. If $ f : \HH \to \CC $ satisfies $ f\br{\gamma z} = \br{cz + d}^kf\br{z} $ for all $ \gamma \in \SL_2\br{\ZZ} $, and if $ D \subseteq \HH $ such that $ D $ meets every $ \SL_2\br{\ZZ} $-orbit in $ \HH $, then $ f $ is determined by its values on $ D $.

\begin{definition}
Let $ G $ be a group acting continuously on a complex analytic space $ X $, such as $ X = \HH $. A subset $ D \subseteq X $ is a \textbf{fundamental domain} for the action of $ G $ if
\begin{itemize}
\item $ D $ meets every $ G $-orbit in $ X $,
\item the subset $ \cbr{x \in D \st \exists g \in G, \ gx \in D, \ gx \ne x} $ has measure zero, and
\item $ D $ is closed in $ X $.
\end{itemize}
\end{definition}

Define
$$ \DDD = \cbr{z \in \HH \st \tfrac{1}{2} \le \Re z \le \tfrac{1}{2}, \ \abs{z} \ge 1} \subseteq \HH, $$
so
$$
\begin{tikzpicture}[scale=2]
\draw [dotted, thick] (-2, 0) to (2, 0);
\draw [dotted, thick] (0, 0) to (0, 2);
\draw [dashed] (-1, 0) arc (180:0:1);
\draw [dashed] (-0.5, 0) to (-0.5, 2);
\draw [dashed] (0.5, 0) to (0.5, 2);
\draw (-0.5, 0.866) arc (120:60:1);
\draw (-0.5, 0.866) to (-0.5, 2);
\draw (0.5, 0.866) to (0.5, 2);
\fill (0, 1) circle (0.025) node[below]{$ i $};
\fill (-0.5, 0.866) circle (0.025) node[below left]{$ -\tfrac{1}{2} + \tfrac{\sqrt{-3}}{2} = e^{\tfrac{2\pi i}{3}} = \rho $};
\fill (0.5, 0.866) circle (0.025) node[below right]{$ \rho' = e^{\tfrac{\pi i}{3}} = \tfrac{1}{2} + \tfrac{\sqrt{-3}}{2} $};
\end{tikzpicture}.
$$
Let
$$ \S = \twobytwo{0}{-1}{1}{0} : z \mapsto -\dfrac{1}{z}, \qquad \T = \twobytwo{1}{1}{0}{1} : z \mapsto z + 1, $$
and let $ \Gamma \subseteq \SL_2\br{\ZZ} $ be the subgroup generated by $ \S $ and $ \T $. We will see later that $ \Gamma = \SL_2\br{\ZZ} $.

\begin{theorem}
\label{thm:fundamentaldomain}
\hfill
\begin{enumerate}
\item For all $ z \in \HH $, there exists $ \gamma \in \Gamma $ such that $ \gamma z \in \DDD $.
\item Suppose $ z, z' \in \DDD $ and $ \gamma \in \SL_2\br{\ZZ} $ with $ \gamma z = z' $. Then either
\begin{itemize}
\item $ z = z' $,
\item $ \Re z = \pm\tfrac{1}{2} $ and $ z' = z \mp 1 $, or
\item $ \abs{z} = 1 $ and $ z' = -1 / z $.
\end{itemize}
In particular, if $ z \ne z' $, then $ z $ and $ z' $ are on the boundary of $ \DDD $.
\item For $ z \in \DDD $, let $ \I_z $ be the stabiliser of $ z $ in $ \SL_2\br{\ZZ} $, that is
$$ \I_z = \cbr{\gamma \in \SL_2\br{\ZZ} \st \gamma z = z}. $$
Then $ \I_z = \cbr{\pm\id} $ unless
\begin{itemize}
\item $ z = i $, where $ \I_z = \cbr{\pm\id, \pm \S} $,
\item $ z = \rho $, where $ \I_z = \cbr{\pm\id, \pm\br{\S\T}, \pm\br{\T^{-1}\S}} $, or
\item $ z = \rho' $, where $ \I_z = \cbr{\pm\id, \pm\br{\T\S}, \pm\br{\S\T^{-1}}} $.
\end{itemize}
\end{enumerate}
\end{theorem}

\begin{corollary}
$ \Gamma = \SL_2\br{\ZZ} $.
\end{corollary}

\begin{proof}
Fix $ \gamma \in \SL_2\br{\ZZ} $ and $ z \in \mathring{\DDD} $ so $ \SL_2\br{\ZZ}z \cap \DDD = \cbr{z} $ and $ \I_z = \cbr{\pm\id} $. Consider $ \gamma z $. There exists $ \gamma' \in \Gamma $ such that $ \gamma'\gamma z \in \DDD $, so $ \gamma'\gamma z = z $. So $ \gamma'\gamma = \pm\id $, so $ \gamma = \pm\gamma'^{-1} $. But $ \gamma'^{-1} \in \Gamma $ and $ -\id = \S^2 \in \Gamma $, so $ \gamma \in \Gamma $.
\end{proof}

\pagebreak

\begin{proof}[Proof of Theorem \ref{thm:fundamentaldomain}]
Recall $ \Im \gamma z = \Im z / \abs{cz + d}^2 $ for $ \gamma \in \SL_2\br{\ZZ} $.
\begin{enumerate}
\item As $ c $ and $ d $ vary, $ \cbr{cz + d} $ forms a lattice in $ \CC $, so there exist only finitely many $ c $ and $ d $ such that $ \abs{cz + d} < 1 $. So $ \Im \gamma z $ attains a maximum as $ \gamma $ varies over $ \Gamma $, so there exists $ \gamma \in \Gamma $ such that $ \Im \gamma z $ is maximal. There exists $ n \in \ZZ $ such that $ \T^n\gamma z $ has real part between $ -\tfrac{1}{2} $ and $ \tfrac{1}{2} $. Consider $ \abs{\T^n\gamma z} $. If this is less than one, then
$$ \Im \S\T^n\gamma z = \Im \dfrac{-1}{\T^n\gamma z} > \Im \T^n\gamma z = \Im \gamma z. $$
Since $ \S\T^n\gamma \in \Gamma $, this contradicts maximality so $ \abs{\T^n\gamma z} \ge 1 $, so $ \T^n\gamma z \in \DDD $.

\lecture{7}{Friday}{18/10/19}

\item[$ 2, 3 $.] Let $ z, z' \in \DDD $ such that $ \gamma z = z' $. Without loss of generality $ \Im z' \ge \Im z $, so $ \abs{cz + d} \le 1 $. Note that $ \abs{cz + d} \ge \Im \br{cz + d} \ge \tfrac{\sqrt{3}}{2}c $, so $ c = -1, 0, 1 $. Note that can replace $ \gamma $ with $ -\gamma $ if convenient.
\begin{itemize}
\item[$ c = 0 $.] Then $ ad = 1 $, so can assume $ a = d = 1 $, so $ \gamma z = z + b $.
Since $ z, z + b \in \DDD $, $ b = \pm 1 $ and $ \Re z = \mp\tfrac{1}{2} $.
\item[$ c = 1 $.] Have $ \abs{z + d} \le 1 $ and $ \abs{z} \ge 1 $, so $ d = -1, 0, 1 $.
\begin{itemize}
\item[$ d = 0 $.] Then $ \abs{z} = 1 $, and $ \gamma z = \br{az - 1} / z = a - 1 / z $.
The only possibilities are
\begin{itemize}
\item $ a = 0 $ and $ \gamma = \S $,
\item $ a = 1 $ and $ \gamma = \T\S $, so $ z = \rho' $, or
\item $ a = -1 $ and $ \gamma = \T^{-1}\S $, so $ z = \rho $.
\end{itemize}
\item[$ d = 1 $.] Then $ z = \rho $, and $ \gamma z = \br{\br{b + 1}z + b} / \br{z + 1} = b + 1 - 1 / \br{z + 1} $, so $ b = 0 $ or $ b = -1 $.
\item[$ d = -1 $.] Then $ z = \rho' $ is similar.
\end{itemize}
\item[$ c = -1 $.] Similar.
\end{itemize}
\end{enumerate}
\end{proof}

\subsubsection{Further review of complex analysis}

Recall that on any compact set, a meromorphic function has only finitely many zeroes and poles. If $ f\br{z} = g\br{e^{2\pi i z}} $ is meromorphic at infinity and $ g $ is meromorphic on $ \D = \abs{q} < 1 $, zeroes and poles of $ g $ are discrete with respect to $ q $, and $ \Im z \gg 0 $ if and only if $ \abs{q} < \epsilon $.

\begin{definition}
Let $ U \subseteq \CC $ be open, and let $ f : U \to \CC $ be meromorphic on $ U $. If $ f $ has a pole at $ p $, can write
$$ f\br{z} = \sum_{n = \ord_p f < 0}^\infty a_n\br{z - p}^n. $$
The coefficient $ a_{-1} $ is called the \textbf{residue} $ \Res_p f $ of $ f $ at $ p $.
\end{definition}

\begin{theorem}[Residue theorem]
Let $ V $ be a region in $ \CC $ whose boundary $ \partial V $ is a simple closed curve. Then
$$ \dfrac{1}{2\pi}\intd{\partial V}{}{f\br{z}}{z} = \sum_{p \in V \ \text{pole of} \ f} \Res_p f. $$
\end{theorem}

\begin{definition}
Let $ f $ be meromorphic on $ U \subseteq \CC $ open. Then the \textbf{logarithmic derivative} $ \d\log f $ is the function $ f' / f $.
\end{definition}

If $ f\br{z} = c_n\br{z - p}^n + c_{n + 1}\br{z - p}^{n + 1} + \dots $, then if $ n \ne 0 $, then the leading term of $ f' $ is $ nc_n\br{z - p}^{n - 1} $ and the leading term of $ f $ is $ c_n\br{z - p}^n $, so the leading term of $ f' / f $ is $ n\br{z - p}^{-1} $. If $ n = 0 $, then $ f' / f $ is holomorphic. So $ f' / f $ is meromorphic with simple poles precisely at the points where $ \ord_p f \ne 0 $, and $ \Res_p f' / f $ at such $ p $ is $ \ord_p f $.

\begin{theorem}[Argument principle]
$$ \dfrac{1}{2\pi i}\intd{\partial V}{}{\dfrac{f'\br{z}}{f\br{z}}}{z} = \sum_{p \in V} \ord_p f. $$
\end{theorem}

\pagebreak

\subsubsection{Controlling modular forms}

\begin{theorem}
Let $ f $ be a non-zero meromorphic modular form of weight $ k $. Then
$$ \ord_\infty f + \dfrac{\ord_\rho f}{3} + \dfrac{\ord_i f}{2} + \sum_{p \in \HH / \SL_2\br{\ZZ}, \ p \nsim \cbr{i, \rho}} \ord_p f = \dfrac{k}{12}. $$
\end{theorem}

\lecture{8}{Friday}{18/10/19}

\begin{proof}
Consider the closed curve $ C_{N, \epsilon} $,
$$
\begin{tikzpicture}[scale=4]
\draw [dotted, thick] (-0.5, 0.866) arc (120:60:1);
\draw [dotted, thick] (-0.5, 0.866) to (-0.5, 2);
\draw [dotted, thick] (0.5, 0.866) to (0.5, 2);
\draw (-0.5, 0.916) arc (90:25:0.05);
\draw (-0.455, 0.89) arc (117:107.5:1);
\draw (-0.3, 0.955) arc (-165:15:0.05);
\draw (-0.205, 0.98) arc (101:92:1);
\draw (-0.05, 0.995) arc (180:0:0.05);
\draw (0.205, 0.98) arc (79:88:1);
\draw (0.3, 0.955) arc (-15:165:0.05);
\draw (0.455, 0.89) arc (63:72.5:1);
\draw (0.5, 0.916) arc (90:155:0.05);
\draw (-0.5, 1.5) to node[above]{$ \Im z = N $} node{$ < $} (0.5, 1.5);
\draw (-0.5, 0.916) to (-0.5, 1.2);
\draw (-0.5, 1.2) arc (270:90:0.05);
\draw (-0.5, 1.3) to (-0.5, 1.5);
\draw (0.5, 0.916) to (0.5, 1.2);
\draw (0.5, 1.2) arc (270:90:0.05);
\draw (0.5, 1.3) to (0.5, 1.5);
\fill (0, 1) circle (0.0125) node[below]{$ i $};
\fill (-0.25, 0.965) circle (0.0125) node[above]{$ z $};
\fill (0.25, 0.965) circle (0.0125) node[below]{$ z $};
\fill (-0.5, 1.25) circle (0.0125) node[right]{$ z $};
\fill (-0.5, 0.866) circle (0.0125) node[below left]{$ \rho $};
\fill (0.5, 1.25) circle (0.0125) node[right]{$ z $};
\fill (0.5, 0.866) circle (0.0125) node[below right]{$ \rho' $};
\end{tikzpicture},
$$
where the $ z $'s are zeroes or poles of $ f $, and the circles are of radius $ \epsilon $. Consider
$$ \dfrac{1}{2\pi i}\intd{C_{N, \epsilon}}{}{\dfrac{f'\br{z}}{f\br{z}}}{z} = \sum_{p \in \HH / \SL_2\br{\ZZ}, \ p \nsim \cbr{i, \rho}} \ord_p f, \qquad \epsilon \to 0. $$
So it suffices to show
$$ \lim_{\epsilon \to 0, \ N \to \infty} \dfrac{1}{2\pi i}\intd{C_{N, \epsilon}}{}{\dfrac{f'\br{z}}{f\br{z}}}{z} = -\ord_\infty f - \dfrac{\ord_\rho f}{3} - \dfrac{\ord_i f}{2} + \dfrac{k}{12}. $$
The vertical parts of the boundary cancel. The integral over the circular part of $ \partial\DDD $ approaches
$$ \dfrac{1}{2\pi i}\intd{\rho}{i}{\dfrac{f'\br{z}}{f\br{z}}}{z} + \dfrac{1}{2\pi i}\intd{i}{\rho'}{\dfrac{f'\br{z}}{f\br{z}}}{z} = \dfrac{1}{2\pi i}\br{\intd{\rho}{i}{\dfrac{f'\br{z}}{f\br{z}}}{z} - \intd{\rho}{i}{\dfrac{f'\br{-1 / z}}{f\br{-1 / z}}}{z}} $$
Since $ f\br{-1 / z} = z^kf\br{z} $,
$$ \d\br{z^kf\br{z}} = \br{kz^{k - 1}f\br{z} + z^kf'\br{z}}\d z, $$
so
$$ \dfrac{1}{2\pi i}\intd{\rho}{i}{\dfrac{f'\br{z}}{f\br{z}}}{z} + \dfrac{1}{2\pi i}\intd{i}{\rho'}{\dfrac{f'\br{z}}{f\br{z}}}{z} = \dfrac{1}{2\pi i}\intd{\rho}{i}{\dfrac{f'\br{z}}{f\br{z}} - \dfrac{kz^{k - 1}f\br{z} + z^kf'\br{z}}{z^kf\br{z}}}{z} = -\dfrac{1}{2\pi i}\intd{\rho}{i}{\dfrac{k}{z}}{z} = \dfrac{k}{12}. $$
Since $ \d q = 2\pi iq\d z $, the top part is
$$ \dfrac{1}{2\pi i}\intd{\tfrac{1}{2} + iN}{\tfrac{1}{2} - iN}{\dfrac{f'\br{z}}{f\br{z}}}{z} = -\dfrac{1}{2\pi i}\intd{\text{circle of radius} \ \epsilon}{}{\dfrac{g'\br{q}}{g\br{q}}}{q} = -\ord_\infty f. $$
Near $ i $, $ f' / f = \ord_i f\br{z - i}^{-1} + h\br{z} $, where $ h\br{z} $ is holomorphic and $ h\br{z} \to 0 $ as $ \epsilon \to 0 $. Then the circle $ C_{\epsilon, i} $ of radius $ \epsilon $ centered at $ i $ is
$$ \lim_{\epsilon \to 0} \dfrac{1}{2\pi i}\intd{C_{\epsilon, i}}{}{\dfrac{f'\br{z}}{f\br{z}}}{z} = \lim_{\epsilon \to 0} \dfrac{1}{2\pi i}\intd{\text{arc of half circle centered at} \ i}{}{\dfrac{\ord_i f}{z - i}}{z} = -\dfrac{\ord_i f}{2}. $$
Similarly, at $ \rho $ and $ \rho' $, get that the circles $ C_{\epsilon, \rho} $ and $ C_{\epsilon, \rho'} $ of radius $ \epsilon $ centered at $ \rho $ and $ \rho' $ are
$$ \lim_{\epsilon \to 0} \dfrac{1}{2\pi i}\intd{C_{\epsilon, \rho}}{}{\dfrac{f'\br{z}}{f\br{z}}}{z} = \lim_{\epsilon \to 0} \dfrac{1}{2\pi i}\intd{C_{\epsilon, \rho'}}{}{\dfrac{f'\br{z}}{f\br{z}}}{z} = -\dfrac{\ord_\rho f}{6}, $$
which gives $ -\ord_\rho f / 3 $.
\end{proof}

\pagebreak

\subsubsection{Holomorphic modular forms}

Let
$$ \M_k = \cbr{\text{holomorphic modular forms of weight} \ k}, $$
and let
$$ \S_k = \cbr{\text{cusp forms of weight} \ k} = \cbr{f \in \M_k \st \ord_\infty f > 0} \subseteq \M_k. $$

\begin{corollary}
\hfill
\begin{itemize}
\item $ \M_k = 0 $ if $ k < 0 $, $ k = 2 $, or $ k $ odd.
\item $ \M_0 $ are constants.
\item $ \M_4 = \CC\E_4 $, where $ \ord_\rho \E_4 = 1 $ and no other zeroes.
\item $ \M_6 = \CC\E_6 $, where $ \ord_i \E_6 = 1 $ and no other zeroes.
\item $ \M_8 = \CC\E_8 $, where $ \ord_\rho \E_8 = 2 $ and no other zeroes.
\item $ \M_{10} = \CC\E_{10} $, where $ \ord_\rho \E_{10} = \ord_i \E_{10} = 1 $ and no other zeroes.
\item $ \M_{12} = \CC\E_{12} \oplus \CC\Delta $, where $ \ord_\infty \Delta = 1 $ and no other zeroes.
\end{itemize}
\end{corollary}

\begin{corollary}
$ \Delta : \M_k \to \S_{k + 12} $ is an isomorphism. On the other hand,
$$ \M_k \cong \CC\E_k \oplus \S_k, \qquad k \ge 4 \ \text{even}, $$
so
$$ \M_k \cong \CC\E_k \oplus \dots \oplus \CC\E_{k - 12r}\Delta^r, \qquad k - 12r \in \cbr{0, 4, 6, 8, 10, 14}. $$
So for $ k \ge 4 $, the set
$$
\begin{cases}
\E_k, \dots, \E_{k - 12\fbr{k / 12}}\Delta^{\fbr{k / 12}} & k \not\equiv 2 \mod 12 \\
\E_k, \dots, \E_{14}\Delta^{\fbr{k / 12} - 1} & k \equiv 2 \mod 12
\end{cases}
$$
is a basis for $ \M_k $.
\end{corollary}

\begin{corollary}
$ \E_4^2 = \E_8 $ and $ \E_4\E_6 = \E_{10} $.
\end{corollary}

\end{document}