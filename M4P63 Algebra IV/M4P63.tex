\def\module{M4P63 Algebra IV}
\def\lecturer{Dr John Britnell}
\def\term{Spring 2020}
\def\cover{}
\def\syllabus{}
\def\thm{section}

\documentclass{article}

% Packages

\usepackage{amssymb}
\usepackage{amsthm}
\usepackage[UKenglish]{babel}
\usepackage{commath}
\usepackage{enumitem}
\usepackage{etoolbox}
\usepackage{fancyhdr}
\usepackage[margin=1in]{geometry}
\usepackage{graphicx}
\usepackage[hidelinks]{hyperref}
\usepackage[utf8]{inputenc}
\usepackage{listings}
\usepackage{mathtools}
\usepackage{stmaryrd}
\usepackage{tikz-cd}
\usepackage{csquotes}

% Formatting

\addto\captionsUKenglish{\renewcommand{\abstractname}{Syllabus}}
\delimitershortfall5pt
\ifx\thm\undefined\newtheorem{n}{}\else\newtheorem{n}{}[\thm]\fi
\newcommand\newoperator[1]{\ifcsdef{#1}{\cslet{#1}{\relax}}{}\csdef{#1}{\operatorname{#1}}}
\setlength{\parindent}{0cm}

% Environments

\theoremstyle{plain}
\newtheorem{algorithm}[n]{Algorithm}
\newtheorem*{algorithm*}{Algorithm}
\newtheorem{algorithm**}{Algorithm}
\newtheorem{conjecture}[n]{Conjecture}
\newtheorem*{conjecture*}{Conjecture}
\newtheorem{conjecture**}{Conjecture}
\newtheorem{corollary}[n]{Corollary}
\newtheorem*{corollary*}{Corollary}
\newtheorem{corollary**}{Corollary}
\newtheorem{lemma}[n]{Lemma}
\newtheorem*{lemma*}{Lemma}
\newtheorem{lemma**}{Lemma}
\newtheorem{proposition}[n]{Proposition}
\newtheorem*{proposition*}{Proposition}
\newtheorem{proposition**}{Proposition}
\newtheorem{theorem}[n]{Theorem}
\newtheorem*{theorem*}{Theorem}
\newtheorem{theorem**}{Theorem}

\theoremstyle{definition}
\newtheorem{aim}[n]{Aim}
\newtheorem*{aim*}{Aim}
\newtheorem{aim**}{Aim}
\newtheorem{axiom}[n]{Axiom}
\newtheorem*{axiom*}{Axiom}
\newtheorem{axiom**}{Axiom}
\newtheorem{condition}[n]{Condition}
\newtheorem*{condition*}{Condition}
\newtheorem{condition**}{Condition}
\newtheorem{definition}[n]{Definition}
\newtheorem*{definition*}{Definition}
\newtheorem{definition**}{Definition}
\newtheorem{example}[n]{Example}
\newtheorem*{example*}{Example}
\newtheorem{example**}{Example}
\newtheorem{exercise}[n]{Exercise}
\newtheorem*{exercise*}{Exercise}
\newtheorem{exercise**}{Exercise}
\newtheorem{fact}[n]{Fact}
\newtheorem*{fact*}{Fact}
\newtheorem{fact**}{Fact}
\newtheorem{goal}[n]{Goal}
\newtheorem*{goal*}{Goal}
\newtheorem{goal**}{Goal}
\newtheorem{law}[n]{Law}
\newtheorem*{law*}{Law}
\newtheorem{law**}{Law}
\newtheorem{plan}[n]{Plan}
\newtheorem*{plan*}{Plan}
\newtheorem{plan**}{Plan}
\newtheorem{problem}[n]{Problem}
\newtheorem*{problem*}{Problem}
\newtheorem{problem**}{Problem}
\newtheorem{question}[n]{Question}
\newtheorem*{question*}{Question}
\newtheorem{question**}{Question}
\newtheorem{warning}[n]{Warning}
\newtheorem*{warning*}{Warning}
\newtheorem{warning**}{Warning}
\newtheorem{acknowledgements}[n]{Acknowledgements}
\newtheorem*{acknowledgements*}{Acknowledgements}
\newtheorem{acknowledgements**}{Acknowledgements}
\newtheorem{annotations}[n]{Annotations}
\newtheorem*{annotations*}{Annotations}
\newtheorem{annotations**}{Annotations}
\newtheorem{assumption}[n]{Assumption}
\newtheorem*{assumption*}{Assumption}
\newtheorem{assumption**}{Assumption}
\newtheorem{conclusion}[n]{Conclusion}
\newtheorem*{conclusion*}{Conclusion}
\newtheorem{conclusion**}{Conclusion}
\newtheorem{claim}[n]{Claim}
\newtheorem*{claim*}{Claim}
\newtheorem{claim**}{Claim}
\newtheorem{notation}[n]{Notation}
\newtheorem*{notation*}{Notation}
\newtheorem{notation**}{Notation}
\newtheorem{note}[n]{Note}
\newtheorem*{note*}{Note}
\newtheorem{note**}{Note}
\newtheorem{remark}[n]{Remark}
\newtheorem*{remark*}{Remark}
\newtheorem{remark**}{Remark}

% Lectures

\newcommand{\lecture}[3]{ % Lecture
  \marginpar{
    Lecture #1 \\
    #2 \\
    #3
  }
}

% Blackboard

\renewcommand{\AA}{\mathbb{A}} % Blackboard A
\newcommand{\BB}{\mathbb{B}}   % Blackboard B
\newcommand{\CC}{\mathbb{C}}   % Blackboard C
\newcommand{\DD}{\mathbb{D}}   % Blackboard D
\newcommand{\EE}{\mathbb{E}}   % Blackboard E
\newcommand{\FF}{\mathbb{F}}   % Blackboard F
\newcommand{\GG}{\mathbb{G}}   % Blackboard G
\newcommand{\HH}{\mathbb{H}}   % Blackboard H
\newcommand{\II}{\mathbb{I}}   % Blackboard I
\newcommand{\JJ}{\mathbb{J}}   % Blackboard J
\newcommand{\KK}{\mathbb{K}}   % Blackboard K
\newcommand{\LL}{\mathbb{L}}   % Blackboard L
\newcommand{\MM}{\mathbb{M}}   % Blackboard M
\newcommand{\NN}{\mathbb{N}}   % Blackboard N
\newcommand{\OO}{\mathbb{O}}   % Blackboard O
\newcommand{\PP}{\mathbb{P}}   % Blackboard P
\newcommand{\QQ}{\mathbb{Q}}   % Blackboard Q
\newcommand{\RR}{\mathbb{R}}   % Blackboard R
\renewcommand{\SS}{\mathbb{S}} % Blackboard S
\newcommand{\TT}{\mathbb{T}}   % Blackboard T
\newcommand{\UU}{\mathbb{U}}   % Blackboard U
\newcommand{\VV}{\mathbb{V}}   % Blackboard V
\newcommand{\WW}{\mathbb{W}}   % Blackboard W
\newcommand{\XX}{\mathbb{X}}   % Blackboard X
\newcommand{\YY}{\mathbb{Y}}   % Blackboard Y
\newcommand{\ZZ}{\mathbb{Z}}   % Blackboard Z

% Brackets

\renewcommand{\eval}[1]{\left. #1 \right|}          % Evaluation
\newcommand{\br}{\del}                              % Brackets
\newcommand{\abr}[1]{\left\langle #1 \right\rangle} % Angle brackets
\newcommand{\fbr}[1]{\left\lfloor #1 \right\rfloor} % Floor brackets
\newcommand{\lbr}[1]{\left\lfloor #1 \right\rfloor} % Ceiling brackets
\newcommand{\st}{\ \middle| \ }                     % Such that

% Calligraphic

\newcommand{\AAA}{\mathcal{A}} % Calligraphic A
\newcommand{\BBB}{\mathcal{B}} % Calligraphic B
\newcommand{\CCC}{\mathcal{C}} % Calligraphic C
\newcommand{\DDD}{\mathcal{D}} % Calligraphic D
\newcommand{\EEE}{\mathcal{E}} % Calligraphic E
\newcommand{\FFF}{\mathcal{F}} % Calligraphic F
\newcommand{\GGG}{\mathcal{G}} % Calligraphic G
\newcommand{\HHH}{\mathcal{H}} % Calligraphic H
\newcommand{\III}{\mathcal{I}} % Calligraphic I
\newcommand{\JJJ}{\mathcal{J}} % Calligraphic J
\newcommand{\KKK}{\mathcal{K}} % Calligraphic K
\newcommand{\LLL}{\mathcal{L}} % Calligraphic L
\newcommand{\MMM}{\mathcal{M}} % Calligraphic M
\newcommand{\NNN}{\mathcal{N}} % Calligraphic N
\newcommand{\OOO}{\mathcal{O}} % Calligraphic O
\newcommand{\PPP}{\mathcal{P}} % Calligraphic P
\newcommand{\QQQ}{\mathcal{Q}} % Calligraphic Q
\newcommand{\RRR}{\mathcal{R}} % Calligraphic R
\newcommand{\SSS}{\mathcal{S}} % Calligraphic S
\newcommand{\TTT}{\mathcal{T}} % Calligraphic T
\newcommand{\UUU}{\mathcal{U}} % Calligraphic U
\newcommand{\VVV}{\mathcal{V}} % Calligraphic V
\newcommand{\WWW}{\mathcal{W}} % Calligraphic W
\newcommand{\XXX}{\mathcal{X}} % Calligraphic X
\newcommand{\YYY}{\mathcal{Y}} % Calligraphic Y
\newcommand{\ZZZ}{\mathcal{Z}} % Calligraphic Z

% Fraktur

\newcommand{\aaa}{\mathfrak{a}}   % Fraktur a
\newcommand{\bbb}{\mathfrak{b}}   % Fraktur b
\newcommand{\ccc}{\mathfrak{c}}   % Fraktur c
\newcommand{\ddd}{\mathfrak{d}}   % Fraktur d
\newcommand{\eee}{\mathfrak{e}}   % Fraktur e
\newcommand{\fff}{\mathfrak{f}}   % Fraktur f
\renewcommand{\ggg}{\mathfrak{g}} % Fraktur g
\newcommand{\hhh}{\mathfrak{h}}   % Fraktur h
\newcommand{\iii}{\mathfrak{i}}   % Fraktur i
\newcommand{\jjj}{\mathfrak{j}}   % Fraktur j
\newcommand{\kkk}{\mathfrak{k}}   % Fraktur k
\renewcommand{\lll}{\mathfrak{l}} % Fraktur l
\newcommand{\mmm}{\mathfrak{m}}   % Fraktur m
\newcommand{\nnn}{\mathfrak{n}}   % Fraktur n
\newcommand{\ooo}{\mathfrak{o}}   % Fraktur o
\newcommand{\ppp}{\mathfrak{p}}   % Fraktur p
\newcommand{\qqq}{\mathfrak{q}}   % Fraktur q
\newcommand{\rrr}{\mathfrak{r}}   % Fraktur r
\newcommand{\sss}{\mathfrak{s}}   % Fraktur s
\newcommand{\ttt}{\mathfrak{t}}   % Fraktur t
\newcommand{\uuu}{\mathfrak{u}}   % Fraktur u
\newcommand{\vvv}{\mathfrak{v}}   % Fraktur v
\newcommand{\www}{\mathfrak{w}}   % Fraktur w
\newcommand{\xxx}{\mathfrak{x}}   % Fraktur x
\newcommand{\yyy}{\mathfrak{y}}   % Fraktur y
\newcommand{\zzz}{\mathfrak{z}}   % Fraktur z

% Geometry

\newcommand{\CP}{\mathbb{CP}}                                              % Complex projective space
\newcommand{\iintd}[4]{\iint_{#1} \, #2 \, \dif #3 \, \dif #4}             % Double integral
\newcommand{\RP}{\mathbb{RP}}                                              % Real projective space
\newcommand{\intd}[4]{\int_{#1}^{#2} \, #3 \, \dif #4}                     % Single integral
\newcommand{\iiintd}[5]{\iint_{#1} \, #2 \, \dif #3 \, \dif #4 \, \dif #5} % Triple integral

% Logic

\newcommand{\iffb}[2]{\br{#1 \leftrightarrow #2}} % Biconditional
\newcommand{\andb}[2]{\br{#1 \land #2}}           % Conjunction
\newcommand{\orb}[2]{\br{#1 \lor #2}}             % Disjunction
\newcommand{\nib}[2]{\br{#1 \notin #2}}           % Element of
\newcommand{\eqb}[2]{\br{#1 = #2}}                % Equal to
\newcommand{\teb}[1]{\br{\exists #1}}             % Existential quantifier
\newcommand{\impb}[2]{\br{#1 \rightarrow #2}}     % Implication
\newcommand{\ltb}[2]{\br{#1 < #2}}                % Less than
\newcommand{\leb}[2]{\br{#1 \le #2}}              % Less than or equal to
\newcommand{\notb}[1]{\br{\neg #1}}               % Negation
\newcommand{\inb}[2]{\br{#1 \in #2}}              % Not element of
\newcommand{\neb}[2]{\br{#1 \ne #2}}              % Not equal to
\newcommand{\subb}[2]{\br{#1 \subseteq #2}}       % Subset
\newcommand{\fab}[1]{\br{\forall #1}}             % Universal quantifier

% Maps

\newcommand{\bijection}[7][]{    % Bijection
  \ifx &#1&
    \begin{array}{rcl}
      #2 & \longleftrightarrow & #3 \\
      #4 & \longmapsto         & #5 \\
      #6 & \longmapsfrom       & #7
    \end{array}
  \else
    \begin{array}{ccrcl}
      #1 & : & #2 & \longrightarrow & #3 \\
         &   & #4 & \longmapsto     & #5 \\
         &   & #6 & \longmapsfrom   & #7
    \end{array}
  \fi
}
\newcommand{\birational}[7][]{   % Birational map
  \ifx &#1&
    \begin{array}{rcl}
      #2 & \dashrightarrow & #3 \\
      #4 & \longmapsto     & #5 \\
      #6 & \longmapsfrom   & #7
    \end{array}
  \else
    \begin{array}{ccrcl}
      #1 & : & #2 & \dashrightarrow & #3 \\
         &   & #4 & \longmapsto     & #5 \\
         &   & #6 & \longmapsfrom   & #7
    \end{array}
  \fi
}
\newcommand{\correspondence}[2]{ % Correspondence
  \cbr{
    \begin{array}{c}
      #1
    \end{array}
  }
  \qquad
  \leftrightsquigarrow
  \qquad
  \cbr{
    \begin{array}{c}
      #2
    \end{array}
  }
}
\newcommand{\function}[5][]{     % Function
  \ifx &#1&
    \begin{array}{rcl}
      #2 & \longrightarrow & #3 \\
      #4 & \longmapsto     & #5
    \end{array}
  \else
    \begin{array}{ccrcl}
      #1 & : & #2 & \longrightarrow & #3 \\
         &   & #4 & \longmapsto     & #5
    \end{array}
  \fi
}
\newcommand{\functions}[7][]{    % Functions
  \ifx &#1&
    \begin{array}{rcl}
      #2 & \longrightarrow & #3 \\
      #4 & \longmapsto     & #5 \\
      #6 & \longmapsto     & #7
    \end{array}
  \else
    \begin{array}{ccrcl}
      #1 & : & #2 & \longrightarrow & #3 \\
         &   & #4 & \longmapsto     & #5 \\
         &   & #6 & \longmapsto     & #7
    \end{array}
  \fi
}
\newcommand{\rational}[5][]{     % Rational map
  \ifx &#1&
    \begin{array}{rcl}
      #2 & \dashrightarrow & #3 \\
      #4 & \longmapsto     & #5
    \end{array}
  \else
    \begin{array}{ccrcl}
      #1 & : & #2 & \dashrightarrow & #3 \\
         &   & #4 & \longmapsto     & #5
    \end{array}
  \fi
}

% Matrices

\newcommand{\onebytwo}[2]{      % One by two matrix
  \begin{pmatrix}
    #1 & #2
  \end{pmatrix}
}
\newcommand{\onebythree}[3]{    % One by three matrix
  \begin{pmatrix}
    #1 & #2 & #3
  \end{pmatrix}
}
\newcommand{\twobyone}[2]{      % Two by one matrix
  \begin{pmatrix}
    #1 \\
    #2
  \end{pmatrix}
}
\newcommand{\twobytwo}[4]{      % Two by two matrix
  \begin{pmatrix}
    #1 & #2 \\
    #3 & #4
  \end{pmatrix}
}
\newcommand{\threebyone}[3]{    % Three by one matrix
  \begin{pmatrix}
    #1 \\
    #2 \\
    #3
  \end{pmatrix}
}
\newcommand{\threebythree}[9]{  % Three by three matrix
  \begin{pmatrix}
    #1 & #2 & #3 \\
    #4 & #5 & #6 \\
    #7 & #8 & #9
  \end{pmatrix}
}
\newcommand{\twobytwosmall}[4]{ % Two by two small matrix
  \begin{psmallmatrix}
    #1 & #2 \\
    #3 & #4
  \end{psmallmatrix}
}

% Number theory

\renewcommand{\symbol}[2]{\br{\tfrac{#1}{#2}}} % Power residue symbol
\newcommand{\unit}[1]{\br{\ZZ / #1\ZZ}^\times} % Unit group

% Operators

\newoperator{ab}    % Abelian
\newoperator{AG}    % Affine geometry
\newoperator{alg}   % Algebraic
\newoperator{Ann}   % Annihilator
\newoperator{area}  % Area
\newoperator{Aut}   % Automorphism
\newoperator{card}  % Cardinality
\newoperator{ch}    % Characteristic
\newoperator{Cl}    % Class
\newoperator{Coker} % Cokernel
\newoperator{col}   % Column
\newoperator{Corr}  % Correspondence
\newoperator{diam}  % Diameter
\newoperator{Disc}  % Discriminant
\newoperator{dom}   % Domain
\newoperator{Eig}   % Eigenvalue
\newoperator{Em}    % Embedding
\newoperator{End}   % Endomorphism
\newoperator{fin}   % Finite
\newoperator{Fix}   % Fixed
\newoperator{Frac}  % Fraction
\newoperator{Frob}  % Frobenius
\newoperator{Fun}   % Function
\newoperator{Gal}   % Galois
\newoperator{GL}    % General linear
\newoperator{Ham}   % Hamming
\newoperator{Homeo} % Homeomorphism
\newoperator{Hom}   % Homomorphism
\newoperator{id}    % Identity
\newoperator{Im}    % Image
\newoperator{Ind}   % Index
\newoperator{Ker}   % Kernel
\newoperator{lcm}   % Least common multiple
\newoperator{Mat}   % Matrix
\newoperator{mult}  % Multiplicity
\newoperator{new}   % New
\newoperator{Nm}    % Norm
\newoperator{old}   % Old
\newoperator{op}    % Opposite
\newoperator{ord}   % Order
\newoperator{Pay}   % Payley
\newoperator{PG}    % Projective geometry
\newoperator{PGL}   % Projective general linear
\newoperator{PSL}   % Projective special linear
\newoperator{rad}   % Radical
\newoperator{ran}   % Range
\newoperator{Res}   % Residue
\newoperator{rk}    % Rank
\newoperator{Re}    % Real
\newoperator{row}   % Row
\newoperator{sgn}   % Sign
\newoperator{Sing}  % Singular
\newoperator{SK}    % Skeleton
\newoperator{sp}    % Span
\newoperator{SL}    % Special linear
\newoperator{SO}    % Special orthogonal
\newoperator{Spec}  % Spectrum
\newoperator{Stab}  % Stabiliser
\newoperator{star}  % Star
\newoperator{srg}   % Strongly regular graph
\newoperator{supp}  % Support
\newoperator{Sym}   % Symmetric
\newoperator{tors}  % Torsion
\newoperator{Tr}    % Trace
\newoperator{vol}   % Volume
\newoperator{wt}    % Weight

% Roman

\newcommand{\A}{\mathrm{A}}   % Roman A
\newcommand{\B}{\mathrm{B}}   % Roman B
\newcommand{\C}{\mathrm{C}}   % Roman C
\newcommand{\D}{\mathrm{D}}   % Roman D
\newcommand{\E}{\mathrm{E}}   % Roman E
\newcommand{\F}{\mathrm{F}}   % Roman F
\newcommand{\G}{\mathrm{G}}   % Roman G
\renewcommand{\H}{\mathrm{H}} % Roman H
\newcommand{\I}{\mathrm{I}}   % Roman I
\newcommand{\J}{\mathrm{J}}   % Roman J
\newcommand{\K}{\mathrm{K}}   % Roman K
\renewcommand{\L}{\mathrm{L}} % Roman L
\newcommand{\M}{\mathrm{M}}   % Roman M
\newcommand{\N}{\mathrm{N}}   % Roman N
\renewcommand{\O}{\mathrm{O}} % Roman O
\renewcommand{\P}{\mathrm{P}} % Roman P
\newcommand{\Q}{\mathrm{Q}}   % Roman Q
\newcommand{\R}{\mathrm{R}}   % Roman R
\renewcommand{\S}{\mathrm{S}} % Roman S
\newcommand{\T}{\mathrm{T}}   % Roman T
\newcommand{\U}{\mathrm{U}}   % Roman U
\newcommand{\V}{\mathrm{V}}   % Roman V
\newcommand{\W}{\mathrm{W}}   % Roman W
\newcommand{\X}{\mathrm{X}}   % Roman X
\newcommand{\Y}{\mathrm{Y}}   % Roman Y
\newcommand{\Z}{\mathrm{Z}}   % Roman Z

\renewcommand{\a}{\mathrm{a}} % Roman a
\renewcommand{\b}{\mathrm{b}} % Roman b
\renewcommand{\c}{\mathrm{c}} % Roman c
\renewcommand{\d}{\mathrm{d}} % Roman d
\newcommand{\e}{\mathrm{e}}   % Roman e
\newcommand{\f}{\mathrm{f}}   % Roman f
\newcommand{\g}{\mathrm{g}}   % Roman g
\newcommand{\h}{\mathrm{h}}   % Roman h
\renewcommand{\i}{\mathrm{i}} % Roman i
\renewcommand{\j}{\mathrm{j}} % Roman j
\renewcommand{\k}{\mathrm{k}} % Roman k
\renewcommand{\l}{\mathrm{l}} % Roman l
\newcommand{\m}{\mathrm{m}}   % Roman m
\renewcommand{\n}{\mathrm{n}} % Roman n
\renewcommand{\o}{\mathrm{o}} % Roman o
\newcommand{\p}{\mathrm{p}}   % Roman p
\newcommand{\q}{\mathrm{q}}   % Roman q
\renewcommand{\r}{\mathrm{r}} % Roman r
\newcommand{\s}{\mathrm{s}}   % Roman s
\renewcommand{\t}{\mathrm{t}} % Roman t
\renewcommand{\u}{\mathrm{u}} % Roman u
\renewcommand{\v}{\mathrm{v}} % Roman v
\newcommand{\w}{\mathrm{w}}   % Roman w
\newcommand{\x}{\mathrm{x}}   % Roman x
\newcommand{\y}{\mathrm{y}}   % Roman y
\newcommand{\z}{\mathrm{z}}   % Roman z

% Tikz

\tikzset{
  arrow symbol/.style={"#1" description, allow upside down, auto=false, draw=none, sloped},
  subset/.style={arrow symbol={\subset}},
  cong/.style={arrow symbol={\cong}}
}

% Fancy header

\pagestyle{fancy}
\lhead{\module}
\rhead{\nouppercase{\leftmark}}

% Make title

\title{\module}
\author{Lectured by \lecturer \\ Typed by David Kurniadi Angdinata}
\date{\term}

\begin{document}

% Title page
\maketitle
\cover
\vfill
\begin{abstract}
\noindent\syllabus
\end{abstract}

\pagebreak

% Contents page
\tableofcontents

\pagebreak

% Document page
\setcounter{section}{-1}

\setcounter{section}{0}

\section{Modules}

\subsection{Modules over rings}

\lecture{1}{Friday}{10/01/20}

Let $ R $ be an \textbf{associative ring with unity}, that is an abelian group written additively with a multiplication which is associative but not necessarily commutative, with an identity $ 1 $ and distributive laws $ a\br{b + c} = ab + ac $ and $ \br{a + b}c = ac + bc $. Then
$$ R^* = \cbr{r \in R \st \exists s \in R, \ rs = 1 = sr} $$
is the unit group of $ R $. If $ R^* = R \setminus \cbr{0} $ then $ R $ is a \textbf{division ring}, or a \textbf{skew field}. In the case that $ R $ is commutative, $ R $ is a \textbf{field}.

\begin{example*}
\hfill
\begin{itemize}
\item Fields $ \CC $, $ \RR $, $ \QQ $, and $ \FF_q $, the field with $ q = p^a $ elements with $ p $ a prime and $ a \ge 1 $.
\item Skew fields $ \HH = \cbr{a + bi + cj + dk \st a, b, c, d \in \RR} $ where $ i^2 = j^2 = k^2 = ijk = -1 $.
\item Other rings are polynomial rings $ k\sbr{x} $ for $ k $ a field, more generally $ k\sbr{x_1, \dots, x_p} $, and $ \Mat_n k $, the $ n \times n $ matrices with entries from $ k $, a field.
\end{itemize}
\end{example*}

\begin{definition}
Let $ R $ be a ring. A \textbf{left $ R $-module} is an abelian group $ M $, written additively, together with a function $ * : R \times M \to M $ satisfying
$$ r * \br{m_1 + m_2} = r * m_1 + r * m_2, \qquad \br{r_1 + r_2} * m = r_1 * m + r_2 * m, \qquad \br{r_1r_2} * m = r_1 * \br{r_2 * m}, \qquad 1 * m = m. $$
\end{definition}

We write $ rm $ for $ r * m $.

\begin{example*}
\hfill
\begin{itemize}
\item $ R $ is itself a left $ R $-module, with $ * $ as ring multiplication. More generally, let $ I $ be a left ideal of $ R $, so $ I $ is an additive subgroup, and $ rI \subseteq I $ for all $ r \in R $. Then $ I $ is an $ R $-module with $ * $ as ring multiplication.
\item Let $ k $ be a field. Then any vector space over $ k $ is a $ k $-module, and vice versa.
\item Any abelian group is a $ \ZZ $-module, with $ * $ defined by $ na = a + \dots + a $ for $ n \in \ZZ^+ $ and $ a \in A $, and $ \br{-n}a = -\br{na} $.
\item Let $ k $ be a field. Let $ k^n $ be column vectors. Then $ k^n $ is a left $ \Mat_n k $-module, with $ * $ as the usual matrix-vector multiplication.
\item Let $ M \in \Mat_n k $. Then we can define a left $ k\sbr{x} $-module structure on $ k^* $ by letting $ x $ act as $ M $ on $ k^* $. So $ \br{x^2 + 3x - 2} * v = M^2v + 3Mv - 2v $.
\item Let $ G $ be a group. Any representation of $ G $ over the field $ k $ is a left module for $ k\sbr{G} $, the \textbf{group algebra}, a vector space over $ k $ with elements of $ G $ as a basis, with multiplication derived from that of $ G $.
\end{itemize}
\end{example*}

\begin{definition}
A \textbf{right $ R $-module} is defined similarly, with the $ R $-multiplication on the right, so $ M $ an abelian group under $ + $, and a map $ M \times R \to M $ satisfying
$$ \br{m_1 + m_2} * r = m_1 * r + m_2 * r, \qquad m * \br{r_1 + r_2} = m * r_1 + m * r_2, \qquad m * \br{r_1r_2} = \br{m * r_1} * r_2, \qquad m * 1 = m. $$
\end{definition}

Left and right modules are not quite the same. If we amend this definition by putting the ring multiplication on the left, the third axiom becomes $ \br{r_1r_2}m = r_2\br{r_1m} $. But in a left module, we have $ \br{r_1r_2}m = r_1\br{r_2m} $.

\begin{definition}
Let $ R $ be a ring. The \textbf{opposite ring} $ R^{\op} $ is $ R $ with a redefined multiplication $ r *_{R^{\op}} s = s *_R r $.
\end{definition}

It is easy to see that a left $ R $-module is the same as a right $ R^{\op} $-module and vice versa. If $ R $ is commutative then $ R = R^{\op} $.

\begin{exercise*}
Show that $ \Mat_n k \cong \Mat_n k^{\op} $.
\end{exercise*}

Except where otherwise stated, $ R $-modules are assumed to be left $ R $-modules.

\pagebreak

\begin{definition}
Let $ M_1 $ and $ M_2 $ be $ R $-modules. A map $ f : M_1 \to M_2 $ is an \textbf{$ R $-module homomorphism} if
\begin{itemize}
\item $ f $ is a group homomorphism, with respect to the $ + $ operation, and
\item $ f\br{rm} = rf\br{m} $, for $ r \in R $ and $ m \in M $.
\end{itemize}
If $ f $ is bijective, then it is an \textbf{$ R $-module isomorphism}.
\end{definition}

\begin{definition}
An additive subgroup $ L \le M $ is a \textbf{submodule} if $ rL \le L $ for $ r \in R $. In this case we automatically get an $ R $-module structure on the quotient $ M / L $ with multiplication given by $ r\br{m + L} = rm + L $.
\end{definition}

\begin{theorem}[First isomorphism theorem]
Let $ f : M_1 \to M_2 $ be an $ R $-module homomorphism. Then $ \Im f \le M_2 $, $ \Ker f \le M_1 $, and $ \Im f \cong M / \Ker f $.
\end{theorem}

The other isomorphism theorems have $ R $-module versions too.

\lecture{2}{Monday}{13/01/20}

Let $ S $ be a set. We have a collection of $ R $-modules $ \br{M_s}_S $ indexed by $ S $.

\begin{definition}
The \textbf{direct product} is
$$ \prod_{s \in S} M_s = \cbr{\br{m_s}_S \st m_s \in M_s}, $$
with coordinate-wise addition and $ R $-multiplication, so
$$ \br{m_s}_S + \br{n_s}_S = \br{m_s + n_s}_S, \qquad r\br{m_s}_S = \br{rm_s}_S. $$
If $ M_s = M $ for all $ s \in S $, then we write $ M^S $ for $ \prod_{s \in S} M_s $. The \textbf{direct sum} is
$$ \bigoplus_{s \in S} M_s = \cbr{\br{m_s}_S \st \text{all but finitely many coordinates} \ m_s \ \text{are zero}} \le \prod_{s \in S} M_s. $$
If $ S $ is finite then the direct product and the direct sum are equal.
\end{definition}

\begin{example*}
Let $ M = \ZZ_2 $, as a $ \ZZ $-module, and let $ S = \NN $. Then $ \bigoplus_{s \in \NN} \ZZ_2 $ is a countable $ \ZZ $-module but $ \prod_{s \in \NN} \ZZ_2 = \ZZ_2^\NN $ is uncountable.
\end{example*}

When $ \abs{S} = 2 $, generally we write $ M_1 \oplus M_2 $ for the direct sum or product. There are natural injective maps
$$ \function[\iota_A]{A}{A \oplus B}{a}{\br{a, 0}}, \qquad \function[\iota_B]{B}{A \oplus B}{b}{\br{0, b}}, $$
and surjective maps
$$ \function[\pi_A]{A \oplus B}{A}{\br{a, b}}{a}, \qquad \function[\pi_B]{A \oplus B}{B}{\br{a, b}}{b}. $$

\subsection{Exact sequences}

\begin{definition}
Suppose we have a sequence of $ R $-modules
$$ \dots \xrightarrow{f_{n - 1}} M_n \xrightarrow{f_n} M_{n + 1} \xrightarrow{f_{n + 1}} \dots, $$
with maps $ f_n : M_n \to M_{n + 1} $. Say the sequence is \textbf{exact at $ M_n $} if
$$ \Im f_{n - 1} = \Ker f_n. $$
The sequence is \textbf{exact} if it is exact everywhere. A \textbf{short exact sequence} is an exact sequence
$$ 0 \to A \xrightarrow{\alpha} B \xrightarrow{\beta} C \to 0. $$
Note that $ \alpha $ is injective and $ \beta $ is surjective. The first isomorphism theorem implies that $ B / \Im \alpha \cong C $, where $ \Im \alpha \cong A $. An easy case is
$$ B \cong A \oplus C, $$
with $ \Im \alpha = A \oplus 0 $ and $ \Im \beta = C $, so $ \alpha = \iota_A $ and $ \beta = \pi_\beta $. We say that the short exact sequence \textbf{splits} in this case.
\end{definition}

\pagebreak

\begin{example*}
A non-split short exact sequence of $ \ZZ $-modules, or abelian groups, is
$$ 0 \to \ZZ / 2\ZZ \to \ZZ / 4\ZZ \to \ZZ / 2\ZZ \to 0. $$
\end{example*}

\begin{proposition}
A short exact sequence
$$ 0 \to A \xrightarrow{\alpha} B \xrightarrow{\beta} C \to 0 $$
is split if and only if there exists an $ R $-module homomorphism $ \sigma : C \to B $ such that $ \beta \circ \sigma = \id_C $.
\end{proposition}

Such a $ \sigma $ is called a \textbf{section} of $ \beta $.

\begin{proof}
\hfill
\begin{itemize}
\item[$ \implies $] Suppose that the short exact sequence is split. So assume $ B = A \oplus C $, with $ \alpha = \iota_A $ and $ \beta = \pi_C $. Now $ \iota_C $ is a section for $ \beta $.
\item[$ \impliedby $] For the converse, suppose that $ \sigma $ is a section for $ \beta $. We want $ f : A \oplus C \xrightarrow{\sim} B $ such that $ f \circ \iota_A = \alpha $ and $ \beta \circ f = \pi_C $, so
$$
\begin{tikzcd}
& & A \oplus C \arrow{dr}{\pi_C} \arrow{dd}{f} & & \\
0 \arrow{r} & A \arrow{ur}{\iota_A} \arrow[swap]{dr}{\alpha} & & C \arrow{r} & 0 \\
& & B \arrow[swap]{ur}{\beta} & &
\end{tikzcd}.
$$
Define
$$ \function[f]{A \times C}{B}{\br{a, c}}{\alpha\br{a} + \sigma\br{c}}. $$
Need to check the following.
\begin{itemize}
\item $ f $ is an $ R $-module homomorphism. \footnote{Exercise}
\item $ f $ is injective. Suppose $ f\br{a, c} = 0 $. Then $ \alpha\br{a} + \sigma\br{c} = 0 $. Now $ \alpha\br{a} \in \Im \alpha = \Ker \beta $, so $ \beta\br{\alpha\br{a} + \sigma\br{c}} = \beta\br{\sigma\br{c}} = c $. Since $ \alpha\br{a} + \sigma\br{c} = 0 $, we have $ c = 0 $. Hence $ \alpha\br{a} = 0 $, and so $ a = 0 $ since $ \alpha $ is injective. We have shown that $ f $ is injective.
\item $ f $ is surjective. Let $ b \in B $. Let $ c = \beta\br{b} $. We have $ \br{\beta \circ \sigma}\br{c} = c = \beta\br{b} $, so $ b - \sigma\br{c} \in \Ker \beta = \Im \alpha $. So there exists $ a \in A $ with $ \alpha\br{a} = b - \sigma\br{c} $. Then $ b = \alpha\br{a} + \sigma\br{c} = f\br{a, c} $.
\item $ f \circ \iota_A = \alpha $ and $ \beta \circ f = \pi_C $. Immediate from the construction of $ f $.
\end{itemize}
\end{itemize}
\end{proof}

\begin{proposition}
The short exact sequence
$$ 0 \to A \xrightarrow{\alpha} B \xrightarrow{\beta} C \to 0 $$
is split if and only if there exists $ \rho : B \to A $ such that $ \rho \circ \alpha = \id_A $.
\end{proposition}

Such a $ \rho $ is a \textbf{retraction} of $ \alpha $.

\begin{proof}
\hfill
\begin{itemize}
\item[$ \implies $] Once again, if the short exact sequence is split then the existence of $ \rho $ is clear.
\item[$ \impliedby $] Suppose that $ \rho $ is a retraction for $ \alpha $. We define $ f : B \xrightarrow{\sim} A \oplus C $ such that $ f \circ \alpha = \iota_A $ and $ \pi_C \circ f = \beta $. Do this by
$$ \function[g]{B}{A \oplus C}{b}{\br{\rho\br{a}, \beta\br{c}}}. $$
Details are omitted.
\end{itemize}
\end{proof}

\pagebreak

\subsection{Projective modules}

\lecture{3}{Tuesday}{14/01/20}

\begin{definition}
An $ R $-module $ M $ is \textbf{projective} if any surjective map $ \beta : B \to M $ has a section. In other words, any short exact sequence
$$ 0 \to A \to B \to M \to 0 $$
splits.
\end{definition}

\begin{example*}
The $ R $-module $ R $ is projective. Let
$$ 0 \to A \to B \xrightarrow{\beta} R \to 0 $$
be a short exact sequence. Since $ \beta $ is surjective, there exists $ b \in B $ such that $ \beta\br{b} = 1 $. Now for all $ r \in R $, $ \beta\br{rb} = r $. Now define
$$ \function[\sigma]{R}{B}{r}{rb}. $$
Then $ \sigma $ is a section for $ \beta $.
\end{example*}

\begin{proposition}
An $ R $-module $ M $ is projective if and only if whenever $ \beta : B \to C $ is surjective, and $ f : M \to C $, there exists $ g : M \to B $ such that $ f = \beta \circ g $, so
$$
\begin{tikzcd}
& & & M \arrow[dashed, swap]{dl}{g} \arrow{d}{f} & \\
0 \arrow{r} & A \arrow{r} & B \arrow[swap]{r}{\beta} & C \arrow{r} & 0
\end{tikzcd}.
$$
\end{proposition}

Such a $ g $ is called a \textbf{lift} of $ f $.

\begin{proof}
\hfill
\begin{itemize}
\item[$ \impliedby $] Suppose that whenever $ \beta : B \to C $ is surjective and $ f : M \to C $ then there exists $ g : M \to B $ with $ f = \beta \circ g $. Suppose $ \beta : B \to M $ is a surjective map. Define $ f : M \to M $ to be $ \id_M $. Then there exists $ g : M \to B $ such that $ f = \beta \circ g $, so $ \id_M = \beta \circ g $. So $ g $ is a section for $ \beta $, and so $ M $ is projective.
\item[$ \implies $] For the converse, suppose $ \beta : B \to C $ is surjective, and $ f : M \to C $. We construct a module $ X $ to complete a commuting square
$$
\begin{tikzcd}
X \arrow{r}{\epsilon} \arrow[swap]{d}{\delta} & M \arrow{d}{f} \\
B \arrow[swap]{r}{\beta} & C
\end{tikzcd}.
$$
Let $ X $ be the submodule of $ B \oplus M $ defined by
$$ X = \cbr{\br{b, m} \st \beta\br{b} = f\br{m}}. $$
The maps $ \delta $ and $ \epsilon $ are just $ \pi_B $ and $ \pi_M $ respectively, in their restrictions to $ X $. It is clear that $ X \le B \oplus M $, and that the square above commutes. Now suppose that $ M $ is projective. Since $ \beta $ is surjective, we see that for all $ m \in M $ there exists $ b \in B $ with $ \beta\br{b} = f\br{m} $. It follows that $ \epsilon : X \to M $ is surjective. So $ \epsilon $ has a section $ \sigma : M \to X $. Define $ g = \delta \circ \sigma : M \to B $, so
$$
\begin{tikzcd}
X \arrow[bend left=15]{r}{\epsilon} \arrow[swap]{d}{\delta} & M \arrow[bend left=15, dashed]{l}{\sigma} \arrow[dashed]{dl}{g} \arrow{d}{f} \\
B \arrow[swap]{r}{\beta} & C
\end{tikzcd}.
$$
Since $ \beta \circ \delta = f \circ \epsilon $, for all $ m \in M $ we have
$$ \br{\beta \circ g}\br{m} = \br{\beta \circ \delta \circ \sigma}\br{m} = \br{f \circ \epsilon \circ \sigma}\br{m} = \br{f \circ \id_M}\br{m} = f\br{m}. $$
So $ \beta \circ g = f $ as required.
\end{itemize}
\end{proof}

\pagebreak

Such an $ X $ is the \textbf{pullback} of $ \beta $ and $ f $, and there is a short exact sequence
$$ 0 \to A \to X \to M \to 0. $$

\begin{definition}
An $ R $-module $ M $ is \textbf{free} if $ M $ is a direct sum of copies of $ R $, so
$$ M = \bigoplus_{s \in S} R. $$
A \textbf{basis} for a module $ M $ is a set $ T $ of elements such that every element $ m \in M $ has a unique expression as
$$ m = \sum_{i = 1}^m r_it_i, \qquad r_i \in R, \qquad t_i \in T. $$
\end{definition}

If $ M = \bigoplus_{s \in S} R $, then $ M $ has a basis consisting of elements with exactly one coordinate one, and the rest zero. On the other hand, if $ M $ has a basis $ T $ then it is straightforward to show that $ M \cong \bigoplus_{t \in T} R $.

\begin{proposition}
Let $ F $ be a free $ R $-module with basis $ T $. Let $ M $ be some $ R $-module, and let $ \psi : T \to M $ be a set map. Then $ \psi $ extends uniquely to a $ R $-module homomorphism $ \psi : F \to M $.
\end{proposition}

\begin{proof}
Each element of $ F $ has a unique expression as $ \sum_i r_it_i $ for $ r_i \in R $ and $ t_i \in T $. Now define
$$ \function[\psi]{F}{M}{\sum_i r_it_i}{\sum_i r_i\psi\br{t_i}}. $$
It is easy to check that this respects $ + $ and $ R $-multiplication.
\end{proof}

\begin{proposition}
A module $ M $ is projective if and only if there exists $ N $ such that $ M \oplus N $ is free, so projective modules are direct summands of free modules.
\end{proposition}

\begin{proof}
\hfill
\begin{itemize}
\item[$ \implies $] Suppose $ M $ is projective. Let $ F $ be the free module with basis $ \cbr{b_m \st m \in M} $. Now the map $ b_m \mapsto m $ extends to an $ R $-module homomorphism $ F \to M $, which is clearly surjective. Then if $ K = \Ker \psi $, we have a short exact sequence
$$ 0 \to K \to F \xrightarrow{\psi} M \to 0. $$
Since $ M $ is projective, there is a section $ \sigma $ for $ \psi $, and so the short exact sequence splits, and $ F \cong K \oplus M $.

\lecture{4}{Friday}{17/01/20}

\item[$ \impliedby $] Suppose that $ M \oplus N = F $, a free module with basis $ T $. Suppose $ \beta : B \to C $ is surjective, and that $ f : M \to C $. Note that $ f \circ \pi_M : F \to C $. For each $ t \in T $, let $ b_t \in B $ be such that $ \beta\br{b_t} = \br{f \circ \pi_M}\br{t} $. The set map
$$ \function{T}{B}{t}{b_t} $$
extends to a homomorphism $ \widehat{g} : F \to B $. Now define $ g : M \to B $ by $ g = \widehat{g} \circ \iota_M $. We need to show $ f = \beta \circ g $. Take $ m \in M $. Then $ \iota_M\br{m} = \br{m, 0} \in F $ can be written as $ \sum_i r_it_i $, where $ t_i \in T $ and $ r_i \in R $. Applying $ \pi_M $, $ m = \sum_i r_im_{t_i} $. Then
$$ g\br{m} = \br{\widehat{g} \circ \iota_M}\br{m} = \widehat{g}\br{\sum_i r_it_i} = \sum_i r_ib_{t_i}. $$
So
$$ \br{\beta \circ g}\br{m} = \beta\br{\sum_i r_ib_{t_i}} = \sum_i r_i\beta\br{b_{t_i}} = \sum_i r_if\br{m_{t_i}} = f\br{\sum_i r_im_{t_i}} = f\br{m}. $$
Hence $ \beta \circ g = f $. So $ M $ is projective.
\end{itemize}
\end{proof}

\pagebreak

\subsection{Injective modules}

\begin{definition}
Let $ M $ be an $ R $-module. Then $ M $ is \textbf{injective} if whenever $ \alpha : M \to B $ is an injective map, it has a retraction $ \rho : B \to M $, so $ \rho \circ \alpha = \id_M $. Equivalently, every short exact sequence
$$ 0 \to M \to B \to C \to 0 $$
splits.
\end{definition}

\begin{example*}
Let $ k $ be a field. Then $ k $-modules are vector spaces. Every $ k $-module is injective. Suppose $ M $ and $ N $ are $ k $-vector spaces and $ \alpha : M \to N $ is a injective map. Then $ \Im \alpha $ is a submodule, or subspace, of $ N $. Take a basis for $ \Im \alpha $, and extend to a basis for $ N $. The basis vectors not in $ \Im \alpha $ form a basis for a complementary subspace $ U $, so $ N = \Im \alpha \oplus U $. Now $ \pi_{\Im \alpha} $ is surjective, and $ \alpha : M \to \Im \alpha $ is an isomorphism. This gives a retraction $ N \to M $.
\end{example*}

If $ R $ is a general ring, the module $ R $ need not be injective.

\begin{example*}
Let $ R = \ZZ $. Then $ R $-modules are abelian groups. There exists an injective $ \alpha : \ZZ \to \QQ $. But $ \ZZ $ is not a quotient of $ \QQ $, \footnote{Exercise} so no retraction exists for $ \alpha $.
\end{example*}

\begin{proposition}
An $ R $-module $ M $ is injective if and only if whenever $ \alpha : A \to B $ is injective, and $ f : A \to M $, there exists $ g : B \to M $ such that $ f = g \circ \alpha $.
\end{proposition}

\begin{proof}
\hfill
\begin{itemize}
\item[$ \impliedby $] Suppose that whenever $ \alpha : A \to B $ is injective, and $ f : A \to M $, there exists $ g : B \to M $ such that $ f = g \circ \alpha $. Suppose that $ \alpha : M \to B $ is injective. We have a map $ M \to M $, namely $ \id_M $. There exists $ g : B \to M $ such that $ \id_M = g \circ \alpha $. So $ g $ is a retraction for $ \alpha $, and so $ M $ is injective.
\item[$ \implies $] For the converse, suppose $ \alpha : A \to B $ is injective, and $ M $ is an injective module, with $ f : A \to M $. We define a module $ Y $ completing a square
$$
\begin{tikzcd}
A \arrow{r}{\alpha} \arrow[swap]{d}{f} & B \arrow{d}{\delta} \\
M \arrow[swap]{r}{\epsilon} & Y
\end{tikzcd},
$$
with $ \epsilon \circ f = \delta \circ \alpha $. Let $ Y $ be a quotient of $ B \oplus M $, by the kernel
$$ K = \cbr{\br{\alpha\br{a}, -f\br{a}} \st a \in A}. $$
Let $ \gamma : B \oplus M \to \br{B \oplus M} / K $ be the canonical quotient map. Then we define $ \delta = \gamma \circ \iota_B $ and $ \epsilon = \gamma \circ \iota_M $. By construction, we have
\begin{align*}
\br{\epsilon \circ f}\br{a}
& = \br{\gamma \circ \iota_M \circ f}\br{a}
= \gamma\br{0, f\br{a}}
= \br{0, f\br{a}} + K \\
& = \br{\alpha\br{a}, 0} + K
= \gamma\br{\alpha\br{a}, 0}
= \br{\gamma \circ \iota_B \circ \alpha}\br{a}
= \br{\delta \circ \alpha}\br{a}.
\end{align*}
Hence $ \epsilon \circ f = \delta \circ \alpha $. Claim that $ \epsilon $ is injective. Suppose $ \epsilon\br{m} = 0 $. Then $ \iota_M\br{m} \in K $, so $ \br{0, m} = \br{\alpha\br{a}, -f\br{a}} $ for some $ a \in A $. But $ \alpha\br{a} = 0 $ implies that $ a = 0 $, and so $ m = -f\br{0} = 0 $. Since $ M $ is injective, $ \epsilon $ has a retraction $ \rho : Y \to M $. Define $ g : B \to M $ by $ g = \rho \circ \delta $, so
$$
\begin{tikzcd}
A \arrow{r}{\alpha} \arrow[swap]{d}{f} & B \arrow[dashed, swap]{dl}{g} \arrow{d}{\delta} \\
M \arrow[bend right=15, swap]{r}{\epsilon} & Y \arrow[bend right=15, dashed, swap]{l}{\rho}
\end{tikzcd},
$$
We know that $ \br{\epsilon \circ f}\br{a} = \br{\delta \circ \alpha}\br{a} $ for all $ a \in A $. So
$$ f\br{a} = \br{\id_M \circ f}\br{a} = \br{\rho \circ \epsilon \circ f}\br{a} = \br{\rho \circ \delta \circ \alpha}\br{a} = \br{g \circ \alpha}\br{a}, $$
so $ f = g \circ \alpha $ as required.
\end{itemize}
\end{proof}

\pagebreak

We know that projectives are direct summands of free modules. We might hope for a dual version of this for injective modules. But there is no straightforward way of doing this.

\end{document}