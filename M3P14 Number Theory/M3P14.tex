\documentclass{article}

\usepackage{amssymb}
\usepackage{amsthm}
\usepackage[UKenglish]{babel}
\usepackage{enumitem}
\usepackage{fancyhdr}
\usepackage[margin=1in]{geometry}
\usepackage{graphicx}
\usepackage[utf8]{inputenc}
\usepackage{listings}
\usepackage{mathtools}
\usepackage{tikz-cd}
\usepackage{csquotes}

\newcommand{\F}{\mathbb{F}}
\newcommand{\N}{\mathbb{N}}
\newcommand{\Z}{\mathbb{Z}}
\newcommand{\Q}{\mathbb{Q}}
\newcommand{\R}{\mathbb{R}}
\newcommand{\C}{\mathbb{C}}
\newcommand{\A}{\mathbb{A}}
\renewcommand{\P}{\mathbb{P}}

\newcommand{\val}[1]{\left. #1 \right\rvert}
\newcommand{\rb}[1]{\left( #1 \right)}
\renewcommand{\sb}[1]{\left[ #1 \right]}
\newcommand{\cb}[1]{\left\{ #1 \right\}}
\newcommand{\ab}[1]{\left\langle #1 \right\rangle}
\newcommand{\abs}[1]{\left\lvert #1 \right\rvert}
\newcommand{\two}[2]{\begin{pmatrix} #1 \\ #2 \end{pmatrix}}
\newcommand{\three}[3]{\begin{pmatrix} #1 & #2 & #3 \end{pmatrix}}

\theoremstyle{definition}\newtheorem{definition}{Definition}
\theoremstyle{definition}\newtheorem*{remark}{Remark}
\theoremstyle{definition}\newtheorem*{example}{Example}
\theoremstyle{definition}\newtheorem*{note}{Note}
\newtheorem{proposition}[definition]{Proposition}
\newtheorem{lemma}[definition]{Lemma}
\newtheorem{theorem}[definition]{Theorem}
\newtheorem{corollary}[definition]{Corollary}

\pagestyle{fancy}
\lhead{M3P14 Number Theory}
\rhead{Autumn 2018}

\title{M3P14 Number Theory}
\author{Lectured by Prof Toby Gee \\ Typeset by David Kurniadi Angdinata}
\date{Autumn 2018}

\setcounter{section}{-1}

\begin{document}

\maketitle

\vfill

\tableofcontents

\pagebreak

\marginpar{Lecture 1 \\ Friday \\ 05/10/18}

\section{Introduction}

Roughly speaking number theory is the study of the integers. More specifically, problems in number theory often have a lot to do with primes and divisibility, congruences, and include problems about the rational numbers. For example, solving equations in integers or in the rationals, such as $ x^2 - 2y^2 = 1 $, etc. We will be looking at problems that can be tackled by elementary means, but this does not mean easy. Also the statements of problems can be elementary without the solution being elementary, such as Fermat's Last Theorem, or even known, such as the twin prime conjecture. Sometimes we will state interesting things, like the prime number theorem, without proving them. Typically these will be things that we could prove if the course was much longer. We will start the course with a look at prime numbers and factorisation, a review of Euclid's algorithm and consequences, congruences, the structure of $ \rb{\Z / n\Z}^* $, RSA algorithm, and quadratic reciprocity. We will return to primes at the end, too. Typical questions here include:
\begin{enumerate}
\item How do you tell if a number is prime?
\item How many primes are there congruent to $ a $ modulo $ b $ for given $ a, b $?
\item How many primes are there less than $ n $?
\end{enumerate}
A warning is that we will be using plenty of things from previous algebra courses, about groups, rings, ideals, fields, Lagrange's theorem, the first isomorphism theorem, and so on. You may want to revise this material if you are not comfortable with it. The course is not based on any particular book, although some material, such as continued fractions, was drawn from the following.
\begin{enumerate}
\item A Baker, A concise introduction to the theory of numbers, 1984
\end{enumerate}
Not everything we will do is in that book, though.

\section{Euclidean algorithm and unique factorisation}

\subsection{Divisibility}

\begin{definition}
Let $ a, b \in \Z $. We say that $ a $ \textbf{divides} $ b $, written $ a \mid b $, if there exists $ c \in \Z $ such that $ b = ac $. If $ a $ does not divide $ b $, write $ a \nmid b $.
\end{definition}

\begin{note}
If $ a, b, c \in \Z $ such that $ a \mid b $ and $ a \mid c $, then $ a \mid \rb{rb + sc} $ for any $ r, s \in \Z $.
\end{note}

\begin{definition}
Let $ a, b \in \Z $, not both zero. The \textbf{greatest common divisor} (gcd) or \textbf{highest common factor} (hcf) of $ a $ and $ b $, written $ \rb{a, b} $, is the largest positive integer dividing both $ a $ and $ b $.
\end{definition}

Such an integer always exists since if $ a \ne 0 $ and $ c \mid a $, then $ -a \le c \le a $.

\begin{example}
$ \rb{-10, 15} = 5 $.
\end{example}

\begin{note}
This notation is consistent with notation from ring theory. The ring $ \Z $ is a principal ideal domain (PID), that is it is an integral domain, and every ideal can be generated by one element. The ideal generated by $ f_1, \dots, f_n \in R $ for some ring $ R $ is usually written $ \rb{f_1, \dots, f_n} $, and indeed the ideal $ \rb{a, b} $ is generated by the highest common factor of $ a $ and $ b $, by Theorem \ref{thm:6} below.
\end{note}

\begin{definition}
$ n \in \Z $ is \textbf{prime} if $ n $ has exactly two positive divisors, namely $ 1 $ and $ n $.
\end{definition}

\begin{note}
By definition, primes can be both positive and negative. In spite of this, frequently when people talk about prime numbers they restrict to the positive case. In this course when we say 'Let $ p $ be a prime number' we will generally mean $ p > 0 $. Also $ 1 $ is not prime.
\end{note}

\subsection{Euclid's algorithm}

\begin{proposition}
Let $ a, b \in \Z $, not both zero. Then for any $ n \in \Z $, we have $ \rb{a, b} = \rb{a, b - na} $.
\end{proposition}

\begin{proof}
By definition of $ \rb{a, b} $, it suffices to show that any $ r \in \Z $ divides both $ a $ and $ b $ if and only if it divides both $ a $ and $ b - na $. But if $ r $ divides $ a $ and $ b $, it clearly divides $ b - na $, and if it divides $ a $ and $ b - na $, it clearly divides $ b $.
\end{proof}

This suggests an approach to computing $ \rb{a, b} $ by replacing $ \rb{a, b} $ by a pair $ \rb{a, b - na} $, and repeat until the numbers involved are small enough that it is easy to compute the greatest common divisor. The key to being able to do this is the following innocuous looking result.

\begin{theorem}
\label{thm:5}
Let $ a, b \in \Z $ with $ b > 0 $. Then there exist unique $ q, r \in \Z $ such that $ a = qb + r $ and $ 0 \le r < b $.
\end{theorem}

\begin{proof}
Let $ q = \lfloor a / b \rfloor $ be the largest integer less than $ a / b $. Then by definition $ 0 \le a / b - q < 1 $. Thus $ 0 \le a - qb < b $, so we can take $ r = a - bq $. Uniqueness is easy.
\end{proof}

This gives us \textbf{Euclid's algorithm} for finding $ \rb{a, b} $ for any $ a, b \in \Z $ not both zero. Without loss of generality, assume $ 0 \le b \le a $ and $ a > 0 $.
\begin{enumerate}
\item Check if $ b = 0 $. If so then $ \rb{a, b} = a $.
\item Otherwise, replace $ \rb{a, b} $ with $ \rb{b, r} $ as in Theorem \ref{thm:5}. Then return to step 1.
\end{enumerate}
Since at every stage $ \abs{a} + \abs{b} $ is decreasing, this algorithm terminates. We have shown that $ \rb{a, b} = \rb{b, r} $ so the output is always equal to $ \rb{a, b} $.

\begin{example}
Let us make this explicit:
\begin{align*}
\rb{120, 87}
& = \rb{87, 33} & 120 = 87 + 33 \\
& = \rb{33, 21} & 87 = 2\rb{33} + 21 \\
& = \rb{21, 12} & 33 = 21 + 12 \\
& = \rb{12, 9} & 21 = 12 + 9 \\
& = \rb{9, 3} & 12 = 9 + 3 \\
& = \rb{3, 0} & 9 = 3\rb{3} + 10
\end{align*}
Now run this backwards, writing out the equations, to get:
\begin{align*}
3
& = 12 - 9 \\
& = 12 - \rb{21 - 12} \\
& = 2\rb{12} - 21 \\
& = 2\rb{33 - 21} - 21 \\
& = 2\rb{33} - 3\rb{21} \\
& = 2\rb{33} - 3\rb{87 - 2\rb{33}} \\
& = 8\rb{33} - 3\rb{87} \\
& = 8\rb{120 - 87} - 3\rb{87} \\
& = 8\rb{120} - 11\rb{87}.
\end{align*}
\end{example}

The same works in general, that is the algorithm gives us more than just a way to compute $ \rb{a, b} $. It also allows us to express $ \rb{a, b} $ in terms of $ a $ and $ b $.

\begin{theorem}
\label{thm:6}
Let $ a, b \in \Z $, not both zero. Then there exist $ r, s \in \Z $ such that $ \rb{a, b} = ra + sb $.
\end{theorem}

\begin{proof}
Let $ a_0 = a $ and $ b_0 = b $, and for each $ i $ let $ \rb{a_i, b_i} $ be the result after running $ i $ steps of Euclid's algorithm on the pair $ \rb{a, b} $. For some $ r $ we have $ a_r = \rb{a, b} $ and $ b_r = 0 $. We will show, by downwards induction on $ i $, that there exist $ n_i, m_i \in \Z $ such that $ \rb{a, b} = n_ia_i + m_ib_i $. For $ i = r $ this is clear. On the other hand, for any $ i $ we have $ a_i = b_{i - 1} $ and $ b_i = a_{i - 1} - q_ib_{i - 1} $ for some $ q_i \in \Z $. Thus if $ \rb{a, b} = n_ia_i + m_ib_i $, we have
$$ \rb{a, b} = n_ib_{i - 1} + m_i\rb{a_{i - 1} - q_ib_{i - 1}} = \rb{n_i - m_iq_i}b_{i - 1} + m_ia_{i - 1}, $$
and the claim follows.
\end{proof}

\subsection{Unique factorisation}

The fact that $ \rb{a, b} $ is an integer linear combination of $ a $ and $ b $ has strong consequences for factorisation and divisibility. First note the following.

\begin{proposition}
\label{prop:7}
Let $ n, a, b \in \Z $, and suppose that $ n \mid ab $ and $ \rb{n, a} = 1 $. Then $ n \mid b $.
\end{proposition}

\begin{proof}
Since $ \rb{n, a} = 1 $, there exists $ r, s \in \Z $ such that $ rn + sa = 1 $. Thus $ rnb + sab = b $. But $ n $ clearly divides $ rnb $ and $ sab $, so $ n \mid b $.
\end{proof}

By definition, if $ n $ is prime, then either $ n \mid a $ or $ \rb{n, a} = 1 $. If $ \rb{n, a} = 1 $, we say that $ n, a $ are \textbf{coprime}.

\marginpar{Lecture 2 \\ Tuesday \\ 09/10/18}

\begin{corollary}
\label{cor:8}
If $ p $ is prime, and $ a, b \in \Z $ are such that $ p \mid ab $, then either $ p \mid a $ or $ p \mid b $.
\end{corollary}

\begin{proof}
If $ p \nmid a $ then $ \rb{p, a} = 1 $, so \ref{prop:7} implies $ p \mid b $.
\end{proof}

\begin{proposition}
\label{prop:9}
If $ \rb{a, b} = 1 $, and $ a \mid n $ and $ b \mid n $, then $ ab \mid n $.
\end{proposition}

\begin{proof}
By \ref{thm:6}, we can write $ n = n\rb{a, b} = nra + nsb $ with $ r, s \in \Z $. Each term is divisible by $ ab $, so $ ab \mid n $.
\end{proof}

We say that $ m_1, \dots, m_n \in \Z $ are \textbf{pairwise coprime} if $ \rb{m_i, m_j} = 1 $ for all $ i \ne j $.

\begin{corollary}
\label{cor:10}
Suppose that $ m_1, \dots, m_n $ are pairwise coprime. If $ m_i \mid N $ for all $ i $, then $ \rb{m_1 \dots m_n} \mid N $.
\end{corollary}

\begin{proof}
Induction on $ n $. $ n = 2 $ is Proposition \ref{prop:9}. (TODO Exercise)
\end{proof}

We can now prove the existence and uniqueness of prime factorisations.

\begin{proposition}
\label{prop:11}
Every $ n \in \Z^* $ can be written as $ \pm p_1 \dots p_r $ for some $ r \ge 0 $ and some primes $ p_1, \dots, p_r $.
\end{proposition}

\begin{proof}
Use induction on $ \abs{n} $. The case $ \abs{n} $ is trivial, so suppose $ \abs{n} > 1 $. Then either $ \abs{n} $ is prime, or $ \abs{n} = ab $ with $ 1 < a, b < \abs{n} $, and by induction each of $ a, b $ is a product of primes.
\end{proof}

\begin{theorem}
Let $ n \in \Z_{> 0} $. Then $ n $ can be written as $ p_1 \dots p_r $ where the $ p_i $ are prime, and are uniquely determined up to reordering.
\end{theorem}

\begin{proof}
Existence is Proposition \ref{prop:11}. For uniqueness, suppose that
$$ n = p_1 \dots p_r = q_1 \dots q_s, $$
with $ p_i, q_i $ prime. Then without loss of generality suppose $ r, s \ge 1 $. Then $ p_1 \mid p_1 \dots p_r $, so $ p_1 \mid q_1 \dots q_s $. By Corollary \ref{cor:8}, either $ p_1 \mid q_1 $ or $ p_1 \mid q_2 \dots q_s $. Proceeding inductively, eventually $ p_1 \mid q_i $ for some $ i $. Since $ q_i $ is prime this means $ p_1 = q_i $. We then have
$$ p_2 \dots p_r = q_1 \dots q_i \dots q_s. $$
Since this product is smaller than $ n $, by the inductive hypothesis we must have $ r - 1 = s - 1 $ and the $ p_i $ except $ p_1 $ are a rearrangement of the $ q_i $ except $ q_i $.
\end{proof}

Put together, these are the fundamental theorem of arithmetic.

\subsection{Linear diophantine equations}

Suppose now that we are given $ a, b, c \in \Z^* $ and we want to solve $ ax + by = c $ for $ x, y \in \Z $. We first note that $ \rb{a, b} $ divides both $ a $ and $ b $, so for there to be any solutions, we must have $ \rb{a, b} \mid c $.

\begin{example}
$ 2x + 6y = 3 $ has no solutions.
\end{example}

From now on, suppose this is true. Let $ a' = a / \rb{a, b} $, $ b' = b / \rb{a, b} $, and $ c' = c / \rb{a, b} $. Then $ ax + by = c $ if and only if $ a'x + b'y = c' $. By Theorem \ref{thm:6}, since $ \rb{a', b'} = 1 $, we can find $ r, s \in \Z $ with $ a'r + b's = 1 $, so $ a'rc' + b'sc' = c' $. So $ x = rc' $, $ y = sc' $ is a solution. $ X, Y $ is another solution if and only if $ a'X + b'Y = a'x + b'y $, if and only if $ a'\rb{X - x} = b'\rb{y - Y} $. For this to hold, we need $ a' \mid \rb{y - Y} $, $ b' \mid \rb{X - x} $. Putting this all together, we find that if $ x, y $ is one solution to $ ax + by = c $, then the other solutions are exactly of the form
$$ X = x + n\dfrac{b}{\rb{a, b}}, \qquad Y = y - n\dfrac{a}{\rb{a, b}} $$
for all $ n \in \Z $.

\begin{example}
Using the example above where we have $ 8\rb{120} - 11\rb{87} = 3 $, we can solve $ 120x + 87y = 9 $. One solution is $ x = 24 $ and $ y = -33 $. The general solution is $ x = 24 + 29n $ and $ y = -33 - 40n $. Taking $ n = -1 $, we have for example, $ x = -5 $ and $ y = 7 $.
\end{example}

\section{Congruences and modular arithmetic}

\subsection{Congruences}

\begin{definition}
Let $ n \in \Z^* $, and let $ a, b \in \Z $. We say $ a $ is \textbf{congruent to $ b $ modulo $ n $}, written $ a \equiv b \mod n $, if $ n \mid \rb{a - b} $.
\end{definition}

For $ n $ fixed, it is easy to verify that congruence modulo $ n $ is an equivalence relation, and therefore partitions $ \Z $ into equivalence classes. The set of equivalence classes modulo $ n $ is denoted $ \Z / n\Z $.

\begin{example}
If $ a \equiv b \mod n $, $ c \equiv d \mod n $, then $ a + c \equiv b + d \mod n $ and $ ac \equiv bd \mod n $.
\end{example}

In fact $ \Z / n\Z $ is a ring, with the obvious addition and multiplication. Indeed $ n\Z = \cb{nr \mid r \in \Z} $ is an ideal in $ \Z $, and $ \Z / n\Z $ is just the quotient ring. For any $ a \in \Z $, we sometimes write $ \overline{a} $ for the image of $ a $ in $ \Z / n\Z $. We can write $ a = qn + r $ with $ 0 \le r < n $. Then $ a \equiv r \mod n $, so $ \overline{a} = \overline{r} $.

\begin{example}
If $ n = 12 $, then $ \overline{25} = \overline{1} $.
\end{example}

It follows that $ 0, \dots, n - 1 $ are representatives for the elements of $ \Z / n\Z $, so every element of $ \Z / n\Z $ is equal to $ \overline{r} $ for some unique $ r \in \cb{0, \dots, n - 1} $. It will also be convenient to write $ \Z / n\Z = \cb{0, \dots, n - 1} $.

\begin{example}
If $ n = 6 $, we could write $ 3 + 4 = 1 $ and $ 3 \times 4 = 0 $.
\end{example}

Recall that if $ R $ is a commutative ring, a \textbf{unit} of $ R $ is an element with a multiplicative inverse, that is $ x $ such that there exists $ y \in R $ with $ xy = 1 $. Write $ R^* $ for the set of units in $ R $. This is a group under multiplication.

\begin{example}
$ \Z^* = \cb{\pm 1} $ and $ \Q^* = \Q \setminus \cb{0} = \cb{x \in \Q \mid x \ne 0} $.
\end{example}

We want to understand $ \rb{\Z / n\Z}^* $. Which elements of $ \cb{0, \dots, n - 1} $ are in $ \rb{\Z / n\Z}^* $? If $ r \in \Z $ and $ \overline{r} \in \rb{\Z / n\Z}^* $ then there exists $ s \in \Z $ such that $ rs \equiv 1 \mod n $. This implies that $ \rb{r, n} = 1 $. Conversely, if $ \rb{r, n} = 1 $, then there exists $ x, y \in \Z $ such that $ rx + ny = 1 $, so $ \overline{rx} = 1 $, so $ \overline{r} $ is a unit. Thus we have $ \rb{\Z / n\Z}^* = \cb{\overline{i} \mid \rb{i, n} = 1} $.

\begin{note}
If $ p $ is a prime, then either $ a \equiv 0 \mod p $ or $ \rb{a, p} = 1 $, so $ \rb{\Z / p\Z}^* = \cb{1, \dots, p - 1} $. Thus every non-zero congruence class modulo $ p $ is a unit, that is $ \Z / p\Z $ is a ring with the property that every non-zero element has a multiplicative inverse, so it is a field. Another equivalent way to see this is to check that $ p\Z $ is a maximal ideal of $ \Z $.
\end{note}

\marginpar{Lecture 3 \\ Wednesday \\ 10/10/18}

\subsection{Linear congruence equations}

Fix $ a, b \in \Z $ and $ c \in \Z^* $. Suppose we want to solve $ ax \equiv b \mod c $. This is equivalent to finding $ x, y $ such that $ ax + cy = b $. In particular, by our analysis of linear diophantine equations, there is a solution precisely when $ \rb{a, c} \mid b $. Furthermore, there is a unique solution modulo $ c' = c / \rb{a, c} $, because all the solutions are obtained by adding multiples of $ c' $ to our given $ x $, and subtracting the corresponding multiple of $ a / \rb{a, c} $ from $ y $. This implies that there are a total of $ \rb{a, c} $ solutions to the original congruence modulo $ c $. If $ x $ is a solution, the other solutions are of the form $ X = x + c'j $ for $ 0 \le j < \rb{a, c} $. In particular, if $ \rb{a, c} = 1 $, then there is a unique solution to $ ax \equiv b \mod c $. Indeed $ a \in \rb{\Z / c\Z}^* $, so it has an inverse $ a^{-1} $, and $ x \equiv a^{-1}b \mod c $ is the unique solution.

\begin{example}
$ 2x \equiv 3 \mod 6 $ has no solutions as $ \rb{2, 6} = 2 \nmid 3 $. $ 2x \equiv 4 \mod 6 $, which is equivalent to $ x \equiv 2 \mod 3 $, has solutions $ x \equiv 2 \mod 6 $ and $ x \equiv 5 \mod 6 $.
\end{example}

\subsection{Chinese remainder theorem}

\begin{theorem}[Chinese remainder theorem]
Let $ m_1, \dots, m_n \in \Z_{\ge 0} $ be pairwise coprime. Then the natural map
$$ \Z / m_1 \dots m_n\Z \to \rb{\Z / m_1\Z} \times \dots \times \rb{\Z / m_n\Z} $$
is an isomorphism of rings, and the induced map
$$ \rb{\Z / m_1 \dots m_n\Z}^* \to \rb{\Z / m_1\Z}^* \times \dots \times \rb{\Z / m_n\Z}^* $$
is an isomorphism of abelian groups.
\end{theorem}

\begin{remark}
This is false without the assumption that $ m_i $ pairwise coprime, for example $ m_1 = m_2 = 2 $.
\end{remark}

\begin{proof}
Note firstly that the map exists and is a ring homomorphism. This follows from the fact that if $ x \equiv y \mod m_1 \dots m_n $ then certainly $ x \equiv y \mod m_i $ for each $ i $. The source and target of the ring homomorphism both have order $ m_1 \dots m_n $, so it suffices to show that the map is injective to show that it is an isomorphism. So we only need to check that the kernel is zero. So we need to know that if $ m_i \mid N $ for all $ i $, that is $ \overline{N} = 0 $ in $ \Z / m_i\Z $, then $ m_1 \dots m_n \mid N $, that is $ \overline{N} = 0 $ in $ \Z / m_1 \dots m_n\Z $. This is just Corollary \ref{cor:10}. The statement about unit groups follows by noting that if $ R, S $ are rings, then $ \rb{R \times S}^* = R^* \times S^* $.
\end{proof}

\begin{note}
This can be reformulated more concretely as a statement about congruences. It says that for any $ a_i $, there is a unique $ x \mod m_1 \dots m_n $ such that $ x \equiv a_i \mod m_i $. The proof does not tell us how to find $ x $, but it is actually quite easy in practice. Here is one way to do it. Write $ M = m_1 \dots m_n $ and $ M_i = M / m_i $. Choose $ q_i $ such that $ q_iM_i \equiv 1 \mod m_i $, using Euclid's algorithm and $ \rb{M_i, m_i} = 1 $ because $ \rb{m_j, m_i} = 1 $ for all $ j \ne i $. Then set
$$ x = a_1q_1M_1 + \dots + a_nq_nM_n. $$
For each $ i $ we have $ q_j \equiv 0 \mod m_i $ if $ i \ne j $, so $ x \equiv a_iq_iM_i \equiv a_i \mod m_i $ for each $ i $.
\end{note}

\section{The structure of $ \rb{\Z / n\Z}^* $}

For the next few lecture we will study the abelian group $ \rb{\Z / n\Z}^* $.

\subsection{The Euler $ \Phi $ function}

We define a function $ \Phi\rb{n} $ on $ \Z_{> 0} $ by letting $ \Phi\rb{n} $ denote the order of $ \rb{\Z / n\Z}^* $. Explicitly we have $ \Phi\rb{n} = \#\cb{1 \le i < n \mid \rb{i, n} = 1} $, that is, $ \Phi\rb{n} $ is the number of integers between $ 0 $ and $ n - 1 $ coprime to $ n $.

\begin{example}
If $ p $ is prime, $ \Phi\rb{p} = p - 1 $.
\end{example}

$ \Phi $ is called \textbf{Euler's $ \Phi $ function}.

\begin{definition}
A function $ f $ on $ \Z_{> 0} $ is \textbf{multiplicative} if for all $ m, n \in \Z $ such that $ \rb{m, n} = 1 $, we have $ f\rb{mn} = f\rb{m}f\rb{n} $. We say $ f $ is \textbf{strongly multiplicative} if for any pair of $ m, n \in \Z_{> 0} $ we have $ f\rb{mn} = f\rb{m}f\rb{n} $.
\end{definition}

\begin{note}
By the Chinese Remainder Theorem, $ \Phi $ is multiplicative, because if $ \rb{m, n} = 1 $ then $ \rb{\Z / mn\Z}^* \cong \rb{\Z / m\Z}^* \times \rb{\Z / n\Z}^* $, but not strongly multiplicative, since $ \Phi\rb{4} = 2 \ne 1 = \Phi\rb{2}\Phi\rb{2} $.
\end{note}

It is clear that a multiplicative function is determined by its values on prime powers. For $ p $ prime we have $ \rb{i, p^a} = 1 $ if and only if $ p $ does not divide $ i $, so $ \Phi\rb{p^a} $ is the number of integers between $ 0 $ and $ p^a - 1 $ that are not divisible by $ p $. There are $ p^{a - 1} $ numbers in this range divisible by $ p $, so we have
$$ \Phi\rb{p^a} = \#\cb{1 \le i < p^a \mid \rb{i, p^a} = 1} = \#\cb{1 \le i < p^a \mid p \nmid i} = p^a - p^{a - 1} = p^a\rb{1 - \dfrac{1}{p}}. $$
Write $ n = \prod_i p_i^{a_i} $ where $ p_i $ are distinct primes. From this and multiplicativity of $ \Phi $ one has that
$$ \Phi\rb{n} = \prod_i \Phi\rb{p_i^{a_i}} = \prod_i p_i^{a_i}\rb{1 - \dfrac{1}{p_i}} = n\prod_i \rb{1 - \dfrac{1}{p_i}} = n\prod_{p \mid n} \rb{1 - \dfrac{1}{p}}, $$
where $ p $ runs over the primes dividing $ n $.

\subsection{Euler's theorem}

The units $ \rb{\Z / n\Z}^* $ form a group under multiplication. By definition, $ \phi\rb{n} $ is the order of this group. Recall that for any group $ G $ of finite order $ d $, Lagrange's theorem states that for all $ g \in G $, $ g^d $ is the identity in $ G $. For the group $ \rb{\Z / n\Z}^* $, this means the following.

\begin{theorem}[Euler's theorem]
\label{thm:16}
Let $ a \in \Z $ with $ \rb{a, n} = 1 $. Then $ a^{\Phi\rb{n}} \equiv 1 \mod n $.
\end{theorem}

\begin{proof}
This is equivalent to saying that $ \overline{a}^{\Phi\rb{n}} = 1 $ in $ \rb{\Z / n\Z}^* $. This is a group of order $ \Phi\rb{n} $, so this is immediate from Lagrange's theorem.
\end{proof}

\begin{corollary}[Fermat's little theorem]
If $ p $ is a prime and $ p \nmid a $ then $ a^{p - 1} \equiv 1 \mod p $.
\end{corollary}

\begin{proof}
Theorem \ref{thm:16} with $ n = p $, so $ \Phi\rb{n} = p - 1 $.
\end{proof}

Of course knowing the order of an abelian group does not tell you its structure.

\begin{example}
Let $ n = 5 $. $ \rb{\Z / 5\Z}^* = \cb{1, 2, 3, 4} $. This has order $ 4 $. There are two isomorphism classes of abelian groups of order $ 4 $, namely $ \Z / 4\Z $ and $ \Z / 2\Z \times \Z / 2\Z $. So it is either cyclic of order $ 4 $ or a product of two cyclic groups of order $ 2 $. $ 2^2 = 4 $, $ 2^3 = 3 $, $ 2^4 = 1 $ in $ \rb{\Z / 5\Z}^* $. So $ \rb{\Z / 5\Z}^* $ is cyclic of order $ 4 $.
\end{example}

By the Chinese Remainder Theorem, to understand the structure of $ \rb{\Z / n\Z}^* $, it is enough to understand the structure of $ \Z / p^m\Z $ where $ p $ is prime and $ m \ge 1 $. We will do this next, beginning with the case $ m = 1 $.

\marginpar{Lecture 4 \\ Friday \\ 12/10/18}

\begin{definition}
If $ G $ is a group and $ g \in G $ is an element, the \textbf{order} of $ g $ is the least $ a \ge 1 $ such that $ g^a = 1 $. In particular, if $ \rb{g, n} = 1 $, then we write $ ord_n\rb{g} $ for the order of $ g $ in $ \rb{\Z / n\Z}^* $, or the order of $ g $ modulo $ n $.
\end{definition}

\begin{proposition}
\label{prop:19}
If $ G $ is a group and $ g $ is an element of order $ a $, then $ g^n = 1 $ if and only if $ a \mid n $.
\end{proposition}

\begin{proof}
If $ n = ab $ then $ g^n = \rb{g^a}^b = 1^b = 1 $. Conversely write $ n = ab + r $ with $ 0 \le r < a $. Then $ g^r = 1 $ and since $ r < a $ we have $ r = 0 $.
\end{proof}

In particular, if $ \rb{g, n} = 1 $, then $ g^{\Phi\rb{n}} = 1 $ by Euler's theorem, so Proposition \ref{prop:19} gives $ ord_n\rb{g} \mid \Phi\rb{n} $. We want to prove that if $ p $ is prime, then $ \rb{\Z / p\Z}^* $ is cyclic. Equivalently, we need to show that there exists $ g $ such that $ ord_p\rb{g} = \Phi\rb{p} = p - 1 $. We will do this by counting the number of elements of each order. Key point is that $ \Z / p\Z $ is a field. For any $ d \ge 1 $, the elements of $ \rb{\Z / p\Z}^* $ of order dividing $ d $ are exactly the roots of the equation $ X^d - 1 $ in $ \Z / p\Z $ by Proposition \ref{prop:19}.

\begin{example}
The equation $ X^2 = 1 $ has exactly two solutions modulo $ p $ for any prime $ p $, namely $ \pm 1 $, but it can have more modulo $ n $ if $ n $ is composite. If $ n = 15 $, then $ 4, 11 $ are also solutions. $ X^2 - 1 \equiv 0 \mod n $ if and only if $ n \mid \rb{X + 1}\rb{X - 1} $, so $ 15 \mid \rb{4 + 1}\rb{4 - 1} $.
\end{example}

\begin{definition}
$ g \in \Z $ with $ \rb{g, p} = 1 $ is a \textbf{primitive root} if $ ord_p\rb{g} = p - 1 $, so $ \rb{\Z / p\Z}^* = \ab{g} $.
\end{definition}

\begin{lemma}
\label{lem:21}
Let $ R $ be a commutative ring, and let $ P\rb{X} \in R\sb{X} $. If $ \alpha \in R $ has $ P\rb{\alpha} = 0 $, then there exists $ Q\rb{X} \in R\sb{X} $ such that $ P\rb{X} = \rb{X - \alpha}Q\rb{X} $.
\end{lemma}

\begin{example}
If $ R = \Z / 15\Z $, $ X^2 - 1 = \rb{X + 1}\rb{X - 1} = \rb{X + 4}\rb{X - 4} $.
\end{example}

\begin{proof}
Induction on $ d = \deg\rb{P} $. $ d = 0 $ is obvious. Assume the result holds for degree less than $ d - 1 $. Let $ P\rb{X} = cX^d + \dots $ and $ S\rb{X} = P\rb{X} - cX^{d - 1}\rb{X - \alpha} $. Then $ S\rb{X} $ has degree less than $ d - 1 $. Also $ S\rb{\alpha} = 0 $. By induction, we can write $ S\rb{X} = \rb{X - \alpha}R\rb{X} $. Set $ Q\rb{X} = cX^{d - 1} + R\sb{X} $. Then
$$ \rb{X - \alpha}Q\rb{X} = cX^{d - 1}\rb{X - \alpha} + S\rb{X} = P\rb{X}. $$
\end{proof}

\begin{theorem}
\label{thm:22}
Let $ F $ be a field. Let $ P\rb{X} $ be a polynomial in $ F\sb{X} $. Then $ P\rb{X} $ has at most $ d $ distinct roots in $ F $.
\end{theorem}

\begin{proof}
Induction on $ d = \deg\rb{P} $. $ d = 1 $ is obvious. If $ P $ has no roots, then we are done. Otherwise, let $ \alpha $ be a root. By Lemma \ref{lem:21}, $ P\rb{X} = \rb{X - \alpha}Q\rb{X} $, $ Q\rb{X} $ has degree $ d - 1 $, so we are done by induction.
\end{proof}

\begin{corollary}
\label{cor:23}
Let $ d $ be any divisor of $ p - 1 $. Then there are exactly $ d $ elements of $ \rb{\Z / p\Z}^* $ of order dividing $ d $.
\end{corollary}

\begin{proof}
We have to show that $ X^d - 1 $ has exactly $ d $ roots in $ \Z / p\Z $. $ X^{p - 1} - 1 $ has exactly $ p - 1 $ roots, by Fermat's little theorem. Since $ d \mid \rb{p - 1} $, we can write
$$ X^{p - 1} - 1 = \rb{X^d - 1}\rb{\rb{X^d}^{\tfrac{p - 1}{d} - 1} + \dots + 1} = \rb{X^d - 1}Q\rb{X}, \qquad \deg\rb{Q} = p - 1 - d. $$
$ X^{p - 1} - 1 $ has exactly $ p - 1 $ roots, $ X^d - 1 $ has at most $ d $ roots, and $ Q\rb{X} $ has at most $ p - 1 - d $ roots by Theorem \ref{thm:22}. So $ X^d - 1 $ has exactly $ d $ roots.
\end{proof}

\begin{example}
Let $ p = 7 $. Then $ \rb{\Z / p\Z}^* $ has:
\begin{enumerate}
\item $ 1 $ element of order $ 1 $,
\item $ 2 $ elements of order dividing $ 2 $, so $ 1 $ element of order $ 2 $,
\item $ 3 $ elements of order dividing $ 3 $, so $ 2 $ elements of order $ 3 $, and
\item $ 6 $ elements of order dividing $ 6 $, so $ 2 $ elements of order $ 6 $.
\end{enumerate}
\end{example}

\begin{lemma}
\label{lem:24}
For any $ n \ge 1 $, we have $ \sum_{d \mid n} \Phi\rb{d} = n $.
\end{lemma}

\begin{proof}
For each $ d \mid n $, the elements of $ \cb{1, \dots, n} $ with $ \rb{i, n} = n / d $ are exactly those of the form $ i = \rb{n / d}j $ with $ 1 \le j \le d $ and $ \rb{j, d} = 1 $. There are exactly $ \Phi\rb{d} $ such elements. Since the $ n / d $ run over all the divisors of $ n $, we are done.
\end{proof}

\begin{theorem}
\label{thm:26}
Let $ p $ be prime, and let $ d \mid \rb{p - 1} $. Then there are exactly $ \Phi\rb{d} $ elements of $ \rb{\Z / p\Z}^* $ of order $ d $. In particular, there are $ \Phi\rb{p - 1} $ primitive roots, and $ \rb{\Z / p\Z}^* $ is cyclic.
\end{theorem}

\begin{proof}
Induction on $ d $. $ d = 1 $ is obvious. Assume the result holds for all $ d' \mid d $, $ d' \ne d $. Then by Lemma \ref{lem:24},
$$ \Phi\rb{d} = d - \sum_{d' \mid d, \ d' \ne d} \Phi\rb{d'}. $$
Now use inductive hypothesis and Corollary \ref{cor:23}.
\end{proof}

\marginpar{Lecture 5 \\ Tuesday \\ 16/10/18}

\begin{proposition}
Let $ p $ be an odd prime and $ n \ge 1 $. Then $ \rb{\Z / p^n\Z}^* $ is cyclic.
\end{proposition}

\begin{proof}
Consider three cases.
\begin{itemize}
\item[$ n = 1 $] Theorem \ref{thm:26}.
\item[$ n = 2 $] Let $ g $ be a primitive root modulo $ p $. Claim that either $ g^{p - 1} \not\equiv 1 \mod p^2 $, and $ g $ is a generator for $ \rb{\Z / p^2\Z}^* $, or $ g^{p - 1} \equiv 1 \mod p^2 $, and $ g + p $ is a generator for $ \rb{\Z / p^2\Z}^* $. Either way, $ \rb{\Z / p^2\Z}^* $ is cyclic. Suppose firstly that $ g^{p - 1} \not\equiv 1 \mod p^2 $.
$$ \#\rb{\Z / p^2\Z}^* = \Phi\rb{p^2} = p\rb{p - 1}. $$
So $ ord_{p^2}\rb{g} \mid p\rb{p - 1} $. On the other hand, $ g^{ord_{p^2}\rb{g}} \equiv 1 \mod p^2 $ gives $ g^{ord_{p^2}\rb{g}} \equiv 1 \mod p $, so $ \rb{p - 1} \mid ord_{p^2}\rb{g} $ because $ ord_p\rb{g} = p - 1 $ by assumption. But $ ord_{p^2}\rb{g} \ne p - 1 $, as $ g^{p - 1} \not\equiv 1 \mod p^2 $. So $ ord_{p^2}\rb{g} = p\rb{p - 1} $ as required. Suppose now that $ g^{p - 1} \equiv 1 \mod p $. It suffices to show that $ \rb{g + p}^{p - 1} \not\equiv 1 \mod p^2 $, as we can then apply the analysis above with $ g + p $ in place of $ g $. By the binomial theorem,
$$ \rb{g + p}^{p - 1} \equiv g^{p - 1} + \rb{p - 1}g^{p - 2}p \equiv 1 + \rb{p - 1}g^{p - 2}p \mod p^2. $$
Since $ p \nmid \rb{p - 1}g^{p - 2} $, $ \rb{g + p}^{p - 1} \not\equiv 1 \mod p^2 $, as required.
\item[$ n \ge 2 $] It suffices to show that if $ ord_{p^2}\rb{g} = p\rb{p - 1} $, then $ ord_{p^n}\rb{g} = p^{n - 1}\rb{p - 1} $. We do this by induction on $ n $. So assume that $ ord_{p^n}\rb{g} = p^{n - 1}\rb{p - 1} $. Then
$$ p^{n - 1}\rb{p - 1} = ord_{p^n}\rb{g} \mid ord_{p^{n + 1}}\rb{g} \mid \Phi\rb{p^{n + 1}} = p^n\rb{p - 1}. $$
So either $ ord_{p^{n + 1}}\rb{g} = p^n\rb{p - 1} $, or $ ord_{p^{n + 1}}\rb{g} = p^{n - 1}\rb{p - 1} $. So we need to show that $ g^{p^{n - 1}\rb{p - 1}} \not\equiv 1 \mod p^{n + 1} $. To do this, consider $ g^{p^{n - 2}\rb{p - 1}} \mod p^{n - 1} $ and $ g^{p^{n - 2}\rb{p - 1}} \mod p^n $. Since $ \Phi\rb{p^{n - 1}} = p^{n - 2}\rb{p - 1} $, $ g^{p^{n - 2}\rb{p - 1}} \equiv 1 \mod p^{n - 1} $ by Euler's theorem. Write $ g^{p^{n - 2}\rb{p - 1}} = 1 + p^{n - 1}t $. Since $ ord_{p^n}\rb{g} = p^{n - 1}\rb{p - 1} $ by assumption, $ g^{p^{n - 2}\rb{p - 1}} \not\equiv 1 \mod p^n $, that is $ p \nmid t $. Then
$$ g^{p^{n - 1}\rb{p - 1}} = \rb{g^{p^{n - 2}\rb{p - 1}}}^p = \rb{1 + p^{n - 1}t}^p \equiv 1 + p^nt + \two{p}{2}p^{2\rb{n - 1}}t^2 + \dots + p^{p\rb{n - 1}}t^p \mod p^{n + 1}, $$
Now $ r\rb{n - 1} \ge n + 1 $ if and only if $ \rb{r - 1}n \ge r + 1 $. Since $ p > 2 $,
$$ p \ \Big| \ \two{p}{2} \qquad \implies \qquad p^{n + 1} \ \Big| \ p^{2n - 1} = p^{2\rb{n - 1} + 1} \ \Big| \ \two{p}{2}p^{2\rb{n - 1}}. $$
So $ g^{p^{n - 1}\rb{p - 1}} \equiv 1 + p^nt \not\equiv 1 \mod p^{n + 1} $, because $ p \nmid t $.
\end{itemize}
\end{proof}

\begin{example}
Let $ p = 2 $.
\begin{enumerate}
\item $ \rb{\Z / 2\Z}^* = \cb{1} $.
\item $ \rb{\Z / 4\Z}^* = \cb{1, 3} $ is cyclic of order $ 2 $, with $ 3 $ as a generator.
\item $ \rb{\Z / 8\Z}^* = \cb{1, 3, 5, 7} $ is not cyclic. $ 1^2 \equiv 3^2 \equiv 5^2 \equiv 7^2 \equiv 1 \mod 8 $, so every element has order two.
\end{enumerate}
\end{example}

\begin{lemma}
\label{lem:27}
For $ n \ge 0 $ we have $ 5^{2^n} \equiv 1 + 2^{n + 2} \mod 2^{n + 3} $.
\end{lemma}

\begin{proof}
Induction on $ n $. $ n = 0 $ is obvious. Assume that $ 5^{2^n} = 1 + 2^{n + 2}t $ with $ t $ odd. Then
$$ 5^{2^{n + 1}} = \rb{1 + 2^{n + 1}t}^2 = 1 + 2^{n + 3}t + 2^{2\rb{n + 2}}t^2 = 1 + 2^{n + 3}\rb{t + 2^{n + 1}t^2}, $$
where and $ t + 2^{n + 1}t^2 $ is odd.
\end{proof}

\begin{proposition}
If $ n \ge 2 $ then there is an isomorphism $ \rb{\Z / 2^n\Z}^* \to \rb{\Z / 2\Z} \times \rb{\Z / 2^{n - 2}\Z} $. In particular, if $ n \ge 3 $, then $ \rb{\Z / 2^n\Z}^* $ is not cyclic.
\end{proposition}

\begin{proof}
Let $ \ab{g} $ denote the group $ \cb{1, \dots, g^{ord\rb{g} - 1}} $ generated by $ g $. Consider the natural map $ \ab{-1} \times \ab{5} \to \rb{\Z / 2^n\Z}^* $. This is injective, because if $ \pm 1\rb{5}^s \equiv 1 \mod 2^n $ then in particular $ \pm 1\rb{5}^s \equiv 1 \mod 4 $ so $ \pm 1 \equiv 1 \mod 4 $, so we must have $ 5^s \equiv 1 \mod 2^n $, that is $ 5^s = 1 $ in $ \ab{5} $. $ \ab{-1} $ has order $ 2 $ and $ \ab{5} $ has order $ ord_{2^n}\rb{5} = 2^{n - 2} $ by Lemma \ref{lem:27}. So $ \ab{-1} \times \ab{5} $ has order $ 2\rb{2^{n - 2}} = 2^{n - 1} = \Phi\rb{2^n} = \#\rb{\Z / 2^n\Z}^* $. So the map $ \ab{-1} \times \ab{5} \to \rb{\Z / 2^n\Z}^* $ is an injection of groups of the same order, so it is a bijection.
\end{proof}

\begin{theorem}
$ \rb{\Z / n\Z}^* $ is cyclic if and only if
\begin{enumerate}
\item $ n = 1, 2, 4 $,
\item $ n = p^r $ for $ p > 2 $ prime and $ r \ge 1 $, and
\item $ n = 2p^r $ for $ p > 2 $ prime and $ r \ge 1 $.
\end{enumerate}
\end{theorem}

\end{document}